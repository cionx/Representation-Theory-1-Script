\section{Finite dimensional representations of \texorpdfstring{$\sll_2(k)$}{sl2(k)}}
Recall that the standard basis $(e,h,f)$ of $\sll_2(k)$ is given by
\[
 e = \begin{pmatrix} 0 & 1 \\ 0 & 0 \end{pmatrix}, \quad
 h = \begin{pmatrix} 1 & 0 \\ 0 & -1 \end{pmatrix}, \quad
 f = \begin{pmatrix} 0 & 0 \\ 1 & 0 \end{pmatrix}
\]
and that this basis satisfies the relations
\[
 [h,e] = 2e, \quad
 [h,f] = -2f, \quad
 [e,f] = h.
\]


\begin{defi}
 Let $V$ be a representation of $\sll_2(k)$. For any $\lambda \in k$ let
 \[
  V_\lambda \coloneqq \{v \in V \mid h.v = \lambda v\}
 \]
 be the \emph{weight space} of $V$ with respect to $\lambda$. An element $\lambda \in k$ is called a \emph{weight} of $V$ if $V_\lambda \neq 0$.
\end{defi}


\begin{lem}\label{lem: e and f moving the eigenspaces}
 If $V$ is a representation of $\g$ then
 \[
  e.V_\lambda \subseteq V_{\lambda+2}
  \quad\text{and}\quad
  f.V_\lambda \subseteq V_{\lambda-2}
  \quad\text{for every $\lambda \in k$}.
 \]
\end{lem}
\begin{proof}
 Let $v \in V_\lambda$. Then
 \begin{gather*}
  h.(e.v)
  = e.h.v + [h,e].v
  = \lambda e.v + 2e.v
  = (\lambda+2) e.v
 \shortintertext{and}
  h.(f.v)
  = h.f.v + [h,f].v
  = \lambda h.v - 2f.v
  = (\lambda-2) f.v.
 \qedhere
 \end{gather*}
\end{proof}


\begin{expl}\label{expl: natural representation of sl2}
 Let $e_1$, $e_2$ be the standard basis of $V \coloneqq k^2$, which is the natural representation of $\sll_2(k)$. Then $h.e_1 = e_1$ and $h.e_2 = -e_2$, so $V_1 = k e_1$ and $V_{-1} = k e_2$ with $V = V_{-1} \oplus V_1$. That $e.V_{-1} = V_1$ and $f.V_1 = V_{-1}$ can then by seen by a glance at the following:
 \[
  e = \begin{pmatrix} 0 & 1 \\ 0 & 0 \end{pmatrix}
  \quad\text{and}\quad
  f = \begin{pmatrix} 0 & 0 \\ 1 & 0 \end{pmatrix}.
 \]
\end{expl}





\subsection{Finite dimensional irreducible representations}


\begin{thrm}[Classification of finite dimensional irreducible representations of $\sll_2(k)$]
 For every $n \in \N$ there exists up to isomorphism a unique $(n+1)$-dimensional irreducible representation of $\sll_2(k)$. More explicitely:
 \begin{enumerate}[leftmargin=*]
  \item
   For every $n \in \N$ there exists an $(n+1)$-dimensional irreducible representation $V^{(n)}$ of $\sll_2(k)$.
  \item
   Let $V$ be an $(n+1)$-dimensional irreducible representation of $\sll_2(k)$ for some $n \in \N$. Then
   \[
    V = V_{-n} \oplus V_{-n+2} \oplus \dotsb \oplus V_{n-2} \oplus V_n
   \]
   and $V_i$ is one-dimensional for every $i = -n, -n+2, \dotsc, n-2, n$. In particular all occuring weights are integral and the heighest weight is $n$.
   
   More explicitely there exists a basis $v_{-n}, v_{-n+2}, \dotsc, v_{n-2}, v_n$ of $V$ with $v_i \in V_i$ for every $i = -n, -n+2, \dotsc, n-2, n$, with respect to which the actions of $e$ and $f$ are given as in the following diagram, where the dashed arrows represent the action of $f$.
   \[
    \begin{tikzcd}
       v_{-n}
       \arrow[bend left]{r}[above]{1}
       &
       v_{-n+2}
       \arrow[bend left]{r}[above]{2}
       \arrow[bend left, dashed]{l}[below]{n-1}
       &
       v_{-n+4}
       \arrow[bend left]{r}[above]{3}
       \arrow[bend left, dashed]{l}[below]{n-2}
       &
       v_{-n+6}
       \arrow[bend left]{r}
       \arrow[bend left, dashed]{l}[below]{n-3}
       &
       \dots
       \arrow[bend left]{r}
       \arrow[bend left, dashed]{l}
       &
       v_{n-6}
       \arrow[bend left]{r}[above]{n-3}
       \arrow[bend left, dashed]{l}
       &
       v_{n-4}
       \arrow[bend left]{r}[above]{n-2}
       \arrow[bend left, dashed]{l}[below]{3}
       &
       v_{n-2}
       \arrow[bend left]{r}[above]{n-1}
       \arrow[bend left, dashed]{l}[below]{2}
       &
       v_n
       \arrow[bend left, dashed]{l}[below]{1}
     \end{tikzcd}
   \]
   In particular
   \begin{gather*}
    e.V_i =
    \begin{cases}
     V_{i+2} & \text{for $i = -n, -n+2, \dotsc, n-2$}, \\
     0       & \text{otherwise},
    \end{cases}
   \shortintertext{and}
    f.V_i =
    \begin{cases}
     V_{i-2} & \text{if $i = -n, \dotsc, n-2, n$}, \\
     0       & \text{otherwise}.
    \end{cases}
   \end{gather*}
 \end{enumerate}
\end{thrm}
\begin{proof}
 \begin{enumerate}[leftmargin=*]
  \item
   Let $\sll_2(k)$ act on $V \coloneqq k[x,y]$ via
   \[
    \sll_2(k) \to \gl(V), \quad
    e \mapsto y\dd{x}, \quad
    h \mapsto y\dd{y}-x\dd{x}, \quad
    f \mapsto x\dd{y}.
   \]
   It was already shown in Examples~\ref{expls: representations of Lie algebras} that this defines a representation of $\sll_2(k)$. Let $V^{(n)} \subseteq V$ be the linear subspace consisting of the homogeneous polynomials of degree $n$, i.e.\
   \[
    V^{(n)} = \vspan_k\left( x^n, x^{n-1} y, \dotsc, x y^{n-1}, y^n \right).
   \]
   Then $V^{(n)}$ is an $(n+1)$-dimensional subrepresentation of $V$.
   
   Let $U \subseteq V^{(n)}$ be a nonzero subrepresentation. If $p \in U$ is a nonzero polynomial then by applying $f$ often enough it follows that $x^n \in U$, from which it follows from applying $e$ often enough that $x^{n-i} y^i \in U$ for every $i = 0, \dotsc, n$. Hence $U = V$. So $V^{(n)}$ is an irreducible representation of $\sll_2(k)$.
   
  \item
   Because $V \neq 0$ is finite dimensional and $k$ is algebraically closed there exists some $\lambda \in k$ with $V_\lambda \neq 0$. Because $v$ is finite dimensional $\lambda$ can be choosen such that $V_{\lambda-2} = 0$. Let $w \in V_\lambda$ with $w \neq 0$. Set
   \[
    w_i \coloneqq e^i.w \quad \text{for every $i \in \N$}.
   \]
   \begin{claim*}
    \begin{enumerate}[leftmargin=*]
     \item
      $h.w_i = (\lambda+2i) w_i$ for every $i \in \N$.
     \item
      $f.w_0 = 0$ and $f.w_{i+1} = -(i+1)(\lambda+i)w_i$ for every $n \in \N$.
    \end{enumerate}
   \end{claim*}
   \begin{proof}
    \begin{enumerate}[leftmargin=*]
     \item
      This follows from $w \in V_\lambda$ and Lemma~\ref{lem: e and f moving the eigenspaces}.
     \item
      From $w_0 = w \in V_\lambda$ and Lemma~\ref{lem: e and f moving the eigenspaces} it follows that $f.w_0 \in V_{\lambda-2}$. Because $V_{\lambda-2} \neq 0$ it follows that $f.w_0 = 0$.
      
      The second formula will be shown by induction over $i \in \N$. It holds for $i = 0$ because
      \[
       f.w_1
       = f.e.w_0
       = [f,e].w_0 - e.f.w_0
       = -h.w_0
       = -\lambda w_0.
      \]
      Now let $i > 0$ and suppose the formula holds for $i-1$. Then
      \begin{align*}
       f.w_{i+1}
       &= f.e.w_i
       = [f,e].w_i + e.f.w_i
       = -h.w_i + e.f.w_i \\
       &= -(\lambda+2i) w_i -i(\lambda+i-1) e.w_{i-1} \\
       &= (-\lambda -2i -i\lambda -i^2 +i) w_i
       =  -(i+1)(\lambda+i) w_i.
      \qedhere
      \end{align*}
     \qedhere
    \end{enumerate}
   \end{proof}
   
   Because $w_i \in V_{\lambda+2i}$ for every $i \in \N$ with $w_0 = v \neq 0$ and $V$ is finite dimensional it follows that there exists a maximal $m \in \N$ such that $w_0, \dotsc, w_m$ are nonzero but $w^{m+1} = 0$. By the previous claim $\vspan_k(w_0, \dotsc, w_m)$ is a subrepresentation of $V$. Because $V$ is irreducible it follows that $V = \vspan_k(w_0, \dotsc, w_m)$. Because $w_0, \dotsc, w_m$ are linearly independent it follows that $w_0, \dotsc, w_m$ is a basis of $V$.
   
   As $V$ is ($n+1$)-dimensional it follows that $m = n$. By the claim
   \[
    0
    = f.w_{n+1}
    = -n(\lambda+n).w_n
   \]
   and therefore $\lambda = -n$. Because $w_i \in V_{\lambda + 2i} = V_{-n+2i}$ for every $i \in \N$ it follows that
   \[
    V
    = kw_0 \oplus \dotsb \oplus kw_n
    = V_{-n} \oplus V_{-n+2} \oplus \dotsb \oplus V_{n-2} \oplus V_n
   \]
   with $V_i$ being one-dimensional for every $i = -n, -n+2, \dotsc, n-2, n$. From the definition of $w_0, \dotsc, w_n$ and the claim it follows that the actions of $e$ and $f$ are given as in the following diagram, where the dashed arrows represent the action of $f$.
   \[
      \begin{tikzcd}
        w_0
        \arrow[bend left]{r}[above]{1}
        &
        w_1
        \arrow[bend left]{r}[above]{1}
        \arrow[bend left, dashed]{l}[below]{n-1}
        &
        w_2
        \arrow[bend left]{r}[above]{1}
        \arrow[bend left, dashed]{l}[below]{2(n-2)}
        &
        w_3
        \arrow[bend left]{r}
        \arrow[bend left, dashed]{l}[below]{3(n-3)}
        &
        \dots
        \arrow[bend left]{r}
        \arrow[bend left, dashed]{l}
        &
        w_{n-3}
        \arrow[bend left]{r}[above]{1}
        \arrow[bend left, dashed]{l}
        &
        w_{n-2}
        \arrow[bend left]{r}[above]{1}
        \arrow[bend left, dashed]{l}[below]{(n-3)3}
        &
        w_{n-1}
        \arrow[bend left]{r}[above]{1}
        \arrow[bend left, dashed]{l}[below]{(n-2)2}
        &
        w_n
        \arrow[bend left, dashed]{l}[below]{n-1}
     \end{tikzcd}
   \]
   The desired basis $v_1, \dotsc, v_n$ can now be defined as
   \[
    v_{-n+2i} \coloneqq \frac{w_i}{i!}
    \quad \text{for every $i = 0, 1, \dotsc, n+1$}.
    \qedhere
   \]
 \end{enumerate}
\end{proof}


\begin{expls}
 \begin{enumerate}[leftmargin=*]
  \item
   The one-dimensional irreducible representation of $\sll_2(k)$ can be realized by $\sll_2(k)$ acting trivially on $k$ (or any one-dimensional vector space for that matter.)
  \item
   The two-dimensional irreducible representation of $\sll_2(k)$ can be realized as the natural representation of $\sll_2(k)$. The detailed calculations were already shown in Example~\ref{expl: natural representation of sl2}.
  \item
   Let $V = \sll_2(k)$ be the adjoint representation of $\sll_2(k)$. Because $\sll_2(k)$ is simple it follows that $V$ is irreducible and $V = V_{-2} \oplus V_0 \oplus V_2$ with $V_{-2} = k f$, $V_0 = k h$ and $V_2 = k e$.
 \end{enumerate}
\end{expls}





\subsection{Arbitrary finite dimensional representations}


\begin{expl}
 The map
 \[
  \phi \colon \sll_2(k) \to \sll_3(k), \quad
  A \mapsto \begin{pmatrix} A & 0 \\ 0 & 0 \end{pmatrix}
 \]
 is a homomorphism of Lie algebras. Therefore $V \coloneqq \sll_3(k)$ can be made into a representation of $\sll_2(k)$ via
 \[
  \rho \colon \sll_2(k) \xlongrightarrow{\phi} \sll_3(k) \xlongrightarrow{\ad} \gl(\sll_3(k)) = \gl(V).
 \]
 Then $h$ acts on $V$ by $\ad(H)$ for
 \[
%   E \coloneqq
%   \begin{pmatrix}
%    0 & 1 & 0 \\
%    0 & 0 & 0 \\
%    0 & 0 & 0
%   \end{pmatrix},
%   \quad
  H \coloneqq \phi(h) =
  \begin{pmatrix}
   1 &  0 & 0 \\
   0 & -1 & 0 \\
   0 &  0 & 0
  \end{pmatrix},
%   \quad
%   F \coloneqq
%   \begin{pmatrix}
%    0 & 0 & 0 \\
%    1 & 0 & 0 \\
%    0 & 0 & 0
%   \end{pmatrix},
 \]

 Notice that this representation is not irreducible because $\phi(\sll_2(k))$ is a non-trivial subrepresentation. However, by Weyl’s theorem $V$ decomposes into the direct sum of irreducible subrepresentations and it will be shown how to do so:
 
 Let $e_{12}, e_{13}, e_{23}, e_{21}, e_{31}, e_{32}, h_1, h_2$ be the basis of $V$ with
 \[
  h_1 \coloneqq \diag(1, -1, 0) \quad\text{and}\quad h_2 \coloneqq \diag(0, 1, -1).
 \]
 Notice that
 \[
  [\diag(a_1, a_2, a_3), e_{ij}] = (a_i-a_j) e_{ij}
  \quad\text{for all $a_1, a_2, a_3 \in k$ and $i,j = 1, \dotsc, 3$}.
 \]
 It follows that
 \[
  e_{21} \in V_{-2}, \quad
  e_{23}, e_{31} \in V_{-1}, \quad
  h_1, h_2 \in V_0, \quad
  e_{13}, e_{32} \in V_1, \quad
  e_{12} \in V_2.
 \]
 So $V = \bigoplus_{i \in \Z} V_i$ with dimensions
 \begin{equation}\label{eqn: dimensions of eigenspaces of sl3}
  \dim V_i =
  \begin{cases}
   1 & \text{if $i = -2, 2$}, \\
   2 & \text{if $i = -1, 0, 1$}, \\
   0 & \text{otherwise}.
  \end{cases}
 \end{equation}
 
 Suppose that $V = W^1 \oplus W^2$ for two four-dimensional irreducible subrepresentations $W^1$ and $W^2$. Then
 \[
  W^1 = W^1_{-3} \oplus W^1_{-1} \oplus W^1_1 \oplus W^1_3
  \quad\text{and}\quad
  W^2 = W^2_{-3} \oplus W^2_{-1} \oplus W^2_1 \oplus W^2_3.
 \]
 and it follows that
 \[
  V = \underbrace{W^1_{-3} \oplus W^2_{-3}}_{= V_{-3}} \oplus
      \underbrace{W^1_{-1} \oplus W^2_{-1}}_{= V_{-1}} \oplus
      \underbrace{W^1_1 \oplus W^2_1}_{= V_{1}}  \oplus
      \underbrace{W^1_3 \oplus W^2_3}_{= V_{3}},
 \]
 contradicting \eqref{eqn: dimensions of eigenspaces of sl3}.
 
 To emphasize the chosen approach suppose that there exists a decomposition $V = W^{0,1} \oplus W^{0,2} \oplus W^{0,3} \oplus W^{1,1} \oplus W^{2,1}$ into irreducible subrepresentations with $\dim W^{0,1} = \dim W^{0,2} = \dim W^{0,3} = 1$, $\dim W^{1,1} = 2$ and $\dim W^{1,1} = 3$. Then
 \begin{align*}
  V
  &= W^{0,1} \oplus W^{0,2} \oplus W^{0,3} \oplus W^{1,1} \oplus W^{2,1} \\
  &= \underbrace{W^{2,1}_{-2}}_{= V_{-2}} \oplus
     \underbrace{W^{1,1}_{-1}}_{= V_{-1}} \oplus
     \underbrace{W^{0,1}_0 \oplus W^{0,2}_0 \oplus W^{0,3}_0 \oplus W^{2,1}_0}_{= V_0} \oplus
     \underbrace{W^{1,1}_1}_{= V_1} \oplus
     \underbrace{W^{2,1}_1}_{= V_2},
 \end{align*}
 also contradicting \eqref{eqn: dimensions of eigenspaces of sl3}.
 
 The above observations can easily be generalized: Let $V = \bigoplus_{d \in \N} \bigoplus_{j=1}^{\nu_d} W^{d,j}$ be a decomposition into irreducible subrepresentations with $W^{d,j}$ being $(d+1)$-dimensional, i.e.\ having heighest weight $d$. As in the previous examples it follows that $V_i = \bigoplus_{p \in \N, d=|i|+2p} \bigoplus_{j=1}^{\nu_d} W^{d,j}_i$ for every $i \in \Z$ and therefore $\dim V_i = \sum_{p \in \N, d=|i|+2p} \nu_d$. Hence
 \[
  \nu_d = \dim V_d - \dim V_{d+2}
  \quad\text{for every $d \in \N$}.
 \]
 With this it follows from \eqref{eqn: dimensions of eigenspaces of sl3} that $\nu_0 = 1$, $\nu_1 = 2$ and $\nu_2 = 1$.
 
 The three-dimensional irreducible subrepresentation is given by
 \[
  W^{2,1} \coloneqq \phi(\sll_2(k)) = \vspan_k(e_{12}, h_1, e_{21})
 \]
 The two two-dimensional irreducible subrepresentations $W^{1,1}$ and $W^{1,2}$ must satisfy
 \[
  W^{1,1} \oplus W^{1,2} = V_{-1} \oplus V_1 = \vspan_k(e_{23}, e_{31}, e_{13}, e_{32}).
 \]
 They then can be choosen as $W^{1,1} \coloneqq \vspan_k(e_{23}, e_{13})$ and $W^{1,2} \coloneqq \vspan_k(e_{31}, e_{32})$. That these are indeed subrepresentations follows from direct calculation. To find the remaining one-dimensional irreducible subrepresentation notice that
 \begin{gather*}
  e.h_1 = -2e_{12}, \quad h.h_1 = 0, \quad f.h_1 = 2e_{21},
 \shortintertext{and}
  e.h_2 = e_{12}, \quad h.h_2 = 0, \quad f.h_2 = -e_{21}.
 \end{gather*}
 Hence $\sll_2(k)$ acts trivially on the one-dimensional linear subspace $W^{0,1} \coloneqq k (h_1 + 2h_2)$, which is why it is a subrepresentation.
\end{expl}


\begin{thrm}\label{thrm: finite dimensional representations of sl2}
 Let $V$ be a finite dimensional representation of $\sll_2(k)$.
 \begin{enumerate}[leftmargin=*]
  \item
   $V = \bigoplus_{d \in \N} \bigoplus_{j=1}^{\nu_d} W^{d,j}$ where $W^{d,j} \subseteq V$ is an irreducible $(d+1)$-dimensional subrepresentation for all $d \in \N$ and $j = 1, \dotsc, \nu_d$.
  \item
   $V = \bigoplus_{i \in \Z} V_i$ with
   \begin{equation}\label{eqn: eigenspace of a finite dimensional representation}
    V_i = \bigoplus_{\substack{p \in \N \\ d = |i|+2p}} \bigoplus_{j=1}^{\nu_d} W^{d,j}_i
    \quad \text{for every $i \in \Z$}
   \end{equation}
   and $\dim V_i = \sum_{p \in \N, d = |i|+2p} \nu_d$ for every $i \in \Z$. In particular all occuring weights are integral.
  \item 
   The numbers $\nu_d$ for $d \in \N$ are unique with $\nu_d = \dim V_d - \dim V_{d+2}$ for every $d \in \N$.
  \item
   $\dim V_i = \dim V_{-i}$ for every $i \in \Z$.
 \end{enumerate}
\end{thrm}
\begin{proof}
 That $V$ decomposes into the direct sum of irreducible subrepresentations follows directly from Weyl’s theorem. From the classification of finite dimensional irreducible representations of $\sll_2(k)$ it follows that
 \[
  W^{d,j}
  = W^{d,j}_{-d} \oplus W^{d,j}_{-d+2} \oplus \dotsb \oplus W^{d,j}_{d-2} \oplus W^{d,j}_d
  = \bigoplus_{p=0}^d W^{d,j}_{-d+2p}
 \]
 for every $d \in \N$ and $j = 1, \dotsc, \nu_d$. It follows that
 \[
  V
  = \bigoplus_{d \in \N} \bigoplus_{j=1}^{\nu_d} W^{d,j}
  = \bigoplus_{d \in \N} \bigoplus_{j=1}^{\nu_d} \bigoplus_{p=0}^d W^{d,j}_{-d+2p}.
 \]
 This shows that $V_i = \bigoplus_{i \in \Z} V_i$. Formula \eqref{eqn: eigenspace of a finite dimensional representation} follows from reordering the summands. It follows that
 \[
  \dim V_i
  = \dim \bigoplus_{\substack{p \in \N \\ d = |i|+2p}} \bigoplus_{j=1}^{\nu_d} W^{d,j}_i
  = \sum_{\substack{p \in \N \\ d = |i|+2p}} \sum_{j=1}^{\nu_d} \underbrace{\dim W^{d,j}_i}_{=1}
  = \sum_{\substack{p \in \N \\ d = |i|+2p}} \nu_d.
 \]
 As a direct consequence it follows that $\dim V_i = \dim V_{-i}$ for every $i \in \Z$ and
 \[
  \dim V_d - \dim V_{d+2}
  = \sum_{\substack{p \in \N \\ d' = i+2p}} \nu_{d'} - \sum_{\substack{p \in \N \\ d' = i+2+2p}} \nu_{d'}
  = \nu_d
  \quad\text{for every $d \in \N$}.
 \qedhere
 \]
\end{proof}





\subsection{The Clebsch--Gordan decomposition}


\begin{expl}\label{expl: decomposition of the tensor product of 4 and 5}
 For every $n \in \N$ let $V^{(n)}$ be the $(n+1)$-dimensional irreducible representation of $V$ and abbreviate $V \coloneqq V^{(3)}$ and $W \coloneqq V^{(4)}$. Then
 \begin{equation}\label{eqn: decomposition of 4 tensor 5}
  V \otimes W \cong V^{(7)} \otimes V^{(5)} \otimes V^{(3)} \otimes V^{(1)}.
 \end{equation}
 To see this notice that for $i,j \in \Z$ with $v \in V_\lambda$ and $w \in W_\lambda$ it follows that
 \[
  h.(v \otimes w)
  = (h.v) \otimes w + v \otimes (h.w)
  = iv \otimes w + jv \otimes w
  = (i+j) v \otimes w
 \]
 and therefore $V_i \otimes W_j \subseteq (V \otimes W)_{i+j}$. It follows that
 \begin{align*}
  (V \otimes W)_{-7}
  &= V_{-3} \otimes W_{-4}, \\
  (V \otimes W)_{-5}
  &= (V_{-3} \otimes W_{-2}) \oplus (V_{-1} \oplus W_{-4}), \\
  (V \otimes W)_{-3}
  &= (V_{-3} \otimes W_0) \oplus (V_{-1} \otimes W_{-2}) \oplus (V_1 \otimes W_{-4}), \\
  (V \otimes W)_{-1}
  &= (V_{-3} \otimes W_2) \oplus (V_{-1} \otimes W_0) \oplus (V_1 \otimes W_{-2}) \oplus (V_3 \otimes W_{-4}), \\
  (V \otimes W)_1
  &= (V_{-3} \otimes W_4) \oplus (V_{-1} \otimes W_2) \oplus (V_1 \otimes W_0) \oplus (V_3 \otimes W_{-2}), \\
  (V \otimes W)_3
  &= (V_{-1} \otimes W_4) \oplus (V_1 \otimes W_2) \oplus (V_3 \otimes W_0), \\
  (V \otimes W)_5
  &= (V_1 \otimes W_4) \oplus (V_3 \otimes W_2), \\
  (V \otimes W)_7
  &= V_3 \otimes W_4.
 \end{align*}
 with each of the summands being one-dimensional. Therefore the weights spaces of $V \otimes W$ have the following dimensions:
 \[
  \begin{array}{c|rrrrrrrrc}
   i \in \Z             & -7 & -5 & -3 & -1 & 1 & 3 & 5 & 7 & \text{otherwise} \\
   \hline
   \dim (V \otimes W)_i &  1 &  2 &  3 &  4 & 4 & 3 & 2 & 1 & 0
  \end{array}
 \]
 Formula \eqref{eqn: decomposition of 4 tensor 5} follows from this by Theorem \ref{thrm: finite dimensional representations of sl2}.
\end{expl}


\begin{prop}
 For all $n \in \N$ let $V^{(n)}$ denote the $(n+1)$-dimensional irreducible representation of $\sll_2(k)$. Then
 \[
  V^{(n)} \otimes V^{(m)} \cong V^{(n+m)} \oplus V^{(n+m-2)} \oplus \dotsb \oplus V^{(|n-m|)}
  \quad\text{for all $n,m \in \N$}.
 \]
\end{prop}
\begin{proof}
 It can be assummed w.l.o.g.\ that $n \geq m$. Abbreviate $V \coloneqq V^{(n)}$ and $W \coloneqq W^{(m)}$. Then $V = \bigoplus_{p=0}^n V_{-n+2p}$ and $W = \bigoplus_{q=0}^m W_{-m+2q}$ and therefore
 \[
  V \otimes W
  = \bigoplus_{p=0}^n \bigoplus_{q=0}^m \underbrace{V_{-n+2p} \otimes W_{-m+2q}}_{\subseteq (V \otimes W)_{-(n+m)+2(p+q)}}.
 \]
 Because $V \otimes W$ decomposes into weight spaces it follows that
 \[
  (V \otimes W)_i
  = \bigoplus_{\substack{j_1 = -n, -n+2, \dotsc, n-2, n \\ j_2 = -m, -m+2 \dotsc, m-2, m \\ p+q=i}} V_p \otimes W_q
  \quad\text{for every $i \in \Z$}.
 \]
 Therefore $\dim (V \otimes W)_i$ is the number of solutions of the equation
 \begin{equation}\label{eqn: Clebsch Gordan by counting}
  p+q = i
  \quad\text{with}\quad
  p \in \{-n, -n+2, \dotsc, n-2, n\},
  q \in \{-m, -m+2, \dotsc, m-2, m\}.
 \end{equation}
 To count these solutions notice that \eqref{eqn: Clebsch Gordan by counting} has no solution for $i < -n-m$ or $i > n+m$. Also notice that $p+q \equiv n+m\pmod(2)$ in \eqref{eqn: Clebsch Gordan by counting}. So it is enough to count the solutions of \eqref{eqn: Clebsch Gordan by counting} for $i = -(n+m), -(n+m)+2, \dotsc, n+m-2, n+m$. The solutions are given in the table in Figure~\ref{fig: solutions for counting dimensions} (page \pageref{fig: solutions for counting dimensions}).
 \begin{figure}
 \[
  \setlength\arraycolsep{4pt}
  \begin{array}{c|c|c}
  i      & \text{solutions $(p,q)$ for $p+q = i$}        & \text{number} \\
  \hline
  n+m    & (n, m)                                        & 1      \\
  n+m-2  & (n, m-2), (n-2, m),                           & 2      \\
  n+m-4  & (n, m-4), (n-2, m-2), (n-4, m),               & 3      \\
  \vdots & \vdots                                        & \vdots \\
  n-m    & (n, -m), (n-2, -m+2), \dotsc,  (n-2m, m)      & m+1    \\
  n-m-2  & (n-2, -m), (n-4, -m+2), \dotsc, (n-2m-2, m)   & m+1    \\
  \vdots & \vdots                                        & \vdots \\
  -n+m+2 & (-n+2m+2,-m), (-n+2m, -m+2), \dotsc, (-n+2,m) & m+1    \\
  -n+m   & (-n+2m,-m), (-n+2m-2, -m+2), \dotsc, (-n,m)   & m+1    \\
  \vdots & \vdots                                        & \vdots \\
  -n-m+4 & (-n+4,-m), (-n+2,-m+2), (-n,-m+4)             & 3      \\
  -n-m+2 & (-n+2,-m), (-n,-m+2)                          & 2      \\
  -n-m   & (-n,-m)                                       & 1
  \end{array}
 \]
 \caption{Solutions for counting dimensions of weight spaces.}
 \label{fig: solutions for counting dimensions}
 \end{figure}
 This results in the following dimensions:
 \[
  \setlength\arraycolsep{2pt}
  \begin{array}{c:c:c:c:c:c:c:c:c:c}
   i \in \Z            & -n-m & -n-m+2 & \cdots & -n+m & \cdots & n-m & \cdots & n+m-2 & n+m \\
   \hline
   \dim(V \otimes W)_i & 1    & 2      & \cdots & m+1  & \cdots & m+1 & \cdots & 2     & 1
  \end{array}
 \]
 From this it follows from Theorem~\ref{thrm: finite dimensional representations of sl2} that
 \[
  V \otimes W \cong V^{(n+m)} \oplus V^{(n+m-2)} \oplus \dotsb \oplus V^{(n-m)}.
  \qedhere
 \]
\end{proof}


We also give another proof, taken from \cite[\S 1.4]{Lectures_on_sl2_modules}.


\begin{proof}
 It can be assummed w.l.o.g.\ that $n \geq m$. The formula can then be shown by induction on $m$. As $\sll_2(k)$ acts trivially on the one-dimensional representation $V^{(0)} \cong \Cbb$ it follows that the map $V^{(n)} \to V^{(n)} \otimes V^{(0)}, v \mapsto v \otimes 1$ is an isomorphism representations. This shows the formula for $m = 0$.
 
 To show the formula for $m = 1$ first notice the following:
 \begin{claim*}
  Let $V$ be a finite dimenisonal representation of $\sll_2(k)$. Suppose there exist a nonzero $v \in V$ with $h.v =  r v$ and $e.v = 0$ for $r \in \N$. Then $V$ contains a subrepresentation which is isomorphic to $V^{(r)}$.
 \end{claim*}
 \begin{proof}
  Let $V = \bigoplus_{d \in \N} \bigoplus_{j=1}^{\nu_d} V^{d,j}$ be a decomposition into irreducible subrepresentations with $V^{d,j}$ having heighest weight $d$ for every $d \in \N$ and $j = 1, \dotsc, \nu_d$. Then $v \in V_r = \bigoplus_{p \in \N, d = r+2p} \bigoplus_{j=0}^{\nu_d} V^{d,j}_r$. Suppose that $V$ contains no submodule which is isomorphic to $V^{(r)}$. Then $\nu_r = 0$ and therefore $V_r = \bigoplus_{p \geq 1, d = r+2p} \bigoplus_{j=1}^{\nu_d} V^{d,j}_r$. Because $e.V^{d,j}_r = V^{d,j}_{r+2}$ for every $d > r$ and $j = 1, \dotsc, \nu_d$ it follows that $e$ maps $V_r = \bigoplus_{p \geq 1, d = r+2p} \bigoplus_{j=1}^{\nu_d} V^{d,j}_r$ isomorphically into $\bigoplus_{p \geq 1, d = r+2p} \bigoplus_{j=1}^{\nu_d} V^{d,j}_{r+2} \subseteq V_{r+2}$. Because $e.v = 0$ it follows that $e = 0$, contradicting the assumption that $v$ is nonzero.
 \end{proof}
 
 Let $v_{-n}, v_{-n+2}, \dotsc, v_{n-2}, v_n$ be a basis of $V^{(n)}$ with $v_i \in V_i$ for every $i$ and $e.v_{n-2} = v_n$ (notice that $n \geq m = 1$, so $v_{n-2}$ is well-defined). Similarly and $w_{-1}, w_1$ a basis of $V^{(1)}$ with $w_j \in W_j$ for $j = -1,1$. Then $v_n \otimes w_1 \in V^{(n)} \otimes V^{(1)}$ with $h.(v_n \otimes w_1) = (n+1) v_n \otimes w_1$ and $e.(v_n \otimes w_1) = 0$. By the previous claim $V^{(n)} \otimes V^{(1)}$ contains a subrepresentation $W^1 \cong V^{(n+1)}$. Similarly $x \coloneqq v_{n-2} \otimes w_1 - v_n \otimes w_2 \in V^{(n)} \otimes V^{(1)}$ with $e.x = 0$ (here it is used that $e.v_{n-2} = v_n$) and $h.x = (n-1)x$. By the claim $V$ contains another subrepresentation $W^2 \cong V^{(n-1)}$. Because $W^1$ and $W^2$ are irreducible and not equal it follows that $W^1 \cap W^2 = 0$. Because
 \[
  \dim W^1 + \dim W^2 = 2n+2
  = \left(\dim V^{(n)}\right) \cdot \left(\dim V^{(1)}\right)
  = \dim\left( V^{(n)} \otimes V^{(1)} \right)
 \]
 it follows that $V = W^1 \oplus W^2 \cong V^{(n+1)} \oplus V^{(n-1)}$. This shows the formula for $m = 1$.
 
 Suppose that $m \geq 2$ and the statement holds for $0, 1, \dotsc, m-1$. Then on the one hand
 \begin{align*}
  V^{(n)} \otimes V^{(m-1)} \otimes V^{(1)}
  &\cong V^{(n)} \otimes \left( V^{(m)} \oplus V^{(m-2)} \right)
  \cong V^{(n)} \otimes V^{(m)} \oplus V^{(n)} \otimes V^{(m-2)} \\
  &\cong V^{(n)} \otimes V^{(m)} \oplus \left( V^{(n+m-2)} \oplus V^{(n+m-4} \oplus \dotsb \otimes V^{(n-m+2)} \right)
 \end{align*}
 while on the other hand
 \begin{align*}
       &\, V^{(n)} \otimes V^{(m-1)} \otimes V^{(1)} \\
  \cong&\, \left( V^{(n+m-1)} \oplus V^{(n+m-3)} \oplus \dotsb \oplus V^{n-m+1} \right) \otimes V^{(1)} \\
  \cong&\, \left( V^{(n+m-1)} \otimes V^{(1)} \right)
           \oplus \left( V^{(n+m-3)} \otimes V^{(1)} \right)
           \oplus \dotsb
           \oplus \left( V^{(n-m+1)} \otimes V^{(1)} \right) \\
  \cong&\, \left( V^{(n+m)} \oplus V^{(n+m-2)} \right)
           \oplus \left( V^{(n+m-2)} \oplus V^{(n+m-4)} \right)
           \oplus \dotsb
           \oplus \left( V^{(n-m+2)} \oplus V^{(n-m)} \right) \\
  \cong&\, \left( V^{(n+m)} \oplus V^{(n+m-2)} \oplus \dotsb \oplus V^{(n-m)} \right) \\
       &\,  \oplus \left( V^{(n+m-2)} \oplus V^{(n+m-4)} \oplus \dotsb \oplus V^{(n-m+2)} \right).
 \end{align*}
 By the uniqueness of the decomposition of $V^{(n)} \otimes V^{(m-1)} \otimes V^{(1)}$ into irreducible subrepresentations is follows that
 \[
  V^{(n)} \otimes V^{(m)} \cong V^{(n+m)} \oplus V^{(n+m-2)} \oplus \dotsb \oplus V^{(n-m)},
 \]
 which finishes the proof.
\end{proof}


\begin{rem}
 \cite{Lectures_on_sl2_modules} has a pretty cool cover and we advise the reader to check it out.
\end{rem}


% TODO: Why it is named after Clebsch and Gordan and its origin in invariant theory.






% TODO: Remark about point diagrams


% TODO: Remark about dual representation for nice basis


% TODO: Decomposition of the tensor product












