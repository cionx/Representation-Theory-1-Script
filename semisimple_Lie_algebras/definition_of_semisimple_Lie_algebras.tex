\section{Definition and basic properties}


\begin{defi}
 A Lie algebra $\g$ is called \emph{semisimple} if it is the sum of finitely many simple ideals, i.e.\ if there exists ideals $I_1, \dotsc, I_n \subideal \g$ which are simple (as Lie algebras) such that $\g = I_1 \oplus \dotsb \oplus I_n$.
\end{defi}


\begin{lem}
 Let $\g$ be a semisimple Lie algebra. Then $[\g,\g] = \g$ and $Z(\g) = 0$.
\end{lem}
\begin{proof}
 Let $I_1, \dotsc, I_n \subideal \g$ be simple ideals such that $\g = I_1 \oplus \dotsb \oplus I_n$. Then
 \begin{gather*}
  [\g,\g] = [I_1, I_1] \oplus \dotsb \oplus [I_n, I_n] = I_1 \oplus \dotsb \oplus I_n = \g
 \shortintertext{as well as}
  Z(\g) = Z(I_1) \oplus \dotsb \oplus Z(I_n) = 0 \oplus \dotsb \oplus 0 = 0.
  \qedhere
 \end{gather*}
\end{proof}


\begin{cor}\label{cor: decomposition of semisimple Lie algebra unique up to order}
 Let $\g$ be a semisimple Lie algebra and $I_1, \dotsc, I_n \subideal \g$ simple ideals with $\g = I_1 \oplus \dotsb \oplus I_n$. Then the ideals $I_1, \dotsc, I_n$ are unique up to reordering.
\end{cor}
\begin{proof}
 Let $J_1, \dotsc, J_m \subideal \g$ be simple ideals with $\g = J_1 \oplus \dotsm \oplus J_m$. Then
 \[
  \g
  = [\g,\g]
  = [I_1 \oplus \dotsb \oplus I_n, J_1 \oplus \dotsb \oplus J_m]
  = \bigoplus_{i=1}^n \bigoplus_{j=1}^m [I_i, J_j].
 \]
 For every $i = 1, \dotsc, n$ and $j = 1, \dotsc, m$ the ideals $I_i$ and $J_j$ are simple, which is why
 \[
  [I_i, J_j] =
  \begin{cases}
   I_i & \text{if $I_i = J_j$}, \\
     0 & \text{otherwise}.
  \end{cases}
 \]
 It follows that for every $i = 1, \dotsc, n$ there exists some $1 \leq j \leq m$ with $I_i = J_j$, and similarly that for every $j = 1, \dotsc, m$ there exists some $1 \leq i \leq n$ with $J_j = I_i$. Hence $n = m$ and the statement follows.
\end{proof}


\begin{rem}
 If $\g$ is a semisimple Lie algebra then we will often just talk about \emph{the} decomposition $\g = I_1 \oplus \dotsb \oplus I_n$ into simple ideals, as this decomposition is unique up to reordering of the summands by Corollary~\ref{cor: decomposition of semisimple Lie algebra unique up to order}.
\end{rem}


\begin{thrm}\label{thrm: characterisations of finite-dimensional semisimple Lie algebras}
 For a finite dimensional Lie algebra $\g$ the following are equivalent:
 \begin{enumerate}[leftmargin=*]
  \item\label{enum: definition semisimple Lie algebra product of simple Lie algebras}
   $\g \cong \g_1 \times \dotsb \times \g_r$ for some $r \in \N$ and simple Lie algebras $\g_i$, $i = 1, \dotsc, r$.
  \item\label{enum: definition semisimple Lie algebra sum of simple ideals}
   $\g = I_1 \oplus \dotsb \oplus I_s$ for some $s \in \N$ and simple Ideals $I_1, \dotsc, I_s \subideal \g$.
  \item\label{enum: definition semisimple Lie algebra killing form nondegenerate}
   The Killing form $\kappa$ of $\g$ is non-degenerate (which is equivalent to $\rad \kappa = 0$).
  \item\label{enum: definition semisimple Lie algebra radical is zero}
   $\g$ has no nonzero solvable ideals (which is equivalent to $\rad \g = 0$).
  \item\label{enum: definition semisimple Lie algebra no nonzero abelian ideals}
   $\g$ has no nonzero abelian ideals.
 \end{enumerate}
\end{thrm}
\begin{proof}
 (\ref{enum: definition semisimple Lie algebra radical is zero} $\Rightarrow$ \ref{enum: definition semisimple Lie algebra killing form nondegenerate}) This directly follows from the fact that $\rad \kappa$ is a solvable ideal in $\g$.
 
 (\ref{enum: definition semisimple Lie algebra killing form nondegenerate} $\Rightarrow$ \ref{enum: definition semisimple Lie algebra sum of simple ideals}) The implication can be shown by induction over $\dim \g$. If $\dim \g = 0$ then $\g = 0$ is the empty sum over zero simple ideals in $\g$. Suppose that $\dim \g \geq 1$ and the implication holds for all smaller dimensions. If $\g$ is simple then there is nothing left to show. Otherwise $\g$ contains a non-trivial ideal $I \subideal \g$, i.e.\ $I \neq 0$ and $I \neq \g$. Then
 \[
  I^\perp \coloneqq \{y \in \g \mid \text{$\kappa(x,y) = 0$ for every $x \in I$}\}
 \]
 is an ideal in $\g$ and because $\kappa$ is non-degenerate it follows that $\dim \g = \dim I + \dim I^\perp$. Because $I \cap I^\perp$ is an ideal in $\g$ it also follows that
 \[
  \kappa_{I \cap I^\perp}(x,y) = \kappa(x,y) = 0 \quad \text{for all $x,y \in I \cap I^\perp$}.
 \]
 By Cartan’s criterion $I \cap I^\perp$ is a solvable ideal in $\g$, from which it follows from the previous implication (\ref{enum: definition semisimple Lie algebra radical is zero} $\Rightarrow$ \ref{enum: definition semisimple Lie algebra killing form nondegenerate}) that $I \cap I^\perp = 0$. Therefore $\g = I \oplus I^\perp$, where $I$ and $I^\perp$ are proper ideals in $\g$. By Lemma \ref{lem: orthogonal ideals with respect to the killing form} the Killing form $\kappa$ is given by the sum of the Killing forms $\kappa_{I_1}$ and $\kappa_{I_2}$. AS $\kappa$ is non-degenerate it follows that the same goes for $\kappa_{I_1}$ and $\kappa_{I_2}$. Hence by induction hypothesis both $I_1$ and $I_2$ are the sum of simple ideals $I_1 = J_1 \oplus \dotsb \oplus J_r$ and $I_2 = K_1 \oplus \dotsb \oplus K_s$. It follows that
 \[
  \g = I_1 \oplus I_2 = J_1 \oplus \dotsb J_r \oplus K_1 \oplus \dotsb \oplus K_s
 \]
 is a decomposition into simple ideals.

 (\ref{enum: definition semisimple Lie algebra sum of simple ideals} $\Rightarrow$ \ref{enum: definition semisimple Lie algebra product of simple Lie algebras}) Follows from $\g = I_1 \oplus \dotsb \oplus I_s \cong I_1 \times \dotsb \times I_s$.
 
 (\ref{enum: definition semisimple Lie algebra product of simple Lie algebras} $\Rightarrow$ \ref{enum: definition semisimple Lie algebra radical is zero}) For each $i = 1, \dotsc, r$ let $\pi_i \colon \g \to \g_i$ be the canonical projection and let $I \subideal \g$ be a solvable ideal. Then for any $i = 1, \dotsc, r$ the image $\pi_i(I) \subseteq \g_i$ is a solvable ideal. Because $\g_i$ is simple it follows that $\pi_i(I) = 0$ for every $i = 1, \dotsc, n$. Hence $I = 0$.
 
 (\ref{enum: definition semisimple Lie algebra radical is zero} $\Rightarrow$ \ref{enum: definition semisimple Lie algebra no nonzero abelian ideals}) This directly follows from the fact that every abelian ideal is solvable.
 
 (\ref{enum: definition semisimple Lie algebra no nonzero abelian ideals} $\Rightarrow$ \ref{enum: definition semisimple Lie algebra radical is zero}) Suppose that there exists a nonzero solvable ideal $I \subideal \g$. Then let $i \geq 0$ such that $I^{(i+1)} = 0$ but $I^{(i)} \neq 0$. Then $I^{(i)}$ is a nonzero abelian ideal in $\g$.
\end{proof}


\begin{cor}
 Let $\g$ be a semisimple Lie algebra and $\g = I_1 \oplus \dotsb \oplus I_n$ the de\-com\-po\-si\-tion into simple ideals. If $I \subseteq \g$ is any ideal, then $I = I_{i_1} \oplus \dotsb \oplus I_{i_m}$ for some indices $1 \leq i_1 < \dotsb < i_m \leq n$.
\end{cor}
\begin{proof}
 The statement holds if $I = \g$ or $I = 0$, hence it can be assumed that $I$ is a non-trivial ideal of $\g$. From the proof of the implication (\ref{enum: definition semisimple Lie algebra killing form nondegenerate} $\Rightarrow$ \ref{enum: definition semisimple Lie algebra sum of simple ideals}) it follows that there exists a decomposition $\g = J_1 \oplus \dotsb \oplus J_m$ into simple ideals such that $I = J_1 \oplus \dotsb \oplus J_\ell$ for some $1 \leq \ell \leq m$. As such a decomposition is unique up to reordering of the summands by Corollary~\ref{cor: decomposition of semisimple Lie algebra unique up to order} the statement follows.
\end{proof}


\begin{cor}\label{cor: ideals and quotients of semisimple again semisimple}
 If $\g$ is a semisimple Lie algebra and $I \subideal \g$ any ideal then both $I$ and $\g/I$ are also semisimple.
\end{cor}


% TODO: Subalgebras not necessarily also semisimple


\begin{cor}\label{cor: semisimple Lie algebra identification with dual space}
 Let $\g$ be a finite-dimensional, semisimple Lie algebra. Then the map
 \[
  \g \to \g^*, \quad x \mapsto \kappa(x,\cdot)
 \]
 is an isomorphism of representations of $\g$.
\end{cor}
\begin{proof}
 The statement follows from Corollary \ref{cor: associative non-degenerate bilinear forms induce isomorphism to the dual} because $\kappa$ is non-degenerate.
\end{proof}


\begin{cor}
 Let $\g$ be a finite dimensional simple Lie algebra and $\beta \colon \g \times \g \to k$ an associative bilinear form. Then $\beta$ is a scalar multiple of the killing form $\kappa \colon \g \times \g \to k$.
\end{cor}
\begin{proof}
 Because $\beta$ is associative the map
 \[
  \varphi \colon \g \to \g^*, \quad x \mapsto \beta(x, \cdot)
 \]
 is a homomorphism of representations of $\g$ by Lemma~\ref{lem: associative bilinear form induces homomorphism of representations}. Because $\g$ is simple, and therefore also semisimple, the Killing form $\kappa$ is non-degenerate. Therefore the map
 \[
  \psi \colon \g \to \g^*, \quad x \mapsto \kappa(x, \cdot)
 \]
 is an isomorphism of representations of $\g$ by Corollary~\ref{cor: semisimple Lie algebra identification with dual space}. It follows that
 \[
  \alpha \coloneqq \varphi \circ \psi^{-1} \colon \g \to \g, 
 \]
 is a homomorphism of representations of $\g$. Because $\g$ is simple the adjoint representation of $\g$ is irreducible. By Schur’s Lemma $\alpha$ is given by multiplication with a scalar $\lambda \in k$. It follows that $\varphi = \lambda \psi$.
\end{proof}














