\section{Cartan subalgebras}
Throughout this section $\g$ denotes the finite dimensional semisimple Lie algebra.





\subsection{Definition and root space decomposition}


\begin{defi}
 A subalgebra $\h \subseteq \g$ is \emph{toral} if it consists of semisimple elements.
\end{defi}


\begin{lem}
 Let $\h \subseteq \g$ be a toral subalgebra. Then $\h$ is abelian.
\end{lem}
\begin{proof}
 Let $x \in \h$. Because $x$ is $\ad_\g$-semisimple and $\h$ is $\ad_\g(x)$-invariant it follows that $\ad_\h(x) = \ad_\g(x)|_\h$ is also semisimple. It is enough to show that all eigenvalues of $\ad_\h(x)$ are zero.
 
 Let $y \in \h$ be an eigenvector of $\ad_\h(x)$ with eigenvalue $\mu$ (in particular $y \neq 0$). In the same way as for $\ad_\h(x)$ it follows that $\ad_\h(y)$ is semisimple. Fer every $\lambda \in k$ let $\h_\lambda$ denote the eigenpsace of $\ad_\h(y)$ with respect to the eigenvalue $\lambda$.
 
  On the one hand $[y,x] \in \h_0$ because
 \[
  \ad_\h(y)([y,x])
  = [y,[y,x]]
  = [y, -\mu y]
  = 0.
 \]
 On the other hand
 \[
  [y,x]
  = \ad_\h(y)(x)
  \in \ad_\h(y)(\h)
  = \ad_\h(y)\left( \bigoplus_{\lambda \in k} \h_\lambda \right)
  = \bigoplus_{\lambda \neq 0} \h_\lambda.
 \]
 By the directness of the sum $\h = \bigoplus_{\lambda \in k} \h_\lambda$ it follows that $0 = [y,x] = -\mu y$. Because $y \neq 0$ it follows that $\mu = 0$.
\end{proof}


\begin{defi}
 A \emph{Cartan subalgebra} of $\g$ is a maximal toral subalgebra.
\end{defi}


\begin{rem}
 The toral subalgebra $0 \subseteq \g$ is contained in a toral subalgebra (of $\g$) of maximal dimensional, which is then a Cartan subalgebra. Therefore $\g$ contains a Cartan subalgebra.
 
 Also notice that if $\g \neq 0$ then any Cartan subalgebra of $\g$ is non-zero. Too see this first notice thet $\g$ then contains a non-zero semisimple element $x \in \g$, because otherwise $\g$ would be nilpotent by Engel’s Theorem. Then the one-dimensional linear subspace $kx$ is a toral subalgebra properly containing $0$.
\end{rem}


% TODO: Remark about usual definition.


\begin{defi}
 Let $\h$ be a Cartan subalgebra of $\g$. For any $\alpha \in \h^*$ let
 \[
  \g_\alpha \coloneqq \{y \in \g \mid \text{$[x,y] = \alpha(x)y$ for every $x \in \h$}\}
 \]
 be the \emph{root space} of $\g$ with respect to $\alpha$. Then $\Phi(\g, \h) \coloneqq \{\alpha \in \h^* \setminus \{0\} \mid \g_\alpha \neq 0\}$ is the set of \emph{roots} of $\g$ with respect to $\h$.
\end{defi}


\begin{rem}
 Notice that $\g_0 = \{y \in \g \mid \text{$[x,y] = 0$ for every $x \in \h$}\} = Z_\g(\h)$.
\end{rem}



\begin{lem}
 Let $\h$ be a Cartan subalgebra of $\g$ with roots $\Phi \coloneqq \Phi(\g, \h)$. Then $\g = Z_\g(\h) \oplus \bigoplus_{\alpha \in \Phi} \g_\alpha$.
\end{lem}
\begin{proof}
 Because $\h$ is abelian the same goes for $\ad_\g(\h) \subseteq \gl(V)$. Therefore $\ad_\g(\h)$ consists of semisimple pairwise commuting endomorphisms. Hence $\ad_\g(\h)$ is simultaneously diagonalizable, which is why
 \[
  \g
  = \bigoplus_{\lambda \in \h^*} \g_\lambda
  = \g_0 \oplus \bigoplus_{\alpha \in \Phi} \g_\alpha
  = Z_\g(\h) \oplus \bigoplus_{\alpha \in \Phi} \g_\alpha.
  \qedhere
 \]
\end{proof}


% TODO: Beispiele hinzufügen


\begin{lem}
 Let $\h$ be a Cartan subalgebra of $\g$. Then $[\g_\alpha, \g_\beta] \subseteq \g_{\alpha+\beta}$ for all $\alpha, \beta \in \h^*$.
\end{lem}
\begin{proof}
 Let $x \in \g_\alpha$ and $y \in \g_\beta$. Then for every $h \in \h$
 \[
  [h,[x,y]]
  = [[h,x],y] + [x,[h,y]]
  = \alpha(h)[x,y] + \beta(h)[x,y]
  = (\alpha+\beta)(h) [x,y].
  \qedhere
 \]
\end{proof}


\begin{lem}
Let $\h \subseteq \g$ be a Cartan subalgebra and $\alpha, \beta \in \h^*$. If $\alpha \neq -\beta$ then $\g_\alpha$ and $\g_\beta$ are orthogonal with respect to the Killing form.
\end{lem}
\begin{proof}
 Because $\alpha \neq -\beta$ it follows that there exists $h \in \h$ with $(\alpha+\beta)(h) \neq 0$. For every $x \in \g_\alpha$ and $y \in \g_\beta$ it then follows that
 \[
  \alpha(h) \kappa(x,y)
  = \kappa([h,x],y)
  = -\kappa([x,h],y)
  = -\kappa(x,[h,y])
  = -\beta(h)\kappa(x,y)
 \]
 and therefore $(\alpha+\beta)(h)\kappa(x,y) = 0$. Because $(\alpha+\beta)(h) \neq 0$ it follows that $\kappa(x,y) = 0$ for every $x \in \g_\alpha$ and $y \in \g_\beta$.
\end{proof}


\begin{cor}\label{cor: restriction of killing form to centralizer is non-degenerate}
 Let $\h \subseteq \g$ be a Cartan subalgebra. Then $\kappa_\g|_{Z_\g(\h) \times Z_\g(\h)}$ is non-degenerate.
\end{cor}
\begin{proof}
 Because $\g$ is semisimple $\kappa_\g$ is non-degenerate. Because $Z_\g(\h) = \g_0$ is orthogonal to $\g_\alpha$ for every $\alpha \in \Phi$ with $\alpha \neq 0$ the statement follows from $\g = Z_\g(\h) \oplus \bigoplus_{\alpha \in \Phi} \g_\alpha$.
\end{proof}


% TODO: Add warning about restriction of Killing form


\begin{prop}\label{prop: CSA are self-centralizing}
 Let $\h$ be a Cartan subalgebra of $\g$. Then $Z_\g(\h) = \h$, i.e.\ $\h$ is self-centralizing.
\end{prop}
\begin{proof}
 Throughout this proof abbreviate $\cl \coloneqq Z_\g(\h)$, $\ad \coloneqq \ad_\g$ and $\kappa \coloneqq \kappa_\g$.

 \begin{claim*}\label{claim: technical and traceless}
  Let $x \in \g$ be nilpotent and $y \in \g$ commuting with $x$. Then $\kappa(x,y) = 0$.
 \end{claim*}
 \begin{proof}
  Because $x$ and $y$ commute so do $\ad(x)$ and $\ad(y)$. Because $\ad(x)$ is nilpotent it follows that $\ad(x)\ad(y)$ is also nilpotent. Therefore $\kappa(x,y) = \tr(\ad(x)\ad(y)) = 0$.
 \end{proof}
 
 Start by noticing that $\cl$ contains the semisimple and nilpotent parts of all its elements: If $x \in \cl$ then $y \in \g$ commutes with $x$ if and only if it commutes with both $y_s$ and $y_n$. Therefore $y \in \cl$ if and only if $y_s, y_n \in \cl$.
 
 Let $s \in \cl$ be semisimple. Because $s$ is semisimple and commutes with $\h$ it then follows that $\h + ks$ is a Lie subalgebra of $\g$ consisting of semisimple elements. By the maximality of $\h$ it follows that $s \in \h$.
 
 Let $x \in \cl$ and let $x = x_s + x_n$ be the Jordan decomposition of $x$. It was already shown above that $x_s \in \h$, so $\ad_\cl x_s = 0$. Hence $\ad_\cl x = \ad_\cl x_n = \ad_\g x_n|_\cl$ is nilpotent. By Engel’s Theorem $\cl$ is nilpotent.
 
 Notice that $\kappa|_{\h \times \h}$ is non-degenerate. To see this let $x \in \h$ with $\kappa(x, \h) = 0$. It needs to be shown that $x = 0$, and for this it sufficies to show that $\kappa(x, \cl) = 0$ by Corollary~\ref{cor: restriction of killing form to centralizer is non-degenerate}. Because $\cl$ contains the semisimple and nilpotent parts of all its elements is sufffices to show that $\kappa(x,s) = \kappa(x,n) = 0$ for every semisimple $s \in \cl$ and nilpotent $n \in \cl$. It was already shown that $s \in \h$ so $\kappa(x,s) = 0$ by assumption. That $\kappa(x,n) = 0$ follows from the claim above.
 
 It follows that $\h \cap [\cl,\cl] = 0$ because $[\h,\cl] = 0$ and thus $\kappa(\h, [\cl,\cl]) = \kappa([\h, \cl], \cl) = 0$. It further follows that $\cl$ is abelian. Otherwise $\cl$ is nilpotent with $[\cl, \cl] \neq 0$. Then $Z(\cl) \cap [\cl,\cl] \neq 0$ by Corollary~\ref{cor: ideal in nilpotent has nontrivial intersection with center}. Let $x \in Z(\cl) \cap [\cl,\cl]$ be non-zero. Notice that $x$ cannot be semisimple because $\h \cap [\cl,\cl] = 0$. So $x_n$ is nonzero and it was already shown that $x_n \in \cl$. Because $x \in Z(\cl)$ it follows that $x_n \in Z(\cl)$. From the claim above it follows that $\kappa(x_n, \cl) = 0$. Because $\kappa_{\cl \times \cl}$ is non-degenerate it follows that $x_n = 0$, contradicting that $x_n$ is non-zero.
 
 Suppose now that $\h \subsetneq \cl$. Let $x \in \cl$ with $\h \neq 0$. Because $x_s, x_n \in \cl$ with $x_s \in \h$ it can be assumed w.l.o.g.\ that $x$ is nilpotent by replacing it with $x_n$. Then $\kappa(x_n, \cl) = 0$ by the above claim, contradicting $\kappa_{\cl \times \cl}$ being non-degenerate.
\end{proof}


\begin{cor}
 Let $\h \subseteq \g$ be a Cartan subalgebra. Then the restriction $\kappa_\g|_{\h \times \h}$ is non-degenerate and $\g = \h \oplus \bigoplus_{\alpha \in \Phi(\g,\h)} \g_\alpha$.
\end{cor}


\begin{rem}
 The proof of Proposition~\ref{prop: CSA are self-centralizing} is taken from \cite[\S 8.2]{Humphreys}.
\end{rem}


\begin{defi}
 Let $\h \subseteq \g$ be a Cartan subalgebra. Then the decomposition
 \[
  \g = \h \oplus \bigoplus_{\alpha \in \Phi} \g_\alpha
  \quad\text{with}\quad
  \Phi \coloneqq \Phi(\g,\h)
 \]
 is the \emph{root space decomposition} of $\g$ with respect to $\h$.
\end{defi}

























