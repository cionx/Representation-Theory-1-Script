\section{Cartan subalgebras}
Throughout this section $\g$ denotes the finite dimensional semisimple Lie algebra.





\subsection{Definition and root space decomposition}


\begin{defi}
 A subalgebra $\h \subseteq \g$ is \emph{toral} if it consists of semisimple elements.
\end{defi}


\begin{lem}
 Let $\h \subseteq \g$ be a toral subalgebra. Then $\h$ is abelian.
\end{lem}
\begin{proof}
 Let $x \in \h$. Because $x$ is $\ad_\g$-semisimple and $\h$ is $\ad_\g(x)$-invariant it follows that $\ad_\h(x) = \ad_\g(x)|_\h$ is also semisimple. It is enough to show that all eigenvalues of $\ad_\h(x)$ are zero.
 
 Let $y \in \h$ be an eigenvector of $\ad_\h(x)$ with eigenvalue $\mu$ (in particular $y \neq 0$). In the same way as for $\ad_\h(x)$ it follows that $\ad_\h(y)$ is semisimple. Fer every $\lambda \in k$ let $\h_\lambda$ denote the eigenpsace of $\ad_\h(y)$ with respect to the eigenvalue $\lambda$.
 
  On the one hand $[y,x] \in \h_0$ because
 \[
  \ad_\h(y)([y,x])
  = [y,[y,x]]
  = [y, -\mu y]
  = 0.
 \]
 On the other hand
 \[
  [y,x]
  = \ad_\h(y)(x)
  \in \ad_\h(y)(\h)
  = \ad_\h(y)\left( \bigoplus_{\lambda \in k} \h_\lambda \right)
  = \bigoplus_{\lambda \neq 0} \h_\lambda.
 \]
 By the directness of the sum $\h = \bigoplus_{\lambda \in k} \h_\lambda$ it follows that $0 = [y,x] = -\mu y$. Because $y \neq 0$ it follows that $\mu = 0$.
\end{proof}


\begin{defi}
 A \emph{Cartan subalgebra} of $\g$ is a maximal toral subalgebra.
\end{defi}


\begin{rem}
 The toral subalgebra $0 \subseteq \g$ is contained in a toral subalgebra (of $\g$) of maximal dimensional, which is then a Cartan subalgebra. Therefore $\g$ contains a Cartan subalgebra.
 
 Also notice that if $\g \neq 0$ then any Cartan subalgebra of $\g$ is non-zero. Too see this first notice thet $\g$ then contains a non-zero semisimple element $x \in \g$, because otherwise $\g$ would be nilpotent by Engel’s Theorem. Then the one-dimensional linear subspace $kx$ is a toral subalgebra properly containing $0$.
\end{rem}


% TODO: Remark about usual definition.


\begin{defi}
 Let $\h$ be a Cartan subalgebra of $\g$. For any $\alpha \in \h^*$ let
 \[
  \g_\alpha \coloneqq \{y \in \g \mid \text{$[x,y] = \alpha(x)y$ for every $x \in \h$}\}
 \]
 be the \emph{root space} of $\g$ with respect to $\alpha$. Then $\Phi(\g, \h) \coloneqq \{\alpha \in \h^* \setminus \{0\} \mid \g_\alpha \neq 0\}$ is the set of \emph{roots} of $\g$ with respect to $\h$.
\end{defi}


\begin{rem}
 Notice that $\g_0 = \{y \in \g \mid \text{$[x,y] = 0$ for every $x \in \h$}\} = Z_\g(\h)$.
\end{rem}



\begin{lem}
 Let $\h$ be a Cartan subalgebra of $\g$ with roots $\Phi \coloneqq \Phi(\g, \h)$. Then $\g = Z_\g(\h) \oplus \bigoplus_{\alpha \in \Phi} \g_\alpha$.
\end{lem}
\begin{proof}
 Because $\h$ is abelian the same goes for $\ad_\g(\h) \subseteq \gl(V)$. Therefore $\ad_\g(\h)$ consists of semisimple pairwise commuting endomorphisms. Hence $\ad_\g(\h)$ is simultaneously diagonalizable, which is why
 \[
  \g
  = \bigoplus_{\lambda \in \h^*} \g_\lambda
  = \g_0 \oplus \bigoplus_{\alpha \in \Phi} \g_\alpha
  = Z_\g(\h) \oplus \bigoplus_{\alpha \in \Phi} \g_\alpha.
  \qedhere
 \]
\end{proof}


% TODO: Beispiele hinzufügen


\begin{lem}
 Let $\h$ be a Cartan subalgebra of $\g$. Then $[\g_\alpha, \g_\beta] \subseteq \g_{\alpha+\beta}$ for all $\alpha, \beta \in \h^*$.
\end{lem}
\begin{proof}
 Let $x \in \g_\alpha$ and $y \in \g_\beta$. Then for every $h \in \h$
 \[
  [h,[x,y]]
  = [[h,x],y] + [x,[h,y]]
  = \alpha(h)[x,y] + \beta(h)[x,y]
  = (\alpha+\beta)(h) [x,y].
  \qedhere
 \]
\end{proof}


\begin{lem}\label{lem: root spaces orthogonal with respect to the Killing form}
Let $\h \subseteq \g$ be a Cartan subalgebra and $\alpha, \beta \in \h^*$. If $\alpha \neq -\beta$ then $\g_\alpha$ and $\g_\beta$ are orthogonal with respect to the Killing form.
\end{lem}
\begin{proof}
 Because $\alpha \neq -\beta$ it follows that there exists $h \in \h$ with $(\alpha+\beta)(h) \neq 0$. For every $x \in \g_\alpha$ and $y \in \g_\beta$ it then follows that
 \[
  \alpha(h) \kappa(x,y)
  = \kappa([h,x],y)
  = -\kappa([x,h],y)
  = -\kappa(x,[h,y])
  = -\beta(h)\kappa(x,y)
 \]
 and therefore $(\alpha+\beta)(h)\kappa(x,y) = 0$. Because $(\alpha+\beta)(h) \neq 0$ it follows that $\kappa(x,y) = 0$ for every $x \in \g_\alpha$ and $y \in \g_\beta$.
\end{proof}


\begin{cor}\label{cor: restriction of killing form to centralizer is non-degenerate}
 Let $\h \subseteq \g$ be a Cartan subalgebra. Then $\kappa_\g|_{Z_\g(\h) \times Z_\g(\h)}$ is non-degenerate.
\end{cor}
\begin{proof}
 Because $\g$ is semisimple $\kappa_\g$ is non-degenerate. Because $Z_\g(\h) = \g_0$ is orthogonal to $\g_\alpha$ for every $\alpha \in \Phi$ with $\alpha \neq 0$ the statement follows from $\g = Z_\g(\h) \oplus \bigoplus_{\alpha \in \Phi} \g_\alpha$.
\end{proof}


% TODO: Add warning about restriction of Killing form


\begin{prop}\label{prop: CSA are self-centralizing}
 Let $\h$ be a Cartan subalgebra of $\g$. Then $Z_\g(\h) = \h$, i.e.\ $\h$ is self-centralizing.
\end{prop}
\begin{proof}
 Throughout this proof abbreviate $\cl \coloneqq Z_\g(\h)$, $\ad \coloneqq \ad_\g$ and $\kappa \coloneqq \kappa_\g$.

 \begin{claim*}\label{claim: technical and traceless}
  Let $x \in \g$ be nilpotent and $y \in \g$ commuting with $x$. Then $\kappa(x,y) = 0$.
 \end{claim*}
 \begin{proof}
  Because $x$ and $y$ commute so do $\ad(x)$ and $\ad(y)$. Because $\ad(x)$ is nilpotent it follows that $\ad(x)\ad(y)$ is also nilpotent. Therefore $\kappa(x,y) = \tr(\ad(x)\ad(y)) = 0$.
 \end{proof}
 
 Start by noticing that $\cl$ contains the semisimple and nilpotent parts of all its elements: If $x \in \cl$ then $y \in \g$ commutes with $x$ if and only if it commutes with both $y_s$ and $y_n$. Therefore $y \in \cl$ if and only if $y_s, y_n \in \cl$.
 
 Let $s \in \cl$ be semisimple. Because $s$ is semisimple and commutes with $\h$ it then follows that $\h + ks$ is a Lie subalgebra of $\g$ consisting of semisimple elements. By the maximality of $\h$ it follows that $s \in \h$.
 
 Let $x \in \cl$ and let $x = x_s + x_n$ be the Jordan decomposition of $x$. It was already shown above that $x_s \in \h$, so $\ad_\cl x_s = 0$. Hence $\ad_\cl x = \ad_\cl x_n = \ad_\g x_n|_\cl$ is nilpotent. By Engel’s Theorem $\cl$ is nilpotent.
 
 Notice that $\kappa|_{\h \times \h}$ is non-degenerate. To see this let $x \in \h$ with $\kappa(x, \h) = 0$. It needs to be shown that $x = 0$, and for this it sufficies to show that $\kappa(x, \cl) = 0$ by Corollary~\ref{cor: restriction of killing form to centralizer is non-degenerate}. Because $\cl$ contains the semisimple and nilpotent parts of all its elements is sufffices to show that $\kappa(x,s) = \kappa(x,n) = 0$ for every semisimple $s \in \cl$ and nilpotent $n \in \cl$. It was already shown that $s \in \h$ so $\kappa(x,s) = 0$ by assumption. That $\kappa(x,n) = 0$ follows from the claim above.
 
 It follows that $\h \cap [\cl,\cl] = 0$ because $[\h,\cl] = 0$ and thus $\kappa(\h, [\cl,\cl]) = \kappa([\h, \cl], \cl) = 0$. It further follows that $\cl$ is abelian. Otherwise $\cl$ is nilpotent with $[\cl, \cl] \neq 0$. Then $Z(\cl) \cap [\cl,\cl] \neq 0$ by Corollary~\ref{cor: ideal in nilpotent has nontrivial intersection with center}. Let $x \in Z(\cl) \cap [\cl,\cl]$ be non-zero. Notice that $x$ cannot be semisimple because $\h \cap [\cl,\cl] = 0$. So $x_n$ is nonzero and it was already shown that $x_n \in \cl$. Because $x \in Z(\cl)$ it follows that $x_n \in Z(\cl)$. From the claim above it follows that $\kappa(x_n, \cl) = 0$. Because $\kappa_{\cl \times \cl}$ is non-degenerate it follows that $x_n = 0$, contradicting that $x_n$ is non-zero.
 
 Suppose now that $\h \subsetneq \cl$. Let $x \in \cl$ with $\h \neq 0$. Because $x_s, x_n \in \cl$ with $x_s \in \h$ it can be assumed w.l.o.g.\ that $x$ is nilpotent by replacing it with $x_n$. Then $\kappa(x_n, \cl) = 0$ by the above claim, contradicting $\kappa_{\cl \times \cl}$ being non-degenerate.
\end{proof}


\begin{cor}
 Let $\h \subseteq \g$ be a Cartan subalgebra. Then the restriction $\kappa_\g|_{\h \times \h}$ is non-degenerate and $\g = \h \oplus \bigoplus_{\alpha \in \Phi(\g,\h)} \g_\alpha$.
\end{cor}


\begin{cor}\label{cor: g alpha and g -alpha pair non degenerate with the Killing form}
 Let $\h \subseteq \g$ be a Cartan subalgebra and $\alpha \in \Phi(\g,\h)$. Then $\kappa(\g_\alpha, \g_{-\alpha}) \neq 0$, i.e.\ there exists $x \in \g_\alpha$ and $y \in \g_{-\alpha}$ with $\kappa(x,y) \neq 0$.
\end{cor}
\begin{proof}
 Because $\alpha \in \Phi(\g,\h)$ it follows that $\g_\alpha \neq 0$. Because with respect to the Killing form $\g_\alpha$ is orthogonal to $\g_\beta$ for $\beta \neq -\alpha$ and $\h = \g_0$ by Lemma~\ref{lem: root spaces orthogonal with respect to the Killing form} and $\kappa|_{\h \times \h}$ is non-degenerate is follows that $0 \neq \kappa(\g_\alpha, \g) = \kappa(\g_\alpha, \g_{-\alpha})$.
\end{proof}


\begin{rem}
 The proof of Proposition~\ref{prop: CSA are self-centralizing} is taken from \cite[\S 8.2]{Humphreys}.
\end{rem}


\begin{defi}
 Let $\h \subseteq \g$ be a Cartan subalgebra. Then the decomposition
 \[
  \g = \h \oplus \bigoplus_{\alpha \in \Phi} \g_\alpha
  \quad\text{with}\quad
  \Phi \coloneqq \Phi(\g,\h)
 \]
 is the \emph{root space decomposition} of $\g$ with respect to $\h$.
\end{defi}


\subsection{Properties of the root space decomposition}
Troughout this subsection let $\h \subseteq \g$ be a Cartan subalgebra and $\Phi \coloneqq \Phi(\g,\h)$ the associated roots.


\begin{defi}
 Becaus $\kappa|_{\h \times \h}$ is non-degenerate the map
 \[
  \h \to \h^*, \quad h \mapsto \kappa(h, \cdot)
 \]
 is an isomorphism of vector spaces. For every $\phi \in \h^*$ let $t_\phi \in \h$ be the unique element with $\kappa(t_\phi, \cdot) = \phi$.
\end{defi}


\begin{prop}
 \begin{enumerate}[leftmargin=*]
  \item 
   $\Phi$ generates $\h^*$ as a vector space.
  \item
   If $\alpha \in \Phi$ then $-\alpha \in \Phi$.
  \item
   $[x,y] = \kappa(x,y) t_\alpha$ for every $\alpha \in \Phi$, $x \in \g_\alpha$ and $y \in \g_{-\alpha}$.
  \item
   Lef $\alpha \in \Phi$. Then $[\g_\alpha, \g_{-\alpha}]$ is one-dimensional and has $t_\alpha$ as basis.
  \item
   If $\alpha \in \Phi$ then $\alpha(t_\alpha) = \kappa(t_\alpha, t_\alpha) \neq 0$.
 \end{enumerate}
\end{prop}
\begin{proof}
 \begin{enumerate}[leftmargin=*]
  \item
   Suppose $\Phi$ does not generates $\h^*$ as a vector space. Then there exists some $\phi \in \h^*$ with $\phi \notin \vspan_k \Phi$. Then there exists some $y \in \h^{**}$ with $y|_{\vspan_k \Phi} = 0$ but $y(\phi) \neq 0$. By the natural isomorphism $\h \xrightarrow{\sim} \h^{**}, x \mapsto (\psi \mapsto \psi(x))$ there exists some $x \in \h$ with $y(\psi) = \psi(x)$ for every $\psi \in \h^*$. In particular $\alpha(x) = 0$ for every $\alpha \in \h^*$ but $\phi(x) \neq 0$ and thus $x \neq 0$. Using the root space decomposition $\g = \h \oplus \bigoplus_{\alpha \in \Phi} \g_\alpha$ it follows that $x \in Z(\g)$. Because $Z(\g) = 0$ this contradicts $x$ being non-zero.
   
  \item
  Let $\alpha \in \Phi$. Then $\kappa(\g_\alpha, \g_{-\alpha}) \neq 0$ by Corollary~\ref{cor: g alpha and g -alpha pair non degenerate with the Killing form}, from which it follows that $\g_{-\alpha} \neq 0$ and therefore $-\alpha \in \Phi$.
   
  \item
   Let $h \in \h$. Then
   \begin{align*}
    \kappa(h, [x,y])
    &= \kappa([h,x], y)
    = \alpha(h) \kappa(x,y) \\
    &= \kappa(t_\alpha, h) \kappa(x,y)
    = \kappa(\kappa(x,y) t_\alpha, h)
    = \kappa(h, \kappa(x,y) t_\alpha).
   \end{align*}
   Becasue $\kappa_{\h \times \h}$ is non-degenerate it follows that $[x,y] = \kappa(x,y) t_\alpha$.
   
  \item
   Let $x \in \g_\alpha$ and $y \in \g_{-\alpha}$ with $\kappa(x,y) \neq 0$ (such elements exist by Corollary~\ref{cor: g alpha and g -alpha pair non degenerate with the Killing form}). Then $[x,y] = \kappa(x,y) t_\alpha \neq 0$ and therefore $[\g_\alpha, \g_{-\alpha}] = \kappa(\g_\alpha, \g_{-\alpha}) t_\alpha = k t_\alpha$.
   
  \item
   Suppose that there exist some $\alpha \in \Phi$ with $\kappa(t_\alpha, t_\alpha) = 0$. Because $\alpha = \kappa(t_\alpha, \cdot)$ it follows that $\alpha(t_\alpha) = \kappa(t_\alpha, t_\alpha) = 0$. Let $x \in \g_\alpha$ and $y \in \g_{-\alpha}$ with $\kappa(x,y) \neq 0$. By rescaling it can be additionally assumed that $\kappa(x,y) = 1$ and thus $[x,y] = \kappa(x,y) t_\alpha = t_\alpha$. Then $L \coloneqq \vspan_k(x,t_\alpha,y)$ is a three-dimensional solvable Lie subalgebra of $\g$ because $[t_\alpha, x] = \alpha(t_\alpha)x = 0$ and $[t_\alpha, y] = -\alpha(t_\alpha)y = 0$.
   
   It follows that $\ad(L)$ is a three-dimensional solvable Lie subalgebra of $\gl(\g)$ and thus $[\ad(L), \ad(L)] = \ad([L,L]) = k \ad(t_\alpha)$ consists of nilpotent endomorphisms of $\g$. (To see this notice that by Lie’s theorem there exists a basis of $\g$ with respect to which $\ad(L)$ is represented by upper triangular matrices. Then with respect to this basis $[\ad(L), \ad(L)]$ is represented by strictly upper triangular matrices.) In particular $\ad(t_\alpha)$ is nilpotent. On the other hand $t_\alpha$ is semisimple (because $t_\alpha \in \h$ and $\h$ consists of semisimple elements by the definition of a toral subalgebra) and thus $\ad(t_\alpha)$ is semisimple. Because $\ad(t_\alpha)$ is both nilpotent is semisimple it follows that $\ad(t_\alpha) = 0$ and thus $t_\alpha = 0$, contradicting $\kappa(t_\alpha, \cdot) = \alpha \neq 0$.
  \qedhere
 \end{enumerate}
\end{proof}






















