\section{Representations of Lie algebras}





\subsection{Definition of a representation}


\begin{defi}
 Let $\g$ be a $k$-Lie algebra. A \emph{representation} of $\g$ is a $k$-vector space $V$ together with a homomorphism of Lie algebras $\rho \colon \g \to \gl(V)$. This representation is called \emph{faithful} if $\rho$ is injective. The \emph{dimension} of this representation is the dimension of $V$.
\end{defi}


\begin{rem}
 Equivalently a representation of $\g$ is a $k$-vector space $V$ together with a $k$-bilinear map $\g \times V \to V, (x,v) \mapsto x.v$ such that
 \[
  x.(y.v) - y.(x.v) = [x,y].v \quad \text{for all $x,y \in \g$ and $v \in V$}.
 \]
 
 Such an action results in a homomorphism of Lie algebras $\rho \colon \g \to \gl(V)$ by setting
 \[
  \rho(x) \colon V \to V, \quad v \mapsto x.v
  \quad \text{for every $x \in \g$ and $v \in V$}.
 \]
 On the other hand any homomorphism $\rho \colon \g \to \gl(V)$ an action as above by setting
 \[
  x.v \coloneqq \rho(x)(v) \quad \text{for evevery $x \in \g$ and $v \in V$}.
 \]
 Both constructions are inverse to each other.

 We will not distinguish between these two concepts and choose whichever is more useful in the given situation.
\end{rem}


\begin{rem}
 Ado’s theorem is equivalent to every finite dimenisonal Lie algebra having a faithful representation.
\end{rem}


\begin{defi}
 Let $V$ und $W$ be representations of a $k$-Lie algebra $\g$. A $k$-linear map $f \colon V \to W$ is called a \emph{homomorphism of representations} if
 \[
  f(x.v) = x.f(v) \quad \text{for every $x \in \g$ and $v \in V$}.
 \]
 If $\rho_V \colon \g \to \gl(V)$ and $\rho_W \colon \g \to \gl(W)$ are the corresponding Lie algebra homomomorphisms this is equivalent to
 \[
  f \circ \rho_V(x) = \rho_W(x) \circ f \quad \text{for every $x \in X$}.
 \]
 $f$ is an \emph{isomorphism of representations} if it is additionally bijective.
\end{defi}


\begin{rem}
 If $f \colon V \to W$ is an isomorphism of representations of a Lie algebra $\g$ then the $k$-linear map $f^{-1} \colon W \to V$ is also a homomorphism of representations (and therefore an isomorphism of representations) because
 \[
  f^{-1}(x.v) = f^{-1}(x.f(f^{-1}(v))) = f^{-1}(f(x.f^{-1}(v))) = x.f^{-1}(v)
 \]
 for every $x \in \g$ and $v \in V$. It also follows for every $x \in \g$ from $f \rho_V(x) = \rho_W(x) f$ that $\rho_V(x) f^{-1} = f^{-1} \rho_W(x)$.
\end{rem}





\subsection{The adjoint representation}


\begin{expl}
 For any Lie algebra $\g$ the adjoint representation $\ad \colon \g \to \gl(V)$ is a representation of $\g$: That $\ad$ is a homomorphism of Lie algebras is equivalent to
 \begin{gather*}
  \ad([x,y])(z) = [\ad(x),\ad(y)](z) \quad \text{for all $x,y,z \in \g$}.
 \shortintertext{Because}
  \ad([x,y])(z) = [[x,y],z] = -[z,[x,y]]
 \shortintertext{and}
  \begin{aligned}
   [\ad(x),\ad(y)](z)
   &= (\ad(x) \circ \ad(y))(z) - (\ad(y) \circ \ad(x))(z) \\
   &= [x,[y,z]] - [y,[x,z]]
   = [x,[y,z]] + [y,[z,x]]
  \end{aligned}
 \end{gather*}
 this is equivalent to the Jacobi identity. Hence $\ad$ really is a representation of $\g$.

 As $\ker \ad = Z(\g)$ the adjoint representation is faithful if and only if $Z(\g) = 0$.
\end{expl}


\begin{rem}
 Together with Lemma~\ref{lem: Lie algebras act adjoint by derivations} it follows that every Lie algebras $\g$ acts on itself not only by any endomorphisms but by derivations of itself. This is where much of the structure of Lie algebras comes from and why the Jacobi identity is of interest (I suppose).
\end{rem}


\begin{defi}
 If $\g \subseteq \gl(V)$ is a Lie subalgebra then $V$ is a representation of $\g$ via the inclusion $\g \inc \gl(V)$. This corresponds to the action
 \[
  x.v = x(v) \quad \text{for every $x \in \g$ and $v \in V$}.
 \]
 This is the called the \emph{natural representation} of $\g$. If $\g \subseteq \gl_n(k)$ is a Lie subalgebra then the \emph{natural representation} of $\g$ is defined as
 \[
  x.v = x \cdot v \quad \text{for every $x \in \g$ and $v \in k^n$}.
 \]
\end{defi}





\subsection{New representations from old ones}


\begin{prop}
 Let $\g$ be a Lie algebra over a field $k$.
 \begin{enumerate}[leftmargin=*]
  \item
   If $(V_i)_{i \in I}$ is a collection of representations of $\g$ then $\bigoplus_{i \in I} V_i$ is a representation of $\g$ via
   \[
    x.\sum_{i \in I} v_i = \sum_{i \in I} x.v_i
   \]
   or every $x \in \g$ and $v_i \in V_i$ for all $i \in I$, with $v_i = 0$ for all but finitely many $i \in I$.
  \item
   If $V$ and $W$ are representations of $\g$ then $V \otimes W$ is a representation of $\g$ via
   \[
    x.(v \otimes w) = (x.v) \otimes w + v \otimes (x.w) \quad \text{for every $x \in \g$, $v \in V$ and $w \in W$}.
   \]
   More generally: If $V_1, \dotsc V_n$ are representations of $\g$ then $\bigotimes_{i=1}^n V_i$ is a representation of $\g$ via
   \[
    x.(v_1 \otimes \dotsb \otimes v_n)
    = \sum_{i=1}^n v_1 \otimes \dotsb \otimes v_{i-1} \otimes (x.v_i) \otimes v_{i+1} \otimes \dotsb \otimes v_n.
   \]
   for every $x \in \g$ and $v_i \in V_i$ for every $i = 1, \dotsc, n$.
  \item
   If $V, W$ are representations of $V$ then $\Hom_k(V,W)$ is a representation of $\g$ via
   \[
    (x.f)(v) = x.f(v) - f(x.v) \quad \text{for every $x \in \g$, $f \in \Hom(V,W)$ and $v \in V$}.
   \]
  \item
   By letting $\g$ act trivially on $k$ the dual $V^* = \Hom_k(V,k)$ becomes a representation of $\g$ in the above way, i.e.
   \[
    (x.\varphi)(v) = -\varphi(x.v) \quad \text{for every $x \in \g$, $\varphi \in V^*$ and $v \in V$}.
   \]
 \end{enumerate}
\end{prop}


\begin{rem}
 Given two representations $V$ and $W$ of a Lie algebra $\g$ a linear map $f \colon V \to W$ is a homomorphism of representations if and only if
 \[
  x.f(v) - f(x.v) = 0 \quad \text{for every $x \in \g$ and $v \in V$},
 \]
 which is equivalent to $x.f = 0$ for every $x \in \g$ with respect to the induced action of $\g$ on $\Hom_k(V,W)$. Hence the homomorphisms of representations are precisely the ``invariant'' linear maps under the action of $\g$.
\end{rem}



\begin{prop}
 Let $\g$ be a Lie algebra.
 \begin{enumerate}[leftmargin=*]
  \item
   If $V$ and $W$ are finite dimensional representations of $\g$ then the map
   \[
    V^* \otimes W \to \Hom_k(V,W), \quad \varphi \otimes w \mapsto (v \mapsto \varphi(v) w)
   \]
   is a homomorphism of representations. If $V$ and $W$ are both finite dimensional this is an isomorphism of vector spaces and thus already an isomorphism of representations.
  \item
   If $V^i_j$ with $i = 1, \dotsc, r$ and $j = 1, \dotsc, n_i$ is a collection of representations of $\g$ then the isomorphism of vector spaces
   \begin{align*}
    \bigotimes_{i=1}^r \bigotimes_{j=1}^{n_j} V^{i}_j
    &\to V^1_1 \otimes \dotsb \otimes V^1_{n_1} \otimes \dotsb \otimes V^r_1 \otimes \dotsb \otimes V^{r}_{n_r}, \\
    (v^1_1 \otimes \dotsb \otimes v^1_{n_1}) \otimes \dotsb \otimes (v^r_1 \otimes \dotsb \otimes v^r_{n_r})
    &\mapsto v^1_1 \otimes \dotsb \otimes v^1_{n_1} \otimes \dotsb \otimes v^r_1 \otimes \dotsb \otimes v^r_{n_r}
   \end{align*}
   is already an isomorphism of representations.
  \item
   If $V_1, \dotsc, V_n$ and $W_1, \dotsc, W_n$ are representations of $\g$ and $f_i \colon V_i \to W_i$ for every $i = 1, \dotsc, n$ a homomorphism of representations it follows that the map
   \[
    f_1 \otimes \dotsb \otimes f_n \colon \bigotimes_{i=1}^n V_i \to \bigotimes_{i=1}^n W_i,
    v_1 \otimes \dotsb \otimes v_n \mapsto f(v_1) \otimes \dotsb \otimes f(v_n)
   \]
   is also a homomorphism of representations.
 \end{enumerate}
\end{prop}





\subsection{Subrepresentations and irreducible representations}


\begin{defi}
 Let $\g$ be a Lie algebra and $\rho \colon \g \to \gl(V)$ a representation of $\g$. A \emph{subrepresentation} of $V$ is a linear subpace $U \subseteq V$ such that $x.u \in U$ for every $x \in \g$ and $u \in U$. Equivalently $U$ is $\rho(x)$ invariant for every $x \in \g$.
 
 If $(U_i)_{i \in I}$ is a collection of subrepresentations of $\g$ then $V$ is called the \emph{direct sum} of the $U_i$ if $V = \bigoplus_{i \in I} U_i$ as vector spaces.
\end{defi}


\begin{expls}
 Let $\g$ be a Lie algebra.
 \begin{enumerate}[leftmargin=*]
  \item
   If $V$ is a representation of $\g$ then $0$ and $V$ itself are subrepresentations. These are also called the \emph{trivial subrepresentations}.
  \item
   If $V$ is a representation and $(U_i)_{i \in I}$ a collection of subrepresentations $U_i \subseteq V_i$ then $\sum_{i \in I} U_i$ is also a subrepresentation of $V$.
  \item
   The subrepresentations of the adjoint representation of $\g$ are precisely the ideals in $\g$.
 \end{enumerate}
\end{expls}


\begin{defi}
 A representation $V$ of a Lie algebra $\g$ is called \emph{irreducible} or \emph{simple} if it has precisely two subrepresentations. Equivalently $V$ is nonzero and has only the trivial subrepresentations.
 
 The representation $V$ is called \emph{decomposable} if there exist non-trivial subrepresentations $U_1, U_2 \subseteq V$ with $V = U_1 \oplus U_2$. Otherwise $V$ is called \emph{indecomposable}.
 
 The representation $V$ is called \emph{completely decomposible} or \emph{semisimple} if there exists a decomposition $V = \bigoplus_{i \in I} U_i$ into irreducible subrepresentations $U_i \subseteq V$.
\end{defi}


\begin{rem}
 By definition every irreducible representation is also indecomposable. The converse does not hold. Irreducible representations are precisely the ones which are both indecomposable and completely reducible.
\end{rem}



\begin{expl}
 The adjoint representation of a Lie algebra $\g$ is irreducible if and only if $\g \neq 0$ and $\g$ has no ideals besides $0$ and $\g$ itself. So $\g$ is either the onedimensional abelian Lie algebra or a simple Lie algebra.
\end{expl}