\section{Representations of Lie algebras}





\subsection{Definition and examples}


\begin{defi}
 Let $\g$ be a $k$-Lie algebra. A \emph{representation} of $\g$ is a $k$-vector space $V$ together with a homomorphism of Lie algebras $\rho \colon \g \to \gl(V)$. This representation is called \emph{faithful} if $\rho$ is injective. The \emph{dimension} of this representation is the dimension of $V$.
\end{defi}


\begin{rem}
 Equivalently a representation of $\g$ is a $k$-vector space $V$ together with a $k$-bilinear map $\g \times V \to V, (x,v) \mapsto x.v$ such that
 \begin{equation}\label{eqn: representation via action}
  x.(y.v) - y.(x.v) = [x,y].v \quad \text{for all $x,y \in \g$ and $v \in V$}.
 \end{equation}
 
 Such an action results in a homomorphism of Lie algebras $\rho \colon \g \to \gl(V)$ by setting
 \[
  \rho(x) \colon V \to V, \quad v \mapsto x.v
  \quad \text{for every $x \in \g$ and $v \in V$}.
 \]
 On the other hand any homomorphism $\rho \colon \g \to \gl(V)$ an action as above by setting
 \[
  x.v \coloneqq \rho(x)(v) \quad \text{for evevery $x \in \g$ and $v \in V$}.
 \]
 Both constructions are inverse to each other.

 We will not distinguish between these two concepts and choose whichever is more useful in the given situation.
\end{rem}


\begin{rem}
 If $(x_i)_{i \in I}$ is a basis of a Lie algebra $\g$ then for $\rho \colon \g \to \gl(V)$ to be a homomorphism of Lie algebras it is enough to check that $\rho([x_i,x_j])= [\rho(x_i), \rho(x_j)]$ for all $i,j \in I$. Therefore it also sufficies to check \eqref{eqn: representation via action} for basis elements, i.e.\ that
 \[
  x_i.(x_j.v) - x_j.(x_i.v) = [x_i, x_j].v \quad \text{for all $i,j \in I$ and $v \in V$}.
 \]
\end{rem}


\begin{rem}
 Ado’s theorem is equivalent to every finite dimenisonal Lie algebra having a faithful representation.
\end{rem}


\begin{expls}\label{expls: representations of Lie algebras}
 \begin{enumerate}[leftmargin=*]
  \item 
   If $\g \subseteq \gl(V)$ is a Lie subalgebra then $V$ is a representation of $\g$ via the inclusion $\g \inc \gl(V)$. This corresponds to the action
   \[
    x.v = x(v) \quad \text{for every $x \in \g$ and $v \in V$}.
   \]
   This is then called the \emph{natural representation} of $\g$.
   
  \item
   If $\g \subseteq \gl_n(k)$ is a Lie subalgebra then $\g$ acts on $k^n$ via
   \[
    x.v = x \cdot v \quad \text{for every $x \in \g$ and $v \in k^n$},
   \]
   which correspondings to the Lie algebra homomorphism $\g \to \gl(k^n), x \mapsto (v \mapsto x \cdot v)$. This is then called the \emph{natural representation} of $\g$.
   
  \item
   Let $\g \coloneqq \sll_2(k)$ for any field $k$. Then $k[x,y]$ becomes a representation of $\g$ via the homomorphism of Lie algebras $\rho \colon \g \to \gl(k[x,y])$ given by
   \[
    \rho(e) = y\dd{x}, \quad
    \rho(h) = y\dd{y} - x\dd{x}, \quad
    \rho(f) = x\dd{y},
   \]
   where $x$ and $y$ also denote the multiplication with the respective variable and $(e,h,f)$ denotes the standard basis of $\sll_2(k)$. To see that this is a homomorphism of representations notice that for all $n, m \geq 0$
   \begin{align*}
    e.(x^n y^m) &= n x^{n-1} y^{m+1}, \\
    h.(x^n y^m) &= (m-n) x^n y^m, \\
    f.(x^n y^m) &= m x^{n+1} y^{m-1},
   \end{align*}
   where we write $x^\nu = 0$ and $y^\mu = 0$ for every $\nu, \mu < 0$. From this it follows that for all $n,m \geq 0$
   \begin{gather*}
    \begin{aligned}
     e.f.(x^n y^m) - f.e.(x^n y^m)
     &= (n+1)m x^n y^m - n(m+1) x^n y^m \\
     &= (m-n) x^n y^m
     = h.m = [e,f].(x^n y^m)
    \end{aligned}
   \shortintertext{as well as}
    \begin{aligned}
     h.e.(x^n y^m) - e.h.(x^n y^m)
     &=  n(m-n+2) x^{n-1} y^{m+1} - n(m-n) x^{n-1} y^{m+1} \\
     &= 2 x^{n-1} y^{m+1}
     = 2e.(x^n y^m)
     = [h,e].(x^n y^m)
    \end{aligned}
   \shortintertext{and}
     \begin{aligned}
      h.f.(x^n y^m) - f.h.(x^n y^m)
      &= m(m-n-2) x^{n+1} y^{m-1} - m(m-n) x^{n+1} y^{m-1} \\
      &= -2 x^{n+1} y^{m-1}
      = -2f.(x^n y^{m-1})
      = [h,f].(x^n y^{m-1}).
    \end{aligned}
   \end{gather*}
   
  \item
   Let $\g \coloneqq \sll_2(k)$ for any field $k$. Then the polynomial ring in one variable $k[x]$ is a representation of $\g$ via the action the homomorphism of Lie algebras $\rho \colon \g \to \gl(k[x])$ with
   \[
    \rho(e) = \dd{x}, \quad
    \rho(h) = -2x\dd{x}, \quad
    \rho(f) = -\dd{x}.
   \]
   Then $\g$ acts on $k[x]$ via
   \[
    e.x^n = n x^{n-1}, \quad
    h.x^n = -2n x^n, \quad
    f.x^n = n x^{n+1} \quad
    \text{for every $n \geq 0$},
   \]
   where we set $x^m \coloneqq 0$ for $m < 0$. To see that this is really a representation of $\g$ notice that for every $n \geq 0$
   \begin{gather*}
    e.f.x^n - f.e.x^n
    = -n(n+1) x^n + n(n-1) x^n
    = -2n x^n
    = h.x^n
    = [e,f].x^n
   \shortintertext{as well as}
    h.e.x^n - e.h.x^n
    = -2n(n-1) x^{n-1} + 2n^2 x^{n-1}
    = 2n x^{n-1}
    = 2e.x^n
    = [h,e].x^n
   \shortintertext{and}
    h.f.x^n - f.h.x^n
    = 2n(n+1) x^{n+1} - 2 n^2 x^{n+1}
    = 2n x^{n+1}
    = -2f.x^n
    = [h,f].x^n.
   \end{gather*}
   
  \item
   If $\rho \colon \g \to \gl(V)$ is a representation of a Lie algebra $\g$ and $\phi \colon \g' \to \g$ a homomorphism of Lie algebras then via the composition $\rho \circ \phi \colon \g' \to \gl(V)$ the vector space $V$ becomes a representation of $\g$. This corresponds to the action
   \[
    x.v = \rho(x).v = \rho(\phi(x))(v) \quad \text{for every $x \in \g$ and $v \in V$}.
   \]
   
  \item
   The map $\ad \colon \g \to \gl(\g), x \mapsto \ad(x)$ is a homomorphism of Lie algebras and hence a representation of $\g$.
 \end{enumerate}
\end{expls}


\begin{defi}
 Let $\g$ be a Lie algebra. Then
 \[
  \ad \colon \g \to \gl(\g), \quad x \mapsto [x,\cdot]
 \]
 is called the \emph{adjoint representation} of $\g$.
\end{defi}


\begin{rem}
 Together with Lemma~\ref{lem: Lie algebras act adjoint by derivations} it follows that every Lie algebras $\g$ acts on itself by derivations of itself via the adjoint representation. This is where much of the structure of Lie algebras comes from and why the Jacobi identity is of interest (I guess).
\end{rem}


\begin{prop}[New representations from old ones]\label{prop: new representations from old ones}
 Let $\g$ be a Lie algebra over an arbitrary field $k$.
 \begin{enumerate}[leftmargin=*]
  \item
   If $(V_i)_{i \in I}$ is a collection of representations of $\g$ then $\bigoplus_{i \in I} V_i$ is a representation of $\g$ via
   \[
    x.\sum_{i \in I} v_i = \sum_{i \in I} x.v_i
   \]
   or every $x \in \g$ and $v_i \in V_i$ for all $i \in I$, with $v_i = 0$ for all but finitely many $i \in I$.
  \item
   If $V$ and $W$ are representations of $\g$ then $V \otimes W$ is a representation of $\g$ via
   \[
    x.(v \otimes w) = (x.v) \otimes w + v \otimes (x.w) \quad \text{for every $x \in \g$, $v \in V$ and $w \in W$}.
   \]
   More generally: If $V_1, \dotsc V_n$ are representations of $\g$ then $V_1 \otimes \dotsb \otimes V_n$ is a representation of $\g$ via
   \[
    x.(v_1 \otimes \dotsb \otimes v_n)
    = \sum_{i=1}^n v_1 \otimes \dotsb \otimes v_{i-1} \otimes (x.v_i) \otimes v_{i+1} \otimes \dotsb \otimes v_n.
   \]
   for every $x \in \g$ and $v_i \in V_i$ for every $i = 1, \dotsc, n$.
  \item
   If $V, W$ are representations of $V$ then $\Hom_k(V,W)$ is a representation of $\g$ via
   \[
    (x.f)(v) = x.f(v) - f(x.v) \quad \text{for every $x \in \g$, $f \in \Hom(V,W)$ and $v \in V$}.
   \]
  \item
   By letting $\g$ act trivially on $k$ the dual $V^* = \Hom_k(V,k)$ becomes a representation of $\g$ in the above way, i.e.
   \[
    (x.\varphi)(v) = -\varphi(x.v) \quad \text{for every $x \in \g$, $\varphi \in V^*$ and $v \in V$}.
   \]
 \end{enumerate}
\end{prop}
\begin{proof}
 \begin{enumerate}[leftmargin=*]
  \item
   Let $x,y \in \g$ and $v_i \in V_i$ for every $i \in I$. Then
   \[
    x.y.\sum_{i=1}^n v_i - y.x.\sum_{i=1}^n v_i
    = \sum_{i=1}^n (x.y.v_i - y.x.v_i)
    = \sum_{i=1}^n [x,y].v_i
    = [x,y].\sum_{i=1}^n v_i.
   \]
   
  \item
   Let $x,y \in \g$ and $v_i \in V_i$ for every $i = 1, \dotsc, n$. For all $i,j,m = 1, \dotsc, n$ set
   \[
    \tilde{w}^{ij}_m \coloneqq
    \begin{cases}
     \phantom{y.}x.v_i & \text{if $i = m \neq j$}, \\
     \phantom{x.}y.v_i & \text{if $i \neq m = j$}, \\
               x.y.v_i & \text{if $i = m= j$}, \\
     \phantom{x.y.}v_i & \text{otherwise},
    \end{cases}
    \quad\text{and}\quad
    \hat{w}^{ij}_m \coloneqq
    \begin{cases}
     \phantom{y.}x.v_i & \text{if $i = m \neq j$}, \\
     \phantom{x.}y.v_i & \text{if $i \neq m = j$}, \\
               y.x.v_i & \text{if $i = m = j$}, \\
     \phantom{y.x.}v_i & \text{otherwise}.
    \end{cases}
   \]
   In particular $\tilde{w}^{ij}_m = \hat{w}^{ij}_m$ holds for all $m = 1, \dotsc, n$ and $i \neq j$. Therefore
   \begin{align*}
     &\, x.y.(v_1 \otimes \dotsb \otimes v_n) - y.x.(v_1 \otimes \dotsb \otimes v_n) \\
    =&\, \sum_{i,j=1}^n \tilde{w}^{ij}_1 \otimes \dotsb \otimes \tilde{w}^{ij}_m
         - \sum_{i,j=1}^n \hat{w}^{ij}_1 \otimes \dotsb \otimes \hat{w}^{ij}_m \\
    =&\, \sum_{i=1}^n ( \tilde{w}^{ii}_1 \otimes \dotsb \otimes \tilde{w}^{ii}_m
                       -\hat{w}^{ii}_1 \otimes \dotsb \otimes \hat{w}^{ii}_m ) \\
    =&\, \sum_{i=1}^n \left( v_1 \otimes \dotsb \otimes (x.y.v_i) \otimes \dotsb \otimes v_n
                            -v_1 \otimes \dotsb \otimes (y.x.v_i) \otimes \dotsb \otimes v_n \right) \\
    =&\, \sum_{i=1}^n v_1 \otimes \dotsb \otimes v_{i-1} \otimes (x.y.v_i - y.x.v_i)
                          \otimes v_{i+1} \otimes \dotsb v_n \\
    =&\, \sum_{i=1}^n v_1 \otimes \dotsb \otimes v_{i-1} \otimes ([x,y].v_i) \otimes v_{i+1} \otimes \dotsb v_n
    = [x,y].(v_1, \otimes \dotsb \otimes v_n).
   \end{align*}
   
  \item
   For all $x,y \in \g$, $f \in \Hom(V,W)$ it follows for every $v \in V$ that
   \begin{align*}
    (x.y.\varphi)(v) - (y.x.\varphi)(v)
    &= -(y.\varphi)(x.v) + (x.\varphi(y.v)
    = \varphi(y.x.v) - (\varphi(x.y.v) \\
    &= -\varphi(x.y.v - y.x.v)
    = -\varphi([x,y].v)
    = ([x.y].\varphi)(v).
   \qedhere
   \end{align*}
 \end{enumerate}
\end{proof}





\subsection{Homomorphisms of representations}


\begin{defi}
 Let $V$ und $W$ be representations of a $k$-Lie algebra $\g$. A $k$-linear map $f \colon V \to W$ is called a \emph{homomorphism of representations} if
 \[
  f(x.v) = x.f(v) \quad \text{for every $x \in \g$ and $v \in V$}.
 \]
 $f$ is an \emph{isomorphism of representations} if it is additionally bijective. If $\rho_V \colon \g \to \gl(V)$ and $\rho_W \colon \g \to \gl(W)$ are the corresponding Lie algebra homomomorphisms then $f$ is a homomorphism of representations if and only if
 \[
  f \circ \rho_V(x) = \rho_W(x) \circ f \quad \text{for every $x \in X$}.
 \]
 The linear subspace of $\Hom_k(V,W)$ consisting of the homomorphisms of representations is denoted by $\Hom_\g(V,W)$, and for $V = W$ by $\End_\g(V) \coloneqq \Hom_\g(V,V)$. 
\end{defi}


\begin{rem}
 If $f \colon V \to W$ is an isomorphism of representations of a Lie algebra $\g$ then the $k$-linear map $f^{-1} \colon W \to V$ is also a homomorphism of representations (and therefore an isomorphism of representations) because
 \[
  f^{-1}(x.v) = f^{-1}(x.f(f^{-1}(v))) = f^{-1}(f(x.f^{-1}(v))) = x.f^{-1}(v)
 \]
 for every $x \in \g$ and $v \in V$. It also follows for every $x \in \g$ from $f \rho_V(x) = \rho_W(x) f$ that $\rho_V(x) f^{-1} = f^{-1} \rho_W(x)$.
\end{rem}


\begin{rem}
 Given two representations $V$ and $W$ of a Lie algebra $\g$ a linear map $f \colon V \to W$ is a homomorphism of representations if and only if
 \[
  x.f(v) - f(x.v) = 0 \quad \text{for every $x \in \g$ and $v \in V$},
 \]
 which is equivalent to $x.f = 0$ for every $x \in \g$ with respect to the induced action of $\g$ on $\Hom_k(V,W)$. Hence the homomorphisms of representations are precisely the ``invariant'' linear maps under the action of $\g$.
\end{rem}


\begin{prop}\label{prop: list of homomorphism of representations}
 Let $\g$ be a Lie algebra.
 \begin{enumerate}[leftmargin=*]
  \item
   If $V$ and $W$ are finite dimensional representations of $\g$ then the map
   \[
    \Phi_1 \colon V^* \otimes W \to \Hom_k(V,W), \quad \varphi \otimes w \mapsto (v \mapsto \varphi(v) w)
   \]
   is a homomorphism of representations. If $V$ and $W$ are both finite dimensional this is an isomorphism of vector spaces and thus already an isomorphism of representations.
  \item
   If $V_1, \dotsc, V_r$ and $W_1, \dotsc, W_s$ are representations of $\g$ then the isomorphism of vector spaces
   \begin{align*}
    \Phi_2 \colon
    (V_1 \otimes \dotsb \otimes V_r) \otimes (W_1 \otimes \dotsb \otimes W_s)
    &\longrightarrow V_1 \otimes \dotsb \otimes V_r \otimes W_1 \otimes \dotsb \otimes W_s, \\
    (v_1 \otimes \dotsb \otimes v_r) \otimes (w_1 \otimes \dotsb \otimes w_s)
    &\longmapsto v_1 \otimes \dotsb \otimes v_r \otimes w_1 \otimes \dotsb \otimes w_s
   \end{align*}
   is already an isomorphism of representations.
  \item
   If $V$ and $W$ are representations of $\g$ then the isomorphism of vector spaces
   \[
    \Phi_3 \colon V \otimes W \to W \otimes V, \quad v \otimes w \mapsto w \otimes v
   \]
   is already an isomorphism of representations.
  \item
   If $V_1$, $V_2$ and $W$ are representations of $\g$ then the isomorphism of vector spaces
   \begin{align*}
    \Phi_4 \colon
    (V_1 \otimes V_2) \otimes W &\to (V_1 \otimes W) \oplus (V_2 \otimes W), \\
    (v_1 + v_2) \otimes w &\mapsto (v_1 \otimes w) + (v_2 \otimes w)
   \end{align*}
   is already an isomorphism of representations.
  \item
   If $V_1, \dotsc, V_n$ are representations of $\g$ and $\sigma \in S_n$ then the isomorphism of vector spaces
   \[
    \Phi_5 \colon 
    V_1 \otimes \dotsb \otimes V_n \to V_{\sigma(1)} \otimes \dotsb \otimes V_{\sigma(n)}, \quad
    v_1 \otimes \dotsb \otimes v_n \mapsto v_{\sigma(1)} \otimes \dotsb \otimes v_{\sigma(n)}
   \]
   is already an isomorphism of representations.
  \item
   If $V_1, \dotsc, V_n$ and $W_1, \dotsc, W_n$ are representations of $\g$ and $f_i \colon V_i \to W_i$ for every $i = 1, \dotsc, n$ a homomorphism of representations it follows that the map
   \[
    \Phi_6 \colon 
    f_1 \otimes \dotsb \otimes f_n \colon \bigotimes_{i=1}^n V_i \to \bigotimes_{i=1}^n W_i, \quad
    v_1 \otimes \dotsb \otimes v_n \mapsto f(v_1) \otimes \dotsb \otimes f(v_n)
   \]
   is also a homomorphism of representations.
 \end{enumerate}
\end{prop}
\begin{proof}
 \begin{enumerate}[leftmargin=*]
  \item
   For $x \in \g$, $\varphi \in V^*$ and $w \in W$ it follows for every $v \in V$ that
   \begin{align*}
    \Phi_1(x.(\varphi \otimes w))(v)
    &= \Phi_1((x.\varphi) \otimes w + \varphi \otimes (x.w))(v)
    = (x.\varphi)(v) w + \varphi(v) x.w \\
    &= \varphi(v) x.w -\varphi(x.v) w
    = x.(\varphi(v) w) - \varphi(x.v) w \\
    &= x.\Phi_1(\varphi \otimes w)(v) - \Phi_1(\varphi \otimes w)(x.v)
    = (x.\Phi_1(\varphi \otimes w))(v).
   \end{align*}
 \end{enumerate}
 The other statements follow from similar calculations.
\end{proof}





\subsection{Subrepresentations and irreducible representations}


\begin{defi}
 Let $\g$ be a Lie algebra and $\rho \colon \g \to \gl(V)$ a representation of $\g$. A \emph{subrepresentation} of $V$ is a linear subpace $U \subseteq V$ such that $x.u \in U$ for every $x \in \g$ and $u \in U$. Equivalently $U$ is $\rho(x)$ invariant for every $x \in \g$.
 
 If $(U_i)_{i \in I}$ is a collection of subrepresentations of $\g$ then $V$ is called the \emph{direct sum} of the $U_i$ if $V = \bigoplus_{i \in I} U_i$ as vector spaces.
\end{defi}


\begin{expls}
 Let $\g$ be a Lie algebra.
 \begin{enumerate}[leftmargin=*]
  \item
   If $V$ is a representation of $\g$ then $0$ and $V$ itself are subrepresentations. These are also called the \emph{trivial subrepresentations}.
  \item
   If $V$ is a representation and $(U_i)_{i \in I}$ a collection of subrepresentations $U_i \subseteq V_i$ then $\sum_{i \in I} U_i$ is also a subrepresentation of $V$.
  \item
   Let $V$ and $W$ be representations of a Lie algebra $\g$ and $\varphi \colon V \to W$ a homomorphism of representations. Then $\ker \varphi \subseteq V$ and $\im \varphi \subseteq W$ are subrepresentations.
  \item
   The subrepresentations of the adjoint representation of $\g$ are precisely the ideals in $\g$.
 \end{enumerate}
\end{expls}


\begin{defi}
 A representation $V$ of a Lie algebra $\g$ is called \emph{irreducible} or \emph{simple} if it has precisely two subrepresentations. Equivalently $V$ is nonzero and has only the trivial subrepresentations.
 
 The representation $V$ is called \emph{decomposable} if there exist non-trivial subrepresentations $U_1, U_2 \subseteq V$ with $V = U_1 \oplus U_2$. Otherwise $V$ is called \emph{indecomposable}.
 
 The representation $V$ is called \emph{completely decomposible} or \emph{semisimple} if there exists a decomposition $V = \bigoplus_{i \in I} U_i$ into irreducible subrepresentations $U_i \subseteq V$.
\end{defi}


\begin{rem}
 By definition every irreducible representation is also indecomposable. The converse does not hold. Irreducible representations are precisely the ones which are both indecomposable and completely reducible.
\end{rem}


\begin{expl}
 \begin{enumerate}[leftmargin=*]
  \item
   Every one-dimensional representation is irreducible.
  \item
   The adjoint representation of a Lie algebra $\g$ is irreducible if and only if $\g \neq 0$ and $\g$ has no ideals besides $0$ and $\g$ itself. So $\g$ is either the onedimensional abelian Lie algebra or a simple Lie algebra.   
 \end{enumerate}
\end{expl}


\begin{lem}[Schur]
 Let $\g$ be a Lie algebra over a field $k$.
 \begin{enumerate}
  \item
   Let $V$ and $W$ be irreducible representations of $\g$ and $\varphi \colon V \to W$ a homomorphism of representation. Then either $\varphi = 0$ or $\varphi$ is an isomorphism of representations. In particular $\Hom_\g(V,W) = 0$ if $V \ncong W$, and $\End_\g(V)$ is a skew field.
  \item
   If $k$ is algebraically closed and $V$ a finite-dimensional irreducible representation of $\g$ then every $\varphi \in \End_\g(V)$ is given by multiplication with some $\lambda \in k$. In particular $\End_\g(V) \cong k$ as rings.
 \end{enumerate}
\end{lem}
\begin{proof}
 \begin{enumerate}[leftmargin=*]
  \item
   From $V \neq 0$ and $W \neq 0$ it follows that $\varphi$ cannot be zero and an isomorohpism at the same time. Suppose that $\varphi \neq 0$. Then $\im \varphi$ is a nonzero subrepresentation of $W$, hence $\im \varphi = W$ because $W$ is irreducible. Similarly it follows that $\ker \varphi$ is a proper subrepresentation of $V$, hence $\ker \varphi = 0$ because $V$ is irreducible. By combining these two observations it follows $\varphi$ is an isomorphism.
  \item
   As $k$ is algebraically closed it follows that $\varphi$ has an eigenvalue $\lambda \in k$. Therefore $\varphi - \lambda$ is an endomorphism of $V$ as a representation of $\g$ with nonzero kernel. Hence it is no isomorphism, so $\varphi - \lambda \id_V = 0$ because $\End_\g(V)$ is a skew field.
  \qedhere
 \end{enumerate}
\end{proof}








