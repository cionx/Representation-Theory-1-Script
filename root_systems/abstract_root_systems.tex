\section{Abstract root systems}
For this section let $k$ be an arbitrary field with $\chara k = 0$.


\begin{defi}\label{defi: notation of natural pairing}
 Let $V$ be a $k$-vector space. For every $v \in V$ and $\lambda \in V^*$ set
 \[
  \pair{v, \lambda} \coloneqq \pair{\lambda, v} \coloneqq2 \lambda(v).
 \]
\end{defi}


\begin{rem}\label{rem: natural pairing is nondegenerate}
 Notice that the bilinear form $V^* \times V \to k, (\lambda,  v) \mapsto \pair{\lambda, v} = \lambda(v)$ is non-degenerate: If $v \in V$ with $\lambda(v) = 0$ for every $\lambda \in V^*$ then $v = 0$ and if $\lambda \in V^*$ with $\lambda(v) = 0$ for every $v \in V$ then $\lambda = 0$.
\end{rem}


\begin{rem}
 Notice that the notation of Definition~\ref{defi: notation of natural pairing} is compatible with the natural monomorphism $\iota \colon V \to V^{**}, v \mapsto (\lambda \mapsto \lambda(v))$ in the sense that the following diagram commutes:
 \[
   \begin{tikzcd}
     V^* \otimes V
     \arrow{r}[above]{\lambda \otimes v \mapsto \pair{\lambda, v}}
     \arrow{d}[left]{\id_V \otimes \iota}
     &
     k
     \\
     V^* \otimes V^{**}
     \arrow{ur}[below]{\lambda \otimes \phi \mapsto \pair{\lambda, \phi}}
     &
     {}
   \end{tikzcd}
 \]
 
 If $V$ is finite dimensional then $\iota$ is already an isomorphism and identifies $V$ with $V^{**}$. Then $\pair{\phi, v}$ can be understood as $\phi(v)$ or $\iota(v)(\phi)$ and the above shows that both are actually the same.
\end{rem}


\begin{defi}
 Let $V$ be a finite dimensional $k$-vector space. A subset $R \subseteq V$ is called an \emph{(abstract) root system (in $V$)} if the following hold:
 \begin{enumerate}
  \item
   $R$ is finite, $0 \notin R$ and $R$ generates $V$ as a $k$-vector space.
  \item
   For every $\alpha \in R$ exists some $\alpha^\vee \in V^*$ such that $\alpha^\vee(\alpha) = 2$ and $S_{\alpha, \alpha^\vee}(R) \subseteq R$ for the linear map
   \[
    S_{\alpha, \alpha^\vee} \colon V \to V, \quad
    \lambda \mapsto \lambda - \pair{\lambda, \alpha^\vee}\alpha.
   \]
   \item
    $\pair{\alpha, \beta^\vee} \in \Z$ for all $\alpha, \beta \in R$.
 \end{enumerate}
 The root system $R$ is \emph{reduced} if $k \alpha \cap R = \{-\alpha, \alpha\}$ for every $\alpha \in R$, i.e.\ if the only multiples of $\alpha$ which are also roots are $\alpha$ and $-\alpha$. The \emph{rank} of the root system $R$ is the dimension of $V$.
\end{defi}


\begin{rem}\label{rem: -alpha also in the root system}
 Notice that if $V$ is a finite dimensional vector space and $R \subseteq V$ a root system with $\alpha \in R$ then also
 \[
  -\alpha = \alpha - 2 \alpha = \alpha - \pair{\alpha, \alpha^\vee} \alpha = S_{\alpha, \alpha^\vee}(\alpha) \in R.
 \]
\end{rem}


\begin{lem}
 Let $V$ be a finite dimensional $k$-vector space and $R \subseteq V$ a root system. Then the element $\alpha^\vee \in V$ with $\alpha^\vee(\alpha) = 2$ is unique.
\end{lem}
\begin{proof}
 Let $\alpha^\vee, \tilde{\alpha}^\vee \in V^*$ with $\alpha^\vee(\alpha) = \tilde{\alpha}^\vee(\alpha) = 2$ and $s(R) \subseteq R$ as well as $t(R) \subseteq R$ for $s \coloneqq S_{\alpha, \alpha^\vee}$ and $t \coloneqq S_{\alpha, \tilde{\alpha}^\vee}$. Notice that $s(\alpha) = t(\alpha) = -\alpha$ as already seen in Remark~\ref{rem: -alpha also in the root system}.
 
 By induction on $n$ it follows that
 \begin{equation}\label{eqn: iterated composition of reflections}
  (s \circ t)^n(x) =  x - n\pair{x,\alpha^\vee - \tilde{\alpha}^\vee} \alpha
  \quad \text{for every $x \in V$ and $n \geq 1$}.
 \end{equation}
 For $n = 1$ this holds because
 \begin{align*}
  s(t(x))
  &= s(x - \pair{x, \tilde{\alpha}^\vee}\alpha)
  = s(x) - \pair{x, \tilde{\alpha}^\vee} s(\alpha) \\
  &= x - \pair{x, \alpha^\vee}\alpha + \pair{x, \tilde{\alpha}^\vee}\alpha
  = x - \pair{x, \alpha^\vee - \tilde{\alpha}^\vee} \alpha.
 \end{align*}
 If the formula holds for some $n \geq 1$ then
 \begin{align*}
  (s \circ t)^{n+1}(x)
  &= s(t(x - n\pair{x,\alpha^\vee - \tilde{\alpha}^\vee} \alpha))
  = s(t(x)) - n\pair{x,\alpha^\vee - \tilde{\alpha}^\vee} s(t(\alpha)) \\
  &= x - \pair{x, \alpha^\vee - \tilde{\alpha}^\vee} \alpha - n\pair{x,\alpha^\vee - \tilde{\alpha}^\vee} s(t(\alpha)) \\
  &= x - (n+1)\pair{x, \alpha^\vee - \tilde{\alpha}^\vee} \alpha.
 \end{align*}
 
 Because $s$ and $t$ are automorphisms of $V$ (as a $k$-vector space) the same goes for $s \circ t$. Because $(s \circ t)(R) = s(t(R)) \subseteq s(R) \subseteq R$ and $R$ is finite it follows that the restriction $(s \circ t)|_R$ is a permutation of $R$. Since $R$ is finite $(s \circ t)|_R$ has finite order, i.e.\ there exists some $n \geq 1$ with $(s \circ t)^n|_R = \id_R$. Because $R$ spans $V$ it already follows that $(s \circ t)^n = \id_V$. Together with \eqref{eqn: iterated composition of reflections} it follows that $\pair{x, \alpha^\vee - \tilde{\alpha}^\vee} = 0$ for every $x \in V$, so $\alpha^\vee - \tilde{\alpha}^\vee = 0$ as seen in Remark~\ref{rem: natural pairing is nondegenerate}.
\end{proof}


















