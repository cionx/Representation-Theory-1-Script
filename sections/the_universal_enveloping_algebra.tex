\chapter[The Universal Enveloping Algebra]{The Universal Enveloping \texorpdfstring{\\}{} Algebra}
\label{universal enveloping algebra}


\begin{convention}
	For this \lcnamecref{universal enveloping algebra} we fix the following conventions.
	
	The field~$\kf$ is arbitrary.
	We denote by~$\cAlg{\kf}$\glsadd{category algebras} the category of~\algebras{$\kf$}.
	If~$A$ is a~\algebra{$\kf$}, then by an~\defemph{\module{$A$}} we mean a left, unitial~\module{$A$}.
	The resulting category of~\modules{$A$}\index{category!of A-modules@of $A$-modules} is denoted by~$\cMod{A}$\glsadd{category modules}.
\end{convention}





%\section{Yoneda’s Lemma for Algebras}
%
%
%% \begin{fluff}
%%   If~$A$ and~$B$ are two~{\algebras{$\kf$}} and~$\varphi \colon A \to B$ is a homomorphism of~{\algebras{$\kf$}} then~$\varphi$ induces for every other~{\algebra{$\kf$}}~$C$ a map~$\varphi^*_C \colon \Hom_{\cAlg{\kf}}(B,C) \to \Hom_{\cAlg{\kf}}(A,C)$.
%%   These induced maps are compatible in the sense that for all~{\algebras{$\kf$}}~$C$ and~$D$ and every homomorphism of~{\algebras{$\kf$}}~$f \colon C \to D$ the following square diagram commutes:
%%   \[
%%     \begin{tikzcd}
%%       \Hom_{\cAlg{\kf}}(A,C)
%%       \arrow{r}[above]{\varphi^*_C}
%%       \arrow{d}[left]{f_*}
%%       &
%%       \Hom_{\cAlg{\kf}}(B,C)
%%       \arrow{d}[right]{f_*}
%%       \\
%%       \Hom_{\cAlg{\kf}}(A,D)
%%       \arrow{r}[below]{\varphi^*_D}
%%       &
%%       \Hom_{\cAlg{\kf}}(B,D)
%%     \end{tikzcd}
%%   \]
%%   If~$\varphi$ is an isomorphism then the induced map~$\varphi^*_C$ is for every~{\algebra{$\kf$}}~$C$ a bijection.
%%
%%   The following \lcnamecref{yoneda lemma very weak version}, which is a weak form of Yoneda’s lemma applied to the category of~{\algebras{$\kf$}}, gives a converse to this observation.
%% \end{fluff}
%
%
%
%\begin{proposition}[Yoneda’s lemma for~{\algebras{$\kf$}}]
%	\label{yoneda lemma very weak version}
%	Let~$A$ and~$B$ be two~{\algebras{$\kf$}}.
%	\begin{enumerate}
%		\item
%			Let~$\Phi$ be a homomorphism of~{\algebras{$\kf$}} from~$A$ to~$B$.
%			This homomorphism induces for every other~{\algebra{$\kf$}}~$T$ a map
%			\[
%				\Phi^*_T
%				\colon
%				\Hom_{\cAlg{\kf}}(B,T) \to \Hom_{\cAlg{\kf}}(A,T) \,,
%				\quad
%				\Psi
%				\mapsto
%				\Psi \circ \Phi \,.
%			\]
%			These induced maps are compatible in the sense that for any two~{\algebras{$\kf$}}~$T$ and~$U$ and every homomorphism of~{\algebras{$\kf$}}~$\Psi$ from~$T$ to~$U$ the following square diagram commutes.
%			\[
%				\begin{tikzcd}
%					\Hom_{\cAlg{\kf}}(B,T)
%					\arrow{r}[above]{\Phi^*_T}
%					\arrow{d}[left]{\Psi_*}
%					&
%					\Hom_{\cAlg{\kf}}(A,T)
%					\arrow{d}[right]{\Psi_*}
%					\\
%					\Hom_{\cAlg{\kf}}(B,U)
%					\arrow{r}[below]{\Phi^*_U}
%					&
%					\Hom_{\cAlg{\kf}}(A,U)
%				\end{tikzcd}
%			\]
%			If~$\Phi$ is an isomorphism, then the induced map~$\Phi^*_T$ is a bijection for every~{\algebra{$\kf$}}~$T$.
%		\item
%			\label{natural homomorphisms}
%			Let conversely~$(\eta_T)_T$ be a family of maps
%			\[
%				\eta_T
%				\colon
%				\Hom_{\cAlg{\kf}}(B,T)
%				\to
%				\Hom_{\cAlg{\kf}}(A,T)
%			\]
%			where~$T$ runs through the class of~{\algebras{$\kf$}}, and suppose that for any two~\algebras{$\kf$}~$T$ and~$U$ and every homomorphism of~\algebras{$\kf$}~$\Psi$ from~$T$ to~$U$ the square diagram
%			\begin{equation}
%				\label{natural square}
%				\begin{tikzcd}
%					\Hom_{\cAlg{\kf}}(B,T)
%					\arrow{r}[above]{\eta_T}
%					\arrow{d}[left]{\Psi_*}
%					&
%					\Hom_{\cAlg{\kf}}(A,T)
%					\arrow{d}[right]{\Psi_*}
%					\\
%					\Hom_{\cAlg{\kf}}(B,U)
%					\arrow{r}[below]{\eta_U}
%					&
%					\Hom_{\cAlg{\kf}}(A,U)
%				\end{tikzcd}
%			\end{equation}
%			commutes.
%			Then there exists an algebra homomorphism~$\Phi$ from~$A$ to~$B$ such that~$\eta_T = \Phi^*_T$ for every~{\algebra{$\kf$}}~$T$.
%			This homomorphism~$\Phi$ is unique and given by~$\Phi = \eta_B(\id_B)$.
%		\item
%			\label{onetoone correspondence morphisms and natural trans}
%			The above constructions give a {\onetoonetext} correspondence
%			\begin{align*}
%				\left\{
%					\begin{tabular}{c}
%						algebra homomorphism \\
%						$\Phi \colon A \to B$
%					\end{tabular}
%				\right\}
%				&\onetoone
%				\left\{
%					\begin{tabular}{c}
%						families~$(\eta_T)_T$ of maps \\
%						$\eta_T \colon \Hom_{\cAlg{\kf}}(B,T) \to \Hom_{\cAlg{\kf}}(A,T)$ \\
%						such that the square diagram~\eqref{natural square} commutes \\
%						for every algebra homomorphism~$f \colon T \to U$
%					\end{tabular}
%				\right\}  \,,
%			\\
%				\Phi
%				&\mapsto
%				(\Phi^*_T)_T \,,
%			\\
%				\eta_B(\id_B)
%				&\mapsfrom
%				(\eta_T)_T  \,.
%			\end{align*}
%		\item
%			A homomorphism of algebras~$\Phi$ from~$A$ to~$B$ is an isomorphism if and only if the map~$\Phi^*_T$ is bijective for every~\algebra{$\kf$}~$T$.
%			The {\onetoonetext} correspondence from part~\ref*{onetoone correspondence morphisms and natural trans} therefore restricts to a {\onetoonetext} correspondence.
%			\[
%				\left\{
%					\begin{tabular}{c}
%						algebra isomorphism \\
%						$\Phi \colon A \to B$
%					\end{tabular}
%				\right\}
%				\onetoone
%				\left\{
%					\begin{tabular}{c}
%						families~$(\eta_T)_T$ of bijections \\
%						$\eta_T \colon \Hom_{\cAlg{\kf}}(B,T) \to \Hom_{\cAlg{\kf}}(A,T)$ \\
%						such that the square diagram~\eqref{natural square} commutes \\
%						for every algebra homomorphism~$f \colon T \to U$
%					\end{tabular}
%				\right\}  \,.
%			\]
%	\end{enumerate}
%\end{proposition}
%
%
%\begin{proof}
%	\leavevmode
%	\begin{enumerate}
%		\item
%			The maps~$\eta_T$ are well-defined because the composite of two homomorphisms of algebras is again a homomorphism of algebras.
%			The square diagram~\eqref{natural square} commutes because
%			\[
%				\Psi_*( \Phi^*_T( \Theta ) )
%				=
%				\Psi_* (\Theta \circ \Phi)
%				=
%				\Psi \circ \Theta \circ \Phi
%				=
%				\Phi^*_U(\Psi \circ \Theta)
%				=
%				\Phi^*_U(\Psi_*(\Theta))
%			\]
%			for every~$\Theta \in \Hom_{\cAlg{\kf}}(B,T)$.
%		\item
%			Suppose first that there exists a homorphism of algebras~$\Phi$ from~$A$ to~$B$ with~$(\eta_T)_T = (\Phi^*_T)_T$.
%			Then
%			\[
%				\eta_B(\id_B)
%				=
%				\Phi^*_B(\id_B)
%				=
%				\id_B \circ \Phi
%				=
%				\Phi \,.
%			\]
%			This shows the uniqueness of~$\Phi$ as well as the claimed formula for~$\Phi$.
%
%			To show the existence of the desired homomorphism~$\Phi$ we set~$\Phi$ to be~$\eta_B(\id_B)$.
%			This is an element of~$\Hom_{\cAlg{\kf}}(A,B)$, and thus a homomorphism of algebras from~$A$ to~$B$.
%			If~$T$ is any~\algebra{$\kf$}, then it follows for every element~$\Psi$ of~$\Hom_{\cAlg{\kf}}(B,T)$ from the commutativity of the square diagram
%			\[
%				\begin{tikzcd}
%					\Hom_{\cAlg{\kf}}(B,B)
%					\arrow{r}[above]{\eta_B}
%					\arrow{d}[left]{\Psi_*}
%					&
%					\Hom_{\cAlg{\kf}}(A,B)
%					\arrow{d}[right]{\Psi_*}
%					\\
%					\Hom_{\cAlg{\kf}}(B,T)
%					\arrow{r}[below]{\eta_T}
%					&
%					\Hom_{\cAlg{\kf}}(A,T)
%				\end{tikzcd}
%			\]
%			that
%			\[
%				\eta_T(\Psi)
%				=
%				\eta_T(\Psi \circ \id_B)
%				=
%				\eta_T(\Psi_*(\id_B))
%				=
%				\Psi_*(\eta_B(\id_B))
%				=
%				\Psi_*(\Phi)
%				=
%				\Psi \circ \Phi \,.
%			\]
%			This shows that~$\eta_T = \Phi^*_T$, as claimed.
%		\item
%			Let~$\Phi$ be a homomorphism of algebras from~$A$ to~$B$.
%			Then
%			\[
%				\Phi^*_B(\id_B)
%				=
%				\id_B \circ \Phi
%				=
%				\Phi \,.
%			\]
%			Together with part~\ref*{onetoone correspondence morphisms and natural trans} this shows that the two constructions are mutually inverse.
%		\item
%			If the homomorphism~$\Phi$ is an isomorphism, then the map~$\Phi^*_T$ is bijective for every~{\algebra{$\kf$}} because the two maps~$\Phi^*_T$ and~$(\Phi^{-1})^*_T$ are mutually inverse.
%
%			Suppose on the other hand that the map~$\Phi^*_T$ is a bijection for every~{\algebra{$\kf$}}~$T$.
%			Then the family of maps~$( (\Phi^*_T)^{-1} )_T$ makes for every homomorphism of algebras~$\Psi$ from~$T$ to~$U$ the square diagram
%			\[
%				\begin{tikzcd}[column sep = large]
%					\Hom_{\cAlg{\kf}}(A,T)
%					\arrow{r}[above]{ (\Phi^*_T)^{-1} }
%					\arrow{d}[left]{\Psi_*}
%					&
%					\Hom_{\cAlg{\kf}}(B,T)
%					\arrow{d}[right]{\Psi_*}
%					\\
%					\Hom_{\cAlg{\kf}}(A,U)
%					\arrow{r}[below]{ (\Phi^*_U)^{-1} }
%					&
%					\Hom_{\cAlg{\kf}}(B,U)
%				\end{tikzcd}
%			\]
%			commute.
%			There hence exists by part~\ref*{natural homomorphisms} a (unique) homomorphism of algebras~$\Theta$ from~$B$ to~$A$ for which~$(\Phi^*_T)^{-1}$ equals~$\Theta^*_T$ for every~{\algebra{$\kf$}}~$T$.
%			We find for the composite~$\Theta \circ \Phi$ that
%			\[
%				(\Theta \circ \Phi)^*_T
%				=
%				\Phi^*_T \circ \Theta^*_T
%				=
%				\Phi^*_T \circ (\Phi^*_T)^{-1}
%				=
%				\id_{\Hom_{\cAlg{\kf}}(A,T)}
%				=
%				(\id_A)^*_T
%			\]
%			for every~{\algebra{$\kf$}}~$T$.
%%       It hence follows from the commutativity of the diagram
%%       \[
%%         \begin{tikzcd}
%%           \Hom_{\cAlg{\kf}}(A,T)
%%           \arrow[bend left]{rr}[above]{\varphi^*_T \circ \psi^*_T}
%%           \arrow{r}[above]{\psi^*_T}
%%           \arrow{d}[left]{f_*}
%%           &
%%           \Hom_{\cAlg{\kf}}(B,T)
%%           \arrow{r}[above]{\varphi^*_T}
%%           \arrow{d}[left]{f_*}
%%           &
%%           \Hom_{\cAlg{\kf}}(A,T)
%%           \arrow{d}[left]{f_*}
%%           \\
%%           \Hom_{\cAlg{\kf}}(A,U)
%%           \arrow{r}[above]{\psi^*_U}
%%           \arrow[bend right]{rr}[below]{\varphi^*_U \circ \psi^*_U}
%%           &
%%           \Hom_{\cAlg{\kf}}(B,U)
%%           \arrow{r}[above]{\varphi^*_U}
%%           &
%%           \Hom_{\cAlg{\kf}}(A,U)
%%         \end{tikzcd}
%%       \]
%%       that the outer square diagram
%%       \[
%%         \begin{tikzcd}[column sep = large]
%%           \Hom_{\cAlg{\kf}}(A,T)
%%           \arrow{r}[above]{(\psi \circ \varphi)^*_T}
%%           \arrow{d}[left]{f_*}
%%           &
%%           \Hom_{\cAlg{\kf}}(A,T)
%%           \arrow{d}[right]{f_*}
%%           \\
%%           \Hom_{\cAlg{\kf}}(A,U)
%%           \arrow{r}[below]{(\psi \circ \varphi)^*_U}
%%           &
%%           \Hom_{\cAlg{\kf}}(A,U)
%%         \end{tikzcd}
%%       \]
%%       commutes.
%%       If we replace~$\psi \circ \varphi$ in this diagram by the identity~$\id_A$ then the resulting square diagram
%%       \[
%%         \begin{tikzcd}[column sep = large]
%%           \Hom_{\cAlg{\kf}}(A,T)
%%           \arrow{r}[above]{(\id_A)^*_T}
%%           \arrow{d}[left]{f_*}
%%           &
%%           \Hom_{\cAlg{\kf}}(A,T)
%%           \arrow{d}[right]{f_*}
%%           \\
%%           \Hom_{\cAlg{\kf}}(A,U)
%%           \arrow{r}[below]{(\id_A)^*_U}
%%           &
%%           \Hom_{\cAlg{\kf}}(A,U)
%%         \end{tikzcd}
%%       \]
%%       also commutes.
%			Hence~$\Theta \circ \Phi = \id_A$ by the uniqueness in part~\ref*{natural homomorphisms}.
%			We find in the same way that also~$\Phi \circ \Theta = \id_B$.
%			This shows that~$\Phi$ and~$\Theta$ are mutually inverse isomorphisms.
%		\qedhere
%	\end{enumerate}
%\end{proof}
%
%
%\begin{definition}
%	A family~$(\eta_T)_T$ of maps~$\eta_T \colon \Hom_{\cAlg{\kf}}(B,T) \to \Hom_{\cAlg{\kf}}(A,T)$, where~$A$ and~$B$ are two~{\algebras{$\kf$}} and~$T$ ranges over all~{\algebras{$\kf$}}, is \defemph{natural}\index{natural} if the square diagram~\eqref{natural square} commutes for ever homomorphism of algebras~$\Psi$ from~$A$ to~$B$.
%\end{definition}
%
%
%\begin{remark}
%	\Cref{yoneda lemma very weak version} holds with the same proof for every kind of mathematical structure that has a suitable notion of homomorphisms between them, i.e.\ in every category.
%	It is then know as the \defemph{Yoneda lemma}\index{Yoneda lemma}, which is one of the most important statement in all of category theory (and mathematics).
%\end{remark}


\section{Tensor Algebra and Symmetric Algebra}


% \begin{example}[Monoid algebra]
%   In the following all monoids will be written multiplicaitvely unless otherwise mentioned.
%   The neutral element of a monoid~$M$ will be denoted by~$1$ or~$1_M$.
%   Given two monoids~$M$ and~$N$ a map~$f \colon M \to N$ is a homomorphism of monids if~$f(m \cdot m') = f(m) \cdot f(m')$ for all~$m, m' \in M$ and~$f(1_M) = 1_N$.
%   If~$M$ is any monoid then the identity~$\id_M$ is a homomorphism and if~$f \colon M \to N$ and~$g \colon N \to P$ are composable homomorphisms of monoids then their composition~$g \circ f \colon M \to P$ is again a homomorphism of monoids.
%   The resulting category of monoids is denoted by~$\cMon$.
%   
%   \begin{description}
%     \item[Construction:]
%       If~$M$ is a monoid then the monoid algebra~\gls*{monoid algebra} is the (free) vector space with basis~$M$ together with the unique bilinear extension~$\kf[M] \times \kf[M] \to \kf[M]$ of the multiplication~$M \times M \to M$ as its multiplication.
%   
%       This means that the elements of~$\kf[M]$ are formal {\linear{$\kf$}} combinations~$\sum_{m \in M} a_m m$ with~$a_m = 0$ for all but finitely many~$m \in M$.
%       The multiplication of two such elements is given by
%       \[
%         \left(
%           \sum_{m \in M} a_m m
%         \right)
%         \left(
%           \sum_{n \in M} b_n n
%         \right)
%         =
%         \sum_{m, n \in M} (a_m b_n) m n  \,.
%       \]
%       We identify every element~$m \in M$ with the corresponding element~$1 \cdot m \in \kf[M]$.
%       The product~$m \cdot n$ of two elements~$m, n \in M$ in~$\kf[M]$ is then the same as their product in~$M$.
%       The associativity of the multiplication of~$\kf[M]$ follows from the associativity of the multiplication of~$M$, and the neutral element of~$M$ is given by the multiplicative neutral element for~$\kf[M]$.
%       
%     \item[Universal Property:]
%       If~$A$ is any~{\algebra{$\kf$}} then~$(A, \cdot)$ is a multiplicative monoid, which we will denote by~$A^-$.
%       If~$M$ is any monoid and~$f \colon M \to A^-$ is a monoid hommorphism then~$f$ extends uniquely to an algebra homomorphism~$F \colon \kf[M] \to A$.
%       The algebra homomorphism~$F$ is given on elements by
%       \[
%         F\left( \sum_{m \in M} a_m m \right)
%         =
%         \sum_{m \in M} a_m f(m) \,.
%       \]
%       On the other hand every algebra homomorphism~$\kf[M] \to A$ restricts to a monoid homomorphism~$M \to A^-$.
%       This construction results in a {\onetoonetext} correspondence
%       \[
%         \{
%           \text{monoid homomorphisms~$M \to A^-$}
%         \}
%         \onetoone
%         \{
%           \text{algebra homomorphisms~$\kf[M] \to A$}
%         \}  \,.
%       \]
%     
%     \item[Uniqueness:]
%       The monoid algebra~$\kf[M]$ together with the inclusion~$i \colon M \to \kf[M]$ is uniquely determined by its universal property up to isomorphism:
%   \end{description}
% \end{example}


% \begin{recall}[Free algebra]
%   Let~$I$ be any set.
%   The \defemph{noncommutative polynomial algebra}~$\kf\gen{X_i \suchthat i \in I}$ has as a basis the set of all monomials
%   \[
%     X_{i_1} \dotsm X_{i_n}
%     \qquad
%     \text{with~$i_1, \dotsc, i_n$}
%   \]
%   and the multiplication is on these basis elements given by
%   \[
%     X_{i_1} \dotsm X_{i_n}
%     \cdot
%     X_{j_1} \dotsm X_{j_m}
%     =
%     X_{i_1} \dotsm X_{i_n} X_{j_1} \dotsm X_{j_m} \,.
%   \]
%   In contrast to the usual (commutative) polynomial algebra~$\kf[X_i \suchthat i \in I]$ the variables~$X_i$ are not required to commute with each other.
%   
%   We can alternatively construct~$\kf\gen{X_i \suchthat i \in I}$ as the monomial algebra of the free monoid on~$I$:
%   Let~$M$ be the set of all words in~$I$, i.e.\ the set of all finite sequences
%   \[
%     (i_1, \dotsc, i_n)
%     \qquad
%     \text{with~$i_1, \dotsc, i_n \in I$}  \,.
%   \]
%   Then~$M$ is a monoid with respect to concatenation of words given by
%   \[
%     (i_1, \dotsc, i_n) (j_1, \dotsc, j_m)
%     =
%     (i_1, \dotsc, i_n, j_1, \dotsc, j_m)
%   \]
%   for all words~$(i_1, \dotsc, i_n), (j_1, \dotsc, j_m) \in M$.
%   The neutral element of~$M$ is given by the empty word~$()$.
% \end{recall}





\subsection{Reviewing the Tensor Algebra}


\begin{recall}[Tensor algebra]
	Let~$V$ be a vector space.
	\begin{description}
		\item[Construction]
			For all vectors~$v_1, \dotsc, v_d$ of~$V$ we denote the resulting simple tensor~$v_1 \tensor \dotsb \tensor v_d$ in the tensor power~$V^{\tensor d}$\glsadd{tensor power}\index{tensor power} by~$(v_1, \dotsc, v_d)$\glsadd{simple tensor}.
			The zeroth tensor power~$V^{\tensor 0}$ has as the empty simple tensor~$()$ as a basis.
			We will therefore identify the tensor power~$V^{\tensor 0}$ with the ground field~$\kf$, so that empty simple tensor~$()$ corresponds to the element~$1$ of~$\kf$.
			
			For all exponents~$p, q \geq 0$ we define a multiplication map
			\[
				\mu_{p,q}
				\colon
				V^{\tensor p} \times V^{\tensor q}
				\to
				V^{\tensor (p+q)} \,,
				\quad
				(x,y)
				\mapsto
				x y
			\]
			that is on simple tensors~$(v_1, \dotsc, v_p)$ and~$(v_{p+1}, \dotsc, v_{p+q})$ given by concatination, i.e. by
			\[
				(v_1, \dotsc, v_p) \cdot (v_{p+1}, \dotsc, v_{p+q})
				\defined
				(v_1, \dotsc, v_{p+q})  \,.
			\]
			We note that for~$p = 0$ or~$q = 0$ this multiplication is just the scalar multiplication that is part of the vector space structure of~$V$.
			These multiplication maps~$\mu_{p,q}$ fit together associatively in the sense that we have for all~$p, q, r \geq 0$ and all tensors~$x \in V^{\tensor p}$,~$y \in V^{\tensor q}$ and~$z \in V^{\tensor r}$ the equality
			\[
				x \cdot (y \cdot z)
				=
				(x \cdot y) \cdot z \,.
			\]
			
			Let~$\Tensor(V) \defined \bigoplus_{d \geq 0} V^{\tensor d}$\glsadd{tensor algebra}.
			We can fit together the multiplication maps~$\mu_{p,q}$ with~$p, q \geq 0$ into a single multiplication map
			\[
				\mu
				\colon
				\Tensor(V) \times \Tensor(V)
				\to
				\Tensor(V)  \,,
				\quad
				(x,y)
				\mapsto
				xy 
			\]
			that is given on any two elements~$x$,~$y$ of~$\Tensor(V)$ with~$x = (x_d)_{d \geq 0}$ and~$y = (y_d)_{d \geq 0}$ by
			\[
				x y
				=
				\left(
					\sum_{p+q = d} x_p y_q
				\right)_{d \geq 0} \,.
			\]
			This multiplication~$\mu$ is built precisely such that it follows from the bilinearity of the multiplications~$\mu_{p,q}$ that the multipliation~$\mu$ is again bilinear.
			It follows from the associativities of the multiplications~$\mu_{p,q}$ that the multiplication~$\mu$ is associative.
			We may identify the ground field~$\kf$ with the zeroth tensor power~$V^{\tensor 0}$ and then with the corresponding direct summand of~$\Tensor(V)$.
			In this way, we regard~$\kf$ as a linear subspace of~$\Tensor(V)$.
			We have seen above that the element~$1$ of~$\kf$ is then unital for the multiplication of~$\Tensor(V)$.
			We have thus altogether constructed a~{\algebra{$\kf$}}~$\Tensor(V)$.
			The algebra~$\Tensor(V)$ is the \defemph{tensor algebra}\index{tensor algebra} of~$V$.
			
			We may identify the vector space~$V$ with the first tensor power~$V^{\otimes 1}$ and then with the corresponding direct summand of~$\Tensor(V)$.
			In this way, we regard the vector space~$V$ as a linear subspace of~$\Tensor(V)$.
			We have for all elements~$v_1, \dotsc, v_n$ of~$V$ that
			\[
				v_1 \dotsm v_n
				=
				(v_1) \dotsm (v_n)
				=
				(v_1, \dotsc, v_n)
				=
				v_1 \tensor \dotsb \tensor v_n  \,.
			\]
			It follows in particular that the algebra~$\Tensor(V)$ is generated by~$V$.
			
			We will more generally identify for every natural number~$d$ the tensor power~$V^{\tensor d}$ with the corresponding direct summand of~$\Tensor(V)$.
			Every element of the tensor algebra~$\Tensor(V)$ is then a linear combination of simple tensors.
		
		\item[Universal Property]
			The tensor algebra~$\Tensor(V)$ can be though of as the \enquote{free~{\algebra{$\kf$}} on~$V$} in at least two ways:
			\begin{description}
				\item[Informal]
					The tensor algebra~$\Tensor(V)$ arises from the vector space~$V$ by starting with the elements of~$V$ and then adding all kinds of expressions that can be constructed from the elements of~$V$ via algebra operations.
					But it follows from the axioms of a~\algebra{$\kf$} that many of these expressions have to be the same, so that we only end up with expressions of a certain form.
					
					Let us be a bit more explicit:
					Suppose that a~{\algebra{$\kf$}}~$A$ contains the vector space~$V$ as a linear subspace.
					Then it also contains every product of the form~$v_1 \dotsm v_n$ with~$v_1, \dotsc, v_n$ in~$V$, and hence sums of such products, i.e. elements of the form
					\begin{equation}
						\label{general form of an element in the generated subalgebra}
						\sum_{i=1}^r v_{1i} \dotsm, v_{n_i i}
					\end{equation}
					for natural numbers~$r, n_1, \dotsc, n_r$ and vectors~$v_{ij}$ in~$V$.
					If we continue to combine elements of the form~\eqref{general form of an element in the generated subalgebra} via algebra operations then we do not gain any new elements, since we can simplify the resulting expressions via the axioms of a~\algebra{$\kf$} into something of the form~\eqref{general form of an element in the generated subalgebra}.
					
					But it may happen that two expressions of the form~\eqref{general form of an element in the generated subalgebra} are equal in~$A$ even though this does not follow pureley from the axioms of a~\algebra{$\kf$}.
					Consider for example the polynomial ring~$A = \kf[x, y]$ and the linear subspace~$V = \gen{x, y}_{\kf}$.
					It follows from the axioms of a~\algebra{$\kf$} that the two elements~$x (x+y)$ and~$x^2 + xy$ of~$A$ are equal.
					But it does not follow from these axioms that~$xy = yx$, even though this holds in~$A$.
					There are hence certain additional \emph{relations} between the elements~$x$ and~$y$ of~$V$ in the ambient \algebra{$\kf$}~$\kf[x,y]$.
					
					In the tensor algebra~$\Tensor(V)$ this does not happen.
					If two elements of~$A$ built from elements of~$V$ via algebra operations are equal, then this equality can be derived from the algebra axioms.
					There hence exist no additional relations between the elements of~$V$ in~$\Tensor(V)$.
					(The only required condition is~$V$ being a linear subspace of~$\Tensor(V)$, i.e. that addition and scalar multiplication in~$V$ does coincide with the one coming from~$\Tensor(V)$.)
					
					The tensor algebra~$\Tensor(V)$ is in this way the \enquote{freest} way of expanding the vector space~$V$ into a~\algebra{$\kf$}.
				\item[Formal]
					Let~$i$ be the inclusion map from~$V$ to~$\Tensor(V)$.
					This map is~{\linear{$\kf$}}.
					If~$A$ is a~{\algebra{$\kf$}} and~$f$ is a linear map from~$V$ to~$A$, then any~\linear{$\kf$} map~$f$ extends uniquely to a homomorphism of algebras~$f^+$ from~$\Tensor(V)$ to~$A$.
					In other words, there exists a unique homomorphism of algebras~$f^+$ from~$\Tensor(V)$ to~$A$ that makes the triangular diagram
					\[
						\begin{tikzcd}
							V
							\arrow{r}[above]{f}
							\arrow{d}[left]{i}
							&
							A
							\\
							\Tensor(V)
							\arrow[dashed]{ur}[below right]{f^+}
							&
							{}
						\end{tikzcd}
					\]
					commute.
					The algebra homomorphism~$f^+$ is given by
					\[
						f^+(v_1 \tensor \dotsb \tensor v_d)
						=
						f(v_1) \dotsm f(v_d)
					\]
					for all~$d \geq 0$ and~$v_1, \dotsc, v_n \in V$.
					This construction results in a {\onetoonetext} correspondence
					\begin{align*}
						\left\{
							\begin{tabular}{c}
								\linear{$\kf$} maps \\
								$f \colon V \to A$
							\end{tabular}
					\right\}
						&\onetoone
						\left\{
							\begin{tabular}{c}
								algebra homomorphisms \\
								$\Phi \colon \Tensor(V) \to A$
							\end{tabular}
						\right\} \,,
						\\
						f
						&\mapsto
						f^+ \,,
						\\
						\restrict{\Phi}{V}
						&\mapsfrom
						\Phi \,.
					\end{align*}
					The tensor algebra~$\Tensor(V)$ together with the linear map~$i$ from~$V$ to~$\Tensor(V)$ is in this sense the \enquote{universal way}\index{universal property!of the tensor algebra} of assigning a~{\algebra{$\kf$}} to the vector space~$V$.
			\end{description}
			The formal explanation relates to the informal explanation as follows:
			If~$A$ is any~{\algebra{$\kf$}} that contains~$V$ as a linear subspace, then the inclusion map~$i$ from~$V$ to~$A$ extend uniquely to a homomorphism of algebras~$i^+$ from~$\Tensor(V)$ to~$A$.
			Every relation between expression of the form~\eqref{general form of an element in the generated subalgebra} that holds in~$\Tensor(V)$ must then also hold in~$A$.
			The relations that hold in~$\Tensor(V)$ are the therefore precisely those relations which hold in \emph{every}~{\algebra{$\kf$}} containing~$V$.
			
		\item[Uniqueness]
			The above universal property determines the tensor algebra~$\Tensor(V)$ together with the inclusion map~$i$ from~$V$ to~$\Tensor(V)$ uniquely up to unique isomorphism, in the following sense.

			Let~$T$ be another~{\algebra{$\kf$}} and let~$j$ be a linear map from~$V$ to~$T$.
			Suppose that for every~{\algebra{$\kf$}}~$A$ and every~{\linear{$\kf$}} map~$f$ from~$V$ to~$A$ there exists a unique homomorphism of algebras~$\Phi$ from~$T$ to~$A$ that makes the triangular diagram
			\[
				\begin{tikzcd}
					V
					\arrow{r}[above]{f}
					\arrow{d}[left]{j}
					&
					A
					\\
					T
					\arrow[dashed]{ur}[below right]{\Phi}
					&
					{}
				\end{tikzcd}
			\]
			commute.
			Then there exist unique homomorphisms of algebras~$\Phi$ from~$A$ to~$T$ and a unique homomorphism of algebras~$\Psi$ from~$T$ to~$A$ that make the triangular diagrams
			\[
				\begin{tikzcd}[column sep = {3em,between origins}]
					{}
					&
					V
					\arrow{dl}[above left]{i}
					\arrow{dr}[above right]{j}
					&
					{}
					\\
					\Tensor(V)
					\arrow[dashed]{rr}[below]{\Phi}
					&
					{}
					&
					T
				\end{tikzcd}
				\qquad\text{and}\qquad
				\begin{tikzcd}[column sep = {3em,between origins}]
					{}
					&
					V
					\arrow{dl}[above left]{j}
					\arrow{dr}[above right]{i}
					&
					{}
					\\
					T
					\arrow[dashed]{rr}[below]{\Psi}
					&
					{}
					&
					\Tensor(V)
				\end{tikzcd}
			\]
			commute.
			It then follows that the composites~$\Psi \circ \Phi$ and~$\Phi \circ \Psi$ make the triangular diagrams
			\[
				\begin{tikzcd}[column sep = small]
					{}
					&
					V
					\arrow{dl}[above left]{i}
					\arrow{dr}[above right]{i}
					&
					{}
					\\
					\Tensor(V)
					\arrow[dashed]{rr}[below]{\Psi \circ \Phi}
					&
					{}
					&
					\Tensor(V)
				\end{tikzcd}
				\qquad\text{and}\qquad
				\begin{tikzcd}[column sep = small]
					{}
					&
					V
					\arrow{dl}[above left]{j}
					\arrow{dr}[above right]{j}
					&
					{}
					\\
					T
					\arrow[dashed]{rr}[below]{\Phi \circ \Psi}
					&
					{}
					&
					T
				\end{tikzcd}
			\]
			commute.
			The algebra homomorphisms~$\Psi \circ \Phi$ and~$\Phi \circ \Psi$ are unique with this propert by the universal properties of~$(\Tensor(V), i)$ and~$(T, j)$.
			But the identities~$\id_{\Tensor(V)}$ and~$\id_T$ also make these diagrams commute.
			We therefore find that~$\Psi \circ \Phi = \id_{\Tensor(V)}$ and~$\Phi \circ \Psi = \id_{T}$.
			The homomorphisms of algebras~$\Phi$ and~$\Psi$ are therefore mutually inverse isomorphisms of algebras.
		
		\item[Functoriality]
			Let~$V$ and~$W$ be two vector spaces and let~$f$ be a~\linear{$\kf$} map from~$V$ to~$W$.
			We can consider the following diagram.
			\[
				\begin{tikzcd}
					V
					\arrow{r}[above]{f}
					\arrow{d}[left]{i_V}
					&
					W
					\arrow{d}[right]{i_W}
					\\
					\Tensor(V)
					&
					\Tensor(W)
				\end{tikzcd}
			\]
			By applying the universal property of the tensor algebra~$\Tensor(V)$ to the composite~$i_W \circ f$, it follows that there exists a unique algebra homomorphism~$\Tensor(f)$\glsadd{tensor algebra on morphisms}\index{induced homomorphism!of tensor algebras} from~$\Tensor(V)$ to~$\Tensor(W)$ that makes the square diagram
			\[
				\begin{tikzcd}[column sep = large, row sep = huge]
					V
					\arrow{r}[above]{f}
					\arrow{d}[left]{i_V}
					\arrow[dashed]{dr}[above right]{i_W \circ f}
					&
					W
					\arrow{d}[right]{i_W}
					\\
					\Tensor(V)
					\arrow[dashed]{r}[below]{\Tensor(f)}
					&
					\Tensor(W)
				\end{tikzcd}
			\]
			commute.
			This homomorphism is given by
			\[
				f(v_1 \tensor \dotsb v_d)
				=
				f(v_1) \dotsm f(v_d)
			\]
			for all~$d \geq 0$ and~$v_1, \dotsc, v_d \in V$.

			This induced algebra homorphism is functorial in the following sense:
			\begin{itemize}
				\item
					It holds that~$\Tensor(\id_V) = \id_{\Tensor(V)}$.
					Indeed, the commutativity of the square 
					\[
						\begin{tikzcd}[column sep = large]
							V
							\arrow{r}[above]{f}
							\arrow{d}
							&
							V
							\arrow{d}
							\\
							\Tensor(V)
							\arrow[dashed]{r}[below]{\Tensor(\id_V)}
							&
							\Tensor(V)
						\end{tikzcd}
					\]
					shows that the identity~$\id_{\Tensor(V)}$ satifies the defining property of the induced algebra homomorphism~$\Tensor(f)$.
				\item
					Let~$U$,~$V$,~$W$ be three vector spaces.
					Let~$f$ be a linear map from~$U$ to~$V$ and let~$g$ be a linear map from~$V$ to~$W$.
					Then
					\[
						\Tensor(g \circ f)
						=
						\Tensor(g) \circ \Tensor(f) \,.
					\]
					Indeed, it follows from the commutativity of the diagram
					\[
						\begin{tikzcd}[column sep = large]
							U
							\arrow[dashed, bend left=45]{rr}[above]{g \circ f}
							\arrow{r}[above]{f}
							\arrow{d}
							&
							V
							\arrow{r}[above]{g}
							\arrow{d}
							&
							W
							\arrow{d}
							\\
							\Tensor(U)
							\arrow{r}[below]{\Tensor(f)}
							\arrow[dashed, bend right=45]{rr}[below]{\Tensor(g) \circ \Tensor(f)}
							&
							\Tensor(V)
							\arrow{r}[below]{\Tensor(g)}
							&
							\Tensor(W)
						\end{tikzcd}
					\]
					that the subdiagram
					\[
						\begin{tikzcd}[column sep = huge]
							U
							\arrow{r}[above]{g \circ f}
							\arrow{d}
							&
							W
							\arrow{d}
							\\
							\Tensor(U)
							\arrow[dashed]{r}[below]{\Tensor(g) \circ \Tensor(f)}
							&
							\Tensor(W)
						\end{tikzcd}
					\]
					commutes.
					This shows that the composite~$\Tensor(g) \circ \Tensor(f)$ satisfies the defining property of the induced homomorphism of algebras~$\Tensor(g \circ f)$.
			\end{itemize}
			
			We see from the above discussion that the assignment~$\Tensor$ defines a functor from~$\cVect{\kf}$ to~$\cAlg{\kf}$.
			The universal property of the tensor algebra states that the functor~$\Tensor$ is left adjoint\index{adjunction} to the forgetful functor from~$\cAlg{\kf}$ to~$\cVect{\kf}$.
		
		\item[Description via a basis]
			Let~$(v_i)_{i \in I}$ be a basis of~$V$
			Then for every natural number~$d$ the tensor power~$V^{\tensor d}$ inherits a basis from~$V$, given by all simple tensors
			\begin{equation}
				\label{basis element of tensor power}
				v_{i_1} \tensor \dotsb \tensor v_{i_d}
			\end{equation}
			with~$i_1, \dotsc, i_d \in I$.
			It follows that the tensor algebra~$\Tensor(V)$ has a basis given by all such simple tensors with~$d \geq 0$ and~$i_1, \dotsc, i_d \in I$.

			The product of two such basis vectors is again a basis vector.
			We may think about the basis vector~\eqref{basis element of tensor power} of~$\Tensor(V)$ as the finite word~$i_1 \dotso i_d$ in the alphabet~$I$, and about the multiplication of two basis vectors as the concatenation of words.
			
			If we think about the basis vector~$v_i$ of~$V$ as a formal variable~$X_i$ then we see that the tensor algebra~$\Tensor(V)$ is isomorphic to the noncommutative polynomial algebra~$\kf\gen{X_i \suchthat i \in I}$\glsadd{noncommutative polynomial algebra}.

			This can also seen via adjunctions.
			Indeed, we have the following commutative diagram of forgetful functors.
			\[
				\begin{tikzcd}
					\cAlg{\kf}
					\arrow{d}
					\arrow[bend left = 60]{dd}
					\\
					\cVect{\kf}
					\arrow{d}
					\\
					\cSet
				\end{tikzcd}
			\]
			It follow that the resulting diagram of left-adjoint functors
			\[
				\begin{tikzcd}
					\cAlg{\kf}
					\\
					\cVect{\kf}
					\arrow{u}[left]{\Tensor}
					\\
					\cSet
					\arrow{u}[left]{F}
					\arrow[bend right = 60]{uu}[right]{\kf\gen{X_i \suchthat i \in (-)}}
				\end{tikzcd}
			\]
			commutes up to natural isomorphism.
			The vector space~$V$ has a basis indexed by~$I$ and is therefore isomorphic to free vector space~$F(I)$.
			Hence
			\[
				\Tensor(V)
				\cong
				\Tensor(F(I))
				\cong
				\kf\gen{X_i \suchthat i \in I} \,.
			\]
	\end{description}
\end{recall}





\subsection{Reviewing the Symmetric Algebra}


\begin{recall}[Symmetric power]
	Let~$V$ be a vector space and let~$d$ be a natural number.
	The~{\howmanyth{$d$}} \defemph{symmetric power}~$\Symm^d(V)$\glsadd{symmetric power}\index{symmetric power} is the quotient vector space of the tensor power~$V^{\tensor d}$ by its linear subspace~$U_d$ that is generated by all differences of the form
	\[
			v_1 \tensor \dotsb \tensor v_d
		- v_{\sigma(1)} \tensor \dotsb \tensor v_{\sigma(d)}
	\]
	with~$v_1, \dotsc, v_d \in V$ and~$\sigma \in \symm_d$.
	Hence
	\[
		\Symm^d(V)
		=
		V^{\tensor d} / U_d
		=
		V^{\tensor d}
		/
		\gen{
				v_1 \tensor \dotsb \tensor v_d
			- v_{\sigma(1)} \tensor \dotsb \tensor v_{\sigma(d)} 
		\suchthat
			v_1, \dotsc, v_n \in V,
			\sigma \in \symm_n
		}_{\kf} \,.
	\]
	We can identify the zeroth symmetric power~$\Symm^0(V)$ with the zeroth tensor prower~$V^{\tensor 0}$ and thus with~$\kf$, because~$U_0$ vanishes.
	For all vectors~$v_1, \dotsc, v_n$ in~$V$ we denote the residue class of the simple tensor~$v_1 \tensor \dotsb \tensor v_d$ in~$\Symm^d(V)$ by~$v_1 \dotsm v_d$\glsadd{simple symmetric tensor}, and call this a \defemph{simple symmetric tensor}\index{simple symmetric tensor}\index{symmetric tensor}.
	We have by construction of~$\Symm^d(V)$ that
	\[
		v_1 \dotsm v_d
		=
		v_{\sigma(1)} \dotsm v_{\sigma(d)}
	\]
	for all~$v_1, \dotsc, v_n \in V$ and~$\sigma \in \symm_d$.

	The symmetric power~$\Symm^d(V)$ is in the following sense universal with this property:
	The map
	\[
		\alpha
		\colon
		\underbrace{ V \times \dotsb \times V }_d
		\to
		\Symm^d(V)  \,,
		\quad
		(v_1, \dotsc, v_d)
		\mapsto
		v_1 \dotsm v_d
	\]
	is symmetric and multilinear.
	If~$\beta$ is any symmetric, multilinear map from the~\fold{$d$} product~$V \times \dotsb \times V$ to another vector space~$W$, then there exists a unique linear map~$f$ from~$\Symm^d(V)$ to~$W$ that makes the triangular diagram
	\[
		\begin{tikzcd}[row sep = large]
			V \times \dotsb \times V
			\arrow{d}[left]{\alpha}
			\arrow{dr}[above right]{\beta}
			&
			{}
			\\
			\Symm^d(V)
			\arrow[dashed]{r}[below]{f}
			&
			W
		\end{tikzcd}
	\]
	commute.
	A linear map from~$\Symm^d(V)$ to~$W$ is in this sense \enquote{the same} as a symmetric, multilinear map from the~\fold{$d$} product~$V \times \dotsb \times V$ to~$W$.
	
	If~$(v_i)_{i \in I}$ is a basis of~$V$ such that~$(I, \leq)$ is a linearly ordered set then the ordered monomials
	\[
		v_{i_1} \dotsm v_{i_d}
		\qquad
		\text{with~$i_1 \leq \dotsb \leq i_d$}
	\]
	form a basis of the symmetric power~$\Symm^d(V)$.
	If~$V$ is of finite dimension~$n$ then it follows that
	\[
		\dim \Symm^d(V)
		=
		\binom{n+d-1}{d}  \,.
	\]
	(This can be seen via stars and bars.)
\end{recall}


\begin{recall}[Symmetric algebra]
	Let~$V$ be a vector space.
	Just as the tensor powers~$V^{\otimes d}$ can be used to construct the tensor algebra~$\Tensor(V)$, we can also use the symmetric powers~$\Symm^d(V)$ to construct the \defemph{symmetric algebra}~$\Symm(V)$\glsadd{symmetric algebra}\index{symmetric algebra}.
	Just as the tensor algebra~$\Tensor(V)$ is the free~\algebra{$\kf$} on~$V$, the symmetric algebra~$\Symm(V)$ is the free symmetric algebra on~$V$.

	The argumentation for the symmetric algebra is analogous to that for the tensor algebra.
	We will therefore skip some of the details.
	
	\begin{description}
		\item[Construction]
			For all vectors~$v_1, \dotsc, v_d$ in~$V$ we denote the corresponding simple symmetric tensor in~$\Symm^d(V)$ by~$v_1 \dotsm v_d$.
			We can define on the direct sum~$\Symm(V) \defined \bigoplus_{d \geq 0} \Symm^d(V)$ a multiplication such that
			\[
				(v_1 \dotsm v_p) \cdot (v_{p+1} \dotsm v_{p+q})
				=
				v_1 \dotsm v_p v_{p+1} \dotsm v_{p+q}
			\]
			for all~$p, q \geq 0$ and~$v_1, \dotsc, v_{p+q} \in V$.
			By identifying the zeroth symmetric power~$\Symm^0(V)$ with the ground field~$\kf$ we make~$\Symm(V)$ into an associative~{\algebra{$\kf$}}.
			This algebra is commutative since we have
			\begin{align*}
				\SwapAboveDisplaySkip
				(v_1 \dotsm v_p) \cdot (v_{p+1} \dotsm v_{p+q})
				&=
				v_1 \dotsm v_p v_{p+1} \dotsm v_{p+q}
				\\
				&=
				v_{p+1} \dotsm v_{p+q} v_1 \dotsm v_p
				\\
				&=
				(v_{p+1} \dotsm v_{p+q}) \cdot (v_1 \dotsm v_p)
			\end{align*}
			for all~$p, q \geq 0$ and~$v_1, \dotsc, v_{p+q} \in V$. 
			We can identify the vector space~$V$ with the first symmetric power~$\Symm^1(V)$ and then with the corresponding direct summand of~$\Symm(V)$.
			We can more generally identify every symmetric power~$\Symm^d(V)$ with the corresponding direct summand of~$\Symm(V)$.
			Every element of the symmetric algebra~$\Symm(V)$ is then a a linear combination of simple symmetric tensors.
			
		\item[Universal property]
			The symmetric algebra~$\Symm(V)$ is in the following sense the \enquote{free commutative~{\algebra{$\kf$}}} on the vector space~$V$.

			Let~$i$ be the inclusion map from~$V$ to~$\Symm(V)$.
			There exists for every symmetric~{\algebra{$\kf$}}~$A$ and every linear map~$f$ from~$V$ to~$A$ a unique homomorphism of algebras~$f^+$ from~$\Symm(V)$ to~$A$ that makes the triangular diagram
			\[
				\begin{tikzcd}
					V
					\arrow{r}[above]{f}
					\arrow{d}[left]{i}
					&
					A
					\\
					\Symm(V)
					\arrow[dashed]{ur}[below right]{f^+}
					&
					{}
				\end{tikzcd}
			\]
			commute.
			The homomorphism~$f^+$ is given by
			\[
				f^+(v_1 \dotsm v_d)
				=
				f(v_1) \dotsm f(v_d)
			\]
			for all~$d \geq 0$ and~$v_1, \dotsc, v_d \in V$.
			This construction results in a {\onetoonetext} correspondence\index{universal property!of the symmetric algebra}
			\begin{align*}
				\left\{
					\begin{tabular}{c}
						\linear{$\kf$} maps \\
						$f \colon V \to A$
					\end{tabular}
			\right\}
				&\onetoone
				\left\{
					\begin{tabular}{c}
						algebra homomorphisms \\
						$\Phi \colon \Symm(V) \to A$
					\end{tabular}
				\right\} \,,
				\\
				f
				&\mapsto
				f^+ \,,
				\\
				\restrict{\Phi}{V}
				&\mapsfrom
				\Phi \,.
			\end{align*}
			
			It follows that a relations between elements of~$V$ holds in the symmetric algebra~$\Symm(V)$ if and only if it holds in every commutative algebra that contains~$V$.
			
		\item[Uniqueness]
			Let~$S$ be a commutative~{\algebra{$\kf$}} and let~$j$ be a linear map from~$V$ to~$S$.
			Suppose that the pair~$(S, j)$ satisfies the same universal property as the symmetric algebra~$(\Symm(V), i)$.
			Then there exists a unique homomorphism of algebras~$\Phi$ from~$\Symm(V)$ and a unique homomorphisms of algebras~$\Psi$ from~$S$ to~$\Symm(V)$ that make the triangular diagrams
			\[
				\begin{tikzcd}[column sep = {3em,between origins}]
					{}
					&
					V
					\arrow{dl}[above left]{i}
					\arrow{dr}[above right]{j}
					&
					{}
					\\
					\Symm(V)
					\arrow[dashed]{rr}[below]{\Phi}
					&
					{}
					&
					S
				\end{tikzcd}
				\qquad\text{and}\qquad
				\begin{tikzcd}[column sep = {3em,between origins}]
					{}
					&
					V
					\arrow{dl}[above left]{j}
					\arrow{dr}[above right]{i}
					&
					{}
					\\
					S
					\arrow[dashed]{rr}[below]{\Psi}
					&
					{}
					&
					\Symm(V)
				\end{tikzcd}
			\]
			commute.
			These homomorphisms~$\Phi$ and~$\Psi$ are mutually inverse isomorphisms of algebras.
			
		\item[Functoriality]
			Let~$V$ and~$W$ be two vector spaces and let~$f$ be a linear map from~$V$ to~$W$.
			Then there exists a unique homomorphism of algebras~$\Symm(f)$\glsadd{symmetric algebra on morphisms}\index{induced homomorphism!on symmetric algebras} from~$\Symm(V)$ to~$\Symm(W)$ that makes the square diagram
			\[
				\begin{tikzcd}
					V
					\arrow{r}[above]{f}
					\arrow{d}
					&
					W
					\arrow{d}
					\\
					\Symm(V)
					\arrow[dashed]{r}[below]{\Symm(f)}
					&
					\Symm(W)
				\end{tikzcd}
			\]
			commmute.
			This homomorphism~$\Symm(f)$ is given by
			\[
				\Symm(f)(v_1 \dotsm v_d)
				=
				f(v_1) \dotsm f(v_d)
			\]
			for all~$d \geq 0$ and~$v_1, \dotsc, v_d \in V$.

			It holds that~$\Symm(\id_V) = \id_{\Symm(V)}$ and it holds for all composable~{\linear{$\kf$}} maps~$f$ from~$U$ to~$V$ and~$g$ from~$V$ to~$W$ that~$\Symm(g \circ f) = \Symm(g) \circ \Symm(f)$.
			The assignment~$\Symm$ is thus a functor from~$\cVect{\kf}$ to~$\cCAlg{\kf}$, where~$\cCAlg{\kf}$\glsadd{category commutative algebras}\index{category!of commutative algebras} denotes the category of commutative~{\algebras{$\kf$}}.
			
		\item[Description via a basis]
			Let~$(v_i)_{i \in I}$ be a basis of$~V$ where~$(I, \leq)$ is a linearly ordered set.
			Then the symmetric power~$\Symm^d(V)$ inherits a basis from~$V$ given by all simple symmetric tensors
			\[
				v_{i_1} \dotsm v_{i_d}
				\qquad
				\text{where~$i_1 \leq \dotsb \leq i_d$} \,.
			\]
			It follows that the symmetric algebra~$\Symm(V)$ has the simple symmetric tensors~$v_{i_1} \dotsm v_{i_d}$ with~$d \geq 0$ and~$i_1, \dotsc, i_d \in I$ such that~$i_1 \leq \dotsb \leq i_d$ as a basis.
			This basis may also be written as
			\[
				v_{i_1}^{n_1} \dotsm v_{i_r}^{n_r}
			\]
			with~$r \geq 0$ and~$i_1, \dotsc, i_r \in I$ such that~$i_1 < \dotsb < i_r$, and~$n_1, \dotsc, n_r \geq 0$.
			(This second description of the induced basis on~$\Symm(V)$ is connected to the first description via~$d = n_1 + \dotsb + n_r$).
			
			We see from this description that the symmetric algebra~$\Symm(V)$ is isomorphic to the commutative polynomial algebra~$\kf[X_i \suchthat i \in I]$, which is the free commutative~{\algebra{$\kf$}} on the generators~$X_i$ with~$i \in I$.
			This can again be explained by considering the commutative diagram of forgetful functors
			\[
				\begin{tikzcd}
					\cCAlg{\kf}
					\arrow{d}
					\arrow[bend left = 60]{dd}
					\\
					\cVect{\kf}
					\arrow{d}
					\\
					\cSet
				\end{tikzcd}
			\]
			from which we see that the resulting diagram of left adjoints\index{adjunction}
			\[
				\begin{tikzcd}
					\cCAlg{\kf}
					\\
					\cVect{\kf}
					\arrow{u}[left]{\Symm}
					&
					\\
					\cSet
					\arrow{u}[left]{F}
					\arrow[bend right = 60]{uu}[right]{\kf[X_i \suchthat i \in (-)]}
					&
					{}
				\end{tikzcd}
			\]
			commutes up to natural isomorphism.
			
		\item[Contruction via the tensor algebra]
			The symmetric algebra~$\Symm(V)$ can also be constructed as a quotient of the tensor algebra~$\Tensor(V)$.
			We give multiple ways how to seethis and think about it.
			Let in the following~$i$ denote the inclusion map from~$V$ to~$\Tensor(V)$ and let~$j$ denote the inclusion map from~$V$ to~$\Symm(V)$.
			\begin{itemize}
				\item
					Let~$I$ be the commutator ideal\index{commutator ideal} of~$\Tensor(V)$, i.e. the two-sided ideal of~$\Tensor(V)$ generated by all elements of the form
					\[
						x y - y x
					\]
					where~$x$ and~$y$ are elements of~$\Tensor(V)$.
					Let~$\Pi$ denote the canonical projection homomorphism from~$\Tensor(V)$ onto~$\Tensor(V)/I$.
					The quotient algebra~$\Tensor(V)/I$ is commutative, whence there exists by the universal property of the symmetric algebra~$\Symm(V)$ a unique homomorphism of algebras~$\Phi$ from~$\Symm(V)$ to~$\Tensor(V)/I$ that makes the diagram
					\[
						\begin{tikzcd}[column sep = small]
							{}
							&
							V
							\arrow[bend right]{ddl}[above left]{i}
							\arrow[bend left]{dr}[above right]{j}
							&
							{}
							\\
							{}
							&
							{}
							&
							\Tensor(V)
							\arrow{d}[right]{\Pi}
							\\
							\Symm(V)
							\arrow[dashed]{rr}[above]{\Phi}
							&
							{}
							&
							\Tensor(V)/I
						\end{tikzcd}
					\]
					commute.
					The homomorphism~$\Phi$ is on the generating set~$V$ of~$\Symm(V)$ given by~$\Phi(v) = \class{v}$ for every~$v \in V$.

					We get on the other hand from the universal property of the tensor algebra~$\Tensor(V)$ a unique homomorphism ofalgebras~$\Psi'$ from~$\Tensor(V)$ to~$\Symm(V)$ that makes the diagram
					\[
						\begin{tikzcd}[column sep = small]
							{}
							&
							V
							\arrow[bend right]{dl}[above left]{j}
							\arrow[bend left]{ddr}[above right]{i}
							&
							{}
							\\
							\Tensor(V)
							\arrow[bend left=25, dashed]{drr}[above right, pos=0.35]{\Psi'}
							\arrow{d}[left]{\Pi}
							&
							{}
							&
							{}
							\\
							\Tensor(V)/I
							&
							{}
							&
							\Symm(V)
						\end{tikzcd}
					\]
					commute.
					The commutator ideal~$I$ is contained in the kernel of~$\Psi'$ because the algebra~$\Symm(V)$ is commutative.
					There hence exists a unique homomorphism of algebras~$\Psi$ from~$\Tensor(V)/I$ to~$\Symm(V)$ that makes the diagram
					\[
						\begin{tikzcd}[column sep = small]
							{}
							&
							V
							\arrow[bend right]{dl}[above left]{j}
							\arrow[bend left]{ddr}[above right]{i}
							&
							{}
							\\
							\Tensor(V)
							\arrow[bend left=25]{drr}[above right, pos=0.37]{\Psi'}
							\arrow{d}[left]{\Pi}
							&
							{}
							&
							{}
							\\
							\Tensor(V)/I
							\arrow[dashed]{rr}[above]{\Psi}
							&
							{}
							&
							\Symm(V)
						\end{tikzcd}
					\]
					commute.
					The homomorphism~$\Psi'$ is on the algebra generating set~$\{ \class{v} \suchthat v \in V \}$ of~$\Tensor(V) / I$ given by by~$\Psi(\class{v}) = v$ for every~$v \in V$.
					
					It follows from the explicit descriptions of~$\Phi$ and~$\Psi$ on algebra generatorthat these homomorphisms are mutually inverse isomorphisms of~\algebras{$\kf$}.
					Thus~$\Symm(V) \cong \Tensor(V)/I$ via the isomorphism~$f$.
					
					We note that the commutator ideal~$I$ is already generated by the commutators~$v \tensor w - w \tensor v$ where~$v$ and~$w$ range through~$V$.
					Indeed, let~$J$ be the ideal generated by these elements.
					Then on the one hand~$J$ is contained in~$I$.
					But on the other hand the quotient~$\Tensor(V)/J$ is already commutative because it is generated by the residue classes~$\class{v}$ with~$v$ in~$V$, and these generators commute.
					The commutator ideal~$I$ is therefore contained in the kernel of the canonical projection homomorphism from~$\Tensor(V)$ to~$\Tensor(V)/J$.
					But this kernel is the ideal~$J$, so~$I$ is contained in~$J$.
					
				\item
					The above argumentation is not surprising if we remember that the tensor algebra~$\Tensor(V)$ is the universal~{\algebra{$\kf$}} on~$V$ and that quotiening out the commutator ideal~$I$ is the universal way of making an algebra commutative.
					The quotient algebra~$\Tensor(V)/I$ therefore ought to be the universal commutative~{\algebra{$\kf$}}.
					
					This motivation can be formalized by observing that the diagram of forgetful functors
					\[
						\begin{tikzcd}
							\cCAlg{\kf}
							\arrow{d}
							\arrow[bend left = 60]{dd}
							\\
							\cAlg{\kf}
							\arrow{d}
							\\
							\cVect{\kf}
						\end{tikzcd}
					\]
					commutes.
					It follows that the resulting diagram of left adjoint functors\index{adjunction}
					\[
						\begin{tikzcd}
							\cCAlg{\kf}
							\\
							\cAlg{\kf}
							\arrow{u}[left]{C}
							\\
							\cVect{\kf}
							\arrow{u}[left]{\Tensor}
							\arrow[bend right = 60]{uu}[right]{\Symm}
						\end{tikzcd}
					\]
					commutes up to natural isomorphism.
					In this diagram, the adjoint fuctor~$C$ of the forgetful functor from~$\cCAlg{\kf}$ to~$\cAlg{\kf}$ is given by quotiening out the commutator ideal, i.e. by
					\[
						C(A)
						=
						A / \ideal{ ab - ba \suchthat a, b \in A }
					\]
					on objects.
					It follows that~$\Symm(V) \cong C(\Tensor(V)) = \Tensor(V)/I$.
				\item
					The above argumentation be also expressed via Yoneda’s lemma.
					Inddeed, for every commutative~\algebra{$\kf$}~$A$ there exist natural bijections
					\begin{align*}
						{}&
						\{ \textstyle\text{algebra homomorphisms~$\Symm(V) \to A$} \}
						\\
						\cong{}&
						\{ \textstyle\text{{\linear{$\kf$}} maps~$V \to A$} \}
						\\
						\cong{}&
						\{ \textstyle\text{algebra homomorphisms~$\Tensor(V) \to A$} \}
						\\
						\cong{}&
						\{ \textstyle\text{algebra homomorphisms~$\Tensor(V)/I \to A$} \} \,,
					\end{align*}
					It now follows from Yoneda’s~lemma that~$\Symm(V) \cong \Tensor(V)/I$.
			\end{itemize}
	\end{description}
\end{recall}


\begin{remark}
	One can similarly construct the \defemph{exterior algebra}~$\Exterior(V) = \bigoplus_{d \geq 0} \Exterior^d(V)$\glsadd{exterior algebra}\index{exterior algebra} of a vector space~$V$ by using the exterior powers~$\Exterior^d(V)$\glsadd{exterior power}\index{exterior power} instead of the the tensor powers~$V^{\otimes d}$ or symmetric powers~$\Symm^d(V)$.
	If~$A$ is another~\algebra{$\kf$}, then a homomorphism of algebras~$\Phi$ from~$\Exterior(V)$ to~$A$ is \enquote{the same} as a~\linear{$\kf$} map~$f$ from~$V$ to~$A$ such that~$f(v)^2 = 0$ for ever~$v \in V$.
% TODO: Do we have to worry about char(k) = 2?
	It thus follows from a similar argumentation as for the symmetric algebra that~$\Exterior(V)$ is isomorphic to the quotient algebra~$\Tensor(V)/I$ where~$I$ is the two-sided ideal in~$\Tensor(V)$ generated by the elements~$v \tensor v$ where~$v$ ranges through~$V$.
	
	If the vector space~$V$ is finite-dimensional, then the exterior algebra~$\Exterior(V)$ is again finite-dimensional.
	Its dimension is given by~$\dim( \Exterior(V) ) = 2^{\dim(V)}$.
	This behavior is different to both that of the tensor algebra~$\Tensor(V)$ and that of the symmetric algebra~$\Symm(V)$, which are infinite-dimensional whenever the vector space~$V$ is nonzero.
\end{remark}






\section{Definition, Construction and First Properties}





\subsection{Definition}


\begin{definition}
	Let~$\glie$ be a~\liealgebra{$\kf$}.
	A \defemph{universal enveloping algebra}\index{universal enveloping algebra} of the Lie~algebra~$\glie$ is a~\algebra{$\kf$}~$\Univ(\glie)$\glsadd{universal enveloping algebra} together with a homomorphism of Lie~algebras~$\iota$ from~$\glie$ to~$\Univ(\glie)$ such that the following universal property\index{universal property! of the universal enveloping algebra} holds:
	For every~{\algebra{$\kf$}}~$A$ and every homomorphism of Lie~algebras~$\varphi$ from~$\glie$ to~$A$ there exists a unique homomorphism of~\algebras{$\kf$}~$\Phi$  from~$\Univ(\glie)$ to~$A$ that makes the triangular diagram
	\[
		\begin{tikzcd}
			\glie
			\arrow{r}[above]{\varphi}
			\arrow{d}[left]{\iota}
			&
			A
			\\
			\Univ(\glie)
			\arrow[dashed]{ur}[below right]{\Phi}
			&
			{}
		\end{tikzcd}
	\]
	commute, i.e.\ such that~$\varphi = \Phi \circ \iota$.
	In other words, every homomorphism of Lie~algebras from~$\glie$ to~$A$ extends along~$\iota$ uniquely to a homomorphism of algebras from~$\Univ(\glie)$ to~$A$.
	The homomorphism~$\iota$ is the \defemph{canonical homomorphism}\index{canonical homomorphism} from~$\glie$ to~$\Univ(\glie)$.
\end{definition}


\begin{remark}[Uniqueness of universal enveloping algebras]
	\label{uniqueness of universal enveloping algebras}
	Let~$\glie$ be a Lie algebra and suppose that~$(\Univ(\glie)_1, \iota_1)$ and~$(\Univ(\glie)_2, \iota_2)$ are two~{\uas} of~$\glie$.
	Then there exist unique homomorphisms of algebras~$\Phi$ from~$\Univ(\glie)_1$ to~$\Univ(\glie)_2$ and~$\Psi$ from~$\Univ(\glie)_2$ to~$\Univ(\glie)_1$ that make the triangular diagrams
	\[
		\begin{tikzcd}[column sep = small]
			{}
			&
			\glie
			\arrow{dl}[above left]{\iota_1}
			\arrow{dr}[above right]{\iota_2}
			&
			{}
			\\
			\Univ(\glie)_1
			\arrow[dashed]{rr}[below]{\Phi}
			&
			{}
			&
			\Univ(\glie)_2
		\end{tikzcd}
		\qquad\text{and}\qquad
		\begin{tikzcd}[column sep = small]
			{}
			&
			\glie
			\arrow{dl}[above left]{\iota_2}
			\arrow{dr}[above right]{\iota_1}
			&
			{}
			\\
			\Univ(\glie)_2
			\arrow[dashed]{rr}[below]{\Psi}
			&
			{}
			&
			\Univ(\glie)_1
		\end{tikzcd}
	\]
	commute.
	It follows that the composites~$\Psi \circ \Phi$ and~$\Phi \circ \Psi$ make the triangle diagrams
	\[
		\begin{tikzcd}[column sep = small]
			{}
			&
			\glie
			\arrow{dl}[above left]{\iota_1}
			\arrow{dr}[above right]{\iota_1}
			&
			{}
			\\
			\Univ(\glie)_1
			\arrow[dashed]{rr}[below]{\Psi \circ \Phi}
			&
			{}
			&
			\Univ(\glie)_1
		\end{tikzcd}
		\qquad\text{and}\qquad
		\begin{tikzcd}[column sep = small]
			{}
			&
			\glie
			\arrow{dl}[above left]{\iota_2}
			\arrow{dr}[above right]{\iota_2}
			&
			{}
			\\
			\Univ(\glie)_2
			\arrow[dashed]{rr}[below]{\Phi \circ \Psi}
			&
			{}
			&
			\Univ(\glie)_2
		\end{tikzcd}
	\]
	commute.

	The algebra homomorphisms~$\Phi \circ \Psi$ and~$\Psi \circ \Phi$ are unique with this property by the universal properties of the {\uas}~$(\Univ(\glie)_1, \iota_1)$ and~$(\Univ(\glie)_2, \iota_2)$.
	But the identities~$\id_{\Univ(\glie)_1}$ and~$\id_{\Univ(\glie)_2}$ also makes these diagrams commute.
	We thus find that the composite~$\Psi \circ \Phi$ equals~$\id_{\Univ(\glie)_1}$ and that the composite~$\Phi \circ \Psi$ equals~$\id_{\Univ(\glie)_2}$.
	The homomorphisms~$\Phi$ and~$\Psi$ are therefore mutually inverse isomorphisms of algebras.
	
	This shows that a universal enveloping algebra of~$\glie$ is unique up to unique isomorphism.
	We will therefore talk about \emph{the} universal enveloping algebra of~$\glie$.
	We will often also surpress the homorphism of Lie~algebra~$\iota$ from~$\glie$ to~$\Univ(\glie)$ from our notation.
\end{remark}


% \begin{remark}
%   One can also formulate the above argument is a more categorical way:
%   Consider the category~$\catC$ where
%   \begin{itemize}
%     \item
%       objects of~$\catC$ is a pairs~$(A, i)$ consisting of a~{\algebra{$\kf$}}~$A$ and a Lie~algebra homomorphism~$i \colon \glie \to A$,
%     \item
%       a morphism~$\phi \colon (A, i) \to (B, j)$ is an algebra homomorphism~$\phi \colon A \to B$ that makes the triangular diagram
%       \[
%         \begin{tikzcd}[column sep = small]
%         {}
%         &
%         \glie
%         \arrow{dl}[above left]{i}
%         \arrow{dr}[above right]{j}
%         &
%         {}
%         \\
%         A
%         \arrow[dashed]{rr}[below]{\phi}
%         &
%         {}
%         &
%         B
%       \end{tikzcd}
%     \]
%       commute, and
%     \item
%       the composition of two morphisms is just their usual set-theoretic composition.
%   \end{itemize}
%   A universal enveloping algebra of~$\glie$ is nothing but an inital object in this category~$\catC$.
%   The argumentation from \cref{uniqueness of universal enveloping algebras} is then the usual argument for the uniqueness of inital objects up to unique isomorphism.
% \end{remark}



\subsection{Construction}


\begin{fluff}
	Let~$\glie$ be a Lie~algebra.
	We will in the following show that the universal enveloping algebra of~$\glie$ exists.
	To figure out the construction of~$\Tensor(\glie)$ we will use the universal property of~$\Univ(\glie)$ to explain that we should be able to construct~$\Univ(\glie)$ as a certain quotient algebra of the tensor algebra~$\Tensor(\glie)$.
	We then show that this quotient does indeed have the correct universal property.

	Suppose that~$\glie$ admits a universal enveloping algebra~$\Univ(\glie)$ and let~$\iota$ be the canonical homomorphism of Lie~algebras from~$\glie$ to~$\Univ(\glie)$.
	Let us first observe that the algebra~$\Univ(\glie)$ is generated by the image of~$\iota$.

	Indeed, let~$U$ be the subalgebra of~$\Univ(\glie)$ that is generated by the image of~$\iota$, and let~$\iota'$ be the restriction of~$\iota$ to a homomorphism of Lie~algebras from~$\glie$ to~$U$.
	For every~{\algebra{$\kf$}}~$A$ and every Lie~algebra homomorphism~$\varphi$ from~$\glie$ to~$A$ the induced homomorphism of algebras~$\Phi$ from~$\Univ(\glie)$ to~$A$ restricts to an homomorphism of algebras~$\Phi'$ from~$U$ to~$A$.
	This homomorphism~$\Phi'$ makes the triangular diagram
	\[
		\begin{tikzcd}
			\glie
			\arrow{r}[above]{\varphi}
			\arrow{d}[left]{\iota'}
			&
			A
			\\
			U
			\arrow[dashed]{ur}[below right]{\Phi'}
			&
			{}
		\end{tikzcd}
	\]
	commute.
	The homomorphism~$\Phi'$ is unique with this property because the algebra~$U$ is generated by the image of~$\iota'$, and the composite~$\Phi \circ \iota'$ equals the predetermined homomorphism of Lie~algebras~$\varphi$.
	This shows that the algebra~$U$ together with the homomorphism of Lie~algebras~$\iota'$ is again a universal enveloping algebra for~$\glie$.
	
	It follows from the uniqueness of the universal enveloping algebra of~$\glie$, as discussed in \cref{uniqueness of universal enveloping algebras}, that there exists a unique homomorphism of algebras~$\Iota$ from~$U$ to~$\Univ(\glie)$ that makes the triangular diagram
	\[
		\begin{tikzcd}[column sep = small]
			{}
			&
			\glie
			\arrow{dl}[above left]{\iota'}
			\arrow{dr}[above right]{\iota}
			&
			{}
			\\
			U
			\arrow[dashed]{rr}[below]{\Iota}
			&
			{}
			&
			\Univ(\glie)
		\end{tikzcd}
	\]
	commute, and that this homomorphism is already an isomorphism.%
	\footnote{The letter~$\Iota$ that we use here is the uppercase version of the greek letter iota.}
	The homomorphism~$\Iota$ is the inclusion map from~$U$ to~$\Univ(\glie)$ because this is a homomorphism of algebras from~$U$ to~$\Univ(\glie)$ that makes the above triangular diagram commute.
	We have thus found that the inclusion map from~$U$ to~$\Univ(\glie)$ is an isomorphism of algebras, whence~$U$ equals~$\Univ(\glie)$.

	We now apply the universal property of the tensor algebra~$\Tensor(\glie)$ to the linear map~$\iota$.
	We find that there exists a unique homomorphism of algebras~$\Phi'$ from~$\Tensor(\glie)$ to~$\Univ(\glie)$ that makes the triangular diagram
	\[
		\begin{tikzcd}[column sep = small]
			{}
			&
			\glie
			\arrow{dl}
			\arrow{dr}[above right]{\iota}
			&
			{}
			\\
			\Tensor(\glie)
			\arrow[dashed]{rr}[below]{\Phi'}
			&
			{}
			&
			\Univ(\glie)
		\end{tikzcd}
	\]
	commute.
	The homomorphism~$\Phi'$ is surjective because~$\Univ(\glie)$ is generated by the image of~$\iota$ as an algebra.
	It follows that~$\Phi'$ induces an isomorphism of algebras
	\[
		\Phi
		\colon
		\Tensor(\glie) / I
		\to
		\Univ(\glie)
	\]
	where the ideal~$I$ is the kernel of~$\Phi'$.
	This isomorphism makes the resulting diagram
	 \[
		\begin{tikzcd}[column sep = small]
			{}
			&
			\glie
			\arrow[bend right]{dl}
			\arrow[bend left]{ddr}[above right]{\iota}
			&
			{}
			\\
			\Tensor(\glie)
			\arrow[bend left]{drr}[below left]{\Phi'}
			\arrow{d}
			&
			{}
			&
			{}
			\\
			\Tensor(\glie)/I
			\arrow[dashed]{rr}[below]{\Phi}
			&
			{}
			&
			\Univ(\glie)
		\end{tikzcd}
	\]
	commute.
 
	Let~$A$ be another~\algebra{$\kf$}.
	Every linear map~$g$ from~$\glie$ to~$A$ factors through a homomorphism of algebras~$\Psi'$ from~$\Tensor(\glie)$ to~$A$.
	It follows thanks to the above isomorphism~$\Phi$ between~$\Tensor(\glie) / I$ and~$\Univ(\glie)$ from the universal property of~$\Univ(\glie)$ that the homomorphism~$\Psi'$ factors trough a homorphism of algebras from~$\Tensor(\glie) / I$ to~$A$ if and only if the linear map~$g$ is a homomorphism of Lie~algebras.

	That~$g$ is a homomorphism of Lie~algebras means that
	\[
		g(x) g(y) - g(y) g(x) - g([x,y]) = 0
	\]
	for all~$x, y \in \glie$.
	This is equivalent to the condition
	\[
		\Psi(x) \Psi(y) - \Psi(y) \Psi(x) - \Psi([x,y]_{\glie})
		=
		0
	\]
	for all~$x, y \in \glie$, and furthermore equivalent to the condition
	\[
		\Psi( xy - yx - [x,y]_{\glie} )
		=
		0
	\]
	for all~$x, y \in \glie$.

	We have now seen that an algebra homomorphism~$\Psi$ from~$\Tensor(\glie)$ to some~\algebra{$\kf$}~$A$ factors trough the quotient~$\Tensor(\glie)/I$ if and only if~$\Psi$ annihilates all those elements of~$\Tensor(\glie)$ that are of the form~$xy - yx - [x,y]_{\glie}$ where~$x$ and~$y$ range through~$\glie$.
	This means that the ideal~$I$ needs to be generated by those elements.
	
	We have now altogether seen that the universal enveloping algebra~$\Univ(\glie)$ needs to be constructable as the quotient of the tensor algebra~$\Tensor(\glie)$ by the ideal~$I$ that is generated by all those elements of the form~$x y - y x - [x,y]_{\glie}$ where~$x$ and~$y$ range through~$\glie$.
	We will conversely show in the following \lcnamecref{existence of uea} that this construction will indeed give us the universal enveloping algebra.
\end{fluff}


\begin{proposition}[Existence of the universal enveloping algebra]
	\label{existence of uea}
	Let~$\glie$ be a~\liealgebra{$\kf$}.
	Let~$\Tensor(\glie)$ be the tensor algebra of the underlying vector space of~$\glie$ and let~$I$ the two-sided ideal of~$\Tensor(\glie)$ generated by all elements of the form~$x y - y x - [x,y]_{\glie}$ where~$x$ and~$y$ range through~$\glie$.
	The quotient algebra~$U \defined T(\glie)/I$ together with the~{\linear{$\kf$}} map
	\[
		\iota
		\colon
		\glie
		\to
		\Univ(\glie) \,,
		\quad
		x
		\mapsto
		\class{x}
	\]
	is a universal enveloping algebra for~$\glie$.
\end{proposition}


\begin{proof}
	The map~$\iota$ is~{\linear{$\kf$}}, and it is compatible with the Lie brackets of~$\glie$ and~$\Univ(\glie)$ because
	\[
		[\iota(x), \iota(y)]
		=
		[\class{x}, \class{y}]
		=
		\class{x} \, \class{y} - \class{y} \, \class{x}
		=
		\class{x y - y x}
		=
		\class{[x,y]_{\glie}}
		=
		\iota([x,y]_{\glie})
	\]
	for all~$x, y \in \glie$.
	Given any~\algebra{$\kf$}~$A$ and Lie algebra homomorphism~$\varphi$ from~$\glie$ to~$A$, there exists a unique homorphism of~\algebras{$\kf$}~$\Phi'$ from~$\Tensor(\glie)$ to~$A$ that makes the triangular diagram
	\[
		\begin{tikzcd}
			\glie
			\arrow{r}[above]{\varphi}
			\arrow{d}
			&
			A
			\\
			\Tensor(\glie)
			\arrow[dashed]{ur}[below right]{\Phi'}
			&
			{}
		\end{tikzcd}
	\]
	commute.
	The homomorphism~$\Phi'$ is given by~$\Phi'(x) = \varphi(x)$ for every~$x \in \glie$.
	It follows that
	\begin{align*}
		\Phi'(x y - y x)
		&=
		\Phi'(x) \Phi'(y) - \Phi'(y) \Phi'(x)
		\\
		&=
		\varphi(x) \varphi(y) - \varphi(y) \varphi(x)
		\\
		&=
		[ \varphi(x), \varphi(y) ]
		\\
		&=
		\varphi( [x,y]_{\glie} )
		\\
		&=
		\Phi'( [x,y]_{\glie} )
	\end{align*}
	for all~$x, y \in \glie$.
	The ideal~$I$ is therefore contained in the kernel of the homomorphism~$\Phi'$.
	It follows that there exists a unique homomorphism of algebras~$\Phi$ from~$U$ to~$A$ that makes the diagram
	\[
		\begin{tikzcd}
			\glie
			\arrow{r}[above]{\varphi}
			\arrow{d}
			&
			A
			\\
			\Tensor(\glie)
			\arrow[bend right= 20]{ur}[above left]{\Phi'}
			\arrow{d}[left]{\Pi}
			&
			{}
			\\
			U
			\arrow[dashed, bend right = 30]{uur}[below right]{\Phi}
			&
			{}
		\end{tikzcd}
	\]
	commute, where~$\Pi$ denotes the canonical projection from~$\Tensor(V)$ to~$\Tensor(V)/I$.
	We may add the homomorphism of Lie~algebras~$\iota$ to this diagram.
	We then arrive at the following commutative diagram.
	\[
		\begin{tikzcd}
			\glie
			\arrow{r}[above]{\varphi}
			\arrow{d}
			\arrow[bend right = 55]{dd}[left]{\iota}
			&
			A
			\\
			\Tensor(\glie)
			\arrow[bend right= 20]{ur}[above left]{\Phi'}
			\arrow{d}[left]{\Pi}
			&
			{}
			\\
			U
			\arrow[bend right = 30]{uur}[below right]{\Phi}
			&
			{}
		\end{tikzcd}
	\]
	We have in particular the following commutative subdiagram.
	\[
		\begin{tikzcd}
			\glie
			\arrow{r}[above]{\varphi}
			\arrow{d}[left]{i}
			&
			A
			\\
			U
			\arrow{ur}[below right]{\Phi}
			&
			{}
		\end{tikzcd}
	\]
	We have thus shown that every homomorphism of Lie~algebras~$\varphi$ from~$\glie$ to~$A$ extends along~$\iota$ to a homomorphism of algebra~$\Phi$ from~$U$ to~$A$.
	The algebra~$U$ is generated by the image of~$\iota$ whence the homomorphism of algebras~$\Phi$ is unique with this property.
\end{proof}


\begin{remark}
	The above proof may be summarized as observing that we have bijections
	\begingroup
	\allowdisplaybreaks
	\begin{align*}
		{}&
		\{ \textstyle\text{algebra homomorphisms~$\Phi \colon \Tensor(\glie)/I \to A$} \}
		\\
		\cong{}&
		\{ \text{algebra homomorphisms~$\Phi' \colon \Tensor(\glie) \to A$ with~$I \subseteq \ker(\Phi')$} \}
		\\
		\cong{}&
		\left\{
			\begin{tabular}{c}
				algebra homomorphisms~$\Phi' \colon \Tensor(\glie) \to A$ with  \\
				$\Phi'(x y - y x - [x,y]_{\glie}) = 0$ for all~$x, y \in \glie$
			\end{tabular}
		\right\}
		\\
		\cong{}&
		\left\{
			\begin{tabular}{c}
				algebra homomorphisms~$\Phi' \colon \Tensor(\glie) \to A$ with  \\
				$\Phi'(x) \Phi'(y) - \Phi'(y) \Phi'(x) - \Phi'([x,y]_{\glie}) = 0$ for all~$x, y \in \glie$
			\end{tabular}
		\right\}
		\\
		\cong{}&
		\left\{
			\begin{tabular}{c}
				algebra homomorphisms~$\Phi' \colon \Tensor(\glie) \to A$ with  \\
				$\Phi'(x) \Phi'(y) - \Phi'(y) \Phi'(x) = \Phi'([x,y]_{\glie})$ for all~$x, y \in \glie$
			\end{tabular}
		\right\}
		\\
		\cong{}&
		\left\{
			\begin{tabular}{c}
				{\linear{$\kf$}} maps~$\varphi \colon \glie \to A$ with  \\
				$\varphi(x) \varphi(y) - \varphi(y) \varphi(x) = \varphi([x,y])$ for all~$x, y \in \glie$
			\end{tabular}
		\right\}
		\\
		\cong{}&
		\left\{
			\begin{tabular}{c}
				{\linear{$\kf$}} maps~$\varphi \colon \glie \to A$ with  \\
				$[\varphi(x), \varphi(y)] = \varphi([x,y])$ for all~$x, y \in \glie$ 
			\end{tabular}
		\right\}
		\\
		={}&
		\{ \textstyle\text{Lie~algebra homomorphisms~$\varphi \colon \glie \to A$} \} \,,
	\end{align*}
	\endgroup
	and that these bijections are natural in~$A$.
	This shows that the~\algebra{$\kf$}~$\Tensor(\glie)/I$ represents the correct functor to be the universal enveloping algebra of~$\glie$.
	We can also see that the identity map of~$\Tensor(\glie)/I$ corresponds under the above bijections (for~$A = \Tensor(\glie)/I$) to the specified map~$\iota$ from~$\glie$ to~$\Tensor(\glie)/I$.
\end{remark}



\subsection{Properties}


\begin{convention}
	Let~$\glie$ be a Lie~algebra.
	For every element~$x$ of~$\glie$ the image of~$x$ of~$\Univ(\glie)$ is denoted by~$\class{x}$\glsadd{element in universal enveloping algebra}.
\end{convention}

\subsubsection{Anti-Homomorphisms}

\begin{proposition}
	Let~$\glie$ be a Lie~algebra, let~$\iota$ be the canonical homomorphism of Lie~algebras from~$\glie$ to~$\Univ(\glie)$ and let~$A$ be a~\algebra{$\kf$}.
	We have a well-defined {\onetoonetext} correspondence given by
	\begin{align*}
		\SwapAboveDisplaySkip
		\left\{
			\begin{tabular}{c}
				anti-homomorphisms\index{anti-homomorphism!of Lie algebras} \\
				of Lie~algebras
				$\varphi \colon \glie \to A$
			\end{tabular}
		\right\}
		&\onetoone
		\left\{
			\begin{tabular}{c}
				anti-homomorphisms\index{anti-homomorphism!of algebras} \\
				of algebras
				$\Phi \colon \Univ(\glie) \to A$
			\end{tabular}
		\right\} \,,
		\\
		\Phi \circ \iota
		&\mapsfrom
		\Phi \,.
	\end{align*}
\end{proposition}

\begin{proof}
	We have {\onetoonetext} correspondence given by
	\begin{align*}
		\left\{
			\begin{tabular}{c}
				homomorphisms \\
				of Lie~algebras
				$\varphi \colon \glie \to A^{\op}$
			\end{tabular}
		\right\}
		&\onetoone
		\left\{
			\begin{tabular}{c}
				homomorphisms \\
				of algebras
				$\Phi \colon \Univ(\glie) \to A^{\op}$
			\end{tabular}
		\right\} \,,
		\\
		\Phi \circ \iota
		&\mapsfrom
		\Phi \,.
	\end{align*}
	An anti-homomorphism of Lie~algebras from~$\glie$ to~$A$ is the same as a homomorphim of Lie~algebras from~$\glie$ to~$A^{\op}$, and an anti-homomorphism of algebras from~$\Univ(\glie)$ to~$A$ is the same as a homomorphism of algebras from~$\Univ(\glie)$ to~$A^{\op}$.%
	\footnote{
		We use here implicitely that the opposite Lie algebra of~$A$ is the same as the underlying Lie~algebra of the opposite algebra of~$A$.
		In other words, the operations \enquote{taking the opposite} and \enquote{taking the underlying Lie~algebra} commute.
	}
\end{proof}

\subsubsection{Representations and Modules}

\begin{proposition}
	\label{representations are modules}
	Let~$M$ be a~{\vectorspace{$\kf$}} and let~$\glie$ be a Lie~algebra.
	Let~$\Univ(\glie)$ be the universal enveloping algebra of~$\glie$ and let~$\iota$ be the canonical homorphism of Lie~algebras from~$\glie$ to~$\Univ(\glie)$.
	\begin{enumerate}
		\item
			For every homomorphism of Lie~algebras~$\rho$ from~$\glie$ to~$\gllie(M)$ let~$\widehat{\rho}$ denote the corresponding homomorphism of algebras from~$\Univ(\glie)$ to~$\End_{\kf}(M)$.
			Then the assignments
			\begin{align*}
				\left\{
				\begin{tabular}{c}
					representations\index{representation} \\
					$\rho \colon \glie \to \gllie(M)$
				\end{tabular}
				\right\}
				&\onetoone
				\left\{
				\begin{tabular}{c}
					$\Univ(\glie)$-module structures \\
					$\Rho \colon \Univ(\glie) \to \End_{\kf}(M)$
				\end{tabular}
				\right\}  \,,
				\\
				\rho
				&\mapsto
				\widehat{\rho} \,,
				\\
				\Rho \circ \iota
				&\mapsfrom
				\Rho  \,,
			\end{align*}
			constitute a {\onetoonetext} correspondence.
		\item
			For every anti-homomorphism of Lie~algebras~$\rho$ from~$\glie$ to~$\gllie(M)$ let~$\widehat{\rho}$ denote the corresponding anti-homomorphism of algebras from~$\Univ(\glie)$ to~$\End_{\kf}(M)$.
			Then the assignments
			\begin{align*}
				\left\{
				\begin{tabular}{c}
					right representations \\
					$\rho \colon \glie \to \gllie(M)$
				\end{tabular}
				\right\}
				&\onetoone
				\left\{
				\begin{tabular}{c}
					right $\Univ(\glie)$-module structures \\
					$\Rho \colon \Univ(\glie) \to \End_{\kf}(M)$
				\end{tabular}
				\right\}  \,,
				\\
				\rho
				&\mapsto
				\widehat{\rho} \,,
				\\
				\Rho \circ \iota
				&\mapsfrom
				\Rho  \,,
			\end{align*}
			constitute a {\onetoonetext} correspondence.
		\qed
	\end{enumerate}
\end{proposition}

\begin{fluff}
	Let~$\glie$ be a Lie~algebra.
	If~$M$ is a left~\module{$\Univ(\glie)$}, then the corresponding action of~$\glie$ on~$M$ is given by
	\[
		x \act m
		=
		\class{x} \cdot m
	\]
	for all~$x \in \glie$ and~$m \in M$.
	Similarly, if~$M$ is a right~\module{$\Univ(\glie)$}, then the corresponding right action of~$\glie$ on~$M$ is given by
	\[
		m \act x
		=
		m \cdot \class{x}
	\]
	for all~$x \in \glie$ and~$m \in M$.
\end{fluff}

\begin{remark}
	Let~$\glie$ be a Lie~algebra and let~$\Univ(\glie)$ be the universal enveloping algebra of~$\glie$.
	\Cref{representations are modules} shows that representations of~$\glie$ are the same as~{\modules{$\Univ(\glie)$}}.
	We get from this correspondence an isomorphism of categories between~$\cRep{\glie}$ and~$\cMod{\Univ(\glie)}$.
\end{remark}

\subsubsection{Functoriality}

\begin{proposition}[Functoriality of the universal enveloping algebra]
	\label{functoriality of universal enveloping algebra}
	Let~$\glie$,~$\hlie$ and~$\klie$ be Lie~algebras.
	\begin{enumerate}
		\item
			For every homomorphism of Lie~algebras~$\varphi$ from~$\glie$ to~$\hlie$ there exists a unique homomorphism of algebras~$\Univ(\varphi)$\glsadd{universal enveloping algebra on morphisms}\index{induced homomorphism!of universal enveloping algebras} from~$\Univ(\glie)$ to~$\Univ(\hlie)$ that makes the following square diagram commute.
			\[
				\begin{tikzcd}[column sep = large]
					\glie
					\arrow{r}[above]{\varphi}
					\arrow{d}[left]{\iota_{\glie}}
					&
					\hlie
					\arrow{d}[right]{\iota_{\hlie}}
					\\
					\Univ(\glie)
					\arrow[dashed]{r}[below]{\Univ(\varphi)}
					&
					\Univ(\hlie)
				\end{tikzcd}
			\]
		\item
			It holds that~$\Univ(\id_{\glie}) = \id_{\Univ(\glie)}$.
		\item
			It holds for overy homomorphisms of Lie~algebras~$\varphi$ from~$\glie$ to~$\hlie$ and every homomorphism of Lie~algebra~$\psi$ from~$\hlie$ to~$\klie$ that
			\[
				\Univ( \psi \circ \varphi )
				=
				\Univ( \psi ) \circ \Univ( \varphi ) \,.
			\]
	\end{enumerate}
\end{proposition}


\begin{proof}
	\leavevmode
	\begin{enumerate}
		\item
			The composite~$\iota_{\hlie} \circ \varphi$ is a homomorphism of Lie~algebras from~$\glie$ to~$\Univ(\hlie)$.
			By the universal property of the universal enveloping algebra~$\Univ(\glie)$ there exists a unique homomorphism of algebras~$\Univ(\varphi)$ from~$\Univ(\glie)$ to~$\Univ(\hlie)$ with~$\Univ(\varphi) \circ \iota_{\glie} = \iota_{\hlie} \circ \varphi$.
		\item
			The square diagram
			\[
				\begin{tikzcd}[column sep = huge]
					\glie
					\arrow{r}[above]{\id_{\glie}}
					\arrow{d}
					&
					\glie
					\arrow{d}
					\\
					\Univ(\glie)
					\arrow[dashed]{r}[below]{\id_{\Univ(\glie)}}
					&
					\Univ(\glie)
				\end{tikzcd}
			\]
			commutes, which shows that the identity homomorphism~$\id_{\Univ(\glie)}$ satisfies the defining property of the induced algebra homomorphism~$\Univ( \id_{\glie} )$.
		\item
			We have the following commutative diagram:
			\[
				\begin{tikzcd}[column sep = large]
					\glie
					\arrow[dashed, bend left = 40]{rr}[above]{\psi \circ \varphi}
					\arrow{r}[above]{\varphi}
					\arrow{d}
					&
					\hlie
					\arrow{r}[above]{\psi}
					\arrow{d}
					&
					\klie
					\arrow{d}
					\\
					\Univ(\glie)
					\arrow{r}[below]{\Univ(\varphi)}
					\arrow[dashed, bend right = 40]{rr}[below]{\Univ(\psi) \circ \Univ(\varphi)}
					&
					\Univ(\hlie)
					\arrow{r}[below]{\Univ(\psi)}
					&
					\Univ(\klie)
				\end{tikzcd}
			\]
			The commutativity of the outer square diagram
			\[
				\begin{tikzcd}[column sep = huge]
					\glie
					\arrow{r}[above]{\psi \circ \varphi}
					\arrow{d}
					&
					\klie
					\arrow{d}
					\\
					\Univ(\glie)
					\arrow[dashed]{r}[below]{\Univ(\psi) \circ \Univ(\varphi)}
					&
					\Univ(\klie)
				\end{tikzcd}
			\]
			shows that the composite~$\Univ(\psi) \circ \Univ(\varphi)$ satisfies the defining property of the induced algebra homomorphism~$\Univ(\psi \circ \varphi)$.
		\qedhere
	\end{enumerate}
\end{proof}


\begin{remark}
	\Cref{functoriality of universal enveloping algebra} shows that the assignment~$\glie \mapsto \Univ(\glie)$ of a Lie~algebra~$\glie$ to its universal eveloping algebra~$\Univ(\glie)$ can be extended to a (covariant) functor~$\Univ$ from~$\cLie{\kf}$ to~$\cAlg{\kf}$.
	The universal property of the universal enveloping algebra states that the functor~$\Univ$ is left adjoint\index{adjunction} to the forgetful functor from~$\cAlg{\kf}$ to~$\cLie{\kf}$, which assigns to each~{\algebra{$\kf$}} its underlying Lie~algebra.
\end{remark}


\begin{remark}
	\label{actions on the universal enveloping algebra}
	Let~$\glie$ be a Lie~algebra.
	Its universal enveloping algebra~$\Univ(\glie)$ can be made into a representation of~$\glie$ in two ways.
	\begin{enumerate}
		\item
			The algebra~$\Univ(\glie)$ becomes a left module over itself via left multiplication.
			This~\module{$\Univ(\glie)$} structure corresponds to an action of~$\glie$ on~$\Univ(\glie)$ given by
			\[
				x \act t
				=
				\class{x} \cdot t
			\]
			for all~$x \in \glie$ and~$t \in \Univ(\glie)$.
		\item
			For the second action let~$\iota$ denote the canonical homomorphism from~$\glie$ to~$\Univ(\glie)$.
			The adjoint representation of~$\Univ(\glie)$ is a homomorphism of Lie~algebras from~$\Univ(\glie)$ to~$\gllie(\Univ(\glie))$.
			We can pull-back this representation via~$\iota$ to an action of~$\glie$ on~$\Univ(\glie)$, given by
			\[
				x \act t
				=
				[ \class{x}, t ]
			\]
			for all~$x \in \glie$ and~$t \in \Univ(\glie)$.
			Unter this section action, the homomorphism of Lie~algebras~$\iota$ becomes a homomorphism of representations from the adjoint representation of~$\glie$ to~$\Univ(\glie)$.
	\end{enumerate}
\end{remark}


\begin{definition}
	Let~$\glie$ be a Lie~algebra.
	The first action from \cref{actions on the universal enveloping algebra} is the \defemph{regular action} of~$\glie$ on~$\Univ(\glie)$, and the second action is the \defemph{adjoint action} of~$\glie$ on~$\Univ(\glie)$.
\end{definition}

\subsubsection{Functoriality for Anti-Homomorphisms}

\begin{proposition}
	\label{functoriality of universal enveloping algebra for anti-homomorphisms}
	Let~$\glie$,~$\hlie$, and~$\klie$ be Lie~algebras.
	\begin{enumerate}
		\item
			Let~$\varphi$ be an anti-homomorphism of Lie~algebras from~$\glie$ to~$\hlie$.
			There exists a unique anti-homomorphism of algebras~$\widetilde{\Univ}(\varphi)$\index{induced anti-homomorphism of universal enveloping algebras} from~$\glie$ to~$\hlie$ that makes the following square diagram commute.
			\[
				\begin{tikzcd}[column sep = large]
					\glie
					\arrow{r}[above]{\varphi}
					\arrow{d}[left]{\iota_{\glie}}
					&
					\hlie
					\arrow{d}[right]{\iota_{\hlie}}
					\\
					\Univ(\glie)
					\arrow[dashed]{r}[above]{\widetilde{\Univ}(\varphi)}
					&
					\Univ(\hlie)
				\end{tikzcd}
			\]
		\item
			Let~$\varphi$ be a map from~$\glie$ to~$\hlie$ and let~$\psi$ be a map from~$\hlie$ to~$\klie$.
			\begin{enumerate}
				\item
					If~$\varphi$ and~$\psi$ are homomorphisms of Lie~algebras, then their composite~$\psi \circ \varphi$ is again a homomorphism of Lie~algebras and~$\Univ(\psi \circ \varphi) = \Univ(\psi) \circ \Univ(\varphi)$.
				\item
					If~$\varphi$ is a homomorphisms of Lie~algebras and~$\psi$ is an anti-homomorphism of Lie~algebras, then their composite~$\psi \circ \varphi$ is an anti-homomorphism and~$\widetilde{\Univ}(\psi \circ \varphi) = \widetilde{\Univ}(\psi) \circ \Univ(\varphi)$.
				\item
					If~$\varphi$ is an anti-homomorphisms of Lie~algebras and~$\psi$ is a homomorphism of Lie~algebras, then their composite~$\psi \circ \varphi$ is an anti-homomorphism and~$\widetilde{\Univ}(\psi \circ \varphi) = \Univ(\psi) \circ \widetilde{\Univ}(\varphi)$.
				\item
				If~$\varphi$ and~$\psi$ are anti-homomorphisms of Lie~algebras, then their composite~$\psi \circ \varphi$ is a homomorphism of Lie~algebras and~$\Univ(\psi \circ \varphi) = \widetilde{\Univ}(\psi) \circ \widetilde{\Univ}(\varphi)$.
			\end{enumerate}
		\item
			\label{induced homomorphism same for homomorphisms and anti-homomorphisms}
			Let~$\varphi$ be a map from~$\glie$ to~$\hlie$ that is both a homomorphism of Lie~algebras and an anti-homomorphism of Lie~algebras.
			Then the induced anti-homomorphism of algebras~$\Univ(\varphi)$ agrees with the induced anti-homomorphism of algebras~$\widetilde{\Univ}(\varphi)$.
	\end{enumerate}
\end{proposition}


\begin{proof}
	\leavevmode
	\begin{enumerate}
		\item
			The composite~$\iota_{\hlie} \circ \varphi$ is an anti-homomorphism of Lie~algebras from~$\glie$ to~$\Univ(\hlie)$, and thus a homomorphism of Lie~algebras from~$\glie$ to~$\Univ(\hlie)^{\op}$.
			It thus extends uniquely to a homomorphism of algebras~$\Univ(\varphi)$ from~$\Univ(\glie)$ to~$\Univ(\hlie)^{\op}$ such that~$\Univ(\varphi) \circ \iota_{\glie} = \iota_{\hlie} \circ \varphi$.
			We can now regard~$\Univ(\varphi)$ as an anti-homomorphism of algebras~$\widetilde{U}(\varphi)$ from~$\Univ(\glie)$ to~$\Univ(\hlie)$.
		\item
			If~$\varphi$ and~$\psi$ are homomorphism of Lie~algebras, then this is known from \cref{functoriality of universal enveloping algebra}.
			The other cases can be shown in the same way is in the proof of \cref{functoriality of universal enveloping algebra}.
		\item
			Suppose first that the Lie~algebra~$\hlie$ is abelian.
			Its universal enveloping algebra~$\Univ(\hlie)$ is commutative because it is generated by the image of~$\hlie$ in~$\Univ(\hlie)$.
			The anti-homomorphism of algebras~$\widetilde{\Univ}(\varphi)$ is therefore a homomorphism of algebras, and it satisfies the defining property of the induced homomorphism of algebras~$\Univ(\varphi)$.
			We thus find that~$\widetilde{\Univ}(\varphi)$ equals~$\Univ(\varphi)$.

			For the general case let~$\ilie$ be the image of~$\varphi$.
			This is a Lie~subalgebra of~$\hlie$ because~$\varphi$ is a homomorphism of Lie~algebras.
			Let~$\varphi'$ be the corestriction of~$\varphi$ to a map from~$\glie$ to~$\varphi'$ and let~$\iota$ be the inclusion map from~$\ilie$ to~$\hlie$.
			The map~$\varphi'$ is again both a homomorphism of Lie~algebras and an anti-homomorphism of Lie~algebras, and the inclusion map~$\iota'$ is a homomorphism of Lie~algebras.
			The Lie~algebra~$\ilie$ is abelian because
			\[
				[\varphi(x), \varphi(y)]
				=
				\varphi( [x,y] )
				=
				[\varphi(y), \varphi(x)]
			\]
			for all~$x, y \in \glie$.
			It follows from the previously considered special case that~$\Univ(\varphi') = \widetilde{\Univ}(\varphi')$.
			It finally follows that
			\[
				\widetilde{\Univ}(\varphi)
				=
				\widetilde{\Univ}(\iota \circ \varphi')
				=
				\Univ(\iota) \circ \widetilde{\Univ}(\varphi')
				=
				\Univ(\iota) \circ \Univ(\varphi')
				=
				\Univ(\iota \circ \varphi')
				=
				\Univ(\varphi) \,,
			\]
			as desired.
		\qedhere
	\end{enumerate}
\end{proof}


\begin{notation}
	Let~$\glie$ and~$\hlie$ be two Lie~algebras and let~$\varphi$ be a anti-homomorphism of Lie~algebras from~$\glie$ to~$\hlie$.
	Instead of~$\widetilde{\Univ}(\varphi)$ we will write~$\Univ(\varphi)$\glsadd{universal enveloping algebra on morphisms}.
	This does not need to conflicts in the case that~$\varphi$ is also a homomorphism of Lie~algebras, as shown in part~\ref{induced homomorphism same for homomorphisms and anti-homomorphisms} of \cref{functoriality of universal enveloping algebra for anti-homomorphisms}.
\end{notation}


\begin{corollary}
	\label{anti-isomorphism of lie-algebras induces anti-isomorphism of algebras}
	Let~$\glie$ and~$\hlie$ be two Lie~algebras and let~$\varphi$ be an anti-isomorphism of Lie~algebras from~$\glie$ to~$\hlie$.
	The induced anti-homorphism of algebras~$\Univ(\varphi)$ from~$\Univ(\glie)$ to~$\Univ(\hlie)$ is an isomorphism of algebras.
	\qed
\end{corollary}


\begin{example}
	\label{construction of antipode}
	Let~$\glie$ be a Lie~algebra.
	The map
	\[
		\sigma
		\colon
		\glie
		\to
		\glie \,,
		\quad
		x
		\mapsto
		-x
	\]
	is an anti-involution of Lie~algebras of~$\glie$.
	The induced anti-homomorphism of Lie~algebras~$\Univ(\sigma)$ is an anti-involution of algebras of~$\Univ(\glie)$.
\end{example}


\begin{definition}
	Let~$\glie$ be a Lie~algebra.
	In the situation of \cref{construction of antipode}, the induced anti-involution~$\Univ(\sigma)$ of~$\Univ(\glie)$ is the \defemph{antipode}\index{antipode} of~$\Univ(\glie)$.
	It is denoted by~$S_{\glie}$, or simply by~$S$\glsadd{antipode}.
\end{definition}


\subsubsection{Derivations}

\begin{proposition}
	\label{extending derivation to universal enveloping algebra}
	Let~$\glie$ be a Lie~algebra.
	Every derivation\index{derivation!of Lie algebra} of~$\glie$ extends uniquely to a derivation of~$\Univ(\glie)$.
	More explicitely, there exists for every Lie~algebra derivation~$\delta$ of~$\glie$ a unique algebra derivation~$\Delta$ of~$\Univ(\glie)$ such that the following square diagram commutes.
	\[
		\begin{tikzcd}
			\glie
			\arrow{r}[above]{\delta}
			\arrow{d}
			&
			\glie
			\arrow{d}
			\\
			\Univ(\glie)
			\arrow[dashed]{r}[above]{\Delta}
			&
			\Univ(\glie) \,.
		\end{tikzcd}
	\]
\end{proposition}


\begin{fluff}
	To prove \cref{extending derivation to universal enveloping algebra} we need some preparation.
	To employ the universal property of the universal enveloping algebra we want to be able to translate between derivations and homomorphisms.
	This is what the next \lcnamecref{translating between derivations and homomorphisms} is about.
\end{fluff}


\begin{lemma}
	\label{translating between derivations and homomorphisms}
	\leavevmode
	\begin{enumerate}
		\item
			Let~$A$ be a~\algebra{$\kf$} and let~$B$ be the~\algebra{$\kf$}
			\[
				B
				\defined
				\begin{pmatrix}
					A & A \\
					0 & A
				\end{pmatrix} \,.
			\]
			A map~$\Delta$ from~$A$ to~$A$ is a derivation of~$A$ if and only if the map
			\[
				\Phi
				\colon
				A
				\to
				B \,,
				\quad
				a
				\mapsto
				\begin{pmatrix}
					a & \Delta(a) \\
					0 & a
				\end{pmatrix}
			\]
			is a homomorphism of algebras.
		\item
			Let~$\glie$ be a Lie~algebra and let
			\[
				B
				\defined
				\begin{pmatrix}
					\Univ(\glie)  & \Univ(\glie)  \\
					0             & \Univ(\glie)
				\end{pmatrix} \,.
			\]
			We regard~$\Univ(\glie)$ as a representation of~$\glie$ via the adjoint action of~$\glie$ on~$\Univ(\glie)$.
			A map~$\delta$ from~$\glie$ to~$\Univ(\glie)$ is a derivation if and only if the map
			\[
				\varphi
				\colon
				\glie
				\to
				B \,,
				\quad
				x
				\mapsto
				\begin{pmatrix}
					\class{x} & \delta(x) \\
					0         & \class{x}
				\end{pmatrix}
			\]
			is a homomorphism of Lie~algebras
	\end{enumerate}
\end{lemma}


\begin{proof}
	\leavevmode
	\begin{enumerate}
		\item
			The map~$\Phi$ is linear if and only if the map~$\Delta$ is linear.
			We have
			\[
				\Phi(a) \cdot \Phi(b)
				=
				\begin{pmatrix}
					a & \Delta(a) \\
					0 & a
				\end{pmatrix}
				\begin{pmatrix}
					b & \Delta(b) \\
					0 & b
				\end{pmatrix}
				=
				\begin{pmatrix}
					ab  & \Delta(a) b + a \Delta(b) \\
					0   & ab
				\end{pmatrix}
			\]
			for all~$a, b \in A$.
			It follows that the map~$\Phi$ is multiplicaitve if and only if~$\Delta(ab) = \Delta(a) b + a \Delta(b)$ for all~$a, b \in A$.
			If~$\Delta$ is a derivation then we have~$\Phi(1) = 1$.
			This shows altogether that~$\Delta$ is a derivation if and only if~$\Phi$ is a homomorphism of algebras.
		\item
			We have
			\begin{align*}
				[ \varphi(x), \varphi(y) ]
				&=
				\Biggl[
					\begin{pmatrix}
						\class{x} & \delta(x) \\
						0         & \class{x}
					\end{pmatrix},
					\begin{pmatrix}
						\class{y} & \delta(y) \\
						0         & \class{y}
					\end{pmatrix}
				\Biggr]
				\\
				&=
				\begin{pmatrix}
					\class{x} & \delta(x) \\
					0         & \class{x}
				\end{pmatrix}
				\begin{pmatrix}
					\class{y} & \delta(y) \\
					0         & \class{y}
				\end{pmatrix}
				-
				\begin{pmatrix}
					\class{y} & \delta(y) \\
					0         & \class{y}
				\end{pmatrix}
				\begin{pmatrix}
					\class{x} & \delta(x) \\
					0         & \class{x}
				\end{pmatrix}
				\\
				&=
				\begin{pmatrix}
					\class{xy}  & \class{x} \delta(y) + \delta(x) \class{y} \\
					0           & \class{xy}
				\end{pmatrix}
				-
				\begin{pmatrix}
					\class{yx}  & \class{y} \delta(x) + \delta(y) \class{x} \\
					0           & \class{yx}
				\end{pmatrix}
				\\
				&=
				\begin{pmatrix}
					\class{x}\class{y} - \class{y}\class{x} & \class{x} \delta(y) - \delta(y) \class{x} + \delta(x) \class{y} - \class{y} \delta(x) \\
					0                                       & \class{x}\class{y} - \class{y}\class{x}
				\end{pmatrix}
				\\
				&=
				\begin{pmatrix}
					\class{[x,y]} & [\class{x}, \delta(y)] - [\class{y}, \delta(x)] \\
					0             & \class{[x,y]}
				\end{pmatrix}
				\\
				&=
				\begin{pmatrix}
					\class{[x,y]} & x \act \delta(y) -  y \act \delta(x) \\
					0             & \class{[x,y]}
				\end{pmatrix}
			\end{align*}
			for all~$x, y \in \glie$.
			It follows that the map~$\varphi$ is a homomorphism of Lie~algebras if and only if~$\delta([x,y]) = x \act \delta(x) - y \act \delta(x)$ for all~$x, y \in \glie$, i.e. if and only if the map~$\delta$.
		\qedhere
	\end{enumerate}
\end{proof}


\begin{proof}[Proof of \cref{extending derivation to universal enveloping algebra}]
	The uniqueness of~$\Delta$ follows from \cref{dervation is uniquely determined by algebra generators} because the universal enveloping algebra~$\Univ(\glie)$ is generated by the image of~$\glie$ in~$\Univ(\glie)$.

	Let~$B$ be the~\algebra{$\kf$} given by
	\[
		B
		\defined
		\begin{pmatrix}
			\Univ(\glie)  & \Univ(\glie) \\
			0             & \Univ(\glie)
		\end{pmatrix} \,.
	\]
	We regard~$\Univ(\glie)$ as a representation of~$\glie$ via the adjoint action of~$\glie$ on~$\Univ(\glie)$.
	The canonical homomorphism of Lie~algebras from~$\glie$ to~$\Univ(\glie)$ is a homomorphism of representations, whence it follows that the map
	\[
		\glie
		\to
		\Univ(\glie) \,,
		\quad
		x
		\mapsto
		\class{\delta(x)}
	\]
	is again a derivation.
	It follows from \cref{translating between derivations and homomorphisms} that the map
	\[
		\varphi
		\colon
		\glie
		\to
		B \,,
		\quad
		x
		\mapsto
		\begin{pmatrix}
			\class{x} & \class{\delta(x)} \\
			0         & \class{x}
		\end{pmatrix}
	\]
	is a homomorphism of Lie~algebras because~$\delta$ is a derivation of~$\glie$.
	It follows from the universal property of the universal enveloping algebra~$\Univ(\glie)$ that the homomorphism of Lie~algebras~$\varphi$ extends uniquely to a homomorphism of algebras~$\Phi$ from~$\Univ(\glie)$ to~$B$.
	This homomorphism~$\Phi$ is of the form
	\[
		\Phi(y)
		=
		\begin{pmatrix}
			\Phi_1(y) & \Delta(y) \\
			0         & \Phi_2(y)
		\end{pmatrix}
		\qquad
		\text{for every~$y \in \Univ(\glie)$}
	\]
	for some unique linear maps
	\[
		\Phi_1, \Phi_2, \Delta
		\colon
		\Univ(\glie)
		\to
		\Univ(\glie) \,.
	\]
	The maps~$\Phi_1$ and~$\Phi_2$ are homomorphisms of algebras because~$\Phi$ is a homomorphism of algebras.
	They satisfy the equalities~$\Phi_1(\class{x}) = \varphi(x) = \class{x}$ and~$\Phi_2(\class{x}) = \varphi(x) = \class{x}$ for every~$x \in \glie$.
	It follows that~$\Phi_1(y) = y$ and~~$\Phi_2(y) = y$ for every~$y \in \Univ(\glie)$ because~$\glie$ generates~$\Univ(\glie)$ as an algebra.
	The homomorphism~$\Phi$ is thus of the form
	\[
		\Phi(y)
		=
		\begin{pmatrix}
			y & \Delta(y) \\
			0 & y
		\end{pmatrix}
		\qquad
		\text{for every~$y \in \Univ(\glie)$.}
	\]
	It follows from \cref{translating between derivations and homomorphisms} that the linear map~$\Delta$ is an algebra derivation of~$\Univ(\glie)$.
	This darivation satisfies the equality~$\Delta(\class{x}) = \class{\delta(x)}$ for every~$x \in \glie$ because~$\Phi$ is an extension of~$\varphi$.
	In other words,~$\Delta$ is an extension of~$\delta$.
\end{proof}



\section{Examples}


% TODO: UEA of semidirect product.


\subsection{Abelian Lie~Algebras}


\begin{examples}
	Let~$\glie$ be an abelian Lie~algebra.
	It follows from the explicit construction of the universal enveloping algebra~$\Univ(\glie)$ that
	\[
		\Univ(\glie)
		\cong
		\Tensor(\glie)/(x \tensor y - y \tensor x \suchthat x, y \in \glie)
		\cong
		\Symm(\glie) \,,
	\]
	with the canonical homomorphism of Lie~algebras from~$\glie$ to~$\Univ(\glie)$ corresponding to the inclusion map from~$\glie$ to~$\Symm(\glie)$\index{universal enveloping algebra!of an abelian Lie algebra}.

	This can also be seen more abstractly, as follows.
	
	We observe that if~$V$ is any~\vectorspace{$\kf$} and~$A$ is any~\algebra{$\kf$}, then a linear map~$f$ from~$V$ to~$A$ extends to a homomorphism of algebras from~$\Symm(V)$ to~$A$ if and only if the image of~$f$ is contained in a commutative subalgebra of~$A$, if and only if the image of~$f$ is commutative in~$A$.
	(By this we mean that any two elements of the image of~$f$ commute with each other.)
	It follows that we have for any~\algebra{$\kf$}~$A$ bijections
	\begin{align*}
		{}&
		\{ \textstyle \text{Lie~algebra homomorphisms~$\glie \to A$} \}
		\\
		\cong{}&
		\{ \textstyle \text{{\linear{$\kf$}} maps~$\glie \to A$ with commutative image} \}
		\\
		\cong{}&
		\{ \textstyle \text{algebra homomorphisms~$\Symm(\glie) \to A$} \} \,.
	\end{align*}
	These bijections are natural in~$A$.
	This shows that the symmetric algebra~$\Symm(\glie)$ together with the inclusion map from~$\glie$ to~$\Symm(\glie)$ satisfies the universal property of the universal enveloping algebra of~$\glie$.
\end{examples}


\begin{example}
	We find for~$\glie = 0$ that~$\Univ(\glie) = \kf$.
\end{example}


\begin{definition}
	\leavevmode
	\begin{enumerate}
		\item
			Let~$A$ be an algebra.
			An \defemph{augumentation}\index{augumentation} of~$A$ is a homomorphism of algebras~$\varepsilon$\glsadd{augumentation} from~$A$ to~$\kf$.
		\item
			An \defemph{augumented~\algebra{$\kf$}}\index{augumented algebra} is a~\algebra{$\kf$}~$A$ together with an augumentation of~$A$.
		\item
			Let~$(A, \varepsilon)$\glsadd{augumented algebra} be an augumented algebra.
			The kerrnel of~$\varepsilon$ is the \defemph{augumentation ideal}\index{augumentation ideal} of~$A$.
	\end{enumerate}
\end{definition}


\begin{proposition}
	\label{decomposition for augumented algebra}
	Let~$(A, \varepsilon) $ be an augumented~\algebra{$\kf$}.
	Then~$A = \kf \oplus \ker(\varepsilon)$ as vector spaces.
\end{proposition}


\begin{proof}[First proof]
	We have~$\varepsilon(1) = 1$ because~$\varepsilon$ is a homomorphism of algebras.
	This shows that~$1$ is not contained in the augumentation ideal, and that the augumentation~$\varepsilon$ is nonzero.
	The image of~$\varepsilon$ is thus one-dimensional, whence the kernel of~$\varepsilon$ has codimension~$1$ in~$A$.
	It follows that~$A = \gen{1}_{\kf} \oplus \ker(\varepsilon) = \kf \oplus \ker(\varepsilon)$.
\end{proof}


\begin{proof}[Second proof]
	Let~$\eta$ be the inclusion from~$\kf$ to~$A$.
	The composito~$\varepsilon \circ \eta$ is the identity on~$\kf$ whence the composite~$\eta \circ \varepsilon$ is idempotent.
	It follows that
	\[
		A
		=
		\im(\eta \circ \varepsilon)
		\oplus \ker(\eta \circ \varepsilon) \,.
	\]
	It follows from the surjectivity of~$\varepsilon$ that~$\im(\eta \circ \varepsilon) = \im(\eta) = \kf$ and from the injectivity of~$\eta$ that~$\ker(\eta \circ \varepsilon) = \ker(\varepsilon)$.
\end{proof}


\begin{corollary}
	Every augumented~\algebra{$\kf$} is nonzero.
	\qedhere
\end{corollary}


\begin{construction}
	\label{construction of counit}
	Let~$\glie$ be a Lie~algebra.
	We have a unique homomorphism of Lie~algebras from~$\glie$ to the zero Lie~algebra.
	This homomorphism of Lie~algebras induces a homomorphism of algebras
	\[
		\varepsilon
		\colon
		\Univ(\glie)
		\to
		\kf \,.
	\]
	This homomorphism of algebras is uniquely determined by the condition
	\[
		\varepsilon( \class{x} ) = 0
		\qquad
		\text{for every~$x \in \glie$}
	\]
	because~$\Univ(\glie)$ is generated as an algebra by the image of~$\glie$ in~$\Univ(\glie)$.
\end{construction}


\begin{definition}
	Let~$\glie$ be a Lie~algebra.
	The homomorphism of algebras~$\varepsilon$\glsadd{counit} from~$\Univ(\glie)$ to~$\kf$ from \cref{construction of counit} is the \defemph{counit}\index{counit} of~$\glie$.
	Its makes~$\Univ(\glie)$ into an augumented algebra.
\end{definition}


\begin{remark}
	Let~$\glie$ be a Lie~algebra.
	We can regard the vector space~$\kf$ as the trivial representation of~$\glie$, and thus as a~\module{$\Univ(\glie)$}.
	This~\module{$\Univ(\glie)$} structure can also be explained with help of the counit~$\varepsilon$, because~$\varphi$ is a homomorphism of algebras from~$\Univ(\glie)$ to~$\kf = \End_{\kf}(\kf)$.
\end{remark}


\begin{proposition}
	\label{augumentation ideal is spanned by monomials}
	The augumentation ideal of~$\Univ(\glie)$ is spanned as a vector space by all the monomials~$\class{x_1} \dotsm \class{x_n}$ with~$n \geq 1$ and~$x_1, \dotsc, x_n \in \glie$.
\end{proposition}


\begin{proof}
	Let~$\varepsilon$ be the counit of~$\Univ(\glie)$, let~$J$ be the linear subspace of~$\Univ(\glie)$ spanned by all monomials~$\class{x_1} \dotsm \class{x_n}$ with~$n \geq 0$,~$x_1, \dotsc, x_n \in \glie$.
	We have
	\[
		\varepsilon( \class{x_1} \dotsm \class{x_n} )
		=
		\varepsilon( \class{x_1} ) \dotsm \varepsilon( \class{x_n} )
		=
		0 \dotsm 0
		=
		0
	\]
	for each such monomial, whence the linear space~$J$ is contained in~$\ker(\varepsilon)$.
	We know on the other hand that~$\Univ(\glie)$ is generated by the image of~$\glie$ in~$\Univ(\glie)$ as an algebra.
	This means that the monomials
	\[
		\class{x_1}  \dotsm \class{x_n}
		\qquad
		\text{with~$n \geq 0$ and~$x_1, \dotsc, x_n \in \glie$}
	\]
	span the algebra~$\Univ(\glie)$ as a vector space.
	We thus have~$\Univ(\glie) = \kf + J$.
	Together with the decomposition~$\Univ(\glie) = \kf \oplus \ker(\varepsilon)$ and the inclusion~$U \subseteq \ker(\varepsilon)$ we find that~$J = \ker(\varepsilon)$.
\end{proof}



\subsection{Opposite Lie~Algebra}

\begin{example}
	\label{uea of opposite by first principles}
	\leavevmode
	\begin{enumerate}
		\item
			Let~$\glie$ be a Lie~algebra.
			We have for every~{\algebra{$\kf$}}~$A$ bijections
			\begin{align*}
				{}&
				\{ \text{algebra homomorphisms~$\Univ(\glie^\op) \to A$   } \}
				\\
				\cong{}&
				\{ \text{Lie~algebra homomorphisms~$\glie^\op \to A$} \}
				\\
				={}&
				\{ \text{Lie~algebra homomorphisms~$\glie \to A^\op$} \}
				\\
				\cong{}&
				\{ \text{algebra homomorphisms~$\Univ(\glie) \to A^\op$} \}
				\\
				={}&
				\{ \text{algebra homomorphisms~$\Univ(\glie)^\op \to A$} \}
			\end{align*}
			that are natural in~$A$.%
			\footnote{
				We use here implicitely that taking the underlying Lie~algebra of a~{\algebra{$\kf$}} commutes with taking opposites.}
				It follows from Yoneda’s~lemma that~$\Univ(\glie^\op) \cong \Univ(\glie)^\op$\index{universal enveloping algebra!of the opposite Lie~algebra}.
			The canonical homomorphism of Lie~algebras from~$\glie^{\op}$ to~$\Univ(\glie^{\op})$ corresponds to the homomorphism from~$\glie^{\op}$ to~$\Univ(\glie)^{\op}$ given by~$\class{x^{\op}} \mapsto \class{x}{}^{\,\op}$ for every~$x \in \glie$.

			We can also derive the above isomorphism in a more explicit way.
			Indeed, the map
			\[
				\sigma
				\colon
				\glie^{\op}
				\to
				\glie \,,
				\quad
				x^{\op}
				\mapsto
				x
			\]
			is an anti-isomorphism of Lie~algebras, and thus induces the anti-isomorphism of algebras~$\Univ(\sigma)$ from~$\Univ(\glie^{\op})$ to~$\Univ(\glie)$.
			We can regard this anti-isomorphism~$\Univ(\sigma)$ as an isomorphism of algebras~$\Phi$ from~$\Univ(\glie^{\op})$ to~$\Univ(\glie)^{\op}$.
			This isomorphism is uniquely determined by
			\[
				\Phi\Bigl( \class{x^{\op}} \Bigr)
				=
				- \class{x}
				\qquad
				\text{for every~$x \in \glie$.}
			\]
		\item
			The map
			\[
				\tau
				\colon
				\glie^{\op}
				\to
				\glie \,,
				\quad
				x^{\op}
				\mapsto
				-x
			\]
			is an isomorphism of Lie~algebras and thus induces an isomorphism of algebras~$\Psi \defined \Univ(\tau)$ from~$\Univ(\glie^{\op})$ to~$\Univ(\glie)$.
			This isomorphism is uniquely determined by
			\[
				\Univ(\tau)\Bigl( \class{x^{\op}} \Bigr)
				=
				- \class{x} \,,
				\qquad
				\text{for every~$x \in \glie$.}
			\]
	\end{enumerate}
\end{example}




%\begin{remark}
%	Let~$\glie$ be a Lie~algebra.
%	For every representation~$M$ of~$\glie$ its dual~$M^*$ becomes again a representation of~$\glie$ via the action
%	\[
%		(x \act \varphi)(m)
%		=
%		- \varphi(m)
%	\]
%	for all~$x \in \glie$,~$\varphi \in M^*$,~$m \in M$.
%	In other words, for every~\module{$\Univ(\glie)$}~$M$ its dual~$M^*$ becomes again a~\module{$\Univ(\glie)$}.
%
%	This can also be explained via the antipode.
%	Let~$M$ be a~\module{$\Univ(\glie)$}.
%	The the dual~$M^*$ becomes a right~\module{$\Univ(\glie)$} via the multiplication
%	\[
%		(\varphi \cdot y)(m)
%		=
%		\varphi(ym)
%	\]
%	for all~$y \in \Univ(\glie)$,~$\varphi \in M^*$,~$m \in M$.
%	This right~\module{$\Univ(\glie)$} structure on~$M^*$ corresponds to a left~\module{$\Univ(\glie)^{\op}$} structure on~$M^*$ given by
%	\[
%		y^{\op} \cdot \varphi
%		=
%		\varphi \cdot y
%	\]
%	for all~$y \in \Univ(\glie)$,~$\varphi \in M^*$.
%	By using the isomorphism of algebras~$S$ from~$\Univ(\glie)$ to~$\Univ(\glie)^{\op}$ we can pull back this~\module{$\Univ(\glie)^{\op}$} structure to a~\module{$\Univ(\glie)$} structure given by
%	\[
%		y \cdot \varphi
%		=
%		S(y) \cdot \varphi
%	\]
%	for all~$y \in \Univ(\glie)$,~$\varphi \in M^*$.
%
%	For every element~$x$ of~$\glie$ we have
%	\[
%		(\class{x} \cdot \varphi)(m)
%		=
%		( S(\class{x}) \cdot \varphi )(m)
%		=
%		( - \class{x}^{\,\op} \cdot \varphi )(m)
%		=
%		( \varphi \cdot (- \class{x}) )(m)
%		=
%		\varphi( - \class{x} \cdot m )
%		=
%		- \varphi( \class{x} \cdot m )
%	\]
%	for all~$\varphi \in M^*$,~$m \in M$.
%	Both constructed~\module{$\Univ(\glie)$} structures on~$M^*$ hence coincide.
%\end{remark}



\subsection{Direct Sum of Lie~algebras}

\begin{recall}
	\label{homomorphism out of a tensor product}
	Let~$A$ and~$B$ be two~{\algebras{$\kf$}}.
	Then the inclusion maps
	\begin{alignat*}{2}
		\Iota_A
		&\colon
		A
		\to
		A \tensor B \,,
		&
		\quad
		a
		&\mapsto
		a \tensor 1 \,,
		\\
		\Iota_B
		&\colon
		B
		\to
		A \tensor B \,,
		&
		\quad
		b
		&\mapsto
		1 \tensor b
	\end{alignat*}
	are injective homomorphisms of algebras (unless~$A$ or~$B$ is the zero algebra, in which case~$A \tensor B$ is again the zero algebra)
	We may therefore identify the algebras~$A$ and~$B$ with the associated subalgebras~$A \tensor 1$ and~$1 \tensor B$ of~$A \tensor B$.
	We note that~$A$ and~$B$ commute in~$A \tensor B$ because
	\[
		\Iota_A(a) \Iota_B(b)
		=
		(a \tensor 1) (b \tensor 1)
		=
		a \tensor b
		=
		(b \tensor 1) (a \tensor 1)
		=
		\Iota_B(b) \Iota_A(a)
	\]
	for all~$a \in A$ and~$b \in B$.
	
	Let now~$C$ be another~\algebra{$\kf$}.
	
	If~$\Phi$ is a homomorphism of algebras from~$A \tensor B$ to~$C$, then the restrictions~$\Phi_A$ and~$\Phi_B$ given by~$\Phi_A \defined \Phi \circ \Iota_A$ and~$\Phi_B \defined \Phi \circ \Iota_B$ are again homomorphisms of algebras.
	The images of~$\Phi_A$ and~$\Phi_B$ commute with each other in~$C$ because~$A$ and~$B$ commute in~$A \tensor B$.
	Indeed, we have
	\[
		[ \Phi_A(a), \Phi_B(b) ]
		=
		[ \Phi(\Iota_A(a)), \Phi(\Iota_B(b)) ]
		=
		\Phi( [ \Iota_A(a), \Iota_B(b) ] )
		=
		\Phi( 0 )
		=
		0
	\]
	for all~$a \in A$ and~$b \in B$.
	
	Suppose on the other hand that~$\Psi_A$ is a homomorphism of algebras from~$A$ to~$C$ and that~$\Psi_B$ is a homomorphism of algebras from~$B$ to~$C$.
	There exists a unique linear map~$\Psi$ from~$A \tensor B$ to~$C$ given by
	\[
		\Psi(a \tensor b)
		=
		\Psi_A(a) \Psi_B(b)
	\]
	for all~$a \in A$ and~$b \in B$.
	Suppose that the images of~$\Psi_A$ and~$\Psi_B$ commute with each other in~$C$.
	The linear map~$\Psi$ is then again an homomorphism of algebras because
	\begin{align*}
		\Psi(a_1 \tensor b_1) \Psi(a_2 \tensor b_2)
		&=
		\Psi_A(a_1) \Psi_B(b_1) \Psi_A(a_2) \Psi_B(b_2)
		\\
		&=
		\Psi_A(a_1) \Psi_A(a_2) \Psi_B(b_1) \Psi_B(b_2)
		\\
		&=
		\Psi_A(a_1 a_2) \Psi_B(b_1 b_2)
		\\
		&=
		\Psi( (a_1 a_2) \tensor (b_1 b_2) )
		\\
		&=
		\Psi( (a_1 \tensor b_1) (a_2 \tensor b_2) )
	\end{align*}
	for all~$a_1, a_2 \in A$ and~$b_1, b_2 \in B$, as well as
	\[
		\Psi( 1_{A \tensor B} )
		=
		\Psi( 1_A \tensor 1_B )
		=
		\Psi_A( 1_A ) \Psi_B( 1_B )
		=
		1_C \cdot 1_C
		=
		1_C \,.
	\]
	
	These above two constructions are mutually inverse and hence result in a {\onetoonetext} correspondence
	\begin{align*}
		\SwapAboveDisplaySkip
		\left\{
			\begin{tabular}{c}
				algebra homomorphisms \\
				$\Phi \colon A \tensor B \to C$
			\end{tabular}
		\right\}
		&\onetoone
		\left\{
			\begin{tabular}{c}
				$(\Phi_A, \Phi_B)$
			\end{tabular}
		\suchthat*
			\begin{tabular}{c}
				algebra homomorphisms   \\
				$\Phi_A \colon A \to C$ \\
				$\Phi_B \colon B \to C$ \\
				whose images commute
			\end{tabular}
		\right\}  \,,
		\\
		\Phi
		&\mapsto
		(\Phi \circ \Iota_A, \Phi \circ \Iota_B)  \,,
		\\
		\biggl( a \tensor b \mapsto \Phi_A(a) \Phi_B(b) \biggr)
		&\mapsfrom
		(\Phi_A, \Phi_B)  \,.
	\end{align*}
\end{recall}


\begin{example}
	\label{explicit isomorphism for uea of direct sum}
	Let~$\glie$ and~$\hlie$ be two Lie~algebras.
	We show in the following that
	\[
		\Univ(\glie \oplus \hlie)
		\cong
		\Univ(\glie) \tensor \Univ(\hlie) \,.
		\index{universal eneveloping algebra!of the direct sum of Lie algebras}
	\]
	The isomorphism from~$\Univ(\glie \oplus \hlie)$ to~$\Univ(\glie) \tensor \Univ(\hlie)$ is given on the algebra generators~$\class{(x,y)}$ with~$(x,y)$ in~$\glie \oplus \hlie$ by
	\[
		\class{(x,y)}
		\mapsto
		\class{x} \tensor 1 + 1 \tensor \class{y} \,.
	\]
	The inverse isomorphism from~$\Univ(\glie) \tensor \Univ(\hlie)$ to~$\Univ(\glie \oplus \hlie)$ is given on the simple tensors~$\class{x} \tensor \class{y}$ with~$x$ in~$\Univ(\glie)$ and~$y$ in~$\Univ(\hlie)$ by
	\[
		\class{x} \tensor \class{y}
		\mapsto
		\Univ( \iota_1 )( \class{x} )
		\cdot
		\Univ( \iota_2 )( \class{y} ) \,.
	\]
	Here we denote by~$\iota_1$ is the canonical homomorphism of Lie~algebras from~$\glie$ to~$\glie \oplus \hlie$ (i.e. the inclusion into the first sammand) and similarly by~$\iota_2$ is the canonical homomorphism of Lie~algebras from~$\hlie$ to~$\glie \oplus \hlie$ (i.e. the inclusion into the second summand).

	We present two ways in which the above isomorphism(s) can be derived.
	\begin{itemize}
		\item
			It follows from \cref{homomorphism out of direct sum} and \cref{homomorphism out of a tensor product} that we get for every~\algebra{$\kf$}~$A$ bijections
			\begin{align*}
				{}&
				\left\{
					\begin{tabular}{c}
						algebra homomorphisms \\
						$\Phi \colon \Univ(\glie \oplus \hlie) \to A$
					\end{tabular}
				\right\}
				\\
				\cong{}&
				\left\{
					\begin{tabular}{c}
						Lie~algebra homomorphisms \\
						$\varphi \colon \glie \oplus \hlie \to A$
					\end{tabular}
				\right\}
				\\
				\cong{}&
				\left\{
					\begin{tabular}{c}
						$( \varphi_1, \varphi_2 )$
					\end{tabular}
				\suchthat*
					\begin{tabular}{c}
						Lie~algebra homomorphisms \\
						$\varphi_1 \colon \glie \to A$ and~$\varphi_2 \colon \hlie \to A$ \\
						whose images commute
					\end{tabular}
				\right\}
				\\
				\cong{}&
				\left\{
					\begin{tabular}{c}
						$(\Phi_1, \Phi_2)$
					\end{tabular}
				\suchthat*
					\begin{tabular}{c}
						algebra homomorphisms               \\
						$\Phi_1 \colon \Univ(\glie) \to A$  \\
						$\Phi_2 \colon \Univ(\hlie) \to A$  \\
						whose images commute
					\end{tabular}
				\right\}
				\\
				\cong{}&
				\left\{
					\begin{tabular}{c}
						 algebra homomorphims \\
						 $\Phi \colon \Univ(\glie) \tensor \Univ(\hlie) \to A$
					\end{tabular}
				\right\} \,.
			\end{align*}
			The claimed isomorphism therefore follows from Yoneda’s lemma.
		\item
			We can construct the isomorphism(s) more explicitely, as follows.

			We note that for the induced homomorphisms of algebras
			\begin{align*}
				\Univ(\iota_1) &\colon \Univ(\glie) \to \Univ(\glie \oplus \hlie) \,, \\
				\Univ(\iota_2) &\colon \Univ(\hlie) \to \Univ(\glie \oplus \hlie)
			\end{align*}
			the images of~$\Univ(\iota_1)$ and~$\Univ(\iota_2)$ commute.
			Indeed, the image of~$\Univ(\iota_1)$ in~$\Univ(\glie \oplus \hlie)$ is generated by the image of~$\glie \oplus 0$ in~$\Univ(\glie \oplus \hlie)$ as an algebra, and the image of~$\Univ(\iota_2)$ in~$\Univ(\glie \oplus \hlie)$ is generated by the image of~$0 \oplus \hlie$ in~$\Univ(\glie \oplus \hlie)$.
			But~$\glie \oplus 0$ and~$0 \oplus \hlie$ commute in~$\glie \oplus \hlie$, whence their images in~$\Univ(\glie \oplus \hlie)$ commute with each other.
			Therefore, the images of~$\Univ(\iota_1)$ and~$\Univ(\iota_2)$ commute with each other.

			It follows from this observation that the two homomorphisms of algebras~$\Univ( \iota_1 )$ and~$\Univ( \iota_2 )$ induce a homomorphism of algebras
			\[
				\Phi
				\colon
				\Univ(\glie) \tensor \Univ(\hlie)
				\to
				\Univ(\glie \oplus \hlie) \,.
			\]
			This homomorphism is given on simple tensors by
			\[
				\Phi(t \tensor u)
				=
				\Univ(\iota_1)(t) \cdot \Univ(\iota_2)(u)
			\]
			for all~$t \in \Univ(\glie)$ and~$u \in \Univ(\hlie)$.
			It holds in particular for all~$x$ in~$\glie$ and~$y$ in~$\hlie$ that
			\[
				\Phi(\class{x} \tensor \class{y})
				=
				\Univ(\iota_1)( \class{x} )
				\cdot
				\Univ(\iota_2)( \class{y} )
				=
				\class{ \iota_1(x) }
				\cdot
				\class{ \iota_2(y) }
				=
				\class{(x,0)} \cdot \class{(0,y)}  \,,
			\]
%      We observe that the map
%      \[
%        \psi'
%        \colon
%        \glie \times \hlie
%        \to
%        \Univ(\glie) \tensor \Univ(\hlie) \,,
%        \quad
%        (x,y)
%        \mapsto
%        \class{(x,0)} \tensor 1 + 1 \tensor \class{(0,y)} \,.
%      \]

			To construct the inverse~$\Psi$ of~$\Phi$ we observe that the equality
			\[
				\Psi\Bigl( \class{(x,0)} \Bigr)
				=
				\Psi\Bigl( \class{(x,0)} \cdot 1 \Bigr)
				=
				\Psi\Bigl( \Univ(\iota_1)( \class{x} ) \cdot \Univ(\iota_2)(1) \Bigr)
				=
				\Psi( \Phi( \class{x} \tensor 1 ) )
				=
				\class{x} \tensor 1
			\]
			has to hold for every~$x \in \glie$, and similarly
			
			\[
				\Psi\Bigl( \class{(0,y)} \Bigr)
				=
				1 \tensor \class{y}
			\]
			for every~$y \in \hlie$.
			It then follows that more generally
			\[
				\Psi\Bigl( \class{(x,y)} \Bigr)
				=
				\Psi\Bigl( \class{(x,0)} + \class{(0,y)} \Bigr)
				=
				\class{x} \tensor 1 + 1 \tensor \class{y}
			\]
			for every~$(x,y) \in \glie \oplus \hlie$.
			
			Motivated by these calculations we consider the map
			\[
				\psi
				\colon
				\glie \oplus \hlie
				\to
				\Univ(\glie) \tensor \Univ(\hlie) \,,
				\quad
				(x,y)
				\mapsto
				\class{x} \tensor 1 + 1 \tensor \class{y} \,.
			\]
			This map is a homomorphism of Lie~algebras because it is linear and we have
			\begin{align*}
				{}&
				[\psi((x_1, y_1)), \psi((x_2, y_2))]
				\\
				={}&
				[
					\class{x_1} \tensor 1 + 1 \tensor \class{y_1} \,,
					\class{x_2} \tensor 1 + 1 \tensor \class{y_2}
				]
				\\
				={}&
				\underbrace{ [\class{x_1} \tensor 1, \class{x_2} \tensor 1] }_{= [\class{x_1}, \class{x_2}] \tensor 1}
				+ \underbrace{ [\class{x_1} \tensor 1, 1 \tensor \class{y_2}] }_{=0}
				+ \underbrace{ [1 \tensor \class{y_1} \,, \class{x_2} \tensor 1] }_{=0}
				+ \underbrace{ [1 \tensor \class{y_1}, 1 \tensor \class{y_2}] }_{= 1 \tensor [\class{y_1}, \class{y_2}]}
				\\
				={}&
					[\class{x_1}, \class{x_2}] \tensor 1
				+ 1 \tensor [\class{y_1}, \class{y_2}]
				\\
				={}&
					\class{[x_1, x_2]} \tensor 1
				+ 1 \tensor \class{[y_1, y_2]}  \,.
				\\
				={}&
				\psi\bigl( ( [x_1, x_2], [y_1, y_2] ) \bigr)
				\\
				={}&
				\psi\bigl( [(x_1, y_1), (x_2, y_2)] \bigr)
			\end{align*}
			for all~$(x_1, y_1), (x_2, y_2) \in \glie \oplus \hlie$.
			It hence follows from the universal property of the {\ua}~$\Univ(\glie \oplus \hlie)$ that there exists a unique homomorphism of algebras~$\Psi$ from~$\Univ(\glie \oplus \hlie)$ to~$\Univ(\glie) \tensor \Univ(\hlie)$ that makes the triangular diagram
			\[
				\begin{tikzcd}[row sep = large]
					\glie \oplus \hlie
					\arrow{r}[above]{\psi}
					\arrow{d}[left]{\class{(\ph)}}
					&
					\Univ(\glie) \tensor \Univ(\hlie)
					\\
					\Univ(\glie \oplus \hlie)
					\arrow[dashed]{ur}[below right]{\Psi}
					&
					{}
				\end{tikzcd}
			\]
			commute.
			This homomorphism~$\psi$ is given by
			\[
				\Psi\Bigl( \class{(x,y)} \Bigr)
				=
				\class{x} \tensor 1 + 1 \tensor \class{y} \,.
			\]
			for every~$(x,y) \in \glie \oplus \hlie$.
			
			We now check that the two homomorphisms~$\Phi$ and~$\Psi$ are mutually inverse.
			We have on the one hand
			\begin{align*}
				\Phi\Bigl( \Psi\Bigl( \class{(x,y)} \Bigr) \Bigr)
				&=
				\Phi( \class{x} \tensor 1 + 1 \tensor \class{y} )
				\\
				&=
				\Phi( \class{x} \tensor 1 ) + \Phi( 1 \tensor \class{y} )
				\\
				&=
				\Univ(\iota_1)(\class{x}) \cdot \Univ(\iota_2)(1)
				+ \Univ(\iota_1)(1) \cdot \Univ(\iota_2)(\class{y})
				\\
				&=
				\class{\iota_1(x)} \cdot 1
				+ 1 \cdot \class{\iota_2(y)}
				\\
				&=
				\class{(x,0)} + \class{(0,y)}
				\\
				&=
				\class{(x,y)}
			\end{align*}
			for every~$(x,y) \in \glie \oplus \hlie$, and thus~$\Phi \circ \Psi = \id_{\Univ(\glie \oplus \hlie)}$.
			We have on the other hand
			\begin{align*}
				\Psi( \Phi( \class{x} \tensor \class{y} ) )
				&=
				\Psi( \Univ(\iota_1)(\class{x}) \cdot \Univ(\iota_2)(\class{y}) )
				\\
				&=
				\Psi\Bigl( \class{(x,0)} \cdot \class{(0,y)} \Bigr)
				\\
				&=
				\Psi\Bigl( \class{(x,0)} \Bigr)
				\Psi\Bigl( \class{(0,y)} \Bigr)
				\\
				&=
				( \class{x} \tensor 1 + 1 \tensor 0 )
				\cdot ( 0 \tensor 1 + 1 \tensor \class{y} )
				\\
				&=
				(\class{x} \tensor 1)
				\cdot (1 \tensor \class{y})
				\\
				&=
				\class{x} \tensor \class{y}
			\end{align*}
			for all~$x \in \glie$ and~$y \in \hlie$, and thus~$\Psi \circ \Phi = \id_{\Univ(\glie) \tensor \Univ(\hlie)}$.
	\end{itemize}
\end{example}


\begin{construction}
	\label{construction of comultiplication}
	Let~$\glie$ be a Lie~algebra.
	We have a homomorphism of Lie~algebras
	\[
		\delta
		\colon
		\glie
		\to
		\glie \oplus \glie \,,
		\quad
		x
		\mapsto
		(x,x) \,.
	\]
	This homomorphism of Lie~algebras extends a homomorphism of algebras
	\[
		\Delta'
		\defined
		\Univ(\delta)
		\colon
		\Univ(\glie)
		\to
		\Univ(\glie \oplus \glie) \,.
	\]
	As seen in \cref{explicit isomorphism for uea of direct sum} we have an isomorphism of algebras from~$\Univ(\glie \oplus \glie)$ to~$\Univ(\glie) \tensor \Univ(\glie)$ given by~$\class{(x,y)} \mapsto \class{x} \tensor 1 + 1 \tensor \class{y}$ for all~$x, y \in \glie$.
	Under this isomorphism, we can regard the homomorphism~$\Delta'$ as a homomorphism of algebras
	\[
		\Delta
		\colon
		\Univ(\glie)
		\to
		\Univ(\glie) \tensor \Univ(\glie) \,.
	\]
	This homomorphism is uniquely determined by
	\[
		\Delta( \class{x} )
		=
		\class{x} \tensor 1 + 1 \tensor \class{x}
		\qquad
		\text{for every~$x \in \glie$.}
	\]
\end{construction}


\begin{definition}
	Let~$\glie$ be a Lie~algebra
	The homomorphism of algebras~$\Delta$\glsadd{comultiplication} from \cref{construction of comultiplication}\index{comultiplication} is the \defemph{comultiplication} of~$\glie$.
\end{definition}


\begin{remark}
	Let~$\glie$ be a Lie~algebra and let~$M$ and~$N$ be two representations of~$\glie$.
	We have corresponding~\module{$\Univ(\glie)$} structures on~$M$ and~$N$ that are uniquely determined by
	\[
		\class{x} \cdot m
		=
		x \act m
		\qquad\text{and}\qquad
		\class{x} \cdot n
		=
		x \act n
	\]
	for all~$x \in \glie$,~$m \in M$,~$n \in N$.
	\begin{enumerate}
		\item
			These module structure correspond to homomorphisms of algebras
			\[
				\Phi_M
				\colon
				\Univ(\glie)
				\to
				\End_{\kf}(M) \,,
				\quad
				\Phi_N
				\colon
				\Univ(\glie)
				\to
				\End_{\kf}(N) \,.
			\]
			These homomorphism fit together into a homomorphism of algebras
			\[
				\Phi_M \tensor \Phi_N
				\colon
				\Univ(\glie) \tensor \Univ(\glie)
				\to
				\End_{\kf}(M) \tensor \End_{\kf}(N) \,.
			\]
			We have also a homomorphism of algebras
			\[
				\Psi
				\colon
				\End_{\kf}(M) \tensor \End_{\kf}(N)
				\to
				\End_{\kf}(M \tensor_{\kf} N) \,,
				\quad
				f \tensor g
				\mapsto
				f \tensor g \,.
			\]
			Let~$\Delta$ be the comultiplication of~$\Univ(\glie)$.
			The composite
			\[
				\Phi_{M \tensor N}
				\defined
				\Psi \circ (\Phi_M \tensor \Phi_N) \circ \Delta
				\colon
				\Univ(\glie)
				\to
				\End_{\kf}(M \tensor N) \,.
			\]
			is again a homomorphism of algebras.
			This homomorphism corresponds to a~\module{$\Univ(\glie)$} structure on~$M \tensor_{\kf} N$.
			For every element~$x$ of~$\glie$ we have
			\begin{align*}
				\class{x} \cdot (m \tensor n)
				&=
				\Phi_{M \tensor N}( \class{x} )( m \tensor n )
				\\
				&=
				\Psi( (\Phi_M \tensor \Phi_N)( \Delta(\class{x}) ) )( m \tensor n)
				\\
				&=
				\Psi( (\Phi_M \tensor \Phi_N)( \class{x} \tensor 1 + 1 \tensor \class{x} ) )( m \tensor n)
				\\
				&=
				\Psi( \Phi_M(\class{x}) \tensor \Phi_N(1) + \Phi_M(1) \tensor \Phi_N(\class{x}) )( m \tensor n)
				\\
				&=
				\Psi( \Phi_M(\class{x}) \tensor \id_N + \id_M \tensor \Phi_N(\class{x}) )( m \tensor n)
				\\
				&=
				\Phi_M(\class{x})(m) \tensor \id_N(n) + \id_M(m) \tensor \Phi_N(\class{x})(n)
				\\
				&=
				(\class{x} \cdot m) \tensor n + m \tensor (\class{x} \cdot n)
				\\
				&=
				(x \act m) \tensor n + m \tensor (x \act n)
			\end{align*}
			for all~$m \in M$,~$n \in N$.
			This~\module{$\Univ(\glie)$} structure on~$M \tensor_{\kf} N$ corresponds to an action of~$\glie$ on~$M \otimes_{\kf} N$.
			The above calculation shows that this is precisely the action from \cref{new representations from old ones}\index{tensor product of representations}.
		\item
			The~\module{$\Univ(\glie)$} structure on~$N$ induces a left~\module{$\Univ(\glie)$} structure on~$\Hom_{\kf}(M, N)$ given by
			\[
				(y \cdot f)(m)
				\defined
				y \cdot f(m)
			\]
			for all~$y \in \Univ(\glie)$,~$f \in \Hom_{\kf}(M, N)$,~$m \in M$.
			The~\module{$\Univ(\glie)$} structure on~$M$ induces a right~\module{$\Univ(\glie)$} structure on~$\Hom_{\kf}(M, N)$ given by
			\[
				(f \cdot y)(m)
				\defined
				f(y \cdot m)
			\]
			for all~$y \in \Univ(\glie)$,~$f \in \Hom_{\kf}(M, N)$,~$m \in M$.
			These two module structures commute with each other because
			\[
				((y_1 \cdot f) \cdot y_2)(m)
				=
				(y_1 \cdot f)(y_2 \cdot m)
				=
				y_1 \cdot f(y_2 \cdot m)
				=
				y_1 \cdot (f \cdot y_2)(m)
				=
				(y_1 \cdot (f \cdot y_2))(m)
			\]
			for all~$y_1, y_2 \in \Univ(\glie)$,~$f \in \Hom_{\kf}(M,N)$,~$m \in M$.
			We have thus a~\bimodule{$\Univ(\glie)$}{$\Univ(\glie)$} structure on~$\Hom_{\kf}(M,N)$ given by
			\[
				y_1 \cdot f \cdot y_2
				\defined
				y_1 \cdot (f \cdot y_2)
				=
				(y_1 \cdot f) \cdot y_2
			\]
			for all~$y_1, y_2 \in \Univ(\glie)$ and~$f \in \Hom_{\kf}(M, N)$.
			This~\bimodule{$\Univ(\glie)$}{$\Univ(\glie)$} structure corresponds to a left~\module{$\Univ(\glie) \tensor \Univ(\glie)^{\op}$} structure given by
			\[
				\Bigl( y_1 \tensor y_2^{\op} \Bigr) \cdot f
				\defined
				y_1 \cdot f \cdot y_2
			\]
			for all~$y_1, y_2 \in \Univ(\glie)$ and~$f \in \Hom_{\kf}(M, N)$.
			By using the isomorphism between~$\Univ(\glie^{\op})$ and~$\Univ(\glie)^{\op}$ from \cref{uea of opposite by first principles} we get a~\module{$\Univ(\glie) \tensor \Univ(\glie)$} structure on~$\Hom_{\kf}(M,N)$ given by
			\[
				( \class{x_1} \tensor \class{x_2} ) \cdot f
				\defined
				\class{x_1} \cdot f \cdot (-\class{x_2})
				=
				- \class{x_1} \cdot f \cdot \class{x_2}
			\]
			for all~$x_1, x_2 \in \glie$ and~$f \in \Hom_{\kf}(M,N)$.
			We can pull back this~\module{$\Univ(\glie) \tensor \Univ(\glie)$} structure along the comultiplication~$\Delta$ to get a~\module{$\Univ(\glie)$} structure on~$\Hom_{\kf}(M,N)$ given by
			\[
				\class{x} \cdot f
				\defined
				(\class{x} \tensor 1 + 1 \tensor \class{x}) \cdot f
				=
				\class{x} \cdot f - f \cdot \class{x}
			\]
			for all~$x_1, x_2 \in \glie$ and~$f \in \Hom_{\kf}(M,N)$.
			We have more explicitely
			\[
				(\class{x} \cdot f)(m)
				=
				\class{x} \cdot f(m) - f(\class{x} \cdot m)
			\]
			for all~$x \in \glie$,~$f \in \Hom_{\kf}(M,N)$ and~$m \in M$.
			This~\module{$\Univ(\glie)$} structure on~$\Hom_{\kf}(M,N)$ corresponds to an action of~$\glie$ on~$\Hom_{\kf}(M,N)$ given by
			\begin{equation}
				\label{induced action on homomorphisms made explicit}
				(x \act f)(m)
				=
				x \act f(m) - f(x \act m)
			\end{equation}
			for all~$x \in \glie$,~$f \in \Hom_{\kf}(M,N)$,~$m \in M$.
			We see from the explicit formula~\eqref{induced action on homomorphisms made explicit} that this is precisely the induced action of~$\glie$ on~$\Hom_{\kf}(M,N)$ from \cref{new representations from old ones}\index{Hom-representation@$\Hom$-representation}.
	\end{enumerate}
\end{remark}


\begin{remark}
	Let~$\glie$ be a Lie~algebra.
	The comultiplication~$\Delta$, counit~$\varepsilon$, and antipode~$S$ satisfy certain compatibility conditions.
	More precisely, the comultiplication~$\Delta$ satisfies the following \defemph{coassociativity}\index{coassociative} diagram.
	\[
		\begin{tikzcd}[sep = large]
			\Univ(\glie)
			\arrow{r}[above]{\Delta}
			\arrow{d}[left]{\Delta}
			&
			\Univ(\glie) \tensor \Univ(\glie)
			\arrow{d}[right]{\Delta \tensor \id}
			\\
			\Univ(\glie) \tensor \Univ(\glie)
			\arrow{r}[above]{\id \tensor \Delta}
			&
			\Univ(\glie) \tensor \Univ(\glie) \tensor \Univ(\glie)
		\end{tikzcd}
	\]
	The comultiplication~$\Delta$ and counit~$\varepsilon$ satisfy the folloing \defemph{counital}\index{counital} diagram.
	\[
		\begin{tikzcd}[row sep = large]
			\Univ(\glie) \tensor \Univ(\glie)
			\arrow{d}[left]{\id \tensor \varepsilon}
			&
			\Univ(\glie)
			\arrow{l}[above]{\Delta}
			\arrow{r}[above]{\Delta}
			\arrow[equal]{d}
			&
			\Univ(\glie) \tensor \Univ(\glie)
			\arrow{d}[right]{\varepsilon \tensor \id}
			\\
			\Univ(\glie) \tensor \kf
			\arrow{r}[above]{\cong}
			&
			\Univ(\glie)
			&
			\kf \tensor \Univ(\glie)
			\arrow{l}[above]{\cong}
		\end{tikzcd}
	\]
	The comultiplication~$\Delta$, counit~$\varepsilon$, and antipode~$S$ satisfy the following \defemph{antipode}\index{antipodal} diagram.
	\[
		\begin{tikzcd}[column sep = small, row sep = large]
			{}
			&
			\Univ(\glie) \tensor \Univ(\glie)
			\arrow{rr}{S \tensor \id}
			&
			{}
			&
			\Univ(\glie) \tensor \Univ(\glie)
			\arrow{dr}{\mathrm{mult}}
			&
			{}
			\\
			\Univ(\glie)
			\arrow{ur}[above left]{\Delta}
			\arrow{rr}[above]{\varepsilon}
			\arrow{dr}[below left]{\Delta}
			&
			{}
			&
			\kf
			\arrow{rr}[above]{\mathrm{incl}}
			&
			{}
			&
			\Univ(\glie)
			\\
			{}
			&
			\Univ(\glie) \tensor \Univ(\glie)
			\arrow{rr}[above]{\id \tensor S}
			&
			{}
			&
			\Univ(\glie) \tensor \Univ(\glie)
			\arrow{ur}[below right]{\mathrm{mult}}
			&
			{}
		\end{tikzcd}
	\]
	This means altogether that the algebra~$\Univ(\glie)$ together with the comultiplication~$\Delta$, counit~$\varepsilon$ and antipode~$S$ is a \defemph{Hopf algebra}.
\end{remark}


% TODO: A more detailed discussion of Hopf algebras.




\subsection{Quotient Lie~Algebra}


\begin{example}
	\label{universal enveloping algebra of quotient}
	Let~$\glie$ be a Lie~algebra, let~$I$ be an ideal of~$\glie$ and let~$\ideal{I}$ be the two-sided ideal of~$\Univ(\glie)$ generated by all elements of the form~$\class{x}$ with~$x$ in~$I$.
	Let~$\pi$ be the canonical quotient homomorphism from~$\glie$ to~$\glie/I$.
	Then the induced homomorphism of algebras~$\Univ(\pi)$ from~$\Univ(\glie)$ to~$\Univ(\glie/I)$ factors through~$\Univ(\glie) / \ideal{I}$, and induces an isomorphism
	\[
		\Univ(\glie) / \ideal{I}
		\to
		\Univ(\glie/I) \,,
		\index{universal enveloping algebra!of a quotient Lie algebra}
	\]
	which is is given by
	\[
		\class{ \class{x} }
		\mapsto
		\class{ \class{x} }
	\]
	for all~$x \in \glie$.

	Indeed, we have for every~{\algebra{$\kf$}}~$A$ bijections
	\begin{align*}
		{}&
		\{ \textstyle\text{algebra homomorphisms~$\Psi \colon \Univ(\glie/I) \to A$} \}
		\\
		\cong{}&
		\{ \textstyle\text{Lie~algebra homomorphisms~$\psi \colon \glie/I \to A$} \}
		\\
		\cong{}&
		\{ \textstyle\text{Lie~algebra homomorphisms~$\varphi \colon \glie \to A$ with~$\varphi(x) = 0$ for every~$x \in I$} \}
		\\
		\cong{}&
		\{ \textstyle\text{algebra homomorphisms~$\Phi \colon \Univ(\glie) \to A$ with~$\Phi(\class{x}) = 0$ for every~$x \in I$} \}
		\\
		={}&
		\{ \textstyle\text{algebra homomorphisms~$\Phi \colon \Univ(\glie) \to A$ with~$\Phi(y) = 0$ for every~$y \in \ideal{I}$} \}
		\\
		\cong{}&
		\{ \textstyle\text{algebra homomorphisms~$\Psi \colon \Univ(\glie) / \ideal{I} \to A$} \} \,.
	\end{align*}
	These bijections are natural in~$A$, whence~$\Univ(\glie/I) \cong \Univ(\glie) / \ideal{I}$ by Yoneda’s lemma.
	
	More explicitely, the composite
	\[
		\glie
		\to
		\Univ(\glie)
		\to
		\Univ(\glie) / \ideal{I} \,,
		\quad
		x \mapsto
		\class{ \class{x} }
	\]
	is a homomorphisms of Lie~algebras.
	This homomorphism annihilates the ideal~$I$ of~$\glie$ and thus induces a homomorphism of Lie~algebras
	\[
		\glie/I
		\to
		\Univ(\glie) / \ideal{I} \,,
		\quad
		\class{x}
		\mapsto
		\class{ \class{x} } \,,
	\]
	which in turn induces a homomorphism of algebras given by
	\begin{alignat*}{3}
		\Phi
		&\colon
		\Univ(\glie/I)
		\to
		\Univ(\glie) / \ideal{I}  \,,
		&
		\quad
		\class{ \class{x} }
		&\mapsto
		\class{ \class{x} }
		&
		\qquad
		&\text{for every~$x \in \glie$.}
	\intertext{
	On the other hand, the canonical quotient homomorphism~$\pi$ from~$\glie$ to~$\glie/I$ induces the homomorphism of algebras~$\Univ(\pi)$ from~$\Univ(\glie)$ to~$\Univ(\glie/I)$, given by
	}
		\Univ(\pi)
		&\colon
		\Univ(\glie)
		\to
		\Univ( \glie / I ) \,,
		&
		\quad
		\class{x}
		&\mapsto
		\class{ \class{x} }
		&
		\qquad
		&\text{for every~$x \in \glie$.}
	\intertext{
	This homomorphism annihilates all residue classes~$\class{x}$ with~$x \in I$, and hence induces a homomorphism of algebras given by
	}
		\Psi
		&\colon
		\Univ(\glie) / \ideal{I}
		\to
		\Univ(\glie/I)  \,,
		&
		\quad
		\class{ \class{x} }
		&\mapsto
		\class{ \class{x} }
		&
		\qquad
		&\text{for every~$x \in \glie$.}
	\end{alignat*}
	We have thus constructed two mutually inverse isomorphisms of algebras~$\Phi$ and~$\Psi$.
\end{example}

\subsubsection{Abelianzation of Lie Algebras}

\begin{example}
	Let~$\glie$ be a Lie~algebra.
	It follows from \cref{universal enveloping algebra of quotient} that
	\begin{align*}
		\Univ( \glie^{\ab} )
		&=
		\Univ( \glie/[\glie, \glie] )
		\\
		&\cong
		\Univ(\glie) / \ideal{[\glie, \glie]}
		\\
		&=
		\Univ(\glie)
		/
		\ideal[\Big]{ \class{[x,y]} \suchthat x, y \in \glie }
		\\
		&=
		\Univ(\glie)
		/
		\ideal{
			\class{x} \, \class{y} - \class{y} \, \class{x}
		\suchthat
			x, y \in \glie
		} \,.
	\end{align*}
	The ideal~$\ideal{ \class{x} \, \class{y} - \class{y} \, \class{x} \suchthat x, y \in \glie }$ is the commutator ideal of~$\Univ(\glie)$ because the elements~$\class{x}$ with~$x \in \glie$ form an algebra generating set of~$\Univ(\glie)$.
	The universal enveloping algebra of the abelianization of~$\glie$ is hence the abelianization (commutativization?) of the universal enveloping algebra of~$\glie$.\index{universal enveloping algebra!of the abelianization}
	
	This can also be seen from Yoneda’s lemma since we have for every commutative~{\algebra{$\kf$}}~$A$ bijections
	\begin{align*}
		\SwapAboveDisplaySkip
		{}&
		\{ \textstyle\text{algebra homomorphisms~$\Univ(\glie/[\glie,\glie]) \to A$} \}
		\\
		\cong{}&
		\{ \textstyle\text{Lie~algebra homomorphism~$\glie/[\glie, \glie] \to A$} \}
		\\
		\cong{}&
		\{ \textstyle\text{Lie~algebra homomorphisms~$\glie \to A$} \}
		\\
		\cong{}&
		\{ \textstyle\text{algebra homomorphisms~$\Univ(\glie) \to A$} \}
		\\
		\cong{}&
		\{ \textstyle\text{algebra homomorphisms~$\Univ(\glie)/I \to A$} \}
	\end{align*}
	that are natural in~$A$, where~$I$ denotes the commutator ideal of~$\Univ(\glie)$.
\end{example}





\subsection{Free Lie~Algebra}


\begin{definition}
	\leavevmode
	\begin{enumerate}
		\item
			Let~$X$ be a set.
			A \defemph{free~\liealgebra{$\kf$} on the set~$X$}\index{free Lie algebra!on a set} is a~\liealgebra{$\kf$}~$\freelieset(X)$\glsadd{free lie algebra set} together with a set-theoretic map~$i$ from~$X$ to~$\freelieset(X)$ such that the following universal property holds:
			For every~\liealgebra{$\kf$}~$\glie$ and every map~$f$ from~$X$ to~$\glie$ there exists a unique homomorphism of Lie~algebras~$\varphi$ from~$\freelieset(X)$ to~$\glie$ that makes the following triangular diagram commute.
			\[
				\begin{tikzcd}[sep = large]
					X
					\arrow{dr}[above right]{f}
					\arrow{d}[left]{i}
					&
					{}
					\\
					\freelieset(X)
					\arrow[dashed]{r}[below]{\varphi}
					&
					\glie
				\end{tikzcd}
			\]
		\item
			Let~$V$ be a~\vectorspace{$\kf$}.
			A \defemph{free~\liealgebra{$\kf$} on the vector space~$V$}\index{free Lie algebra!on a vector space} is a~\liealgebra{$\kf$}~$\freelievect(V)$\glsadd{free lie algebra vector space} together with a linear map~$i$ from~$V$ to~$\freelievect$ such that the following universal property holds:
			For every~\liealgebra{$\kf$}~$\glie$ and every linear map~$f$ from~$V$ to~$\glie$ there exists a unique homomorphism of Lie algebras~$\varphi$ from~$\freelievect(V)$ to~$\glie$ that makes the following diagram commute.
			\[
				\begin{tikzcd}[sep = large]
					V
					\arrow{dr}[above right]{f}
					\arrow{d}[left]{i}
					&
					{}
					\\
					\freelievect(V)
					\arrow[dashed]{r}[below]{\varphi}
					&
					\glie
				\end{tikzcd}
			\]
	\end{enumerate}
\end{definition}


\begin{remark}
	\leavevmode
	\begin{enumerate}
		\item
			Let~$X$ be a set.
			A free Lie~algebra on the set~$X$ consists of a Lie~algebra~$\freelieset(X)$ together with an element~$\class{x}$ of~$\freelieset(X)$ for every element~$x$ of~$X$, such that the following universal property holds:
			For every Lie~algebra~$\glie$ and every famile~$(y_x)_{x \in X}$ of elements of~$\glie$ there exists a unique homomorphism of Lie~algebras~$\varphi$ from~$\freelieset(X)$ to~$\glie$ such that~$\varphi( \class{x} ) = y_x$ for every~$x \in X$.
		\item
			The free Lie~algebra on a set~$X$ is unique up to unique isomorphism, in the following sense.

			If~$(L_1, i_1)$ and~$(L_2, i_2)$ are two free Lie~algebras on a set~$X$ then there exist unique homomorphisms of Lie~algebras~$\varphi$ from~$L_1$ to~$L_2$ and~$\psi$ from~$L_2$ to~$L_1$ that make the triangular diagrams
			\[
				\begin{tikzcd}[column sep = small]
					{}
					&
					X
					\arrow{dl}[above left]{\iota_1}
					\arrow{dr}[above right]{\iota_2}
					&
					{}
					\\
					F_1
					\arrow[dashed]{rr}[below]{\varphi}
					&
					{}
					&
					F_2
				\end{tikzcd}
				\qquad\text{and}\qquad
				\begin{tikzcd}[column sep = small]
					{}
					&
					X
					\arrow{dl}[above left]{\iota_2}
					\arrow{dr}[above right]{\iota_1}
					&
					{}
					\\
					F_2
					\arrow[dashed]{rr}[below]{\psi}
					&
					{}
					&
					F_1
				\end{tikzcd}
			\]
			commute.
			These homomorphisms~$\varphi$ and~$\psi$ are mutually inverse isomorphisms of Lie~algebras.

			Because of this uniqueness we will talk about \emph{the} free~\liealgebra{$\kf$} on~$X$.
		\item
			We find in the same way that the free~\liealgebra{$\kf$} on a~\vectorspace{$\kf$} is unique up to unique isomorphism.
		\item
			Let~$X$ and~$Y$ be two sets and let~$f$ be a map from~$X$ to~$Y$.
			There exists a unique homomorphism of Lie~algebras~$\freelieset(f)$\glsadd{free lie algebra set on morphisms}\index{induced homomorphism!on free Lie algebras} from~$\freelieset(X)$ to~$\freelieset(Y)$ that makes the square diagram
			\[
				\begin{tikzcd}[column sep = large]
					X
					\arrow{r}[above]{f}
					\arrow{d}
					&
					Y
					\arrow{d}
					\\
					\freelieset(X)
					\arrow[dashed]{r}[above]{\freelieset(f)}
					&
					\freelieset(Y)
				\end{tikzcd}
			\]
			commute.
			It holds for every set~$X$ that~$\freelieset(\id_X) = \id_{\freelieset(X)}$, and it holds for any two composable maps of sets~$f$ from~$X$ to~$Y$ and~$g$ from~$Y$ to~$Z$ that~$\freelieset(g \circ f) = \freelieset(g) \circ \freelieset(f)$.
			The assignment~$\freelieset$ is thus a functor from the category~$\cSet$ to the category~$\cLie{\kf}$.
			
			The universal property of the free Lie~algebra on a set states that the functor~$\freelieset$ is left adjoint\index{adjunction} to the forgetful functor from~$\cLie{\kf}$ to~$\cSet$, which assigns to each Lie~algebra its underlying set.
		\item
			We find similarly that for any two vector spaces~$V$ and~$W$ and every linear maps~$f$ from~$V$ to~$W$ there exist a unique homomorphism of Lie~algebras~$\freelievect(f)$\glsadd{free lie algebra vector space on morphisms} from~$\freelievect(V)$\index{induced homomorphism!on free Lie algebras} to~$\freelievect(W)$ that makes the square diagram
			\[
				\begin{tikzcd}[column sep = huge]
					V
					\arrow{r}[above]{f}
					\arrow{d}
					&
					W
					\arrow{d}
					\\
					\freelievect(V)
					\arrow[dashed]{r}[above]{\freelievect(f)}
					&
					\freelievect(W)
				\end{tikzcd}
			\]
			commute.
			The assignment~$\freelievect$ is then a functor from~$\cVect{\kf}$ to~$\cLie{\kf}$.
			The universal property of the free Lie~algebra on a vector space states that the functor~$\freelievect$ is left adjoint\index{adjunction} to the forgetful functor from~$\cLie{\kf}$ to~$\cVect{\kf}$, which assigns to each \liealgebra{$\kf$} its underlying~\vectorspace{$\kf$}.
		\item
			For every set~$X$ let~$F(X)$ denote the free~\vectorspace{$\kf$} on the set~$X$.
			The resulting Lie~algebra~$\freelievect(F(X))$ together with the composite
			\[
				X
				\to
				F(X)
				\to
				\freelievect(F(X))
			\]
			is a free~\liealgebra{$\kf$} on the set~$X$.

			Indeed, for every Lie~algebra~$\glie$ and every map~$f$ from~$X$ to~$\glie$ we can first extend~$f$ uniquely to a linear map~$g$ from~$F(X) \to \glie$ such that the following diagram commutes.
			\[
				\begin{tikzcd}[column sep = large]
					X
					\arrow{d}
					\arrow[bend left = 45]{ddr}[above right]{f}
					&
					{}
					\\[0.5em]
					F(X)
					\arrow{d}
					\arrow[dashed, bend left = 20]{dr}[above right]{g}
					&
					{}
					\\[0.5em]
					\freelievect(F(X))
					&
					\glie
				\end{tikzcd}
			\]
			We can then extend~$g$ uniquely to a homomorphism of Lie~algebras~$\psi$ from~$\freelievect(F(X))$ to~$\glie$.
			\[
				\begin{tikzcd}[column sep = large]
					X
					\arrow{d}
					\arrow[bend left = 45]{ddr}[above right]{f}
					&
					{}
					\\[0.5em]
					F(X)
					\arrow{d}
					\arrow[bend left = 20]{dr}[above right]{g}
					&
					{}
					\\[0.5em]
					\freelievect(F(X))
					\arrow[dashed]{r}[above]{\psi}
					&
					\glie
				\end{tikzcd}
			\]

			We have more abstractly for every~\liealgebra{$\kf$}~$\glie$ bijections
			\begin{align*}
				{}&
				\{
					\text{maps~$f \colon X \to \glie$}
				\}
				\\
				\cong{}&
				\{
					\text{linear maps~$g \colon F(X) \to \glie$}
				\}
				\\
				\cong{}&
				\{
					\text{homomorphism of Lie~algebras~$\psi \colon \freelievect(F(X)) \to \glie$}
				\} \,,
			\end{align*}
			which are natural in~$\glie$.
			It therefore follows from Yoneda’s lemma that~$\freelievect(F(X)) \cong \freelieset(X)$.

			We can also understand this isomorphism via adjunctions:
			We have the following commutative diagram of forgetful functors.
			\[
				\begin{tikzcd}
					\cLie{\kf}
					\arrow{d}
					\arrow[bend left = 60]{dd}
					\\
					\cVect{\kf}
					\arrow{d}
					\\
					\cSet
				\end{tikzcd}
			\]
			It follows that the following diagram of left adjoint functors commutes up to isomorphism:
			\[
				\begin{tikzcd}[row sep = large]
					\cLie{\kf}
					\\
					\cVect{\kf}
					\arrow{u}[right]{\freelievect}
					\\
					\cSet
					\arrow{u}[right]{F}
					\arrow[bend left = 60]{uu}[left]{\freelieset}
				\end{tikzcd}
			\]
	\end{enumerate}
\end{remark}


\begin{example}[Free Lie~algebras]
	\label{uea of free lie algebra}
	\leavevmode
	\begin{enumerate}
		\item
			Let~$V$ be a~\vectorspace{$\kf$}.
			For every~\algebra{$\kf$}~$A$ we have bijections
			\begin{align*}
				{}&
				\{
					\text{algebra homomorphisms~$\Phi \colon \Univ(\freelievect(V)) \to A$}
				\}
				\\
				\cong{}&
				\{
					\text{Lie~algebra homomorphisms~$\varphi \colon \freelievect(V) \to A$}
				\}
				\\
				\cong{}&
				\{
					\text{linear maps~$f \colon V \to A$}
				\}
				\\
				\cong{}&
				\{
					\text{algebra homomorphisms~$\Phi \colon \Tensor(V) \to A$}
				\} \,,
			\end{align*}
			which are natural in~$A$.
			It follows from Yoneda’s lemma that we have an isomorphism of algebras
			\[
				\Univ(\freelievect(V)) \cong \Tensor(V) \,.
				\index{universal enveloping algebra!of the free Lie algebra!on a vector space}
			\]

			We can also make this argumentation more explicit:
			For this, let~$i$ be the canonical linear map from~$V$ to its free Lie~algebra~$\freelievect(V)$, and let~$j$ be the canonical homomorphism of Lie~algebras from~$\freelievect(V)$ to its universal enveloping algebra~$\Univ(\freelievect(V))$.
			The composite~$j \circ i$ is a linear map from~$V$ to~$\Univ(\freelievect(V))$ and thus extends to a homomorphism of algebras
			\[
				\Phi
				\colon
				\Tensor(V)
				\to
				\Univ(\freelievect(V))
			\]
			that is given by
			\[
				\Phi(v)
				=
				j(i(v))
				=
				\class{i(v)}
			\]
			for every~$v \in V$.
			We find on the other hand that the inclusion map from~$V$ to~$\Tensor(V)$ is linear and therefore extends to a homomorphism of Lie~algebras
			\[
				\psi
				\colon
				\freelievect(V)
				\to
				\Tensor(V)
			\]
			that is given by
			\[
				\psi(i(v)) = v
			\]
			for every~$v \in V$.
			This homomorphism of Lie~algebras extends by the universal property of the universal enveloping algebra to a homomorphism of algebras
			\[
				\Psi
				\colon
				\Univ(\freelievect(V))
				\to
				\Tensor(V)
			\]
			that is given by
			\[
				\Psi(\class{i(v)})
				=
				\psi(i(v))
				=
				v
			\]
			for every~$v \in V$.
			The composite~$\Psi \circ \Phi$ is a homorphism of algebras from~$\Tensor(V)$ to~$\Tensor(V)$ with
			\[
				(\Psi \circ \Phi)(v)
				=
				\Psi(\Phi(v))
				=
				\Psi\Bigl( \class{i(v)} \Bigr)
				=
				v
			\]
			for every~$v \in V$.
			The tensor algebra~$\Tensor(V)$ is generated by the elements of~$V$, whence find that~$\Psi \circ \Phi = \id_{\Tensor(V)}$.
			The composite~$\Phi \circ \Psi$ on the other hand satisfies the condition
			\[
				(\Phi \circ \Psi)\Bigl(\class{i(v)} \Bigr)
				=
				\Phi\Bigl (\Psi\Bigl( \class{i(v)} \Bigr) \Bigr)
				=
				\Phi(v)
				=
				\class{i(v)}
			\]
			for every~$v \in V$.
			This shows that
			\[
				\Phi \circ \Psi \circ j \circ i
				=
				j \circ i \,.
			\]
			It follows from the universal property of the free Lie~algebra~$\freelievect(V)$ that
			\[
				\Phi \circ \Psi \circ j
				=
				j \,,
			\]
			and thus by the universal property of the universal enveloping algebra finally
			\[
				\Phi \circ \Psi
				=
				\id_{\Univ(\freelievect(V))} \,.
			\]
			We have shown altogether that~$\Phi$ and~$\Psi$ are mutually inverse isomorphisms of algebras.
		\item
			Let~$X$ be a set and let~$F(X)$ be the free vector space on~$X$.
			It follows from our previous discussions that
			\[
				\Univ(\freelieset(X))
				\cong
				\Univ(\freelievect(F(X))
				\cong
				\Tensor(F(X))
				\cong
				\kf\gen{ t_x \suchthat x \in X } \,.
				\index{universal enveloping algebra!of the free Lie algebra!on a set}
			\]
			The composite of canonical maps
			\[
				X
				\to
				\freelieset(X)
				\to
				\Univ(\freelieset(X))
				\cong
				\kf\gen{ t_x \suchthat x \in X }
			\]
			is given by~$x \mapsto t_x$ for every~$x \in X$.
	\end{enumerate}
\end{example}


\begin{remark}
	The Shirshov--Witt~theorem\index{Shirshow--Witt theorem}\index{theorem of!Shirshov--Witt} asserts that Lie~subalgebras of free Lie~algebras are again free.
	We refer to \cite{shirshov_subalgebras_of_free_lie_algebras} for a proof of this theorem.
\end{remark}








% TODO: Hopf algebra structure
%       Explain this also via hopf objects (applying U to a Hopf object in Lie(k) gives a Hopf algebra)





