\section{Definition and Construction of the UEA}





\subsection{Definition and Functoriality}


\begin{definition}
  A \defemph{universal enveloping algebra}\index{universal enveloping algebra} of a Lie~algebra~$\glie$ is a~\algebra{$\kf$}~\gls*{universal enveloping algebra} together with a homomorphism of Lie~algebras~$\iota \colon \glie \to \Univ(\glie)$ such that for every other~{\algebra{$\kf$}}~$A$ and homomorphism of Lie~algebras~$\phi \colon \glie \to A$ there exists a unique homomorphism of~\algebras{$\kf$}~$\Phi \colon \Univ(\glie) \to A$ that makes the triangular diagram
  \[
    \begin{tikzcd}
      \glie
      \arrow{r}[above]{\phi}
      \arrow{d}[left]{\iota}
      &
      A
      \\
      \Univ(\glie)
      \arrow[dashed]{ur}[below right]{\Phi}
      &
      {}
    \end{tikzcd}
  \]
  commute, i.e.\ such that~$\phi = \Phi \circ \iota$.
\end{definition}


\begin{remark}[Uniqueness of universal enveloping algebras]
  \label{uniqueness of universal enveloping algebras}
  Suppose that a Lie~algebra~$\glie$ admits two {\uas}~$(\Univ(\glie)_1, \iota_1)$ and~$(\Univ(\glie)_2, \iota_2)$.
  Then there exists unique algebra homomorphisms~$\phi \colon \Univ(\glie)_1 \to \Univ(\glie)_2$ and~$\psi \colon \Univ(\glie)_2 \to \Univ(\glie)_1$ that make the triangular diagrams
  \[
    \begin{tikzcd}[column sep = small]
      {}
      &
      \glie
      \arrow{dl}[above left]{\iota_1}
      \arrow{dr}[above right]{\iota_2}
      &
      {}
      \\
      \Univ(\glie)_1
      \arrow[dashed]{rr}[below]{\phi}
      &
      {}
      &
      \Univ(\glie)_2
    \end{tikzcd}
    \qquad\text{and}\qquad
    \begin{tikzcd}[column sep = small]
      {}
      &
      \glie
      \arrow{dl}[above left]{\iota_1}
      \arrow{dr}[above right]{\iota_2}
      &
      {}
      \\
      \Univ(\glie)_2
      \arrow[dashed]{rr}[below]{\phi}
      &
      {}
      &
      \Univ(\glie)_1
    \end{tikzcd}
  \]
  commute.
  It follows that the compositions~$\phi \circ \psi \colon \Univ(\glie)_1 \to \Univ(\glie)_1$ and~$\psi \circ \phi \colon \Univ(\glie)_2 \to \Univ(\glie)_2$ make the triangular diagrams
  \[
    \begin{tikzcd}[column sep = small]
      {}
      &
      \glie
      \arrow{dl}[above left]{\iota_1}
      \arrow{dr}[above right]{\iota_1}
      &
      {}
      \\
      \Univ(\glie)_1
      \arrow[dashed]{rr}[below]{\psi \circ \phi}
      &
      {}
      &
      \Univ(\glie)_1
    \end{tikzcd}
    \qquad\text{and}\qquad
    \begin{tikzcd}[column sep = small]
      {}
      &
      \glie
      \arrow{dl}[above left]{\iota_2}
      \arrow{dr}[above right]{\iota_2}
      &
      {}
      \\
      \Univ(\glie)_2
      \arrow[dashed]{rr}[below]{\phi \circ \psi}
      &
      {}
      &
      \Univ(\glie)_2
    \end{tikzcd}
  \]
  commutes.
  The algebra homomorphisms~$\phi \circ \psi$ and~$\psi \circ \phi$ are unique with this property by the universal property of the {\uas}~$(\Univ(\glie)_1, \iota_1)$ and~$(\Univ(\glie)_2, \iota_2)$.
  But the identities~$\id_{\Univ(\glie)_1}$ and~$\id_{\Univ(\glie)_2}$ also makes these diagrams commute.
  We thus find that~$\psi \circ \phi = \id_{\Univ(\glie)_1}$ and~$\phi \circ \psi = \id_{\Univ(\glie)_2}$, so that~$\phi$ and~$\psi$ are mutually inverse isomorphisms of~{\algebras{$\kf$}}.
  
  This shows that the {\ua} (if it exists) is \enquote{unique up to unique isomorphisms}.
  We will therefore talk about \emph{the} {\ua} of~$\glie$.
  We will often also surpress the algebra homorphism~$\iota \colon \glie \to \Univ(\glie)$ from our notation.
\end{remark}


% \begin{remark}
%   One can also formulate the above argument is a more categorical way:
%   Consider the category~$\catC$ where
%   \begin{itemize}
%     \item
%       objects of~$\catC$ is a pairs~$(A, i)$ consisting of a~{\algebra{$\kf$}}~$A$ and a Lie~algebra homomorphism~$i \colon \glie \to A$,
%     \item
%       a morphism~$\phi \colon (A, i) \to (B, j)$ is an algebra homomorphism~$\phi \colon A \to B$ that makes the triangular diagram
%       \[
%         \begin{tikzcd}[column sep = small]
%         {}
%         &
%         \glie
%         \arrow{dl}[above left]{i}
%         \arrow{dr}[above right]{j}
%         &
%         {}
%         \\
%         A
%         \arrow[dashed]{rr}[below]{\phi}
%         &
%         {}
%         &
%         B
%       \end{tikzcd}
%     \]
%       commute, and
%     \item
%       the composition of two morphisms is just their usual set-theoretic composition.
%   \end{itemize}
%   A {\ua} of~$\glie$ is nothing but an inital object in this category~$\catC$.
%   The argumentation from \cref{uniqueness of universal enveloping algebras} is then the usual argument for the uniqueness of inital objects up to unique isomorphism.
% \end{remark}


\begin{proposition}
  \label{representations are modules}
  Let~$V$ be a~{\vectorspace{$\kf$}}.
  Let~$\glie$ be a Lie~algebra and let~$\iota \colon \glie \to \Univ(\glie)$ be the canonical Lie~algebra homomorphism.
  Then the assignments
  \begin{align*}
    \left\{
    \begin{tabular}{@{}c@{}}
      representations of~$\glie$, \\
      $\rho \colon \glie \to \gllie(V)$
    \end{tabular}
    \right\}
    &\longonetoone
    \left\{
    \begin{tabular}{@{}c@{}}
      $\Univ(\glie)$-module structures \\
      $\theta \colon \Univ(\glie) \to \End_{\kf}(V)$
    \end{tabular}
    \right\}  \,,
    \\
    \rho
    &\longmapsto
    \hat{\rho} \,,
    \\
    \theta \circ \iota
    &\longmapsfrom
    \theta  \,,
  \end{align*}
  constitute a {\onetoone} correspondence,~where $\hat{\rho} \colon \Univ(\glie) \to \End_{\kf}(V)$ is the unique~\algebra{$\kf$} homomorphism induced by the homomorphism of Lie~algebras~$\rho \colon \glie \to \gllie(V)$ via the universal property of the~{\ua}~$\Univ(\glie)$.
  \qed
\end{proposition}


\begin{remark}
  \Cref{representations are modules} shows that representations of~$\glie$ are the same as~{\modules{$\Univ(\glie)$}}.
  The categories~$\cRep{\glie}$ and~$\cMod{\Univ(\glie)}$ are hence isomorphic.
\end{remark}


\begin{lemma}[Functoriality of the universal enveloping algebra]
  \label{functoriality of universal enveloping algebra}
  Let~$\glie$,~$\hlie$ and~$\klie$ be Lie~algebras.
  \begin{enumerate}
    \item
      For every homomorphism of Lie~algebras~$\phi \colon \glie \to \hlie$ there exists a unique induced homomorphism of~\algebras{$\kf$}~$\phi^* \colon \Univ(\glie) \to \Univ(\hlie)$ that makes the following square diagram commute:
      \[
        \begin{tikzcd}[column sep = large]
          \glie
          \arrow{r}[above]{\phi}
          \arrow{d}
          &
          \hlie
          \arrow{d}
          \\
          \Univ(\glie)
          \arrow[dashed]{r}[below]{\phi_*}
          &
          \Univ(\hlie)
        \end{tikzcd}
      \]
    \item
      It holds that~$(\id_{\glie})_* = \id_{\Univ(\glie)}$.
    \item
      It holds for all composable homomorphisms of Lie~algebras~$\phi \colon \glie \to \hlie$ and~$\psi \colon \hlie \to \klie$ that
      \[
        (\psi \circ \phi)_*
        =
        \psi_* \circ \phi_* \,.
      \]
  \end{enumerate}
\end{lemma}


\begin{proof}
  \leavevmode
  \begin{enumerate}
    \item
      This follows from the universal property of the universal enveloping algebra~$\Univ(\glie)$ by applying it to the composition~$\glie \to \hlie \to \Univ(\hlie)$.
    \item
      The square diagram
      \[
        \begin{tikzcd}[column sep = huge]
          \glie
          \arrow{r}[above]{\id_{\glie}}
          \arrow{d}
          &
          \glie
          \arrow{d}
          \\
          \Univ(\glie)
          \arrow[dashed]{r}[below]{\id_{\Univ(\glie)}}
          &
          \Univ(\glie)
        \end{tikzcd}
      \]
      commutes, which shows that~$\id_{\Univ(\glie)}$ satisfies the defining property of the induced algebra homomorphism~$(\id_{\glie})_*$.
    \item
      We have the following commutative diagram:
      \[
        \begin{tikzcd}[column sep = large]
          \glie
          \arrow[dashed, bend left = 40]{rr}[above]{\psi \circ \phi}
          \arrow{r}[above]{\phi}
          \arrow{d}
          &
          \hlie
          \arrow{r}[above]{\psi}
          \arrow{d}
          &
          \klie
          \arrow{d}
          \\
          \Univ(\glie)
          \arrow{r}[below]{\phi_*}
          \arrow[dashed, bend right = 40]{rr}[below]{\psi_* \circ \phi_*}
          &
          \Univ(\hlie)
          \arrow{r}[below]{\psi_*}
          &
          \Univ(\klie)
        \end{tikzcd}
      \]
      The commutativity of the outer square diagram
      \[
        \begin{tikzcd}[column sep = huge]
          \glie
          \arrow{r}[above]{\psi \circ \phi}
          \arrow{d}
          &
          \klie
          \arrow{d}
          \\
          \Univ(\glie)
          \arrow[dashed]{r}[below]{\psi_* \circ \phi_*}
          &
          \Univ(\klie)
        \end{tikzcd}
      \]
      shows that~$\psi_* \circ \phi_*$ satisfies the defining property of the induced algebra homomorphism~$(\psi \circ \phi)_*$.
    \qedhere
  \end{enumerate}
\end{proof}





\subsection{Construction of the Universal Enveloping Algebra}


\begin{remark}
  \Cref{functoriality of universal enveloping algebra} shows that the assignment~$\glie \mapsto \Univ(\glie)$ of a Lie~algebra~$\glie$ to its universal eveloping algebra~$\Univ(\glie)$ can be extended to a (covariant) functor~$\Univ \colon \cLie{\kf} \to \cAlg{\kf}$.
  The universal property of the {\ua} states that the functor~$\Univ$ is left adjoint to the forgetful functor~$\cAlg{\kf} \to \cLie{\kf}$ that assigns to each~{\algebra{$\kf$}} its underlying Lie~algebra.
\end{remark}


\begin{remark}
  Let~$\glie$ be a Lie~algebra.
  It follows from the universal propery of the {\ua} that~$\Univ(\glie)$ is generated by the image of the canonical homomorphism~$\iota \colon \glie \to \Univ(\glie)$:
  
  Indeed, let~$U$ be the subalgebra of~$\Univ(\glie)$ that is generated by the image of~$\iota$, and let~$i \colon \glie \to U$ be the restriction of~$\iota$.
  Then for every~{\algebra{$\kf$}}~$A$ and every Lie~algebra homomorphism~$\phi \colon \glie \to A$ the induced algebra homomorphism~$\Phi' \colon \Univ(\glie) \to A$ restricts to an algebra homomorphism~$\Phi \colon U \to A$ that makes the triangular diagram
  \[
    \begin{tikzcd}
      \glie
      \arrow{r}[above]{\phi}
      \arrow{d}[left]{j}
      &
      A
      \\
      U
      \arrow[dashed]{ur}[below right]{\Phi}
      &
      {}
    \end{tikzcd}
  \]
  commute.
  The homomorphism~$\Phi$ is unique with this property because it is uniquely determined by the restriction~$\Phi \circ j = \phi$, since~$U$ is generated by the image of~$j$.
  This shows that~$(U, j)$ is again a {\ua} for~$\glie$.
  
  It follows from the uniqueness of the {\ua}, as discussed in \cref{uniqueness of universal enveloping algebras}, that the unique algebra homomorphism~$i \colon U \to \Univ(\glie)$ that makes the triangular diagram
  \[
    \begin{tikzcd}[column sep = small]
      {}
      &
      \glie
      \arrow{dl}[above left]{j}
      \arrow{dr}[above right]{\iota}
      &
      {}
      \\
      U
      \arrow[dashed]{rr}[below]{i}
      &
      {}
      &
      \Univ(\glie)
    \end{tikzcd}
  \]
  commute is already an isomorphism.
  This homomorphism is necessarily the inclusion of~$U$ into~$\Univ(\glie)$, and that it is an isomorphism means that already~$U = \Univ(\glie)$.
  
  The canonical homomorphism~$\iota \colon \glie \to \Univ(\glie)$ is in particular~{\linear{$\kf$}} and hence induces an algebra homomorphism~$\phi \colon \Tensor(\glie) \to \Univ(\glie)$ that makes the triangular diagram
  \[
    \begin{tikzcd}[column sep = small]
      {}
      &
      \glie
      \arrow{dl}
      \arrow{dr}[above right]{\iota}
      &
      {}
      \\
      \Tensor(\glie)
      \arrow[dashed]{rr}[below]{\phi}
      &
      {}
      &
      \Univ(\glie)
    \end{tikzcd}
  \]
  commute.
  That~$\Univ(\glie)$ is generated by the image of~$\iota$ means that~$\phi$ is surjective.
  Therefore~$\phi$ induces an algebra isomorphism
  \[
    \Phi
    \colon
    \Tensor(\glie)/I
    \to
    \Univ(\glie)
  \]
  for~$I = \ker \phi$, that makes the resulting diagram
   \[
    \begin{tikzcd}[column sep = small]
      {}
      &
      \glie
      \arrow[bend right]{dl}
      \arrow[bend left]{ddr}[above right]{\iota}
      &
      {}
      \\
      \Tensor(\glie)
      \arrow[bend left]{drr}[below left]{\phi}
      \arrow{d}
      &
      {}
      &
      {}
      \\
      \Tensor(\glie)/I
      \arrow[dashed]{rr}[below]{\Phi}
      &
      {}
      &
      \Univ(\glie)
    \end{tikzcd}
  \]
  commute.
  
  If~$A$ is any other~{\algebra{$\kf$}} then we know on the one hand that algebra homorphisms~$\Univ(\glie) \to A$ correspond to Lie~algebra homomorphisms~$\glie \to A$.
  We know find on the other hand that~{\algebra{$\kf$}} homomorphisms~$\Tensor(V)/I \to A$ correspond to algebra homomorphisms~$\Tensor(V) \to A$ that annihilate~$I$, with the algebra homorphisms~$\Tensor(V) \to A$ corresponding to~{\linear{$\kf$}} maps~$\glie \to A$.
  A linear map~$f \colon \glie \to A$ is a homorphism of Lie~algebras if and only if
  \[
      f(x)f(y)
    - f(y)f(x)
    - f([x,y])
    =
    0 \,,
  \]
  so it seems reasonable to assume that the corresponding ideal~$I$ of~$\Tensor(V)$ ought to be given by
  \[
    I
    =
    (x \tensor y - y \tensor x - [x,y] \suchthat x, y \in \glie)
  \]
  We will now show that this is indeed the case.
\end{remark}


\begin{proposition}[Existence of the universal enveloping algebra]
  Let~$\glie$ be a Lie~algebra.
  Let~$\Tensor(\glie)$ be the tensor algebra of the underlying vector space of~$\glie$ and let~$I$ the two-sided ideal in~$\Tensor(\glie)$ generated by the elements $x \tensor y - y \tensor x - [x,y]$ with~$x,y \in \glie$.
  The the quotient algebra~$U \defined T(\glie)/I$ together with the~{\linear{$\kf$}} map
  \[
    i
    \colon
    \glie
    \to
    \Univ(\glie) \,,
    \quad
    x
    \mapsto
    \class{x}
  \]
  is a {\ua} for~$\glie$.
\end{proposition}


\begin{proof}
  The map~$i$ is~{\linear{$\kf$}} and it compatible with the Lie brackets because
  \[
    [i(x), i(y)]
    =
    [\class{x}, \class{y}]
    =
    \class{x} \, \class{y} - \class{y} \, \class{x}
    =
    \class{x \tensor y - y \tensor x}
    =
    \class{[x,y]}
    =
    i([x,y]) \,.
  \]
  Given any~{\algebra{$\kf$}}~$A$ and Lie algebra homomorphism~$\phi \colon \glie \to A$ there exists a unique homorphism of~{\algebras{$\kf$}}~$\Phi' \colon \Tensor(\glie) \to A$ that makes the triangular diagram
  \[
    \begin{tikzcd}
      \glie
      \arrow{r}[above]{\phi}
      \arrow{d}[left]{\iota}
      &
      A
      \\
      \Tensor(V)
      \arrow[dashed]{ur}[below right]{\Phi'}
      &
      {}
    \end{tikzcd}
  \]
  commute, where~$\iota \colon \glie \to \Tensor(\glie)$ is the inclusion.
  The homomorphism~$\Phi'$ is given by~$\Phi'(x) = \phi(x)$ for every~$x \in \glie$.
  It follows that
  \begin{align*}
    \Phi'(x \tensor y - y \tensor x)
    &=
    \Phi'(x \tensor y) - \Phi'(y \tensor x)
    \\
    &=
    \Phi'(x) \Phi'(y) - \Phi'(y) \Phi'(x)
    \\
    &=
    \phi(x) \phi(y) - \phi(y) \phi(x)
    \\
    &=
    [\phi(x), \phi(y)]
    \\
    &=
    \phi([x,y])
    \\
    &=
    \Phi'([x,y])
  \end{align*}
  for all~$x, y \in \glie$, so that the ideal~$I$ is contained in the kernel of~$\Phi'$.
  It follows that there exists a unique algebra homomorphism~$\Phi \colon \Tensor(\glie)/I \to A$ that makes the triangular diagram
  \[
    \begin{tikzcd}
      \glie
      \arrow{r}[above]{\phi}
      \arrow{d}[left]{\iota}
      \arrow[bend right = 55]{dd}[left]{i}
      &
      A
      \\
      \Tensor(\glie)
      \arrow[bend right= 20]{ur}[above left]{\Phi'}
      \arrow{d}[left]{\pi}
      &
      {}
      \\
      \Tensor(\glie)/I
      \arrow[dashed, bend right = 30]{uur}[below right]{\Phi}
      &
      {}
    \end{tikzcd}
  \]
  commute, where~$\pi \colon \Tensor(V) \to \Tensor(V)/I$ denotes the canonical projection.
  Then the subdiagram
  \[
    \begin{tikzcd}
      \glie
      \arrow{r}[above]{\phi}
      \arrow{d}[left]{i}
      &
      A
      \\
      \Tensor(\glie)/I
      \arrow{ur}[below right]{\Phi}
      &
      {}
    \end{tikzcd}
  \]
  commutes.
  That~$\Phi$ is unique with this property follows from the uniqueness of~$\Phi'$.
\end{proof}


\begin{remark}
  The above proof may be reorganized by observing that we have bijections
  \begin{align*}
    {}&
    \{ \text{algebra homomorphisms~$\Phi \colon \Tensor(\glie)/I \to A$} \}
    \\
    \cong{}&
    \{ \text{algebra homomorphisms~$\Phi' \colon \Tensor(\glie) \to A$ with~$\Phi'(I) = 0$} \}
    \\
    \cong{}&
    \left\{
      \begin{tabular}{@{}c@{}}
        algebra homomorphisms~$\Phi' \colon \Tensor(\glie) \to A$ with  \\
        $\Phi'(x \tensor y - y \tensor x - [x,y]) = 0$ for all~$x, y \in \glie$
      \end{tabular}
    \right\}
    \\
    \cong{}&
    \left\{
      \begin{tabular}{@{}c@{}}
        algebra homomorphisms~$\Phi' \colon \Tensor(\glie) \to A$ with  \\
        $\Phi'(x) \Phi'(y) - \Phi'(y) \Phi'(x) - \Phi'([x,y]) = 0$ for all~$x, y \in \glie$
      \end{tabular}
    \right\}
    \\
    \cong{}&
    \left\{
      \begin{tabular}{@{}c@{}}
        algebra homomorphisms~$\Phi' \colon \Tensor(\glie) \to A$ with  \\
        $\Phi'(x) \Phi'(y) - \Phi'(y) \Phi'(x) = \Phi'([x,y])$ for all~$x, y \in \glie$
      \end{tabular}
    \right\}
    \\
    \cong{}&
    \left\{
      \begin{tabular}{@{}c@{}}
        {\linear{$\kf$}} maps~$\phi \colon \glie \to A$ with  \\
        $\phi(x) \phi(y) - \phi(y) \phi(x) = \phi([x,y])$ for all~$x, y \in \glie$
      \end{tabular}
    \right\}
    \\
    \cong{}&
    \left\{
      \begin{tabular}{@{}c@{}}
        {\linear{$\kf$}} maps~$\phi \colon \glie \to A$ with  \\
        $[\phi(x), \phi(y)] = \phi([x,y])$ for all~$x, y \in \glie$ 
      \end{tabular}
    \right\}
    \\
    ={}&
    \{ \text{Lie~algebra homomorphisms~$\phi \colon \glie \to A$} \}
  \end{align*}
  that are natural in~$A$.
  This shows that the~{\algebra{$\kf$}}~$\Tensor(\glie)/I$ represented the right kind of functor;
  and the identity~$\Tensor(\glie)/I \to \Tensor(\glie)/I$ corresponds under the above bijections to the map~$\iota \colon \glie \to \Tensor(\glie)/I$ as desired.
\end{remark}





\subsection{Examples}



\subsubsection{Abelian Lie~Algebras}


\begin{examples}
  Let~$\glie$ be an abelian Lie~algebra.
  It follows from the explicit construction of the universal enveloping algebra~$\Univ(\glie)$ that
  \[
    \Univ(\glie)
    \cong
    \Tensor(\glie)/(x \tensor y - y \tensor x \suchthat x, y \in \glie)
    \cong
    \Symm(\glie)
  \]
  with the canonical Lie~algebra homomorphism~$\glie \to \Univ(\glie)$ corresponding to the inclusion~$\glie \to \Symm(\glie)$.
  This can also be seen more abstractly:
  
  We observe that if~$V$ is a vector space and~$A$ is an algebra then a linear map~$f \colon V \to A$ extends to an algebra homomorphism~$\Symm(V) \to A$ (necessarily uniquely) if and only if the image of~$f$ is contained in a commutative subalgebra of~$A$, if and only if the image of~$f$ is commutative.
  It hence follows that for any~{\algebra{$\kf$}} we have bijections
  \begin{align*}
    {}&
    \{ \text{Lie~algebra homomorphisms~$\glie \to A$} \}
    \\
    \cong{}&
    \{ \text{{\linear{$\kf$}} maps~$\glie \to A$ with commutative image} \}
    \\
    \cong{}&
    \{ \text{algebra homomorphisms~$\Symm(\glie) \to A$} \}
  \end{align*}
  that are natural in~$A$.
  This shows that the symmetric algebra~$\Symm(\glie)$ together with the inclusion~$\glie \to \Symm(\glie)$ satisfies the universal property of the universal enveloping algebra of~$\glie$.
\end{examples}



\subsubsection{Products of Lie~Algebras}


\begin{recall}
  \label{homomorphism out of a tensor product}
  Let~$A$ and~$B$ be two~{\algebras{$\kf$}}.
  Then the inclusions~$i \colon A \to A \tensor B$ and~$j \colon B \to A \tensor B$ are injective algebra homomorphisms.
  We may therefore identify~$A$ and~$B$ with the subalgebras~$A \tensor 1$ and~$1 \tensor B$ of~$A \tensor B$.
  Note that~$A$ and~$B$ commute in~$A \tensor B$ because
  \[
    i(a) j(b)
    =
    (a \tensor 1) (b \tensor 1)
    =
    a \tensor b
    =
    (b \tensor 1) (a \tensor 1)
    =
    j(b) i(a)
  \]
  for all~$a \in A$ and~$b \in B$.
  
  Let~$C$ be another~{\algebra{$\kf$}}.
  
  If~$f \colon A \tensor B \to C$ is an algebra homomorphism then the restrictions~$\phi = f \circ i$ and~$\psi = f \circ j$ are again algebra homomorphisms~$\phi \colon A \to C$ and~$\psi \colon B \to C$.
  The images of~$\phi$ and~$\psi$ in~$C$ commute with each other because~$A$ and~$B$ commute in~$A \tensor B$.
  More explicitely,
  \begin{align*}
    \phi(a) \psi(b)
    &=
    f(a \tensor 1) f(1 \tensor b)
    \\
    &=
    f((a \tensor 1) (1 \tensor b))
    \\
    &=
    f(a \tensor b)
    \\
    &=
    f((1 \tensor b) (a \tensor 1))
    \\
    &=
    f(1 \tensor b) f(a \tensor 1)
    \\
    &=
    \psi(b) \phi(a)
  \end{align*}
  for all~$a \in A$ and~$b \in B$.
  
  If on the other hand~$\phi \colon A \to C$ and~$\psi \colon B \to C$ are two algebra homomorphisms whose images commute with each other then the map
  \[
    f'
    \colon
    A \times B
    \to
    C \,,
    \quad
    (a,b)
    \mapsto
    \phi(a) \psi(b)
  \]
  is~{\bilinear{$\kf$}} and hence induces a~{\linear{$\kf$}} map
  \[
    f
    \colon
    A \tensor B
    \to
    C \,,
    \quad
    a \tensor b
    \mapsto
    \phi(a) \psi(b) \,.
  \]
  The map~$f$ is again an algebra homomorphism because
  \begin{align*}
    f(a_1 \tensor b_1) f(a_2 \tensor b_2)
    &=
    \phi(a_1) \psi(b_1) \phi(a_2) \psi(b_2)
    \\
    &=
    \phi(a_1) \phi(a_2) \psi(b_1) \psi(b_2)
    \\
    &=
    \phi(a_1 a_2) \psi(b_1 b_2)
    \\
    &=
    f( (a_1 a_2) \tensor (b_1 b_2) )
    \\
    &=
    f( (a_1 \tensor b_1) (a_2 \tensor b_2) )
  \end{align*}
  for all simple tensors~$a \tensor b \in A \tensor B$.
  
  These constructions are mutually inverse and hence result in a {\onetoone} correspondence
  \begin{align*}
    \left\{
      \begin{tabular}{@{}c@{}}
        algebra homomorphisms \\
        $f \colon A \tensor B \to C$
      \end{tabular}
    \right\}
    &\longonetoone
    \left\{
      (\phi, \psi)
    \suchthat*
      \begin{tabular}{@{}c@{}}
        algebra homomorphisms \\
        $\phi \colon A \to C$ and~$\psi \colon B \to C$ \\
        whose images commute with each other
      \end{tabular}
    \right\}  \,,
    \\
    f
    &\longmapsto
    (f \circ i, f \circ j)  \,,
    \\
    \biggl( a \tensor b \mapsto \phi(a)\psi(b) \biggr)
    &\longmapsfrom
    (\phi, \psi)  \,.
  \end{align*}
\end{recall}


\begin{example}
  If~$\glie$ and~$\hlie$ be two Lie~algebras then
  \[
    \Univ(\glie \times \hlie)
    \cong
    \Univ(\glie) \tensor \Univ(\hlie) \,.
  \]
  This can be seen in various ways:
  \begin{itemize}
    \item
      It follows from \cref{homomorphism out of a product} and \cref{homomorphism out of a tensor product} that we get for every~{\algebra{$\kf$}}~$A$ bijections
      \begin{align*}
        {}&
        \{ \text{algebra homomorphisms~$F \colon \Univ(\glie \times \hlie) \to A$} \}
        \\
        \cong{}&
        \{ \text{Lie~algebra homomorphisms~$f \colon \glie \times \hlie \to A$} \}
        \\
        \cong{}&
        \left\{
          \begin{tabular}{@{}c@{}}
            Lie~algebra homomorphisms \\
            $\phi \colon \glie \to A$ and~$\psi \colon \hlie \to A$ \\
            whose images commute with each other
          \end{tabular}
        \right\}
        \\
        \cong{}&
        \left\{
          \begin{tabular}{@{}c@{}}
            algebra homomorphisms \\
            $\Phi \colon \Univ(\glie) \to A$ and~$\Psi \colon \Univ(\hlie) \to A$ \\
            whose images commute with each other
          \end{tabular}
        \right\}
        \\
        \cong{}&
        \{ \text{algebra homomorphims~$F \colon \Univ(\glie) \tensor \Univ(\hlie) \to A$} \}  \,.
      \end{align*}
      The claimed isomorphism therefore follows from Yoneda’s lemma.
    \item
      More explicitely let~$i \colon \glie \to \glie \times \hlie$ and~$j \colon \hlie \to \glie \times \hlie$ be the canonical inclusions.
      Then for the induced algebra homomorphisms~$i_* \colon \Univ(\glie) \to \Univ(\glie \times \hlie)$ and~$j_* \colon \Univ(\hlie) \to \Univ(\glie \times \hlie)$ the images of~$i_*$ of~$j_*$ commute with each other.
      Indeed, the algebra~$\Univ(\glie \times \hlie)$ is generated by the image of~$\glie \times \hlie$ in~$\Univ(\glie \times \hlie)$, and~$i(\glie)$ and~$j(\hlie)$ commute in~$\glie \times \hlie$.
      Therefore
      \[
        i_*(\class{x}) j_*(\class{y})
        =
        \class{(x,0)} \class{(0,y)}
        =
        \class{(0,y)} \class{(x,0)}
        =
        j_*(\class{y}) i_*(\class{x}) \,,
      \]
      for all~$x \in \glie$ and~$y \in \hlie$, where we denote by~$\class{(-)}$ the corresponding elements of the {\uas}.
      We used for the middle equality that~$(x,0)$ and~$(0,y)$ commute in~$\glie \times \hlie$ and hence also in~$\Univ(\glie \times \hlie)$.
      
      It follows that~$i_*$ and~$j_*$ induce a common algebra homomorphism
      \[
        \phi
        \colon
        \Univ(\glie) \tensor \Univ(\hlie)
        \to
        \Univ(\glie \times \hlie) \,,
      \]
      that is given on elements by
      \[
        \phi(x \tensor y)
        =
        i_*(x) j_*(y)
      \]
      for all simple tensors~$x \tensor y \in \Univ(\glie) \tensor \Univ(\hlie)$.
      It holds in particular for all~$x \in \glie$ and~$y \in \glie$ that
      \[
        \phi(\class{x} \tensor \class{y})
        =
        i_*(\class{x}) j_*(\class{y})
        =
        \class{i(x)} \cdot \class{j(y)}
        =
        \class{(x,0)} \cdot \class{(0,y)}  \,,
      \]
      We observe that the map
      \[
        \psi'
        \colon
        \glie \times \hlie
        \to
        \Univ(\glie) \tensor \Univ(\hlie) \,,
        \quad
        (x,y)
        \mapsto
        \class{(x,0)} \tensor 1 + 1 \tensor \class{(0,y)} \,.
      \]
      To construct the inverse~$\psi$ of~$\phi$ we observe that the equalities
      \[
        \psi( \class{(x,0)} )
        =
        \psi( \class{(x,0)} \cdot 1 )
        =
        \psi( i_*(\class{x}) \cdot j_*(1) )
        =
        \psi( \phi(\class{x} \tensor 1) )
        =
        \class{x} \tensor 1
      \]
      and similarly~$\psi( \class{(0,y)} ) = 1 \tensor \class{y}$ have to hold for all~$x \in \glie$ and~$y \in \glie$.
      It then follows that more generally
      \[
        \psi( \class{(x,y)} )
        =
        \psi( \class{(x,0)} + \class{(0,y)} )
        =
        \class{x} \tensor 1 + 1 \tensor \class{y}
      \]
      for all~$(x,y) \in \glie \times \hlie$.
      
      Motivated by these observations we consider the map
      \[
        \psi'
        \colon
        \glie \times \hlie
        \to
        \Univ(\glie) \tensor \Univ(\hlie)
      \]
      that is given by
      \[
        \psi'((x,y))
        =
        \class{x} \tensor 1 + 1 \tensor \class{y} \,.
      \]
      This map is~{\linear{$\kf$}} and it is a homomorphism of Lie~algebras because
      \begin{align*}
        {}&
        [\psi'((x_1, y_1)), \psi'((x_2, y_2))]
        \\
        ={}&
        [
          \class{x_1} \tensor 1 + 1 \tensor \class{y_1},
          \class{x_2} \tensor 1 + 1 \tensor \class{y_2}
        ]
        \\
        ={}&
          [\class{x_1} \tensor 1, \class{x_2} \tensor 1]
        + \underbrace{ [\class{x_1} \tensor 1, 1 \tensor \class{y_2}] }_{=0}
        + \underbrace{ [1 \tensor \class{y_1}, \class{x_2} \tensor 1] }_{=0}
        + [1 \tensor \class{y_1}, 1 \tensor \class{y_2}]
        \\
        ={}&
          [\class{x_1}, \class{x_2}] \tensor 1
        + 1 \tensor [\class{y_1}, \class{y_2}]
        \\
        ={}&
          \class{[x_1, x_2]} \tensor 1
        + 1 \tensor \class{[y_1, y_2]}  \,.
        \\
        ={}&
        \psi'( ( [x_1, x_2], [y_1, y_2] ) )
        \\
        ={}&
        \psi'( [(x_1, y_1), (x_2, y_2)] ) \,.
      \end{align*}
      It hence follows from the universal property of the {\ua}~$\Univ(\glie \times \hlie)$ that there exists a unique algebra homomorphism~$\psi \colon \Univ(\glie \times \hlie) \to \Univ(\glie) \tensor \Univ(\hlie)$ that makes the triangular diagram
      \[
        \begin{tikzcd}[row sep = large]
          \glie \times \hlie
          \arrow{r}[above]{\psi'}
          \arrow{d}[left]{\class{(-)}}
          &
          \Univ(\glie) \tensor \Univ(\hlie)
          \\
          \Univ(\glie \times \hlie)
          \arrow[dashed]{ur}[below right]{\psi}
          &
          {}
        \end{tikzcd}
      \]
      commute.
      This homomorphism~$\psi$ is given for all~$(x, y) \in \glie \times \hlie$ by
      \[
        \psi( \class{(x,y)} )
        =
        \class{x} \tensor 1 + 1 \tensor \class{y} \,.
      \]
      
      The homomorphisms~$\phi$ and~$\psi$ are mutually inverse:
      It sufficies to check this on the generators~$\class{x} \tensor \class{y}$ of~$\Univ(\glie) \tensor \Univ(\hlie)$ with~$x \in \glie$ and~$y \in \hlie$, and on the generators~$\class{(x,y)}$ of~$\Univ(\glie \times \hlie)$ with~$(x,y) \in \glie \times \hlie$.
      And indeed, we find that
      \begin{align*}
        \phi( \psi( \class{(x,y)} ) )
        &=
        \phi( \class{x} \tensor 1 + 1 \tensor \class{y} )
        \\
        &=
        \phi( \class{x} \tensor 1 ) + \phi( 1 \tensor \class{y} )
        \\
        &=
        i_*(\class{x}) j_*(1) + i_*(1) j_*(\class{y})
        \\
        &=
        \class{i(x)} + \class{j(y)}
        \\
        &=
        \class{(x,0)} + \class{(0,y)}
        \\
        &=
        \class{(x,y)}
      \end{align*}
      and
      \begin{align*}
        \psi( \phi( \class{x} \tensor \class{y} ) )
        &=
        \psi( i_*(\class{x}) j_*(\class{y}) )
        \\
        &=
        \psi( \class{(x,0)} \class{(0,y)} )
        \\
        &=
        \psi( \class{(x,0)} ) \psi( \class{(0,y)} )
        \\
        &=
        ( \class{x} \tensor 1 + 1 \tensor 0 ) ( 0 \tensor 1 + 1 \tensor \class{y} )
        \\
        &=
        (\class{x} \tensor 1) (1 \tensor \class{y})
        \\
        &=
        \class{x} \tensor \class{y} \,.
      \end{align*}
  \end{itemize}
\end{example}



\subsubsection{Free Lie~Algebras}


\begin{definition}
  Let~$X$ be a set.
  A~\defemph{free~{\liealgebra{$\kf$}} on~$X$}\index{free Lie algebra}\index{Lie algebra!free} is a~{\liealgebra{$\kf$}}~$F(X)$ together with a map~$\iota \colon X \to F(X)$ such that for every~{\liealgebra{$\kf$}}~$\glie$ and every map~$\phi \colon X \to \glie$ there exists a unique homomorphism of Lie~algebras~$\Phi \colon F(X) \to \glie$ with~$\phi = \Phi \circ \iota$, i.e.\ that makes the following triangular diagram commute:
  \[
    \begin{tikzcd}
      X
      \arrow{dr}[above right]{\phi}
      \arrow{d}[left]{\iota}
      &
      {}
      \\
      F(X)
      \arrow[dashed]{r}[below]{\Phi}
      &
      \glie
    \end{tikzcd}
  \]
\end{definition}


\begin{remark}
  \leavevmode
  \begin{enumerate}
    \item
      One may reformulate the above definition as saying that there exists for every~$x \in X$ an associated element~$\class{x} \in F(X)$ and that whenever~$\hlie$ is any Lie~algebra then there exist for every choice of images~$y_x \in \hlie$ with~$x \in X$ a unique homomorphism of Lie~algebras~$F(X) \to \hlie$ that maps~$\class{x}$ to~$y_x$.
    \item
      As usual with free objects it follows that any two free Lie~algebras over a set~$X$ are unique up to unique isomorphism:
      If~$F_1$ with~$\iota_1 \colon X \to F_1$ and~$F_2$ with~$\iota_2 \colon X \to F_2$ are two~free~{\liealgebras{$\kf$}} over~$X$ then there exist unique homomorphisms of Lie~algebras~$\phi \colon F_1 \to F_2$ and~$\psi \colon F_2 \to F_1$ that make the triangular diagrams
      \[
        \begin{tikzcd}[column sep = small]
          {}
          &
          X
          \arrow{dl}[above left]{\iota_1}
          \arrow{dr}[above right]{\iota_2}
          &
          {}
          \\
          F_1
          \arrow[dashed]{rr}[below]{\phi}
          &
          {}
          &
          F_2
        \end{tikzcd}
        \qquad\text{and}\qquad
        \begin{tikzcd}[column sep = small]
          {}
          &
          X
          \arrow{dl}[above left]{\iota_2}
          \arrow{dr}[above right]{\iota_1}
          &
          {}
          \\
          F_2
          \arrow[dashed]{rr}[below]{\psi}
          &
          {}
          &
          F_1
        \end{tikzcd}
      \]
      commute.
      The homomorphisms~$\phi$ and~$\psi$ are now mutually inverse isomorphisms.
      We will therefore always talk about \emph{the} free~{\liealgebra{$\kf$}} over~$X$.
    \item
      For every map between sets~$f \colon X \to Y$ there exists a unique homomorphism of Lie~algebras~$f_* \colon F(X) \to F(Y)$ that makes the square diagram
      \[
        \begin{tikzcd}
          X
          \arrow{r}[above]{f}
          \arrow{d}[left]{\iota_X}
          &
          Y
          \arrow{d}[right]{\iota_Y}
          \\
          F(X)
          \arrow[dashed]{r}[below]{f_*}
          &
          F(Y)
        \end{tikzcd}
      \]
      commute.
      It holds for every set~$X$ that~$(\id_X)_* = \id_{F(X)}$ and it holds for all composable maps of sets~$f \colon X \to Y$ and~$g \colon Y \to Z$ that~$(g \circ f)_* = g_* \circ f_*$.
      We therefore get a functor~$F \colon \cSet \to \cLie{\kf}$.
      
      The universal property of the free Lie~algebra states that the functor~$F$ is left adjoint to the forgetful functor~$\cLie{\kf} \to \cSet$.
  \end{enumerate}
\end{remark}


\begin{example}[Free Lie~algebras]
  \label{uea of free lie algebra}
  Let~$I$ be a set and let~$F(I)$ be the free~{\liealgebra{$\kf$}} on~$I$.
  We have for every~{\algebra{$\kf$}}~$A$ bijections
  \begin{align*}
    {}&
    \{ \text{algebra homomorphisms~$\Univ(F(I)) \to A$} \}
    \\
    \cong&
    \{ \text{Lie~algebra homomorphisms~$F(I) \to A$} \}
    \\
    \cong&
    \{ \text{set-theoretic maps~$I \to A$} \}
    \\\
    \cong&
    \{ \text{algebra homomorphisms~$\kf\gen{x_i \suchthat i \in I} \to A$} \} \,,
  \end{align*}
  and these bijections are natural in~$A$.
  We hence find by Yoneda’s~lemma that~$\Univ(F(I)) \cong \kf\gen{x_i \suchthat i \in I}$.
  More precisely, there exists for the composition~$I \to F(I) \to \Univ(F(I))$ a unique algebra homomorphism~$\phi \colon \kf\gen{x_i \suchthat i \in I} \to \Univ(F(I))$ that makes the diagram
  \[
    \begin{tikzcd}
        {}
      & I
        \arrow[bend right, out = -45, in=225]{ddl}[above left]{i \mapsto x_i}
        \arrow[bend left]{dr}[above right]{\iota}
      & {}
      \\
        {}
      & {}
      & F(I)
        \arrow{d}
      \\
        \kf\gen{x_i \suchthat i \in I}
        \arrow[dashed]{rr}[below]{\phi}
      & {}
      & \Univ(F(I))
    \end{tikzcd}
  \]
  commute, and~$\phi$ is an isomorphism.
\end{example}




