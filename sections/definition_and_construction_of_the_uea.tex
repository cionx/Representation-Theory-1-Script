\section{Universal Enveloping Algebra}





\subsection{Definition}


\begin{definition}
  Let~$\glie$ be a~\liealgebra{$\kf$}.
  A \defemph{universal enveloping algebra}\index{universal enveloping algebra} of~$\glie$ is a~\algebra{$\kf$}~$\Univ(\glie)$\glsadd{universal enveloping algebra} together with a homomorphism of Lie~algebras~$\iota$ from~$\glie$ to~$\Univ(\glie)$ such that the following universal property holds:
  for every~{\algebra{$\kf$}}~$A$ and every homomorphism of Lie~algebras~$\varphi$ from~$\glie$ to~$A$ there exists a unique homomorphism of~\algebras{$\kf$}~$\Phi$  from~$\Univ(\glie)$ to~$A$ that makes the triangular diagram
  \[
    \begin{tikzcd}
      \glie
      \arrow{r}[above]{\phi}
      \arrow{d}[left]{\iota}
      &
      A
      \\
      \Univ(\glie)
      \arrow[dashed]{ur}[below right]{\Phi}
      &
      {}
    \end{tikzcd}
  \]
  commute, i.e.\ such that~$\varphi = \Phi \circ \iota$.
\end{definition}


\begin{remark}[Uniqueness of universal enveloping algebras]
  \label{uniqueness of universal enveloping algebras}
  Let~$\glie$ be a Lie algebra and suppose that~$(\Univ(\glie)_1, \iota_1)$ and~$(\Univ(\glie)_2, \iota_2)$ are two~{\uas} of~$\glie$.
  Then there exist unique algebra homomorphisms~$\Phi$ from~$\Univ(\glie)_1$ to~$\Univ(\glie)_2$ and~$\Psi$ from~$\Univ(\glie)_2$ to~$\Univ(\glie)_1$ that make the triangular diagrams
  \[
    \begin{tikzcd}[column sep = small]
      {}
      &
      \glie
      \arrow{dl}[above left]{\iota_1}
      \arrow{dr}[above right]{\iota_2}
      &
      {}
      \\
      \Univ(\glie)_1
      \arrow[dashed]{rr}[below]{\Phi}
      &
      {}
      &
      \Univ(\glie)_2
    \end{tikzcd}
    \qquad\text{and}\qquad
    \begin{tikzcd}[column sep = small]
      {}
      &
      \glie
      \arrow{dl}[above left]{\iota_2}
      \arrow{dr}[above right]{\iota_1}
      &
      {}
      \\
      \Univ(\glie)_2
      \arrow[dashed]{rr}[below]{\Psi}
      &
      {}
      &
      \Univ(\glie)_1
    \end{tikzcd}
  \]
  commute.
  It follows that the composites~$\Psi \circ \Phi$ and~$\Phi \circ \Psi$ make the triangle diagrams
  \[
    \begin{tikzcd}[column sep = small]
      {}
      &
      \glie
      \arrow{dl}[above left]{\iota_1}
      \arrow{dr}[above right]{\iota_1}
      &
      {}
      \\
      \Univ(\glie)_1
      \arrow[dashed]{rr}[below]{\Psi \circ \Phi}
      &
      {}
      &
      \Univ(\glie)_1
    \end{tikzcd}
    \qquad\text{and}\qquad
    \begin{tikzcd}[column sep = small]
      {}
      &
      \glie
      \arrow{dl}[above left]{\iota_2}
      \arrow{dr}[above right]{\iota_2}
      &
      {}
      \\
      \Univ(\glie)_2
      \arrow[dashed]{rr}[below]{\Phi \circ \Psi}
      &
      {}
      &
      \Univ(\glie)_2
    \end{tikzcd}
  \]
  commute.

  The algebra homomorphisms~$\Phi \circ \Psi$ and~$\Psi \circ \Phi$ are unique with this property by the universal properties of the {\uas}~$(\Univ(\glie)_1, \iota_1)$ and~$(\Univ(\glie)_2, \iota_2)$.
  But the identities~$\id_{\Univ(\glie)_1}$ and~$\id_{\Univ(\glie)_2}$ also makes these diagrams commute.
  We thus find that the composite~$\Psi \circ \Phi$ equals~$\id_{\Univ(\glie)_1}$ and the composite~$\Phi \circ \Psi$ equals~$\id_{\Univ(\glie)_2}$.
  The homomorphisms~$\Phi$ and~$\Psi$ are therefore mutually inverse isomorphisms.
  
  This shows that a {\ua} of~$\glie$ is unique up to unique isomorphism.
  We will therefore talk about \emph{the} {\ua} of~$\glie$.
  We will often also surpress the algebra homorphism~$\iota$ from~$\glie$ to~$\Univ(\glie)$ from our notation.
\end{remark}


% \begin{remark}
%   One can also formulate the above argument is a more categorical way:
%   Consider the category~$\catC$ where
%   \begin{itemize}
%     \item
%       objects of~$\catC$ is a pairs~$(A, i)$ consisting of a~{\algebra{$\kf$}}~$A$ and a Lie~algebra homomorphism~$i \colon \glie \to A$,
%     \item
%       a morphism~$\phi \colon (A, i) \to (B, j)$ is an algebra homomorphism~$\phi \colon A \to B$ that makes the triangular diagram
%       \[
%         \begin{tikzcd}[column sep = small]
%         {}
%         &
%         \glie
%         \arrow{dl}[above left]{i}
%         \arrow{dr}[above right]{j}
%         &
%         {}
%         \\
%         A
%         \arrow[dashed]{rr}[below]{\phi}
%         &
%         {}
%         &
%         B
%       \end{tikzcd}
%     \]
%       commute, and
%     \item
%       the composition of two morphisms is just their usual set-theoretic composition.
%   \end{itemize}
%   A {\ua} of~$\glie$ is nothing but an inital object in this category~$\catC$.
%   The argumentation from \cref{uniqueness of universal enveloping algebras} is then the usual argument for the uniqueness of inital objects up to unique isomorphism.
% \end{remark}



\subsection{Construction}


\begin{fluff}
  Let~$\glie$ be a Lie~algebra.
  We will in the following show that the {\ua} of~$\glie$ exists.
  For this we will first conclude from the universal property of~$\Univ(\glie)$ that we should be able to construct~$\Univ(\glie)$ as a certain quotient algebra of the tensor algebra~$\Tensor(\glie)$.
  We then show that this quotient does indeed have the correct universal property.

  Suppose that~$\glie$ admits a universal enveloping algebra~$\Univ(\glie)$ and let~$\iota$ be the canonical homomorphism of Lie~algebras from~$\glie$ to~$\Univ(\glie)$.
  We first observe that the algebra~$\Univ(\glie)$ is generated by the image of~$\iota$.

  Indeed, let~$U$ be the subalgebra of~$\Univ(\glie)$ which is generated by the image of~$\iota$, and let~$\iota'$ be the restriction of~$\iota$ to a homomorphism of Lie~algebras from~$\glie$ to~$U$.
  For every~{\algebra{$\kf$}}~$A$ and every Lie~algebra homomorphism~$\varphi$ from~$\glie$ to~$A$ the induced homomorphism of algebras~$\Phi$ from~$\Univ(\glie)$ to~$A$ restricts to an homomorphism of algebras~$\Phi'$ from~$U$ to~$A$.
  This homomorphism~$\Phi'$ makes the triangular diagram
  \[
    \begin{tikzcd}
      \glie
      \arrow{r}[above]{\varphi}
      \arrow{d}[left]{\iota'}
      &
      A
      \\
      U
      \arrow[dashed]{ur}[below right]{\Phi'}
      &
      {}
    \end{tikzcd}
  \]
  commute.
  The homomorphism~$\Phi'$ is unique with this property because the algebra~$U$ is generated by the image of~$\iota'$, and the composite~$\Phi \circ \iota'$ equals the fixed homomorphism of Lie~algebras~$\varphi$.
  This shows that the algebra~$U$ together with the homomorphism of Lie~algebras~$\iota'$ is again a {\ua} for~$\glie$.
  
  It follows from the uniqueness of the {\ua} of~$\glie$, as discussed in \cref{uniqueness of universal enveloping algebras}, that there exists a unique homomorphism of algebras~$\Iota$ from~$U$ to~$\Univ(\glie)$ that makes the triangular diagram
  \[
    \begin{tikzcd}[column sep = small]
      {}
      &
      \glie
      \arrow{dl}[above left]{\iota'}
      \arrow{dr}[above right]{\iota}
      &
      {}
      \\
      U
      \arrow[dashed]{rr}[below]{\Iota}
      &
      {}
      &
      \Univ(\glie)
    \end{tikzcd}
  \]
  commute, and that this homomorphism is already an isomorphism.
  The homomorphism~$\Iota$ is the inclusion map from~$U$ to~$\Univ(\glie)$ because this is a homomorphism of algebras from~$U$ to~$\Univ(\glie)$ which makes the above triangular diagram commute.
  We have thus found that the inclusion map from~$U$ to~$\Univ(\glie)$ is an isomorphism of algebras, whence~$U$ equals~$\Univ(\glie)$.

  We now apply the universal property of the tensor algebra~$\Tensor(\glie)$ to the linear map~$\iota$.
  We find that there exists a unique homomorphism of algebras~$\Phi'$ from~$\Tensor(\glie)$ to~$\Univ(\glie)$ that makes the triangular diagram
  \[
    \begin{tikzcd}[column sep = small]
      {}
      &
      \glie
      \arrow{dl}
      \arrow{dr}[above right]{\iota}
      &
      {}
      \\
      \Tensor(\glie)
      \arrow[dashed]{rr}[below]{\Phi'}
      &
      {}
      &
      \Univ(\glie)
    \end{tikzcd}
  \]
  commute.
  The homomorphism~$\Phi'$ is surjective because~$\Univ(\glie)$ is generated by the image of~$\iota$ as an algebra.
  It follows that~$\Phi'$ induces an isomorphism of algebras
  \[
    \Psi
    \colon
    \Tensor(\glie) / I
    \to
    \Univ(\glie)
  \]
  where the ideal~$I$ is the kernel of~$\Phi'$.
  This isomorphism makes the resulting diagram
   \[
    \begin{tikzcd}[column sep = small]
      {}
      &
      \glie
      \arrow[bend right]{dl}
      \arrow[bend left]{ddr}[above right]{\iota}
      &
      {}
      \\
      \Tensor(\glie)
      \arrow[bend left]{drr}[below left]{\Phi}
      \arrow{d}
      &
      {}
      &
      {}
      \\
      \Tensor(\glie)/I
      \arrow[dashed]{rr}[below]{\Psi}
      &
      {}
      &
      \Univ(\glie)
    \end{tikzcd}
  \]
  commute.
 
  Let~$A$ be another~\algebra{$\kf$}.
  Every linear map~$g$ from~$\glie$ to~$A$ factors through a homomorphism of algebras~$\Psi'$ from~$\Tensor(\glie)$ to~$A$.
  It follows from the above isomorphism~$\Phi$ between~$\Tensor(\glie) / I$ and~$\Univ(\glie)$ that the homomorphism~$\Psi'$ factors trough a homorphism from~$\Tensor(\glie) / I$ to~$A$ if and only if the linear map~$g$ is a homomorphism of Lie~algebras.

  That~$g$ is a homomorphism of Lie~algebras means that
  \[
    g(x) g(y) - g(y) g(x) - g([x,y]) = 0
  \]
  for all~$x, y \in \glie$.
  This is equivalent to the condition
  \[
    \Psi(x) \Psi(y) - \Psi(y) \Psi(x) - \Psi([x,y]_{\glie})
    =
    0
  \]
  for all~$x, y \in \glie$, and further äquivalent to the condition
  \[
    \Psi( xy - yx - [x,y]_{\glie} )
    =
    0
  \]
  for all~$x, y \in \glie$.

  We have now seen that an algebra homomorphism~$\Psi$ from~$\Tensor(\glie)$ to some~\algebra{$\kf$}~$A$ factors trough the quotient~$\Tensor(\glie)/I$ if and only if~$\Psi$ annihilates all those elements of~$\Tensor(\glie)$ that are of the form~$xy - yx - [x,y]_{\glie}$ with~$x$,~$y$ in~$\glie$.
  This means that the ideal~$I$ needs to be generated by those elements.
  
  We have now altogether seen that the universal enveloping algebra~$\Univ(\glie)$ needs to be constructable as the quotient of the tensor algebra~$\Tensor(\glie)$ by the ideal~$I$ which is generated by all those elements of the form~$x y - y x - [x,y]_{\glie}$ with~$x$,~$y$ in~$\glie$.
  We will conversely show in the following \lcnamecref{existence of uea} that this construction will indeed give us the universal enveloping algebra.
\end{fluff}


\begin{proposition}[Existence of the universal enveloping algebra]
  \label{existence of uea}
  Let~$\glie$ be a Lie~algebra.
  Let~$\Tensor(\glie)$ be the tensor algebra of the underlying vector space of~$\glie$ and let~$I$ the two-sided ideal of~$\Tensor(\glie)$ generated by all the elements $x y - y x - [x,y]_{\glie}$ with~$x$,~$y$ in~$\glie$.
  The quotient algebra~$U \defined T(\glie)/I$ together with the~{\linear{$\kf$}} map
  \[
    \iota
    \colon
    \glie
    \to
    \Univ(\glie) \,,
    \quad
    x
    \mapsto
    \class{x}
  \]
  is a {\ua} for~$\glie$.
\end{proposition}


\begin{proof}
  The map~$\iota$ is~{\linear{$\kf$}} and it compatible with the Lie brackets because
  \[
    [\iota(x), \iota(y)]
    =
    [\class{x}, \class{y}]
    =
    \class{x} \, \class{y} - \class{y} \, \class{x}
    =
    \class{x y - y x}
    =
    \class{[x,y]_{\glie}}
    =
    \iota([x,y]_{\glie}) \,.
  \]
  for all~$x, y \in \glie$.
  Given any~\algebra{$\kf$}~$A$ and Lie algebra homomorphism~$\varphi$ from~$\glie$ to~$A$ there exists a unique homorphism of~\algebras{$\kf$}~$\Phi'$ from~$\Tensor(\glie)$ to~$A$ that makes the triangular diagram
  \[
    \begin{tikzcd}
      \glie
      \arrow{r}[above]{\varphi}
      \arrow{d}
      &
      A
      \\
      \Tensor(V)
      \arrow[dashed]{ur}[below right]{\Phi'}
      &
      {}
    \end{tikzcd}
  \]
  commute.
  The homomorphism~$\Phi'$ is given by~$\Phi'(x) = \varphi(x)$ for all~$x \in \glie$.
  It follows that
  \begin{align*}
    \Phi'(x y - y x)
    &=
    \Phi'(x) \Phi'(y) - \Phi'(y) \Phi'(x)
    \\
    &=
    \varphi(x) \varphi(y) - \varphi(y) \varphi(x)
    \\
    &=
    [ \varphi(x), \varphi(y) ]
    \\
    &=
    \varphi( [x,y]_{\glie} )
    \\
    &=
    \Phi'( [x,y]_{\glie} )
  \end{align*}
  for all~$x, y \in \glie$.
  The ideal~$I$ is therefore contained in the kernel of the homomorphism~$\Phi'$.
  It follows that there exists a unique homomorphism of algebras~$\Phi$ from~$U$ to~$A$ that makes the diagram
  \[
    \begin{tikzcd}
      \glie
      \arrow{r}[above]{\varphi}
      \arrow{d}
      &
      A
      \\
      \Tensor(\glie)
      \arrow[bend right= 20]{ur}[above left]{\Phi'}
      \arrow{d}[left]{\pi}
      &
      {}
      \\
      U
      \arrow[dashed, bend right = 30]{uur}[below right]{\Phi}
      &
      {}
    \end{tikzcd}
  \]
  commute, where~$\Pi$ denotes the canonical projection from~$\Tensor(V)$ to~$\Tensor(V)/I$.
  We may add the homomorphism of Lie~algebras~$\iota$ to this diagram.
  We then arrive at the following commutative diagram.
  \[
    \begin{tikzcd}
      \glie
      \arrow{r}[above]{\varphi}
      \arrow{d}
      \arrow[bend right = 55]{dd}[left]{\iota}
      &
      A
      \\
      \Tensor(\glie)
      \arrow[bend right= 20]{ur}[above left]{\Phi'}
      \arrow{d}[left]{\pi}
      &
      {}
      \\
      U
      \arrow[bend right = 30]{uur}[below right]{\Phi}
      &
      {}
    \end{tikzcd}
  \]
  We have in particular the following commutative subdiagram.
  \[
    \begin{tikzcd}
      \glie
      \arrow{r}[above]{\varphi}
      \arrow{d}[left]{i}
      &
      A
      \\
      U
      \arrow{ur}[below right]{\Phi}
      &
      {}
    \end{tikzcd}
  \]
  We have thus shown that every homomorphism of Lie~algebras~$\varphi$ from~$\glie$ to~$A$ extends to a homomorphism of algebra~$\Phi$ from~$U$ to~$A$ .
  The algebra~$U$ is generated by the image of~$\iota$ whence the homomorphism of algebras~$\Phi$ is unique with this property.
\end{proof}


\begin{remark}
  The above proof may be summarized by observing that we have bijections
  \begin{align*}
    {}&
    \{ \textstyle\text{algebra homomorphisms~$\Phi \colon \Tensor(\glie)/I \to A$} \}
    \\
    \cong{}&
    \{ \text{algebra homomorphisms~$\Phi' \colon \Tensor(\glie) \to A$ with~$\Phi'(I) = 0$} \}
    \\
    \cong{}&
    \left\{
      \begin{tabular}{@{}c@{}}
        algebra homomorphisms~$\Phi' \colon \Tensor(\glie) \to A$ with  \\
        $\Phi'(x y - y x - [x,y]_{\glie}) = 0$ for all~$x, y \in \glie$
      \end{tabular}
    \right\}
    \\
    \cong{}&
    \left\{
      \begin{tabular}{@{}c@{}}
        algebra homomorphisms~$\Phi' \colon \Tensor(\glie) \to A$ with  \\
        $\Phi'(x) \Phi'(y) - \Phi'(y) \Phi'(x) - \Phi'([x,y]_{\glie}) = 0$ for all~$x, y \in \glie$
      \end{tabular}
    \right\}
    \\
    \cong{}&
    \left\{
      \begin{tabular}{@{}c@{}}
        algebra homomorphisms~$\Phi' \colon \Tensor(\glie) \to A$ with  \\
        $\Phi'(x) \Phi'(y) - \Phi'(y) \Phi'(x) = \Phi'([x,y]_{\glie})$ for all~$x, y \in \glie$
      \end{tabular}
    \right\}
    \\
    \cong{}&
    \left\{
      \begin{tabular}{@{}c@{}}
        {\linear{$\kf$}} maps~$\varphi \colon \glie \to A$ with  \\
        $\varphi(x) \varphi(y) - \varphi(y) \varphi(x) = \varphi([x,y])$ for all~$x, y \in \glie$
      \end{tabular}
    \right\}
    \\
    \cong{}&
    \left\{
      \begin{tabular}{@{}c@{}}
        {\linear{$\kf$}} maps~$\varphi \colon \glie \to A$ with  \\
        $[\varphi(x), \varphi(y)] = \varphi([x,y])$ for all~$x, y \in \glie$ 
      \end{tabular}
    \right\}
    \\
    ={}&
    \{ \textstyle\text{Lie~algebra homomorphisms~$\varphi \colon \glie \to A$} \} \,,
  \end{align*}
  and that these bijections are natural in~$A$.
  This shows that the~{\algebra{$\kf$}}~$\Tensor(\glie)/I$ represents the right kind of functor.
  We can also see that the identity of~$\Tensor(\glie)/I$ corresponds under the above bijections (for~$A = \Tensor(\glie)/I$) to the map~$\iota$ from~$\glie$ to~$\Tensor(\glie)/I$.
\end{remark}



\subsection{Properties}

\subsubsection{Anti-Homomorphisms}

\begin{proposition}
  Let~$\glie$ be a Lie~algebra, let~$\iota$ be the canonical homomorphism of Lie~algebras from~$\glie$ to~$\Univ(\glie)$ and let~$A$ be a~\algebra{$\kf$}.
  We have a well-defined {\onetoonetext} correspondence given by
  \begin{align*}
    \SwapAboveDisplaySkip
    \left\{
      \begin{tabular}{@{}c@{}}
        anti-homomorphisms \\
        of Lie~algebras
        $\varphi \colon \glie \to A$
      \end{tabular}
    \right\}
    &\onetoone
    \left\{
      \begin{tabular}{@{}c@{}}
        anti-homomorphisms \\
        of algebras
        $\Phi \colon \Univ(\glie) \to A$
      \end{tabular}
    \right\} \,,
    \\
    \Phi \circ \iota
    &\mapsfrom
    \Phi \,.
  \end{align*}
\end{proposition}

\begin{proof}
  We have {\onetoonetext} correspondence given by
  \begin{align*}
    \left\{
      \begin{tabular}{@{}c@{}}
        homomorphisms \\
        of Lie~algebras
        $\varphi \colon \glie \to A^{\op}$
      \end{tabular}
    \right\}
    &\onetoone
    \left\{
      \begin{tabular}{@{}c@{}}
        homomorphisms \\
        of algebras
        $\Phi \colon \Univ(\glie) \to A^{\op}$
      \end{tabular}
    \right\} \,,
    \\
    \Phi \circ \iota
    &\mapsfrom
    \Phi \,.
  \end{align*}
  An anti-homomorphism of Lie~algebras from~$\glie$ to~$A$ is the same as a homomorphim of Lie~algebras from~$\glie$ to~$A^{\op}$, and an anti-homomorphism of algebras from~$\Univ(\glie)$ to~$A$ is the same as a homomorphism of algebras from~$\Univ(\glie)$ to~$A^{\op}$.
\end{proof}

\subsubsection{Representations and Modules}

\begin{proposition}
  \label{representations are modules}
  Let~$M$ be a~{\vectorspace{$\kf$}} and let~$\glie$ be a Lie~algebra.
  Let~$\Univ(\glie)$ be the universal enveloping algebra of~$\glie$ and let~$\iota$ be the canonical homorphism of Lie~algebras from~$\glie$ to~$\Univ(\glie)$.
  \begin{enumerate}
    \item
      For every homomorphism of Lie~algebras~$\rho$ from~$\glie$ to~$\gllie(M)$ let~$\widehat{\rho}$ denote the corresponding homomorphism of algebras from~$\Univ(\glie)$ to~$\End_{\kf}(M)$.
      Then the assignments
      \begin{align*}
        \left\{
        \begin{tabular}{@{}c@{}}
          representations \\
          $\rho \colon \glie \to \gllie(M)$
        \end{tabular}
        \right\}
        &\onetoone
        \left\{
        \begin{tabular}{@{}c@{}}
          $\Univ(\glie)$-module structures \\
          $\Rho \colon \Univ(\glie) \to \End_{\kf}(M)$
        \end{tabular}
        \right\}  \,,
        \\
        \rho
        &\mapsto
        \widehat{\rho} \,,
        \\
        \Rho \circ \iota
        &\mapsfrom
        \Rho  \,,
      \end{align*}
      constitute a {\onetoonetext} correspondence.
    \item
      For every anti-homomorphism of Lie~algebras~$\rho$ from~$\glie$ to~$\gllie(M)$ let~$\widehat{\rho}$ denote the corresponding anti-homomorphism of algebras from~$\Univ(\glie)$ to~$\End_{\kf}(M)$.
      Then the assignments
      \begin{align*}
        \left\{
        \begin{tabular}{@{}c@{}}
          right representations \\
          $\rho \colon \glie \to \gllie(M)$
        \end{tabular}
        \right\}
        &\onetoone
        \left\{
        \begin{tabular}{@{}c@{}}
          right $\Univ(\glie)$-module structures \\
          $\Rho \colon \Univ(\glie) \to \End_{\kf}(M)$
        \end{tabular}
        \right\}  \,,
        \\
        \rho
        &\mapsto
        \widehat{\rho} \,,
        \\
        \Rho \circ \iota
        &\mapsfrom
        \Rho  \,,
      \end{align*}
      constitute a {\onetoonetext} correspondence.
  \end{enumerate}
\end{proposition}

\begin{fluff}
  Let~$\glie$ be a Lie~algebra.
  If~$M$ is a left~\module{$\Univ(\glie)$} then the corresponding act of~$\glie$ on~$M$ is given by
  \[
    x \act m
    =
    \class{x} \cdot m
  \]
  for all~$x \in \glie$,~$m \in M$.
  Similarly, if~$M$ is a right~\module{$\Univ(\glie)$} then the corresponding right action of~$\glie$ on~$M$ is given by
  \[
    m \act x
    =
    m \cdot \class{x}
  \]
  for all~$x \in \glie$,~$m \in M$.
\end{fluff}

\subsubsection{Functoriality}

\begin{remark}
  Let~$\glie$ be a Lie~algebra with universal enveloping algebra~$\Univ(\glie)$.
  \Cref{representations are modules} shows that representations of~$\glie$ are the same as~{\modules{$\Univ(\glie)$}}.
  We get from this correspondence an isomorphism of categories between~$\cRep{\glie}$ and~$\cMod{\Univ(\glie)}$.
\end{remark}


\begin{lemma}[Functoriality of the universal enveloping algebra]
  \label{functoriality of universal enveloping algebra}
  Let~$\glie$,~$\hlie$ and~$\klie$ be Lie~algebras.
  \begin{enumerate}
    \item
      For every homomorphism of Lie~algebras~$\varphi$ from~$\glie$ to~$\hlie$ there exists a unique homomorphism of algebras~$\Univ(\varphi)$ from~$\Univ(\glie)$ to~$\Univ(\hlie)$ that makes the following square diagram commute.
      \[
        \begin{tikzcd}[column sep = large]
          \glie
          \arrow{r}[above]{\varphi}
          \arrow{d}[left]{\iota_{\glie}}
          &
          \hlie
          \arrow{d}[right]{\iota_{\hlie}}
          \\
          \Univ(\glie)
          \arrow[dashed]{r}[below]{\Univ(\varphi)}
          &
          \Univ(\hlie)
        \end{tikzcd}
      \]
    \item
      It holds that~$\Univ(\glie) = \id_{\Univ(\glie)}$.
    \item
      It holds for all composable homomorphisms of Lie~algebras~$\varphi$ from~$\glie$ to~$\hlie$ and~$\psi$ from~$\hlie$ to~$\klie$ that
      \[
        \Univ( \psi \circ \varphi )
        =
        \Univ( \psi ) \circ \Univ( \varphi ) \,.
      \]
  \end{enumerate}
\end{lemma}


\begin{proof}
  \leavevmode
  \begin{enumerate}
    \item
      The composite~$\iota_{\hlie} \circ \varphi$ is a homomorphism of Lie~algebras from~$\glie$ to~$\Univ(\hlie)$.
      By the universal property of the universal enveloping algebra~$\Univ(\glie)$ there exists a unique homomorphism of algebras~$\Univ(\varphi)$ from~$\Univ(\glie)$ to~$\Univ(\hlie)$ with~$\Univ(\varphi) \circ \iota_{\glie} = \iota_{\hlie} \circ \varphi$.
    \item
      The square diagram
      \[
        \begin{tikzcd}[column sep = huge]
          \glie
          \arrow{r}[above]{\id_{\glie}}
          \arrow{d}
          &
          \glie
          \arrow{d}
          \\
          \Univ(\glie)
          \arrow[dashed]{r}[below]{\id_{\Univ(\glie)}}
          &
          \Univ(\glie)
        \end{tikzcd}
      \]
      commutes, which shows that the identity homomorphism~$\id_{\Univ(\glie)}$ satisfies the defining property of the induced algebra homomorphism~$\Univ( \id_{\glie} )$.
    \item
      We have the following commutative diagram:
      \[
        \begin{tikzcd}[column sep = large]
          \glie
          \arrow[dashed, bend left = 40]{rr}[above]{\psi \circ \varphi}
          \arrow{r}[above]{\varphi}
          \arrow{d}
          &
          \hlie
          \arrow{r}[above]{\psi}
          \arrow{d}
          &
          \klie
          \arrow{d}
          \\
          \Univ(\glie)
          \arrow{r}[below]{\Univ(\varphi)}
          \arrow[dashed, bend right = 40]{rr}[below]{\Univ(\psi) \circ \Univ(\varphi)}
          &
          \Univ(\hlie)
          \arrow{r}[below]{\Univ(\psi)}
          &
          \Univ(\klie)
        \end{tikzcd}
      \]
      The commutativity of the outer square diagram
      \[
        \begin{tikzcd}[column sep = huge]
          \glie
          \arrow{r}[above]{\psi \circ \varphi}
          \arrow{d}
          &
          \klie
          \arrow{d}
          \\
          \Univ(\glie)
          \arrow[dashed]{r}[below]{\Univ(\psi) \circ \Univ(\varphi)}
          &
          \Univ(\klie)
        \end{tikzcd}
      \]
      shows that the composite~$\Univ(\psi) \circ \Univ(\varphi)$ satisfies the defining property of the induced algebra homomorphism~$\Univ(\psi \circ \varphi)$.
    \qedhere
  \end{enumerate}
\end{proof}


\begin{remark}
  \Cref{functoriality of universal enveloping algebra} shows that the assignment~$\glie \mapsto \Univ(\glie)$ of a Lie~algebra~$\glie$ to its universal eveloping algebra~$\Univ(\glie)$ can be extended to a (covariant) functor~$\Univ$ from~$\cLie{\kf}$ to~$\cAlg{\kf}$.
  The universal property of the {\ua} states that the functor~$\Univ$ is left adjoint to the forgetful functor from~$\cAlg{\kf}$ to~$\cLie{\kf}$, which assigns to each~{\algebra{$\kf$}} its underlying Lie~algebra.
\end{remark}

\subsubsection{Derivations}

\begin{proposition}
  \label{extending derivation to universal enveloping algebra}
  Let~$\glie$ be a Lie~algebra.
  Every derivation of~$\glie$ extends uniquely to an extension of~$\Univ(\glie)$.
  More explicitely, there exists for every Lie~algebra derivation~$\delta$ of~$\glie$ a unique algebra derivation~$\Delta$ of~$\Univ(\glie)$ such that the following square diagram commutes.
  \[
    \begin{tikzcd}
      \glie
      \arrow{r}[above]{\delta}
      \arrow{d}
      &
      \glie
      \arrow{d}
      \\
      \Univ(\glie)
      \arrow[dashed]{r}[above]{\Delta}
      &
      \Univ(\glie) \,.
    \end{tikzcd}
  \]
\end{proposition}


\begin{lemma}
  \label{translating between derivations and homomorphisms}
  \leavevmode
  \begin{enumerate}
    \item
      Let~$A$ be a~\algebra{$\kf$} and let~$B$ be the~\algebra{$\kf$}
      \[
        B
        \defined
        \begin{pmatrix}
          A & A \\
          0 & A
        \end{pmatrix} \,.
      \]
      A map~$\Delta$ from~$A$ to~$A$ is a derivation of~$A$ if and only if the map
      \[
        \Phi
        \colon
        A
        \to
        B \,,
        \quad
        a
        \mapsto
        \begin{pmatrix}
          a & \Delta(a) \\
          0 & a
        \end{pmatrix}
      \]
      is a homomorphism of algebras.
    \item
      Let~$\glie$ be a Lie~algebra and let
      \[
        B
        \defined
        \begin{pmatrix}
          \Univ(\glie)  & \Univ(\glie)  \\
          0             & \Univ(\glie)
        \end{pmatrix} \,.
      \]
      A map~$\delta$ from~$\glie$ to~$B$ is a derivation of~$\glie$ if and only if the map
      \[
        \varphi
        \colon
        \glie
        \to
        B \,,
        \quad
        x
        \mapsto
        \begin{pmatrix}
          x & x \\
          0 & x
        \end{pmatrix}
      \]
      is a homomorphism of Lie~algebras
  \end{enumerate}
\end{lemma}


\begin{proof}
  \leavevmode
  \begin{enumerate}
    \item
      The map~$\Phi$ is linear if and only if the map~$\Delta$ is linear.
      We have
      \[
        \Phi(a) \cdot \Phi(b)
        =
        \begin{pmatrix}
          a & \Delta(a) \\
          0 & a
        \end{pmatrix}
        \begin{pmatrix}
          b & \Delta(b) \\
          0 & b
        \end{pmatrix}
        =
        \begin{pmatrix}
          ab  & \Delta(a) b + a \Delta(b)
          0   & ab
        \end{pmatrix}
      \]
      for all~$a, b \in A$.
      The map~$\Phi$ is thus multiplicaitve if and only if~$\Delta(ab) = \Delta(a) b + a \Delta(b)$ for all~$a, b \in A$.
      If~$\Delta$ is a derivation then we also have~$\Delta(1) = 0$ and thus~$\Phi(1) = 1$.
      This shows altogether that~$\Delta$ is a derivation if and only if~$\Phi$ is a homomorphism of algebras.
    \item
      We have
      \begin{align*}
        [ \varphi(x), \varphi(y) ]
        &=
        \Biggl[
          \begin{pmatrix}
            x & \delta(x) \\
            0 & x
          \end{pmatrix},
          \begin{pmatrix}
            y & \delta(y) \\
            0 & y
          \end{pmatrix}
        \Biggr]
        \\
        &=
        \begin{pmatrix}
          x & \delta(x) \\
          0 & x
        \end{pmatrix}
        \begin{pmatrix}
          y & \delta(y) \\
          0 & y
        \end{pmatrix}
        -
        \begin{pmatrix}
          y & \delta(y) \\
          0 & y
        \end{pmatrix}
        \begin{pmatrix}
          x & \delta(x) \\
          0 & x
        \end{pmatrix}
        \\
        &=
        \begin{pmatrix}
          xy  & x \delta(y) + \delta(x) y \\
          0   & xy
        \end{pmatrix}
        -
        \begin{pmatrix}
          yx  & y \delta(x) + \delta(y) x \\
          0   & yx
        \end{pmatrix}
        \\
        &=
        \begin{pmatrix}
          xy - yx & \delta(x) y - y \delta(x) + x \delta(y) - \delta(y) x \\
          0       & xy - yx
        \end{pmatrix}
        \\
        &=
        \begin{pmatrix}
          [x,y] & [\delta(x), y] + [x, \delta(y)] \\
          0     & [x,y]
        \end{pmatrix}
      \end{align*}
      for all~$x, y \in \glie$.
      It follows that the map~$\varphi$ is a homomorphism of Lie~algebras if and only if~$\delta([x,y]) = [\delta(x), y] + [x, \delta(y)]$ for all~$x, y \in \glie$, i.e. if and only if the map~$\delta$ is a derivation of~$\glie$.
    \qedhere
  \end{enumerate}
\end{proof}


\begin{proof}[Proof of \cref{extending derivation to universal enveloping algebra}]
  The uniqueness of~$\Delta$ follows from \cref{dervation is uniquely determined by algebra generators} because~$\glie$ generates the algebra~$\Univ(\glie)$.

  Let~$B$ be the~\algebra{$\kf$} given by
  \[
    B
    \defined
    \begin{pmatrix}
      \Univ(\glie)  & \Univ(\glie) \\
      0             & \Univ(\glie)
    \end{pmatrix} \,.
  \]
  It follows from \cref{translating between derivations and homomorphisms} that the map
  \[
    \varphi
    \colon
    \glie
    \to
    A \,,
    \quad
    x
    \mapsto
    \begin{pmatrix}
      x & \delta(x) \\
      0 & x
    \end{pmatrix}
  \]
  is a homomorphism of Lie~algebras because~$\delta$ is a derivation of~$\glie$.
  It follows from the universal property of the universal enveloping algebra~$\Univ(\glie)$ that the homomorphism of Lie~algebras~$\varphi$ extends uniquely to a homomorphism of algebras~$\Phi$ from~$\Univ(\glie)$ to~$B$.
  This homomorphism~$\Phi$ is of the form
  \[
    \Phi(x)
    =
    \begin{pmatrix}
      \Phi_1(x) & \Delta(x) \\
      0         & \Phi_2(x)
    \end{pmatrix}
  \]
  for all~$x \in \Univ(\glie)$ for some unique linear mape
  \[
    \Phi_1, \Phi_2, \Delta
    \colon
    \Univ(\glie)
    \to
    \Univ(\glie) \,.
  \]
  The maps~$\Phi_1$ and~$\Phi_2$ are homomorphisms of algebras because~$\Phi$ is a homomorphism of algebras.
  They satisfy the equalities~$\Phi_1(x) = \varphi(x) = x$ and~$\Phi_2(x) = \varphi(x) = x$ for all~$x \in \glie$.
  It follows that~$\Phi_1(x) = x$ and~~$\Phi_2(x) = x$ for all~$x \in \Univ(\glie)$ because~$\glie$ generates~$\Univ(\glie)$ as an algebra.
  The homomorphism~$\Phi$ is thus of the form
  \[
    \Phi(x)
    =
    \begin{pmatrix}
      x & \Delta(x) \\
      0 & x
    \end{pmatrix}
  \]
  for all~$x \in \glie$.
  It follows from \cref{translating between derivations and homomorphisms} that the linear map~$\Delta$ is an algebra derivation of~$\Univ(\glie)$.
  This darivation satisfies the equality~$\Delta(x) = \delta(x)$ for all~$x \in \glie$ because~$\Phi$ is an extension of~$\varphi$.
  In other words,~$\Delta$ is an extension of~$\delta$.
\end{proof}


