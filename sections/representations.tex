\chapter{Representations of Lie~Algebras}


% TODO: Tensor hom adjunction for internal hom.





\section{Definition and Examples}


\begin{definition}
  Let~$\glie$ be a~{\liealgebra{$\kf$}}.
  \begin{enumerate}
    \item
      A \defemph{representation}\index{representation}~$(V, \rho)$ of~$\glie$ is a~{\vectorspace{$\kf$}}~$V$ together with a homomorphism of Lie~algebras~$\rho \colon \glie \to \gllie(V)$.
    \item
      The \defemph{dimension}\index{representation!dimension}\index{dimension of a representation} of a representation~$(V, \rho)$ is the dimension of its underlying vector space~$V$.
    \item
      A representation~$(V, \rho)$ is~\defemph{faithful}\index{faithful}\index{representation!faithful} if the homomorphism~$\rho$ is injective.
    \item
      An \defemph{action} of~$\glie$ on a vector space~$V$ is a bilinear map
      \[
        \glie \times V \to V \,,
        \quad
        (x,v) \mapsto x \act v
      \]
      such that
      \[
        x \act (y \act v) - y \act (x \act v)
        =
        [x,y]
        \qquad
        \text{for all~$x, y \in \glie$,~$v \in V$.}
      \]
  \end{enumerate}
\end{definition}


\begin{proposition}
  \label{correspodence between representations and actions}
  Let~$\glie$ be a~\liealgebra{$\kf$} and let~$V$ be a~\vectorspace{$\kf$}.
  \begin{enumerate}
    \item
      If~$\rho \colon \glie \to \gllie(V)$ is a homomorphism of Lie~algebras then the formula
      \[
        x \act v \defined \rho(x)(v)
        \qquad
        \text{for all~$x \in \glie$,~$v \in V$}
      \]
      defines an action of~$\glie$ on~$V$.
    \item
      Suppose on the other hand that~$\glie$ acts on~$V$.
      Then for every element~$x$ of~$\glie$ the map
      \[
        \rho(x) \colon V \to V \,,
        \quad
        v \mapsto x \act v
      \]
      is~\linear{$\kf$}, and the resulting map
      \[
        \rho \colon \glie \to \gllie(V)
      \]
      is a homomorphism of Lie algebras.
    \item
      The above two constructions are mutually inverse, and result in a {\onetoonetext} correspondence given by
      \[
        \left\{
          \begin{tabular}{@{}c@{}}
            homomorphism of
            \\
            Lie~algebras~$\glie \to \gllie(V)$
          \end{tabular}
        \right\}
        \onetoone
        \{
          \text{actions of~$\glie$ on~$V$}
        \} \,.
      \]
  \end{enumerate}
\end{proposition}


\begin{proof}
  We consider a map
  \[
    \alpha
    \colon
    \glie \times V
    \to
    V \,,
    \quad
    (x,v)
    \mapsto
    x \act v
  \]
  and the corresponding map
  \[
    \rho
    \colon
    \glie
    \to
    \Maps(V,V)
  \]
  given by
  \begin{equation}
    \label{action and homomorphism formula}
    \rho(x)(v) = x \act v
    \qquad
    \text{for all~$x \in \glie$,~$v \in V$.}
  \end{equation}
  The linearity of~$\alpha$ in the second argument is equivalent to~$\rho$ taking values in~$\gllie(V)$.
  The additional linearity of~$\alpha$ in the first argument is equivalent to~$\rho$ being linear.
  That~$\rho$ is a homomorphism of Lie~algebras (when corestricted to a map into~$\gllie(V)$) is equivalent to the condition
  \[
    x \act (y \act v) - y \act (x \act v)
    =
    [x,y] \act v
    \qquad
    \text{for all~$x, y \in \glie$,~$v \in V$.}
  \]
  This shows overall that~$\alpha$ is an action of~$\glie$ on~$V$ if and only if the map~$\rho$ gives a homomorphism of Lie~algebras~$\glie \to \gllie(V)$ (by corestriction).
  That both constructions are mutually inverse follows from the formula~\eqref{action and homomorphism formula}.
\end{proof}


\begin{fluff}
  Let~$\glie$ be a~\liealgebra{$\kf$}.
  According to \cref{correspodence between representations and actions} we can equivalently characterize a representation of~$\glie$ as a~\vectorspace{$\kf$} together with an action of~$\glie$ on~$V$.
  In the following we will most often work with this alternative characterization of representations via actions, but we will feel free to use the original definition whenever it is useful to do so.
\end{fluff}


%\begin{remark}
%  If~$(x_i)_{i \in I}$ is a basis of a Lie~algebra~$\glie$ then for a~{\linear{$\kf$}}~$\rho \colon \glie \to \gllie(V)$ to be a homomorphism of Lie~algebras it sufficies that~$\rho([x_i,x_j])= [\rho(x_i), \rho(x_j)]$ for all~$i, j \in I$.
%  It therefore sufficies to check condition~\eqref{representation via action} for basis elements, i.e.\ it sufficies to check that
%  \[
%    x_i.(x_j.v) - x_j.(x_i.v)
%    = 
%    [x_i, x_j].v
%  \]
%  for all~$i, j \in I$ and~$v \in V$.
%\end{remark}


\begin{fluff}
  Ado’s theorem can be reformulated in the language of representations.
\end{fluff}


\begin{theorem}[Ado, second version]
  \index{Ado’s theorem}
  Every finite-dimensional Lie~algebra admits a finite-dimensional faithful representation.
\end{theorem}


\begin{examples}
  \label{examples for representations}
  \leavevmode
  \begin{enumerate}
    \item 
      Let~$\glie$ be a Lie~subalgebra of~$\gllie(V)$ for some vector space~$V$.
      Then the inclusion~$\glie \to \gllie(V)$ is a homomorphism of Lie~algebras, which makes~$V$ into a faithful representation of~$\glie$.
      The corresponding action of~$\glie$ on~$V$ is given by
      \[
        x \act v
        =
        x(v)
        \qquad
        \text{for all~$x \in \glie$,~$v \in V$.}
      \]
      This representation is the \defemph{natural representation}\index{natural representation}\index{representation!natural} of~$\glie$.
    \item
      Let~$\glie$ be a Lie~subalgebra of~$\gllie(n, \kf)$ for natural number~$n$.
      Then the inclusion
      \[
        \glie \to \gllie(n, \kf) \cong \gllie(\kf^n)
      \]
      is a homomorphism of Lie~algebras, which makes~$\kf^n$ into a faithful representation of~$\gllie$.
      The corresponding action of~$\glie$ on~$\kf^n$ is given by
      \[
        x \act v
        =
        x \cdot v
        \qquad
        \text{for all~$x \in \glie$,~$v \in \kf^n$.}
      \]
      This representation is the the \defemph{natural representation}\index{natural representation}\index{representation!natural} of~$\glie$.
    \item
      Let~$V \defined \kf[x_1, \dotsc, x_n]$ and let~$p_1, \dotsc, p_n, q_1, \dotsc, q_n, c$ be the endomorphisms of~$V$ given as follows:
      $p_i \defined \dd{x_i}$ is the~{\howmanyth{$i$}} partial derivate,~$q_i$ is the multiplication with the variable~$x_i$, and~$c \defined \id_V$ is the identiy function.
      These endomorphisms are linearly independent and satisfy the relations
      \begin{itemize}
        \item
          $[p_i, p_j] = 0$ and $[q_i, q_j] = 0$ for all~$i, j = 1, \dotsc, n$,
        \item
          $[p_i, c] = 0$ and~$[q_i, 0] = 0$ for every~$i = 1, \dotsc, n$,
        \item
          $[p_i, q_j] = \delta_{ij} c$ for all~$i, j = 1, \dotsc, n$.
      \end{itemize}
      These relations are that of the Heisenberg Lie~algebra~$\Hlie$ from \cref{examples for lie algebras}.
      The unique linear map
      \[
        \rho
        \colon
        \Hlie
        \to
        \gllie(V)
      \]
      that is given on the basis~$P_1, \dotsc, P_n, Q_1, \dotsc, Q_n, C$ of~$\Hlie$ by
      \[
        \rho(P_i) \defined p_i \,,
        \quad
        \rho(Q_i) \defined q_i \,,
        \quad
        \rho(C) \defined c
      \]
      is therefore a homomorphism of Lie~algebras.
      We have thus constructed a representation of~$\Hlie$ on~$V = \kf[x_1, \dotsc, x_n]$.
      Moreover, the elements~$p_1, \dotsc, p_n, q_1, \dotsc, q_n, c$ of~$\gllie(V)$ are linearly independent, so~$\rho$ is injective.
      The constructed representation is therefore faifthul.

      This also shows that~$\Hlie$ is indeed a Lie~algebra, as it can be realized as a Lie~subalgebra of~$\gllie(V)$.
    \item
      Let~$\glie \defined \sllie(2, \kf)$ for any field~$\kf$.
      Then the polynomial ring~$\kf[x,y]$ becomes a representation of~$\glie$ via the homomorphism of Lie~algebras~$\rho \colon \glie \to \gllie(\kf[x,y])$ given by
      \[
        \rho(e) = y \dd{x} \,,
        \qquad
        \rho(h) = y \dd{y} - x \dd{x} \,,
        \qquad
        \rho(f) = x \dd{y}  \,.
      \]
      Here~$e$,~$h$,~$f$ denotes the standard basis of~$\sllie(2, \kf)$.
      The corresponding action of~$\sllie_2(\kf)$ on~$\kf[x,y]$ is given by
      \begin{align*}
        \SwapAboveDisplaySkip
        e \act (x^n y^m)
        &=
        n x^{n-1} y^{m+1} \,,
        \\
        h \act (x^n y^m)
        &=
        (m-n) x^n y^m \,,
        \\
        f \act (x^n y^m)
        &=
        m x^{n+1} y^{m-1}
      \end{align*}
      for all~$n, m \geq 0$.
      This is indeed an action of~$\sllie(2, \kf)$ on~$\kf[x]$ because
      \begin{gather*}
        \begin{aligned}
        e \act f \act (x^n y^m) - f \act e \act (x^n y^m)
        &=
        (n+1)m x^n y^m - n(m+1) x^n y^m
        \\
        &=
        (m-n) x^n y^m
        \\
        &= h \act m
        = [e,f] \act (x^n y^m) \,,
        \end{aligned}
      \shortintertext{as well as}
        \begin{aligned}
        h \act e \act (x^n y^m) - e \act h \act (x^n y^m)
        &=
        n(m-n+2) x^{n-1} y^{m+1} - n(m-n) x^{n-1} y^{m+1}
        \\
        &=
        2 x^{n-1} y^{m+1}
        \\
        &=
        2e \act (x^n y^m)
        \\
        &=
        [h,e] \act (x^n y^m) \,,
        \end{aligned}
      \shortintertext{and}
        \begin{aligned}
          h \act f \act (x^n y^m) - f \act h \act (x^n y^m)
          &=
          m(m-n-2) x^{n+1} y^{m-1} - m(m-n) x^{n+1} y^{m-1}
          \\
          &=
          -2 x^{n+1} y^{m-1}
          \\
          &=
          -2 f \act (x^n y^{m-1})
          \\
          &=
          [h,f] \act (x^n y^{m-1})
        \end{aligned}
      \end{gather*}
      for all~$n, m \geq 0$.
    \item
      Then polynomial ring in one variable~$\kf[x]$ is a representation of~$\sllie(2, \kf)$ via the homomorphism of Lie~algebras~$\rho \colon \sllie(2, \kf) \to \gllie(\kf[x])$ given by
      \[
        \rho(e)
        =
        \dd{x} \,,
        \qquad
        \rho(h)
        =
        -2x\dd{x} \,,
        \qquad
        \rho(f)
        =
        -\dd{x} \,,
      \]
      on the standard basis~$e$,~$h$,~$f$ of~$\sllie(2, \kf)$.
      The correpsonding action of~$\sllie(2, \kf)$ on~$\kf[x]$ is given by
      \[
        e \act x^n = n x^{n-1} \,,
        \qquad
        h \act x^n = -2n x^n \,,
        \qquad
        f \act x^n = n x^{n+1}
      \]
      for every~$n \geq 0$.
      This is indeed an action of~$\sllie(2, \kf)$ on~$\kf[x]$ because
      \begin{gather*}
        e \act f \act x^n - f \act e \act x^n
        = -n(n+1) x^n + n(n-1) x^n
        = -2n x^n
        = h \act x^n
        = [e,f] \act x^n \,,
      \intertext{as well as}
        h \act e \act x^n - e \act h \act x^n
        = -2n(n-1) x^{n-1} + 2n^2 x^{n-1}
        = 2n x^{n-1}
        = 2e \act x^n
        = [h,e] \act x^n \,,
      \shortintertext{and}
        h \act f \act x^n - f \act h \act x^n
        = 2n(n+1) x^{n+1} - 2 n^2 x^{n+1}
        = 2n x^{n+1}
        = -2 f \act x^n
        = [h,f] \act x^n
      \end{gather*}
      for every~$n \geq 0$.
    \item
      If~$\rho \colon \glie \to \gllie(V)$ is a representation of a Lie~algebra~$\glie$ and~$\phi \colon \hlie \to \glie$ a homomorphism of Lie~algebras then via the composition~$\rho \circ \phi \colon \hlie \to \gllie(V)$ the vector space~$V$ becomes a representation of~$\hlie$.
      This representation comes from the action of~$\hlie$ on~$V$ given by
      \[
        x \act v = \phi(x).v
        \qquad
        \text{for all~$x \in \hlie$,~$v \in V$,}
      \]
      where the right hand side denotes the action of~$\glie$ on~$V$ corresponding to the representation~$\rho$.
    \item
      Let~$A$ be a~\algebra{$\kf$} and let~$V$ be a left~\module{$A$}.
      Then~$V$ is also a representation of~$A$ as a Lie~algebra because
      \[
        [a,b] \act v
        =
        (ab - ba) \act v
        =
        (ab - ba)v
        =
        abv - bav
        =
        a \act (b \act v) - b \act (a \act v)
      \]
      for all elements~$a$,~$b$ of~$A$ and every element~$v$ of~$V$.
    \item
      The map~$\ad \colon \glie \to \gllie(\glie)$ is a homomorphism of Lie~algebras and hence a representation of~$\glie$.
  \end{enumerate}
\end{examples}


\begin{definition}
  Let~$\glie$ be a Lie~algebra.
  The  Lie~algebra homomorphism
  \[
    \ad
    \colon
    \glie
    \to
    \gllie(\glie) \,,
    \quad
    x
    \mapsto
    \ad(x)
    =
    [x,-]
  \]
  is the \defemph{adjoint representation}\index{adjoint representation} of~$\glie$.
\end{definition}


\begin{remark}
  It follows together with \cref{lie algebras act adjoint by derivations} that every Lie~algebras~$\glie$ acts on itself via derivations of itself through the adjoint representation.
  The author suspects that this is where much of the structure of Lie~algebras comes from, and why the Jacobi identity is of interest.
\end{remark}


\begin{remark}
  \label{right representations}
  Let~$\glie$ be a~{\liealgebra{$\kf$}} and let~$V$ be a~\vectorspace{$\kf$}.
  A \defemph{right action}\index{action!right}\index{right action} of~$\glie$ on~$V$ is a~{\bilinear{$\kf$}} map
  \[
    V \times \glie
    \to
    V \,,
    \quad
    (v,x)
    \mapsto
    v \act x
  \]
  such that
  \begin{equation}
    \label{right action}
    (v \act x) \act y - (v \act y) \act x
    =
    v \act [x,y]
    \qquad
    \text{for all~$x, y \in \glie$,~$v \in V$.}
  \end{equation}
  The actions that we have considered so far are \defemph{left actions}\index{action!left}\index{left action}.
  It turns out that both concepts are equivalent, as we will now explain.

  We first note that \cref{correspodence between representations and actions} can be generalized to right actions:
  a map
  \[
    V \times \glie
    \to
    V \,,
    \quad
    (v,x)
    \mapsto
    v \act x
  \]
  is a right action if and only if the map
  \[
    \rho
    \colon
    \glie
    \to
    \gllie(V)
  \]
  given by
  \[
    \rho(x)(v)
    \defined
    v \act x
    \qquad
    \text{for all~$x \in \glie$,~$v \in V$}
  \]
  is an anti-homomorphism of Lie~algebras.
  This means that~$\rho$ is a homomorphism of Lie~algebras from~$\glie^{\op}$ to~$\gllie(V)$.
  Such a homomorphism does in turn correspond to a left action of~$\glie^{\op}$ on~$V$ via
  \[
    x^{\op} \act v
    \defined
    v \act x
    \qquad
    \text{for all~$x \in \glie$,~$v \in V$.}
  \]
  We have thus found {\onetoonetext} correspondences between the following concepts.
  \begin{equivalenceslist*}
    \item
      Right actions of~$\glie$ on~$V$.
    \item
      Left actions of~$\glie^{\op}$ on~$V$.
    \item
      Anti-homomorphisms of Lie~algebras from~$\glie$ to~$\gllie(V)$.
    \item
      Homomorphisms of Lie~algebras from~$\glie^{\op}$ to~$\gllie(V)$.
  \end{equivalenceslist*}
  We now recall \cref{lie algebra isomorphic to its opposite}, which states that the map
  \[
    \glie
    \to
    \glie^{\op} \,,
    \quad
    x
    \mapsto
    -x^{\op}
  \]
  is an isomorphism of Lie~algebras.
  By using this isomorphism we can now translate between right actions of~$\glie$ on~$V$, left actions of~$\glie^{\op}$ on~$V$, and then left actions of~$\glie$ on~$V$.

  More explicitely, if~$\glie$ acts on the vector space~$V$ from the right via
  \[
    V \times \glie
    \to
    V \,,
    \quad
    (v,x)
    \mapsto
    v \act x
    \qquad
    \text{for all~$x \in \glie$,~$v \in V$,}
  \]
  then~$\glie$ acts on~$V$ from the left via
  \[
    x \act v
    \defined
    -v \act x
    \qquad
    \text{for all~$x \in \glie$,~$v \in V$.}
  \]

  In the following we will sometimes encounter constructions which turn left actions into right actions.
  We are now able to reinterpret these right actions as left actions.
  In this way we only have to deal with left actions.
\end{remark}





\section{New Representations from Old Ones}


\begin{proposition}
  \label{new representations from old ones}
  Let~$\glie$ be a Lie~algebra over an arbitrary field~$\kf$.
  \begin{enumerate}
    \item
      Let~$V_\lambda$ with~$\lambda \in \Lambda$ be a collection of representations of~$\glie$.
      Then the product~$\prod_{\lambda \in \Lambda} V_\lambda$ becomes again a representation of~$\glie$ via the action given by
      \[
        x \act (v_\lambda)_\lambda
        \defined
        ( x \act v_\lambda )_\lambda
        \qquad
        \text{for all~$x \in \glie$,~$(v_\lambda)_\lambda \in \prod_{\lambda \in \Lambda} V_\lambda$.}
      \]
    \item
      Let~$V_\lambda$ with~$\lambda \in \Lambda$ be a collection of representations of~$\glie$.
      Then the direct sum~$\bigoplus_{\lambda \in \Lambda} V_\lambda$ becomes again a representation of~$\glie$ via the action given by
      \[
        x \act (v_\lambda)_\lambda
        \defined
        ( x \act v_\lambda )_\lambda
        \qquad
        \text{for all~$x \in \glie$,~$(v_\lambda)_\lambda \in \bigoplus_{\lambda \in \Lambda} V_\lambda$.}
      \]
    \item
      If~$V$ and~$W$ are two representations of~$\glie$ then their tensor product~$V \tensor W$ becomes again a representation of~$\glie$ via the action
      \[
        x \act (v \tensor w)
        \defined
        (x \act v) \tensor w + v \tensor (x \act w)
        \qquad
        \text{for all~$x \in \glie$,~$v \in V$,~$w \in W$.}
      \]
      Let more generall~$V_1, \dotsc, V_n$ be a finite collection of representations of~$\glie$.
      Then their tensor product~$V_1 \tensor \dotsb \tensor V_n$ becomes again a representation of~$\glie$ via the action
      \[
        x \act (v_1 \tensor \dotsb \tensor v_n)
        \defined
        \sum_{i=1}^n
                v_1
        \tensor \dotsb
        \tensor v_{i-1}
        \tensor (x.v_i)
        \tensor v_{i+1}
        \tensor \dotsb
        \tensor v_n
      \]
      for every~$x \in \glie$ and all~$v_i \in V_i$ for~$i = 1, \dotsc, n$.
    \item
      Let~$V$ and~$W$ be two representations of~$\glie$.
      The vector space~$\Hom_{\kf}(V,W)$ becomes again a representation of~$\glie$ via the action
      \[
        (x \act f)(v)
        \defined
        x \act f(v) - f(x \act v)
      \]
      for all~$x \in \glie$,~$f \in \Hom_{\kf}(V,W)$ and~$v \in V$.
    \item
      Every~{\vectorspace{$\kf$}}~$V$ becomes a representation of~$\glie$ via the \defemph{trivial action}\index{trivial!action} given by
      \[
        x \act v
        \defined
        0
        \qquad
        \text{for all~$x \in \glie$ and~$v \in V$.}
      \]
    \item
      Let~$V$ be a representation of~$\glie$.
      The dual space~$V^*$ becomes again a representation of~$\glie$ via the action
      \[
        (x \act \varphi)(v)
        \defined
        - \varphi(x \act v)
        \qquad
        \text{for all~$x \in \glie$,~$\varphi \in V^*$,~$v \in V$.}
      \]
  \end{enumerate}
\end{proposition}


\begin{proof}
  \leavevmode
  \begin{enumerate}
    \item
      We have for all elements~$x$,~$y$ of~$\glie$ and every element~$(v_\lambda)_\lambda$ of~$\prod_{\lambda \in \Lambda} V_\lambda$ that
      \begin{align*}
        x \act y \act (v_\lambda)_\lambda
        - y \act x \act (v_\lambda)_\lambda
        &=
        (x \act y \act v_\lambda)_\lambda
        - (y \act x \act v_\lambda)_\lambda
        \\
        &=
        (x \act y \act v_\lambda - y \act x \act v_\lambda)_\lambda
        \\
        &=
        ([x,y] \act v_\lambda)_\lambda
        \\
        &=
        [x,y] \act (v_\lambda)_\lambda \,.
      \end{align*}
    \item
      That the given formula gives an action of~$\glie$ on the direct sum~$\bigoplus_{\lambda \in \Lambda} V_\lambda$ follows from the same calculation as for the product~$\prod_{\lambda \in \Lambda} V_\lambda$.
    \item
      It sufficies by induction to consider the case~$n = 2$, i.e.\ the tensor product~$V \tensor W$.
      We find in this case for every element~$x$ of~$\glie$, every element~$v$ of~$V$ and every element~$w$ of~$W$ that
      \begin{align*}
        x \act y \act (v \tensor w)
        &=
        x \act ( (y \act v) \tensor w
        + v \tensor (y \act w) )
        \\
        &=
        x \act ( (y \act v) \tensor w)
        + x \act (v \tensor (y \act w) )
        \\
        &=
        (x \act y \act v) \tensor w
        + (y \act v) \tensor (x \act w)
        + (x \act v) \tensor (y \act w)
        + v \tensor (x \act y \act w)
      \end{align*}
      and therefore
      \begin{align*}
        {}&
        x \act y \act (v \tensor w)
        - y \act x \act (v \tensor w)
        \\
        ={}&
        (x \act y \act v) \tensor w
        + v \tensor (x \act y \act w)
        - (y \act x \act v) \tensor w
        - v \tensor (y \act x \act w)
        \\
        ={}&
        (x \act y \act v - y \act x \act v) \tensor w
        + v \tensor (x \act y \act w - y \act x \act w)
        \\
        ={}&
        ([x,y] \act v) \tensor w
        + v \tensor ([x,y] \act w) \,.
      \end{align*}
    \item
      We find all elements~$x$,~$y$ of~$\glie$, every element~$f$ of~$\Hom(V,W)$ and every vector~$v$ of~$V$ that
      \begin{align*}
        (x \act y \act \phi - y \act x \act \phi)(v)
        &=
        (x \act y \act \phi)(v) - (y \act x \act \phi)(v)
        \\
        &=
        -(y \act \phi)(x.v) + (x \act \phi(y.v)
        \\
        &=
        \phi(y \act x \act v) - \phi(x \act y \act v)
        \\
        &=
        -\phi(x \act y \act v - y \act x \act v)
        \\
        &=
        -\varphi([x,y] \act v)
        \\
        &=
        ([x.y] \act \varphi)(v)  \,,
      \end{align*}
      and therefore
      \[
        x \act y \act \phi - y \act x \act \phi
        =
        [x,y] \act \phi \,.
      \]
    \item
      It holds for all element~$x$,~$y$ of~$\glie$ and every vector~$v$ of~$V$ that
      \[
        x \act y \act v - y \act x \act v
        =
        0 - 0
        =
        0
        =
        [x,y] \act v \,.
      \]
    \item
      The action of~$\glie$ on~$V$ and the trivial action of~$\glie$ on~$\kf$ induce an action of~$\glie$ on~$V^* = \Hom(V, \kf)$, making it into a representation of~$\glie$.
      This action is precisely the given one.
    \qedhere
  \end{enumerate}
\end{proof}


\begin{remark}
  Let~$V$ be a representation of~$\glie$.
  The action of~$\glie$ on the dual space~$V^*$ can also be explained in a more systematic way, as we will now explain.

  Let~$\rho \colon \glie \to \gllie(V)$ be the homomorphism of Lie~algebras corresponding to the given representation.
  The algebra anti-homomorphism
  \[
    \End_{\kf}(V)
    \to
    \End_{\kf}(V^*) \,,
    \quad
    \phi
    \mapsto
    \phi^*
  \]
  is in particular an anti-homomorphism of Lie~algebras from~$\gllie(V)$ to~$\gllie(V^*)$.
  The resulting composition
  \[
    \glie
    \xto{\rho}
    \gllie(V)
    \to
    \gllie(V^*)
  \]
  is again an anti-homorphism of Lie~algebras.
  This anti-homomorphism corresponds to a a right action of~$\glie$ on~$V^*$ given by
  \[
    \phi \act x
    \defined
    \rho(x)^*(\phi)
    =
    \phi \circ \rho(x)
    \qquad
    \text{for all~$x \in \glie$,~$\phi \in V^*$.}
  \]
  This action can more explicitely be rewritten as
  \[
    (\phi \act x)(v)
    =
    \phi(x \act v)
    \qquad
    \text{for all~$x \in \glie$,~$\phi \in V^*$,~$v \in V$.}
  \]
  This right action of~$\glie$ on~$V^*$ corresponds to a left action of~$\glie$ on~$V$ via the formula
  \[
    x \act \phi
    \defined
    -\phi \act x
    \qquad
    \text{for all~$x \in \glie$,~$\phi \in V^*$.}
  \]
  This left action can more explicitely be rewritten as
  \[
    (x \act \phi)(v)
    =
    -\phi(x \act v)
    \qquad
    \text{for all~$x \in \glie$,~$\phi \in V^*$,~$v \in V$.}
  \]
  This is precisely the action of~$\glie$ on~$V^*$ which we have seen in \cref{new representations from old ones}.
\end{remark}


\begin{definition}
  Let~$V$ be a representation of a Lie~algebra~$\glie$.
  A \defemph{subrepresentation} of~$V$ is a linear subpace~$U$ of~$V$ such that
  \[
    x \act u \in U
    \qquad
    \text{for all~$x \in \glie$,~$u \in U$.}
  \]
\end{definition}


\begin{remark}
  Let~$(V, \rho)$ be a representation of a Lie~algebra~$\glie$.
  A linear subspace~$U$ of~$V$ is a subrepresentation if and only if~$U$ is~{\invariant{$\rho(x)$}} for every~$x \in \glie$, in the sense that~$\rho(x)(U) \subseteq U$.
\end{remark}


\begin{examples}
  Let~$\glie$ be a Lie~algebra.
  \begin{enumerate}
    \item
      Let~$V$ be a representation of~$\glie$.
      The linear subspaces~$0$ and~$V$ are subrepresentations of~$V$.%
    \footnote{
      These two subrepresentations are often called the \defemph{trivial} ones.
      We will abstain from doing so, as we have already defined the notion of a trivial representation in \cref{trivial representations}.
    }
    \item
      Let~$V$ be a representation of~$\glie$.
      Let~$U_\lambda$ with~$\lambda \in \lambda$ be a collection of subrepresentations~$U_\lambda$ of~$V$.
      Then both the intersection~$\bigcap_{\lambda \in \Lambda} U_\lambda$ and the sum~$\sum_{\lambda \in \Lambda} U_\lambda$ are again subrepresentations of~$V$.
    \item
      Let~$V_\lambda$ with~$\lambda \in \Lambda$ be a collection of representations of~$\glie$.
      The direct sum~$\bigoplus_{\lambda \in \Lambda} V_\lambda$ is a subrepresentation of the product~$\prod_{\lambda \in \Lambda} V_\lambda$.
    \item
      The subrepresentations of the adjoint representation of~$\glie$ are precisely the ideals of~$\glie$.
    \item
      Let~$V$ be any representation of~$\glie$.
      The the linear subspace~$\glie V$ of~$V$ given by
      \[
        \glie V
        \defined
        \gen{
          x \act v
          \suchthat
          x \in \glie,
          v \in V
        }_{\kf}
      \]
      is a subrepresentation of~$\glie$.
  \end{enumerate}
\end{examples}


\begin{definition}
  \label{trivial representations}
  Let~$V$ be a representation of a Lie~algebra~$\glie$.
  \begin{enumerate}
    \item
      The representation~$V$ is \defemph{trivial}\index{trivial!representation} if every element~$x$ of~$\glie$ acts by multiplication with zero on~$V$.
    \item
      An element~$v$ of~$V$ is \defemph{\invariant{$\glie$}}\index{invariants}, or simply~\defemph{invariant}, if
      \[
        x \act v = 0
        \qquad
        \text{for every~$x \in \glie$.}
      \]
      The set of invariants of~$V$ is denoted by
      \[
        V^{\glie}\glsadd{invariants}
        \defined
        \{
          v \in V
        \suchthat
          \text{$x \act v = 0$ for every~$x \in \glie$}
        \}  \,.
      \]
  \end{enumerate}  
\end{definition}


% TODO: Explain connection to representations of groups.


\begin{lemma}
  The space of invariants~$V^{\glie}$ is for every representation~$V$ of~$\glie$ the uniquely maximal trivial subrepresentation of~$V$.
  \qed
\end{lemma}


\begin{example}
  Let~$\glie$ be a Lie~algebra.
  The space of invariants of the adjoint representation of~$\glie$ is given by
  \begin{align*}
    \glie^{\glie}
    &=
    \{
      y \in \glie
    \suchthat
      \text{$x \act y = 0$ for every~$x \in \glie$}
    \}
    \\
    &=
    \{
      y \in \glie
    \suchthat
      \text{$\ad(x)(y) = 0$ for every~$x \in \glie$}
    \}
    \\
    &=
    \{
      y \in \glie
    \suchthat
      \text{$[x,y] = 0$ for every~$x \in \glie$}
    \}
    \\
    &=
    \centerlie(\glie) \,.
  \end{align*}
\end{example}


\begin{example}[Quotient representations]
  \label{quotient representation}
  Let~$V$ be a representation of a Lie~algebra~$\glie$ and let~$U$ be a subrepresentation of~$V$.
  The quotient vector space~$V/U$ inherits from~$V$ the structure of a~{\representation{$\glie$}} via the action of~$\glie$ on~$V/U$ given by
  \[
    x \act \class{v}
    \defined
    \class{x \act v}
    \qquad
    \text{for all~$x \in \glie$,~$v \in V$}.
  \]
  Indeed, we have for any two element~$x$,~$y$ of~$\glie$ every vector~$\class{v}$ of~$V$ that
  \[
    [x,y] \act \class{v}
    =
    \class{[x,y] \act v}
    =
    \class{x \act (y \act v) - y \act (x \act v)}
    =
    \class{x \act (y \act v)} - \class{y \act (x \act v)}
    =
    x \act (y \act \class{v}) - y \act (x \act \class{v}) \,.
  \]
  
  Alternatively, let~$\rho \colon \glie \to \gllie(V)$ be the Lie~algebra homomorphism corresponding to the action of~$\glie$ on~$V$.
  Then the linear subspace~$U$ of~$V$ is~{\invariant{$\rho(x)$}} for every~$x \in \glie$, and hence the endomorphism~$\rho(x)$ induces an endomorphism
  \[
    \induced{\rho(x)}
    \colon
    V/U
    \to
    V/U \,,
    \quad
    \class{v}
    \mapsto
    \class{\rho(x)(v)}
  \]
  for every element~$x$ of~$\glie$.
  The resulting map
  \[
    \induced{\rho}
    \colon
    \glie
    \to
    \gllie(V/U) \,,
    \quad
    x
    \mapsto
    \induced{\rho(x)}
  \]
  given is again a homomorphism of Lie~algebras because
  \[
    \induced{\rho}([x,y])
    =
    \induced{\rho([x,y])}
    =
    \induced{[\rho(x), \rho(y)]}
    =
    [\induced{\rho(x)}, \induced{\rho(y)}]
    =
    [\induced{\rho}(x), \induced{\rho}(y)]
  \]
  for all~$x, y \in \glie$.
\end{example}


\begin{definition}
  Let~$V$ be a representation of a Lie~algebra~$\glie$ and let~$U$ be a subrepresentation of~$V$.
  The representation~$V/U$\glsadd{quotient representation} from \cref{quotient representation} is the \defemph{quotient representation}\index{quotient!representations}\index{representation!quotient} of~$V$ by~$U$.
\end{definition}


\begin{lemma}
  Let~$\glie$ be a Lie~algebra, let~$V$ be a representation of~$\glie$, and let~$d$ be a natural number.
  \begin{enumerate}
    \item
      The exterior power~$\Exterior^d(V)$ becomes a representation of~$\glie$ via the action
      \[
        x \act (v_1 \wedge \dotsb \wedge v_d)
        \defined
        \sum_{i=1}^d
        v_1 \wedge \dotsb \wedge v_{i-1} \wedge (x \act v_i) \wedge v_{i+1} \wedge \dotsb \wedge v_d
      \]
      for every element~$x$ of~$\glie$ and all vectors~$v_1, \dotsc, v_n$ of~$V$.
    \item
      The symmetric power~$\Symm^d(V)$ becomes a representation of~$\glie$ via the action
      \[
        x \act (v_1 \dotsm v_d)
        \defined
        \sum_{i=1}^d
        v_1 \dotsm v_{i-1} (x \act v_i) v_{i+1} \dotsm v_d
      \]
      for every element~$x$ of~$\glie$ and all vectors~$v_1, \dotsc, v_n$ of~$V$.
  \end{enumerate}
\end{lemma}


\begin{proof}
  \leavevmode
  \begin{enumerate}
    \item
      The exterior power~$\Exterior^d(V)$ is given by~$\Exterior^d(V) = V^{\tensor d} / U$ where the linear suspace~$U$ of~$V^{\tensor d}$ is given by
      \[
        U
        =
        \gen{
          v_1 \tensor \dotsb \tensor v_d
        \suchthat
          \text{$v_1, \dotsc, v_d \in V$,~$v_i = v_j$ for some~$i \neq j$}
        }_{\kf} \,.
      \]
      It sufficies to show that~$U$ is a subrepresentation of~$V^{\tensor d}$.
      To see this let~$v_1 \tensor \dotsb \tensor v_d$ be a simple tensor in~$V^{\otimes d}$ with~$v_i = v_j$ for some~$i < j$.
      It holds for every element~$x$ of~$\glie$ that
      \[
        x \act (v_1 \tensor \dotsb \tensor v_d)
        =
        \sum_{k=1}^d v_1
        \tensor \dotsb \tensor v_{k-1}
        \tensor (x \act v_k)
        \tensor v_{k+1} \tensor \dotsb \tensor v_d  \,.
      \]
      The summands for~$k \neq i,j$ are again contained in~$U$ as the tensor factors in the~{\howmanyth{$i$}} and~{\howmanyth{$j$}} place remain unchanged.
      With the vector~$w \defined v_i = v_j$ the two remaining summands may be rewritten as
      \begin{align*}
        {}&
          v_1 \tensor \dotsb
          \tensor (x \act w)
          \tensor \dotsb
          \tensor w
          \tensor \dotsb \tensor v_d
        \\
        {}&
        + v_1 \tensor \dotsb
          \tensor w
          \tensor \dotsb
          \tensor (x \act w)
          \tensor \dotsb \tensor v_d
        \\
        ={}&
          v_1 \tensor \dotsb
          \tensor (x \act w + w)
          \tensor \dotsb
          \tensor (x \act w + w)
          \tensor \dotsb \tensor v_d
        \\
        {}&
        - v_1 \tensor \dotsb
          \tensor (x \act w)
          \tensor \dotsb
          \tensor (x \act w)
          \tensor \dotsb \tensor v_d
        \\
        {}&
        - v_1 \tensor \dotsb
          \tensor w
          \tensor \dotsb
          \tensor w
          \tensor \dotsb \tensor v_d  \,.
      \end{align*}
      This term is again contained in~$U$.
      This shows altogether that~$U$ is indeed a subrepresentation of~$V^{\tensor d}$.
    \item
      Similarly to before we have to show that the linear subspace~$U$ of~$V^{\tensor d}$ given by
      \[
        U
        \defined
        \gen{
          v_1 \tensor \dotsb \tensor v_d
          - v_{\sigma(1)} \tensor \dotsb \tensor v_{\sigma(d)}
        \suchthat
          v_1, \dotsc, v_d \in V
        }_{\kf}
      \]
      is a subrepresentation of~$V^{\tensor d}$.
      We denote the permutation action of the symmetric group~$\Symm_d$ on the tensor power~$V^{\tensor d}$ from the right by
      \[
        t \act \sigma
        \qquad
        \text{for all~$\sigma \in \symm_d$,~$t \in V^{\tensor d}$.}
      \]
      This action is on simple tensors given by
      \[
        (v_1 \tensor \dotsb \tensor v_d) \act \sigma
        =
        v_{\sigma(1)} \tensor \dotsb \tensor v_{\sigma(d)}
        \qquad
        \text{for all~$\sigma \in \symm_d$,~$v_1, \dotsc, v_d \in V$.}
      \]
      We fix any element~$x$ of~$\glie$ and denote for every position~$i = 1, \dotsc, n$ by
      \[
        X_i
        \colon
        V^{\tensor d}
        \to
        V^{\tensor d}
      \]
      the linear map that is given on simple tensors by
      \[
        X_i(v_1 \tensor \dotsb \tensor v_d)
        =
        v_1 \tensor \dotsb \tensor v_{i-1}
        \tensor (x \act v_i)
        \tensor v_{i+1} \tensor \dotsb \tensor v_d
      \]
      for all vectors~$v_1, \dotsc, v_d$ of~$V$.
      
      We observe the relation
      \begin{equation}
        \label{interplay of permutation and multiplication}
        X_i( t \act \sigma )
        =
        X_{\sigma(i)}(t) \act \sigma
        \qquad
        \text{for all~$\sigma \in \symm_d$ and~$i = 1, \dotsc, n$}.
      \end{equation}
      Indeed, let us consider a simple tensor~$t = v_1 \tensor \dotsb \tensor v_n$ in~$V^{\otimes d}$, and the modified simple tensor
      \[
        t'
        \defined
        X_i( t \act \sigma) \act \sigma^{-1} \,.
      \]
      The simple tensor~$t'$ arises from the simple tensor~$t$ by first permuting the tensor factors according to~$\sigma$, then letting~$x$ act on the currently~\howmanyth{$i$} tensor factor, namely on~$v_{\sigma(i)}$, and then undoing the permutation of the tensor factors.
      We find that the simple tensors~$t$ and~$t'$ agree, except that in~$t'$ the tensor factor~$v_{\sigma(i)}$ has been acted upon by~$x$.
      In other words, we have
      \[
        t' = X_{\sigma(i)}(t) \,.
      \]
      This shows the equality~\eqref{interplay of permutation and multiplication} for simple tensors.
      It then also holds for arbitrary tensors, i.e. for elements of~$V^{\tensor d}$.
      
      
      It follows for every element~$t$ of~$V^{\tensor d}$ that
      \begin{align*}
        x \act ( t - t \act \sigma )
        &=
        x \act t - x \act (t \act \sigma)
        \\
        &=
        \sum_{i=1}^n X_i(t) - \sum_{i=1}^n X_i(t \act \sigma)
        \\
        &=
        \sum_{i=1}^n X_i(t) - \sum_{i=1}^n X_{\sigma(i)}(t) \act \sigma
        \\
        &=
        \sum_{i=1}^n X_i(t) - \sum_{i=1}^n X_i(t) \act \sigma
        \\
        &=
        \sum_{i=1}^n \biggl( X_i(t) - X_i(t) \act \sigma \biggr) \,.
      \end{align*}
      The summands~$X_i(t) - X_i(t) \act \sigma$ are again contained in~$U$.
      We have therefore shown that the linear subspace~$U$ is indeed a subrepresentation of~$V^{\tensor d}$.
    \qedhere
  \end{enumerate}
\end{proof}






\section{Homomorphisms of Representations}


\begin{definition}
  Let~$V$ und~$W$ be two representations of a~{\liealgebra{$\kf$}}~$\glie$.
  \begin{enumerate}
    \item
      A~{\linear{$\kf$}} map~$f$ from~$V$ to~$W$ is a \defemph{homomorphism of representations}\index{homomorphism!of representations} if it satisfies the condition
      \[
        f(x \act v) = x \act f(v)
        \qquad
        \text{for all~$x \in \glie$,~$v \in V$.}
      \]
    \item
      The space of homomorphism from~$V$ to~$W$ is denoted by~$\Hom_{\glie}(V,W)$\glsadd{rep homo}, and for~$V = W$ by~$\End_{\glie}(V)$\glsadd{rep endo}.
  \end{enumerate}
\end{definition}


\begin{remark}
  Let~$U$,~$V$,~$W$ be representations of a Lie~algebra~$\glie$ over~$\kf$.
  \begin{enumerate}
    \item
      Let~$\rho_V \colon \glie \to \gllie(V)$ and~$\rho_W \colon \glie \to \gllie(W)$ be the homomorphism of Lie algebras corresponding to the representations~$V$ and~$W$.
      A~{\linear{$\kf$}} map~$f$ from~$V$ to~$W$ is a homomorphism of representations if and only if
      \[
        f \circ \rho_V(x)
        =
        \rho_W(x) \circ f
        \qquad
        \text{for all~$x \in \glie$,}
      \]
      i.e.\ if and only if the square diagram
      \[
        \begin{tikzcd}[column sep = large]
          V
          \arrow{r}[above]{\rho_V(x)}
          \arrow{d}[left]{f}
          &
          V
          \arrow{d}[right]{f}
          \\
          W
          \arrow{r}[above]{\rho_W(x)}
          &
          W
        \end{tikzcd}
      \]
      commutes for every element~$x$ of~$\glie$.
    \item
      The identity~$\id_V \colon V \to V$ is a homomorphism of the representation~$V$.
    \item
      If~$f \colon U \to V$ and~$g \colon V \to W$ are two composable homomorphism of representations then their composite~$g \circ f$ is a homomorphism of representations from~$U$ to~$V$.
    \item
      Let~$f$,~$g$ be two homomorphisms of representations from~$V$ to~$W$.
      Their sum~$f + g$ is again a homomorphism of representations, and for every scalar~$\lambda$ of~$\kf$ the map~$\lambda f$ is again a homomorphism of representations.
      This means that~$\Hom_{\glie}(V,W)$ is a~{\linear{$\kf$}} subspace of~$\Hom_{\kf}(V,W)$.
    \item
      The above observations show that the representations of~$\glie$ together with the homomorphisms of representations between them form a~(\linear{$\kf$}) category.
      We will denote this category by~$\cRep{\glie}$\glsadd{representation category}.

      The class of objects of~$\cRep{\glie}$ is given by the class of representations of~$\glie$.
      The~\spaces{$\Hom$} of~$\cRep{\glie}$ are given by~$\Hom_{\cRep{\glie}}(V, W) = \Hom_{\glie}(V,W)$ for any two representations~$V$,~$W$ of~$\glie$.
      The composition of morphisms in~$\cRep{\glie}$ is the usual composition of functions.
      The identity morphism of any represenation~$V$ of~$\glie$ is the usual identity function.
    \item
      The notions of an \defemph{isomorphism}\index{isomorphism!of representations} \defemph{monomorphism}\index{monomorphism of representations}, \defemph{epimorphism}\index{epimorphism of representations}, \defemph{endomorphism}\index{endomorphism of representations} and \defemph{automorphism}\index{automorphism of representations} are defined in the usual category-theoretic way.
    \item
      If~$f$ is a bijective homomorphism of representations from~$V$ to~$W$ then its inverse~$f^{-1}$ is again a homomorphism of representations.
      Indeed, we have
      \[
        f^{-1}(x \act v)
        =
        f^{-1}(x \act f(f^{-1}(v)))
        =
        f^{-1}(f(x \act f^{-1}(v)))
        =
        x \act f^{-1}(v)
      \]
      for all~$x \in \glie$ and~$v \in V$. 
      This shows that a homomorphism of representations is an isomorphism if and only if it bijective.
    \item
      One can show in the usual way by using kernels and cokernels -- which are again representations, as we will see below -- that the epimorphisms of representations are precisely the surjective homomorphisms and the monomorphisms of representations are precisely the injective homomorphisms.
  \end{enumerate}
\end{remark}


\begin{lemma}
  Let~$V$ and~$W$ be two representations of a Lie~algebra~$\glie$ let~$f$ be a homomorphism of representations from~$V$ to~$W$.
  Then the kernel of~$f$ is a subrepresentation of~$V$ while the image of~$f$ is a subrepresentation~$W$.
  \qed
\end{lemma}


\begin{remark}
  \label{homomorphisms of representations as invariants}
  Let~$V$ and~$W$ be two representations of a Lie~algebra~$\glie$.
  A linear map~$f$ from~$V$ to~$W$ is a homomorphism of representations if and only if it is invariant under the induced action of~$\glie$ on~$\Hom(V,W)$.
  Indeed, we have that
  \begin{align*}
        {}& \text{$f$ is a homomorphism}  \\
    \iff{}& \text{$f(x \act v) = x \act f(v)$ for all~$x \in \glie$,~$v \in V$}  \\
    \iff{}& \text{$f(x \act v) - x \act f(v) = 0$ for all~$x \in \glie$,~$v \in V$}  \\
    \iff{}& \text{$(x \act f)(v) = 0$ for all~$x \in \glie$,~$v \in V$} \\
    \iff{}& \text{$x \act f = 0$ for every~$x \in \glie$}  \\
    \iff{}& \text{$f$ is invariant.}
  \end{align*}
  This means overall that
  \[
    \Hom_{\glie}(V,W)
    =
    \Hom_{\kf}(V,W)^{\glie} \,.
  \]
\end{remark}


\begin{proposition}
  \label{list of homomorphism of representations}
  Let~$\glie$ be a Lie algebra.
 \begin{enumerate}
    \item
      Let~$V$ be a representation of~$\glie$.
      The natural isomorphism of vector spaces
      \[
        \kf \tensor V
        \to
        V \,,
        \quad
        \lambda \tensor v
        \mapsto
        \lambda v
      \]
      is already a natural isomorphism of representations.
    \item
      Let~$V$ and~$W$ be two representations of~$\glie$.
      For every homomorphism of representations~$f$ from~$V$ to~$W$ the dual linear map
      \[
        f^*
        \colon
        W^*
        \to
        V^* \,,
        \quad
        \phi
        \mapsto
        \phi \circ f
      \]
      is again a homomorphism of representations.
    \item
      Let more generally~$U$,~$V$,~$W$ be three representations of~$\glie$.
      For every homomorphism of representations~$f$ from~$U$ to~$V$ the linear map
      \begin{alignat*}{2}
        f^*
        \colon
        \Hom_{\kf}(V,W)
        &\to
        \Hom_{\kf}(U,W) \,,
        &
        \quad
        \phi
        &\mapsto
        \phi \circ f 
      \intertext{is again a homomorphism of representations, and for every homomomorphism of representation~$g$ from~$V$ to~$W$ the linear map}
        g_*
        \colon
        \Hom_{\kf}(U,V)
        &\to
        \Hom_{\kf}(U,W) \,,
        &
        \quad
        \phi
        &\mapsto
        g \circ \phi
      \end{alignat*}
      is again a homomorphisms of representations.
    \item
      Let~$V$ and~$W$ be two representations of~$\glie$.
      The natural linear map
      \[
        \Phi_1
        \colon
        V^* \tensor W
        \to
        \Hom_{\kf}(V,W) \,,
        \quad
        \phi \tensor w
        \mapsto
        (v \mapsto \phi(v) w)
      \]
      is a natural homomorphism of representations.
      If one (or both) of the representations~$V$ and~$W$ is finite-dimensional then~$\Phi_1$ is already a natural isomorphism of representations.
    \item
      Let~$V^1_1, \dotsc, V^1_{n_1}$ up to~$V^t_1, \dotsc, V^t_{n_t}$ be finitely many collections of finitely many representations of~$\glie$.
      The natural isomorphism of vector spaces
      \begin{align*}
        \Phi_2
        \colon
        \bigl(
          V^1_1 \tensor \dotsb \tensor V^1_{n_1}
        \bigr)
        \tensor
        \dotsb
        \tensor
        \bigl(
          V^t_1 \tensor \dotsb \tensor V^t_{n_t}
        \bigr)
        &\to
        V^1_1 \tensor \dotsb \tensor V^t_{n_t}
      \shortintertext{given by}
        (v^1_1 \tensor \dotsb \tensor v^1_{n_1})
        \tensor
        \dotsb
        \tensor
        (v^t_1 \tensor \dotsb \tensor v^t_{n_t})
        &\mapsto
        v^1_1 \tensor \dotsb \tensor v^t_{n_t}
      \end{align*}
      for all~$v^i_j \in V$ is already a natural isomorphism of representations.
    \item
      Let~$V_\lambda$ with~$\lambda \in \Lambda$ be a collection of representations of~$\glie$ and let~$W$ be another representations of~$\glie$.
      The natural isomorphism of vector spaces
      \begin{align*}
        \Phi_4
        \colon
        \Biggl(
          \bigoplus_{\lambda \in \Lambda}
          V_\lambda
        \Biggr)
        \tensor
        W
        &\to
        \bigoplus_{\lambda \in \Lambda}
        (V_\lambda \tensor W) \,,
        \\
        (v_\lambda)_\lambda \tensor w
        &\mapsto
        (v_\lambda \tensor w)_\lambda
      \end{align*}
      is already a natural isomorphism of representations.

      Let similarly~$V$ be a representation of~$\glie$ and let~$W_\lambda$ with~$\lambda \in \Lambda$ be a collection of representations of~$\glie$.
      Then the natural isomorphism of vector spaces
      \begin{align*}
        \Phi_5
        \colon
        V
        \tensor
        \Biggl(
          \bigoplus_{\lambda \in \Lambda}
          W_\lambda
        \Biggr)
        &\to
        \bigoplus_{\lambda \in \Lambda}
        V \tensor W_\lambda \,,
        \\
        v \tensor (w_\lambda)_\lambda
        \mapsto
        (v \tensor w_\lambda)_\lambda
      \end{align*}
      is already a natural isomorphism of representations.
    \item
      Let~$V_1, \dotsc, V_n$ be representations of~$\glie$ and let~$\sigma \in \symm_n$ be any permuation.
      The natural isomorphism of vector spaces
      \begin{align*}
        \Phi_6
        \colon 
        V_1 \tensor \dotsb \tensor V_n
        &\to
        V_{\sigma(1)} \tensor \dotsb \tensor V_{\sigma(n)} \,,
        \\
        v_1 \tensor \dotsb \tensor v_n
        &\mapsto
        v_{\sigma(1)} \tensor \dotsb \tensor v_{\sigma(n)}
      \end{align*}
      is already a natural isomorphism of representations.
    \item
      Let~$V_1, \dotsc, V_n$ and~$W_1, \dotsc, W_n$ be representations of~$\glie$.
      For every index~$i = 1, \dotsc, n$ let
      \[
        f_i \colon V_i \to W_i
      \]
      be a homomorphisms of representations.
      Then the linear map
      \begin{align*}
        f_1 \tensor \dotsb \tensor f_n
        \colon
        V_1 \tensor \dotsb \tensor V_n
        &\to
        W_1 \tensor \dotsb \tensor W_n \,,
        \\
        v_1 \tensor \dotsb \tensor v_n
        &\mapsto
        f(v_1) \tensor \dotsb \tensor f(v_n)
      \end{align*}
      is again a homomorphism of representations.
    \qed
  \end{enumerate}
\end{proposition}


\begin{proposition}[Homomorphism theorem]
  \label{homomorphism theorem!for representations}
  Let~$V$ be a representation of a Lie~algebra~$\glie$ and let~$U$ be a subrepresentation of~$V$.
  \begin{enumerate}
    \item
      The canonical projction
      \[
        \pi
        \colon
        V
        \to
        V/U \,,
        \quad
        v
        \mapsto
        \class{v}
      \]
      is a homomorphism of representations. 
  \end{enumerate}
  Let~$W$ be another representation of~$\glie$.
  \begin{enumerate}[resume*]
    \item
      Let~$g$ be a homomorphism of representations from~$V/U$ to~$W$.
      The composite~$g \circ \pi$ is a homomorphism of representations from~$V$ to~$W$ with~$U \subseteq \ker(g \circ \pi)$.
    \item
      Let~$f$ be a homomorphism of representations from~$V$ to~$W$.
      The homomorphism~$f$ factors through a homomorphism of representations~$g$ from~$V/U$ to~$W$ that makes the diagram
      \[
        \begin{tikzcd}
          V
          \arrow{r}[above]{f}
          \arrow{d}[left]{\pi}
          &
          W
          \\
          V/U
          \arrow[dashed]{ur}[below right]{g}
          &
          {}
        \end{tikzcd}
      \]
      commute if and only if~$U \subseteq \ker(f)$.
      The homomorphism~$g$ is unique and it is given by
      \[
        g(\class{v}) = f(v)
        \qquad
        \text{for every~$v \in V$.}
      \]
      Its image and kernel are given by~$\im(g) = \im(f)$ and~$\ker(g) = \ker(f)/U$.
    \qed
  \end{enumerate}
\end{proposition}


\begin{corollary}[Isomorphism theorems]
  \index{isomorphism theorems!for representations}
  Let~$V$ be a representation of a Lie~algebra~$\glie$.
  \begin{enumerate}
    \item
      Let~$W$ be another representation of~$\glie$ and let~$f$ be a homomorphism of representations from~$V$ to~$W$.
      The homomorphism~$f$ induces a unique well-defined isomorphism of representations
      \[
        V/{\ker(f)}
        \to
        \im(f) \,,
        \quad
        \class{v}
        \mapsto
        f(v)  \,.
      \]
    \item
      Let~$U$ and~$W$ be two subrepresentations of~$V$ such that~$U$ is contained in~$W$.
      Then~$W/U$ is a subrepresentation of~$V/U$ and the natural isomorphism of vector spaces
      \[
        (V/U) / (W/U)
        \to
        V/W \,,
        \quad
        \class{v}
        \mapsto
        \class{v}
      \]
      is already a natural isomorphism of representations.
    \item
      Let~$U$ and~$W$ be subrepresentations of~$V$.
      Then~$W$ is a subrepresentation of the sum~$U+W$, the intersection~$U \cap W$ is a subrepresentation of~$U$, and the natural isomorphism of vector spaces
      \[
        U/(U \cap W)
        \to
        (U + W)/W  \,,
        \quad
        \class{u}
        \mapsto
        \class{u}
      \]
      is already a natural isomorphism of representations.
  \end{enumerate}
\end{corollary}


\begin{lemma}
  Let~$V$ and~$W$ be two representations of a Lie~algebra~$\glie$ and let~$f$ be a homomorphism of representations from~$V$ to~$W$.
  \begin{enumerate}
    \item
      If~$U$ is a subrepresentation of~$V$ then the image~$f(U)$ is a subrepresentation of~$W$.
    \item
      If~$U$ is a subrepresentation of~$V$ then the preimage~$f^{-1}(U)$ is a subrepresentation of~$W$.
    \qed
  \end{enumerate}
\end{lemma}


\begin{proposition}[Correspondence theorem]
  \label{correspondence theorem!for representations}
  Let~$V$ be a represenation of a Lie~algebra~$\glie$ and let~$U$ be a subrepresentation of~$V$.
  Let~$\pi$ the canonical projection from~$V$ to~$V/U$.
  
  If~$W$ is a subrepresentation of~$V$ which contains~$U$ then the quotient~$W/U$ is a subrepresentation of~$V/U$.
  This construction results in a {\onetoonetext} correspondence
  \begin{align*}
    \{ \text{subrepresentations~$W$ of~$V$which~$U$} \}
    &\longleftrightarrow
    \{ \text{subrepresentations of~$V/U$} \}  \,,
    \\
    W
    &\mapsto
    W/U \,,
    \\
    \pi^{-1}(W')
    &\mapsfrom
    W'  \,.
  \end{align*}
  If~$W$ is a subrepresentation of~$V$ containing~$U$ then it holds for the associated subrepresentation~$W/U$ of~$V/U$ that~$(V/U)/(W/U) \cong V/W$.
\end{proposition}





\section{Irreducible and Semisimple Representations}


\begin{definition}
  Let~$V$ be a representation of a Lie~algebra~$\glie$.
  \begin{enumerate}
    \item
      The representation~$V$ is \defemph{irreducible}\index{irreducible representation}\index{representation!irreducible} or \defemph{simple}\index{simple!representation}\index{representation!simple} if it is nonzero and its only subrepresentations are~$0$ and itself.
    \item
      The representation~$V$ is \defemph{indecomposable}\index{indecomposable representation}\index{representation!indecomposable} if it is nonzero and it does not admit any decomposition~$V = U_1 \oplus U_2$ into subrepresentations~$U_1$ and~$U_2$ apart from the two decompositions~$V = V \oplus 0$ and~$V = 0 \oplus V$.
      If~$V$ is not indecomposable then it is \defemph{decomposable}\index{decomposable representation}\index{representation!decomposable}
    \item
      The representation~$V$ is~\defemph{completely reducible}\index{completely reducible representation}\index{representation!completely reducible} or~\defemph{semisimple}\index{semisimple!representation}\index{representation!semisimple} if it admits a decomposition
      \[
        V = \bigoplus_{i \in I} U_i
      \]
      into irreducible subrepresentations~$U_i$.
  \end{enumerate}
\end{definition}


\begin{remark}
  \leavevmode
  \begin{enumerate}
    \item
      Every irreducible representation indecomposable, but the converse does not hold in general not hold true.
    \item
      A representation is irreducible if and only if it is both indecomposable and semisimple.
  \end{enumerate}
\end{remark}


\begin{example}
  \leavevmode
  \begin{enumerate}
    \item
      Every {\onedimensional} representation is irreducible.
    \item
      The adjoint representation of a Lie~algebra~$\glie$ is irreducible if and only if~$\glie$ is nonzero and contains no ideals beside~$0$ and~$\glie$ itself.
      This is the case if and only if~$\glie$ is either {\onedimensional} and abelian, or simple.
  \end{enumerate}
\end{example}


\begin{proposition}[Characterization of semisimple representations]
  Let~$V$ be a representation of a Lie~algebra~$\glie$.
  The following conditions on~$V$ are equivalent.
  \begin{equivalenceslist}
    \item
      The representation~$V$ is semisimple, i.e.~$V$ admits a decomposition~$V = \bigoplus_{i \in I} U_i$ into some irreducible subrepresentations~$U_i$ of~$V$.
    \item
      The representation~$V$ can be written as a (not necessarily direct) sum~$V = \sum_{i \in I} U_i$ for some irreducible subrepresentations~$U_i$ of~$V$.
    \item
      Every subrepresentation~$U$ of~$V$ admits a direct complement, i.e.\ there exists a subrepresentation~$W$ of~$V$ with~$V = U \oplus W$.
  \end{equivalenceslist}
\end{proposition}


\begin{proof}
  The proof is the same as for semisimple modules over a~\algebra{$\kf$}, which we won’t repeat here.
\end{proof}


\begin{remark}
  Let~$A$ be a~{\algebra{$\kf$}}.
  Every simple~{\module{$A$}}~$M$ is cyclic and therefore satisfies the inequality~$\dim(M) \leq \dim(A)$.
  It follows that if~$A$ is finite-dimensional then all simple~{\modules{$A$}} are also finite-dimensional, with their dimensions uniformly bounded by the dimension of~$A$.
  
  The same does not hold for finite-dimensional Lie~algebras.
  If~$\glie$ is a finite-dimensional Lie~algebra then the irreducible representations of~$\glie$ do not have to be finite-dimensional.
  Moreover, those irreducible representations that are finite-dimensional can have arbitrarily large dimension.
  
  Indeed, we will see in \cref{highest weight irreps for sl2} that the Lie~algebra~$\sllie(2,\kf)$ admits up to isomorphism for every nonzero, finite dimension~$d$ precisely one irreducible representation of dimension~$d$ (if~$\kf$ is algebraically closed), and that~$\sllie(2,\kf)$ also admits infinite-dimensional irreducible representations.
  
  But we note that the following statements still hold any Lie~algebra~$\glie$.
  \begin{enumerate}
    \item
      Any nonzero finite-dimensional representation of~$\glie$ contains an irreducible subrepresentation.
      Indeed, we can take any subrepresentation of minimal nonzero dimension.
    \item
      There exists a finite-dimensional irreducible representation~$V$ for~$\glie$.
      Here we can take the one-dimensional trivial representation.
  \end{enumerate}
\end{remark}


\begin{proposition}[Schur]
  \index{Schur’s Lemma}
  Let~$V$ and~$W$ be two representations of a Lie~algebra~$\glie$ and let~$f$ be a homomorphism of representations from~$V$ to~$W$.
  \begin{enumerate}
    \item
      If~$V$ is irreducible then either~$f$ is injective or~$f = 0$, but not both.
    \item
      If~$W$ is irreducible then either~$f$ is surjective or~$f = 0$, but not both.
    \item
      If both~$V$ and~$W$ are irreducible then either~$f$ is bijective or~$f = 0$, but not both.
    \item
      If~$V$ is irreducible then the endomorphism algebra~$\End_{\glie}(V)$ is a skew field extension of~$\kf$.
    \item
      Let~$\kf$ be algebraically closed.
      If~$V$ is both finite-dimensional and irreducible then every endomorphism~$f$ of~$V$ is of the form~$f = \lambda \id_V$ for some scalar~$\lambda$ in~$\kf$.
      In other words,~$\End_{\glie}(V) = \kf$.
  \end{enumerate}
\end{proposition}


\begin{proof}
  \leavevmode
  \begin{enumerate}
    \item
      The kernel of~$f$ is a subrepresentation of~$V$ and so either~$\ker(f) = 0$ or~$\ker(f) = V$, but not both because~$V$ is nonzero.
    \item
      The image of~$f$ is a subrepresentation of~$W$ and so either~$\im(f) = W$ or~$\im(f) = 0$, but not both because~$W$ is nonzero.
    \item
      This is a combination of the previous two statements.
    \item
      This is a reformulation of the previous statement.
      We only need to observe that the algebra~$\End_{\glie}(V)$ is nonzero because~$V$ is nonzero, and thus~$\id_V$ is nonzero.
    \item
      It follows from the finite-dimensionality of~$V$ that the endomorphism algebra~$\End_{\glie}(V)$ is finite-dimensional.
      It also follows from the irreducibility of~$V$ that~$\End_{\glie}(V)$ is a skew field extension of~$\kf$.
      We thus find that~$\End_{\glie}(V)$ is a finite-dimensional skew field extension of~$\kf$.
      It now follows that~$\End_{\glie}(V) = \kf$ because~$\kf$ is algebraically closed.
    \qedhere
  \end{enumerate}
\end{proof}





