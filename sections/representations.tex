\section{Representations of Lie~Algebras}


% TODO: Irreps of f.d. Lie algebras need not be f.d. (example: Heisenberg)
%       Can have arbitrary dimension (exmaple: sl2)





\subsection{Definition and Examples}


\begin{definition}
  Let~$\glie$ be a~{\liealgebra{$\kf$}}.
  \begin{enumerate}
    \item
      A \defemph{representation}\index{representation}~$(V, \rho)$ of~$\glie$ is a~{\vectorspace{$\kf$}}~$V$ together with a homomorphism of Lie~algebras~$\rho \colon \glie \to \gllie(V)$.
    \item
      The \defemph{dimension}\index{representation!dimension}\index{dimension of a representation} of a representation~$(V, \rho)$ is the dimension of~$V$.
    \item
      A representation~$(V, \rho)$ is~\defemph{faithful}\index{faithful}\index{representation!faithful} if the homomorphism~$\rho$ is injective.
  \end{enumerate}
\end{definition}


\begin{remark}
  Equivalently, a representation of~$\glie$ is a~{\vectorspace{$\kf$}}~$V$ together with a~{\bilinear{$\kf$}} map~$\glie \times V \to V$,~$(x,v) \mapsto x.v$ --- an action of~$\glie$ on~$V$ --- such that
  \begin{equation}
  \label{representation via action}
    x.(y.v) - y.(x.v)
    =
    [x,y].v
  \end{equation}
  for all~$x, y \in \glie$ and~$v \in V$.
  Such an action results in a homomorphism of Lie~algebras~$\rho \colon \glie \to \gllie(V)$ by setting
  \[
    \rho(x)
    \colon
    V
    \to
    V \,,
    \quad
    v
    \mapsto
    x.v
  \]
  for all~$x \in \glie$ and~$v \in V$.
  On the other hand every Lie~algebra homomorphism~$\rho \colon \glie \to \gllie(V)$ results in an action as above by setting
  \[
    x.v
    \defined
    \rho(x)(v)
  \]
  for all~$x \in \glie$ and~$v \in V$.
  
  The two constructions are mutually inverse.
  We will in the following not distinguish between these two concepts of representations and choose whichever is more useful in the given situation.
\end{remark}


\begin{remark}
  If~$(x_i)_{i \in I}$ is a basis of a Lie~algebra~$\glie$ then for a~{\linear{$\kf$}}~$\rho \colon \glie \to \gllie(V)$ to be a homomorphism of Lie~algebras it sufficies that~$\rho([x_i,x_j])= [\rho(x_i), \rho(x_j)]$ for all~$i, j \in I$.
  It therefore sufficies to check condition~\eqref{representation via action} for basis elements, i.e.\ it sufficies to check that
  \[
    x_i.(x_j.v) - x_j.(x_i.v)
    = 
    [x_i, x_j].v
  \]
  for all~$i, j \in I$ and~$v \in V$.
\end{remark}


\begin{remark}
  Ado’s theorem is equivalent to every finite dimenisonal Lie~algebra having a faithful representation.
  \index{Ado’s theorem}
\end{remark}


\begin{examples}
  \label{examples for representations}
  \leavevmode
  \begin{enumerate}
    \item 
      If~$\glie$ is a Lie~subalgebra of some~$\gllie(V)$ then~$V$ is a representation of~$\glie$ via the inclusion~$\glie \inclusion \gllie(V)$.
      This representation corresponds to the action of~$\glie$ on~$V$ given by
      \[
        x.v
        =
        x(v)
      \]
      for all~$x \in \glie$ and~$v \in V$.
      This representation is the \defemph{natural representation}\index{natural representation}\index{representation!natural} of~$\glie$.
    \item
      If~$\glie$ is a Lie~subalgebra of some~$\gllie_n(\kf)$ then~$\glie$ acts on~$\kf^n$ via
      \[
        x.v
        =
        x \cdot v
      \]
      for all~$x \in \glie$ and~$v \in V$, which correspondings to the Lie~algebra homomorphism
      \[
        \glie
        \to
        \gllie(\kf^n) \,,
        \quad
        x
        \mapsto
        (v \mapsto x \cdot v) \,.
      \]
      This representation is the the \defemph{natural representation}\index{natural representation}\index{representation!natural} of~$\glie$.
    \item
      Let~$\glie \defined \sllie_2(\kf)$ for any field~$\kf$.
      Then the polynomial ring~$\kf[x,y]$ becomes a representation of~$\glie$ via the homomorphism of Lie~algebras~$\rho \colon \glie \to \gllie(\kf[x,y])$ given by
      \[
        \rho(e) = y \dd{x} \,,
        \qquad
        \rho(h) = y \dd{y} - x \dd{x} \,,
        \qquad
        \rho(f) = x \dd{y}  \,.
      \]
      Here we denote by~$x$ and~$y$ not only the variables of~$\kf[x,y]$ but also the multiplication with these variables, and~$(e,h,f)$ denotes the standard basis of~$\sllie_2(\kf)$.
      To see that this is a homomorphism of Lie~algebra note that
      \begin{align*}
        e.(x^n y^m)
        &=
        n x^{n-1} y^{m+1} \,, \\
        h.(x^n y^m)
        &=
        (m-n) x^n y^m \,, \\
        f.(x^n y^m)
        &=
        m x^{n+1} y^{m-1}
      \end{align*}
      for all~$n, m \geq 0$.
      It follows that
      \begin{gather*}
        \begin{aligned}
        e.f.(x^n y^m) - f.e.(x^n y^m)
        &=
        (n+1)m x^n y^m - n(m+1) x^n y^m
        \\
        &=
        (m-n) x^n y^m
        \\
        &= h.m = [e,f].(x^n y^m)
        \end{aligned}
      \shortintertext{as well as}
        \begin{aligned}
        h.e.(x^n y^m) - e.h.(x^n y^m)
        &=
        n(m-n+2) x^{n-1} y^{m+1} - n(m-n) x^{n-1} y^{m+1}
        \\
        &=
        2 x^{n-1} y^{m+1}
        \\
        &=
        2e.(x^n y^m)
        \\
        &=
        [h,e].(x^n y^m)
        \end{aligned}
      \shortintertext{and}
        \begin{aligned}
          h.f.(x^n y^m) - f.h.(x^n y^m)
          &=
          m(m-n-2) x^{n+1} y^{m-1} - m(m-n) x^{n+1} y^{m-1}
          \\
          &=
          -2 x^{n+1} y^{m-1}
          \\
          &=
          -2f.(x^n y^{m-1})
          \\
          &=
          [h,f].(x^n y^{m-1})
        \end{aligned}
      \end{gather*}
      for all~$n, m \geq 0$.
    \item
      Let~$\glie \defined \sllie_2(\kf)$ for any field~$\kf$.
      Then the polynomial ring in one variable~$\kf[x]$ is a representation of~$\glie$ via the homomorphism of Lie~algebras~$\rho \colon \glie \to \gllie(k[x])$ given by
      \[
        \rho(e)
        =
        \dd{x} \,,
        \qquad
        \rho(h)
        =
        -2x\dd{x} \,,
        \qquad
        \rho(f)
        =
        -\dd{x} \,.
      \]
      The~$\glie$ acts on~$\kf[x]$ via
      \[
        e.x^n = n x^{n-1} \,,
        \qquad
        h.x^n = -2n x^n \,,
        \qquad
        f.x^n = n x^{n+1}
      \]
      for every~$n \geq 0$.
      This is indeed a representation of~$\glie$ because
      \begin{gather*}
        e.f.x^n - f.e.x^n
        = -n(n+1) x^n + n(n-1) x^n
        = -2n x^n
        = h.x^n
        = [e,f].x^n
      \shortintertext{as well as}
        h.e.x^n - e.h.x^n
        = -2n(n-1) x^{n-1} + 2n^2 x^{n-1}
        = 2n x^{n-1}
        = 2e.x^n
        = [h,e].x^n
      \shortintertext{and}
        h.f.x^n - f.h.x^n
        = 2n(n+1) x^{n+1} - 2 n^2 x^{n+1}
        = 2n x^{n+1}
        = -2f.x^n
        = [h,f].x^n
      \end{gather*}
      for every~$n \geq 0$.
    \item
      If~$\rho \colon \glie \to \gllie(V)$ is a representation of a Lie~algebra~$\glie$ and~$\phi \colon \hlie \to \glie$ a homomorphism of Lie~algebras then via the composition~$\rho \circ \phi \colon \hlie \to \gllie(V)$ the vector space~$V$ becomes a representation of~$\hlie$.
      This homomorphism corresponds to the action given by
      \[
        x.v = \rho(x).v = \rho(\phi(x))(v)
      \]
      for all~$x \in \hlie$, and~$v \in V$.
    \item
      The map~$\ad \colon \glie \to \gllie(\glie)$,~$x \mapsto \ad(x)$ is a homomorphism of Lie~algebras and hence a representation of~$\glie$.
  \end{enumerate}
\end{examples}


\begin{definition}
  Let~$\glie$ be a Lie~algebra.
  The  Lie~algebra homomorphism
  \[
    \ad
    \colon
    \glie
    \to
    \gllie(\glie) \,,
    \quad
    x
    \mapsto
    \ad(x)
    =
    [x,-]
  \]
  is the \defemph{adjoint representation}\index{adjoint representation} of~$\glie$.
\end{definition}


\begin{remark}
  It follows together with \cref{lie algebras act adjoint by derivations} that every Lie~algebras~$\glie$ acts on itself by derivations of itself via the adjoint representation.
  The author suspects that this is where much of the structure of Lie~algebras comes from and why the Jacobi identity is of interest.
\end{remark}





\subsection{New Representations from Old Ones}


\begin{proposition}
  \label{new representations from old ones}
  Let~$\glie$ be a Lie~algebra over an arbitrary field~$\kf$.
  \begin{enumerate}
    \item
      If~$(V_i)_{i \in I}$ is a collection of representations of~$\glie$ then the product~$\prod_{i \in I} V_i$ is again a representation of~$\glie$ via
      \[
        x.(v_i)_{i \in I}
        =
        ( x.v_i )_{i \in I}
      \]
      or all~$x \in \glie$ and~$(v_i)_{i \in I} \in \prod_{i \in I} V_i$.
    \item
      If~$(V_i)_{i \in I}$ is a collection of representations of~$\glie$ then the direct sum~$\bigoplus_{i \in I} V_i$ is again a representation of~$\glie$ via
      \[
        x.(v_i)_{i \in I}
        =
        ( x.v_i )_{i \in I}
      \]
      or all~$x \in \glie$ and~$(v_i)_{i \in I} \in \bigoplus_{i \in I} V_i$.
    \item
      If~$V$ and~$W$ are two representations of~$\glie$ then their tensor product~$V \tensor W$ is again a representation of~$\glie$ via
      \[
        x.(v \tensor w)
        =
        (x.v) \tensor w + v \tensor (x.w)
      \]
      for all~$x \in \glie$,~$v \in V$ and~$w \in W$.
      More generally, if~$V_1, \dotsc V_n$ are representations of~$\glie$ then their tensor product~$V_1 \tensor \dotsb \tensor V_n$ is again a representation of~$\glie$ via
      \[
        x.(v_1 \tensor \dotsb \tensor v_n)
        = \sum_{i=1}^n
                  v_1
          \tensor \dotsb
          \tensor v_{i-1}
          \tensor (x.v_i)
          \tensor v_{i+1}
          \tensor \dotsb
          \tensor v_n
      \]
      for every~$x \in \glie$ and all~$v_i \in V_i$ with~$i = 1, \dotsc, n$.
    \item
      If~$V$ and~$W$ are two representations of~$\glie$ then~$\Hom_\kf(V,W)$ is again a representation of~$\glie$ via
      \[
        (x.f)(v)
        =
        x.f(v) - f(x.v)
      \]
      for all~$x \in \glie$,~$f \in \Hom_\kf(V,W)$ and~$v \in V$.
    \item
      Every~{\vectorspace{$\kf$}}~$V$ becomes a representation of~$\glie$ via the \defemph{trivial action}\index{trivial action}
      \[
        x.v
        =
        0
      \]
      for all~$x \in \glie$ and~$v \in V$.
    \item
      By letting~$\glie$ act trivially on~$\kf$ the dual~$V^* = \Hom_\kf(V, \kf)$ becomes a representation of~$\glie$ in the above way, i.e.\ via
      \[
        (x.\phi)(v)
        =
        -\phi(x.v)
      \]
      for all~$x \in \glie$ and~$v \in V$.
  \end{enumerate}
\end{proposition}


\begin{proof}
  \leavevmode
  \begin{enumerate}
    \item
      Let~$x, y \in \glie$ and~$(v_i)_{i \in I} \in \prod_{i \in I} V_i$, then
      \begin{align*}
        x.y.(v_i)_{i \in I} - y.x.(v_i)_{i \in I}
        &=
        (x.y.v_i)_{i \in I} - (y.x.v_i)_{i \in I}
        \\
        &=
        (x.y.v_i - y.x.v_i)_{i \in I}
        \\
        &=
        ([x,y].v_i)_{i \in I}
        \\
        &=
        [x,y].(v_i)_{i \in I} \,.
      \end{align*}
    \item
      This follows from the same calculation as for the product~$\prod_{i \in I} V_i$.
    \item
      It sufficies by induction to consider the case~$n = 2$, i.e.\ the tensor product~$V \tensor W$.
      Then
      \begin{align*}
        x.y.(v \tensor w)
        &=
        x.((y.v) \tensor w + v \tensor (y.w))
        \\
        &=
        x.((y.v) \tensor w) + x.(v \tensor (y.w))
        \\
        &=
        (x.y.v) \tensor w + (y.v) \tensor (x.w) + (x.v) \tensor (y.w) + v \tensor (x.y.w)
      \end{align*}
      and therefore
      \begin{align*}
        x.y.(v \tensor w) - y.x.(v \tensor w)
        &=
        (x.y.v) \tensor w + v \tensor (x.y.w) - (y.x.v) \tensor w - v \tensor (y.x.w)
        \\
        &=
        (x.y.v - y.x.v) \tensor w + v \tensor (x.y.w - y.x.w)
        \\
        &=
        ([x,y].v) \tensor w + v \tensor ([x,y].w) \,.
      \end{align*}
    \item
      For all~$x,y \in \glie$,~$f \in \Hom(V,W)$ and~$v \in V$ it holds that
      \begin{align*}
        (x.y.\phi)(v) - (y.x.\phi)(v)
        &=
        -(y.\phi)(x.v) + (x.\phi(y.v)
        \\
        &=
        \phi(y.x.v) - \phi(x.y.v)
        \\
        &=
        -\phi(x.y.v - y.x.v)
        \\
        &=
        -\varphi([x,y].v)
        \\
        &=
        ([x.y].\varphi)(v)  \,.
      \end{align*}
    \item
      It holds for all~$x,y \in \glie$ and~$v \in V$ that
      \[
        x.y.v - y.x.v
        =
        0 - 0
        =
        0
        =
        [x,y].v \,.
      \]
    \item
      This is a combination of the previous two constructions.
    \qedhere
  \end{enumerate}
\end{proof}


\begin{definition}
  Let~$V$ be a representation of a Lie~algebra~$\glie$.
  A \defemph{subrepresentation} of~$V$ is a linear subpace~$U$ of~$V$ such that~$x.u \in U$ for all~$x \in \glie$ and~$u \in U$.
\end{definition}


\begin{remark}
  If~$(V, \rho)$ is a representation of a Lie~algebra~$\glie$ then a linear subspace~$U \subseteq V$ is a subrepresentation if and only if~$U$ is~{\invariant{$\rho(x)$}} for every~$x \in \glie$, in the sense that~$\rho(x)(U) \subseteq U$.
\end{remark}


\begin{examples}
  Let~$\glie$ be a Lie~algebra.
  \begin{enumerate}
    \item
      If~$V$ is any representation of~$\glie$ then the linear subspaces~$0$ and~$V$ itself are subrepresentations.%
    \footnote{These two subrepresentations are often called the \defemph{trivial} ones.
      We will abstain from doing so, as we have already defined the notion of a trivial representation in \cref{trivial representations}.}
    \item
      If~$V$ is a representation of~$\glie$ and~$U_i$ with~$i \in I$ a collection of subrepresentations~$U_i$ of~$V$ then~$\sum_{i \in I} U_i$ is again a subrepresentation of~$V$.
    \item
      If~$(V_i)_{i \in I}$ is any collection of representations of~$\glie$ then the direct sum~$\bigoplus_{i \in I} V_i$ is a subrepresentation of the product~$\prod_{i \in I} V_i$.
    \item
      The subrepresentations of the adjoint representation~$\ad \colon \glie \to \gllie(\glie)$ are precisely the ideals in~$\glie$.
    \item
      If~$V$ is any representation of~$\glie$ then the linear subspace
      \[
        \glie V
        \defined
        \vspan_\kf
        \{
          x.v
          \suchthat
          x \in \glie,
          v \in V
        \}
      \]
      is a subrepresentation of~$\glie$.
    \item
      If~$V$ is any representation of~$\glie$ then the linear subspace of invariants~$V^{\glie}$ is a (trivial) subrepresentation.
    \item
      Let~$f \colon V \to V$ be an endomorphism of a representation~$V$ of~$\glie$.
      Then for any scalar~$\lambda \in \kf$ both the eigenspace
      \[
        V_\lambda
        \defined
        \{
          v \in V
        \suchthat
          f(v)
          =
          \lambda v
        \}
      \]
      and the generalized eigenspace
      \[
        V_{(\lambda)}
        \defined
        \bigcup_{n \geq 0} \ker(f - \lambda \id_V)^n
        =
        \{
          v \in V
        \suchthat
        \text{$(f - \lambda \id_V)^n(v) = 0$ for some~$n \geq 0$}
        \}
      \]
      are subrepresentations of~$V$.
  \end{enumerate}
\end{examples}


\begin{definition}
  \label{trivial representations}
  Let~$V$ be a representation of a Lie~algebra~$\glie$.
  \begin{enumerate}
    \item
      The representation~$V$ is \defemph{trivial}\index{trivial representation} if every~$x \in \glie$ acts by multiplication with zero on~$V$.
    \item
      An element~$v \in V$ is \defemph{\invariant{$\glie$}}\index{invariants}, or simply~\defemph{invariant}, if~$x.v = 0$ for every~$x \in \glie$.
      The set of invariants is denoted by
      \[
        \gls*{invariants}
        \defined
        \{
          v \in V
        \suchthat
          \text{$x.v = 0$ for every~$x \in \glie$}
        \}  \,.
      \]
  \end{enumerate}  
\end{definition}


% TODO: Explain connection to representations of groups.


\begin{lemma}
  The space of invariants~$V^{\glie}$ is for every representation~$V$ of~$\glie$ the maximal invariant subrepresentation of~$V$.
  \qed
\end{lemma}


\begin{example}
  The invariants of the adjoint representation (of~$\glie$ on itself) are given by
  \begin{align*}
    \glie^{\glie}
    &=
    \{
      y \in \glie
    \suchthat
      \text{$x.y = 0$ for every~$x \in \glie$}
    \}
    \\
    &=
    \{
      y \in \glie
    \suchthat
      \text{$\ad(x)(y) = 0$ for every~$x \in \glie$}
    \}
    \\
    &=
    \{
      y \in \glie
    \suchthat
      \text{$[x,y] = 0$ for every~$x \in \glie$}
    \}
    \\
    &=
    \centerlie(\glie) \,.
  \end{align*}
\end{example}


\begin{example}[Quotient representations]
  \label{quotient representation}
  Let~$V$ be a representation of a Lie~algebra~$\glie$ and let~$U$ be a subrepresentation of~$V$.
  Then the quotient vector space~$V/U$ inherits from~$V$ the structure of a~{\representation{$\glie$}} via
  \[
    x.\class{v}
    =
    \class{x.v}
  \]
  for all~$x \in \glie$ and~$v \in V$.
  Indeed, we have for all~$x, y \in \glie$ and~$\class{v} \in V$ that
  \[
    [x,y].\class{v}
    =
    \class{[x,y].v}
    =
    \class{x.(y.v) - y.(x.v)}
    =
    \class{x.(y.v)} - \class{y.(x.v)}
    =
    x.(y.\class{v}) - y.(x.\class{v}) \,.
  \]
  
  Alternatively let~$\rho \colon \glie \to \gllie(V)$ be the Lie~algebra homomorphism corresponding to the action of~$\glie$ on~$V$.
  Then the linear subspace~$U$ of~$V$ is~{\invariant{$\rho(x)$}} for every~$x \in \glie$, and hence the endomorphism~$\rho(x)$ induces an endomorphism
  \[
    \induced{\rho(x)}
    \colon
    V/U
    \to
    V/U \,,
    \quad
    \class{v}
    \mapsto
    \class{\rho(x)(v)}
  \]
  for every~$x \in \glie$.
  The resulting map~$\induced{\rho} \colon \glie \to \gllie(V/U)$ given by~$\induced{\rho}(x) = \induced{\rho(x)}$  is a homomorphism of Lie~algebras because
  \[
    \induced{\rho}([x,y])
    =
    \induced{\rho([x,y])}
    =
    \induced{[\rho(x), \rho(y)]}
    =
    [\induced{\rho(x)}, \induced{\rho(y)}]
    =
    [\induced{\rho}(x), \induced{\rho}(y)]
  \]
  for all~$x, y \in \glie$.
\end{example}


\begin{definition}
  Let~$V$ be a representation of a Lie~algebra~$\glie$ and let~$U$ be a subrepresentation of~$V$.
  The representation~\gls*{quotient representation} from \cref{quotient representation} is the \defemph{quotient representation}\index{quotient!representations}\index{representation!quotient} of~$V$ by~$U$.
\end{definition}





\subsection{Homomorphisms of Representations}


\begin{definition}
  Let~$V$ und~$W$ be two representations of a~{\liealgebra{$\kf$}}~$\glie$.
  \begin{enumerate}
    \item
      A~{\linear{$\kf$}} map~$f \colon V \to W$ is a \defemph{homomorphism of representations}\index{homomorphism!of representations} if
      \[
        f(x.v) = x.f(v)
      \]
      for all~$x \in \glie$ and~$v \in V$.
    \item
      A homomorphism of representations~$f$ is an \defemph{isomorphism of representations}\index{isomorphism!of representations} if it is bijective.
    \item
      The space of homomorphism~$V \to W$ is denoted by~\gls*{rep homo}, and for~$V = W$ by~\gls*{rep endo}.
  \end{enumerate}
\end{definition}

\begin{remark}
  Let~$V$,~$W$ and~$U$ be representations of a Lie~algebra~$\glie$ over~$\kf$.
  \begin{enumerate}
    \item
      The notions of a \defemph{monomorphism}\index{monomorphism of representations}, \defemph{epimorphism}\index{epimorphism of representations}, \defemph{endomorphism}\index{endomorphism of representations} and \defemph{automorphism}\index{automorphism of representations} are defined in the usual way.
    \item
      If the representations~$V$ and~$W$ are given by the Lie~algebra homomorpisms~$\rho_V \colon \glie \to \gllie(V)$ and~$\rho_W \colon \glie \to \gllie(W)$ then a~{\linear{$\kf$}} map~$f \colon V \to W$ is a homomorphism of representations if and only if~$f \circ \rho_V(x) = \rho_W(x) \circ f$ for all~$x \in \glie$, i.e.\ if and only if the square diagram
      \[
        \begin{tikzcd}[column sep = large]
          V
          \arrow{r}[above]{\rho_V(x)}
          \arrow{d}[left]{f}
          &
          V
          \arrow{d}[right]{f}
          \\
          W
          \arrow{r}[above]{\rho_W(x)}
          &
          W
        \end{tikzcd}
      \]
      commutes for every~$x \in \glie$.
    \item
      If~$f, g \colon V \to W$ are homomorphisms of representations then~$f + g$ is again a homomorphism of representations, and~$\lambda f$ is for every~$\lambda \in \kf$ again a homomorphism of representations.
      Hence~$\Hom_{\glie}(V,W)$ is a~{\linear{$\kf$}} subspace of~$\Hom_\kf(V,W)$.
    \item
      If~$f \colon V \to W$ is an isomorphism of representations then its inverse~$f^{-1} \colon W \to V$ is again a homomorphism (and thus isomorphism) of representations.
      Indeed, we find that
      \[
        f^{-1}(x.v)
        =
        f^{-1}(x.f(f^{-1}(v)))
        =
        f^{-1}(f(x.f^{-1}(v)))
        =
        x.f^{-1}(v)
      \]
      for all~$x \in \glie$ and~$v \in V$.
    \item
      The identity~$\id_V \colon V \to V$ is always an automorphism of the representation~$V$.
    \item
      If~$f \colon V \to W$ and~$g \colon W \to U$ are homomorphism of representations then their composition~$g \circ f \colon V \to U$ is again a homomorphism of representations.
    \item
      It follows that the representations of~$\glie$ together with the homomorphisms of representations between them form a (\linear{$\kf$}) category.
      We will denote this category by~\gls*{representation category}.
  \end{enumerate}
\end{remark}


\begin{lemma}
  Let~$V$ and~$W$ be two representations of a Lie~algebra~$\glie$ let~$f \colon V \to W$ be a homomorphism of representations.
  Then the kernel of~$f$ is a subrepresentation of~$V$ while the image of~$f$ is a subrepresentation~$W$.
  \qed
\end{lemma}


\begin{remark}
  \label{homomorphisms of representations as invariants}
  Given two representations~$V$ and~$W$ of a Lie~algebra~$\glie$ a linear map~$f \colon V \to W$ is a homomorphism of representations if and only if it is invariant under the induced action of~$\glie$ on~$\Hom(V,W)$:
  Indeed, we see that
  \begin{align*}
        {}& \text{$f$ is a homomorphism}  \\
    \iff{}& \text{$f(x.v) = x.f(v)$ for all~$x \in \glie$ and~$v \in V$}  \\
    \iff{}& \text{$f(x.v) - x.f(v) = 0$ for all~$x \in \glie$ and~$v \in V$}  \\
    \iff{}& \text{$(x.f)(v) = 0$ for all~$x \in \glie$ and~$v \in V$} \\
    \iff{}& \text{$x.f = 0$ for every~$x \in \glie$}  \\
    \iff{}& \text{$f$ is invariant} \,.
  \end{align*}
  Hence~$\Hom_{\glie}(V,W) = \Hom_\kf(V,W)^{\glie}$.
\end{remark}


\begin{proposition}
  \label{list of homomorphism of representations}
  Let~$\glie$ be a Lie algebra.
 \begin{enumerate}
    \item
      For every homomorphism of representations~$f \colon V \to W$ the dual linear map
      \[
        f^*
        \colon
        W^*
        \to
        V^* \,,
        \quad
        \phi
        \mapsto
        f \circ \phi
      \]
      is again a homomorphism of representations.
    \item
      For any two representations~$V$ and~$W$ of~$\glie$ the natural linear map
      \[
        \Phi_1
        \colon
        V^* \tensor W
        \to
        \Hom_k(V,W) \,,
        \quad
        \phi \tensor w
        \mapsto
        (v \mapsto \phi(v) w)
      \]
      is a homomorphism of representations.
      If at least one of the two representations~$V$ and~$W$ is finite dimensional then it is an isomorphism of representations.
    \item
      For all representations~$V_1, \dotsc, V_n$ and~$W_1, \dotsc, W_m$ of~$\glie$ the natural isomorphism of vector spaces
      \begin{align*}
        \Phi_2
        \colon
        (V_1 \tensor \dotsb \tensor V_n) \tensor (W_1 \tensor \dotsb \tensor W_m)
        &\longto
        V_1 \tensor \dotsb \tensor V_n \tensor W_1 \tensor \dotsb \tensor W_m \,,
        \\
        (v_1 \tensor \dotsb \tensor v_n) \tensor (w_1 \tensor \dotsb \tensor w_m)
        &\longmapsto
        v_1 \tensor \dotsb \tensor v_n \tensor w_1 \tensor \dotsb \tensor w_m
      \end{align*}
      is already an isomorphism of representations.
    \item
      For any two representations~$V$ and~$W$ of~$\glie$ the natural isomorphism of vector spaces
      \[
        \Phi_3
        \colon
        V \tensor W
        \to
        W \tensor V \,,
        \quad
        v \tensor w
        \mapsto
        w \tensor v
      \]
      is already an isomorphism of representations.
    \item
      For all representations~$V_1$,~$V_2$ and~$W$ of~$\glie$ the natural isomorphism of vector spaces
      \begin{align*}
        \Phi_4
        \colon
        (V_1 \tensor V_2) \tensor W
        &\longto
        (V_1 \tensor W) \oplus (V_2 \tensor W) \,,
        \\
        (v_1, v_2) \tensor w
        &\longmapsto
        (v_1 \tensor w, v_2 \tensor w)
      \end{align*}
      is already an isomorphism of representations.
    \item
      If~$V_1, \dotsc, V_n$ are representations of~$\glie$ and~$\sigma \in S_n$ is any permuation then the natural isomorphism of vector spaces
      \begin{align*}
        \Phi_5
        \colon 
        V_1 \tensor \dotsb \tensor V_n
        &\longto
        V_{\sigma(1)} \tensor \dotsb \tensor V_{\sigma(n)} \,,
        \\
        v_1 \tensor \dotsb \tensor v_n
        &\longmapsto
        v_{\sigma(1)} \tensor \dotsb \tensor v_{\sigma(n)}
      \end{align*}
      is already an isomorphism of representations.
    \item
      If~$V_1, \dotsc, V_n$ and~$W_1, \dotsc, W_n$ are representations of~$\glie$ and~$f_i \colon V_i \to W_i$ with~$i = 1, \dotsc, n$ are homomorphisms of representations then the natural linear map
      \begin{align*}
        f_1 \tensor \dotsb \tensor f_n
        \colon
        V_1 \tensor \dotsb \tensor V_n
        &\longto
        W_1 \tensor \dotsb \tensor W_n
        \\
        v_1 \tensor \dotsb \tensor v_n
        &\longmapsto
        f(v_1) \tensor \dotsb \tensor f(v_n)
      \end{align*}
      is already a homomorphism of representations.
    \qed
  \end{enumerate}
\end{proposition}


\begin{proposition}[Homomorphism theorem]
  \label{homomorphism theorem!for representations}
  Let~$V$ be a representation of a Lie~algebra~$\glie$ and let~$U$ be a subrepresentation of~$V$.
  Let~$W$ be another representation of~$\glie$.
  For every homomorphism of representations~$f \colon V \to W$ with~$U \subseteq \ker f$ there exists a unique homomorphism of representations~$\induced{f} \colon V/U \to W$ that makes the triangular diagram
  \[
    \begin{tikzcd}
      V
      \arrow{r}[above]{f}
      \arrow{d}[left]{\pi}
      &
      W
      \\
      V/U
      \arrow[dashed]{ur}[below right]{\induced{f}}
      &
      {}
    \end{tikzcd}
  \]
  commute.
  It holds that~$\ker \induced{f} = {\ker f}/I$ and~$\im \induced{f} = \im f$.
\end{proposition}


\begin{corollary}[Isomorphism theorems]
  \index{isomorphism theorems!for representations}
  Let~$V$ be a representation of a Lie~algebra~$\glie$.
  \begin{enumerate}
    \item
      If~$W$ is another representation of~$\glie$ and~$f \colon V \to W$ is any homomorphism of representations then~$f$ induces a unique well-defined isomorphism of representations
      \[
        \induced{f}
        \colon
        V/{\ker f}
        \to
        \im f \,,
        \quad
        \class{v}
        \mapsto
        f(v)  \,.
      \]
    \item
      If~$U$ and~$W$ are subrepresentations of~$V$ with~$U \subseteq W$ then~$W/U$ is a subrepresentation of~$V/U$ and the natural isomorphism of vector spaces
      \[
        (V/U) / (W/U)
        \to
        V/W \,,
        \quad
        \class{v}
        \mapsto
        \class{v}
      \]
      is already an isomorphism of representations.
    \item
      If~$U$ and~$W$ are subrepresentations of~$V$ then~$W$ is a subrepresentation of~$U+W$ and~$U \cap W$ is a subrepresentation of~$U$, and the natural isomorphism of vector spaces
      \[
        U/(U \cap W)
        \to
        (U + W)/W  \,,
        \quad
        \class{u}
        \mapsto
        \class{u}
      \]
      is already an isomorphism of representations.
  \end{enumerate}
\end{corollary}


\begin{proposition}[Correspondence theorem]
  \label{correspondence theorem!for representations}
  Let~$V$ be a represenation of a Lie~algebra~$\glie$ and let~$U \subseteq V$ be a subrepresentation.
  Let~$\pi \colon V \to V/U$ the canonical projection.
  
  If~$W$ is a subrepresentation of~$V$ that contains~$U$ then the quotient~$W/U$ is a subrepresentation of~$V/U$.
  This construction results in a {\onetoone} correspondence
  \begin{align*}
    \{ \text{subrepresentations~$W \subseteq V$ containing~$U$} \}
    &\longleftrightarrow
    \{ \text{subrepresentations of~$V/U$} \}  \,,
    \\
    W
    &\longmapsto
    W/U \,,
    \\
    \pi^{-1}(W')
    &\longmapsfrom
    W'  \,.
  \end{align*}
  If~$W$ is a subrepresentation of~$V$ containing~$U$ then it holds for the associated subrepresentation~$W/U$ of~$V/U$ that~$(V/U)/(W/U) \cong V/W$.
\end{proposition}





\subsection{Irreducible and Semisimple Representations}


\begin{definition}
  Let~$V$ be a representation of a Lie~algebra~$\glie$.
  \begin{enumerate}
    \item
      The representation~$V$ is \defemph{irreducible}\index{irreducible representation}\index{representation!irreducible} or \defemph{simple}\index{simple!representation}\index{representation!simple} if it is nonzero and admits only the subrepresentations~$0$ and~$V$ itself.
    \item
      The representation~$V$ is \defemph{indecomposable}\index{indecomposable representation}\index{representation!indecomposable} if there does not exists a decomposition~$V = U_1 \oplus U_2$ into subrepresentations~$U_1$ and~$U_2$ apart from~$V = V \oplus 0$ and~$V = 0 \oplus V$.
      Otherwise~$V$ is \defemph{decomposable}\index{decomposable representation}\index{representation!decomposable}
    \item
      The representation~$V$ is~\defemph{completely reducible}\index{completely reducible representation}\index{representation!completely reducible} or~\defemph{semisimple}\index{semisimple!representation}\index{representation!semisimple} if it has a decomposition~$V = \bigoplus_{i \in I} U_i$ into irreducible subrepresentations~$U_i$.
  \end{enumerate}
\end{definition}


\begin{remark}
  \leavevmode
  \begin{enumerate}
    \item
      Every irreducible representation indecomposable, but the converse does not hold.
    \item
      A representation is irreducible if and only if it is both indecomposable and completely reducible.
    \item
      For any representation~$V$ the following conditions are equivalent:
      \begin{equivalenceslist}
        \item
          $V$ is semisimple, i.e.~$V$ admits a decomposition~$V = \bigoplus_{i \in I} U_i$ into irreducible subrepresentations~$U_i$.
        \item
          $V$ can be written as a (not necessarily direct) sum~$V = \sum_{i \in I} U_i$ for irreducible subrepresentations~$U_i$.
        \item
          Every subrepresentation~$U$ of~$V$ admits a direct complement, i.e.\ there exists a subrepresentation~$W$ of~$V$ with~$V = U \oplus W$.
      \end{equivalenceslist}
  \end{enumerate}
\end{remark}


\begin{example}
  \leavevmode
  \begin{enumerate}
    \item
      Every {\onedimensional} representation is irreducible.
    \item
      The adjoint representation of a Lie~algebra~$\glie$ is irreducible if and only if~$\glie$ is nonzero and contains no ideals beside~$0$ and~$\glie$ itself.
      This is the case if and only if~$\glie$ is either {\onedimensional} and abelian, or simple.
  \end{enumerate}
\end{example}


\begin{lemma}[Schur]
  \index{Schur’s Lemma}
  Let~$V$ and~$W$ be representations of a Lie~algebra~$\glie$ and let~$f \colon V \to W$ be a homomorphism of representations.
  \begin{enumerate}
    \item
      If~$V$ is irreducible then either~$f$ is injective or~$f = 0$, but not both.
    \item
      If~$W$ is irreducible then either~$f$ is surjective or~$f = 0$, but not both.
    \item
      If both~$V$ and~$W$ are irreducible then either~$f$ is bijective or~$f = 0$, but not both.
    \item
      If~$V$ is irreducible then the endomorphism algebra~$\End_{\glie}(V)$ is a skew field over~$\kf$.
    \item
      If the field~$\kf$ is algebraically closed and~$V$ finite dimensional and irreducible then every endomorphism~$f \in \End_{\glie}(V)$ is given by multiplication with some scalar~$\lambda \in \kf$.
      In particular~$\End_{\glie}(V) = \kf$.
  \end{enumerate}
\end{lemma}


\begin{proof}
  \leavevmode
  \begin{enumerate}
    \item
      The kernel~$\ker f$ is a subrepresentation of~$V$ and so either~$\ker f = 0$ or~$\ker f = V$, but not both.
    \item
      The image~$\im f$ is a subrepresentation of~$W$ and so either~$\im f = W$ or~$\im f = 0$, but not both.
    \item
      This is a combination of the previous two statements.
    \item
      This is a reformulation of the previous statement;
      note that~$\End_{\glie}(V) \neq 0$ because~$\id_V \neq 0$.
    \item
      The endomorphism~$f$ admits an eigenvalue~$\lambda \in \kf$ because~$\kf$ is algebraically closed.
      The endomorphism~$f - \lambda \id_V$ is non-injective and hence~$f - \lambda \id_V = 0$ as seen above.
    \qedhere
  \end{enumerate}
\end{proof}





\subsection{Extension}


\begin{definition}
  A \defemph{short exact sequence of Lie~algebras}\index{short exact sequence!of Lie algebras} is a short exact sequence
  \begin{equation}
    \label{general extension}
    0 
    \to
    I
    \xlongto{f}
    \glie
    \xlongto{g}
    \hlie
    \to
    0
  \end{equation}
  of vector spaces such that~$I$,~$\glie$ and~$\hlie$ are Lie algebras and both~$f$ and~$g$ are Lie algebra homomorphisms.
  The short exact sequence~\eqref{general extension} is then an~\defemph{extension}\index{extension} of~$\hlie$ by~$I$.
\end{definition}


\begin{remark}
  \leavevmode
  \begin{enumerate}
    \item
      By abuse of notation we often say that \enquote{$\glie$ is an extension of~$\hlie$ by~$I$} to mean that there exists such an extension with~$\glie$ as its middle term.
    \item
      If~$0 \to I \xto{f} \glie \xto{g} \hlie \to 0$ is a short exact sequence of Lie~algebras then the image of~$I$ is the kernel of~$g$ and hence an ideal in~$\glie$.
  \end{enumerate}
\end{remark}


\begin{example}
  If~$I$ is any ideal in a Lie~algebra~$\glie$ then
  \[
    0
    \to
    I
    \xlongto{i}
    \glie
    \xlongto{p}
    \glie/I
    \to
    0
  \]
  is a short exact sequence of Lie~algebras where~$i$ is the inclusion~$x \to x$ and~$p$ is the canonical projection~$x \mapsto \class{x}$.
\end{example}


\begin{definition}
  \label{equivalence of extensions}
  Let~$I$ and~$\hlie$ be two Lie~algebras.
  Two extensions
  \[
    0 
    \to
    I
    \xlongto{f}
    \glie
    \xlongto{g}
    \hlie
    \to
    0
    \quad\text{and}\quad
    0 
    \to
    I
    \xlongto{f'}
    \glie'
    \xlongto{g'}
    \hlie
    \to
    0
  \]
  of~$\hlie$ by~$I$ are \defemph{equivalent}\index{equivalent extensions} if there exists an isomorphism of Lie~algebras~$\varphi \colon \glie \to \glie'$ that makes the following diagram commute:
  \[
    \begin{tikzcd}
      0
      \arrow{r}
      &
      I
      \arrow{r}
      \arrow[equal]{d}
      &
      \glie
      \arrow{r}
      \arrow[dashed]{d}[right]{\varphi}
      &
      \hlie
      \arrow{r}
      \arrow[equal]{d}
      &
      0
      \\
      0
      \arrow{r}
      &
      I
      \arrow{r}
      &
      \glie'
      \arrow{r}
      &
      \hlie
      \arrow{r}
      &
      0
    \end{tikzcd}
  \]
\end{definition}


\begin{lemma}
  Let~$I$ and~$\hlie$ be two Lie~algebras.
  Equivalences of extensions is an equivalence relation on the class of extensions of~$\hlie$ by~$I$.
\end{lemma}


\begin{remark}
  It follows from the five lemma that in \cref{equivalence of extensions} we do not have to require the homomorphism~$\varphi$ to be an isomorphism --- this automatically follows from the commutativity of the given diagram.
\end{remark}


\begin{remark}
  \label{general approach to extensions}
  Let~$I$ and~$\hlie$ be Lie~algebras.
  One general approach to understanding (certain equivalence classes of) extensions of~$\hlie$ by~$I$ is the following:
  
  Any such extension
  \begin{equation}
  \label{original extension}
    0
    \to
    I
    \xlongto{f}
    \glie
    \xlongto{g}
    \hlie
    \to
    0
  \end{equation}
  is in particular a short exact sequece of vector spaces, and splits as such.
  In other words, there exists an isomorphism of vector spaces~$\varphi \colon \glie \to \hlie \oplus I$ that makes the diagram
  \[
    \begin{tikzcd}[column sep = large]
      0
      \arrow{r}
      &
      I
      \arrow{r}
      \arrow[equal]{d}
      &
      \glie
      \arrow{r}
      \arrow[dashed]{d}[right]{\varphi}
      &
      \hlie
      \arrow{r}
      \arrow[equal]{d}
      &
      0
      \\
      0
      \arrow{r}
      &
      I
      \arrow{r}[above]{\iota}
      &
      \hlie \oplus I
      \arrow{r}[above]{\pi}
      &
      \hlie
      \arrow{r}
      &
      0
    \end{tikzcd}
  \]
  commute.
  Here we denote~$\iota \colon I \to \hlie \oplus I$ the canonical inclusion~$c \mapsto (0,c)$ and by~$\pi \colon \hlie \oplus I \to \hlie$ the canonical projection~$(x,c) \mapsto x$.
  It follows that the original extension~\eqref{original extension} is equivalent to an extension whose middle term is~$\hlie \oplus I$.
  Up to equivalence we can therefore assume that the extension of interest is of the form
  \[
    0
    \to
    I
    \xlongto{\iota}
    \hlie \oplus I
    \xlongto{\pi}
    \hlie
    \to
    0 \,.
  \]

  Suppose now that we are given such an extension.
  We have for the Lie~bracket on~$\hlie \oplus I$ that
  \begin{align*}
    {}&
    [(x, c), (y,d)]
    \\
    ={}&
      [(x,0), (y,0)]
    + [(x,0), (0,d)]
    + [(0,c), (y,0)]
    + [(0,c), (0,d)]
    \\
    ={}&
      [(x,0), (y,0)]
    + [(x,0), (0,d)]
    - [(y,0), (0,c)]
    + [(0,c), (0,d)]
  \end{align*}
  for all~$(x,c), (y,d) \in \hlie \oplus I$.
  We have that
  \[
    \pi( [(x,0), (y,0)] )
    =
    [ \pi( (x,0) ), \pi( (y,0) ) ]
    =
    [x, y]
  \]
  because~$\pi$ is a homorphism of Lie~algebras.
  The commutator~$[(x,0), (y,0)]$ is therefore of the form
  \[
    [(x,0), (y,0)]
    =
    ( [x,y], \kappa(x,y) )
  \]
  for some function
  \[
    \kappa
    \colon
    \hlie \times \hlie
    \to
    I \,.
  \]
  It follows from the bilinearity of the Lie~bracket~$[-,-]$ on~$\hlie \oplus I$ that the map~$\kappa$ is again bilinear.
  The commutators~$[(x,0), (0,c)]$ and~$[(y,0), (0,d)]$ are again contained in~$0 \oplus I$ because this is an ideal in~$\hlie \oplus I$ (since it is the kernel of~$\pi$).
  It follows that there exists a unique linear map
  \[
    \theta
    \colon
    \hlie
    \to
    \gllie(I)
  \]
  with
  \[
    [(z,0), (0,e)]
    =
    [0, \theta(z)(e)]
  \]
  for all~$z \in \hlie$ and~$e \in I$.
  We have seen in \cref{lie algebras act adjoint by derivations} that the action of~$[(z,0), -] = \ad_{\hlie \oplus I}((z,0))$ on~$\hlie \oplus I$ is by derivations.
  It follows that~$\theta$ is actually a linear map
  \[
    \theta
    \colon
    \hlie
    \to
    \Der(I) \,.
  \]
  We now have that
  \[
    [(x,0), (0,d)]
    =
    [0, \theta(x)(d)]
    \quad\text{and}\quad
    [(y,0), (0,e)]
    =
    [0, \theta(y)(e)] \,.
  \]
  The last remaining commutator~$[(0,c), (0,d)]$ can be computed by observing that
  \[
    [(0,c), (0,d)]
    =
    [\iota(c), \iota(d)]
    =
    \iota([c,d])
    =
    (0, [c,d])
  \]
  because the inclusion~$\iota \colon I \to \hlie \oplus I$ is a homomorphism of Lie~algebras.
  We find altogether that the Lie bracket on~$\hlie \oplus I$ can be expressed with help of the maps~$\kappa$ and~$\theta$ as
  \begin{align*}
    {}&
    [ (x,c), (y,d) ]
    \\
    ={}&
      ( [x,y], \kappa(x,y) )
    + ( 0, \theta(x)(d) )
    - ( 0, \theta(y)(c) )
    + ( 0, [c,d] )
    \\
    ={}&
    ( [x,y], \kappa(x,y) + \theta(x)(d) - \theta(y)(c) + [c,d] ) \,.
  \end{align*}
  The Lie~algebra structure of the extension~$\hlie \oplus I$ is therefore uniquely determined by the two maps~$\kappa \colon \hlie \times \hlie \to I$ and~$\theta \colon \hlie \to \Der(I)$.

  Suppose now that we are not yet given a Lie~bracket on~$\hlie \oplus I$ but instead a bilinear map~$\kappa \colon \hlie \times \hlie \to I$ and a linear map~$\theta \colon \hlie \to \Der(I)$.
  Then we can conversely define a bilinear bracket~$[-,-]$ on~$\hlie \oplus I$ via the above formula
  \[
    [ (x,c), (y,d) ]
    \defined
    ( [x,y], \kappa(x,y) + \theta(x)(d) - \theta(y)(c) + [c,d] )  \,.
  \]
  (Note that this formula involves three brackets:
  The one on~$\hlie \oplus I$ that is being defined, and the ones on~$\hlie$ and~$I$ that are used on the right hand side.)
  If this is a Lie bracket on~$\hlie \oplus I$ then this makes~$\hlie \oplus I$ into an extension of~$\hlie$ of~$I$, as both the inclusion~$\iota \colon I \to \hlie \oplus I$ and the projection~$\pi \colon \hlie \oplus I \to \hlie$ are then Lie~algebra homomorphisms.
  But of course not any choice of~$\kappa$ and~$\theta$ does define a Lie~bracket on~$\hlie \oplus I$ via the above formula:
  We need that~$[-,-]$ is alternating and that~$[-,-]$ satisfies the Jacobi identity.
  
  For the first condition we calculate that
  \[
    [(x,c), (x,c)]
    =
    ( [x,x], \kappa(x,x) + \theta(x)(c) - \theta(x)(c) + [c,c] )
    =
    (0, \kappa(x,x))
  \]
  for all~$(x, c) \in \hlie \oplus I$.
  This shows that~$[-,-]$ is alternating if and only if~$\kappa$ is alternating.
  We can similarly express that~$[-,-]$ satisfies the Jacobi identity in terms of both~$\kappa$ and~$\theta$.
  But we will not do this here, as the resulting terms become rather ugly.
  We will instead do this missing calculation for some special cases.
\end{remark}


\begin{warning}
  If we are given an extension~$0 \to I \to \glie \to \hlie \to 0$ then we may identify~$\glie$ with the direct sum~$\glie \oplus I$ as vector spaces and then describe the Lie~bracket of~$\glie$ in terms of a bilinear map~$\kappa \colon \hlie \times \hlie \to I$ and a linear map~$\theta \colon \hlie \to \Der(I)$.
  But these maps depend not only on the Lie~bracket of~$\glie$ but also on the choosen identification of~$\glie$ with~$\hlie \oplus I$!
\end{warning}



\subsubsection{Central Extensions}


\begin{definition}
  An extension~$0 \to I \to \glie \to \hlie \to 0$ of Lie~algebras is \emph{central}\index{central extension}\index{extension!central} if the image of~$I$ is contained in the center of~$\glie$.
\end{definition}


\begin{example}[Central extensions]
  In the notation of \cref{general approach to extensions} we find that an extension
  \[
    0
    \to
    I
    \xlongto{\iota}
    \hlie \oplus I
    \xlongto{\pi}
    \hlie
    \to
    0
  \]
  is central if and only~$[(x,c), (0,d)] = 0$ for all~$(x,c), (0,d) \in \hlie \oplus I$.
  This means precisely that~$\theta = 0$ and also~$[c,d] = 0$ for all~$c, d \in I$.
  Hence the Lie bracket on~$\hlie \oplus I$ is given by
  \begin{equation}
    \label{central extension formula}
    [ (x,c), (y,d) ]
    =
    ( [x,y], \kappa(x,y) )
  \end{equation}
  for all~$(x,c), (y,d) \in \hlie \oplus I$.
  
  Given any bilinear form~$\kappa \colon \hlie \times \hlie \to I$ we have already seen that \eqref{central extension formula} defines a bilinear bracket on~$\hlie \oplus I$ that is alternating if and only if~$\kappa$ is alternating.
  We find for the Jacobi identity that
  \[
    [ (x,c), [ (y,d), (z,e) ] ]
    =
    [ (x,c), ([y,z], \kappa(y,z)) ]
    =
    ( [x,[y,z]], \kappa(x, [y,z]) )
  \]
  for all~$(x,c), (y,d), (z,e) \in \hlie \oplus I$ and therefore
  \begin{align*}
    {}&
      [(x,c), [(y,d), (z,e)]]
    + [(y,d), [(z,e), (x,c)]]
    + [(z,e), [(x,c), (y,d)]]
    \\
    ={}&
    (
      [x,[y,z]] + [y,[z,x]] + [z,[x,y]],
      \kappa(x, [y,z]) + \kappa(y, [z,x]) + \kappa(z, [x,y])
    )
    \\
    ={}&
    ( 0, \kappa(x, [y,z]) + \kappa(y, [z,x]) + \kappa(z, [x,y]) ) \,.
  \end{align*}
  We hence find that the bracket~$[-,-]$ on~$\hlie \oplus I$ satisfies the Jacobi identity if and only if the bilinear map~$\kappa$ satisfies the similar looking identity
  \[
    \kappa(x, [y,z]) + \kappa(y, [z,x]) + \kappa(z, [x,y])
    =
    0 \,.
  \]
  This condition is the \defemph{{\twococycle} condition}\index{$2$-cocycle condition}, and a bilinear map~$\kappa \colon \hlie \to \hlie \to I$ that is both alternating is satisfies the {\twococycle} condition is a {\twococycle}.
  
  We have overall constructed for all Lie~algebras~$\hlie$ and~$I$ a {\onetoone} correspondence between
  \begin{itemize}
    \item
      Lie~brackets~$[-,-]$ on the vector space~$\hlie \oplus I$ that make the standard short exact sequence
      \[
        0
        \to
        I
        \to
        \hlie
        \oplus
        I
        \to
        \hlie
        \to
        0
      \]
      into a central extension of~$\hlie$ by~$I$ and
    \item
      {\twococycles}~$\kappa \colon \hlie \times \hlie \to I$,
  \end{itemize}
  and this correspondence is given by~$[(x,c), (y,d)] = ([x,y], \kappa(x,y))$.
\end{example}



\subsubsection{Split Extensions and Semidirect Products}


\begin{definition}
  A short exact sequence
  \[
    0
    \to
    I
    \xlongto{f}
    \glie
    \xlongto{g}
    \hlie
    \to
    0
  \]
  \defemph{splits}\index{extension!split} if there exists a Lie~algebra homomorphism~$s \colon \hlie \to \glie$ with~$g \circ s = \id_{\hlie}$.
  The homomorphism~$s$ is then a \defemph{split} for the short exact sequence.
  
  A \emph{split extension} is an extension that splits.%
  \footnote{This is meant to be funny.}
\end{definition}


\begin{example}[Split extensions]
  Let
  \[
    0
    \to
    I
    \xlongto{f}
    \glie
    \xlongto{g}
    \hlie
    \to
    0
  \]
  be any split extension of~$\hlie$ by~$I$, and let~$s \colon \hlie \to \glie$ be a split.
  Then there exists a (unique) homomorphism of vector spaces~$\varphi \colon \glie \to \hlie \oplus I$ that makes the resulting diagram
  \[
    \begin{tikzcd}[column sep = large]
      0
      \arrow{r}
      &
      I
      \arrow{r}[above]{f}
      \arrow[equal]{d}
      &
      \glie
      \arrow{r}[above]{g}
      \arrow[dashed]{d}[right]{\varphi}
      &
      \hlie
      \arrow{r}
      \arrow[equal]{d}
      &
      0
      \\
      0
      \arrow{r}
      &
      I
      \arrow{r}[above]{\iota}
      &
      \hlie \oplus I
      \arrow{r}[above]{\pi}
      &
      \hlie
      \arrow{r}
      &
      0
    \end{tikzcd}
  \]
  commute and such that the square diagram
  \[
    \begin{tikzcd}
      \glie
      \arrow[dashed]{d}[left]{\varphi}
      &
      \hlie
      \arrow{l}[above]{s}
      \arrow[equal]{d}
      \\
      \hlie \oplus I
      &
      \hlie
      \arrow{l}{\sigma}
    \end{tikzcd}
  \]
  commutes.
  Hence we may again assume that~$\glie = \hlie \oplus I$, that~$f$ is the canonical inclusion~$\iota \colon I \to \hlie \oplus I$, that~$g$ is the canonical projection~$\pi \colon \hlie \oplus I \to \hlie$ that~$s$ is the canonical inclusion~$\sigma \colon \hlie \to \hlie \oplus I$.
  
  We can now use the notations from \cref{general approach to extensions} to express the Lie bracket~$[-,-]$ of~$\hlie \oplus I$ via the bilinear form~$\kappa \colon \hlie \times \hlie \to I$ and the linear map~$\theta \colon \hlie \to \Der(I)$.
  
  The bilinear map~$\kappa$ is defined by the equality
  \[
    [(x,0), (y,0)]
    =
    ([x,y], \kappa(x,y))
  \]
  for all~$x, y \in \hlie$.
  The linear subspace~$\hlie \oplus 0$ of~$\hlie \oplus I$ is the image of the Lie~algebra homomorphism~$\sigma \colon \hlie \to \hlie \oplus I$ and hence a Lie~subalgebra of~$\hlie \oplus I$.
  We therefore find that the commutator~$[(x,0), (y,0)]$ is again contained in~$\hlie \oplus 0$ for all~$x, y \in \hlie$.
  This means that~$\kappa = 0$.
  
  The linear map~$\theta \colon \hlie \to I$ takes for~$x \in \hlie$ the restriction of the endomorphism~$\ad_{\hlie \oplus I}((x,0))$ to the ideal~$0 \oplus I$ and then identifying~$0 \oplus I$ with~$I$ to get~$\theta(x) \in \Der(I)$.
  This assignment may be written as the composition
  \[
    \theta
    \colon
    \hlie
    \xlongto{\sigma}
    \hlie \oplus I
    \xlongto{ \restrict{\ad(-)}{0 \oplus I} }
    \Der(0 \oplus I)
    \to
    \Der(I) \,.
  \]
  Each of the intermediate maps is a Lie~algebra homomorphism, so~$\theta$ is again a homomorphism.
  
  It remains to examine what further conditions such a homomorphism~$\theta$ needs to satisfy for the bracket~$[-,-]$ on~$\hlie \oplus I$ to satisfy the Jacobi identity.
  We claim that there are none.
  
  Indeed, we need to check that under the given constraints~($\kappa = 0$ and~$\theta$ is a homomorphism of Lie~algebras~$\hlie \to \Der(I)$) we already have the Jacobi identity
  \[
      [\alpha, [\beta, \gamma]]
    + [\beta, [\gamma, \alpha]]
    + [\gamma, [\alpha, \beta]]
    =
    0
  \]
  for all~$\alpha, \beta, \gamma \in \hlie \oplus I$.
  For this we use a trick taken from \cite{semidirect_jacobi}:
  Let us denote the left hand side of this equation by~$J(\alpha, \beta, \gamma)$.
  It follows from the bilinearity of~$[-,-]$ that~$J$ is trilinear.
  It also follows from~$[-,-]$ being alternating and thus skew-symmetric that~$J$ is again skew-symmetric.
  It therefore sufficies to check the condition~$J(\alpha, \beta, \gamma) = 0$ for the following four cases:
  \begin{itemize}
    \item
      If~$\alpha = (x,0)$,~$\beta = (y,0)$ and~$\gamma = (z,0)$ with~$x, y, z \in \hlie$ then
      \[
        [\alpha, [\beta, \gamma]]
        =
        [(x,0), [(y,0), (z,0)]]
        =
        [(x,0), ([y,z], 0)]
        =
        ([x,[y,z]], 0)
      \]
      and therefore
      \begin{align*}
        J(\alpha, \beta, \gamma)
        &=
          ([x,[y,z]], 0)
        + ([y,[z,x]], 0)
        + ([z,[x,y]], 0)
        \\
        &=
        ([x,[y,z]] + [y,[z,x]] + [z,[x,y]], 0)  \,.
      \end{align*}
      Hence~$J(\alpha, \beta, \gamma) = 0$ precisely because the Lie~bracket of~$\hlie$ satisfies the Jacobi identity.
    \item
      If~$\alpha = (x,0)$,~$\beta = (y,0)$ and~$\gamma = (0,c)$ with~$x, y \in \hlie$ and~$c \in I$ then
      \[
        [\alpha, [\beta, \gamma]]
        =
        [(x,0), [(y,0), (0,c)]]
        =
        [(x,0), [(0, \theta(y)(c)]]
        =
        ( 0, \theta(x)(\theta(y)(c)) )
      \]
      and similarly
      \[
        [\gamma, [\alpha, \beta]]
        =
        [(0,c), [(x,0), (y,0)]]
        =
        [(0,c), ([x,y], 0)]
        =
        ( 0, -\theta([x,y])(c) ) \,.
      \]
      It also follows that
      \[
        [\beta, [\gamma, \alpha]]
        =
        [(y,0), [(0,c), (x,0)]]
        =
        -[(y,0), [(x,0), (0,c)]]
        =
        ( 0, -\theta(y)(\theta(x)(c)) )
      \]
      and therefore
      \begin{align*}
        J(\alpha, \beta, \gamma)
        &=
          [(x,0), [(y,0), (0,c)]]
        + [(y,0), [(0,c), (x,0)]]
        + [(0,c), [(x,0), (y,0)]]
        \\
        &=
        ( 0, \theta(x)(\theta(y)(c)) - \theta([x,y])(c) - \theta(y)(\theta(x)(c)) ) \,.
      \end{align*}
      We hence find for this case that~$J(\alpha, \beta, \gamma) = 0$ precisely because~$\theta$ is a Lie~algebra homomorphism.
    \item
      If~$\alpha = (x,0)$,~$\beta = (0,c)$ and~$\gamma = (0,d)$ with~$x \in \hlie$ and~$c, d \in I$ then
      \[
        [\alpha, [\beta, \gamma]]
        =
        [(x,0), [(0,c), (0,d)]]
        =
        [(x,0), (0,[c,d])]
        =
        ( 0, \theta(x)([c,d]) )
      \]
      and similarly
      \[
        [\beta, [\gamma, \alpha]]
        =
        [(0,c), [(0,d), (x,0)]]
        =
        [(0,c), (0, -\theta(x)(d)]
        =
        ( 0, -[c,\theta(x)(d)] ) \,.
      \]
      It also follows that
      \begin{align*}
        [\gamma, [\alpha, \beta]]
        =
        [(0,d), [(x,0), (0,c)]]
        =
        - [(0,d), [(0,c), (x,0)]]
        &=
        ( 0, [d,\theta(x)(c)] )
        \\
        &=
        ( 0, -[\theta(x)(c), d] ) \,.
      \end{align*}
      It follows in combination that
      \begin{align*}
        J(\alpha, \beta, \gamma)
        &=
          [(x,0), [(0,c), (0,d)]]
        + [(0,c), [(0,d), (0,x)]]
        + [(0,d), [(x,0), (0,c)]]
        \\
        &=
        ( 0, \theta(x)([c,d]) - [c, \theta(x)(d)] - [\theta(x)(c), d] )
      \end{align*}
      which shows that~$J(\alpha, \beta, \gamma) = 0$ precisely because~$\theta(x)$ is a derivation of~$I$.
    \item
      If~$\alpha = (0,c)$,~$\beta = (0,d)$ and~$\gamma = (0,e)$ with~$c, d, e \in I$ then
      \[
        [\alpha, [\beta, \gamma]]
        =
        [(0,c), [(0,d), (0,e)]]
        =
        [(0,c), (0, [d,e])]
        =
        ( 0, [c,[d,e]] )
      \]
      and therefore
      \begin{align*}
        J(\alpha, \beta, \gamma)
        &=
          [(0,c), [(0,d), (0,e)]]
        + [(0,d), [(0,e), (0,c)]]
        + [(0,e), [(0,c), (0,d)]]
        \\
        &=
        ( 0, [c,[d,e]] + [d,[e,c]] + [e,[c,d]] )
      \end{align*}
      Which shows that~$J(\alpha, \beta, \gamma) = 0$ precisely because the Lie~bracket of~$I$ satisfies the Jacobi~identity.
  \end{itemize}
  
  We have now overall constructed for all Lie~algebras~$\hlie$ and~$I$ a {\onetoone} correspondence between
  \begin{itemize}
    \item
      Lie~brackets~$[-,-]$ on the vector space~$\hlie \oplus I$ that makes the standard short exact sequence
      \[
        0
        \to
        I
        \to
        \hlie
        \oplus
        I 
        \to
        \hlie
        \to
        0
      \]
      into a split extension of~$\hlie$ by~$I$ and
    \item
      Lie~algebra homomorphisms~$\theta \colon \hlie \to \Der(I)$,
  \end{itemize}
  and this correspondence is given by~$[(x,c), (y,d)] = ([x,y], \theta(x)(d) - \theta(y)(c) + [c,d])$.
\end{example}


\begin{definition}
  Let~$\hlie$ and~$I$ be two Lie~algebras and let~$\theta \colon \hlie \to \Der(I)$ be a homomorphism of Lie~algebras.
  The \defemph{semidirect product}~\gls*{semidirect product} of~$\hlie$ by~$I$ over~$\theta$ is the Lie~algebra~$\glie$ that is given by
  \begin{itemize}
    \item
      the underlying vector space~$\glie \defined \hlie \oplus I$ together with
    \item
      the Lie bracket~$[-,-]$ given by
      \[
        [(x,c), (y,d)]
        =
        ([x,y], \theta(x)(d) - \theta(y)(c) + [c,d])
      \]
      for all~$(x,c), (y,d) \in \glie$.
  \end{itemize}
\end{definition}


\begin{remark}
  Let~$\hlie$ and~$I$ be Lie~algebras and let~$\theta \colon \hlie \to \Der(I)$ be a Lie~algebra homomorphism.
  \begin{enumerate}
    \item
      It follows from the above discussion that~$\hlie \ltimes_\theta I$ is indeed a Lie algebra and that
      \[
        0
        \to
        I
        \xlongto{\iota}
        \hlie \ltimes_\theta I
        \xlongto{\pi}
        \hlie
        \to
        0
      \]
      is a split extension of~$\hlie$ by~$I$, where~$\iota(x) = (0,y)$ and~$\pi(x,y) = x$.
      Moreover, a split~$\sigma \colon \hlie \to \hlie \ltimes_\theta I$ of this extension is given by~$\sigma(x) = (x,0)$.
    \item
      We have also seen that whenever a Lie algebra~$\glie$ is a split extension of~$\hlie$ by~$I$ then~$\glie$ is already (equivalent to) a semidirect product of~$\hlie$ by~$I$ (over some suitable homomorphism of Lie~algebras~$\theta \colon \hlie \to \Der(I)$).
  \end{enumerate}
  The notions of split extensions and semidirect products are hence equivalent.
\end{remark}


\begin{remark}[Internal semidirect products]
  Given a Lie~algebra~$\glie$ we may ask ourselves if~$\glie$ can be decomposed as a semidirect product~$\glie \cong \hlie \ltimes_\theta I$;
  and if so, how~$I$ and~$\glie$ look like from the point of view of~$\glie$.
  
  We observe first that the semidirect product~$\glie = \hlie \ltimes_\theta I$ (where~$\theta \colon \hlie \to \Der(I)$ is some Lie~algebra homomorphism) contains the Lie~subalgebra~$\hlie' = \{ (x,0) \suchthat x \in \hlie \}$ and the ideal~$I' \defined \{ (0,c) \suchthat c \in I \}$.
  We have that~$\hlie' \cong \hlie$ and~$I' \cong I$ as Lie~algebras, and~$\glie = \hlie' \oplus I'$ as vector spaces.
  The homomorpism~$\theta \colon \hlie \to \Der(I)$ can be described by considering for~$x \in \hlie$ the corresponding element~$(x,0) \in \hlie'$, restricting the endomorphism~$\ad_{\glie}((x,0)) \in \Der(\glie)$ to an endomorphism~$\restrict{ \ad_{\glie}((x,0)) }{I'} \in \Der(I')$ and then using the isomorphism~$I' \cong I$ to arrive at~$\theta(x) \in \Der(I)$.
  
  Suppose on the other hand that~$\glie$ is any Lie~algebra and that~$\glie = \hlie \oplus I$ for some Lie~subalgebra~$\hlie$ of~$\glie$ and some ideal~$I$ of~$\glie$.
  Then~$\theta \colon \hlie \to \Der(I)$ given by~$x \mapsto \restrict{\ad_{\glie}(x)}{I}$ is a well-defined Lie algebra homomorphism, and we have that
  \begin{align*}
    [x + c, y + d]
    &=
    [x,y] + [x,d] + [c,y] + [c,d]
    \\
    &=
    [x,y] + [x,d] - [y,c] + [c,d]
    \\
    &=
    \underbrace{ [x,y] }_{\in \hlie} + \underbrace{ \theta(x)(d) - \theta(y)(c) + [c,d] }_{\in I}
  \end{align*}
  for all~$x, y \in \hlie$ and~$c, d \in I$.
  This shows that the vector space isomorphism
  \[
    \varphi
    \colon
    \hlie \oplus I
    \to
    \glie \,,
    \quad
    (x,c)
    \mapsto
    x + c
  \]
  is already an isomorphism of Lie algebras~$\varphi \colon \hlie \ltimes_\theta I \to \glie$.
\end{remark}


\begin{definition}
  Let~$\glie$ be a Lie~algebra, let~$\hlie$ be a Lie~subalgebra of~$\glie$ and let~$I$ be an ideal in~$\glie$.
  Then~$\glie$ is the \defemph{internal semidirect product}\index{internal semidirect product} of~$\hlie$ by~$I$ if~$\glie = \hlie \oplus I$.
  This is then denoted by~$\glie = \gls*{internal semidirect product}$.
\end{definition}


\begin{remark}
  The notion of an internal semidirect product does not depend on the additional information of a Lie~algebra homomorphism~$\theta \colon \hlie \to \Der(I)$.
  Instead this information is encoded in the Lie~algebra structure of~$\glie$, and related to the expected homomorphism~$\theta$ by~$\theta(x) = \restrict{\ad_{\glie}(x)}{I}$.
\end{remark}


\begin{example}
  The Lie algebra~$\glie = \tlie_n(\kf)$ can be written as a direct sum~$\tlie_n(\kf) = \dlie_n(\kf) \oplus \nlie_n(\kf)$ with~$\dlie_n(\kf)$ a Lie~subalgebra of~$\tlie_n(\kf)$ and~$\nlie_n(\kf)$ an ideal in~$\tlie_n(\kf)$.
  Hence
  \[
    \tlie_n(\kf)
    =
    \dlie_n(\kf) \ltimes \nlie_n(\kf) \,.
  \]
\end{example}


\begin{example}[Trivial extensions]
  \label{trivial extension is semidirect}
  For any two Lie~algebras~$I$ and~$\hlie$ the zero map~$\theta \colon \hlie \to \Der(I)$ is a homomorphism of Lie algebras.
  The resulting semidirect product~$\hlie \ltimes_\theta I$ is precisely the usual product~$\hlie \times I$.
  The corresponding split extension
  \[
    0
    \to
    I
    \xlongto{\iota}
    \hlie \times I
    \xlongto{\pi}
    \hlie
    \to
    0
  \]
  is given by canonical inclusion~$\iota(c) = (0,c)$ and the canonical projection~$\pi(x,y) = x$.
  This extension is the \emph{trivial extension}\index{trivial extension} of~$\hlie$ by~$I$.
  More generally any extension that is equivalent to the trivial extension is called \defemph{trivial}.
\end{example}


% \begin{remark}
%   \label{when semidirect is direct}
%   The converse to the above observation also holds:
%   Let~$I$ and~$\hlie$ be Lie~algebras and let~$\theta \colon \hlie \to \Der(I)$ be a homomorphism of Lie~algebras such that the resulting semidirect product~$\hlie \ltimes_\theta I$ is trivial.
%   Then already~$\theta = 0$.
%   
%   Indeed, if the extension
%   \[
%     0
%     \to
%     I
%     \xlongto{\iota}
%     \hlie \times I
%     \xlongto{\pi}
%     \hlie
%     \to
%     0
%   \]
%   is trivial
% 
% \end{remark}


\begin{warning}
  Let
  \[
    0
    \to
    I
    \xlongto{f}
    \glie
    \xlongto{g}
    \hlie
    \to
    0
  \]
  be an extension of Lie~algebras.
  The observant reader may have noticed that we have so far only considered splits~$s \colon \hlie \to \glie$, but not splits~$t \colon \glie \to I$.
  There is a good reason for this:
  
  Suppose that such a split exists, i.e.\ there exists a Lie~algebra homomorphism~$t \colon \glie \to I$ with~$t f = \id_I$.
  Then~$J \defined \ker t$ is an ideal in~$\glie$.
  If~$I'$ denotes the image of~$I$ in~$\glie$ then it follows that~$\glie = I' \oplus J$, the homomorphism~$f$ restricts to an isomorphism~$I \to I'$ and the homomorphism~$g$ restricts to an isomorphism~$J \to \hlie$.
  This means that the given extension is equivalent to the trivial extension, and hence trivial itself.
  
  We see on the other hand that the trivial extension admits both kinds of split, and thus the same holds for every extension that is trivial.
  
  This means altogether that an extension~$0 \to I \to \glie \to \hlie \to 0$ admits a split~$\glie \to I$ (of Lie~algebras) if and only if the extension is already trivial.
\end{warning}









