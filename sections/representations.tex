\section{Representations of Lie~Algebras}


% TODO: Irreps of f.d. Lie algebras need not be f.d. (example: Heisenberg)
%       Can have arbitrary dimension (exmaple: sl2)





\subsection{Definition and Examples}


\begin{definition}
  Let~$\glie$ be a~{\liealgebra{$\kf$}}.
  \begin{enumerate}
    \item
      A \defemph{representation}\index{representation}~$(V, \rho)$ of~$\glie$ is a~{\vectorspace{$\kf$}}~$V$ together with a homomorphism of Lie~algebras~$\rho \colon \glie \to \gllie(V)$.
    \item
      The \defemph{dimension}\index{representation!dimension}\index{dimension of a representation} of a representation~$(V, \rho)$ is the dimension of~$V$.
    \item
      A representation~$(V, \rho)$ is~\defemph{faithful}\index{faithful}\index{representation!faithful} if the homomorphism~$\rho$ is injective.
  \end{enumerate}
\end{definition}


\begin{remark}
  Equivalently, a representation of~$\glie$ is a~{\vectorspace{$\kf$}}~$V$ together with a~{\bilinear{$\kf$}} map~$\glie \times V \to V$,~$(x,v) \mapsto x.v$ --- an action of~$\glie$ on~$V$ --- such that
  \begin{equation}
  \label{representation via action}
    x.(y.v) - y.(x.v)
    =
    [x,y].v
  \end{equation}
  for all~$x, y \in \glie$ and~$v \in V$.
  Such an action results in a homomorphism of Lie~algebras~$\rho \colon \glie \to \gllie(V)$ by setting
  \[
    \rho(x)
    \colon
    V
    \to
    V \,,
    \quad
    v
    \mapsto
    x.v
  \]
  for all~$x \in \glie$ and~$v \in V$.
  On the other hand every Lie~algebra homomorphism~$\rho \colon \glie \to \gllie(V)$ results in an action as above by setting
  \[
    x.v
    \defined
    \rho(x)(v)
  \]
  for all~$x \in \glie$ and~$v \in V$.
  
  The two constructions are mutually inverse.
  We will in the following not distinguish between these two concepts of representations and choose whichever is more useful in the given situation.
\end{remark}


\begin{remark}
  If~$(x_i)_{i \in I}$ is a basis of a Lie~algebra~$\glie$ then for a~{\linear{$\kf$}}~$\rho \colon \glie \to \gllie(V)$ to be a homomorphism of Lie~algebras it sufficies that~$\rho([x_i,x_j])= [\rho(x_i), \rho(x_j)]$ for all~$i, j \in I$.
  It therefore sufficies to check condition~\eqref{representation via action} for basis elements, i.e.\ it sufficies to check that
  \[
    x_i.(x_j.v) - x_j.(x_i.v)
    = 
    [x_i, x_j].v
  \]
  for all~$i, j \in I$ and~$v \in V$.
\end{remark}


\begin{remark}
  Ado’s theorem is equivalent to every finite dimenisonal Lie~algebra having a faithful representation.
  \index{Ado’s theorem}
\end{remark}


\begin{examples}
  \label{examples for representations}
  \leavevmode
  \begin{enumerate}
    \item 
      If~$\glie$ is a Lie~subalgebra of some~$\gllie(V)$ then~$V$ is a representation of~$\glie$ via the inclusion~$\glie \inclusion \gllie(V)$.
      This representation corresponds to the action of~$\glie$ on~$V$ given by
      \[
        x.v
        =
        x(v)
      \]
      for all~$x \in \glie$ and~$v \in V$.
      This representation is the \defemph{natural representation}\index{natural representation}\index{representation!natural} of~$\glie$.
    \item
      If~$\glie$ is a Lie~subalgebra of some~$\gllie_n(\kf)$ then~$\glie$ acts on~$\kf^n$ via
      \[
        x.v
        =
        x \cdot v
      \]
      for all~$x \in \glie$ and~$v \in V$, which correspondings to the Lie~algebra homomorphism
      \[
        \glie
        \to
        \gllie(\kf^n) \,,
        \quad
        x
        \mapsto
        (v \mapsto x \cdot v) \,.
      \]
      This representation is the the \defemph{natural representation}\index{natural representation}\index{representation!natural} of~$\glie$.
    \item
      Let~$\glie \defined \sllie_2(\kf)$ for any field~$\kf$.
      Then the polynomial ring~$\kf[x,y]$ becomes a representation of~$\glie$ via the homomorphism of Lie~algebras~$\rho \colon \glie \to \gllie(\kf[x,y])$ given by
      \[
        \rho(e) = y \dd{x} \,,
        \qquad
        \rho(h) = y \dd{y} - x \dd{x} \,,
        \qquad
        \rho(f) = x \dd{y}  \,.
      \]
      Here we denote by~$x$ and~$y$ not only the variables of~$\kf[x,y]$ but also the multiplication with these variables, and~$(e,h,f)$ denotes the standard basis of~$\sllie_2(\kf)$.
      To see that this is a homomorphism of Lie~algebra note that
      \begin{align*}
        e.(x^n y^m)
        &=
        n x^{n-1} y^{m+1} \,, \\
        h.(x^n y^m)
        &=
        (m-n) x^n y^m \,, \\
        f.(x^n y^m)
        &=
        m x^{n+1} y^{m-1}
      \end{align*}
      for all~$n, m \geq 0$.
      It follows that
      \begin{gather*}
        \begin{aligned}
        e.f.(x^n y^m) - f.e.(x^n y^m)
        &=
        (n+1)m x^n y^m - n(m+1) x^n y^m
        \\
        &=
        (m-n) x^n y^m
        \\
        &= h.m = [e,f].(x^n y^m)
        \end{aligned}
      \shortintertext{as well as}
        \begin{aligned}
        h.e.(x^n y^m) - e.h.(x^n y^m)
        &=
        n(m-n+2) x^{n-1} y^{m+1} - n(m-n) x^{n-1} y^{m+1}
        \\
        &=
        2 x^{n-1} y^{m+1}
        \\
        &=
        2e.(x^n y^m)
        \\
        &=
        [h,e].(x^n y^m)
        \end{aligned}
      \shortintertext{and}
        \begin{aligned}
          h.f.(x^n y^m) - f.h.(x^n y^m)
          &=
          m(m-n-2) x^{n+1} y^{m-1} - m(m-n) x^{n+1} y^{m-1}
          \\
          &=
          -2 x^{n+1} y^{m-1}
          \\
          &=
          -2f.(x^n y^{m-1})
          \\
          &=
          [h,f].(x^n y^{m-1})
        \end{aligned}
      \end{gather*}
      for all~$n, m \geq 0$.
    \item
      Let~$\glie \defined \sllie_2(\kf)$ for any field~$\kf$.
      Then the polynomial ring in one variable~$\kf[x]$ is a representation of~$\glie$ via the homomorphism of Lie~algebras~$\rho \colon \glie \to \gllie(k[x])$ given by
      \[
        \rho(e)
        =
        \dd{x} \,,
        \qquad
        \rho(h)
        =
        -2x\dd{x} \,,
        \qquad
        \rho(f)
        =
        -\dd{x} \,.
      \]
      The~$\glie$ acts on~$\kf[x]$ via
      \[
        e.x^n = n x^{n-1} \,,
        \qquad
        h.x^n = -2n x^n \,,
        \qquad
        f.x^n = n x^{n+1}
      \]
      for every~$n \geq 0$.
      This is indeed a representation of~$\glie$ because
      \begin{gather*}
        e.f.x^n - f.e.x^n
        = -n(n+1) x^n + n(n-1) x^n
        = -2n x^n
        = h.x^n
        = [e,f].x^n
      \shortintertext{as well as}
        h.e.x^n - e.h.x^n
        = -2n(n-1) x^{n-1} + 2n^2 x^{n-1}
        = 2n x^{n-1}
        = 2e.x^n
        = [h,e].x^n
      \shortintertext{and}
        h.f.x^n - f.h.x^n
        = 2n(n+1) x^{n+1} - 2 n^2 x^{n+1}
        = 2n x^{n+1}
        = -2f.x^n
        = [h,f].x^n
      \end{gather*}
      for every~$n \geq 0$.
    \item
      If~$\rho \colon \glie \to \gllie(V)$ is a representation of a Lie~algebra~$\glie$ and~$\phi \colon \hlie \to \glie$ a homomorphism of Lie~algebras then via the composition~$\rho \circ \phi \colon \hlie \to \gllie(V)$ the vector space~$V$ becomes a representation of~$\hlie$.
      This homomorphism corresponds to the action given by
      \[
        x.v = \rho(x).v = \rho(\phi(x))(v)
      \]
      for all~$x \in \hlie$, and~$v \in V$.
    \item
      The map~$\ad \colon \glie \to \gllie(\glie)$,~$x \mapsto \ad(x)$ is a homomorphism of Lie~algebras and hence a representation of~$\glie$.
  \end{enumerate}
\end{examples}


\begin{definition}
  Let~$\glie$ be a Lie~algebra.
  The  Lie~algebra homomorphism
  \[
    \ad
    \colon
    \glie
    \to
    \gllie(\glie) \,,
    \quad
    x
    \mapsto
    \ad(x)
    =
    [x,-]
  \]
  is the \defemph{adjoint representation}\index{adjoint representation} of~$\glie$.
\end{definition}


\begin{remark}
  It follows together with \cref{lie algebras act adjoint by derivations} that every Lie~algebras~$\glie$ acts on itself by derivations of itself via the adjoint representation.
  The author suspects that this is where much of the structure of Lie~algebras comes from and why the Jacobi identity is of interest.
\end{remark}





\subsection{New Representations from Old Ones}


\begin{proposition}
  \label{new representations from old ones}
  Let~$\glie$ be a Lie~algebra over an arbitrary field~$\kf$.
  \begin{enumerate}
    \item
      If~$(V_i)_{i \in I}$ is a collection of representations of~$\glie$ then the product~$\prod_{i \in I} V_i$ is again a representation of~$\glie$ via
      \[
        x.(v_i)_{i \in I}
        =
        ( x.v_i )_{i \in I}
      \]
      or all~$x \in \glie$ and~$(v_i)_{i \in I} \in \prod_{i \in I} V_i$.
    \item
      If~$(V_i)_{i \in I}$ is a collection of representations of~$\glie$ then the direct sum~$\bigoplus_{i \in I} V_i$ is again a representation of~$\glie$ via
      \[
        x.(v_i)_{i \in I}
        =
        ( x.v_i )_{i \in I}
      \]
      or all~$x \in \glie$ and~$(v_i)_{i \in I} \in \bigoplus_{i \in I} V_i$.
    \item
      If~$V$ and~$W$ are two representations of~$\glie$ then their tensor product~$V \tensor W$ is again a representation of~$\glie$ via
      \[
        x.(v \tensor w)
        =
        (x.v) \tensor w + v \tensor (x.w)
      \]
      for all~$x \in \glie$,~$v \in V$ and~$w \in W$.
      More generally, if~$V_1, \dotsc V_n$ are representations of~$\glie$ then their tensor product~$V_1 \tensor \dotsb \tensor V_n$ is again a representation of~$\glie$ via
      \[
        x.(v_1 \tensor \dotsb \tensor v_n)
        = \sum_{i=1}^n
                  v_1
          \tensor \dotsb
          \tensor v_{i-1}
          \tensor (x.v_i)
          \tensor v_{i+1}
          \tensor \dotsb
          \tensor v_n
      \]
      for every~$x \in \glie$ and all~$v_i \in V_i$ with~$i = 1, \dotsc, n$.
    \item
      If~$V$ and~$W$ are two representations of~$\glie$ then~$\Hom_\kf(V,W)$ is again a representation of~$\glie$ via
      \[
        (x.f)(v)
        =
        x.f(v) - f(x.v)
      \]
      for all~$x \in \glie$,~$f \in \Hom_\kf(V,W)$ and~$v \in V$.
    \item
      Every~{\vectorspace{$\kf$}}~$V$ becomes a representation of~$\glie$ via the \defemph{trivial action}\index{trivial action}
      \[
        x.v
        =
        0
      \]
      for all~$x \in \glie$ and~$v \in V$.
    \item
      By letting~$\glie$ act trivially on~$\kf$ the dual~$V^* = \Hom_\kf(V, \kf)$ becomes a representation of~$\glie$ in the above way, i.e.\ via
      \[
        (x.\phi)(v)
        =
        -\phi(x.v)
      \]
      for all~$x \in \glie$ and~$v \in V$.
  \end{enumerate}
\end{proposition}


\begin{proof}
  \leavevmode
  \begin{enumerate}
    \item
      Let~$x, y \in \glie$ and~$(v_i)_{i \in I} \in \prod_{i \in I} V_i$, then
      \begin{align*}
        x.y.(v_i)_{i \in I} - y.x.(v_i)_{i \in I}
        &=
        (x.y.v_i)_{i \in I} - (y.x.v_i)_{i \in I}
        \\
        &=
        (x.y.v_i - y.x.v_i)_{i \in I}
        \\
        &=
        ([x,y].v_i)_{i \in I}
        \\
        &=
        [x,y].(v_i)_{i \in I} \,.
      \end{align*}
    \item
      This follows from the same calculation as for the product~$\prod_{i \in I} V_i$.
    \item
      It sufficies by induction to consider the case~$n = 2$, i.e.\ the tensor product~$V \tensor W$.
      Then
      \begin{align*}
        x.y.(v \tensor w)
        &=
        x.((y.v) \tensor w + v \tensor (y.w))
        \\
        &=
        x.((y.v) \tensor w) + x.(v \tensor (y.w))
        \\
        &=
        (x.y.v) \tensor w + (y.v) \tensor (x.w) + (x.v) \tensor (y.w) + v \tensor (x.y.w)
      \end{align*}
      and therefore
      \begin{align*}
        x.y.(v \tensor w) - y.x.(v \tensor w)
        &=
        (x.y.v) \tensor w + v \tensor (x.y.w) - (y.x.v) \tensor w - v \tensor (y.x.w)
        \\
        &=
        (x.y.v - y.x.v) \tensor w + v \tensor (x.y.w - y.x.w)
        \\
        &=
        ([x,y].v) \tensor w + v \tensor ([x,y].w) \,.
      \end{align*}
    \item
      For all~$x,y \in \glie$,~$f \in \Hom(V,W)$ and~$v \in V$ it holds that
      \begin{align*}
        (x.y.\phi)(v) - (y.x.\phi)(v)
        &=
        -(y.\phi)(x.v) + (x.\phi(y.v)
        \\
        &=
        \phi(y.x.v) - \phi(x.y.v)
        \\
        &=
        -\phi(x.y.v - y.x.v)
        \\
        &=
        -\varphi([x,y].v)
        \\
        &=
        ([x.y].\varphi)(v)  \,.
      \end{align*}
    \item
      It holds for all~$x,y \in \glie$ and~$v \in V$ that
      \[
        x.y.v - y.x.v
        =
        0 - 0
        =
        0
        =
        [x,y].v \,.
      \]
    \item
      This is a combination of the previous two constructions.
    \qedhere
  \end{enumerate}
\end{proof}


\begin{definition}
  Let~$V$ be a representation of a Lie~algebra~$\glie$.
  A \defemph{subrepresentation} of~$V$ is a linear subpace~$U$ of~$V$ such that~$x.u \in U$ for all~$x \in \glie$ and~$u \in U$.
\end{definition}


\begin{remark}
  If~$(V, \rho)$ is a representation of a Lie~algebra~$\glie$ then a linear subspace~$U \subseteq V$ is a subrepresentation if and only if~$U$ is~{\invariant{$\rho(x)$}} for every~$x \in \glie$, in the sense that~$\rho(x)(U) \subseteq U$.
\end{remark}


\begin{examples}
  Let~$\glie$ be a Lie~algebra.
  \begin{enumerate}
    \item
      If~$V$ is any representation of~$\glie$ then the linear subspaces~$0$ and~$V$ itself are subrepresentations.%
    \footnote{These two subrepresentations are often called the \defemph{trivial} ones.
      We will abstain from doing so, as we have already defined the notion of a trivial representation in \cref{trivial representations}.}
    \item
      If~$V$ is a representation of~$\glie$ and~$U_i$ with~$i \in I$ a collection of subrepresentations~$U_i$ of~$V$ then~$\sum_{i \in I} U_i$ is again a subrepresentation of~$V$.
    \item
      If~$(V_i)_{i \in I}$ is any collection of representations of~$\glie$ then the direct sum~$\bigoplus_{i \in I} V_i$ is a subrepresentation of the product~$\prod_{i \in I} V_i$.
    \item
      The subrepresentations of the adjoint representation~$\ad \colon \glie \to \gllie(\glie)$ are precisely the ideals in~$\glie$.
    \item
      If~$V$ is any representation of~$\glie$ then the linear subspace
      \[
        \glie V
        \defined
        \vspan_\kf
        \{
          x.v
          \suchthat
          x \in \glie,
          v \in V
        \}
      \]
      is a subrepresentation of~$\glie$.
    \item
      If~$V$ is any representation of~$\glie$ then the linear subspace of invariants~$V^{\glie}$ is a (trivial) subrepresentation.
    \item
      Let~$f \colon V \to V$ be an endomorphism of a representation~$V$ of~$\glie$.
      Then for any scalar~$\lambda \in \kf$ both the eigenspace
      \[
        V_\lambda
        \defined
        \{
          v \in V
        \suchthat
          f(v)
          =
          \lambda v
        \}
      \]
      and the generalized eigenspace
      \[
        V_{(\lambda)}
        \defined
        \bigcup_{n \geq 0} \ker(f - \lambda \id_V)^n
        =
        \{
          v \in V
        \suchthat
        \text{$(f - \lambda \id_V)^n(v) = 0$ for some~$n \geq 0$}
        \}
      \]
      are subrepresentations of~$V$.
  \end{enumerate}
\end{examples}


\begin{definition}
  \label{trivial representations}
  Let~$V$ be a representation of a Lie~algebra~$\glie$.
  \begin{enumerate}
    \item
      The representation~$V$ is \defemph{trivial}\index{trivial representation} if every~$x \in \glie$ acts by multiplication with zero on~$V$.
    \item
      An element~$v \in V$ is \defemph{\invariant{$\glie$}}\index{invariants}, or simply~\defemph{invariant}, if~$x.v = 0$ for every~$x \in \glie$.
      The set of invariants is denoted by
      \[
        \gls*{invariants}
        \defined
        \{
          v \in V
        \suchthat
          \text{$x.v = 0$ for every~$x \in \glie$}
        \}  \,.
      \]
  \end{enumerate}  
\end{definition}


% TODO: Explain connection to representations of groups.


\begin{lemma}
  The space of invariants~$V^{\glie}$ is for every representation~$V$ of~$\glie$ the maximal invariant subrepresentation of~$V$.
  \qed
\end{lemma}


\begin{example}
  The invariants of the adjoint representation (of~$\glie$ on itself) are given by
  \begin{align*}
    \glie^{\glie}
    &=
    \{
      y \in \glie
    \suchthat
      \text{$x.y = 0$ for every~$x \in \glie$}
    \}
    \\
    &=
    \{
      y \in \glie
    \suchthat
      \text{$\ad(x)(y) = 0$ for every~$x \in \glie$}
    \}
    \\
    &=
    \{
      y \in \glie
    \suchthat
      \text{$[x,y] = 0$ for every~$x \in \glie$}
    \}
    \\
    &=
    \centerlie(\glie) \,.
  \end{align*}
\end{example}


\begin{example}[Quotient representations]
  \label{quotient representation}
  Let~$V$ be a representation of a Lie~algebra~$\glie$ and let~$U$ be a subrepresentation of~$V$.
  Then the quotient vector space~$V/U$ inherits from~$V$ the structure of a~{\representation{$\glie$}} via
  \[
    x.\class{v}
    =
    \class{x.v}
  \]
  for all~$x \in \glie$ and~$v \in V$.
  Indeed, we have for all~$x, y \in \glie$ and~$\class{v} \in V$ that
  \[
    [x,y].\class{v}
    =
    \class{[x,y].v}
    =
    \class{x.(y.v) - y.(x.v)}
    =
    \class{x.(y.v)} - \class{y.(x.v)}
    =
    x.(y.\class{v}) - y.(x.\class{v}) \,.
  \]
  
  Alternatively let~$\rho \colon \glie \to \gllie(V)$ be the Lie~algebra homomorphism corresponding to the action of~$\glie$ on~$V$.
  Then the linear subspace~$U$ of~$V$ is~{\invariant{$\rho(x)$}} for every~$x \in \glie$, and hence the endomorphism~$\rho(x)$ induces an endomorphism
  \[
    \induced{\rho(x)}
    \colon
    V/U
    \to
    V/U \,,
    \quad
    \class{v}
    \mapsto
    \class{\rho(x)(v)}
  \]
  for every~$x \in \glie$.
  The resulting map~$\induced{\rho} \colon \glie \to \gllie(V/U)$ given by~$\induced{\rho}(x) = \induced{\rho(x)}$  is a homomorphism of Lie~algebras because
  \[
    \induced{\rho}([x,y])
    =
    \induced{\rho([x,y])}
    =
    \induced{[\rho(x), \rho(y)]}
    =
    [\induced{\rho(x)}, \induced{\rho(y)}]
    =
    [\induced{\rho}(x), \induced{\rho}(y)]
  \]
  for all~$x, y \in \glie$.
\end{example}


\begin{definition}
  Let~$V$ be a representation of a Lie~algebra~$\glie$ and let~$U$ be a subrepresentation of~$V$.
  The representation~\gls*{quotient representation} from \cref{quotient representation} is the \defemph{quotient representation}\index{quotient!representations}\index{representation!quotient} of~$V$ by~$U$.
\end{definition}





\subsection{Homomorphisms of Representations}


\begin{definition}
  Let~$V$ und~$W$ be two representations of a~{\liealgebra{$\kf$}}~$\glie$.
  \begin{enumerate}
    \item
      A~{\linear{$\kf$}} map~$f \colon V \to W$ is a \defemph{homomorphism of representations}\index{homomorphism!of representations} if
      \[
        f(x.v) = x.f(v)
      \]
      for all~$x \in \glie$ and~$v \in V$.
    \item
      A homomorphism of representations~$f$ is an \defemph{isomorphism of representations}\index{isomorphism!of representations} if it is bijective.
    \item
      The space of homomorphism~$V \to W$ is denoted by~\gls*{rep homo}, and for~$V = W$ by~\gls*{rep endo}.
  \end{enumerate}
\end{definition}

\begin{remark}
  Let~$V$,~$W$ and~$U$ be representations of a Lie~algebra~$\glie$ over~$\kf$.
  \begin{enumerate}
    \item
      The notions of a \defemph{monomorphism}\index{monomorphism of representations}, \defemph{epimorphism}\index{epimorphism of representations}, \defemph{endomorphism}\index{endomorphism of representations} and \defemph{automorphism}\index{automorphism of representations} are defined in the usual way.
    \item
      If the representations~$V$ and~$W$ are given by the Lie~algebra homomorpisms~$\rho_V \colon \glie \to \gllie(V)$ and~$\rho_W \colon \glie \to \gllie(W)$ then a~{\linear{$\kf$}} map~$f \colon V \to W$ is a homomorphism of representations if and only if~$f \circ \rho_V(x) = \rho_W(x) \circ f$ for all~$x \in \glie$, i.e.\ if and only if the square diagram
      \[
        \begin{tikzcd}[column sep = large]
          V
          \arrow{r}[above]{\rho_V(x)}
          \arrow{d}[left]{f}
          &
          V
          \arrow{d}[right]{f}
          \\
          W
          \arrow{r}[above]{\rho_W(x)}
          &
          W
        \end{tikzcd}
      \]
      commutes for every~$x \in \glie$.
    \item
      If~$f, g \colon V \to W$ are homomorphisms of representations then~$f + g$ is again a homomorphism of representations, and~$\lambda f$ is for every~$\lambda \in \kf$ again a homomorphism of representations.
      Hence~$\Hom_{\glie}(V,W)$ is a~{\linear{$\kf$}} subspace of~$\Hom_\kf(V,W)$.
    \item
      If~$f \colon V \to W$ is an isomorphism of representations then its inverse~$f^{-1} \colon W \to V$ is again a homomorphism (and thus isomorphism) of representations.
      Indeed, we find that
      \[
        f^{-1}(x.v)
        =
        f^{-1}(x.f(f^{-1}(v)))
        =
        f^{-1}(f(x.f^{-1}(v)))
        =
        x.f^{-1}(v)
      \]
      for all~$x \in \glie$ and~$v \in V$.
    \item
      The identity~$\id_V \colon V \to V$ is always an automorphism of the representation~$V$.
    \item
      If~$f \colon V \to W$ and~$g \colon W \to U$ are homomorphism of representations then their composition~$g \circ f \colon V \to U$ is again a homomorphism of representations.
    \item
      It follows that the representations of~$\glie$ together with the homomorphisms of representations between them form a (\linear{$\kf$}) category.
      We will denote this category by~\gls*{representation category}.
  \end{enumerate}
\end{remark}


\begin{lemma}
  Let~$V$ and~$W$ be two representations of a Lie~algebra~$\glie$ let~$f \colon V \to W$ be a homomorphism of representations.
  Then the kernel of~$f$ is a subrepresentation of~$V$ while the image of~$f$ is a subrepresentation~$W$.
  \qed
\end{lemma}


\begin{remark}
  \label{homomorphisms of representations as invariants}
  Given two representations~$V$ and~$W$ of a Lie~algebra~$\glie$ a linear map~$f \colon V \to W$ is a homomorphism of representations if and only if it is invariant under the induced action of~$\glie$ on~$\Hom(V,W)$:
  Indeed, we see that
  \begin{align*}
        {}& \text{$f$ is a homomorphism}  \\
    \iff{}& \text{$f(x.v) = x.f(v)$ for all~$x \in \glie$ and~$v \in V$}  \\
    \iff{}& \text{$f(x.v) - x.f(v) = 0$ for all~$x \in \glie$ and~$v \in V$}  \\
    \iff{}& \text{$(x.f)(v) = 0$ for all~$x \in \glie$ and~$v \in V$} \\
    \iff{}& \text{$x.f = 0$ for every~$x \in \glie$}  \\
    \iff{}& \text{$f$ is invariant} \,.
  \end{align*}
  Hence~$\Hom_{\glie}(V,W) = \Hom_\kf(V,W)^{\glie}$.
\end{remark}


\begin{proposition}
  \label{list of homomorphism of representations}
  Let~$\glie$ be a Lie algebra.
 \begin{enumerate}
    \item
      For every homomorphism of representations~$f \colon V \to W$ the dual linear map
      \[
        f^*
        \colon
        W^*
        \to
        V^* \,,
        \quad
        \phi
        \mapsto
        f \circ \phi
      \]
      is again a homomorphism of representations.
    \item
      For any two representations~$V$ and~$W$ of~$\glie$ the natural linear map
      \[
        \Phi_1
        \colon
        V^* \tensor W
        \to
        \Hom_k(V,W) \,,
        \quad
        \phi \tensor w
        \mapsto
        (v \mapsto \phi(v) w)
      \]
      is a homomorphism of representations.
      If at least one of the two representations~$V$ and~$W$ is finite dimensional then it is an isomorphism of representations.
    \item
      For all representations~$V_1, \dotsc, V_n$ and~$W_1, \dotsc, W_m$ of~$\glie$ the natural isomorphism of vector spaces
      \begin{align*}
        \Phi_2
        \colon
        (V_1 \tensor \dotsb \tensor V_n) \tensor (W_1 \tensor \dotsb \tensor W_m)
        &\longto
        V_1 \tensor \dotsb \tensor V_n \tensor W_1 \tensor \dotsb \tensor W_m \,,
        \\
        (v_1 \tensor \dotsb \tensor v_n) \tensor (w_1 \tensor \dotsb \tensor w_m)
        &\longmapsto
        v_1 \tensor \dotsb \tensor v_n \tensor w_1 \tensor \dotsb \tensor w_m
      \end{align*}
      is already an isomorphism of representations.
    \item
      For any two representations~$V$ and~$W$ of~$\glie$ the natural isomorphism of vector spaces
      \[
        \Phi_3
        \colon
        V \tensor W
        \to
        W \tensor V \,,
        \quad
        v \tensor w
        \mapsto
        w \tensor v
      \]
      is already an isomorphism of representations.
    \item
      For all representations~$V_1$,~$V_2$ and~$W$ of~$\glie$ the natural isomorphism of vector spaces
      \begin{align*}
        \Phi_4
        \colon
        (V_1 \tensor V_2) \tensor W
        &\longto
        (V_1 \tensor W) \oplus (V_2 \tensor W) \,,
        \\
        (v_1, v_2) \tensor w
        &\longmapsto
        (v_1 \tensor w, v_2 \tensor w)
      \end{align*}
      is already an isomorphism of representations.
    \item
      If~$V_1, \dotsc, V_n$ are representations of~$\glie$ and~$\sigma \in S_n$ is any permuation then the natural isomorphism of vector spaces
      \begin{align*}
        \Phi_5
        \colon 
        V_1 \tensor \dotsb \tensor V_n
        &\longto
        V_{\sigma(1)} \tensor \dotsb \tensor V_{\sigma(n)} \,,
        \\
        v_1 \tensor \dotsb \tensor v_n
        &\longmapsto
        v_{\sigma(1)} \tensor \dotsb \tensor v_{\sigma(n)}
      \end{align*}
      is already an isomorphism of representations.
    \item
      If~$V_1, \dotsc, V_n$ and~$W_1, \dotsc, W_n$ are representations of~$\glie$ and~$f_i \colon V_i \to W_i$ with~$i = 1, \dotsc, n$ are homomorphisms of representations then the natural linear map
      \begin{align*}
        f_1 \tensor \dotsb \tensor f_n
        \colon
        V_1 \tensor \dotsb \tensor V_n
        &\longto
        W_1 \tensor \dotsb \tensor W_n
        \\
        v_1 \tensor \dotsb \tensor v_n
        &\longmapsto
        f(v_1) \tensor \dotsb \tensor f(v_n)
      \end{align*}
      is already a homomorphism of representations.
    \qed
  \end{enumerate}
\end{proposition}


\begin{proposition}[Homomorphism theorem]
  \label{homomorphism theorem!for representations}
  Let~$V$ be a representation of a Lie~algebra~$\glie$ and let~$U$ be a subrepresentation of~$V$.
  Let~$W$ be another representation of~$\glie$.
  For every homomorphism of representations~$f \colon V \to W$ with~$U \subseteq \ker f$ there exists a unique homomorphism of representations~$\induced{f} \colon V/U \to W$ that makes the triangular diagram
  \[
    \begin{tikzcd}
      V
      \arrow{r}[above]{f}
      \arrow{d}[left]{\pi}
      &
      W
      \\
      V/U
      \arrow[dashed]{ur}[below right]{\induced{f}}
      &
      {}
    \end{tikzcd}
  \]
  commute.
  It holds that~$\ker \induced{f} = {\ker f}/I$ and~$\im \induced{f} = \im f$.
\end{proposition}


\begin{corollary}[Isomorphism theorems]
  \index{isomorphism theorems!for representations}
  Let~$V$ be a representation of a Lie~algebra~$\glie$.
  \begin{enumerate}
    \item
      If~$W$ is another representation of~$\glie$ and~$f \colon V \to W$ is any homomorphism of representations then~$f$ induces a unique well-defined isomorphism of representations
      \[
        \induced{f}
        \colon
        V/{\ker f}
        \to
        \im f \,,
        \quad
        \class{v}
        \mapsto
        f(v)  \,.
      \]
    \item
      If~$U$ and~$W$ are subrepresentations of~$V$ with~$U \subseteq W$ then~$W/U$ is a subrepresentation of~$V/U$ and the natural isomorphism of vector spaces
      \[
        (V/U) / (W/U)
        \to
        V/W \,,
        \quad
        \class{v}
        \mapsto
        \class{v}
      \]
      is already an isomorphism of representations.
    \item
      If~$U$ and~$W$ are subrepresentations of~$V$ then~$W$ is a subrepresentation of~$U+W$ and~$U \cap W$ is a subrepresentation of~$U$, and the natural isomorphism of vector spaces
      \[
        U/(U \cap W)
        \to
        (U + W)/W  \,,
        \quad
        \class{u}
        \mapsto
        \class{u}
      \]
      is already an isomorphism of representations.
  \end{enumerate}
\end{corollary}


\begin{proposition}[Correspondence theorem]
  \label{correspondence theorem!for representations}
  Let~$V$ be a represenation of a Lie~algebra~$\glie$ and let~$U \subseteq V$ be a subrepresentation.
  Let~$\pi \colon V \to V/U$ the canonical projection.
  
  If~$W$ is a subrepresentation of~$V$ that contains~$U$ then the quotient~$W/U$ is a subrepresentation of~$V/U$.
  This construction results in a {\onetoone} correspondence
  \begin{align*}
    \{ \text{subrepresentations~$W \subseteq V$ containing~$U$} \}
    &\longleftrightarrow
    \{ \text{subrepresentations of~$V/U$} \}  \,,
    \\
    W
    &\longmapsto
    W/U \,,
    \\
    \pi^{-1}(W')
    &\longmapsfrom
    W'  \,.
  \end{align*}
  If~$W$ is a subrepresentation of~$V$ containing~$U$ then it holds for the associated subrepresentation~$W/U$ of~$V/U$ that~$(V/U)/(W/U) \cong V/W$.
\end{proposition}





\subsection{Irreducible and Semisimple Representations}


\begin{definition}
  Let~$V$ be a representation of a Lie~algebra~$\glie$.
  \begin{enumerate}
    \item
      The representation~$V$ is \defemph{irreducible}\index{irreducible representation}\index{representation!irreducible} or \defemph{simple}\index{simple!representation}\index{representation!simple} if it is nonzero and admits only the subrepresentations~$0$ and~$V$ itself.
    \item
      The representation~$V$ is \defemph{indecomposable}\index{indecomposable representation}\index{representation!indecomposable} if there does not exists a decomposition~$V = U_1 \oplus U_2$ into subrepresentations~$U_1$ and~$U_2$ apart from~$V = V \oplus 0$ and~$V = 0 \oplus V$.
      Otherwise~$V$ is \defemph{decomposable}\index{decomposable representation}\index{representation!decomposable}
    \item
      The representation~$V$ is~\defemph{completely reducible}\index{completely reducible representation}\index{representation!completely reducible} or~\defemph{semisimple}\index{semisimple!representation}\index{representation!semisimple} if it has a decomposition~$V = \bigoplus_{i \in I} U_i$ into irreducible subrepresentations~$U_i$.
  \end{enumerate}
\end{definition}


\begin{remark}
  \leavevmode
  \begin{enumerate}
    \item
      Every irreducible representation indecomposable, but the converse does not hold.
    \item
      A representation is irreducible if and only if it is both indecomposable and completely reducible.
    \item
      For any representation~$V$ the following conditions are equivalent:
      \begin{equivalenceslist}
        \item
          $V$ is semisimple, i.e.~$V$ admits a decomposition~$V = \bigoplus_{i \in I} U_i$ into irreducible subrepresentations~$U_i$.
        \item
          $V$ can be written as a (not necessarily direct) sum~$V = \sum_{i \in I} U_i$ for irreducible subrepresentations~$U_i$.
        \item
          Every subrepresentation~$U$ of~$V$ admits a direct complement, i.e.\ there exists a subrepresentation~$W$ of~$V$ with~$V = U \oplus W$.
      \end{equivalenceslist}
  \end{enumerate}
\end{remark}


\begin{example}
  \leavevmode
  \begin{enumerate}
    \item
      Every {\onedimensional} representation is irreducible.
    \item
      The adjoint representation of a Lie~algebra~$\glie$ is irreducible if and only if~$\glie$ is nonzero and contains no ideals beside~$0$ and~$\glie$ itself.
      This is the case if and only if~$\glie$ is either {\onedimensional} and abelian, or simple.
  \end{enumerate}
\end{example}


\begin{lemma}[Schur]
  \index{Schur’s Lemma}
  Let~$V$ and~$W$ be representations of a Lie~algebra~$\glie$ and let~$f \colon V \to W$ be a homomorphism of representations.
  \begin{enumerate}
    \item
      If~$V$ is irreducible then either~$f$ is injective or~$f = 0$, but not both.
    \item
      If~$W$ is irreducible then either~$f$ is surjective or~$f = 0$, but not both.
    \item
      If both~$V$ and~$W$ are irreducible then either~$f$ is bijective or~$f = 0$, but not both.
    \item
      If~$V$ is irreducible then the endomorphism algebra~$\End_{\glie}(V)$ is a skew field over~$\kf$.
    \item
      If the field~$\kf$ is algebraically closed and~$V$ finite dimensional and irreducible then every endomorphism~$f \in \End_{\glie}(V)$ is given by multiplication with some scalar~$\lambda \in \kf$.
      In particular~$\End_{\glie}(V) = \kf$.
  \end{enumerate}
\end{lemma}


\begin{proof}
  \leavevmode
  \begin{enumerate}
    \item
      The kernel~$\ker f$ is a subrepresentation of~$V$ and so either~$\ker f = 0$ or~$\ker f = V$, but not both.
    \item
      The image~$\im f$ is a subrepresentation of~$W$ and so either~$\im f = W$ or~$\im f = 0$, but not both.
    \item
      This is a combination of the previous two statements.
    \item
      This is a reformulation of the previous statement;
      note that~$\End_{\glie}(V) \neq 0$ because~$\id_V \neq 0$.
    \item
      The endomorphism~$f$ admits an eigenvalue~$\lambda \in \kf$ because~$\kf$ is algebraically closed.
      The endomorphism~$f - \lambda \id_V$ is non-injective and hence~$f - \lambda \id_V = 0$ as seen above.
    \qedhere
  \end{enumerate}
\end{proof}




