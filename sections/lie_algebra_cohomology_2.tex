\chapter{Lie Algebra (Co)homology II}
\label{lie algebra cohomology again}





\section{The Koszul Resolution}
\label{koszul resolution}

\begin{convention}
  We fix for this \lcnamecref{koszul resolution} a Lie~algebra~$\glie$.
  For every natural number~$n$ we regard~$\Univ(\glie) \otimes_{\kf} \Exterior^n(\glie)$ as an~\module{$\Univ(\glie)$} via extension of scalars, i.e. via the multiplication given by
  \[
    x \cdot (y \otimes t)
    =
    (xy) \otimes y
  \]
  for all~$x, y \in \Univ(\glie)$,~$t \in \Exterior^n(\glie)$.
\end{convention}


\begin{lemma}
  \label{extension of scalars is free}
  The module~$\Univ(\glie) \otimes_{\kf} \Exterior^n(\glie)$ is free for all~$n \geq 0$.
\end{lemma}


\begin{proof}
  The exterior power~$\Exterior^n(\glie)$ is free as a~\module{$\kf$}.
  The extension of scalars~$\Univ(\glie) \otimes_{\kf} \Exterior^n(\glie)$ is therefore free as a~\module{$\Univ(\glie)$}.
\end{proof}


\begin{recall}
  Let~$A$ be a~\algebra{$\kf$} and let~$M$ be an~\module{$A$}.
  A \defemph{resolution} of~$M$ is an exact sequence of the form
  \[
    \dotsb
    \to
    M_2
    \xto{d_2}
    M_1
    \xto{d_1}
    M_0
    \xto{\varepsilon}
    M
    \to
    0 \,.
  \]
  Such a resolution is \defemph{free} if the modules~$M_i$ are free for all~$i \geq 1$.
\end{recall}


\begin{proposition}
  \label{construction of koszul differential}
  \leavevmode
  \begin{enumerate}
    \item
      There exists for every natural number~$n$ with~$n \geq 1$ a unique linear map
      \[
        d_n
        \colon
        \Univ(\glie) \otimes_{\kf} \Exterior^n(\glie)
        \to
        \Univ(\glie) \otimes_{\kf} \Exterior^{n-1}(\glie)
      \]
      given by
      \begin{align*}
        d_n( y \otimes x_1 \wedge \dotsb \wedge x_n )   
        ={}&
        \sum_{1 \leq i < j \leq n}
        (-1)^{i+j}
        y \otimes [x_i, x_j] \wedge x_1 \wedge \dotsb \wedge \widehat{x_i} \wedge \dotsb \wedge \widehat{x_j} \wedge \dotsb \wedge x_n
        \\
        {}&
        +
        \sum_{i=1}^n
        (-1)^{i+1}
        (y x_i) \otimes x_1 \wedge \dotsb \wedge \widehat{ x_i } \wedge \dotsb \wedge x_n
      \end{align*}
      for all~$y \in \Univ(\glie)$,~$x_1, \dotsc, x_n \in \glie$.
    \item
      These maps are homomorphism of~\modules{$\Univ(\glie)$}.
    \item
      These maps satisfy the condition~$d_{n-1} \circ d_n = 0$ for all~$n \geq 2$.
  \end{enumerate}
\end{proposition}


\begin{proof}
  The have a right~\module{$\Univ(\glie)$} structure on~$\Univ(\glie)$ given by right multiplication.
  This module structure corresponds to a right action of~$\glie$ on~$\Univ(\glie)$ given by
  \[
    y \act x
    =
    yx
  \]
  for all~$x \in \glie$,~$y \in \Univ(\glie)$.
  This right action corresponds to a left action given by
  \[
    x \act y
    =
    -yx
  \]
  for all~$y \in \glie$,~$y \in \Univ(\glie)$.
  We have in this way made~$\Univ(\glie)$ into a representation of~$\glie$.

  We can thus form the Lie algebra chain complex~$\Chain_\bullet(\glie, \Univ(\glie))$, given by the vector spaces
  \[
    \Chain_n(\glie, \Univ(\glie))
    =
    \Exterior^n(\glie) \otimes_{\kf} \Univ(\glie)
  \]
  for all~$n \geq 0$, and the differentials
  \[
    \widetilde{d_n}
    \colon
    \Exterior^n(\glie) \otimes_{\kf} \Univ(\glie)
    \to
    \Exterior^{n-1}(\glie) \otimes_{\kf} \Univ(\glie)
  \]
  given by
  \begin{align*}
    \widetilde{d_n}( x_1 \wedge \dotsb \wedge x_n \otimes y )
    ={}&
    \sum_{1 \leq i < j \leq n}
    (-1)^{i+j}
    [x_i, x_j] \wedge x_1 \wedge \dotsb \wedge \widehat{x_i} \wedge \dotsb \wedge \widehat{x_j} \wedge \dotsb \wedge x_n \otimes y
    \\
    {}&
    +
    \sum_{i=1}^n (-1)^i x_1 \wedge \dotsb \wedge \widehat{x_i} \wedge \dotsb \wedge x_n \otimes (x_i \act y)
  \end{align*}
  for all~$x_1, \dotsc x_n \in \glie$,~$y \in \Univ(\glie)$.
  By using the isomorphism of vector spaces
  \[
    \Exterior^n(\glie) \otimes_{\kf} \Univ(\glie)
    \cong
    \Univ(\glie) \otimes_{\kf} \Exterior^n(\glie)
  \]
  and the formula~$x \act y = -yx$ for all~$x \in \glie$,~$y \in \Univ(\glie)$ we arrive at a chain complex as described. 

  It remains to check that the maps~$d_n$ are homomorphisms of~\modules{$\Univ(\glie)$}.
  This holds because
  \begin{align*}
    \SwapAboveDisplaySkip
    {}&
    d_n(x \cdot (y \otimes x_1 \wedge \dotsb \wedge x_n) )
    \\
    ={}&
    d_n( (xy) \otimes x_1 \wedge \dotsb \wedge x_n )
    \\
    ={}&
    \sum_{1 \leq i < j \leq n}
    (-1)^{i+j}
    (xy) \otimes [x_i, x_j] \wedge x_1 \wedge \dotsb \wedge \widehat{x_i} \wedge \dotsb \wedge \widehat{x_j} \wedge \dotsb \wedge x_n
    \\
    {}&
    +
    \sum_{i=1}^n
    (-1)^{i+1}
    (x y x_i) \otimes x_1 \wedge \dotsb \wedge \widehat{ x_i } \wedge \dotsb \wedge x_n
    \\
    ={}&
    x
    \cdot
    \Biggl(
    \sum_{1 \leq i  j \leq n}
      (-1)^{i+j}
      y \otimes [x_i, x_j] \wedge x_1 \wedge \dotsb \wedge \widehat{x_i} \wedge \dotsb \wedge \widehat{x_j} \wedge \dotsb \wedge x_n
    \\
    {}&
    \phantom{
      x \cdot
      \Biggl(
    }
      +
      \sum_{i=1}^n
      (-1)^{i+1}
      (y x_i) \otimes x_1 \wedge \dotsb \wedge \widehat{ x_i } \wedge \dotsb \wedge x_n
    \Biggr)
    \\
    ={}&
    x \cdot d_n(y \otimes x_1 \wedge \dotsb \wedge x_n)
  \end{align*}
  for all~$x \in \glie$,~$y \in \Univ(\glie)$,~$x_1, \dotsc, x_n \in \glie$.
\end{proof}


\begin{fluff}
  We use in the following the identifications
  \[
    \Univ(\glie) \otimes_{\kf} \Exterior^0(\glie)
    \cong
    \Univ(\glie) \otimes_{\kf} \kf
    \cong
    \Univ(\glie)
  \]
  and
  \[
    \Univ(\glie) \otimes_{\kf} \Exterior^1(\glie)
    \cong
    \Univ(\glie) \otimes \glie \,.
  \]
  We regard~$\kf$ as a trivial representation of~$\glie$ and thus an a~\module{$\Univ(\glie)$}.
\end{fluff}


\begin{theorem}
  \label{koszul is a free resolution}
  \leavevmode
  The sequence
  \begin{equation}
    \label{koszul complex written out}
    \dotsb
    \to
    \Univ(\glie) \otimes_{\kf} \Exterior^2(\glie)
    \xto{d_2}
    \Univ(\glie) \otimes_{\kf} \glie
    \xto{d_1}
    \Univ(\glie)
    \xto{\varepsilon}
    \kf
    \to
    0
  \end{equation}
  is a free resolution of~$\kf$ as an~\module{$\Univ(\glie)$}, where~$\varepsilon$ denotes the counit of~$\Univ(\glie)$.
\end{theorem}


\begin{definition}
  The free resolution from \cref{koszul is a free resolution} is the \defemph{Koszul resolution} or \defemph{Chevalley--Eilenberg-complex} of~$\glie$.
\end{definition}


\begin{fluff}
  We will spend the rest of this \lcnamecref{koszul resolution} on proving \cref{koszul is a free resolution}.
  The presented proof is taken from \cite[IV.6]{knapp}.
  The reader who is familiar with spectral sequences may also check out the proof in \cite[Theorem~7.2.2]{weibel_homological_algebra}.

  We will proceed in four steps.
  We first prove \cref{koszul is a free resolution} under the assumption that the Lie~algebra~$\glie$ is abelian and finite-dimensional.
  Then we prove \cref{koszul is a free resolution} under the assumption that~$\glie$ is abelian, but possibly infinite-dimensional.
  Finally we prove \cref{koszul is a free resolution} without restriction on~$\glie$.
\end{fluff}


\begin{proof}[Proof of \cref{koszul is a free resolution}, general part]
  It follows from~\cref{construction of koszul differential} that~$d_{n-1} \circ d_n = 0$ for all~$n \geq 2$.
  The map~$d_1$ is given by
  \[
    d_1( y \otimes x )
    =
    yx
  \]
  for all~$y \in \Univ(\glie)$,~$x \in \glie$.
  It follows from~\cref{augumentation ideal is spanned by monomials} that the image of~$d_1$ is the augumentation ideal of~$\Univ(\glie)$ and thus the kernel of~$\varepsilon$.
  This shows the exactness at~$\Univ(\glie)$.
  The given sequence is moreover exact at~$\kf$ because the map~$d_0$ is surjective, since~$\kf$ is one-dimensional and the image~$d_0(1)$ is nonzero.

  This shows that the given sequence is a chain complexe, which is exact at~$\Univ(\glie)$ and~$\kf$.

  The terms~$\Univ(\glie) \otimes_{\kf} \Exterior^n(\glie)$ with~$n \geq 0$ are free~\modules{$\Univ(\glie)$} according to \cref{extension of scalars is free}.

  The maps~$d_n$ for~$n \geq 1$ are homomorphism of~\modules{$\Univ(\glie)$} according to \cref{construction of koszul differential}.
  The counit~$\varepsilon$ is also a homomorphism of~\modules{$\Univ(\glie)$}.
  Indeed, every element~$x$ of~$\Univ(\glie)$ acts on~$\kf$ by multiplication with the scalar~$\varepsilon(x)$, whence
  \[
    \varepsilon(x \cdot y)
    =
    \varepsilon(x) \cdot \varepsilon(y)
    =
    x \cdot \varphi(y)
  \]
  for all~$y \in \Univ(\glie)$.

  It remains to check the exactness of the given complex at~$\Univ(\glie) \otimes \Exterior^n(\glie)$ for~$n \geq 1$.
\end{proof}


\begin{recall}
  \label{exactness in ses of chain complex}
  Let
  \[
    0 \to X_\bullet \to Y_\bullet \to Z_\bullet \to 0
  \]
  be a short exact sequence of chain complexes.
  If the chain complexes~$X_\bullet$ and~$Z_\bullet$ are exact (i.e. they are exact sequences) then the chain complex~$Y_\bullet$ is also a chain complex.
  Indeed, in the resulting long exact homology sequence
  \[
    \dotsb
    \to
    \Homology_{n+1}(Z_\bullet)
    \to
    \Homology_n(X_\bullet)
    \to
    \Homology_n(Y_\bullet)
    \to
    \Homology_n(Z_\bullet)
    \to
    \Homology_{n-1}(X_\bullet)
    \to
    \dotsb
  \]
  both~$\Homology_n(X_\bullet)$ and~$\Homology_n(Z_\bullet)$ vanish for all~$n \in \Integer$.
  It then follows from the exactness of this sequence that also~$\Homology_n(Y_\bullet)$ vanishes for all~$n \in \Integer$.
\end{recall}


\begin{recall}
  Let~$X_\bullet$ be a chain complex and let~$Y_\bullet$ be a subcomplex of~$X_\bullet$.
  We can then form the quotient chain complex~$X_\bullet / Y_\bullet$.
  This chain complex is given by the vector spaces~$X_n / Y_n$ for all~$n \in \Integer$.
  The differentials of~$X_\bullet / Y_\bullet$ are given by the unique linear maps that make the diagrams
  \[
    \begin{tikzcd}
      X_n
      \arrow{d}
      \arrow{r}[above]{d_n}
      &
      X_{n-1}
      \arrow{d}
      \\
      Y_n
      \arrow{r}[above]{d_n}
      \arrow{d}
      &
      Y_{n-1}
      \arrow{d}
      \\
      X_n / Y_n
      \arrow[dashed]{r}
      &
      X_{n-1} / Y_{n-1}
    \end{tikzcd}
  \]
  commute for all~$n \in \Integer$.
  The above chain complexes fit into a short exact sequence
  \[
    0
    \to
    Y_\bullet
    \to
    X_\bullet
    \to
    X_\bullet / Y_\bullet
    \to
    0 \,.
  \]
\end{recall}


\begin{proof}[Proof of \cref{koszul is a free resolution}, first part]
  Suppose that the Lie~algebra~$\glie$ is abelian and finite-dimensional.
  The maps~$d_n$ are given by
  \[
    d_n( y \otimes x_1 \wedge \dotsb \wedge x_n )
    =
    \sum_{i=1}^n
    (-1)^{i+1}
    (y x_i) \otimes x_1 \wedge \dotsb \wedge \widehat{x_i} \wedge \dotsb \wedge x_n
  \]
  for all~$n \geq 1$,~$y \in \Univ(\glie)$,~$x_1, \dotsc, x_n \in \glie$ because~$\glie$ is abelian.

  Let~$u_1, \dotsc, u_N$ be a basis of~$\glie$.
  The universal enveloping algebra~$\Univ(\glie)$ isomorphic to the symmetric algebra~$\Symm(\glie)$ because~$\glie$ is abelian.
  The algebra~$\Univ\glie)$ thus isomorphic to the polynomial algebra~$\kf[u_1, \dotsc, u_N]$.
  We show the exactness of the given chain complex~\eqref{koszul complex written out} by induction over~$N$.
  For~$N = 0$ we have~$\Univ(\glie) = 0$ and is chain complex is given by
  \[
    \dotsb
    \to
    0
    \to
    0
    \to
    \kf
    \xto{\id}
    \kf
    \to
    0 \,.
  \]
  This sequence is exact, showing the base case.

  For~$N \geq 1$ we use that we have already shown~\eqref{koszul complex written out} to be a chain complex.
  To show that this chain complex is exact we consider for every~$k = 0, \dotsc, n$ the ideal
  \[
    J_k
    \defined
    \ideal{u_1, \dotsc, u_k}
  \]
  of~$\Univ(\glie)$, and the linear subspaces~$V(n,k)$ of~$\Univ(\glie) \otimes \Exterior^n(\glie)$ given by
  \[
    V(n,k)
    \defined
    \Univ(\glie) \otimes \Exterior^n(u_1, \dotsc, u_k)  \,,
  \]
  where we use the abbrevaiton~$\Exterior^n(u_1, \dotsc, u_k)$ for~$\Exterior^n( \gen{u_1, \dotsc, u_k}_{\kf} )$.
  We have
  \[
    d_n( V(n, k) )
    \subseteq
    V(n-1, k)
  \]
  for all~$n \geq 1$,~$k = 0, \dotsc, n$.
  We thus have for every~$k$ a sequence~$X^{(k)}_\bullet$ given by
  \[
    \dotsb
    \to
    V(3,k)
    \xto{d_{3,k}}
    V(2,k)
    \xto{d_{2,k}}
    V(1,k)
    \xto{d_{1,k}}
    \Univ(\glie)
    \xto{\varepsilon_k}
    \Univ(\glie) / J_k
    \to
    0 \,,
  \]
  where~$\Univ(\glie) = V(0,k)$ and~$\varepsilon$ denotes the quotient homomorphism.
  We will show that~$X^{(k)}_\bullet$ is exact for all~$k = 0, \dotsc, n$.
  For~$k = n$ this then shows the exactness of~\eqref{koszul complex written out}.

  We already know that~\eqref{koszul complex written out} is a chain complex.
  It follows for~$X^{(k)}_\bullet$ that we have~$d_{n-1,k} \circ d_{n,k} = 0$ for all~$n \geq 2$.
  We have~$V(1,k) = \Univ(\glie) \otimes \gen{u_1, \dotsc, u_k}_{\kf}$.
  The image of this space under~$d_{1,k}$ is precisely the ideal of~$\Univ(\glie)$ generated by~$u_1, \dotsc, u_k$, i.e. the ideal~$J_k$, which in turn is precisely the kernel of~$\varepsilon_k$.
  The sequence~$X^{(k)}_\bullet$ is thus exact at~$\Univ(\glie)$.
  The projection~$\varepsilon$ is surjective, which shows the exactness of~$X^{(k)}_\bullet$ at~$\Univ(\glie) / J_k$.
  For the exactness of~$X^{(k)}_\bullet$ it hence sufficies to show the exactness of the chain complex~$Y^{(k)}_\bullet$ given by
  \[
    \dotsb
    \to
    V(3,k)
    \xto{d_{3,k}}
    V(2,k)
    \xto{d_{2,k}}
    V(1,k)
    \xto{d_{1,k}}
    J_k
    \to
    0 \,.
  \]
  We show this exactness by induction over~$k$.

  The complex~$Y^{(0)}_\bullet$ is the zero chain complex, which is exact.
  Suppose now that~$k \geq 1$.
  The complex~$Y^{(k-1)}_\bullet$ is exact by induction hypothesis.
  We note that the chain complex~$Y^{(k-1)}_\bullet$ is a subcomplex of~$Y^{(k)}_\bullet$.
  We will show in the following that the quotient complex~$Y^{(k)}_\bullet / Y^{(k-1)}_\bullet$ is exact.
  It then follows from~\cref{exactness in ses of chain complex} that the complex~$Y^{(k)}_\bullet$ is also exact.

  The quotient complex~$Y^{(k)}_\bullet / Y^{(k-1)}_\bullet$ is of the form
  \[
    \dotsb
    \to
    V(2,k) / V(2, k-1)
    \xto{\induced{d_{2,k}}}
    V(1,k) / V(1, k-1)
    \xto{\induced{d_{1,k}}}
    J_k / J_{k-1}
    \to
    0
  \]

  We have
  \begin{align*}
    V(n,k) / V(n, k-1)
    &=
    \Bigl( \Univ(\glie) \otimes \Exterior^n( u_1, \dotsc, u_k ) \Bigr)
    \mathbin{\big/}
    \Bigl( \Univ(\glie) \otimes \Exterior^n( u_1, \dotsc, u_{k-1} ) \Bigr)
    \\
    &\cong
    \Univ(\glie)
    \otimes
    \Bigl(
      \Exterior^n( u_1, \dotsc, u_k )
      \mathbin{\big/}
      \Exterior^n( u_1, \dotsc, u_{k-1} )
    \Bigr) \,.
  \end{align*}
  The exterior power~$\Exterior^n( u_1, \dotsc, u_k )$ has the simple wedges
  \[
    u_{i_1} \wedge \dotsb \wedge u_{i_n}
    \qquad
    \text{with~$1 \leq i_1 < \dotsb < i_n \leq k$}
  \]
  as a basis.
  Those simple wedges with~$1 \leq i_1 < \dotsb < i_n \leq k-1$ form a basis of~$\Exterior^n( u_1, \dotsc, u_{k-1} )$.
  The quotient~$\Exterior^n(u_1, \dotsc, u_k) \mathbin{/} \Exterior^n(u_1, \dotsc, u_{k-1})$ does therefore have the residue classes
  \[
    \class{ u_{i_1} \wedge \dotsb \wedge u_{i_{n-1}} \wedge u_k }
    \qquad
    \text{with~$1 \leq i_1 < \dotsb < i_{n-1} \leq k-1$}
  \]
  as a basis.
  We can therefore identify the quotient~$\Exterior^n(u_1, \dotsc, u_k) \mathbin{/} \Exterior^n(u_1, \dotsc, u_{k-1})$ with the exterior power~$\Exterior^{n-1}(u_1, \dotsc, u_{k-1})$ via the isomorphism
  \[
    \Exterior^{n-1}(u_1, \dotsc, u_{k-1})
    \to
    \Exterior^n(u_1, \dotsc, u_k)
    \mathbin{\big/}
    \Exterior^n(u_1, \dotsc, u_{k-1}) \,,
    \quad
    t
    \mapsto
    \class{ t \wedge u_k } \,.
  \]
  We thus have
  \[
    V(n, k) / V(n, k-1)
    \cong
    V(n-1, k-1)
  \]
  for all~$n \geq 1$, and such an isomorphism~$f_n$ from~$V(n-1,k-1)$ to~$V(n,k) / V(n, k-1)$ is given by
  \[
    f_{n-1}
    \colon
    V(n-1, k-1)
    \to
    V(n,k) / V(n,k-1) \,,
    \quad
    y \otimes t
    \mapsto
    \class{ y \otimes (t \wedge u_k) }
  \]
  for all~$y \in \Univ(\glie)$,~$t \in \Exterior^{n-1}(u_1, \dotsc, u_{k-1})$.

  We note for the quotient~$J_k / J_{k-1}$ that the ideal~$J_k$ has as a basis all those monomials which contain at least one of the variables~$u_1, \dotsc, u_k$, and that~$J_{k-1}$ has as a basis all those monomials which contain at least on of the variables~$u_1, \dotsc, u_{k-1}$.
  The quotient~$J_k / J_{k-1}$ does therefore have a basis given by all those monomials which contain the variable~$u_k$ but not the variables~$u_1, \dotsc, u_{k-1}$.
  We can thus identify the quotient~$J_k / J_{k-1}$ on the level of vector spaces with the ideal~$\ideal{u_k}_{\kf[u_k, \dotsc, u_N]}$.
  This ideal we can identify with~$\kf[u_k, \dotsc, u_N]$ via the multiplicaiton with~$u_k$.
  But algebra~$\kf[u_k, \dotsc, u_N]$ is the quotient~$\Univ(\glie) / J_{k-1}$.
  We have thus overall an isomorphism of vector spaces
  \[
    f_{-1}
    \colon
    \Univ(\glie) / J_{k-1}
    \to
    J_k / J_{k-1} \,,
    \quad
    \class{y}
    \mapsto
    \class{y u_k} \,.
  \]

  We have now the following diagram.
  \[
    \begin{tikzcd}[row sep = large]
      \dotsb
      \arrow{r}
      &
      V(2, k) / V(2, k-1)
      \arrow{r}[above]{\induced{d_{2,k}}}
      &
      V(1, k) / V(1, k-1)
      \arrow{r}[above]{\induced{d_{1,k}}}
      &
      J_k / J_{k-1}
      \arrow{r}
      &
      0
      \\
      \dotsb
      \arrow{r}
      &
      V(1,k-1)
      \arrow{u}[right]{f_1}
      \arrow{r}[above]{d_{1,k-1}}
      &
      V(0,k-1)
      \arrow{u}[right]{f_0}
      \arrow{r}[above]{\varepsilon_{k-1}}
      &
      \Univ(\glie) / J_{k-1}
      \arrow{u}[right]{f_{-1}}
      \arrow{r}
      &
      0
    \end{tikzcd}
  \]
  The lower row is the chain complex~$X^{(k-1)}_\bullet$, which is exact by induction hypothesis.
  The vertical arrows are isomorphisms.
  To conclude the exactness of the upper row it sufficies to show that this diagram commutes.
  We have
  \begin{align*}
    {}&
    \induced{d_{n,k}}( f_{n-1}( y \otimes x_1 \wedge \dotsb \wedge x_{n-1} ) )
    \\
    ={}&
    \induced{d_{n,k}}\bigl( \class{ y \otimes x_1 \wedge \dotsb \wedge x_{n-1} \wedge u_k} \bigr)
    \\
    ={}&
    \sum_{i=1}^{n-1}
    (-1)^{i+1} \class{ (y x_i) \otimes x_1 \wedge \dotsb \wedge \widehat{x_i} \wedge \dotsb \wedge x_{n-1} \wedge u_k }
    + (-1)^{n+1} \underbrace{ \class{ (y u_k) \otimes x_1 \wedge \dotsb \wedge x_{n-1} } }_{=0}
    \\
    ={}&
    \sum_{i=1}^{n-1}
    (-1)^{i+1} \class{ (y x_i) \otimes x_1 \wedge \dotsb \wedge \widehat{x_i} \wedge \dotsb \wedge x_{n-1} \wedge u_k }
    \\
    ={}&
    f_{n-2}
    \Biggl(
      \sum_{i=1}^{n-1}
      (-1)^{i+1}
      (y x_i) \otimes x_1 \wedge \dotsb \wedge \widehat{x_i} \wedge \dotsb \wedge x_{n-1}
    \Biggr)
    \\
    ={}&
    f_{n-2}( d_{n-1,k-1}( y \otimes x_1 \wedge \dotsb \wedge x_{n-1} ) )
  \end{align*}
  for all~$y \in \Univ(\glie)$,~$x_1, \dotsc, x_{n-1} \in \gen{u_1, \dotsc, u_{k-1}}$.
  This shows the commutativity of all squares except for the rightmost one.
  For the commutativity of this last square we calculate that
  \[
    \induced{d_{1,k}}( f_0( y ) )
    =
    \induced{d_{1,k}}\bigl( \class{ y \otimes u_k } \bigr)
    =
    \class{ y u_k }
    =
    f_{-1}( \class{y} )
    =
    f_{-1}( \varepsilon_{k-1}( y ) )
  \]
  for all~$y \in \Univ(\glie)$.
\end{proof}


\begin{recall}
  Let~$X^{(i)}_\bullet$ with~$i \in I$ be a directed family of chain complexes.
  More explicitely, each~$X^{(i)}_\bullet$ is a chain complex, and for any two element~$i$,~$j$ of~$I$ there exists another element~$k$ of~$I$ such that both~$X^{(i)}_\bullet$ and~$X^{(j)}_\bullet$ are subcomplexes of~$X^{(k)}_\bullet$.
  \begin{enumerate}
    \item
      It follows for every integer~$n$ that the family of vector spaces~$X^{(i)}_n$ with~$i \in I$ is directed.
      The union~$Y_n \defined \bigcup_{i \in I} X^{(i)}_n$ does therefore inherits the structure of a vector space from the~$X^{(i)}_n$.
      Moreover, the differentials~$d^{(i)}_n \colon X^{(i)}_n \to X^{(i)}_{n-1}$ assemble into a unique linear map~$d_n$ from~$X_n$ to~$X_{n-1}$, such that the restriction of~$d_n$ to~$X^{(i)}_n$ is~$d^{(i)}_n$ for every~$i \in I$.
      
      These maps satisfy the condition~$d_{n-1} \circ d_n = 0$ for every integer~$n$.
      Indeed, there exists for every element~$x$ of~$X_n$ some index~$i$ such that~$x$ is contained in~$X^{(i)}_n$.
      Then~$d_n(x) = d^{(i)}_n(x)$, and this element is contained in~$X^{(i)}_{n-1}$.
      Thus
      \[
        d_{n-1}( d_n( x ) )
        =
        d_{n-1}( d^{(i)}_n( x ) )
        =
        d^{(i)}_{n-1}( d^{(i)}_n( x ) )
        =
        0 \,.
      \]

      We have thus constructed a chain complex~$X_{\bullet}$ such that each~$X^{(i)}_\bullet$ is a subcomplex of~$X_\bullet$.
      We denote this complex by~$\bigcup_{i \in I} X^{(i)}_\bullet$.
    \item
      If each complex~$X^{(i)}_\bullet$ is exact then the complex~$X_\bullet$ is again exact.
      Indeed, let~$x$ be a~\cycle{$n$} of~$X_\bullet$.
      The element~$x$ is contained in~$X^{(i)}_\bullet$ for some index~$i$.
      Then
      \[
        0
        =
        d_n(x)
        =
        d^{(i)}_n(x)
      \]
      whence the element~$x$ is an~\cycle{$n$} in~$X^{(i)}_\bullet$.
      It follows from the exactness of~$X^{(i)}_\bullet$ that there exists an element~$y$ of~$X^{(i)}_{n+1}$ with~$x = d^{(i)}_{n+1}(y)$.
      Thus
      \[
        x
        =
        d^{(i)}_{n+1}( y )
        =
        d_{n+1}(y) \,.
      \]
      This shows the exactness of~$X_\bullet$.
  \end{enumerate}
\end{recall}


\begin{proof}[Proof of \cref{koszul is a free resolution}, second part]
  Suppose now that~$\glie$ is an abelian Lie~algebra of arbitrary dimension.
  Let~$(u_\lambda)_{\lambda \in \Lambda}$ be a basis of~$\glie$.
  For every subset~$I$ of~$\Lambda$ let~$\glie_I$ be linear subspace of~$\glie$ with basis~$(u_i)_{i \in I}$.
  This subspace inherts from~$\glie$ the structure of an abelian Lie~algebra.
  Let~$X^{(I)}_\bullet$ denote the Koszul complex of~$\glie_I$.

  Let~$\mathcal{I}$ denote the set of finite subsets of~$\Lambda$.
  We have already seen that the complex~$X^{(I)}_\bullet$ is exact for all~$I \in \mathcal{I}$.
  If~$I$ and~$J$ are two elements of~$\mathcal{I}$ then there exists a third element~$K$ of~$\mathcal{I}$ which contains both~$I$ and~$J$ (e.g.~$K = I \cup J$).
  Then the complexes~$X^{(I)}_\bullet$ and~$X^{(J)}_\bullet$ are both subcomplexes of~$X^{(K)}_\bullet$.
  This shows that the family of complexes~$X^{(I)}_\bullet$ with~$I \in \mathcal{I}$ is directed.

  The directed union~$\bigcup_{I \in \mathcal{I}} X^{(I)}_\bullet$ is thus again an exact complex.
  But this union is the Koszul complex of~$\glie$.
\end{proof}


\begin{lemma}
  \label{kernel of restricted counit}
  Let~$\varepsilon$ be the counit of~$\glie$.
  For all~$p \geq 0$ the kernel of the restriction of~$\varepsilon$ to~$\Univ(\glie)_{(p)}$ is spanned as a vector space by all those monomials~$x_1 \dotsm x_n$ with~$1 \leq n \leq p$,~$x_1, \dotsc, x_n \in \glie$.
\end{lemma}


\begin{proof}
  Let~$\varepsilon_p$ be the restriction of~$\varepsilon$ to~$\Univ(\glie)_{(p)}$.
  Let~$U$ be the span of the proposed vector space generating set of~$\ker(\varepsilon_p)$.

  We know that the vector space~$\Univ(\glie)_{(p)}$ is spanned by the monomials
  \[
    x_1 \dotsm x_n
    \qquad
    \text{with~$0 \leq n \leq p$,~$x_1, \dotsc, x_n \in \glie$.}
  \]
  We know from \cref{augumentation ideal is spanned by monomials} that all those monomials with~$n \geq 1$ are contained in~$\ker(\varepsilon_p)$.
  We therefore know that~$U$ is contained in~$\ker(\varepsilon_p)$ and that the codimension of~$U$ in~$\Univ(\glie)_{(p)}$ is at most one.
  But the codimension of~$\ker(\varepsilon_p)$ in~$\Univ(\glie)_{(p)}$ is precisely one.
  The inclusion~$U \subseteq \ker( \varepsilon_p )$ is therefore an equality.
\end{proof}


\begin{recall}
  \label{restricting decompositions}
  Let~$V$ be a vector space.
  Let~$U$ and~$W$ be two linear subspaces of~$V$ such that~$U$ is contained in~$W$.
  If~$C$ is a direct complement for~$U$ in~$V$ then~$C \cap W$ is a direct complement vor~$U$ in~$W$.
\end{recall}


\begin{lemma}
  \label{decomposition for restricted counit}
  Let~$\varepsilon$ be the counit of~$\Univ(\glie)$ and let~$\varepsilon_p$ be the restriction of~$\varepsilon$ to~$\Univ(\glie)_{(p)}$.
  Then
  \[
    \Univ(\glie)_{(p)}
    =
    \kf \oplus \ker(\varepsilon_p) \,.
  \]
\end{lemma}


\begin{proof}
  We have the decomposition~$\Univ(\glie) = \kf \oplus \ker(\varepsilon)$ by \cref{decomposition for augumented algebra}.
  The direct summand~$\kf$ is contained in~$\Univ(\glie)_{(p)}$.
  It therefore follows~\cref{restricting decompositions} that~$\Univ(\glie)_{(p)} = \kf \oplus \ker(\varepsilon_p)$ because~$\ker(\varepsilon_p) = \ker(\varepsilon) \cap \Univ(\glie)_{(p)}$.
\end{proof}


\begin{recall}
  Let~$X^{(i)}_\bullet$ with~$i \in I$ be a family of chain complexes.
  For every integer~$n$ let
  \[
    X_n \defined \bigoplus_{i \in I} X^{(i)}_n
  \]
  and let~$d_n \defined \bigoplus_{i \in I} d^{(i)}_n$.
  This defines a chain complex~$X_\bullet$ because
  \[
    d_{n-1} \circ d_n
    =
    \bigoplus_{i \in I} d^{(i)}_{n-1}
    \circ
    \bigoplus_{i \in I} d^{(i)}_n
    =
    \bigoplus_{i \in I} \Bigl( d^{(i)}_{n-1} \circ d^{(i)}_n \Bigr)
    =
    \bigoplus_{i \in I} 0
    =
    0
  \]
  for all~$n \in \Integer$.
  For this chain complex~$X_\bullet$ we have
  \[
    \Cycle_n( X_\bullet )
    =
    \ker\biggl( \bigoplus_{i \in I} d^{(i)}_n \biggr)
    =
    \bigoplus_{i \in I} \ker\Bigl( d^{(i)}_n \Bigr)
    =
    \bigoplus_{i \in I} \Cycle_n\Bigl( X^{(i)}_\bullet \Bigr)
  \]
  and similarly
  \[
    \Boundary_n( X_\bullet )
    =
    \im\biggl( \bigoplus_{i \in I} d^{(i)}_{n+1} \biggr)
    =
    \bigoplus_{i \in I} \im\Bigl( d^{(i)}_{n+1} \Bigr)
    =
    \bigoplus_{i \in I} \Boundary_n\Bigl( X^{(i)}_\bullet \Bigr) \,.
  \]
  It follows that
  \begin{align*}
    \Homology_n( X_\bullet )
    &=
    \Cycle_n( X_\bullet ) / \Boundary_n( X_\bullet )
    \\
    &=
    \biggl(
      \bigoplus_{i \in I} \Cycle_n\Bigl( X^{(i)}_\bullet \Bigr)
    \biggr)
    \Big/
    \biggl(
      \bigoplus_{i \in I} \Boundary_n\Bigl( X^{(i)}_\bullet \Bigr)
    \biggr)
    \\
    &\cong
    \bigoplus_{i \in I}
    \bigl(
      \Cycle_n( X_\bullet ) / \Boundary_( X_\bullet )
    \bigr)
    \\
    &=
    \bigoplus_{i \in I} \Homology_n( X_\bullet ) \,.
  \end{align*}
  It follows from this that the chain complex~$X_\bullet$ is exact if and only if the chain complex~$X^{(i)}_\bullet$ is exact for all~$i \in I$.
\end{recall}


\begin{proof}[Proof of \cref{koszul is a free resolution}, third part]
  Suppose now that~$\glie$ is an arbitrary Lie~algebra.
  For every natural number~$p$ let
  \[
    V(n,p)
    \defined
    \Univ(\glie)_{(p-n)} \otimes \Exterior^n(\glie)
  \]
  for all~$n \geq 0$.
  The differentials of the Kozsul complex satisfy
  \[
    d_n( V(n,p) )
    \subseteq
    V(n-1,p)
  \]
  for all~$n \geq 1$.
  The Koszul complex for~$\glie$ therefore restricts for all~$p \geq 0$ to a complex~$X^{(p)}_\bullet$ given by
  \[
    \dotsb
    \to
    V(3,p)
    \xto{d_{3,p}}
    V(2,p)
    \xto{d_{2,p}}
    V(1,p)
    \xto{d_{1,p}}
    \Univ(\glie)_{(p)}
    \xto{\varepsilon_p}
    \kf
    \to
    0 \,,
  \]
  where~$V(0,p) = \Univ(\glie)_{(p)}$. 
  If~$p \leq q$ then~$X^{(p)}_\bullet$ is a subcomplex of~$X^{(q)}_\bullet$.
  It follows that the family of complexes~$X^{(p)}_\bullet$ with~$p \geq 0$ is directed, and the union~$\bigcup_{p \geq 0} X^{(p)}_\bullet$ is the Koszul complex of~$\glie$.
  (One may think about the complexes~$X^{(p)}_\bullet$ as a filtration of the Koszul complex.)
  We show in the following that the complex~$X^{(p)}_\bullet$ is exact for all~$p \geq 0$.
  It then follows that the Koszul complex is also exact, since it is the union~$\bigcup_{p \geq 0} X^{(p)}_\bullet$.

  We note that the restriction~$\varepsilon_p$ from~$\Univ(\glie)_{(p)}$ to~$\kf$ is surjective for all~$p \geq 0$.
  The complex~$X^{(p)}_\bullet$ is therefore exact at~$\kf$ for all~$p \geq 0$.
  We also know from \cref{kernel of restricted counit} that the kernel of~$\varepsilon_p$ is spanned by the monomials~$x_1 \dotsm x_n$ with~$1 \leq n \leq p$,~$x_1, \dotsc, x_n \in \glie$.
  This is precisely the image of~$V(1,p) = \Univ(\glie)_{(p-1)} \otimes \glie$ under~$d_{1,p}$, since~$d_{1,p}$ is given by
  \[
    d_{1,p}( y \otimes x )
    =
    y x
  \]
  for all~$y \in \Univ(\glie)_{(p-1)}$,~$x \in \glie$.
  The complex~$X^{(p)}_\bullet$ is thus also exact at~$\Univ(\glie)_{(p)}$.

  To prove the exactness of~$X^{(p)}_\bullet$ it now sufficies to show the exactness of the chain complex~$Y^{(p)}_\bullet$ given by
  \[
    \dotsb
    \to
    V(3,p)
    \xto{d_{3,p}}
    V(2,p)
    \xto{d_{2,p}}
    V(1,p)
    \xto{d_{1,p}}
    \ker( \varepsilon_p )
    \to
    0 \,.
  \]
  We show this exactness by induction over~$p$.
  The chain complex~$Y^{(0)}_\bullet$ is the zero complex, which is exact.
  This takes care of the base case.

  Let now~$p \geq 1$.
  We note that the chain complex~$Y^{(p-1)}_\bullet$ is a subcomplex of~$Y^{(p)}_\bullet$.
  The chain complex~$Y^{(p-1)}_\bullet$ is exact by induction hypothesis.
  We show in the following that the quotient chain complex~$Y^{(p)}_\bullet / Y^{(p-1)}_\bullet$ is also exact.
  It then follows from \cref{exactness in ses of chain complex} that~$Y^{(p)}_\bullet$ is exact, as desired.

  The quotient~$Y^{(p)}_\bullet / Y^{(p-1)}_\bullet$ is given by
  \[
    \dotsb
    \to
    V(2,p) / V(2, p-1)
    \xto{\induced{d_{2,p}}}
    V(1,p) / V(1, p-1)
    \xto{\induced{d_{1,p}}}
    \ker(\varepsilon_p) / {\ker(\varepsilon_{p-1})}
    \to
    0 \,.
  \]
  The differentials of this chain complex are given by
  \begin{align*}
    {}&
    \induced{d_{n,p}}( \class{y \otimes x_1 \wedge \dotsb \wedge x_n} )
    \\
    ={}&
    \sum_{1 \leq i < j \leq n}
    (-1)^{i+j}
    \,
    \underbrace{
      \class{
        y \otimes [x_i, x_j] \wedge x_1 \wedge \dotsb \wedge \widehat{x_i} \wedge \dotsb \wedge \widehat{x_j} \wedge \dotsb \wedge x_n
      }
    }_{=0}
    \\
    {}&
    +
    \sum_{i=1}^n
    (-1)^{i+1}
    \,
    \class{ (y x_i) \otimes x_1 \wedge \dotsb \wedge \widehat{x_i} \wedge \dotsb \wedge x_n }
    \\
    ={}&
    \sum_{i=1}^n
    (-1)^{i+1}
    \,
    \class{ (y x_i) \otimes x_1 \wedge \dotsb \wedge \widehat{x_i} \wedge \dotsb \wedge x_n }
  \end{align*}
  for all~$y \in \Univ(\glie)_{(p-n)}$,~$x_1, \dotsc, x_n \in \glie$.
  We used above that the term
  \[
    y \otimes [x_i,x_j] \wedge x_1 \wedge \dotsb \wedge \widehat{x_i} \wedge \dotsb \wedge \widehat{x_j} \wedge \dotsb \wedge x_n
  \]
  is contained in~$\Univ(\glie)_{(p-n)} \otimes \Exterior^{n-1}(\glie) = V(n-1,p-1)$ and thus vanishes in~$V(n-1,p) / V(n-1, p-1)$.
  We can also compute the terms of~$Y^{(p)}_\bullet / Y^{(p-1)}_\bullet$.
  We have
  \begin{align*}
    V(n, p) / V(n, p-1)
    &=
    \Bigl(
      \Univ(\glie)_{(p-n)} \otimes \Exterior^n(\glie)
    \Bigr)
    \big/
    \Bigl(
      \Univ(\glie)_{(p-n-1)} \otimes \Exterior^n(\glie)
    \Bigr)
    \\
    &\cong
    \bigl(
      \Univ(\glie)_{(p-n)} / \Univ(\glie)_{(p-n-1)}
    \bigr)
    \otimes
    \Exterior^n(\glie)
    \\
    &=
    \gr[p-n]( \Univ(\glie) )
    \otimes
    \Exterior^n(\glie)
  \end{align*}
  for all~$n \geq 1$.
  This isomorphism of vector spaces is explicitely given by
  \begin{align*}
    f_n
    \colon
    V(n, p) / V(n, p-1)
    &\to
    \gr[p-n]( \Univ(\glie) )
    \otimes
    \Exterior^n(\glie) \,,
    \\
    \class{ y \otimes x_1 \wedge \dotsb \wedge x_n }
    &\mapsto
    \fclass{ y }_p \otimes x_1 \wedge \dotsb \wedge x_n
  \end{align*}
  for all~$y \in \Univ(\glie)_{(p-n)}$,~$x_1, \dotsc, x_n \in \glie$.
  We also have
  \begin{align*}
    \ker(\varepsilon_p)
    /
    {\ker(\varepsilon_{p-1})}
    &\cong
    (\ker(\varepsilon_p) \oplus \kf)
    /
    (\ker(\varepsilon_{p-1}) \oplus \kf)
    \\
    &=
    \Univ(\glie)_{(p)} / \Univ(\glie)_{(p-1})
    \\
    &=
    \gr[p-n]( \Univ(\glie) ) \,.
  \end{align*}
  This isomorphism of vector spaces is explicitely given by
  \[
    f_0
    \colon
    \ker(\varepsilon_p) / {\ker(\varepsilon_{p-1})}
    \to
    \gr[p]( \Univ(\glie) ) \,,
    \quad
    \class{y}
    \mapsto
    \fclass{ y }_p
  \]
  for all~$y \in \ker(\varepsilon_p)$.
  We have~$\ker(\varepsilon_p) / \ker(\varepsilon_{p-1}) \cong \Univ(\glie)_{(p)} / \Univ(\glie)_{(p-1)}$, and the formula for~$f_0$ is the same as for~$f_n$ with~$n \geq 1$.
  In other words, we have
  \[
    f_n( \class{ y \otimes x_1 \wedge \dotsb \wedge x_n } )
    =
    \fclass{ y }_{p-n} \otimes x_1 \wedge \dotsb \wedge x_n
  \]
  for all~$n \geq 0$,~$y \in \Univ(\glie)_{(p-n)}$,~$x_1, \dotsc, x_n \in \glie$.

  We have now a commutative diagram
  \[
    \begin{tikzcd}
      \dotsb
      \arrow{r}
      &[-1em]
      V(2,p) / V(2, p-1)
      \arrow{r}[above]{\induced{d_{2,p}}}
      \arrow{d}[right]{f_2}
      &
      V(1,p) / V(1, p-1)
      \arrow{r}[above]{\induced{d_{1,p}}}
      \arrow{d}[right]{f_1}
      &
      \ker(\varepsilon_p) / {\ker(\varepsilon_{p-1})}
      \arrow{r}
      \arrow{d}[right]{f_0}
      &[-1em]
      0
      \\
      \dotsb
      \arrow{r}
      &
      \gr[p-2]( \Univ(\glie) ) \otimes \Exterior^2(\glie)
      \arrow{r}[above]{\del_{2,p}}
      &
      \gr[p-1]( \Univ(\glie) ) \otimes \glie
      \arrow{r}[above]{\del_{1,p}}
      &
      \gr[p]( \Univ(\glie) )
      \arrow{r}
      &
      0
    \end{tikzcd}
  \]
  for suitable linear maps~$\del_{n,p}$ with~$n \geq 1$.
  The vertical maps are isomorphism of vector spaces.
  The lower row is thus again a chain complex, and it is isomorphic to the upper row as chain complexes.
  We denote this lower row by~$Z^{(p)}_\bullet$.
  We calculate the differentials~$\del_{n,p}$ of~$Z^{(p)}_\bullet$ as
  \begin{align*}
    {}&
    \del_{n,p}( \fclass{ y }_{p-n} \otimes x_1 \wedge \dotsb \wedge x_n )
    \\
    ={}&
    \del_{n,p}\Bigl( f_n\Bigl( \class{ y \otimes x_1 \wedge \dotsb \wedge x_n } \Bigr) \Bigr) 
    \\
    ={}&
    f_{n-1}\Bigl( \induced{d_{n,p}}\Bigl( \class{ y \otimes x_1 \wedge \dotsb \wedge x_n } \Bigr) \Bigr)
    \\
    ={}&
    f_{n-1}
    \Biggl(
      \sum_{i=1}^n
      (-1)^{i+1}
      \,
      \class{ (y x_i) \otimes x_1 \wedge \dotsb \wedge \widehat{x_i} \wedge \dotsb \wedge x_n }
    \Biggr)
    \\
    ={}&
    \sum_{i=1}^n
    (-1)^{i+1}
    \fclass{ y x_i }_{p-n+1} \otimes x_1 \wedge \dotsb \wedge \widehat{x_i} \wedge \dotsb \wedge x_n
  \end{align*}
  for all~$y \in \Univ(\glie)_{(p-n)}$,~$x_1, \dotsc, x_n \in \glie$.
  This means in the space case of~$n = 1$ that
  \[
    \del_{1,p}( \fclass{ y }_{p-1} \otimes x )
    =
    \fclass{ yx }_p
  \]
  for all~$y \in \Univ(\glie)_{(p-1)}$,~$x \in \glie$.
  To show the desired exactness of~$Y^{(p)}_\bullet / Y^{(p-1)}_\bullet$ we show the exactness of~$Z^{(p)}_\bullet$.

  Let~$\hlie$ be the abelian Lie~algebra that has the same underlying vector space of~$\hlie$.
  According to the abstract version of the PBW theorem we may identify the associated graded algebra~$\gr( \Univ(\glie) )$ with the symmetric algebra~$\Symm(\glie)$ as graded algebras.
  Then
  \[
    \gr(\Univ(\glie))
    =
    \Symm(\glie)
    =
    \Symm(\hlie)
    =
    \Univ(\hlie)
  \]
  as graded algebra, because~$\hlie$ is abelian.
  We have already seen that the Koszul complex~$K_\bullet$ of~$\hlie$, given by
  \[
    \dotsb
    \to
    \Symm(\hlie) \otimes \Exterior^2(\hlie)
    \xto{d_2}
    \Symm(\hlie) \otimes \hlie
    \xto{d_1}
    \Symm(\hlie)
    \xto{\varepsilon}
    \kf
    \to
    0 \,,
  \]
  is exact, where the differentials of this chain complex are given by
  \[
    d_n( y \otimes x_1 \wedge \dotsb \wedge x_n)
    =
    \sum_{i=1}^n
    (-1)^{i+1} (y x_i) \otimes x_1 \wedge \dotsb \wedge \widehat{x_i} \wedge \dotsb \wedge x_n \,.
  \]
  The grading~$\Symm(\hlie) = \bigoplus_{d \geq 0} \Symm^d(\hlie)$ induces for all~$n \geq 0$ a grading on~$\Symm(\hlie) \otimes \Exterior^n(\hlie)$ such that the degree~$q$ part of~$\Symm(\hlie) \otimes \Exterior^n(\hlie)$ is given by~$\Symm^{q-n}(\hlie) \otimes \Exterior^n(\hlie)$ (where~$S^d(\hlie) = 0$ for all~$d < 0$).
  We also have a grading on~$\kf$ given by~$\kf_0 = \kf$ and~$\kf_q = 0$ for all~$q > 1$.
  The differentials of the Koszul complex respect these gradings, in these sense that we have for every~$q$ a restricted chain complex~$K^{(q)}_\bullet$ given by
  \[
    \dotsb
    \to
    \Symm^{q-2}(\hlie) \otimes \Exterior^2(\hlie)
    \xto{d_{2,q}}
    \Symm^{q-1}(\hlie) \otimes \hlie
    \xto{d_{1,q}}
    \Symm^q(\hlie)
    \xto{\varepsilon_q}
    \kf_q
    \to
    0 \,,
  \]
  We have conversely~$K_\bullet = \bigoplus_{q \geq 0} K^{(q)}_\bullet$.
  The exactness of the Koszul complex~$K_\bullet$ shows that each subcomplex~$K^{(q)}_\bullet$ is again exact.
  We find for~$q = p$ that the exact chain complex~$K^{(p)}_\bullet$ is given by
  \[
    \dotsb
    \to
    \Symm^{p-2}(\hlie) \otimes \Exterior^2(\hlie)
    \xto{d_{2,p}}
    \Symm^{p-1}(\hlie) \otimes \hlie
    \xto{d_{1,p}}
    \Symm^p(\hlie)
    \xto{\varepsilon_p}
    0
    \to
    0 \,,
  \]
  because~$p \geq 0$ and thus~$\kf_p = 0$.

  We have now
  \[
    \gr[d](\Univ(\glie))
    =
    \gr(\Univ(\glie))_d
    =
    \Symm(\glie)_d
    =
    \Symm(\hlie)_d
    =
    \Symm^d(\hlie)
  \]
  for all~$d \geq 0$ and thus
  \[
    \gr[d]( \Univ(\glie) ) \otimes \Exterior^n(\glie)
    =
    \Symm^d( \hlie ) \otimes \Exterior^n(\hlie)
  \]
  for all~$d \geq 0$,~$n \geq 0$.
  We have therefore the diagram
  \[
    \begin{tikzcd}
      \dotsb
      \arrow{r}
      &
      \gr[p-2]( \Univ(\glie) ) \otimes \Exterior^2(\glie)
      \arrow{r}[above]{\del_{2,p}}
      \arrow[equal]{d}
      &
      \gr[p-1]( \Univ(\glie) ) \otimes \glie
      \arrow{r}[above]{\del_{1,p}}
      \arrow[equal]{d}
      &
      \gr[p]( \Univ(\glie) )
      \arrow{r}
      \arrow[equal]{d}
      &
      0
      \\
      \dotsb
      \arrow{r}
      &
      \Symm^{p-2}( \Univ(\hlie) ) \otimes \Exterior^2(\hlie)
      \arrow{r}[above]{d_{2,p}}
      &
      \Symm^{p-1}( \Univ(\hlie) ) \otimes \hlie
      \arrow{r}[above]{d_{1,p}}
      &
      \Symm^p( \Univ(\hlie) )
      \arrow{r}
      &
      0
    \end{tikzcd}
  \]
  We know from the explicit formulas for~$\delta_{n,p}$ and~$d_{n,p}$ that this diagram commutes.
  The lower row is the exact complex~$K^{(p)}_\bullet$.
  It thus follows that the upper row also commutes.
\end{proof}


\begin{fluff}
  We also have a Koszul resolution as right modules.
  For this we regard~$\Exterior(\glie) \otimes_{\kf} \Univ(\glie)$ as a right~\module{$\Univ(\glie)$} via extension of scalars, i.e. via
  \[
    (t \otimes y) \cdot x
    =
    t \otimes (yx)
  \]
  for all~$t \in \Exterior^n(\glie)$,~$y, x \in \Univ(\glie)$.
\end{fluff}


\begin{corollary}[Koszul resolution for right modules]
  \label{right koszul is a free resolution}
  \leavevmode
  \begin{enumerate}
    \item
      There exists for every~$n \geq 1$ a unique linear map
      \[
        \del_n
        \colon
        \Exterior^n(\glie) \otimes_{\kf} \Univ(\glie)
        \to
        \Exterior^{n-1}(\glie) \otimes_{\kf} \Univ(\glie)
      \]
      given by
      \begin{align*}
        \del_n( x_1 \wedge \dotsb \wedge x_n \otimes y )
        ={}&
        \sum_{1 \leq i < j \leq n}
        (-1)^{i+j}
        [x_i, x_j] \wedge x_1 \wedge \dotsb \wedge \widehat{x_i} \wedge \dotsb \wedge \widehat{x_j} \wedge \dotsb \wedge x_n
        \otimes y
        \\
        {}&
        +
        \sum_{i=1}^n
        (-1)^i
        x_1 \wedge \dotsb \wedge \widehat{x_i} \wedge \dotsb \wedge x_n \otimes (x_i y)
      \end{align*}
      for all~$x_1, \dotsc, x_n \in \glie$,~$y \in \Univ(\glie)$.
    \item
      The maps~$\del_n$ for~$n \geq 1$ are homomorphisms of right~\modules{$\Univ(\glie)$}.
    \item
      The sequence
      \[
        \dotsb
        \to
        \Exterior^2(\glie) \otimes \Univ(\glie)
        \xto{\del_2}
        \glie \otimes_{\kf} \Univ(\glie)
        \xto{\del_1}
        \Univ(\glie)
        \xto{\varepsilon}
        \kf
        \to
        0
      \]
      is a free resolution of~$\kf$ as a right~\module{$\Univ(\glie)$}, where~$\varepsilon$ denotes the counit of~$\Univ(\glie)$.
  \end{enumerate}
\end{corollary}


\begin{proof}
  The existence and uniqueness of the maps~$d_n$ can be shown as in \cref{construction of koszul differential}.
  That these maps are homomorphisms of~\modules{$\Univ(\glie)$} can also be shown as in \cref{construction of koszul differential}.

  To show the exactness of the given sequence we consider the Koszul complex of~$\glie^{\op}$, given by
  \[
    \dotsb
    \to
    \Univ(\glie) \otimes \Exterior^2(\glie)
    \xto{d_2}
    \Univ(\glie) \otimes \glie
    \xto{d_1}
    \Univ(\glie)
    \xto{\varepsilon}
    \kf
    \to
    0 \,.
  \]
  The anti-isomorphism of Lie~algebras
  \[
    \varphi
    \colon
    \glie
    \to
    \glie
    \quad
    x
    \mapsto
    - x
  \]
  extends to an anti-isomorphism of algebras
  \[
    \Phi
    \colon
    \Univ(\glie)
    \to
    \Univ(\glie)\,.
  \]
  We have for all~$n \geq 0$ an induced isomorphism of vector spaces
  \[
    f_n
    \coloneqq
    \Phi \otimes \id
    \colon
    \Univ(\glie) \otimes \Exterior^n(\glie)
    \to
    \Univ(\glie) \otimes \Exterior^n(\glie) \,.
  \]
  We get from this a commutative diagram
  \[
    \begin{tikzcd}[row sep = large]
      \dotsb
      \arrow{r}
      &
      \Univ(\glie) \otimes \Exterior^2(\glie)
      \arrow{r}[above]{\del'_2}
      \arrow{d}[right]{f_2}
      &
      \Univ(\glie) \otimes \glie
      \arrow{r}[above]{\del'_1}
      \arrow{d}[right]{f_1}
      &
      \Univ(\glie)
      \arrow{r}[above]{\varepsilon'}
      \arrow{d}[right]{f_0}
      &
      \kf
      \arrow{r}
      \arrow[equal]{d}
      &
      0
      \\
      \dotsb
      \arrow{r}
      &
      \Univ(\glie) \otimes \Exterior^2(\glie)
      \arrow{r}[above]{d_2}
      &
      \Univ(\glie) \otimes \glie
      \arrow{r}[above]{d_1}
      &
      \Univ(\glie)
      \arrow{r}[above]{\varepsilon}
      &
      \kf
      \arrow{r}
      &
      0
    \end{tikzcd}
  \]
  for suitable linear maps~$\del'_n$ and~$\varepsilon'$.
  The lower row is exact and the vertical linear maps are isomorphisms, so the upper row is exact.
  We can explicitely calculate the maps in the upper row.
  We note that
  \[
    \Phi(y) \cdot x
    =
    - \Phi(y) \cdot \varphi(x)
    =
    - \Phi(y) \cdot \Phi(x)
    =
    - \Phi(x y)
  \]
  for all~$y \in \Univ(\glie)$,~$x \in \glie$, and thus
  \begin{align*}
    {}&
    f_{n-1}( \del'_n( y \otimes x_1 \wedge \dotsb \wedge x_n ) )
    \\
    ={}&
    d_n( f_n( y \otimes x_1 \wedge \dotsb \wedge x_n ) )
    \\
    ={}&
    d_n( \Phi(y) \otimes x_1 \wedge \dotsb \wedge x_n )
    \\
    ={}&
    \sum_{1 \leq i < j \leq n}
    (-1)^{i+j}
    \Phi(y) \otimes
    [x_i, x_j] \wedge
    x_1 \wedge \dotsb \wedge \widehat{x_i} \wedge \dotsb \wedge \widehat{x_j} \wedge \dotsb \wedge x_n
    \\
    {}&
    +
    \sum_{i=1}^n
    (-1)^{i+1}
    ( \Phi(y) \cdot x_i )
    \otimes
    x_1 \wedge \dotsb \wedge \widehat{x_i} \wedge \dotsb \wedge x_n
    \\
    ={}&
    \sum_{1 \leq i < j \leq n}
    (-1)^{i+j}
    \Phi(y) \otimes
    [x_i, x_j] \wedge
    x_1 \wedge \dotsb \wedge \widehat{x_i} \wedge \dotsb \wedge \widehat{x_j} \wedge \dotsb \wedge x_n
    \\
    {}&
    +
    \sum_{i=1}^n
    (-1)^i
    \Phi(x_i y)
    \otimes
    x_1 \wedge \dotsb \wedge \widehat{x_i} \wedge \dotsb \wedge x_n
    \\
    ={}&
    f_{n-1}
    \Biggl(
      \sum_{1 \leq i < j \leq n}
      (-1)^{i+j}
      y \otimes
      [x_i, x_j] \wedge
      x_1 \wedge \dotsb \wedge \widehat{x_i} \wedge \dotsb \wedge \widehat{x_j} \wedge \dotsb \wedge x_n
    \\
      {}&
      \phantom{ f_{n-1} \Biggl( }
      +
      \sum_{i=1}^n
      (-1)^i
      (x_i y)
      \otimes
      x_1 \wedge \dotsb \wedge \widehat{x_i} \wedge \dotsb \wedge x_n
    \Biggr)
  \end{align*}
  for all~$n \geq 1$,~$y \in \Univ(\glie)$,~$x_1, \dotsc, x_n \in \glie$.
  This shows that
  \begin{align*}
    \del'_n( y \otimes x_1 \wedge \dotsb \wedge x_n )
    ={}&
    \sum_{1 \leq i < j \leq n}
    (-1)^{i+j}
    y \otimes
    [x_i, x_j] \wedge
    x_1 \wedge \dotsb \wedge \widehat{x_i} \wedge \dotsb \wedge \widehat{x_j} \wedge \dotsb \wedge x_n
    \\
    {}&
    +
    \sum_{i=1}^n
    (-1)^i
    (x_i y)
    \otimes
    x_1 \wedge \dotsb \wedge \widehat{x_i} \wedge \dotsb \wedge x_n
  \end{align*}
  for all~$n \geq 1$,~$y \in \Univ(\glie)$,~$x_1, \dotsc, x_n \in \glie$.
  The map~$\varepsilon'$ is the composite of the homomorphism of algebras~$\varepsilon$ and the anti-homomorphism of algebras~$\Phi = f_0$.
  It follows that~$\varepsilon'$ is an anti-homomorphism of algebras from~$\Univ(\glie)$ to~$\kf$.
  But the algebra~$\kf$ is commutaitve, so~$\varepsilon'$ is a homomorphism of algebras.
  Moreover, we have
  \[
    \varepsilon'(x)
    =
    \varepsilon( \Phi(x) )
    =
    \varepsilon( \varphi(x) )
    =
    \varepsilon( -x )
    =
    - \varepsilon(x)
    =
    0
  \]
  for all~$x \in \glie$.
  This shows that~$\varepsilon'$ is the antipode of~$\Univ(\glie)$, i.e.~$\varepsilon' = \varepsilon$.
  
  It follows from these calculations that the diagram
  \[
    \begin{tikzcd}[row sep = large]
      \dotsb
      \arrow{r}
      &
      \Univ(\glie) \otimes \Exterior^2(\glie)
      \arrow{r}[above]{\del'_2}
      \arrow{d}[right]{t_2}
      &
      \Univ(\glie) \otimes \glie
      \arrow{r}[above]{\del'_1}
      \arrow{d}[right]{t_1}
      &
      \Univ(\glie)
      \arrow{r}[above]{\varepsilon'}
      \arrow{d}[right]{t_0}
      &
      \kf
      \arrow{r}
      \arrow[equal]{d}
      &
      0
      \\
      \dotsb
      \arrow{r}
      &
      \Exterior^2(\glie) \otimes \Univ(\glie)
      \arrow{r}[above]{\del_2}
      &
      \glie \otimes \Univ(\glie)
      \arrow{r}[above]{\del_1}
      &
      \Univ(\glie)
      \arrow{r}[above]{\varepsilon}
      &
      \kf
      \arrow{r}
      &
      0
    \end{tikzcd}
  \]
  commutes, where~$t_n$ denotes the twist map for all~$n \geq 0$.
  It follows from the exactness of the upper row that the lower row is also exact.
\end{proof}




\section{Consequences of the Koszul Resolution}

\begin{fluff}
  With help of the Koszul complex we can now give an alternative description of Lie algebra homology and Lie algebra cohomology.
\end{fluff}


\begin{construction}
  \label{lie algebra homology and cohomology via koszul complex}
  Let~$\glie$ be a Lie~algebra and let~$M$ be a representation of~$\glie$.
  \begin{enumerate}
    \item
      Let
      \[
        \dotsb
        \to
        \Exterior^2(\glie) \otimes_{\kf} \Univ(\glie)
        \xto{d_2}
        \glie \otimes_{\kf} \Univ(\glie)
        \xto{d_1}
        \Univ(\glie)
        \xto{\varepsilon}
        \kf
        \to
        0
      \]
      be the right Koszul resolution of~$\glie$.
      By removing the term~$\kf$ from this resolution we get the chain complex
      \begin{equation}
        \label{shortened right koszul resolution}
        \dotsb
        \to
        \Exterior^2(\glie) \otimes_{\kf} \Univ(\glie)
        \xto{d_2}
        \glie \otimes_{\kf} \Univ(\glie)
        \xto{d_1}
        \Univ(\glie)
        \to
        0
      \end{equation}
      We can then apply the functor~$(-) \otimes_{\Univ(\glie)} M$ to~\eqref{shortened koszul resolution} to get a new chain complex given by
      \[
        \dotsb
        \to
        \Exterior^2(\glie) \otimes_{\kf} \Univ(\glie) \otimes_{\Univ(\glie)} M
        \xto{d_2 \otimes \id}
        \glie \otimes_{\kf} \Univ(\glie) \otimes_{\Univ(\glie)} M
        \xto{d_1 \otimes \id}
        \Univ(\glie) \otimes_{\Univ(\glie)} M
        \to
        0
      \]
      The differentials of this chain complex are given by
      \begin{align*}
        {}&
        (d_n \otimes \id)(x_1 \wedge \dotsb \wedge x_n \otimes y \otimes m)
        \\
        ={}&
        \sum_{1 \leq i < j \leq n}
        (-1)^{i+j}
        [x_i, x_j] \wedge x_1 \wedge \dotsb \wedge \widehat{x_i} \wedge \dotsb \wedge \widehat{x_j} \wedge \dotsb \wedge x_n
        \otimes y \otimes m
        \\
        {}&
        +
        \sum_{i=1}^n
        (-1)^i
        x_1 \wedge \dotsb \wedge \widehat{x_i} \wedge \dotsb \wedge x_n \otimes (x_i y) \otimes m \,.
      \end{align*}
      We can simplify this chain complex by using the isomorphism
      \begin{equation}
        \label{isomorphism for triple tensor product}
        \Exterior^n(\glie) \otimes_{\kf} \Univ(\glie) \otimes_{\Univ(\glie)} M
        \cong
        \Exterior^n(\glie) \otimes_{\kf} M
      \end{equation}
      which is given by
      \[
        t \otimes y \otimes m
        \mapsto
        t \otimes (y \cdot m)
      \]
      for all~$t \in \Exterior^n(\glie)$,~$y \in \Univ(\glie)$,~$m \in M$.
      We arrive at a chain complex
      \[
        \dotsb
        \to
        \Exterior^2(\glie) \otimes_{\kf} M
        \xto{\del_2}
        \glie \otimes_{\kf} M
        \xto{\del_1}
        M
        \to
        0 \,.
      \]
      To compute the differentials for this chain complex we take the above formula for~$d_n \otimes \id$, and apply the isomorphism~\eqref{isomorphism for triple tensor product}, and then set~$y  = 1$.
      We find that
      \begin{align*}
        {}&
        \del_n(x_1 \wedge \dotsb \wedge x_n \otimes m)
        \\
        ={}&
        \sum_{1 \leq i < j \leq n}
        (-1)^{i+j}
        [x_i, x_j] \wedge x_1 \wedge \dotsb \wedge \widehat{x_i} \wedge \dotsb \wedge \widehat{x_j} \wedge \dotsb \wedge x_n
        \otimes m
        \\
        {}&
        +
        \sum_{i=1}^n
        (-1)^i
        x_1 \wedge \dotsb \wedge \widehat{x_i} \wedge \dotsb \wedge x_n \otimes (x_i \cdot m) \,.
      \end{align*}
      for all~$m \in M$,~$x_1, \dotsc, x_n \in \glie$.
      We have thus constructed the Lie algebra chain complex~$\Chain_\bullet(\glie, M)$ from the Koszul resolution.
    \item
      Let
      \[
        \dotsb
        \to
        \Univ(\glie) \otimes_{\kf} \Exterior^2(\glie)
        \xto{d_2}
        \Univ(\glie) \otimes_{\kf} \glie
        \xto{d_1}
        \Univ(\glie)
        \xto{\varepsilon}
        \kf
        \to
        0
      \]
      be the Koszul resolution of~$\glie$.
      By removing the term~$\kf$ from this resolution we get the chain complex
      \begin{equation}
        \label{shortened koszul resolution}
        \dotsb
        \to
        \Univ(\glie) \otimes_{\kf} \Exterior^2(\glie)
        \xto{d_2}
        \Univ(\glie) \otimes_{\kf} \glie
        \xto{d_1}
        \Univ(\glie)
        \to
        0
      \end{equation}
      We can also apply the functor~$\Hom_{\Univ(\glie)}(-, M)$ to the chain complex~\eqref{shortened koszul resolution} to arrive at a cochain complex
      \begin{align*}
        0
        \to
        \Hom_{\Univ(\glie)}( \Univ(\glie), M )
        &\xto{d_1^*}
        \Hom_{\Univ(\glie)}( \Univ(\glie) \otimes_{\kf} \glie, M )
        \\
        &\xto{d_2^*}
        \Hom_{\Univ(\glie)}\Bigl( \Univ(\glie) \otimes_{\kf} \Exterior^2(\glie), M \Bigr)
        \\
        &\xto{d_3^*}
        \Hom_{\Univ(\glie)}\Bigl( \Univ(\glie) \otimes_{\kf} \Exterior^3(\glie), M \Bigr)
        \to
        \dotsb
      \end{align*}
      The differential of this cochain complex is given by
      \begin{align*}
        d_n^*(f)( y \otimes x_1 \wedge \dotsb \wedge x_n )
        ={}&
        f( d_n(y \otimes x_1 \wedge \dotsb \wedge x_n) )
        \\
        ={}&
        \sum_{1 \leq i < j \leq n}
        (-1)^{i+1}
        f
        (
          y \otimes
          [x_i, x_j] \wedge x_1 \wedge \dotsb \wedge \widehat{x_i} \wedge \dotsb \wedge \widehat{x_j} \wedge \dotsb \wedge x_n
        )
        \\
        {}&
        +
        \sum_{i=1}^n
        (-1)^{i+1}
        f( (y x_i) \otimes x_1 \wedge \dotsb \wedge \widehat{x_i} \wedge \dotsb \wedge x_n )
      \end{align*}
      for all~$f \in \Hom( \Univ(\glie) \otimes_{\kf} \Exterior^{n-1}(\glie), M )$,~$y \in \Univ(\glie)$,~$x_1, \dotsc, x_n \in \glie$.
      We have by the universal property of the extension of scalars an isomorphism of vector spaces
      \[
        \Hom_{\Univ(\glie)}\Bigl( \Univ(\glie) \otimes \Exterior^n( \glie, M ) \Bigr)
        \cong
        \Hom_{\kf}\Bigl( \Exterior^n(\glie), M \Bigr)
      \]
      for all~$n \geq 0$, given by
      \begin{align*}
        f
        &\mapsto
        \bigl( t \mapsto f(1 \otimes t) \bigr) \,, \\
        \bigl( y \otimes t \mapsto y \cdot g(t) \bigr)
        &\mapsfrom
        g \,.
      \end{align*}
      The above cochain complex is thus isomorphic to the cochain complex given by
      \[
        0
        \to
        \Hom_{\kf}( \kf, M )
        \xto{\del_0}
        \Hom_{\kf}( \glie, M )
        \xto{\del_1}
        \Hom_{\kf}\Bigl( \Exterior^2(\glie), M \Bigr)
        \to
        \dotsb
      \]
      with differentials given by
      \begin{align*}
        \del_n(f)(x_1 \wedge \dotsb \wedge x_{n+1})
        ={}&
        \sum_{1 \leq i < j \leq n+1}
        (-1)^{i+1}
        f
        (
          [x_i, x_j] \wedge
          x_1 \wedge \dotsb \wedge \widehat{x_i} \wedge \dotsb \wedge \widehat{x_j} \wedge \dotsb \wedge x_{n+1}
        )
        \\
        {}&
        +
        \sum_{i=1}^{n+1}
        (-1)^{i+1}
        x_i \cdot f( \otimes x_1 \wedge \dotsb \wedge \widehat{x_i} \wedge \dotsb \wedge x_{n+1} )
      \end{align*}
      for all~$f \in \Hom_{\kf}( \Exterior^n(\glie), M )$,~$x_1, \dotsc, x_{n+1} \in \glie$.

      By further identifying~$\Hom_{\kf}( \Exterior^n(\glie), M )$ with~$\Alt^n(\glie, M)$ for all~$n \geq 0$ we arrive at the Lie algebra cochain complex~$\Chain^\bullet(\glie, M)$.
  \end{enumerate}
\end{construction}


\begin{proposition}
  \label{invariant and coinvariants via hom and tensor}
  Let~$\glie$ be a Lie~algebra and let~$M$ be a representation of~$\glie$.
  \begin{enumerate}
    \item
      The map
      \[
        \Hom_{\Univ(\glie)}(\kf, M)
        \to
        M^{\glie} \,,
        \quad
        f
        \mapsto
        f(1)
      \]
      is a well-defined isomorphism of vector spaces.
    \item
      The map
      \[
        \kf \otimes_{\Univ(\glie)} M
        \to
        M_{\glie} \,,
        \quad
        1 \otimes m
        \mapsto
        \class{ m }
      \]
      is a well-defined isomorphism of vector spaces.
  \end{enumerate}
\end{proposition}


\begin{proof}
  \leavevmode
  \begin{enumerate}
    \item
      We have already seen this in \cref{invariants via internal hom}.
    \item
      The counit~$\varepsilon$ of~$\Univ(\glie)$ induces an isomorphism of~\module{$\Univ(\glie)$}
      \[
        \Univ(\glie) / I
        \to
        \kf \,,
        \quad
        \class{x}
        \mapsto
        \varepsilon(x)
      \]
      where~$I$ is the augumentation ideal of~$\Univ(\glie)$.
      It follows that
      \[
        \kf \otimes_{\Univ(\glie)} M
        \cong
        ( \Univ(\glie) / I ) \otimes_{\Univ(\glie)} M
        \cong
        M / I M \,,
      \]
      and this isomorphism is given by
      \[
        1 \otimes m
        =
        \varepsilon(1) \otimes m
        \mapsto
        \class{1} \otimes m
        \mapsto
        \class{ 1 \cdot m }
        =
        \class{ m } \,.
      \]
      The augumentation ideal~$I$ is generated by~$\glie$ as a left ideal of~$\Univ(\glie)$, whence
      \[
        M / I M
        =
        M / \ideal{\glie} M
        =
        M / \glie M
        =
        M_{\glie} \,.
      \]
      We have altogether shown the claimed isomorphism.
    \qedhere
  \end{enumerate}
\end{proof}


\begin{remark}
  \label{invariants and coinvariants via hom and tensor as functors}
  The isomorphisms from \cref{invariant and coinvariants via hom and tensor} are natural in~$M$ and thus assemble into isomorphisms of functors
  \[
    (-)^{\glie}
    \cong
    \Hom_{\Univ(\glie)}( \kf, - ) \,,
    \quad
    (-)_{\glie}
    \cong
    \kf \otimes_{\Univ(\glie)} (-) \,.
  \]
\end{remark}


\begin{fluff}
  Let~$\glie$ be a Lie~algebra.
  The Koszul complex of~$\glie$ is a free, and thus projective, resolution of~$\kf$ as a left~\module{$\Univ(\glie)$}.
  We can therefore use this resolution to compute the right derived functors~$\Ext_{\Univ(\glie)}^n(\kf, -)$ with~$n \geq 0$.
  We can similarly use the right Koszul resolution to compute the left derived functors~$\cramped{\Tor^{\Univ(\glie)}_n(\kf, -)}$ with~$n \geq 0$.
  We see from \cref{lie algebra homology and cohomology via koszul complex} that the result of these computations are precisely the Lie algebra cohomology of~$\glie$ and Lie algebra homology of~$\Univ(\glie)$.
  We thus have
  \[
    \Homology_n(\glie, -)
    \cong
    \Tor^{\Univ(\glie)}_n(\kf, -)
    \quad\text{and}\quad
    \Homology^n(\glie, -)
    \cong
    \Ext^n_{\Univ(\glie)}(\kf, -)
  \]
  for all~$n \geq 0$.
  In other words, the Lie algebra cohomology of~$\glie$ is given by the right derived functors of~$\Hom_{\glie}(\kf, -)$, and the Lie algebra homology of~$\glie$ is given by the left derived functors of~$\kf \otimes_{\Univ(\glie)} (-)$.
  It follows from \cref{invariants and coinvariants via hom and tensor as functors} that the Lie algebra cohomology of~$\glie$ is given by the right derived functors of~$(-)^{\glie}$, and the Lie~algebra homology of~$\glie$ is given by the left derived functors of~$(-)_{\glie}$.
\end{fluff}


\begin{proposition}
  Let~$\glie$ be a Lie~algebra and let~$I$ be the augumentation ideal of~$\glie$.
  Let~$M$ be a representation of~$\glie$.
  Then the map
  \[
    R
    \colon
    \Hom_{\Univ(\glie)}(I, M)
    \to
    \Der(\glie, M) \,,
    \quad
    d
    \mapsto
    \restrict{d}{\glie}
  \]
  is an isomorphism of vector spaces.
\end{proposition}


\begin{proof}
  We first note that the augumentation ideal~$I$ is a~\submodule{$\Univ(\glie)$} of~$\Univ(\glie)$.
  It therefore makes sense to consider~$\Hom_{\Univ(\glie)}(I, M)$.

  We know from the exactness of the Koszul complex that we have an exact sequence
  \[
    \Univ(\glie) \otimes \Exterior^2(\glie)
    \xto{d_2}
    \Univ(\glie) \otimes \glie
    \xto{d_1}
    I
    \to
    0 \,.
  \]
  This sequence is right-exact, so we may apply the contravariant~\functor{$\Hom$}~$\Hom_{\Univ(\glie),-}$ to get the left-exact sequence of vector spaces
  \[
    0
    \to
    \Hom_{\Univ(\glie)}(I, M)
    \xto{d_1^*}
    \Hom_{\Univ(\glie)}( \Univ(\glie) \otimes \glie, M )
    \xto{d_2^*}
    \Hom_{\Univ(\glie)}\Bigl( \Univ(\glie) \otimes \Exterior^2(\glie), M \Bigr)
  \]
  We have by the universal property of the extension of scalars a linear {\onetoonetext} correspondence
  \begin{align*}
    \Hom_{\Univ(\glie)}( \Univ(\glie) \otimes \glie, M )
    &\onetoone
    \Hom_{\kf}( \glie, M ) \,,
    \\
    f
    &\mapsto
    \bigl( x \mapsto f(1 \otimes x) \bigr) \,,
    \\
    \bigl( \induced{g} \colon y \otimes x \mapsto y \cdot g(x) \bigr)
    &\mapsfrom
    g \,.
  \end{align*}
  Under this isomorphism we get the exact sequence
  \begin{equation}
    \label{left exact sequence for derivations}
    0
    \to
    \Hom_{\Univ(\glie)}(I, M)
    \xto{\del_1}
    \Hom_{\kf}( \glie, M )
    \xto{\del_2}
    \Hom_{\Univ(\glie)}\Bigl( \Univ(\glie) \otimes \Exterior^2(\glie), M \Bigr) \,.
  \end{equation}
  
  We have
  \[
    \delta_1(f)(x)
    =
    d_1^*(f)(1 \otimes x)
    =
    f( d_1(1 \otimes x) )
    =
    f(1 \cdot x)
    =
    f(x) \,.
  \]
  In other words~$\delta_1$ is precisely the restriction map~$R$.
  It follows from the left-exactness of the sequence~\eqref{left exact sequence for derivations} that the restriction map~$R$ identifiey~$\Hom_{\Univ(\glie)}(I, M)$ with a certain linear subspace of~$\Hom_{\kf}(\glie, M)$, namely the kernel of~$\del_2$.

  We also have
  \begin{align*}
    \del_2(g)(y \otimes x_1 \wedge x_2)
    &=
    d_2^*(\induced{g})(y \otimes x_1 \wedge x_2)
    \\
    &=
    \induced{g}( d_2( y \otimes x_1 \wedge x_2 ) )
    \\
    &=
    \induced{g}
    (
      - y \otimes [x_1, x_2]
      + (y x_1) \otimes x_2
      - (y x_2) \otimes x_1
    )
    \\
    &=
    - y \cdot g([x_1, x_2])
    + (y x_1) \cdot g(x_2)
    + (y x_2) \cdot g(x_1)
    \\
    &=
    - y \cdot g([x_1, x_2])
    + y \cdot x_1 \cdot g(x_2)
    - y \cdot x_2 \cdot g(x_1)
    \\
    &=
    y \cdot
    (
      x_1 \cdot g(x_2) - x_2 \cdot g(x_1) - g([x_1, x_2])
    ) 
  \end{align*}
  for all~$g \in \Hom_{\kf}(\glie, M)$,~$y \in \Univ(\glie)$,~$x_1, x_2 \in \glie$.
  It follows that~$g$ is contained in the kernel of~$\del_2$ if and only if
  \[
    x_1 \cdot g(x_2) - x_2 \cdot g(x_1)
    =
    g([x_1, x_2])
  \]
  for all~$x_1, x_2 \in \glie$, i.e. if and only if~$g$ is a derivation from~$\glie$ to~$M$.
  We thus find that the kernel of~$\del_2$ is the space of derivations from~$\glie$ to~$M$.

  We have altogether shown that the restriction map~$R$ identifies the space of homomorphisms~$\Hom_{\Univ(\glie)}(I, M)$ with the space of derivations~$\Der(\glie, M)$.
\end{proof}


\begin{proof}[Second proof]
  We first note that the augumentation ideal~$I$ is a~\submodule{$\Univ(\glie)$} of~$\Univ(\glie)$.
  It therefore makes sense to consider~$\Hom_{\Univ(\glie)}(I, M)$.
  Now we calculate for every homomorphisms of~\modules{$\Univ(\glie)$}~$d$ from~$I$ to~$M$ that
  \[
    d([x,y])
    =
    d(xy - yx)
    =
    d(xy) - d(yx)
    =
    x \cdot d(y) - y \cdot d(x)
  \]
  for all~$x, y \in \glie$.
  The restriction~$\restrict{d}{\glie}$ is therefore a derivation from~$\glie$ to~$M$.
  The map~$R$ is therefore well-defined.
  It is also linear.

  We first want to show that every derivation~$\delta$ from~$\glie$ to~$M$ extends to a homomorphism of~\modules{$\Univ(\glie)$} from~$I$ to~$M$.
  We know from the exactness of the Koszul complex that we have an exact sequence
  \[
    \Univ(\glie) \otimes \Exterior^2(\glie)
    \xto{d_2}
    \Univ(\glie) \otimes \glie
    \xto{d_1}
    I
    \to
    0 \,.
  \]
  It follows from the universal property of the extension of scalars that the derivation~$\delta$ -- which is in particular a linear map - extends to a homomorphism of~\modules{$\Univ(\glie)$}~$d'$ from~$\Univ(\glie) \otimes \glie$ to~$M$ given by
  \[
    d'(y \otimes x)
    =
    y \cdot \delta(x)
  \]
  for all~$y \in \Univ(\glie)$,~$x \in \glie$.
  This homomorphism~$d'$ vanishes on the image of~$d_2$ because
  \begin{align*}
    d'( d_2(y \otimes x_1 \wedge x_2) )
    &=
    d'
    (
      - y \otimes [x_1, x_2]
      + (y x_1) \otimes x_2
      - (y x_2) \otimes x_1
    )
    \\
    &=
    - y \cdot \delta([x_1, x_2])
    + (y x_1) \cdot \delta(x_2)
    - (y x_2) \cdot \delta(x_1)
    \\
    &=
    - y \cdot \delta([x_1, x_2])
    + y \cdot x_1 \cdot \delta(x_2)
    - y \cdot x_2 \cdot \delta(x_1)
    \\
    &=
    y \cdot
    (
      x_1 \cdot \delta(x_2)
      - x_2 \cdot \delta(x_1)
      - \delta([x_1, x_2])
    )
    \\
    &=
    y \cdot 0
    \\
    &=
    0
  \end{align*}
  where we have used that~$\delta([x_1, x_2]) = x_1 \cdot \delta(x_2) - x_2 \cdot \delta(x_1)$ because~$\delta$ is a derivation.
  It follows that~$d'$ factors through a homomorphism of~\modules{$\Univ(\glie)$} from~$I$ to~$M$.
  This homomorphism satisfies
  \[
    d(x)
    =
    d( d_1(1 \otimes x) )
    =
    d'( 1 \otimes x )
    =
    1 \cdot \delta(x)
    =
    \delta(x)
  \]
  for all~$x \in \glie$.
  It is therefore an extension of~$\delta$.
  This shows the surjectivity of~$R$.

  It follows from \cref{augumentation ideal is spanned by monomials} that the augumentaiton ideal~$I$ is generated by~$\glie$ as an ideal, and thus as a~\module{$\Univ(\glie)$}.
  Every homomorphism of~\modules{$\Univ(\glie)$}~$d$ from~$I$ to~$M$ is therefore uniquely determined by its restriction~$\restrict{d}{\glie}$.
  This shows the injectivity of~$R$.
\end{proof}


\begin{recall}
  \label{homological algebra for pulling out an exact functor}
  Let~$\Acat$,~$\Bcat$ and~$\Ccat$ be three abelian categories such that the categories~$\Acat$ and~$\Bcat$ admit enough injectives.
  Let~$F$ be an exact functor from~$\Acat$ to~$\Bcat$ which maps injectives to injectives.
  Let~$G$ be a left exact functor from~$\Bcat$ to~$\Ccat$.
  Then the composite~$G \circ F$ is again left exact and for all~$n \geq 0$,
  \[
    \Right^n (G \circ F)
    \cong
    (\Right^n G) \circ F \,.
  \]
  
  Indeed, by precomposing the~\functor{$\delta$}~$\Right^\bullet G$ with the exact functor~$F$ we get again a~\functor{$\delta$}, which one might denote by~$(\Right^\bullet G) \circ F$.
  Every injective object of~$\Acat$ is mapped by~$F$ to an injective object of~$\Bcat$, which is then annihilated by~$\Right^n G$ for every~$n \geq 1$.
  This shows that the functors~$(\Right^n G) \circ F$ for~$n \geq 1$ annihilate all injective objects of~$\Acat$.
  The~\functor{$\delta$}~$(\Right^\bullet G) \circ F$ is therefore universal.
  We also have~$(\Right^0 G) \circ F \cong G \circ F$ because~$\Right^0 G \cong G$.
  We thus find that the universal~\functor{$\delta$}~$(\Right^\bullet G) \circ F$ is the right derived functor of~$G \circ F$.
\end{recall}


\begin{proposition}
  Let~$\glie$ be a Lie~algebra and let~$M$,~$N$ be two representations of~$\glie$.
  Then
  \[
    \Ext_{\Univ(\glie)}^n(M, N)
    \cong
    \Homology^n(\glie, \Hom_{\kf}(M, N)) \,.
  \]
\end{proposition}


\begin{proof}
  The functor~$\Hom_{\kf}(M, -)$ exact.
  It also maps injecitves to injectives because right adjoint to the exact functor~$(-) \otimes_{\kf} M$ by \cref{enriched tensor hom adjunction}.
  It follows from \cref{homological algebra for pulling out an exact functor} that
  \begin{align*}
    \Ext_{\Univ(\glie)}^n(M, -)
    &\cong
    \Right^n \Hom_{\Univ(\glie)}(M, -)
    \\
    &=
    \Right^n \Hom_{\kf}(M, -)^{\glie}
    \\
    &=
    \Right^n \bigl( (-)^{\glie} \circ \Hom_{\kf}(M, -) \bigr)
    \\
    &\cong
    \Right^n\bigl( (-)^{\glie} \bigr) \circ \Hom_{\kf}(M, -)
    \\
    &\cong
    \Homology^n(\glie, -) \circ \Hom_{\kf}(M, -)
    \\
    &=
    \Homology^n(\glie, \Hom_{\kf}(M, -)) \,.
  \end{align*}
  This proves the assertion.
\end{proof}





