\section{Abstract Version of the Poincar\'{e}--Birkhoff--Witt Theorem}


\begin{theorem}[Poincar\'{e}--Birkhoff--Witt (abstract version)]
  \label{pbw abstract}
  Let~$\glie$ be a Lie~algebra and denote by~$\pi$ the canonical projection~$\Tensor(\glie) \to \Univ(\glie)$ given by
  \[
    \pi
    \colon
    \Tensor(\glie)
    \to
    \Univ(\glie) \,,
    \quad
    x_1 \tensor \dotsb \tensor x_n
    \mapsto
    x_1 \dotsm x_n
  \]
  for all $x_1, \dotsc, x_n \in \glie$.
  Then~$\pi$ induces a homomorphism of graded~{\algebras{$\kf$}}
  \[
    \gr(\pi)
    \colon
    \Tensor(\glie)
    =
    \gr(\Tensor(\glie))
    \to
    \gr(\Univ(\glie))
  \]
  and we also have a homomorphism of graded~{\algebras{$\kf$}}~$\pi' \colon \Tensor(\glie) \to \Symm(\glie)$ that  is given by
  \[
    \pi'
    \colon
    \Tensor(\glie)
    \to
    \Symm(\glie) \,,
    \quad
    x_1 \tensor \dotsb \tensor x_n
    \mapsto
    x_1 \dotsm x_n
  \]
  for all~$x_1, \dotsc, x_n \in \glie$.
  Then both~$\pi$ and~$\pi'$ have the same kernel and thus induce an isomorphism of graded algebras~$\Symm(\glie) \to \gr(\Univ(\glie))$ given by
  \begin{equation}
    \label{definition of abstract pbw isomorphism}
    \varphi
    \colon
    \Symm(\glie)
    \to
    \gr(\Univ(\glie)) \,,
    \quad
    x_1 \dotsm x_n
    \mapsto
    [x_1 \dotsm x_n]_n \,.
  \end{equation}
  for all~$x_1, \dotsc, x_n \in \glie$.
\end{theorem}


\begin{proposition}
  The abstract versions of the PBW~theorem is equivalent to the concrete version.
\end{proposition}


\begin{proof}
  It follows from the definition of~$\gr(\pi)$ that
  \[
    \gr(\pi)(x \tensor y - y \tensor x)
    =
    [xy - yx]_2
    \in
    \gr(\Univ(\glie))_2
  \]
  for all~$x, y \in \glie$.
  We have for the occuring representative~$xy - yx \in \Univ(\glie)_{(2)}$ that
  \[
    xy - yx
    =
    [x,y]
    \in
    \Univ(\glie)_{(1)}
  \]
  and hence
  \[
    [xy - yx]_2
    =
    0 \,.
  \]
  This shows that~$\gr(\pi)(x \tensor y - y \tensor x) = 0$ for all~$x, y \in \glie$.
  The kernel of~$\pi'$ is generated by the elements~$x \tensor y - y \tensor x$ with~$x,y \in \glie$ so it follows that~$\pi'$ factors through a homomorphism of graded~\algebras{$\kf$}~$\varphi \colon \Symm(\glie) \to \gr(\Univ(\glie))$ that is given by~\eqref{definition of abstract pbw isomorphism}.
  
  \begin{implicationlist}
    \item[concrete~$\implies$~abstract:]
      It follows for every degree~$d \geq 0$ from \cref{pbw concrete basis part filtered part} that the graded component~$\gr[d](\Univ(\glie))$ has the monomials~$x_\alpha$ with~$\alpha \in I^d$ as a basis.
      The homogeneous component~$\Symm^d(\glie)$ has as a basis the simple symmetric tensors~$x_{i_1} \dotsm x_{i_d}$ with~$i_1, \dotsc, i_d \in I$ and~$i_1 \leq \dotsb \leq$.
      The homomorphism~$\varphi$ restricts to a bijection between these bases and is therefore an isomorphism (of vector spaces and hence of graded~{\algebras{$\kf$}}).
    \item[abstract~$\implies$~concrete:]
      We need to show that the monomials~$x_\alpha$ with~$\alpha \in I^*$ are linearly independent.
      Suppose otherwise.
      Then there exists some minimal~$n \geq 0$ such that the monomials~$x_\alpha$ with~$\alpha \in I^{(n)}$ are linearly dependent.
      Let
      \[
        0
        =
        \sum_{\alpha \in I^{(n)}}
        \lambda_\alpha x_\alpha
      \]
      be a non-trivial linear relation.
      Then
      \[
        0
        =
        \sum_{\alpha \in I^{(n)}}
        \lambda_\alpha x_\alpha
        \equiv
        \sum_{\alpha \in I^n}
        \lambda_\alpha x_\alpha
        \mod
        \Univ(\glie)_{(n-1)}
      \]
      and therefore
      \begin{equation}
        \label{nontrivial linear combination in graded}
        0
        =
        \sum_{\alpha \in I^n}
        \lambda_\alpha [x_\alpha]_n
      \end{equation}
      in~$\gr[n](\Univ(\glie))$.
      The minimality of~$n$ means that~\eqref{nontrivial linear combination in graded} is a non-trivial linear relation, i.e.\ that~$\lambda_\alpha \neq 0$ for some~$\alpha \in I^n$.
      By using the isomorphism~$\varphi$ we get in the corresponding graded component~$\Symm^n(\glie)$ of~$\Symm(\glie)$ a non-trivial linear relation between the simple symmetric tensors.
      But these are linear independent --- a contradiction.
    \qedhere
  \end{implicationlist}
\end{proof}


\begin{corollary}
  \label{uea contains no zero divisors}
  Let~$\glie$ be a Lie~algebra.
  \begin{enumerate}
    \item
      The universal enveloping algebra~$\Univ(\glie)$ contains no nonzero zero divisors.
    \item
      If~$\glie$ is finite-dimensional then~$\Univ(\glie)$ is both left noetherian and right noetherian.
  \end{enumerate}
\end{corollary}


\begin{proof}
  \leavevmode
  \begin{enumerate}
    \item
      The statement holds for the associated graded algebra~$\gr(\Univ(\glie)) \cong \Symm(\glie)$ because it is an integral domain.
      It follows from \cref{associated graded algebra and zero divisors} that this also holds for~$\Univ(\glie)$.
    \item
      If~$\glie$ is~{\dimensional{$n$}} then~$\gr(\Univ(\glie)) \cong \Symm(\glie) \cong k[t_1, \dotsc, t_n]$ is noetherian and therefore~$\Univ(\glie)$ is both left noetherian and right noetherian by \cref{universal enveloping reflects chain conditions}
    \qedhere
  \end{enumerate}
\end{proof}


\begin{lemma}
  If~$\glie$ is any nonzero Lie~algebra then~$\Univ(\glie)$ is neither left artinian nor right artinian.
\end{lemma}


\begin{proof}
  We show that~$\Univ(\glie)$ is not left artinian;
  that~$\Univ(\glie)$ is not right artinian can then be shown in the same way.
  
  Let~$(x_i)_{i \in I}$ be a basis of~$\glie$ such that~$(I, \leq)$ is linearly ordered with maximal element~$j$.
  Then for every~$m \geq 0$ the left ideal~$\Univ(\glie) x_j^m$ has as a basis all monomials~$x_{i_1}^{n_1} \dotsm x_{i_r}^{n_r} x_j^m$ with~$r \geq 0$,~$n_i \geq 0$ and~$i_1 < \dotsb < i_r < j$.
  Therefore
  \[
    \Univ(\glie)
    \supsetneq
    \Univ(\glie) x_j
    \supsetneq
    \Univ(\glie) x_j^2
    \supsetneq
    \Univ(\glie) x_j^3
    \supsetneq
    \dotsb
  \]
  is a strictly decreasing sequence of left ideals in~$\Univ(\glie)$.
\end{proof}


\begin{remark}
  There exist examples for infinite-dimensional Lie~algebras whose universal enveloping algebras are not noetherian, see \cite{uea_of_witt_algebra_not_noetherian}.
  It is (according to \cite[p.\ xix]{introduction_to_noncommutative_noetherian}) an open problem if the universal enveloping algebra of an infinite-dimensional Lie~algebra must be non-noetherian.
\end{remark}


\begin{remark}
  One can think about the universal enveloping algebra~$\Univ(\glie)$ as a deformation of the symmetric algebra~$\Symm(\glie)$, in the sense that we have a family of algebras
  \[
    A_t
    \defined
    \Tensor(\glie)/(x \tensor y - y \tensor x - t[x,y] \suchthat x, y \in \glie)
  \]
  that is parametrized over a parameter~$t \in \kf$ with~$A_0 \cong \Symm(\glie)$ and~$A_1 \cong \Univ(\glie)$.
  One can then prove the abstract version of the PBW~theorem by examining this deformation.
  See~\cite{pbw_deformation} for more details on this.
\end{remark}





