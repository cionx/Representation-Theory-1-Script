\section{Consequences of the PBW Theorem}


\begin{example}
	\label{converse to warning about generating set for the associated graded}
	We can now give a counterexample to part~\ref{generators of associated graded part} of \cref{generators of associated graded} as follows.

	Let~$\glie$ be the Lie algebra~$\sllie(2, \kf)$ and let~$A$ be its universal enveloping algebra with the induced filtration from \cref{examples for filtered algebras}
	We know that~$A$ is generated as a~\algebra{$\kf$} by the standard basis~$e$,~$h$,~$f$ of~$\sllie(2, \kf)$
	It follows from the commutator relation~$[e,f] = h$ that~$A$ is already generated by the two elements~$e$ and~$f$.
	But the algebra~$\gr(A)$ is isomorphic to the polynomial algebra~$\kf[x, y, z]$ with~$x$ corresponding to~$\fclass{ e }_1$,~$y$ corresponding to~$\fclass{h}_1$, and~$z$ corresponding to~$\fclass{f}_1$.
	We find that~$\gr(A)$ is not generated by~$[e]_1$ and~$[f]_1$ because~$\kf[x,y,z]$ is not generated by~$x$,~$z$.
\end{example}



\subsection{Resulting Identifications}

\begin{proposition}
	\label{embedding into uea}
	The canonical homomorphism of Lie~algebras from~$\glie$ to~$\Univ(\glie)$ is injective.
\end{proposition}


\begin{proof}
	It follows from the PBW~theorem that the basis~$(x_i)_{i \in I}$ of~$\glie$ is mapped to linearly independent elements of~$\Univ(\glie)$.
\end{proof}


\begin{convention}
	\label{identification of lie algebra with image in its uea}
	Let~$\glie$ be a Lie~algebra.
	In the rest of these notes we will always identify~$\glie$ with its image in its universal enveloping algebra~$\Univ(\glie)$ via the canonical homomorphism of Lie~algebras from~$\glie$ to~$\Univ(\glie)$.
\end{convention}


%\begin{remark}
%  We may rewrite the isomorphism~$\Univ(\glie^{\op}) \cong \Univ(\glie)^{\op}$ from \cref{uea of opposite by first principles}, that is given by~$x \mapsto x$ for every~$x \in \glie$, as~$\Univ(\glie^{\op}) = \Univ(\glie)^{\op}$ by the aformentioned identification from \cref{identification into uea}.
%\end{remark}


\begin{proposition}
	\label{inclusion of universal enveloping algebras}
	Let~$\glie$ be a Lie~algebra and let~$\hlie$ be a Lie~subalgebra of~$\glie$.
	Let~$\iota$ be the inclusion map from~$\hlie$ to~$\glie$.
	The induced hommorphism of algebras~$\Univ(\iota)$ from~$\Univ(\hlie)$ to~$\Univ(\glie)$ is injective.
\end{proposition}


\begin{proof}
	Let~$(x_i)_{i \in I}$ be a basis of~$\hlie$, and let~$(x_j)_{j \in J}$ be an extended basis of~$\glie$.
	Let~$\leq$ be a linear order on~$J$.
	The homomorphism~$\Univ(\iota)$ maps the resulting PBW-basis of~$\Univ(\hlie)$ injectively into the resulting PBW-basis of~$\Univ(\glie)$.
	It is therefore injective.
\end{proof}


\begin{convention}
	\label{identification of uea of lie subalgebra}
	Let~$\glie$ be a Lie~algebra, let~$\hlie$ be a Lie~subalgebra of~$\glie$ and let~$\iota$ be the inclusion map from~$\hlie$ to~$\glie$.
	In the rest of these notes we will identify~$\Univ(\hlie)$ with its image in~$\Univ(\glie)$ under the injective homomorphism of algebras~$\Univ(\iota)$.
\end{convention}


\begin{remark}
	Let~$\glie$ be a Lie~algebra and let~$\hlie$ be a Lie~subalgebra of~$\glie$.
	Under the identifcations from \cref{identification of lie algebra with image in its uea} and \cref{identification of uea of lie subalgebra} we have the following commutative square diagram of inclusions.
	\[
		\begin{tikzcd}
			\Univ(\hlie)
			\arrow[phantom]{r}{\subseteq}
			&
			\Univ(\glie)
			\\
			\hlie
			\arrow[phantom]{u}[rotate=90]{\subseteq}
			\arrow[phantom]{r}{\subseteq}
			&
			\glie
			\arrow[phantom]{u}[rotate=90]{\subseteq}
		\end{tikzcd}
	\]
\end{remark}



\subsection{Bases}

\begin{proposition}
	Let~$\glie$ be a Lie~algebra and let~$\hlie$ be a Lie~subalgebra of~$\glie$.
	Then~$\Univ(\glie)$ is free as a left~\module{$\Univ(\hlie)$}, and also free as a right~\module{$\Univ(\hlie)$}.
\end{proposition}


\begin{proof}
	Let~$(x_i)_{i \in I}$ be basis of~$\hlie$ and let~$(x_j)_{j \in J}$ be an extended basis of~$\glie$.
	Let~$\leq$ be a linear order on~$\glie$ such that~$i \leq j$ for all~$i \in I$,~$j \in J \setminus I$.
	It follows from the PBW~theorem that
	\begin{align*}
		\Univ(\glie)
		&=
		\bigoplus{}_{
			n \geq 0, \;
			j_1, \dotsc, j_n \in J, \;
			j_1 \leq \dotsb \leq j_n
		}
		\gen{ x_{j_1} \dotsm x_{j_n} }
		\\
		&=
		\bigoplus{}_{
			r, s \geq 0, \;
			i_1, \dotsc, i_r \in I, \,;
			j_1, \dotsc, j_s \in J \setminus I, \;
			i_1 \leq \dotsb \leq i_r,\;
			j_1 \leq \dotsb \leq j_s
		}
		\gen{ x_{i_1} \dotsm x_{i_r} x_{j_1} \dotsm x_{j_s} }_{\kf}
		\\
		&=
		\bigoplus{}_{
			r, s \geq 0, \;
			i_1, \dotsc, i_r \in I, \;
			j_1, \dotsc, j_s \in J \setminus I, \;
			i_1 \leq \dotsb \leq i_r, \;
			j_1 \leq \dotsb \leq j_s
		}
		\gen{ x_{i_1} \dotsm x_{i_r} }_{\kf}
		\cdot x_{j_1} \dotsm x_{j_s}
		\\
		&=
		\bigoplus{}_{
			s \geq 0, \;
			j_1, \dotsc, j_s \in J \setminus I, \;
			j_1 \leq \dotsb \leq j_s
		}
		\cdot
		\bigoplus{}_{
			r \geq 0, \;
			i_1, \dotsc, i_r \in I, \;
			i_1 \leq \dotsb \leq i_r,
		}
		\gen{ x_{i_1} \dotsm x_{i_r} }_{\kf}
		\cdot x_{j_1} \dotsm x_{j_s}
		\\
		&=
		\bigoplus{}_{
			s \geq 0, \;
			j_1, \dotsc, j_s \in J \setminus I, \;
			j_1 \leq \dotsb \leq j_s
		}
		\Univ(\hlie)
		\cdot x_{j_1} \dotsm x_{j_s}
	\end{align*}
	The elements
	\[
		x_{j_1} \dotsm x_{j_s}
		\qquad
		\text{with~$s \geq 0$ and~$j_1, \dotsc, j_s \in J \setminus I$ with~$j_1 \leq \dotsb \leq j_s$}
	\]
	form therefore a basis of~$\Univ(\glie)$ as a left~\module{$\Univ(\hlie)$}.
	
	That~$\Univ(\glie)$ is free as a right~\module{$\hlie$} can be shown in the same way.
\end{proof}


\begin{proposition}
	\label{uea of direct sum of subspaces}
	Let~$\glie$ be a Lie algebra and let~$\hlie$ and~$\klie$ be two Lie~subalgebras of~$\glie$ such that~$\glie = \hlie \oplus \klie$ as vector spaces.
	The multiplication map
	\[
		m
		\colon
		\Univ(\hlie) \tensor_{\kf} \Univ(\klie)
		\to
		\Univ(\glie) \,,
		\quad
		x \tensor y
		\mapsto
		xy
	\]
	is an isomorphism of~\bimodules{$\Univ(\hlie)$}{$\Univ(\klie)$}.
\end{proposition}


\begin{proof}
	The map~$m$ is linear and a homomorphism of~\bimodules{$\Univ(\hlie)$}{$\Univ(\klie)$}.
	It remains to show that~$m$ is an isomorphism of vector spaces.

	Let~$(x_i)_{i \in I}$ be a basis of~$\hlie$ and let $(x_j)_{j \in J}$ be a basis of~$\klie$.
	We may assume that the index sets~$I$ and~$J$ are disjoint.
	Let~$K$ the disjoint union of~$I$ and~$J$.
	The family $(x_k)_{k \in K}$ is then a basis of~$\glie$.

	We endow the index sets~$I$ and~$J$ with linear orders, making them into linearly ordered sets~$(I, \leq)$ and~$(J, \leq)$.
	We extend these linear orders on~$I$ and~$J$ to a linear order on~$K$ such that~$i \leq j$ for all~$i \in I$ and~$j \in J$.

	The induced PBW~bases of~$\Univ(\hlie)$ and~$\Univ(\klie)$ induce a basis on~$\Univ(\hlie) \otimes_{\kf} \Univ(\klie)$.
	The given multiplication map~$m$ restricts to a bijection between this basis of~$\Univ(\hlie) \otimes_{\kf} \Univ(\klie)$ and the induced PBW~basis of~$\Univ(\glie)$.
	It is therefore an isomorphism of vector spaces.
\end{proof}


\begin{remark}
	Let~$\glie$ be the Lie~algebra and suppose that~$\glie$ is the internal direct sum of two Lie~subalgebras~$\hlie_1$ and~$\hlie_2$.
	Let~$i_1$ be the inclusion map from~$\hlie_1$ to~$\glie$ and let~$i_2$ be inclusion map from~$\hlie_2$ to~$\glie$.
	The induced homomorphisms of algebras~$\Univ(\iota_1)$ from~$\Univ(\hlie_1)$ to~$\Univ(\glie)$ and~$\Univ(\iota_2)$ from~$\Univ(\hlie_2)$ to~$\Univ(\glie)$ assemble into a single homomorphism of algebras
	\[
		\Phi
		\colon
		\Univ(\hlie_1) \otimes \Univ(\hlie_2)
		\to
		\Univ(\glie) \,.
	\]
	This homomorphism~$\Phi$ is given on simple tensors by
	\[
		\Phi(x \otimes y)
		=
		x y
	\]
	for all~$x \in \Univ(\hlie_1)$,~$y \in \Univ(\hlie_2)$.
	It follows from \cref{uea of direct sum of subspaces} that the homomorphism~$\Phi$ is an isomorphism of vector spaces and thus an isomorphism of algebras.

	With this we have once again proven that~$\Univ(\glie)$ is isomorphic to~$\Univ(\hlie_1) \otimes \Univ(\hlie_2)$.
\end{remark}



\subsection{Zero Divisors}

\begin{proposition}
	\label{uea contains no zero divisors}
	Let~$\glie$ be a Lie~algebra.
	The universal enveloping algebra~$\Univ(\glie)$ contains no nonzero zero divisors.\index{zero divisors}
\end{proposition}


\begin{proof}
	The associated graded algebra~$\gr(\Univ(\glie))$ is isomorphic to the symmetric algebra~$\Symm(\glie)$, which is an integral domain, since it is isomorphic to a polynomial algebra~$\kf[x_i \suchthat i \in I]$ (where the cardinality of~$I$ is the dimension of~$\glie$).
	It follows from \cref{associated graded algebra and zero divisors} that the original algebra~$\Univ(\glie)$ also has no nonzero zero divisors.
\end{proof}



\subsection{Chain Conditions}

\begin{proposition}
	\label{uea is noetherian if fd}
	Let~$\glie$ be a finite-dimensional Lie~algebra.
	The universal enveloping algebra~$\Univ(\glie)$ is both left noetherian\index{left noetherian}\index{noetherian} and right noetherian\index{right noetherian}\index{noetherian}.
\end{proposition}


\begin{proof}
	Let~$n$ be the dimension of~$\glie$.
	The associated graded algebra~$\gr(\Univ(\glie))$ is isomorphic to~$\Symm(\glie)$, which in turn is isomorphic to the polynomial algebra~$\kf[t_1, \dotsc, t_n]$.
	This algebra is commutative and noetherian.
	It therefore follows from \cref{universal enveloping reflects chain conditions} that the original algebra~$\Univ(\glie)$ is both left noetherian and right noetherian.
\end{proof}


\begin{remark}
	There exist infinite-dimensional Lie~algebras whose universal enveloping algebras are not noetherian\index{noetherian}.
	We refer to \cite{sierra_walten_uea_witt_not_noetherian} for an explicit example.
	It is (according to \cite[p.~xix]{goodearl_warfield_introduction_to_nn_rings}) an open problem if the converse of \cref{uea is noetherian if fd} holds, i.e. if for an infinite-dimensional Lie~algebra~$\glie$ its universal enveloping algebra~$\Univ(\glie)$ is necessarily non-noetherian.
\end{remark}


\begin{proposition}
	Let~$\glie$ be a nonzero Lie~algebra.
	The algebra~$\Univ(\glie)$ is neither left artinian\index{left artinian}\index{artinian} nor right artinian\index{right artinian}\index{artinian}.
\end{proposition}


\begin{proof}
	Let~$(x_i)_{i \in I}$ be a basis of~$\glie$ such that~$(I, \leq)$ is linearly ordered with maximal element~$j$.
	The left ideal~$\Univ(\glie) x_j^m$ of~$\Univ(\glie)$ has by the PBW~theorem the monomials
	\[
		x_{i_1}^{n_1} \dotsm x_{i_r}^{n_r} x_j^{m+l}
		\qquad
		\text{with~$r \geq 0$,~$n_1, \dotsc n_r, l \geq 0$ and~$i_1 < \dotsb < i_r < j$}
	\]
	as a basis.
	It follows that we have strictly decreasing sequence of left ideal of~$\Univ(\glie)$ given by
	\[
		\Univ(\glie)
		\supsetneq
		\Univ(\glie) x_j
		\supsetneq
		\Univ(\glie) x_j^2
		\supsetneq
		\Univ(\glie) x_j^3
		\supsetneq
		\dotsb
	\]
	This shows that the algebra~$\Univ(\glie)$ is not left artinian.

	That~$\Univ(\glie)$ is not right artinian can be shown in the same way.
	It also follows from the isomorphism~$\Univ(\glie)^{\op} \cong \Univ(\glie^{\op})$ because~$\Univ(\glie^{\op})$ is not left-artinian.
\end{proof}



\subsection{Existence of Free Lie Algebras}

\begin{fluff}
	Let~$V$ be a~\vectorspace{$\kf$}.
	We have seen in \cref{uea of free lie algebra} that the universal enveloping algebra of the free Lie~algebra~$\freelievect(V)$\index{free Lie algebra!on a vector space} is given by the tensor algebra~$\Tensor(V)$.
	We have seen in \cref{embedding into uea} and \cref{identification of lie algebra with image in its uea} that we can therefore regard~$\freelievect(V)$ is a Lie~subalgebra of~$\Tensor(V)$.
	We can use this observation to construct free Lie~algebras, and hence show their existence.
\end{fluff}


\begin{proposition}
	\label{construction of free lie algebras}
	Let~$V$ be a vector space.
	Let~$\flie$ be the Lie~subalgebra of~$\Tensor(V)$ generated by~$V$ and let~$i$ be the inclusion map from~$V$ to~$\Tensor(V)$.
	The Lie~algebra~$\flie$ together with the linear map~$i$ is the free~\liealgebra{$\kf$} of~$V$\index{free Lie algebra!on a vector space}.
\end{proposition}


\begin{proof}
	So let~$\hlie$ be a~\liealgebra{$\kf$} and let~$f$ be a linear map from~$V$ to~$\hlie$.
	We need to show that there exists a unique homomorphism of Lie~algebras~$\varphi$ from~$\flie$ to~$\hlie$ such that~$\varphi \circ i = f$.

	There exists by the universal property of the tensor algebra~$\Tensor(V)$ a unique homomorphism of algebras~$\Phi$ from~$\Tensor(V)$ to~$\Univ(\hlie)$ with~$\Phi(v) = f(v)$ for every~$v \in V$, i.e. such that the following square commutes.
	\[
		\begin{tikzcd}[column sep = large]
			V
			\arrow{r}[above]{f}
			\arrow{d}
			&
			\hlie
			\arrow{d}
			\\
			\Tensor(V)
			\arrow[dashed]{r}[above]{\Phi}
			&
			\Univ(\hlie)
		\end{tikzcd}
	\]
	This homomorphism of algebras~$\Phi$ is in particular a homomorphism of Lie~algebras and therefore restricts to a homomorphism of Lie~algebras~$\varphi'$ from~$\flie$ to~$\hlie$, which makes the following square diagram commute.
	\[
		\begin{tikzcd}[column sep = large]
			V
			\arrow{r}[above]{f}
			\arrow{d}[left]{i}
			&
			\hlie
			\arrow{d}
			\\
			\flie
			\arrow[dashed]{r}[above]{\varphi'}
			&
			\Univ(\hlie)
		\end{tikzcd}
	\]
	The Lie~algebra~$\flie$ is generated by the set~$i(V)$, whence the image of~$\varphi'$ is generated as a Lie~algebra by~$\varphi'(i(V)) = f(V)$.
	But~$f(V)$ is contained in~$\hlie$.
	We hence find that the homomorphism of Lie~algebras~$\varphi'$ from~$\flie$ to~$\Univ(\hlie)$ restricts to a homomorphism of Lie~algebras~$\varphi$ from~$\flie$ to~$\hlie$, which makes the following square diagram commute.
	\[
		\begin{tikzcd}[column sep = large]
			V
			\arrow{r}[above]{f}
			\arrow{d}[left]{i}
			&
			\hlie
			\arrow[equal]{d}
			\\
			\flie
			\arrow[dashed]{r}[above]{\varphi}
			&
			\hlie
		\end{tikzcd}
	\]
	We have thus shown the existence of an required homomorphism~$\varphi$.

	The homomorphism~$\varphi$ is uniquely determined by the condition~$\varphi \circ i = f$ because the Lie~algebra~$\flie$ is generated by the image of~$i$.
\end{proof}


\begin{corollary}
	\leavevmode
	\begin{enumerate}
		\item
			For every~\vectorspace{$\kf$}~$V$ the free~\liealgebra{$\kf$} on~$V$ exists\index{free Lie algebra!existence}.
		\item
			For every set~$X$ the free~\liealgebra{$\kf$} on~$X$ exists.
	\end{enumerate}
\end{corollary}


\begin{proof}
	The first assertion follows from the explicit construction in \cref{construction of free lie algebras}.
	The second assertion follows from the first by part~\ref{free lie algebra on set via free lie algebra on vector space} of \cref{remarks about free lie algebras}.
\end{proof}


\begin{proposition}
	\label{subsets give free lie subalgebras}
	\leavevmode
	\begin{enumerate}
		\item
			Let~$V$ and~$W$ be two vector spaces and let~$i$ be an injective linear map from~$V$ to~$W$.
			Then the induced homomorphism of Lie~algebras~$\freelievect(i)$ from~$\freelievect(V)$ to~$\freelievect(W)$ is again injective.
		\item
			Let~$X$ and~$Y$ be two sets and let~$i$ be an injective map from~$X$ to~$Y$.
			Then the induced homomorphism of Lie~algebras~$\freelieset(i)$ from~$\freelieset(X)$ to~$\freelieset(Y)$ is again injective.
	\end{enumerate}
\end{proposition}


\begin{proof}
	Every monomorphism in~$\cVect{\kf}$ or~$\cSet$ is a split monomorphism, and is thus mapped to a split monomorphism, and therefore in particular to a monomorphism.
\end{proof}



\subsection{Description of Lie~Algebras via Generators and Relations}

\begin{fluff}
	By using free Lie~algebras and quotient Lie~algebras we can now both describe and construct Lie~algebras via generators and relations.
\end{fluff}


\begin{definition}
	Let~$X$ be a set and let~$R$ be a subset of the free Lie~algebra~$F(X)$.
	Let~$\ideal{ R }$ denote the Lie~ideal of~$F(X)$ generated by the set~$R$.
	The quotient Lie~algebra
	\[
		F(X) / \ideal{ R }
	\]
	is the Lie~algebra given by the set of \defemph{generators}~$X$\index{generators} and the set of \defemph{relations}~$R$\index{relations}.
\end{definition}


\begin{remark}
	We often write relations as equations, i.e. in the form~$s = t$.
	The corresponding element of~$R$ is given by then the difference~$s - t$.
\end{remark}


\begin{example}
	The Lie~algebra~$\sllie(2, \kf)$ can be described by the generators~$e$,~$h$,~$f$ and the relations
	\begin{equation}
		\label{reltaions on standard basis of sl2}
		[h,e] = 2e \,,
		\quad
		[h,f] = -2f \,,
		\quad
		[e,f] = h \,.
	\end{equation}
	More formally, the set of generators is given by~$X = \{ e, h, f \}$ and the set of relations is given by~$ R = \{ [h,e] - 2e, [hf] + 2f, [e,f] - h \} $.

	To prove this claim let~$\glie$ be the Lie~algebra given by generators~$e'$,~$h'$,~$f'$ and relations
	\begin{equation}
		\label{relations on generators of sl2}
		[h', e'] = 2 e' \,,
		\quad
		[h', f'] = -2 f' \,,
		\quad
		[e', f'] = h' \,.
	\end{equation}
	The standard basis~$e$,~$h$,~$f$ of~$\sllie(2, \kf)$ satisfies the relations~\eqref{reltaions on standard basis of sl2}.
	There hence exists a unique homomorphism of Lie~algebras~$\varphi$ from~$\glie$ to~$\sllie(2, \kf)$ given by
	\[
		\varphi(e') = e \,,
		\quad
		\varphi(h') = h \,,
		\quad
		\varphi(f') = f \,.
	\]
	There exists on the other hand a unique linear map~$\psi$ from~$\sllie(2, \kf)$ to~$\glie$ given by
	\[
		\psi(e) = e' \,,
		\quad
		\psi(h) = h' \,,
		\quad
		\psi(f) = f' \,.
	\]
	We see from the relations~\eqref{relations on generators of sl2} that this linear map~$\psi$ is a homomorphism of Lie~algebras.

	The composite~$\varphi \circ \psi$ is the identity on the basis elements~$e$,~$h$,~$f$ of$~\sllie(2, \kf)$.
	It is thus the identity on~$\sllie(2, \kf)$, whence~$\varphi \circ \psi = \id_{\sllie(2, \kf)}$.
	The composite~$\psi \circ \varphi$ is the identity on the Lie~algebra generators~$e'$,~$h'$,~$f'$ of~$\glie$.
	It is thus the identity on~$\glie$, whence~$\psi \circ \varphi = \id_{\glie}$.
	We have altogether shown that the two homomorphisms~$\varphi$ and~$\psi$ are mutually inverse isomorphisms.
\end{example}


\begin{remark}
	One can more generally describe for every natural number~$n$ with~$n \geq 1$ the Lie~algebra~$\sllie_{n+1}(\kf)$ by generators~$e_i$,~$f_i$,~$h_i$ with~$i = 1, \dotsc, n$ subject to the relations
	\begin{align*}
		[h_i, h_j] = 0  \,,
		\quad
		[h_i, e_j] = a_{ij} e_j \,,
		\quad
		[h_i, f_j] = -a_{ij}f_j  \,,
		\quad
		[e_i, f_j] = \delta_{ij} h_i
	\end{align*}
	for all~$i,j = 1, \dotsc, n$, together with the relations
	\[
		\ad(e_i)^{1-a_{ij}}(e_j) = 0
		\quad\text{and}\quad
		\ad(f_i)^{1-a_{ij}}(f_j) = 0
	\]
	for all~$i, j = 1, \dotsc, n$ with~$i \neq j$, where the numbers~$a_{ij}$ are for all~$i,j = 1, \dotsc, n$ given by
	\[
		a_{ij} =
		\begin{cases}
		\phantom{-}2 & \text{if $i = j$}  \,, \\
							-1 & \text{if $i = j \pm 1$}  \,, \\
		\phantom{-}0 & \text{otherwise} \,.
		\end{cases}
	\]
\end{remark}


\begin{remark}
	\label{existence of small colimits}
	\leavevmode
	\begin{enumerate}
		\item
			It follows from the existence of free Lie algebras that the category~$\cLie{\kf}$ admits all small coproducts.

			Indeed, suppose that~$\glie_\lambda$ with~$\lambda \in \Lambda$ is a family of~$\cLie{\kf}$ Lie algebras.
			Each Lie~algebra~$\glie_\lambda$ can be described by a set of generators~$X_\lambda$ and a set of relations~$R_\lambda$.
			We may assume that the sets~$X_\lambda$ with~$\lambda \in \Lambda$ are pairwise disjoint.
			We set~$X \coloneqq \bigdcup_{\lambda \in \Lambda} X_\lambda$ and denote for every~$\lambda \in \Lambda$ by~$i_\lambda$ the inclusion map from~$X_\lambda$ to~$X$.
			According to \cref{subsets give free lie subalgebras} we can regard each free Lie~algebra~$\freelieset(X_\lambda)$ as a Lie~subalgebra of the free Lie~algebra~$\freelieset(X)$ via the induced homomorphism of Lie~algebras~$\freelieset(i_\lambda)$.

			We set~$R \defined \bigdcup_{\lambda \in \Lambda} R_\lambda$ and let~$\glie$ be the Lie~algebra given by the generators~$X$ and relations~$R$, i.e. the quotient Lie~algebra~$\glie = \freelieset(X) / \ideal{ R }$.
			For each index~$\lambda$ in~$\Lambda$ the homomorphism of Lie~algebras~$\freelieset(i_\lambda)$ from~$\freelieset(X_\lambda)$ to~$\freelieset(X)$ descends to a homomorphism of Lie~algebra~$\iota_\lambda$ from~$\glie_\lambda$ to~$\glie$.

			It now follows from the universal properties of the free Lie~algebra and the universal property of the quotient Lie~algebra that the Lie~algebra~$\glie$ together the homomorphisms~$\iota_\lambda$ for~$\lambda \in \Lambda$ is a coproduct\index{coproduct of Lie algebras} of the Lie~algebras~$\glie_\lambda$ in the category~$\cLie{\kf}$.
		\item
			The category~$\cLie{\kf}$ also admits binary coequalizers.
			Indeed, let~$\glie$ and~$\hlie$ be two Lie~algebras and let~$\varphi$ and~$\psi$ be two homomorphisms of Lie~algebras from~$\glie$ to~$\hlie$.
			Let~$I$ be the Lie~ideal of~$\hlie$ generated by all differences~$\varphi(x) - \psi(x)$ with~$x \in \glie$.
			Then let~$\clie$ be the quotient Lie~algebra~$\hlie / I$ and let~$\varepsilon$ be the quotient homomorphism from~$\hlie$ to~$\clie$.
			The Lie~algebra~$\clie$ together with the homomorphism~$\varepsilon$ is a coequalizer\index{coequalizer of Lie algebras} of the two homomorphisms~$\varphi$ and$~\psi$ in the category~$\cLie{\kf}$.
		\item
			It follows from the existence of small coproducts and the existence of coequalizers that the category~$\cLie{\kf}$ admits all small colimits, i.e. that this category is cocomplete\index{cocomplete}.
	\end{enumerate}
\end{remark}





