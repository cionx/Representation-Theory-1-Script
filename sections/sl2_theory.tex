\section{Representations of \texorpdfstring{$\sllie_2(\kf)$}{sl2(k)}}


\begin{recall}
  The standard basis~$e$,~$h$,~$f$ of~$\sllie_2(\kf)$ is given by
  \[
    e
    =
    \begin{pmatrix}
      0 & 1 \\
      0 & 0
    \end{pmatrix} \,,
    \qquad
    h
    =
    \begin{pmatrix*}[r]
      1 &  0 \\
      0 & -1
      \end{pmatrix*}  \,,
    \qquad
    f
    =
    \begin{pmatrix}
      0 & 0 \\
      1 & 0
    \end{pmatrix}
  \]
  and this basis satisfies the relations
  \[
    [h,e] = 2e \,,
    \qquad
    [h,f] = -2 \spacing f \,,
    \qquad
    [e,f] = h \,.
  \]
\end{recall}


\begin{definition}
  Let~$V$ be a representation of~$\sllie_2(\kf)$.
  For any scalar~$\lambda \in \kf$ the linear subspace
  \[
    \gls*{weight space}
    \defined
    \{
      v \in V
    \suchthat
      h.v = \lambda v
    \}
  \]
  is the \defemph{weight space}\index{weight!space} of~$V$ with respect to~$\lambda$.
  An element~$\lambda \in \kf$ is a \defemph{weight}\index{weight} of~$V$ if~$V_\lambda \neq 0$.
  A nonzero element~$v \in V_\lambda$ is a \defemph{weight vector}\index{weight!vector}.
\end{definition}


\begin{convention}
  For~$\lambda \in \kf$ we write~$\lambda \geq 0$ to mean that~$\lambda$ is a non-negative integer.
\end{convention}


\begin{remark}
  One can define a partial order~$\leq$ on~$\kf$ by requiring that~$\lambda \leq \lambda + 1$ for all~$\lambda \in \kf$.
  The partial order~$\leq$ is then given by~$\lambda \leq \mu$ if and only if the difference~$\mu - \lambda$ is a non-negative integer.
  This then entails that~$\lambda \geq 0$ if and only if~$\lambda$ is a non-negative integer.
\end{remark}





\subsection{Highest Weight Representations}


\begin{lemma}[Shifting lemma]
  \label{shifting lemma}
  If~$V$ is any~{\representation{$\sllie_2(\kf)$}} then
  \[
    e.V_\lambda \subseteq V_{\lambda+2}
    \quad\text{and}\quad
    f.V_\lambda \subseteq V_{\lambda-2} \,.
  \]
\end{lemma}


\begin{proof}
  Let~$v \in V_\lambda$.
  Then
  \[
    h.(e.v)
    =
    e.h.v + [h,e].v
    =
    \lambda e.v + 2e.v
    =
    (\lambda + 2) e.v
  \]
  and thus~$e.v \in V_{\lambda + 2}$, and similarly
  \[
    h.(\spacing f.v)
    =
    h . \spacing f . v + [h,f].v
    =
    \lambda h.v - 2 \spacing f.v
    =
    (\lambda - 2) \spacing f.v
  \]
  and thus~$f.v \in V_{\lambda - 2}$.
\end{proof}


\begin{example}
  \label{natural representation of sl2}
  Let~$e_1$,~$e_2$ be the standard basis of~$V \defined \kf^2$, which is the natural representation of~$\sllie_2(\kf)$.
  Then~$h.e_1 = e_1$ and~$h.e_2 = -e_2$, so~$V_1 = \kf e_1$ and~$V_{-1} = \kf e_2$ with~$V = V_{-1} \oplus V_1$.
  That~$e.V_{-1} = V_1$ and~$f.V_1 = V_{-1}$ can then by seen by a glance at the matrices
  \[
    e
    =
    \begin{pmatrix}
      0 & 1 \\
      0 & 0
    \end{pmatrix}
    \qquad\text{and}\qquad
    f
    =
    \begin{pmatrix}
      0 & 0 \\
      1 & 0
      \end{pmatrix} \,.
  \]
\end{example}


\begin{definition}
  Let~$V$ be an~{\representation{$\sllie_2(\kf)$}}.
  \begin{enumerate}
    \item
      A vector~$v \in V$ is \defemph{primitive}\index{primitive vector} if it is a weight vector (and hence in particular nonzero) with~$e.v = 0$.
    \item
      The representation~$V$ is of \defemph{highest weight~$\lambda$} if~$V$ is generated by a primitive vector of weight~$\lambda$.
  \end{enumerate}
\end{definition}


\begin{theorem}[Structure of highest weight~{\representations{$\sllie_2(\kf)$}}]
  \label{classification of highest weight for sl2}
  Let~$V$ be a representation of~$\sllie_2(\kf)$ of highest weight~$\lambda$, generated by a primitive vector~$v$ of weight~$\lambda$.
  Let~$v_i \defined f^i.v$ for every~$i \geq 0$.
  \begin{enumerate}
    \item
      The representation~$V$ is spanned by the vectors~$v_i$ with~$i \geq 0$ as a vector space, and the actions of~$e$,~$h$ and~$f$ on these vector space generators is given by
      \[
        e.v_{i+1}
        =
        (i+1)(\lambda-i) v_i  \,,
        \quad
        h.v_i
        =
        (\lambda-2i) v_i  \,,
        \quad
        f.v_i
        =
        v_{i+1} \,.
      \]
    \item
      \label{highest weight nondominant}
      If~$\lambda \ngeq 0$ then~$V$ is irreducible with basis~$v_i$,~$i \geq 0$.
    \item
      For~$\lambda \geq 0$ then there are two cases:
      \begin{enumerate}
        \item
          If~$\lambda \geq 0$ and~$V$ is finite dimensional then~$V$ is irreducible with basis~$v_0, \dotsc, v_{\lambda}$.
          The representation~$V$ is in particular~{\dimensional{$(\lambda+1)$}}.
        \item
          If~$\lambda \geq 0$ and~$V$ is infinite dimensional then~$V$ has the vectors~$v_i \geq 0$ with~$i \geq 0$ as a basis.
          Then~$u \defined v_{\lambda+1}$ is a primitive vector of weight~$-\lambda-2$.
          The subrepresentation~$U$ of~$V$ generated by~$u$ is the unique nonzero proper subrepresentation of~$V$.
          The subrepresentation~$U$ is infinite dimensional and irreducible of highest weight~$-\lambda-2$ whereas the quotient~$V/U$ is finite dimensional and irreducible of highest weight~$\lambda$.
      \end{enumerate}
    \item
      If~$V$ is infinite dimensional then the decomposition~$V = \bigoplus_{i \geq 0} \kf v_i$ gives the weight space decomposition~$V = \bigoplus_{i \geq 0} V_{\lambda-2i}$.
      The actions of~$e$ and~$f$ on the basis~vectors~$v_i$ are given as indicated in the following diagram, where the dashed arrows represent the action of~$f$.
      \begin{equation}
        \label{picture of infinite dimensional verma irrep}
        \begin{tikzcd}[column sep = 2em]
          v_0
          \arrow[dashed, bend left = 50]{r}[above]{1}
          &
          v_1
          \arrow[dashed, bend left = 50]{r}[above]{1}
          \arrow[bend left = 50]{l}[below]{\lambda}
          &
          v_2
          \arrow[dashed, bend left = 50]{r}[above]{1}
          \arrow[bend left = 50]{l}[below]{2(\lambda-1)}
          &
          v_3
          \arrow[dashed, bend left = 50]{r}[above]{1}
          \arrow[bend left = 50]{l}[below]{3(\lambda-2)}
          &
          v_4
          \arrow[dashed, bend left = 50]{r}
          \arrow[bend left = 50]{l}[below]{4(\lambda-3)}
          &
          \dotsb
          \arrow[bend left = 50]{l}
        \end{tikzcd}
      \end{equation}
      If~$V$ is finite dimensional --- and thus~$\lambda \geq 0$ --- then the decompositon~$V = \kf v_0 \oplus \dotsb \oplus \kf v_\lambda$ gives the weight space decomposition~$V = V_{\lambda} \oplus V_{\lambda-2} \oplus \dotsb \oplus V_{-\lambda+2} \oplus V_{-\lambda}$.
      On the basis~vectors~$v_0, \dotsc, v_\lambda$ the actions of~$e$ and~$f$ are given as follows:
      \begin{equation}
        \label{picture of finite dimensional verma irrep}
        \begin{tikzcd}[column sep = 2em]
          v_0
          \arrow[dashed, bend left = 50]{r}[above]{1}
          &
          v_1
          \arrow[dashed, bend left = 50]{r}[above]{1}
          \arrow[bend left = 50]{l}[below]{\lambda}
          &
          v_2
          \arrow[dashed, bend left = 50]{r}[above]{1}
          \arrow[bend left = 50]{l}[below]{2(\lambda-1)}
          &
          v_3
          \arrow[dashed, bend left = 50]{r}
          \arrow[bend left = 50]{l}[below]{3(\lambda-2)}
          &
          \dotsb
          \arrow[dashed, bend left = 50]{r}
          \arrow[dashed, bend left = 50]{l}
          &
          v_{\lambda-3}
          \arrow[dashed, bend left = 50]{r}[above]{1}
          \arrow[bend left = 50]{l}
          &
          v_{\lambda-2}
          \arrow[dashed, bend left = 50]{r}[above]{1}
          \arrow[bend left = 50]{l}[below]{(\lambda-2)3}
          &
          v_{\lambda-1}
          \arrow[dashed, bend left = 50]{r}[above]{1}
          \arrow[bend left = 50]{l}[below]{(\lambda-1)2}
          &
          v_{\lambda}
          \arrow[bend left = 50]{l}[below]{\lambda}
        \end{tikzcd}
      \end{equation}
      By normalizing vectors the basis~$v_0, \dotsc, v_\lambda$ we get a basis~$w_{\lambda}, w_{\lambda-2}, \dotsc, w_{-\lambda+2}, w_{-\lambda}$ of~$V$ consisting of weight vectors~$w_\mu \in V_{\mu}$ on which the action of~$e$ and~$f$ is given as follows:
      \begin{equation}
        \label{picture of finite dimensional verma irrep normalized}
        \begin{tikzcd}[column sep = 1.4em]
          w_{\lambda}
          \arrow[dashed, bend left = 50]{r}[above]{1}
          &
          w_{\lambda-2}
          \arrow[dashed, bend left = 50]{r}[above]{2}
          \arrow[bend left = 50]{l}[below]{\lambda}
          &
          w_{\lambda-4}
          \arrow[dashed, bend left = 50]{r}[above]{3}
          \arrow[bend left = 50]{l}[below]{\lambda-1}
          &
          w_{\lambda-6}
          \arrow[dashed, bend left = 50]{r}
          \arrow[bend left = 50]{l}[below]{\lambda-2}
          &
          \dotsb
          \arrow[dashed, bend left = 50]{r}
          \arrow[dashed, bend left = 50]{l}
          &
          w_{-\lambda+6}
          \arrow[dashed, bend left = 50]{r}[above]{\lambda-2}
          \arrow[bend left = 50]{l}
          &
          w_{-\lambda+4}
          \arrow[dashed, bend left = 50]{r}[above]{\lambda-1}
          \arrow[bend left = 50]{l}[below]{3}
          &
          w_{-\lambda+2}
          \arrow[dashed, bend left = 50]{r}[above]{\lambda}
          \arrow[bend left = 50]{l}[below]{2}
          &
          w_{-\lambda}
          \arrow[bend left = 50]{l}[below]{1}
        \end{tikzcd}
      \end{equation}
  \end{enumerate}
\end{theorem}


\begin{proof}
  \leavevmode
  \begin{enumerate}
    \item
      That~$v_i \in V_{\lambda - 2i}$ follows by induction from~\cref{shifting lemma}.
      That~$f.v_i = v_{i+1}$ holds by definition.
      That~$e.v_{i+1} = (i+1)(\lambda-i) v_i$ follows from~$e.v_0 = e.v = 0$ by induction over~$i$:
      We find for~$i = 0$ that
      \[
        e.v_1
        =
        e . \spacing f . v_0
        =
        [e,f].v_0 + f.e.v_0
        =
        h.v_0 + f.e.v_0
        =
        \lambda v_0
      \]
      because~$v_0 = v$ is primitive of weight~$\lambda$.
      For the induction step we find that
      \begin{align*}
        e.v_{i+1}
        &=
        e . \spacing f . v_i
        \\
        &=
        [e,f].v_i + f.e.v_i
        \\
        &=
        h.v_i + f.e.v_i
        \\
        &=
        (\lambda - 2i)v_i + i (\lambda-i+1) \spacing f.v_{i-1}
        \\
        &=
        (\lambda - 2i)v_i + i (\lambda-i+1) v_i
        \\
        &=
        (\lambda - 2i + \lambda i - i^2 + i) v_i
        \\
        &=
        (\lambda i - i^2 + \lambda - i) v_i
        \\
        &=
        (i+1)(\lambda-i) v_i
      \end{align*}
      as claimed.
    \item
      The linear subspace of~$V$ spanned by all~$v_i$,~$i \geq 0$ is closed under the actions of~$h$ and~$f$, and by the formula~$e.v_{i+1} = (i+1)(\lambda-i) v_i$ also under the action of~$e$ (because~$e.v_0 = e.v = 0$).
      This linear span contains the generator~$v_0 = v$ and hence equals~$V$.
    \item
      It follows from our assumption~$\lambda \ngeq 0$ that~$(i+1)(\lambda-i) \neq 0$ for every~$i \geq 0$.
      We hence find for every~$i \geq 0$ from the formula~$e.v_{i+1} = (i+1)(\lambda-i) v_i$ that~$v_{i+1} \neq 0$ if~$v_i \neq 0$.
      It therefore follows inductively from~$v_0 = v \neq 0$ that~$v_{i+1} \neq 0$ for every~$i \geq 0$.
      The vector space generators~$v_i$ with~$i \geq 0$ are thus nonzero.
      They are already linearly independent because they are of pairwise different weights.
      Hence the vector space generating set~$v_i$,~$i \geq 0$ is a basis of~$V$.
      
      For the irreducibility of~$V$ we show that any nonzero vector~$u \in V$ generates~$V$ as a representation.
      Let~$U$ be the subrepresentation of~$V$ generated by~$u$.
      We have seen that~$e$ maps for every~$i \geq 0$ the basis~vector~$v_{i+1}$ to a nonzero multiple of the previous basis~vector~$v_i$ and annihilates the first basis~vector~$v_0$.
      We hence find that~$e^n.u$ is a nonzero multiple of~$e_0$ for a suitable power~$n$ (if~$u = \sum_{i \geq 0} a_i v_i$ with~$a_i \in \kf$ then let~$n$ be maximal with~$a_n \neq 0$).
      This shows~$v_0 \in U$ and whence~$v_i = f^i.v \in U$ for every~$i \geq 0$.
      Thus~$U = V$, which means that~$u$ generates~$V$.
    \item
      We find from the relation~$e.v_{i+1} = (i+1)(\lambda-i)$ for~$i \geq 0$ that~$e$ maps for every~$i = 1, \dotsc, \lambda$ the vector~$v_i$ to a nonzero multiple of the previous vector~$v_{i-1}$.
      But we also find that~$e.v_{\lambda+1} = 0$.
      We now distinguish between the cases~$v_{\lambda+1} = 0$ and~$v_{\lambda+1} \neq 0$.
      \begin{enumerate}
        \item
          If~$v_{\lambda+1} = 0$ then~$v_i = 0$ for all~$i \geq \lambda+1$.
          Then~$V$ is generated by the vectors~$v_0, v_1, \dotsc, v_\lambda$.
          We find as in the case~$\lambda \ngeq 0$ that~$v_0, \dotsc, v_n$ are actually linearly indepedent and thus a basis.
          Hence~$V$ is of finite dimension~$\lambda+1$.
          That~$V$ is irreducible can be seen as in the case~$\lambda \ngeq 0$.
        \item
          If~$v_{\lambda+1} \neq 0$ then~$u \defined v_{\lambda+1}$ is a weight vector for the weight~$\lambda-2(\lambda+1)$ with~$e.u = 0$.
          Hence~$u$ is a primitive vector for the weight~$-\lambda-2$.
          The subrepresentation~$U$ generated by~$u$ is infinite dimensional and irreducible by part~\ref*{highest weight nondominant} because~$-\lambda-2 \ngeq 0$.
          The representation~$V/U$ admits the residue class~$\class{v}$ as a primitive generator of weight~$\lambda$ and~$V/U$ is finite dimensional, namely spanned by the residue classes~$\class{v_0}, \dotsc, \class{v_\lambda}$.
          Hence~$V/U$ is irreducible of highest weight~$\lambda$ and dimension~$\lambda+1$ by the previous case.
          
          It remains to show that~$U$ is the unique nonzero proper subrepresentation of~$U$.
          We first observe that every nonzero subrepresentation of~$V$ contains~$U$:
          For every nonzero~$w \in V$ we have~$f^n.w \in U$ for~$n$ sufficiently large (namely~$n \geq \lambda+1$) and then~$f^n.w$ generates~$U$ because~$U$ is irreducible.
          Hence~$U$ is contained in any nonzero cyclic subrepresentation, and therefore in every nonzero subrepresentation.
          The subrepresentations of~$V$ that contain~$U$ are in {\onetoone} correspondence to the subrepresentations of~$V/U$.
          The representation~$V/U$ is irreducible and hence admits precisely two subrepresentations, namely~$0$ and~$V/U$.
          Thus~$V$ admits sprecisely two subrepresentations that contain~$U$, namely~$U$ and~$V$.
          This shows altogether that~$0$,~$U$ and~$V$ are the only subrepresentations of~$V$.
      \end{enumerate}
    \item
      It remains to show how to normalize in the finite dimensional case the basis~$v_0, \dotsc, v_\lambda$.
      By normalizing the basis~vectors~$v_0, \dotsc, v_\lambda$ as~$\tilde{w}_i \defined v_i/i!$ we get a basis~$\tilde{w}_0, \dotsc, \tilde{w}_\lambda$ of~$V$ on which the actions of~$e$ and~$f$ are as follows:
      \[
        \begin{tikzcd}[column sep = 1.5em]
          \tilde{w}_0
          \arrow[dashed, bend left = 50]{r}[above]{1}
          &
          \tilde{w}_1
          \arrow[dashed, bend left = 50]{r}[above]{2}
          \arrow[bend left = 50]{l}[below]{\lambda}
          &
          \tilde{w}_2
          \arrow[dashed, bend left = 50]{r}[above]{3}
          \arrow[bend left = 50]{l}[below]{\lambda-1}
          &
          \tilde{w}_3
          \arrow[dashed, bend left = 50]{r}
          \arrow[bend left = 50]{l}[below]{\lambda-2}
          &
          \dotsb
          \arrow[dashed, bend left = 50]{r}
          \arrow[dashed, bend left = 50]{l}
          &
          \tilde{w}_{\lambda-3}
          \arrow[dashed, bend left = 50]{r}[above]{\lambda-2}
          \arrow[bend left = 50]{l}
          &
          \tilde{w}_{\lambda-2}
          \arrow[dashed, bend left = 50]{r}[above]{\lambda-1}
          \arrow[bend left = 50]{l}[below]{3}
          &
          \tilde{w}_{\lambda-1}
          \arrow[dashed, bend left = 50]{r}[above]{\lambda}
          \arrow[bend left = 50]{l}[below]{2}
          &
          \tilde{w}_\lambda
          \arrow[bend left = 50]{l}[below]{1}
        \end{tikzcd}
      \]
      We now arrive at the desired basis~$v_{\lambda}, v_{\lambda-2}, \dotsc, v_{-\lambda+2}, v_{-\lambda}$ of~$V$ by relabeling the given basis~vectors~$\tilde{w}_0, \dotsc, \tilde{w}_\lambda$.
    \qedhere
  \end{enumerate}
\end{proof}


\begin{lemma}[Existence of highest weight representations of~$\sllie_2(\kf)$]
  \label{existence of highest weight for sl2}
  For every scalar~$\lambda \in \kf$ there exists an infinite dimensional~{\representation{$\sllie_2(\kf)$}} of highest weight~$\lambda$.
\end{lemma}


\begin{proof}[First proof]
  Let~$V$ be a vector space with basis~$v_i$ where~$i \geq 0$.
  We set~$v_{-1} \defined 0$ and define an action of~$e$,~$h$ and~$f$ on~$V$ via
  \[
    e.v_i
    =
    i(\lambda-i+1) v_{i-1}  \,,
    \qquad
    h.v_i
    =
    (\lambda-2i) v_i  \,,
    \qquad
    f.v_i
    =
    v_{i+1}
  \]
  for all~$i \geq 0$.
  We observe that the identity~$h.v_i = (\lambda-2i) v_i$ also holds for~$i = -1$.
  This gives~$V$ the structure of a~{\representation{$\sllie_2(\kf)$}}:
  We have for all~$i \geq 0$ that
  \begin{align*}
    h.e.v_i - e.h.v_i
    &=
    i(\lambda-i+1)h.v_{i-1} - (\lambda-2i)e.v_i
    \\
    &=
    i(\lambda-i+1)(\lambda-2i+2) v_{i-1} - i(\lambda-i+1)(\lambda-2i) v_{i-1}
    \\
    &=
    2i(\lambda-i+1) v_{i-1}
    \\
    &=
    2e.v_i
    \\
    &=
    [h,e].v_i \,.
  \end{align*}
  We also have for all~$i \geq 0$ that
  \begin{align*}
    h . \spacing f . v_i - f.h.v_i
    &=
    h.v_{i+1} - (\lambda-2i) \spacing f.v_i
    \\
    &=
    (\lambda-2i-2) v_{i+1} - (\lambda-2i) v_{i+1}
    \\
    &=
    -2 v_{i+1}
    \\
    &=
    -2 \spacing f.v_i
    \\
    &=
    [h,f].v_i \,.
  \end{align*}
  Lastly we have to check the relation~$e . \spacing f . v_i - f.e.v = h.v_i$ for all~$i \geq 0$.
  For~$i = 0$ we find
  \[
    e . \spacing f . v_0 - f.e.v_0
    =
    e.v_1 - f.0
    =
    \lambda v_0
    =
    h.v_0
  \]
  and for~$i \geq 1$ we see that
  \begin{align*}
    e . \spacing f . v_i - f.e.v_i
    &=
    e.v_{i+1} - i(\lambda-i+1) \spacing f.v_{i-1}
    \\
    &=
    (i+1)(\lambda-i) v_i - i(\lambda-i+1) v_i
    \\
    &=
    \biggl( i(\lambda-i) + (\lambda-i) - i(\lambda-i) - i \biggr) v_i
    \\
    &=
    (\lambda-2i) v_i
    \\
    &=
    h.v_i \,.
  \end{align*}
  Altogether this shows that we have defined on~$V$ the structure of a representation of~$\sllie_2(\kf)$.
  The vector~$v_0$ is primitive of weight~$\lambda$ and generates the representation~$V$.
  This means that~$V$ is a representation of~$\sllie_2(\kf)$ of highest weight~$\lambda$.
\end{proof}


\begin{proof}[Second proof]
  Let~$\blie$ be the Lie~subalgebra of~$\sllie_2(\kf)$ consisting of traceless upper triangular matrices.
  The standard basis~$e$,~$h$,~$f$ of~$\sllie_2(\kf)$ restricts to a basis~$e$,~$h$ of~$\blie$.
  We can make the {\onedimensional} vector space~$\kf$ into a representation of~$\sllie_2(\kf)$ via~$e.x = 0$ and~$h.x = \lambda x$ for all~$x \in \kf$.
  We denote this representation by~$\kf_{\, \lambda}$.
  We may regard~$\kf_{\, \lambda}$ as a~{\module{$\Univ(\blie)$}} and then use extensions of scalars to form the~\module{$\Univ(\sllie_2(\kf))$}
  \[
    V
    \defined
    \Univ(\sllie_2(\kf)) \tensor_{\Univ(\blie)} \kf_{\, \lambda} \,.
  \]
  It follows from the PBW~theorem that the~{\algebra{$\kf$}}~$\Univ(\sllie_2(\kf))$ is free with basis~$f^i$,~$i \geq 0$ as a right~{\module{$\Univ(\blie)$}}.
  Hence~$V$ has as a~{\basis{$\kf$}} the elements~$f^i \tensor 1$ with~$i \geq 0$.
  The nonzero vector~$v \defined 1 \tensor 1 \in V$ is primitive with weight~$\lambda$ because
  \begin{gather*}
    e.v
    =
    e.(1 \tensor 1)
    =
    e \tensor 1
    =
    1 \tensor (e.1)
    =
    1 \tensor 0
    =
    0
  \shortintertext{and}
    h.v
    =
    h.(1 \tensor 1)
    =
    h \tensor 1
    =
    1 \tensor (h.1)
    =
    1 \tensor \lambda
    =
    \lambda (1 \tensor 1)
    =
    \lambda v \,.
  \end{gather*}
  The element~$1 \in \kf_{\, \lambda}$ generates~$\kf_{\, \lambda}$ as a left~{\module{$\Univ(\blie)$}} whence the induced element~$1 \tensor 1 \in V$ generates~$V$ as a left~{\module{$\Univ(\sllie_2(\kf)$}}.
  Hence~$V$ is an infinite dimensional~{\representation{$\sllie_2(\kf)$}} of highest weight~$\lambda$.
\end{proof}


\begin{proposition}[Classification of irreducible highest weight representations of~$\sllie_2(\kf)$]
  For every~$\lambda \in \kf$ there exists up to isomorphism a unique irreducible~{\representation{$\sllie_2(\kf)$}}~\gls*{sl2 irrep} of highest weight~$\lambda$.
  \begin{enumerate}
    \item
      If~$\lambda \ngeq 0$ then~$\irr(\lambda)$ is infinite dimensional and admits a basis~$v_i$,~$i \geq 0$ of weight vectors~$v_i \in V_{\lambda-2i}$ on which~$e$ and~$f$ act as in~\eqref{picture of infinite dimensional verma irrep}.
    \item
      If~$\lambda \geq 0$ then~$\irr(\lambda)$ is of finite dimension~$\lambda+1$.
      The representation~$V$ admits a basis~$v_0, \dotsc, v_n$ of weight vectors~$v_i \in V_{\lambda-2i}$ on which~$e$ and~$f$ act as in~\eqref{picture of finite dimensional verma irrep};
      the representation~$V$ also admits a basis~$v_{\lambda}, v_{\lambda-2}, \dotsc, v_{-\lambda+2}, v_{-\lambda}$ of weight vectors~$w_\mu \in V_\mu$ on which~$e$ and~$f$ act as in~\eqref{picture of finite dimensional verma irrep normalized}.
  \end{enumerate}
\end{proposition}


\begin{proof}
  Let~$V$ be an infinite dimensional~{\representation{$\sllie_2(\kf)$}} of highest weight~$\lambda$;
  the existence of~$V$ is ensured by \cref{existence of highest weight for sl2}.
  If~$\lambda \ngeq 0$ then~$V$ is irreducible and of the desired form by \cref{classification of highest weight for sl2}.
  If~$\lambda \geq 0$ then~$V$ admits a unique nonzero proper subrepresentation~$U$ by \cref{classification of highest weight for sl2}.
  The representation~$V/U$ is then irreducible, of highest weight~$\lambda$ and of the desired form, again by \cref{classification of highest weight for sl2}.
  The uniqueness up to isomorphism follows from the explicit description of the actions of~$e$,~$h$ and~$f$ on the bases~$v_i$,~$i \geq 0$ (resp.~$i = 0, \dotsc, \lambda$).
\end{proof}


\begin{lemma}
  \label{fd contains a weight sl2}
  Every finite dimensional~{\representation{$\sllie_2(\kf)$}}~$V$ contains a primitive vector (for some weight).
\end{lemma}


\begin{proof}
  The representation~$V$ admits some weight~$\lambda \in \kf$ because~$V$ is finite dimensional and~$\kf$ is algebraically closed.
  For some occuring weight~$\lambda$ we have~$V_{\lambda+2} = 0$ because~$V$ is finite dimensional.
  Then any nonzero~$v \in V_\lambda$ does the job.
\end{proof}


\begin{corollary}[Classification of irreducible finite dimensional representations of~$\sllie_2(\kf)$]
  The representations~$\irr(\lambda)$ with~$\lambda \geq 0$ form a set of representations for the isomorphism classes of finite dimensional irreducible~$\sllie_2(\kf)$ representations.
\end{corollary}


\begin{proof}
  It remains to show that every finite dimensional~{\representation{$\sllie_2(\kf)$}}~$V$ is isomorphic to~$\irr(\lambda)$ for some~$\lambda \geq 0$.
  
  There exists a primitive vector~$v \in V$ for some weight~$\lambda \in \kf$ by \cref{fd contains a weight sl2}.
  The primitive vector~$v$ generates~$V$ as a representation because~$V$ is irreducible, so~$V$ is of highest weight~$\lambda$.
  It follows from \cref{classification of highest weight for sl2} that~$V \cong \irr(\lambda)$ with~$\dim V = \lambda + 1$ because otherwise~$V$ would need to be infinite dimensional. 
\end{proof}

% 
% 
% 
% 
% \begin{theorem}[Classification of finite dimensional irreducible representations of~$\sllie_2(\kf)$]
%   \label{classification of fd sl2 irreps}
%     For every~$\lambda \geq 0$ there exists up to isomorphism a unique~{\dimensional{$(\lambda+1)$}} irreducible representation of~$\sllie_2(\kf)$.
%     More explicitely:
%   \begin{enumerate}
%     \item
%       For every~$\lambda \geq 0$ there exists an~{\dimensional{$(\lambda+1)$}} irreducible representation of~$\sllie_2(\kf)$.
%     \item
%       Let~$V$ be an~{\dimensional{$(\lambda+1)$}} irreducible representation of~$\sllie_2(\kf)$ for some~$\lambda \geq 0$.
%       Then
%       \[
%         V
%         =
%         V_{-\lambda} \oplus V_{-\lambda+2} \oplus \dotsb \oplus V_{\lambda-2} \oplus V_\lambda
%       \]
%       and each occuring weight space~$V_\mu$ is {\onedimensional}.
%       In particular all occuring weights are integral and the highest occuring weight is~$\lambda$.
% 
%       Even more explicitely, there exists a basis~$v_{-\lambda}, v_{-\lambda+2}, \dotsc, v_{\lambda-2}, v_\lambda$ of~$V$ such that~$v_\mu \in V_\mu$ for every~$\mu = -\lambda, -\lambda+2, \dotsc, \lambda-2, \lambda$, with respect to which the actions of~$e$ and~$f$ on~$V$ are given as in the following diagram, where the dashed arrows represent the action of $f$:
%       \begin{equation}
%         \begin{tikzcd}[column sep = 1.5em]
%           v_{-\lambda}
%           \arrow[bend left]{r}[above]{1}
%           &
%           v_{-\lambda+2}
%           \arrow[bend left]{r}[above]{2}
%           \arrow[bend left, dashed]{l}[below]{\lambda-1}
%           &
%           v_{-\lambda+4}
%           \arrow[bend left]{r}[above]{3}
%           \arrow[bend left, dashed]{l}[below]{\lambda-2}
%           &
%           v_{-\lambda+6}
%           \arrow[bend left]{r}
%           \arrow[bend left, dashed]{l}[below]{\lambda-3}
%           &
%           \dots
%           \arrow[bend left]{r}
%           \arrow[bend left, dashed]{l}
%           &
%           v_{\lambda-6}
%           \arrow[bend left]{r}[above]{\lambda-3}
%           \arrow[bend left, dashed]{l}
%           &
%           v_{\lambda-4}
%           \arrow[bend left]{r}[above]{\lambda-2}
%           \arrow[bend left, dashed]{l}[below]{3}
%           &
%           v_{\lambda-2}
%           \arrow[bend left]{r}[above]{\lambda-1}
%           \arrow[bend left, dashed]{l}[below]{2}
%           &
%           v_\lambda
%           \arrow[bend left, dashed]{l}[below]{1}
%         \end{tikzcd}
%       \end{equation}
%       The actions of~$e$ and~$\spacing f$ on the endpoints is given by~$e.v_\lambda = 0$ and~$f.v_{-\lambda} = 0$.
%  \end{enumerate}
% \end{theorem}
% 
% 
% \begin{proof}
%   \leavevmode
%   \begin{enumerate}
%     \item
%       Let~$\sllie_2(\kf)$ act on the polynomial ring~$V \defined \kf[x,y]$ via
%       \[
%         \sllie_2(\kf)
%         \to
%         \gllie(V) \,,
%         \qquad
%         e
%         \mapsto
%         y\dd{x} \,,
%         \quad
%         h
%         \mapsto
%         y\dd{y} - x\dd{x}
%         \,,
%         \quad
%         f
%         \mapsto
%         x\dd{y} \,.
%       \]
%       It was already shown in~\cref{examples for representations} that this defines a representation of~$\sllie_2(\kf)$.
%       Let~$V^{(n)}$ be the linear subspace of~$V$ consisting of the homogeneous polynomials of degree~$n$, i.e.\
%       \[
%         V^{(n)}
%         =
%         \gen*{ x^n, x^{n-1} y, \dotsc, x y^{n-1}, y^n }_{\kf} \,.
%       \]
%       Then~$V^{(n)}$ is an~{\dimensional{$(n+1)$}} subrepresentation of~$V$.
%       The action of~$e$ and~$f$ on~$V^{(n)}$ is on these basis~vectors given by
%       \[
%         e . x^i y^{n-i}
%         =
%         k x^{i-1} y^{n-i+1} \,,
%         \qquad
%         f. x^i y^{n-i}
%         =
%         (n-i) x^{i+1} y^{n-i-1} \,.
%       \]
%       So up to coefficient the actions of~$e$ and~$f$ move around the 
%       
%       Let~$U$ be a nonzero subrepresentation of~$V^{(n)}$.
%       If~$p \in U$ is a nonzero polynomial then by applying~$f$ often enough it follows that~$x^n \in U$, from which it follows from applying~$e$ often enough that~$x^{n-i} y^i \in U$ for all~$i = 0, \dotsc, n$.
%       Hence~$U = V$.
%       We have shown that~$V^{(n)}$ is an irreducible representation of~$\sllie_2(\kf)$.
%     
%     \item
%       Because~$V$ is finite dimensional and nonzero and~$\kf$ is algebraically closed there exists some~$\lambda \in \kf$ with~$V_\lambda \neq 0$.
%       The eigenvalue~$\lambda$ can be choosen such that~$V_{\lambda+2} = 0$ because~$V$ is finite dimensional.
%       Let~$v \in V_\lambda$ be a nonzero weight vector and let
%       \[
%         w_i
%         \defined
%         f^i.v
%       \]
%       for all~$i \geq 0$.
%       We find from \cref{shifting lemma} that~$e.w_{i+1} = (i+1)(\lambda-i)w_i$ for every~$i \geq 0$.
%       
%       There exists by the finite dimensionality of~$V$ some maximal~$m$ for which the vetors~$w_0, \dotsc, w_m$ are nonzero but~$w_{m+1} = 0$.
%       The nonzero elements~$w_0, \dotsc, w_m$ are linearly independent as they live in different weight spaces.
%       The linear space~$\gen{w_0, \dotsc, w_m}$ is a subrepresentation of~$V$:
%       It is closed under the action of~$h$ because the~$w_i$ are weight vectors;
%       it is closed under the action of~$f$ because~$f.w_i = w_{i+1}$ for~$i = 0, \dotsc, m-1$ and~$f.w_m = 0$;
%       and it is closed under the action of~$e$ by \cref{shifting lemma} because~$e.w_0 = 0$.
%       It follows from the irreducibility of~$V$ that~$V = \gen{w_0, \dotsc, w_m}$ so that~$w_0, \dotsc, w_m$ is a basis of~$V$.
%       
%       It follows in particular that~$m = n$.
%       We have~$0 = e.w_{n+1} = (n+1)(\lambda-n)w_n$ with~$w_n \neq 0$ and hence~$\lambda = n$.
%       We have thus constructed a basis~$w_0, \dotsc, w_m$ of~$V$ with~$w_i \in V_{\lambda-i}$ and on which the actions of~$e$ and~$f$ are given as indicated by the following diagram, where the dashed arrows represent the action of~$f$:
%       
%       
%       
%     \qedhere
%   \end{enumerate}
% \end{proof}


\begin{examples}
  \leavevmode
  \begin{enumerate}
    \item
      The irreducible representation~$\irr(0)$ can be realized by~$\sllie_2(\kf)$ acting trivially on~$\kf$ (or on any {\onedimensional} vector space for that matter).
    \item
      The irreducible representation~$\irr(1)$ can be realized as the natural representation of~$\sllie_2(\kf)$ as seen in \cref{natural representation of sl2}.
    \item
      Let~$V = \sllie_2(\kf)$ be the adjoint representation of~$\sllie_2(\kf)$.
      The representation~$V$ is irreducible because the Lie~algebra~$\sllie_2(\kf)$ is simple, and it is {\threedimensional}.
      Whence~$V \cong \irr(2)$ and hence~$V = V_{-2} \oplus V_0 \oplus V_2$.
      The occuring weight spaces are given by~$V_{-2} = \gen{\spacing f}_{\kf}$,~$V_0 = \gen{h}_{\kf}$ and~$V_2 = \gen{e}_{\kf}$.
    \item
      For every~$\lambda \geq 0$ the Lie~algebra~$\sllie_2(\kf)$ acts on the polynomial ring~$\kf[x,y]$ as discussed in \cref{examples for representations}.
      The homogeneous polynomials of degree~$\lambda$ form an irreducible subrepresentation of~$\kf[x,y]$ of dimension~$\lambda + 1$, which gives another construction of~$\irr(\lambda)$.
  \end{enumerate}
\end{examples}





\subsection{Finite Dimensional Representations of~\texorpdfstring{$\sllie_2(\kf)$}{sl2(k)}}



\subsubsection{Classification of Finite Dimensional Representations}


\begin{fluff}
  We know from Weyl’s theorem that every finite dimensional~{\representation{$\sllie_2(\kf)$}}~$V$ decomposes into a direct sum of irreducible representations.
  The occuring irreducible direct summands are again finite dimesional and hence of the form~$\irr(\lambda)$ for some~$\lambda \geq 0$.
  We can determine which irreducible representations~$\irr(\lambda)$ occur in~$V$ and by which multiplicity, as long as we know the dimensions of the weight spaces of~$V$:
  
  How the total dimension of~$V$ it distribuited among its weight spaces can be represented by a \defemph{weight diagram}\index{weight diagram}\footnote{The author takes full blame for this definition.} of the following form:
  \[
    \sldiagram{0,2,1,3,1,2,0}
  \]
  In this example both~$\dim V_2$ and~$\dim V_{-2}$ are {\twodimensional},~$V_1$ and~$V_{-1}$ are {\onedimensional} and~$V_0$ is {\threedimensional}.
  All other weight spaces are zero.
  The representation~$V$ has total dimension~$9$.
  
  The irreducible representations~$\irr(\lambda)$ with~$\lambda \geq 0$ have particularly nice weight diagrams:
  We have a single dot in the columns~$-\lambda, -\lambda+2, \dotsc, \lambda-2, \lambda$ and no dots in the other columns.
  The weight diagrams of~$\irr(\lambda)$ for small values of~$\lambda$ are given as follows:
  \begingroup
  \setlength{\jot}{5ex}
  \begin{gather*}
    \sldiagram{0,0,0,0,1,0,0,0,0}
    \tag{$\lambda=0$}
    \\
    \sldiagram{0,0,0,1,0,1,0,0,0}
    \tag{$\lambda=1$}
    \\
    \sldiagram{0,0,1,0,1,0,1,0,0}
    \tag{$\lambda=2$}
    \\
    \sldiagram{0,1,0,1,0,1,0,1,0}
    \tag{$\lambda=3$}
    \\
    \sldiagram{1,0,1,0,1,0,1,0,1}
    \tag{$\lambda=4$}
  \end{gather*}
  \endgroup
  If~$U$ and~$W$ are two finite dimensional~{\representations{$\sllie_2(\kf)$}} then the weight diagram of their direct sum~$V \oplus W$ results from the weight diagrams of~$U$ and~$W$ by columnwise combination.
  If for example the weight diagram of~$U$ and~$W$ are given by
  \[
    \sldiagram[0.9]{0,2,2,4,2,2,0}
    \qquad\text{and}\qquad
    \sldiagram[0.9]{0,0,0,0,0,0,0\\2,1,3,0,3,1,2}
  \]
  then the weight diagram of~$U \oplus W$ will be given as follows:
  \[
    \sldiagram{0,2,2,4,2,2,0\\2,1,3,0,3,1,2}
  \]
\end{fluff}


\begin{fluff}
  Suppose now that~$V$ is a finite dimensional~{\representation{$\sllie_2(\kf)$}} that has the following weight diagram:
  \[
    \sldiagram{0,1,2,1,3,1,2,1,0}
  \]
  We see that the weight space~$V_3$ is nonzero.
  Hence one of the irreducible direct summands of~$V$ must be of the form~$\irr(\lambda)$ with~$\irr(\lambda)_3 \neq 0$, i.e.\ with~$\lambda \geq 3$ and~$\lambda$ odd.
  But we also see that~$V_\mu = 0$ for every~$\mu \geq 4$, so~$V$ cannot contain any direct summand~$\irr(\lambda)$ with~$\lambda \geq 4$.
  This means that one of the direct summands of~$V$ must be~$\irr(3)$.
  
  If~$W$ is a direct complement of~$\irr(3)$ in~$V$, i.e.\ a subrepresentation of~$V$ with~$V \cong W \oplus \irr(3)$ then we have seen that the weight diagram of~$W$ results from that of~$V$ by substracting the weight diagram of~$\irr(3)$.
  The weight diagram of~$\irr(3)$ looks as follows:
  \[
    \sldiagram{0,1,0,1,0,1,0,1,0}
  \]
  The weight diagram of~$W$ is therefore given as follows:
  \[
    \sldiagram{0,0,2,0,3,0,2,0,0}
  \]
  We can now repeat the above argumentation for~$W$, and find that~$\irr(2)$ is an irreducible direct summand of~$W$.
  By continuing this argumentation we see that
  \begin{align*}
    V
    &\cong
    W \oplus \irr(3)
    \\
    &\cong
    W' \oplus \irr(2) \oplus \irr(3)
    \\
    &\cong
    W'' \oplus \irr(2) \oplus \irr(2) \oplus \irr(3)
    \\
    &\cong
    \irr(0) \oplus \irr(2) \oplus \irr(2) \oplus \irr(3) \,.
  \end{align*}
\end{fluff}


\begin{example}
  The map
  \[
    \phi
    \colon
    \sllie_2(\kf)
    \to
    \sllie_3(\kf) \,,
    \quad
    A
    \mapsto
    \begin{pmatrix}
      A & 0 \\
      0 & 0
    \end{pmatrix}
  \]
  is a homomorphism of Lie~algebras.
  By pulling back the adjoint representation of~$\sllie_3(\kf)$ to a representation of~$\sllie_2(\kf)$ we get a~{\representation{$\sllie_2(\kf)$}}~$V \defined \sllie_3(\kf)$.
  Then~$h$ acts on~$V$ by the endomorphism~$\ad(H)$ where
  \[
    H
    \defined
    \begin{pmatrix*}[r]
    1 &  0 & 0 \\
    0 & -1 & 0 \\
    0 &  0 & 0
    \end{pmatrix*} \,.
  \]

  Let~$e_{12}$,~$e_{13}$,~$e_{23}$,~$e_{21}$,~$e_{31}$~$e_{32}$,~$h_1$,~$h_2$ be the basis of~$V$ consisting of the standard matrices~$e_{ij}$ with~$i \neq j$ together with the two traceless diagonal matrices
  \[
    h_1
    \defined
    \diag(1,-1,0)
    \quad\text{and}\quad
    h_2
    \defined
    \diag(0,1,-1) \,.
  \]
  We observe that
  \[
    [\diag(a_1, a_2, a_3), e_{ij}] = (a_i - a_j) e_{ij}
  \]
  for all~$a_1, a_2, a_3 \in \kf$ and~$i,j = 1, \dotsc, 3$.
  It follows that
  \[
    e_{21}
    \in
    V_{-2}\,,
    \quad
    e_{23}, e_{31}
    \in
    V_{-1}  \,,
    \quad
    h_1, h_2
    \in
    V_0 \,,
    \quad
    e_{13}, e_{32}
    \in
    V_1 \,,
    \quad
    e_{12}
    \in
    V_2 \,.
  \]
  The weight diagram of~$V$ is hence as follows:
  \[
    \sldiagram{1,2,2,2,1}
  \]
  It follows that
  \[
    V
    \cong
    \irr(2) \oplus \irr(1) \oplus \irr(1) \oplus \irr(0)  \,.
  \]
  
  We can also exhibit an explicit decomposition of~$V$ into irreducible subrepresentations:
  We know that the adjoint representation of~$\sllie_2(\kf)$ is irreducible because~$\sllie_2(\kf)$ is simple;
  it is isomorphism to~$\irr(2)$.
  A {\threedimensional} irreducible subrepresentation of~$V$ is hence given by
  \[
    W^{2,1}
    \defined
    \phi(\sllie_2(\kf))
    =
    \gen{ e_{12}, h_1, e_{21} }_{\kf} \,.
  \]
  This is actually the only {\threedimensional} irreducible subrepresentation of~$V$:
  If~$U$ were another one with~$U \neq W^{2,1}$ then~$U \cap W^{2,1} = 0$ because~$U$ and~$W^{2,1}$ are simple.
  But then~$U \oplus W^{2,1}$ is a direct summand of~$V$ isomorphic to~$\irr(2) \oplus \irr(2)$.
  This is not possible because~$\irr(2)$ has multiplicity~$1$ in~$V$.
  
  Any two {\twodimensional} irreducible subrepresentations~$W^{1,1}$ and~$W^{1,2}$ of~$V$ must satisfy
  \[
    W^{1,1} \oplus W^{1,2}
    =
    V_{-1} \oplus V_1
    =
    \gen{ e_{23}, e_{31}, e_{13}, e_{32} }_{\kf}  \,.
  \]
  We can choose~$W^{1,1} \defined \gen{e_{23}, e_{13}}_{\kf}$ and $W^{1,2} \defined \gen{e_{31}, e_{32}}_{\kf}$.
  To find a {\onedimensional} irreducible subrepresentation --- which is actually unique by the same argument as for the {\threedimensional} irreducible subrepresentation --- we observe that
 \begin{alignat*}{3}
    e.h_1 &= -2 e_{12} \,,
    &
    \quad
    h.h_1 &= 0 \,,
    &
    \quad
    f.h_1 &= 2e_{21} \,,
  \shortintertext{and}
    e.h_2 &= e_{12}  \,,
    &
    \quad
    h.h_2 &= 0 \,,
    &
    \quad
    f.h_2 &= -e_{21} \,.
 \end{alignat*}
 Whence~$\sllie_2(\kf)$ acts trivially on the {\onedimensional} linear subspace~$W^{0,1} \defined \gen{ h_1 + 2h_2 }_{\kf}$, which is why it is a subrepresentation.
\end{example}


\begin{example}
  Let~$V \defined \irr(\lambda)$ with~$\lambda \geq 0$ be the irreducible~{\representation{$\sllie_2(\kf)$}} with highest weight~$\lambda$.
  Let~$v_{-\lambda}, v_{-\lambda+2}, \dotsc, v_{\lambda-2}, v_{\lambda}$ be a basis of~$V$ consisting of weight vectors~$v_\mu \in V_\mu$.
  Then~$h.v_\mu^* = -\mu v_\mu^*$ for the dual basis~$v_{-\lambda}^*, v_{-\lambda+2}^*, \dotsc, v_{\lambda-2}^*, v_{\lambda}^*$ of~$V^*$ because
  \[
    (h.v_\mu^*)(v_\kappa)
    =
    v_\mu^*(-h.v_\kappa)
    =
    v_\mu^*(-\kappa v_\kappa)
    =
    -\kappa \delta_{\mu,\kappa}
    =
    -\kappa v_\mu^*(v_\kappa)
  \]
  for every basis~vector~$v_\kappa$.
  Hence the weight diagram for~$V^*$ is the same as for~$\irr(\lambda)$ (and hence~$V$ itself).
  This shows that~$\irr(\lambda)^* \cong \irr(\lambda)$ for all~$\lambda \geq 0$.
\end{example}


\begin{theorem}[Structure of finite dimensional~{\representations{$\sllie_2(\kf)$}}]
  \label{finite dimensional representations of sl2}
  Let~$V$ be a finite dimensional~{\representation{$\sllie_2(\kf)$}}.
  \begin{enumerate}
    \item
      There exists a decomposition
      \begin{equation}
        \label{decomposition into irreps}
        V
        =
        \bigoplus_{\lambda \geq 0}
        \bigoplus_{i=1}^{n_\lambda}
        W^{\lambda,i}
      \end{equation}
      into irreducible subrepresentations where the summands~$W^{\lambda,i}$ are of highest weight~$\lambda$.
    \item
      The representation~$V$ admits a weight space decomposition~$V = \bigoplus_{\mu \in \Integer} V_\mu$.
      In particular all occuring weights are integral.
    \item
      It holds for every potential weight~$\mu \in \Integer$ that~$\dim V_\mu = \dim V_{-\mu}$.
%     \item
%       Let~$\lambda \in \Integer$ be a potential weight.
%       If~$\lambda \geq 0$ then
%       \begin{equation}
%         \label{positive eigenspace of a finite dimensional representation}
%           V_\lambda
%           =
%           \bigoplus_{\substack{\mu \geq \lambda \\ \mu \equiv \lambda}}
%           \bigoplus_{i=1}^{n_\mu} W^{\mu,i}_\lambda
%       \end{equation}
%       where~$\mu \equiv \lambda$ means that~$\mu \equiv \lambda \pmod{2}$.
%       If~$\lambda \leq 0$ then similarly
%       \begin{equation}
%         \label{negative eigenspace of a finite dimensional representation}
%           V_\lambda
%           =
%           \bigoplus_{\substack{\mu \leq \lambda \\ \mu \equiv \lambda}}
%           \bigoplus_{j=1}^{n_\mu} W^{\mu,j}_\lambda \,.
%       \end{equation}
%       In both cases each summand~$W^{\lambda,j}_\lambda$ is {\onedimensional}.
    \item
      \label{calculation of multiplicities}
      The multiplicities~$n_\lambda$ for~$\lambda \geq 0$ are given by~$n_\lambda = \dim V_\lambda - \dim V_{\lambda+2}$.
  \end{enumerate}
\end{theorem}


\begin{proof}
  It follows from Weyl’s~theorem that there exists such a decomposition~$V = \bigoplus_{\lambda \geq 0} \bigoplus_{i=1}^{n_\lambda} W^{\lambda,i}$ into irreducible subrepresentations.
  Each irreducible direct summand~$W^{\lambda,i} \cong \irr(\lambda)$ admits a weight space decomposition, the combination of which gives a weight space decomposition for~$V$.
  We have for all~$\lambda \geq 0$ and~$\mu \in \Integer$ that~$\dim W^{\lambda,i}_\mu = W^{\lambda,i}_{-\mu}$ because~$W^{\lambda,i} \cong \irr(\lambda)$.
  It follows that~$\dim V_\mu = \dim V_{-\mu}$.
  
  For~$\mu \geq 0$ the dimension of the corresponding weight space~$V_\mu$ counts how many direct summands~$W^{\lambda,i}$ with~$\lambda \geq \mu$ and~$\mu \equiv \lambda \pmod{2}$ occur in the given decomposition.
  The difference~$\dim V_\mu - \dim V_{\mu + 2}$ therefore counts the how many direct summands of the form~$W^{\lambda,i}$ with exactly~$\lambda = \mu$ appear in the given decomposition.
\end{proof}



\subsubsection{The Clebsch--Gordan Decomposition}


\begin{lemma}
  \label{tensor of sl2 weights}
  Let~$V$ and~$W$ be two~{\representations{$\sllie_2(\kf)$}} and let~$v \in V_\lambda$ and~$w \in W_\mu$ for some potential weights~$\lambda, \mu \in \kf$.
  Then~$v \tensor w \in (V \tensor W)_{\lambda + \mu}$.
\end{lemma}


\begin{proof}
  We have~$h.(v \tensor w) = (h.v) \tensor w + v \tensor (h.w) = \lambda v \tensor w + \mu v \tensor w = (\lambda + \mu) v \tensor w$.
\end{proof}


\begin{example}
  \label{tensor of 3 and 2}
  The weight diagrams of~$V \defined \irr(3)$ and~$W \defined \irr(2)$ look as follows:
  \[
    \sldiagram{1,0,1,0,1,0,1}
    \qquad
    \qquad
    \sldiagram{1,0,1,0,1}
  \]
  We may choose basis~$v_{-3}$,~$v_1$,~$v_1$,~$v_3$ of~$V$ and~$w_{-2}$,~$w_0$,~$w_2$ of~$W$ consisting of weight vectors~$v_\lambda \in V_\lambda$ and~$w_\mu \in W_\mu$.
  It follows from \cref{tensor of sl2 weights} that~$v_\lambda \tensor w_\mu$ is a weight vector in~$V \tensor W$ of weight~$\lambda + \mu$.
  Hence~$V \tensor w_\mu$ admits a weight diagram (even though it is for~$\mu \neq 0$ not a subrepresentation of~$V \tensor W$).
  The weight diagram of~$V \tensor w_\mu$ results from that of~$V$ by shifting it by~$\mu$:
  \[
    \sldiagram[2][1]{1,0,1,0,1,0,1}[-3+\mu,-2+\mu,-1+\mu,0+\mu,1+\mu,2+\mu,3+\mu]
  \]
  It follows that in the decomposition
  \[
    V \tensor W
    =
    V \tensor (\kf w_{-2} \oplus \kf w_0 \oplus \kf w_2)
    =
    (V \tensor w_{-2}) \oplus (V \tensor w_0) \oplus (V \tensor w_2)
  \]
  the direct summand~$V \tensor w_{-2}$,~$V \tensor w_0$ and~$V \tensor w_2$ have the following weight diagrams:
  \[
    \begingroup
    \renewcommand{\arraystretch}{3.5}
    \begin{array}{rc}
      V \tensor w_{-2}:
      &
      \sldiagram{1,0,1,0,1,0,1,0,0,0,0} \\
      V \tensor w_0:
      &
      \sldiagram{0,0,0,0,0,0,0,0,0,0,0\\0,0,1,0,1,0,1,0,1,0,0} \\
      V \tensor w_2:
      &
      \sldiagram{0,0,0,0,0,0,0,0,0,0,0\\0,0,0,0,0,0,0,0,0,0,0\\0,0,0,0,1,0,1,0,1,0,1}
    \end{array}
    \endgroup
  \]
  We get the total weight diagram of~$V \tensor W$ by adding together these three weight diagrams:
  \[
    \begin{array}{rc}
    V \tensor W:
    &
    \sldiagram{1,0,1,0,1,0,1,0,0,0,0\\0,0,1,0,1,0,1,0,1,0,0\\0,0,0,0,1,0,1,0,1,0,1}
    \end{array}
  \]
  We see from the weight diagram of~$V \tensor W$ that
  \[
    \irr(3) \tensor \irr(2)
    =
    V \tensor W
    \cong
    \irr(5) \oplus \irr(3) \oplus \irr(1) \,.
  \]
\end{example}


\begin{example}
  \label{tensor of 4 and 3}
  We find in the same way as in \cref{tensor of 3 and 2} that the weight diagram of the tensor product~$\irr(4) \tensor \irr(3)$ is as follows:
  \[
    \sldiagram{1,0,1,0,1,0,1,0,1,0,0,0,0,0,0\\0,0,1,0,1,0,1,0,1,0,1,0,0,0,0\\0,0,0,0,1,0,1,0,1,0,1,0,1,0,0\\0,0,0,0,0,0,1,0,1,0,1,0,1,0,1}
  \]
  Hence
  \[
    \irr(4) \tensor \irr(3)
    \cong
    \irr(7) \oplus \irr(5) \oplus \irr(3) \oplus \irr(1)  \,.
  \]
\end{example}


\begin{proposition}[Clebsch--Gordan]
  \label{clebsch gordan}
  For all weights~$\lambda, \mu \geq 0$ with~$\lambda \geq \mu$,
  \[
    \irr(\lambda) \tensor \irr(\mu)
    \cong
    \irr(\lambda+\mu) \oplus \irr(\lambda+\mu-2) \oplus \dotsb \oplus \irr(\lambda-\mu) \,,
  \]
  and hence more generally
  \[
    \irr(\lambda) \tensor \irr(\mu)
    \cong
    \irr(\lambda+\mu) \oplus \irr(\lambda+\mu-2) \oplus \dotsb \oplus \irr(\abs{\lambda-\mu})
  \]
  for all~$\lambda, \mu \geq 0$.
\end{proposition}


\begin{proof}[First proof]
  Let~$V \defined \irr(\lambda)$ and $W \defined \irr(\mu)$.
  Then both~$V = \bigoplus_{p=0}^\lambda V_{-\lambda+2p}$ and~$W = \bigoplus_{q=0}^\mu W_{-\mu+2q}$ and hence
  \[
    V \tensor W
    =
    \bigoplus_{p=0}^\lambda
    \bigoplus_{q=0}^\mu
    V_{-\lambda+2p} \tensor W_{-\mu+2q}
  \]
  We find for every potential weight~$\kappa \geq 0$ that
  \[
    (V \tensor W)_\kappa
    =
    \bigoplus_{
      \substack{
        \lambda'      = -\lambda, -\lambda+2, \dotsc, \lambda-2, \lambda \\
        \mu'          = -\mu, -\mu+2 \dotsc, \mu-2, \mu \\
        \lambda + \mu = \kappa
      }
    }
    V_{\lambda'} \tensor W_{\mu'} \,.
  \]
  The dimension of $(V \tensor W)_\kappa$ is therefore the number of solutions of the equation
  \begin{equation}
    \label{clebsch gordan by counting}
    \lambda' + \mu' = \kappa
    \qquad\text{with}\qquad
    \left\{
    \begin{aligned}
      p &\in \{-\lambda, -\lambda+2, \dotsc, \lambda-2, \lambda\}  \,,  \\
      q &\in \{-\mu, -\mu+2, \dotsc, \mu-2, \mu\}  \,.
    \end{aligned}
    \right.
  \end{equation}
  Let us count these solutions:
  We observe that \eqref{clebsch gordan by counting} has no solution for~$\kappa < -\lambda-\mu$ or~$\kappa > \lambda+\mu$.
  We also observe that~$\lambda' + \mu' \equiv \lambda+\mu \pmod{2}$ in \eqref{clebsch gordan by counting}.
  So we only need to count the solutions of \eqref{clebsch gordan by counting} for~$\kappa = -(\lambda+\mu), -(\lambda+\mu)+2, \dotsc, (\lambda+\mu)-2, (\lambda+\mu)$.
  The solutions to these values of~$\kappa$ are given as follows:
  \[
    \begingroup
    \renewcommand{\arraystretch}{1.3}
    \begin{array}{ccc}
        \kappa
      & \text{solutions $(\lambda', \mu')$ for~$\lambda' + \mu' = \kappa$ as in~\eqref{clebsch gordan by counting}} 
      & \text{\#solutions}
      \\
      \hline
        \lambda + \mu
      & (\lambda, \mu)
      & 1
      \\
        \lambda + \mu - 2  
      & (\lambda, \mu - 2),
        (\lambda - 2, \mu),
      & 2
      \\
      \lambda + \mu - 4
      & (\lambda, \mu - 4),
        (\lambda - 2, \mu - 2),
        (\lambda - 4, \mu),
      & 3
      \\
        \vdots
      & \vdots
      & \vdots
      \\
        \lambda - \mu
      & (\lambda, -\mu),
        (\lambda - 2, - \mu + 2),
        \dotsc,
        (\lambda - 2\mu, \mu)
      & \mu + 1
      \\
        \lambda - \mu - 2
      & (\lambda - 2, - \mu),
        (\lambda - 4, - \mu + 2),
        \dotsc,
        (\lambda - 2 \mu - 2, \mu)
      & \mu + 1
      \\
        \vdots
      & \vdots
      & \vdots
      \\
        - \lambda + \mu + 2 
      & \quad
        (- \lambda + 2 \mu + 2, - \mu),
        (- \lambda + 2 \mu, - \mu + 2),
        \dotsc,
        (- \lambda + 2, \mu)
        \quad
      & \mu + 1
      \\
        - \lambda + \mu
      & (- \lambda + 2 \mu, - \mu),
        (- \lambda + 2 \mu - 2, - \mu + 2),
        \dotsc,
        (- \lambda, \mu)
      & \mu + 1
      \\
        \vdots
      & \vdots
      & \vdots
      \\
        - \lambda - \mu + 4
      & (- \lambda + 4, - \mu),
        (- \lambda + 2, - \mu + 2),
        (- \lambda, - \mu + 4)
      & 3
      \\
        - \lambda - \mu + 2
      & (- \lambda + 2, - \mu),
        (- \lambda, - \mu + 2)
      & 2
      \\
        - \lambda - \mu
      & (- \lambda, - \mu)
      & 1
    \end{array}
    \endgroup
  \]
  This results in the following dimensions:
  \[
    \begingroup
    \renewcommand{\arraystretch}{1.2}
    \begin{array}{cccccccccc}
        \kappa
      & -\lambda-\mu
      & -\lambda-\mu+2
      & \cdots
      & -\lambda+\mu
      & \cdots
      & \lambda-\mu
      & \cdots 
      & \lambda+\mu-2
      & \lambda+\mu
      \\
      \hline
      \dim(V \tensor W)_\kappa
      & 1
      & 2
      & \cdots
      & \mu+1
      & \cdots
      & \mu+1
      & \cdots
      & 2
      & 1
    \end{array}
    \endgroup
  \]
  We find with part~\ref*{calculation of multiplicities}~of~\cref{finite dimensional representations of sl2} (or with the weight diagram of~$V \tensor W$)   that
  \[
    \irr(\lambda) \tensor \irr(\mu)
    =
    V \tensor W
    \cong
            \irr(\lambda+\mu)
    \oplus  \irr(\lambda+\mu-2)
    \oplus  \dotsb
    \oplus  \irr(\lambda-\mu)
  \]
  as claimed.
\end{proof}
 
\begin{proof}[Second proof of \cref{clebsch gordan} {\cite[\S 1.4]{lectures_on_sl2_modules}}]
  We use induction on~$\mu$ (with~$\lambda \geq \mu$).
  For~$\mu = 0$ we have~$\irr(\lambda) \tensor \irr(0) \cong \irr(\lambda) \tensor \kf \cong \irr(\lambda)$.
  
  For~$\mu = 1$ let~$v_{-\lambda}, v_{-\lambda+2}, \dotsc, v_{\lambda-2}, v_\lambda$ be a basis of~$\irr(\lambda)$ consisting of weight vectors~$v_\kappa \in V_\kappa$.
  We may assume~$e.v_{\lambda-2} = v_\lambda$ by rescaling the basis~vectors if necessary.
  Similarly let~$w_{-1}$,~$w_1$ be a basis of~$\irr(1)$ with~$w_1 \in W_1$ and~$w_{-1} \in W_{-1}$.
  We have~$e.v_\lambda = 0$ and~$e.w_1 = 0$.
  
  The element~$v_\lambda \tensor w_1 \in \irr(\lambda) \tensor \irr(1)$ is a primitive vector of weight~$\lambda + \mu$.
  It follows from \cref{classification of highest weight for sl2} that the tensor product~$\irr(\lambda) \tensor \irr(1)$ contains a copy~$U_1$ of~$\irr(\lambda+1)$.
  Similarly the element~$z \defined v_{n-2} \tensor w_1 - v_n \tensor w_2 \in \irr(\lambda) \tensor \irr(1)$ is a primitive vector of weight~$\lambda - 1$ (where it is used that $e.v_{n-2} = v_n$).
  It follows from \cref{classification of highest weight for sl2} that the tensor product~$\irr(\lambda) \tensor \irr(1)$ contains a copy~$U_2$ of~$\irr(\lambda-1)$.
  We have~$U_1 \cap U_2 = 0$ because~$U_1$ and~$U_2$ are two distinct irreducible subrepresentations.
  It follows that~$V = U_1 \oplus U_2 \cong \irr(\lambda+1) \oplus \irr(\lambda-1)$ because
  \[
    \dim U_1 + \dim U_2
    =
    (\lambda+2) + \lambda
    =
    2\lambda + 2
    =
    \dim \irr(\lambda) \cdot \dim \irr(1)
    =
    \dim \irr(\lambda) \tensor \irr(1)  \,.
  \]
  This shows the claimed formula for~$\mu = 1$.
  
  For the induction step let~$\mu \geq 2$.
  To understand the tensor product~$\irr(\lambda) \tensor \irr(\mu)$ we calculate the term~$\irr(\lambda) \tensor \irr(\mu-1) \tensor \irr(2)$ in two ways.
  We have on the one hand that
  \begin{align*}
    {}&
    \irr(\lambda) \tensor \irr(\mu-1) \tensor \irr(2)
    \\
    \cong{}&
    \irr(\lambda) \tensor ( \irr(\mu-1) \tensor \irr(2) )
    \\
    \cong{}&
    \irr(\lambda) \tensor ( \irr(\mu) \oplus \irr(\mu-2) )
    \\
    \cong{}&
    ( \irr(\lambda) \tensor \irr(\mu) )
    \oplus
    ( \irr(\lambda) \tensor \irr(\mu-2) )
  \end{align*}
  where the second direct summand~$\irr(\lambda) \tensor \irr(\mu-2)$ is by induction hypothesis given by
  \[
    \irr(\lambda) \tensor \irr(\mu-2)
    =
            \irr(\lambda+\mu-2)
    \oplus  \irr(\lambda+\mu-4)
    \oplus  \dotsb
    \oplus  \irr(\lambda-\mu+2) \,.
  \]
  We have on the other hand that
  \begin{align*}
    {}&
    \irr(\lambda) \tensor \irr(\mu-1) \tensor \irr(2)
    \\
    \cong{}&
    ( \irr(\lambda) \tensor \irr(\mu-1) ) \tensor \irr(2)
    \\
    \cong{}&
    (
              \irr(\lambda+\mu-1)
      \oplus  \irr(\lambda+\mu-3)
      \oplus  \dotsb
      \oplus  \irr(\lambda-\mu+1)
    )
    \tensor
    \irr(2)
    \\
    \cong{}&
            ( \irr(\lambda+\mu-1) \tensor \irr(2) )
    \oplus  ( \irr(\lambda+\mu-3) \tensor \irr(2) )
    \oplus  \dotsb
    \oplus  ( \irr(\lambda-\mu+1) \tensor \irr(2) )
    \\
    \cong{}&
            \irr(\lambda+\mu)   \oplus \irr(\lambda+\mu-2)
    \oplus  \irr(\lambda+\mu-2) \oplus \dotsb
    \\
    {}& 
            \phantom{\irr(\lambda+\mu)} \,
    \oplus  \irr(\lambda-\mu+2)
    \oplus  \irr(\lambda-\mu+2) \oplus \irr(\lambda-\mu)  \,.
  \end{align*}
  By comparing both decompositions of~$\irr(\lambda) \tensor \irr(\mu-1) \tensor \irr(1)$ and then using the uniqueness of multiplicities for irreducible summands we see that
  \[
    \irr(\lambda) \tensor \irr(\mu)
    \cong
            \irr(\lambda+\mu)
    \oplus  \irr(\lambda+\mu-2)
    \oplus  \dotsb
    \oplus  \irr(\lambda-\mu) \,,
  \]
  as claimed.
\end{proof}


\begin{example}
  For all~$\lambda, \mu \geq 0$ we find that
  \begin{align*}
    \Hom_\kf( \irr(\lambda), \irr(\mu) )
    &\cong
    \irr(\lambda)^* \tensor \irr(\mu)
    \\
    &\cong
    \irr(\lambda) \tensor \irr(\mu)
    \\
    &\cong
            \irr(\lambda+\mu)
    \oplus  \irr(\lambda+\mu-2)
    \oplus  \dotsb
    \oplus  \irr(\abs{\lambda-\mu})
  \end{align*}
  as representations and hence
  \begin{align*}
    \Hom_{\sllie_2(\kf)}( \irr(\lambda), \irr(\mu) )
    &=
    \Hom_{\kf}( \irr(\lambda), \irr(\mu) )^{\sllie_2(\kf)}
    \\
    &\cong
            \irr(\lambda+\mu)^{\sllie_2(\kf)}
    \oplus  \irr(\lambda+\mu-2)^{\sllie_2(\kf)}
    \oplus  \dotsb
    \oplus  \irr(\abs{\lambda-\mu})^{\sllie_2(\kf)} \,.
  \end{align*}
  We have for every weight~$\kappa \geq 0$ that~$\irr(\kappa)^{\sllie_2(\kf)} = 0$ if~$\kappa \neq 0$ and otherwise~$\irr(0)^{\sllie_2(\kf)} = \irr(0)$.
  Whence the space~$\Hom_{\glie}( \irr(\lambda), \irr(\mu) )$ is {\onedimensional} if~$\lambda = \mu$ and zero otherwise.
  This agrees with Schur’s~lemma.
\end{example}




