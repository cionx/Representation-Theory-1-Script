\section{Graded \texorpdfstring{$\kf$}{k}-Algebras}


\begin{definition}
	\label{definition of graded algebras}
	Let~$A$ be a~\algebra{$\kf$}.
	A \defemph{grading}\index{grading} or \defemph{gradation}\index{gradation} of~$A$ is a direct sum decomposition
	\[
		A
		=
		\bigoplus_{p \geq 0} A_p
	\]
	such that
	\[
		A_p A_q
		\subseteq
		A_{p+q}
	\]
	for all~$p, q \geq 0$.
	A \defemph{graded~\algebra{$\kf$}}\index{graded algebra} is a~\algebra{$\kf$}~$A$ together with a grading~$A = \bigoplus_{p \geq 0} A_p$.
	The direct summand~$A_p$\glsadd{homogeneous part} is the~\howmanyth{$p$} \defemph{homogeneous part}\index{homogeneous!part} of~$A$.
\end{definition}


\begin{remark}
	We often say that~\enquote{$A$ is a graded algebra} without explicitely mentioning the involved grading.
\end{remark}


\begin{remark}
	Let~$(S, +)$ be an abelian monoid.
	An~\defemph{\grading{$S$}}\index{grading} of a~\algebra{$\kf$}~$A$ is a direct sum decomposition
	\[
		A
		=
		\bigoplus_{s \in S}
		A_s
	\]
	such that
	\[
		A_s A_t
		\subseteq
		A_{s + t}
	\]
	for all~$s, t \in S$.
	An~\defemph{\graded{$S$}~\algebra{$\kf$}}\index{graded algebra} is a~\algebra{$\kf$}~$A$ together with an~\grading{$S$}~$A = \bigoplus_{s \in S} A_s$.

	We see now that a graded~\algebra{$\kf$} in the sense of \cref{definition of graded algebras} is the special case of an~\graded{$\Natural$}~\algebra{$\kf$}.
	We will restrict our attention troughout these notes to~{\gradings{$\Natural$}}, and refer to the excellent \cite[II.{\S}11, III.{\S}3]{bourbaki_algebra_1} for more general gradings.
\end{remark}


\begin{definition}
	Let~$A$ be a graded~{\algebra{$\kf$}} with grading~$A = \bigoplus_{p \geq 0} A_p$.
	\begin{enumerate}
		\item
			An element~$x$ of~$A$ is \defemph{homogeneous}\index{homogeneous!element} of \defemph{degree}~$p$ if~$x$ is contained in the direct summand~$A_p$.
		\item
			Let~$x$ be an element of~$A$ and let~$x = \sum_{p \geq 0} x_p$ be the decomposition of~$x$ with respect to the grading~$A = \bigoplus_{p \geq 0} A_p$.
			The summands~$x_p$ are the~\defemph{homogeneous components}\index{homogeneous!component} of~$x$, and the decomposition~$x = \sum_{p \geq 0} x_p$\glsadd{homogeneous summand} is the~\defemph{homogeneous decomposition}\index{homogeneous!decomposition} of~$x$.
	\end{enumerate}
\end{definition}


\begin{proposition}
	Let~$A$ be a graded~\algebra{$\kf$}.
	The multiplicative neutral element~$1_A$ is homogeneous of degree~$0$.
\end{proposition}


\begin{proof}
	We consider the homogeneous decomposition~$1 = \sum_{p \geq 0} x_p$.
	We have for every degree~$q$ and every element~$a$ of~$A_q$ that
	\[
		a
		=
		1 \cdot a
		=
		\Biggl( \sum_{p \geq 0} e_p \Biggr) \cdot a
		=
		\sum_{p \geq 0} e_p a  \,.
	\]
	Each summand~$e_p a$ is homogeneous of degree~$p + q$.
	It therefore follows from the directness of the decomposition~$A = \bigoplus_{p \geq 0} A_p$ that the element~$a$ equals the summand~$e_0 a$ (and that all other summands~$e_p a$ with~$p \neq 0$ vanish).
	It follows that~$e_0 a = a$ for all~$a \in A$, which shows that~$e_0$ is a multiplicative left neutral element in~$A$.
	It follows that~$e_0 = 1$.
\end{proof}


\begin{corollary}
	Let~$A$ be a graded~\algebra{$\kf$}.
	The homogenous part~$A_0$ is a~\subalgebra{$\kf$} of~$A$.
	\qed
\end{corollary}


\begin{examples}
	\label{examples for graded algebras}
	\leavevmode
	\begin{enumerate}
		\item
			Any~\algebra{$\kf$}~$A$ becomes a graded~\algebra{$\kf$} by setting~$A_0 \defined A$ and~$A_p \defined 0$ for all~$p \geq 1$.
		\item
			The polynomial ring~$A = \kf[x_i \suchthat i \in I]$ is a graded~\algebra{$\kf$} by setting
			\[
				A_d
				\defined
				\gen[\big]{
					x_{i_1}^{n_1} \dotsm x_{i_r}^{n_r}
				\suchthat[\big]
					r \geq 0,
					i_1, \dotsc, i_r \in I,
					n_1 + \dotsb + n_r = d
				}_{\kf}
			\]
			for every~$d \geq 0$.
			The homogeneous part~$A_d$ consists precisely of the homogeneous polynomials of degree~$d$, and each monomial~$x_{i_1}^{n_1} \dotsm x_{i_r}^{n_r}$ is homogeneous of degree~$n_1 + \dotsb + n_r$.
			
			We can more generally put the variable~$x_i$ is any degree~$d_i$, as follows.
			Given any family~$(d_i)_{i \in I}$ of natural numbers~$d_i$ we can define a grading on~$A$ via
			\[
				A_d
				\defined
				\gen[\big]{
					x_{i_1}^{n_1} \dotsm x_{i_r}^{n_r}
				\suchthat[\big]
					r \geq 0,
					i_1, \dotsc, i_r \in I,
					n_1 d_1 + \dotsb + n_r d_r = d
				}_{\kf}
			\]
			for every~$d \geq 0$.
			Then each monomials~$x_{i_1}^{n_1} \dotsm x_{i_r}^{n_r}$ is homogeneous of degree~$n_1 d_1 + \dotsb + n_r d_r$.
			It holds in particular that each variables~$x_i$ is homogeneous of degree~$d_i$.
		\item
			The noncommutative polynomial \algebra{$\kf$}~$A \defined \kf\gen{x_i \suchthat i \in I}$ can be graded via
			\[
				A_d
				\defined
				\gen{
					x_{i_1}^{n_1} \dotsm x_{i_r}^{n_r}
				\suchthat
					r \geq 0,
					i_1, \dotsc, i_r \in I,
					n_1 + \dotsb + n_r = d
				}_{\kf}
			\]
			for every~$d \geq 0$, and for any family~$(d_i)_{i \in I}$ of natural numbers $d_i$ via
			\[
				A_d
				\defined
				\gen{
					x_{i_1}^{n_1} \dotsm x_{i_r}^{n_r}
				\suchthat
					r \geq 0,
					i_1, \dotsc, i_r \in I,
					n_1 d_1 + \dotsb + n_r d_r = d
				}_{\kf} \,.
			\]
			This grading makes each monomial~$x_{i_1}^{n_1} \dotsm x_{i_r}^{n_r}$ homogeneous of degree~$n_1 d_1 + \dotsb + n_r d_r$.
		\item
			The tensor algebra~$\Tensor(V)$ of a vector space~$V$ admits a grading given by~$\Tensor(V)_p \coloneqq V^{\tensor p}$ for every~$p \geq 0$.
			Similarly, both the symmetric algebra~$\Symm(V)$ and the exterior algebra~$\Exterior(V)$ have gradings given by~$\Symm(V)_p = \Symm^p(V)$ and~$\Exterior(V)_p = \Exterior^p(V)$ for every~$p \geq 0$.
		\item
			Let~$M$ be a multiplicative monoid and let~$M = \coprod_{p \geq 0} M_p$ be a grading of~$M$, i.e. a disjoint decomposition of~$M$ into subsets~$M_p$ with~$p \geq 0$ such that~$M_p \cdot M_q \subseteq M_{p+q}$ for all~$p, q \geq 0$.
			The monoid algebra~$\kf[M]$\index{monoid algebra} inhericts a grading from~$M$ via
			\[
				\kf[M]_p
				\defined
				\gen{ M_p }_{\kf}
			\]
			for every~$p \geq 0$.
			As special cases of this construction we get the following examples:
			\begin{enumerate}
				\item
					Let~$I$ is an index set and let~$M = \Natural^{\oplus I}$.
					A grading on~$M$ is given by
					\[
						M_p
						\defined
						\left\{
							(n_i)_{i \in I} \in M
						\suchthat*
							\sum_{i \in I} n_i = p
						\right\}
					\]
					for every~$p \geq 0$.
					Then~$\kf[M]$ together with the induced grading is the commutative polynomial algebra~$\kf[x_i \suchthat i \in I]$.
				\item
					Let~$I$ be an index set and let let~$\Sigma \defined \{x_i \suchthat i \in I\}$ be an alphabet with letters~$x_i$.
					let~$M \defined \Sigma^*$ be the monoid of words in the alphabet~$\Sigma$ together with concatenation of words as its rule of composition.
					The monoid~$M$ has a grading given by
					\[
						M_d
						\defined
						\{
							w \in M
						\suchthat
							\text{$w$ is a word in~$\Sigma$ of length~$d$}
						\}
					\]
					for all~$d \geq 0$.
					The algebra~$\kf[M]$ together with the induced grading is the noncommutative polynomial algebra~$\kf\gen{x_i \suchthat i \in I}$.
			\end{enumerate}
	\end{enumerate}
\end{examples}


\begin{convention}
	Let~$V$ be a vector space.
	We will always regard the tensor algebra~$\Tensor(V)$\glsadd{tensor algebra}\index{tensor algebra}\index{grading!on the tensor algebra}, the symmetric algebra~$\Symm(V)$\glsadd{symmetric algebra}\index{symmetric algebra}\index{grading!on the symmetric algebra}, and the exterior algebra~$\Exterior(V)$\glsadd{exterior algebra}\index{exterior algebra}\index{grading!on the exterior algebra} is graded algebras, as explained above.
	For every index set~$I$ we will regard the commutative polynomial algebra~$\kf[ x_i \suchthat i \in I]$ and the noncommutative polynomial algebra~$\kf\gen{ x_i \suchthat i \in I}$ as graded~\algebras{$\kf$} such that each variable~$x_i$ is homogeneous of degree~$1$.
\end{convention}



\begin{remark}
	\label{external description of graded algebras}
	The grading of the tensor algebra~$\Tensor(V)$, symmetric algebra~$\Symm(V)$ and exterior algebra~$\Exterior(V)$ are basically built into the construction of these algebras.
	This way of constructing graded~\algebras{$\kf$} can be generalized as follows:
	
	Suppose that we are given a sequence of vector spaces~$A_p$ with~$p \geq 0$ and bilinear maps
	\[
		\mu_{p,q}
		\colon
		A_p \times A_q
		\to
		A_{p+q} \,,
		\quad
		(x,y)
		\mapsto
		xy
	\]
	for all~$p, q \geq 0$ such that
	\begin{itemize}
		\item
			the maps~$\mu_{p, q}$ are relatively associative in the sense that
			\[
				x(yz)
				=
				(xy)z
			\]
			for all~$p, q, r \geq 0$ and~$x \in A_p$,~$y \in A_q$,~$z \in A_r$, and
		\item
			there exists an element~$1$ in~$A_0$ with
			\[
				1 \cdot x
				=
				x
				=
				x \cdot 1
			\]
			for all~$p \geq 0$ and~$x \in A_p$.
	\end{itemize}
	We can then set~$A \defined \bigoplus_{p \geq 0} A_p$ and fit together the partial multiplication maps~$\mu_{p,q}$ into a single multiplication~ map
	\[
		\mu
		\colon
		A \times A
		\to
		A
	\]
	that is given on elements~$x$,~$y$ of~$A$ with~$x = (x_p)_{p \geq 0}$ and~$y = (y_p)_{p \geq 0}$ by
	\[
		x y
		=
		\Biggl( \sum_{q=0}^p x_q y_{p-q} \Biggr)_{p \geq 0}  \,.
	\]
	The bilinearity of the map~$\mu$ follows from the bilinearities of the maps~$\mu_{p,q}$.
	It follows from the relative associativity of the multiplication maps~$\mu_{p,q}$ that the multiplication map~$\mu$ is associative.
	And the element~$1$ of~$A_0$ is a multiplicative neutral element for~$\mu$.
	We identify every vector spacp~$A_p$ with the corresponding direct summand of~$A$.

	We have in this way constructed a grapded algebra~$A$.
	This construction gives an external description of graded~\algebras{$\kf$}, in constrast to the internal description in \cref{definition of graded algebras}.
\end{remark}


\begin{definition}
	Let~$A$ and~$B$ be two graded~\algebras{$\kf$}.
	\begin{enumerate}
		\item
			A \defemph{homomorphism of graded~\algebras{$\kf$}}\index{homomorphism!of graded algebras} from~$A$ to~$B$ is a homomorphism of~\algebras{$\kf$}~$\Phi$ from~$A$ to~$B$ such that the image~$\Phi(A_p)$ is contained in the homogeneous part~$B_p$ for every~$p \geq 0$.
		\item
			Let~$\Phi$ be a homomorphism of graded algebras from~$A$ to~$B$.
			The restriction of~$\Phi$ to a linear map from~$A_p$ to~$B_p$ is denoted by~$\Phi_p$\glsadd{restricted homomorphism of graded algebras} for every~$p \geq 0$.
	\end{enumerate}
\end{definition}


\begin{remark}
	\leavevmode
	\begin{enumerate}
		\item
			Let~$A$,~$B$, and~$C$ be graded~\algebra{$\kf$}.
			\begin{enumerate}
				\item
					The identity map of~$A$ is a homorphism of graded~\algebras{$\kf$} from~$A$ to~$A$.
				\item
					Suppose that~$\Phi$ is a homomorphism of graded algebras from~$A$ to~$B$ and that~$\Psi$ is a homomorphism of graded algebras from~$B$ to~$C$.
					Their composite~$\Psi \circ \Phi$ is a homomorphism of graded algebras from~$A$ to~$C$.
			\end{enumerate}
			It follows from these observations that the graded~\algebras{$\kf$} together with the homomorphisms of graded~\algebras{$\kf$} between them form a category.
			We will denote this category by~$\cgAlg{\kf}$\glsadd{category graded algebras}\index{category!of graded algebras}.
		\item
			Let~$\Phi$ be a homomorphism of graded~\algebras{$\kf$} from~$A$ to~$B$.
			The following conditions on the homomorphism~$\Phi$ are equivalent.
			\begin{equivalenceslist}
				\item
					The homomorphism~$\Phi$ is an isomorphism\index{isomorphism!of graded algebras} of graded~{\algebras{$\kf$}}, i.e.\ there exists a homomorphism of graded algebras~$\Psi$ from~$B$ to~$A$ such that~$\Psi \circ \Phi = \id_A$ and~$\Phi \circ \Psi = \id_B$.
				\item
					The homomorphism~$\Phi$ is bijective.
				\item
					The restriction~$\Phi_p$ is bijective for every~$p \geq 0$.
			\end{equivalenceslist}
	\end{enumerate}
\end{remark}


\begin{example}
	\leavevmode
	\begin{enumerate}
		\item
			Let~$V$ be a vector space.
			The canonical quotient homomorphisms of algebras
			\begin{alignat*}{2}
				\Tensor(V)
				&\to
				\Symm(V) \,,
				&
				\quad
				x_1 \tensor \dotsb \tensor x_n
				&\mapsto
				x_1 \dotsm x_n
			\shortintertext{and}
				\Tensor(V)
				&\to
				\Exterior(V) \,,
				&
				\quad
				x_1 \tensor \dotsb \tensor x_n
				&\mapsto
				x_1 \wedge \dotsb \wedge x_n
			\end{alignat*}
			are homomorphisms of graded~\algebras{$\kf$}.
		\item
			Let~$V$ be a finite-dimensional vector space with basis~$x_1, \dotsc, x_n$.
			The isomorphism of algebras
			\[
				\kf[X_1, \dotsc, X_n]
				\to
				\Symm(V) \,,
				\quad
				X_i
				\mapsto
				x_i
			\]
			is an isomorphism of graded~\algebras{$\kf$}.
	\end{enumerate}
\end{example}


\begin{lemma}
	\label{characterizations of homogeneous ideals}
	Let~$A$ be a graded~\algebra{$\kf$}.
	Let~$I$ be some kind of ideal of~$A$, i.e. a left ideal, a right ideal, or a two-sided ideal.
	\begin{enumerate}
		\item
			The linear subspace~$\bigoplus_{p \geq 0} {} (I \cap A_p)$ is again an ideal of~$A$ of the same kind as~$I$.
		\item
			The following conditions on the ideal~$I$ are equivalent.
			\begin{enumerate}
				\item
					\label{direct sum of linear subspaces}
					There exists linear subspaces~$I_p$ of~$A_p$ with~$p \geq 0$ such that~$I = \bigoplus_{p \geq 0} I_p$.
				\item
					\label{direct sum of intersections}
					It holds that~$I = \bigoplus_{p \geq 0} {} (I \cap A_p)$.
				\item
					\label{contains all homogeneous components}
					For every element~$x$ of~$I$ all homogeneous components of~$x$ are again contained in~$I$.
				\item
					\label{generated by homogeneous}
					The ideal~$I$ is generated by homogeneous elements.
			\end{enumerate}
	\end{enumerate}
\end{lemma}


\begin{proof}
	\leavevmode
	\begin{enumerate}
		\item
			Let~$I' \defined \bigoplus_{p \geq 0} I \cap A_i$.
			We first check that the linear subspace~$I'$ of~$A$ is a left ideal of~$A$ if~$I$ is a left ideal of~$A$.
			Indeed, we have
			\begin{align*}
				A \cdot I'
				&=
							\Biggl( \sum_{p \geq 0} A_p \Biggr)
				\cdot \Biggl( \sum_{q \geq 0} (I \cap A_q) \Biggr)
				\\
				&=
				\sum_{p, q \geq 0} ( A_p \cdot (I \cap A_q) )
				\\
				&\subseteq
				\sum_{p, q \geq 0} ( (A_p \cdot I) \cap (A_p \cdot A_q) )
				\\
				&\subseteq
				\sum_{p, q \geq 0} ( I \cap A_{p+q} )
				\\
				&=
				\sum_{r \geq 0} ( I \cap A_r )
				\\
				&=
				I'  \,.
			\end{align*}
			We find in the same way that the linear subspace~$I'$ of~$A$ is a right ideal of~$A$ if~$I$ is aright ideal of~$A$.
			It follows from these two cases that~$I'$ is a two-sided ideal of~$A$ if~$I$ is a left-sided ideal of~$A$.
		\item
			\begin{implicationlist}
				\item[\ref*{direct sum of linear subspaces}~$\implies$~\ref*{direct sum of intersections}]
					It follows from the given assumption that~$I_p = I \cap A_p$ for every~$p \geq 0$.
					It follows from these equalities that altogether~$\bigoplus_{p \geq 0} {} (I \cap A_p) = \bigoplus_{p \geq 0} I_p = I$.
				\item[\ref*{direct sum of intersections}~$\implies$~\ref*{direct sum of linear subspaces}]
					We may choose the linear subspace~$I_p$ is~$I \cap A_p$ for every~$p \geq 0$.
				\item[\ref*{direct sum of linear subspaces}~$\implies$~\ref*{contains all homogeneous components}]
					Let~$x$ be an element of~$I$.
					Let~$x = \sum_{p \geq 0} x_p$ be the decomposition of~$x$ into homogeneous components, and let~$x = \sum_{p \geq 0} x'_p$ be the decomposition of~$x$ with respect to the direct sum decomposition~$I = \bigoplus_{p \geq 0} I_p$.
					Each summand~$x'_p$ is contained in~$I_p$ and thus in~$A_p$.
					The summands~$x'_p$ are thus the homogeneous components of~$x$.
					This shows that~$x_p = x_p$ is contained in~$I_p$, and thus in~$I$.
				\item[\ref*{contains all homogeneous components}~$\implies$~\ref*{generated by homogeneous}]
					We may start with any generating set for~$I$ and then replace each generator by all its homogeneous components.
				\item[\ref*{generated by homogeneous}~$\implies$~\ref*{direct sum of intersections}]
					Each homogeneous generator of~$I$ is contained in some intersection~$I \cap A_p$.
					It follows that the linear subspace~$\bigoplus_{p \geq 0} {} (I \cap A_p)$ of~$A$ contains all homogeneous generators of~$A$.
					This subspace is again a ideal of~$A$ (of the same type as the ideal~$I$), whence~$I = \bigoplus_{p \geq 0} {} (I \cap A_p)$.
				\qedhere
			\end{implicationlist}
	\end{enumerate}
\end{proof}


\begin{definition}
	Let~$A$ be a graded~\algebra{$\kf$}.
	An ideal~$I$ of~$A$ (of any kind) is \defemph{homogeneous}\index{homogeneous ideal} if it satisfies the equivalent conditions from \cref{characterizations of homogeneous ideals}.
	If~$I$ is a homogeneous ideal of~$A$, then the intersection~$I \cap A_p$ is denoted by~$I_p$ for every~$p \geq 0$.
\end{definition}


\begin{example}
	Let~$A$ and~$B$ be two graded~\algebras{$\kf$} and let~$\Phi$ be a homomorphism of graded algebras from~$A$ to~$B$.
	Then the kernel of~$\Phi$ is a homogeneous ideal of~$A$, with~$\ker(\Phi)_p = \ker( \Phi_p )$ for all~$p \geq 0$.
\end{example}


\begin{proposition}
	\label{construction of quotients of graded algebras}
	Let~$A$ be a graded~\algebra{$\kf$} and let~$I$ be a two-sided homogeneous ideal of~$A$.
	Let~$\Pi$ be the quotient homomorphism of algebras from~$A$ to~$A/I$.
	\begin{enumerate}
		\item
			The quotient algebra~$A/I$ inherits a grading from~$A$ with homogeneous parts
			\[
				(A/I)_p \defined \Pi(A_p)
				\qquad
				\text{for every~$p \geq 0$.}
			\]
			This grading makes~$A/I$ into a graded~\algebra{$\kf$}.
		\item
			This grading is the unique grading on the algebra~$A/I$ that makes the quotient homomorphism~$\Pi$ into a homomorphism of graded~\algebras{$\kf$}.
	\end{enumerate}
\end{proposition}


\begin{proof}
	\leavevmode
	\begin{enumerate}
		\item
			It follows from the isomorphisms
			\[
				A/I
				=
				\Biggl( \bigoplus_{p \geq 0} A_p \Biggr)
				\bigg/
				\Biggl( \bigoplus_{p \geq 0} I_p \Biggr)
				\cong
				\bigoplus_{p \geq 0} {} (A_p / I_p)
				\cong
				\bigoplus_{p \geq 0} {} (A_p / I_p)
			\]
			that~$A/I = \bigoplus_{p \geq 0} {} (A/I)_p$.
			It also holds for all~$p, q \geq 0$ that
			\[
				(A/I)_p \cdot (A/I)_q
				=
				\Pi(A_p) \cdot \Pi(A_q)
				=
				\Pi(A_p A_q)
				\subseteq
				\Pi( A_{p+q} )
				=
				(A/I)_{p+q} \,.
			\]
		\item
			Any such grading~$A/I = \bigoplus_{p \geq 0} {} (A/I)_p$ must satisfy the condition~$\Pi(A_p) \subseteq (A/I)_p$ for every~$p \geq 0$.
			The resulting chain of inclusions
			\[
				A/I
				=
				\bigoplus_{p \geq 0} \Pi(A_p)
				\subseteq
				\bigoplus_{p \geq 0} {} (A/I)_p
				=
				A/I
			\]
			must already consists of equalities, whence the inclusion~$\Pi(A_p) \subseteq (A/I)_p$ is already an equality for every~$p \geq 0$.
		\qedhere
	\end{enumerate}
\end{proof}


\begin{definition}
	Let~$A$ be a graded~\algebra{$\kf$} and let~$I$ be a homogeneous ideal of~$A$.
	The quotient algebra~$A/I$ together with the induced grading from \cref{construction of quotients of graded algebras} is the \defemph{quotient graded algebra}\index{quotient!of graded Lie algebras} of~$A$ by~$I$.
\end{definition}


\begin{examples}
	Let~$V$ be a vector space.
	The two-sided ideal~$I$ of~$\Tensor(V)$ generated by the elements
	\[
		x \tensor y - y \tensor x
		\qquad
		\text{with~$x, y \in V$}
	\]
	is a homogeneous ideal of~$\Tensor(V)$ because it is generated by homogeneous elements.
	(Namely elements that are homogeneous of degree~$2$.)
	Similarly, the two sided ideal~$J$ of~$\Tensor(V)$ generated by the elements
	\[
		x \tensor x
		\qquad
		\text{with~$x \in V$}
	\]
	is homogeneous because it generated by homogeneous elements.
	The quotient algebra~$A/I$ together with its induced grading is the symmetric algebra~$\Symm(V)$, and the quotient algebra~$A/J$ together with its induced grading is the exterior algebra~$\Exterior(V)$.
\end{examples}


% Grading is on the wrong level.
% 
% \begin{remark}
%   One can also consider graded Lie~algebras, and if~$\glie$ is a graded Lie~algebra then~$\Univ(\glie)$ inherits the structure of a graded~{\algebra{$\kf$}}:
%   \begin{enumerate}
%     \item
%       A \defemph{grading} of a vector space~$V$ is a direct sum decomposition~$V = \bigoplus_{i \geq 0} V_i$.
%       A \defemph{graded vector space} is a vector space~$V$ together with a grading of~$V$.
%     \item
%       A \defemph{grading} of a Lie~algebra~$\glie$ is a direct sum decomposition~$\glie = \bigoplus_{i \geq 0} \glie_i$ such that~$[\glie_i, \glie_j] \subseteq \glie_{i+j}$ for all~$i, j \geq 0$.
%       A \defemph{graded Lie~algebra} is a Lie~algebra~$\glie$ together with a grading of~$\glie$.
%     \item
%       If~$V$ is a graded vector space then the tensor algebra~$\Tensor(V)$ inherits a grading from~$V$:
%       We get for every~$d \geq 0$ a decomposition
%       \[
%         V^{\tensor d}
%         =
%         \left(
%           \bigoplus_{i \geq 0} V_i
%         \right)^{\tensor d}
%         =
%         \bigoplus_{i_1, \dotsc, i_d \geq 0} V_{i_1} \tensor \dotsb \tensor V_{i_d}  \,.
%       \]
%       This overall results for the tensor algebra~$\Tensor(V)$ in a decomposition
%       \[
%         \Tensor(V)
%         =
%         \bigoplus_{r \geq 0}
%         V^{\tensor r}
%         =
%         \bigoplus_{\substack{r \geq 0 \\ i_1, \dotsc, i_r \geq 0}}
%         V_{i_1} \tensor \dotsb \tensor V_{i_r}  \,.
%       \]
%       We define for all~$d \geq 0$ the homogeneous component~$\Tensor(V)_d$ as
%       \[
%         \Tensor(V)_d
%         \defined
%         \bigoplus_{
%           \substack{r \geq 0 \\
%                     i_1, \dotsc, i_r \geq 0 \\
%                     i_1 + \dotsb + i_r = d}
%         }
%         V_{i_1} \tensor \dotsb \tensor V_{i_r}  \,.
%       \]
%       This defines a grading on~$\Tensor(V)$ which makes the inclusion~$V \inclusion \Tensor(V)$ into a homomorphism of graded vector spaces.
%     \item
%       If~$\glie$ is a graded Lie~algebra with grading~$\glie = \bigoplus_{i \neq 0} \glie_i$ then we regard the tensor algebra~$\Tensor(\glie)$ as a graded~{\algebra{$\kf$}} in the above way.
%       Let~$I$ be the two-sided ideal of~$\Tensor(\glie)$ generated by all elements~$c_{x,y} \defined x \tensor y - y \tensor x - [x,y]$ with~$x, y \in \glie$.
%       The ideal~$I$ is already generated by all~$c_{x,y}$ with~$x, y \in \glie$ homogeneous because~$c_{x,y}$ is bilinear in~$x$ and~$y$.
%       The ideal~$I$ is hence graded and so the quotient~$\Univ(\glie) = \Tensor(\glie)/I$ inherits a grading from~$\Tensor(\glie)$.
%       This is the unique grading that makes the canonical homomorphism~$\glie \to \Univ(\glie)$ a homomorphism of graded~{\algebras{$\kf$}}.
%   \end{enumerate}
% \end{remark}




