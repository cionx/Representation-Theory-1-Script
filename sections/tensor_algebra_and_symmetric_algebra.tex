\section{Tensor Algebra and Symmetric Algebra}


% \begin{example}[Monoid algebra]
%   In the following all monoids will be written multiplicaitvely unless otherwise mentioned.
%   The neutral element of a monoid~$M$ will be denoted by~$1$ or~$1_M$.
%   Given two monoids~$M$ and~$N$ a map~$f \colon M \to N$ is a homomorphism of monids if~$f(m \cdot m') = f(m) \cdot f(m')$ for all~$m, m' \in M$ and~$f(1_M) = 1_N$.
%   If~$M$ is any monoid then the identity~$\id_M$ is a homomorphism and if~$f \colon M \to N$ and~$g \colon N \to P$ are composable homomorphisms of monoids then their composition~$g \circ f \colon M \to P$ is again a homomorphism of monoids.
%   The resulting category of monoids is denoted by~$\cMon$.
%   
%   \begin{description}
%     \item[Construction:]
%       If~$M$ is a monoid then the monoid algebra~\gls*{monoid algebra} is the (free) vector space with basis~$M$ together with the unique bilinear extension~$\kf[M] \times \kf[M] \to \kf[M]$ of the multiplication~$M \times M \to M$ as its multiplication.
%   
%       This means that the elements of~$\kf[M]$ are formal {\linear{$\kf$}} combinations~$\sum_{m \in M} a_m m$ with~$a_m = 0$ for all but finitely many~$m \in M$.
%       The multiplication of two such elements is given by
%       \[
%         \left(
%           \sum_{m \in M} a_m m
%         \right)
%         \left(
%           \sum_{n \in M} b_n n
%         \right)
%         =
%         \sum_{m, n \in M} (a_m b_n) m n  \,.
%       \]
%       We identify every element~$m \in M$ with the corresponding element~$1 \cdot m \in \kf[M]$.
%       The product~$m \cdot n$ of two elements~$m, n \in M$ in~$\kf[M]$ is then the same as their product in~$M$.
%       The associativity of the multiplication of~$\kf[M]$ follows from the associativity of the multiplication of~$M$, and the neutral element of~$M$ is given by the multiplicative neutral element for~$\kf[M]$.
%       
%     \item[Universal Property:]
%       If~$A$ is any~{\algebra{$\kf$}} then~$(A, \cdot)$ is a multiplicative monoid, which we will denote by~$A^-$.
%       If~$M$ is any monoid and~$f \colon M \to A^-$ is a monoid hommorphism then~$f$ extends uniquely to an algebra homomorphism~$F \colon \kf[M] \to A$.
%       The algebra homomorphism~$F$ is given on elements by
%       \[
%         F\left( \sum_{m \in M} a_m m \right)
%         =
%         \sum_{m \in M} a_m f(m) \,.
%       \]
%       On the other hand every algebra homomorphism~$\kf[M] \to A$ restricts to a monoid homomorphism~$M \to A^-$.
%       This construction results in a {\onetoonetext} correspondence
%       \[
%         \{
%           \text{monoid homomorphisms~$M \to A^-$}
%         \}
%         \onetoone
%         \{
%           \text{algebra homomorphisms~$\kf[M] \to A$}
%         \}  \,.
%       \]
%     
%     \item[Uniqueness:]
%       The monoid algebra~$\kf[M]$ together with the inclusion~$i \colon M \to \kf[M]$ is uniquely determined by its universal property up to isomorphism:
%   \end{description}
% \end{example}


% \begin{recall}[Free algebra]
%   Let~$I$ be any set.
%   The \defemph{noncommutative polynomial algebra}~$\kf\gen{X_i \suchthat i \in I}$\index{noncommutative polynomial algebra} has as a basis the set of all monomials
%   \[
%     X_{i_1} \dotsm X_{i_n}
%     \qquad
%     \text{with~$i_1, \dotsc, i_n$}
%   \]
%   and the multiplication is on these basis elements given by
%   \[
%     X_{i_1} \dotsm X_{i_n}
%     \cdot
%     X_{j_1} \dotsm X_{j_m}
%     =
%     X_{i_1} \dotsm X_{i_n} X_{j_1} \dotsm X_{j_m} \,.
%   \]
%   In contrast to the usual (commutative) polynomial algebra~$\kf[X_i \suchthat i \in I]$ the variables~$X_i$ are not required to commute with each other.
%   
%   We can alternatively construct~$\kf\gen{X_i \suchthat i \in I}$ as the monomial algebra of the free monoid on~$I$:
%   Let~$M$ be the set of all words in~$I$, i.e.\ the set of all finite sequences
%   \[
%     (i_1, \dotsc, i_n)
%     \qquad
%     \text{with~$i_1, \dotsc, i_n \in I$}  \,.
%   \]
%   Then~$M$ is a monoid with respect to concatenation of words given by
%   \[
%     (i_1, \dotsc, i_n) (j_1, \dotsc, j_m)
%     =
%     (i_1, \dotsc, i_n, j_1, \dotsc, j_m)
%   \]
%   for all words~$(i_1, \dotsc, i_n), (j_1, \dotsc, j_m) \in M$.
%   The neutral element of~$M$ is given by the empty word~$()$.
% \end{recall}





\subsection{Reviewing the Tensor Algebra}


\begin{recall}[Tensor algebra]
  Let~$V$ be a vector space.
  \begin{description}
    \item[Construction]
      For all vectors~$v_1, \dotsc, v_d$ of~$V$ we denote the resulting simple tensor~$v_1 \tensor \dotsb \tensor v_d$ in~$V^{\tensor d}$ by~$(v_1, \dotsc, v_d)$.
      Observe that for~$d = 0$ the tensor power~$V^{\tensor d} = V^{\tensor 0}$ has as a basis the empty simple tensor~$()$.
      We will therefore identify the tensor power~$V^{\tensor 0}$ with the ground field~$\kf$, so that empty simple tensor~$()$ corresponds to the element~$1$ of~$\kf$.
      
      For all~$p, q \geq 0$ we define a multiplication
      \[
        \mu_{p,q}
        \colon
        V^{\tensor p} \times V^{\tensor q}
        \to
        V^{\tensor (p+q)} \,,
        \quad
        (x,y)
        \mapsto
        x y
      \]
      that is on simple tensors~$(v_1, \dotsc, v_p)$ and~$(v_{p+1}, \dotsc, v_{p+q})$ given by
      \[
        (v_1, \dotsc, v_p) \cdot (v_{p+1}, \dotsc, v_{p+q})
        \defined
        (v_1, \dotsc, v_{p+q})  \,.
      \]
      We note that for~$p = 0$ or~$q = 0$ this multiplication is just the scalar multiplication.  
      These multiplications fit together associatively in the sense that we have for all~$p, q, r \geq 0$ and all simple tensors~$x \in V^{\tensor p}$,~$y \in V^{\tensor q}$ and~$z \in V^{\tensor r}$ the equality
      \[
        x \cdot (y \cdot z)
        =
        (x \cdot y) \cdot z \,.
      \]
      
      Let~$\Tensor(V) \defined \bigoplus_{d \geq 0} V^{\tensor d}$.
      We can fit together the multiplications~$\mu_{p,q}$ with~$p, q \geq 0$ to a single multiplication
      \[
        \mu
        \colon
        \Tensor(V) \times \Tensor(V)
        \to
        \Tensor(V)  \,,
        \quad
        (x,y)
        \mapsto
        xy 
      \]
      that is given on elements~$x$,~$y$ of~$\Tensor(V)$ with~$x = (x_d)_{d \geq 0}$ and~$y = (y_d)_{d \geq 0}$ by
      \[
        x y
        =
        \left(
          \sum_{p+q = d} x_p y_q
        \right)_{d \geq 0} \,.
      \]
      This multiplication is built precisely so that it follows from the bilinearity of the multiplications~$\mu_{p,q}$ that the multipliation~$\mu$ is again bilinear.
      It follows from the associativities of the multiplications~$\mu_{p,q}$ that the multiplication~$\mu$ is associative.
      We may identify the ground field~$\kf$ with the zeroth tensor power~$V^{\tensor 0}$ and then with the corresponding direct summand of~$\Tensor(V)$.
      In this way we regard~$\kf$ as a linear subspace of~$\Tensor(V)$.
      We have seen above that the element~$1$ of~$\kf$ is then unital for the multiplication of~$\Tensor(V)$.
      We have thus altogether constructed a~{\algebra{$\kf$}}~$\Tensor(V)$.
      
      We may identify the vector space~$V$ with the first tensor power~$V^{\otimes 1}$ and then with the corresponding direct summand of~$\Tensor(V)$.
      In this way we regard the vector space~$V$ as a linear subspace of~$\Tensor(V)$.
      We then have for all elements~$v_1, \dotsc, v_n$ of~$V$ that
      \[
        v_1 \dotsm v_n
        =
        (v_1) \dotsm (v_n)
        =
        (v_1, \dotsc, v_n)
        =
        v_1 \tensor \dotsb \tensor v_n  \,.
      \]
      It follows in particular that~$\Tensor(V)$ is generated by the vector space~$V$ as an algebra.
      The algebra~$\Tensor(V)$\glsadd{tensor algebra} is the \defemph{tensor algebra of~$V$}
      
      We will more generally identify for every natural number~$d$ the tensor power~$V^{\tensor d}$ with the corresponding direct summand of~$\Tensor(V)$.
      Every element of the tensor algebra~$\Tensor(V)$ is then a linear combination of simple tensors.
    
    \item[Universal Property]
      The tensor algebra~$\Tensor(V)$ can be though of as the \enquote{free~{\algebra{$\kf$}} on~$V$}, in at least two ways:
      \begin{description*}
        \item[Informal]
          The tensor algebra~$\Tensor(V)$ arises from the vector space~$V$ by starting with the elements of~$V$ and then adding all kinds of expressions that can be constructed from the elements of~$V$ by algebra operations.
          But it follows from the axioms of a~\algebra{$kf$} that many of these expressions have to be the same, so that we only end up with expressions of a certain form.
          
          Let us be a bit more explicit:
          Suppose that a~{\algebra{$\kf$}}~$A$ contains the vector space~$V$ as a linear subspace.
          Then it also contains products of the form~$v_1 \dotsm v_n$ with~$v_1, \dotsc, v_n \in V$ and hence sums of such products, i.e. elements of the form
          \[
            \sum_{i=1}^r v_{1i} \dotsm, v_{n_i i}
          \]
          for natural numbers~$r, n_1, \dotsc, n_r$ and vectors~$v_{ij}$ in~$V$.
          If we continue to combine elements of this form with algebra operations then we do not gain any new elements, since by the axioms of a~\algebra{$\kf$} they must already be of the above form.
          
          But in an arbitrary~{\algebra{$\kf$}} it may happen that some of these expressions are equal even though this does not follow pureley from the axioms of a~{\algebra{$\kf$}}.
          Consider for example the polynomial ring~$A = \kf[x, y]$ and the linear subspace~$V = \gen{x, y}_{\kf}$.
          It follows from the axioms of a~{\algebra{$\kf$}} that the two elements~$x (x+y)$ and~$x^2 + xy$ of~$A$ are equal.
          But it does not follow from these axioms that~$xy = yx$, even though this holds in~$A$.
          There are hence certain additional \emph{relations} between the elements~$x$ and~$y$ of~$V$ in the ambient {\algebra{$\kf$}}~$A$.
          
          In the tensor algebra~$\Tensor(V)$ this does not happen:
          If two elements of~$A$ built from elements of~$V$ via algebra operations are equal, then this equality can be derived from the algebra axioms alone.
          Hence there exist no additional relations between the elements of~$V$ in~$\Tensor(V)$.

          The only required condition is that~$V$ is a linear subspace of~$\Tensor(V)$, i.e.\ that addition and scalar multiplication in~$V$ does coincide with the one coming from~$\Tensor(V)$.
          
          The tensor algebra~$\Tensor(V)$ is in this way the \enquote{freest} way of expanding the vector space~$V$ into a~{\algebra{$\kf$}}.
        \item[Formal]
          Let~$i$ be the inclusion map from~$V$ to~$\Tensor(V)$.
          This map is~{\linear{$\kf$}}.
          If~$A$ is a~{\algebra{$\kf$}} and~$f$ is a linear map from~$V$ to$~A$ any~{\linear{$\kf$}} map then~$f$ extends uniquely to an algebra homomorphism~$f^+$ from~$\Tensor(V)$ to~$A$, i.e.\ there exists a unique algebra homomorphism~$f^+$ from~$\Tensor(V)$ to~$A$ that makes the triangular diagram
          \[
            \begin{tikzcd}
              V
              \arrow{r}[above]{f}
              \arrow{d}[left]{i}
              &
              A
              \\
              \Tensor(V)
              \arrow[dashed]{ur}[below right]{f^+}
              &
              {}
            \end{tikzcd}
          \]
          commute.
          The algebra homomorphism~$f^+$ is given by
          \[
            f^+(v_1 \tensor \dotsb \tensor v_d)
            =
            f(v_1) \dotsm f(v_d)
          \]
          for all~$d \geq 0$,~$v_1, \dotsc, v_n \in V$.
          This construction results in a {\onetoonetext} correspondence
          \begin{align*}
            \left\{
              \begin{tabular}{@{}c@{}}
                \linear{$\kf$} maps \\
                $f \colon V \to A$
              \end{tabular}
          \right\}
            &\onetoone
            \left\{
              \begin{tabular}{@{}c@{}}
                algebra homomorphisms \\
                $\Phi \colon \Tensor(V) \to A$
              \end{tabular}
            \right\} \,,
            \\
            f
            &\mapsto
            f^+ \,,
            \\
            \restrict{\Phi}{V}
            &\mapsfrom
            \Phi \,.
          \end{align*}
          The tensor algebra~$\Tensor(V)$ together with the linear map~$i$ from~$V$ to~$\Tensor(V)$ is in this sense the \enquote{universal way} of assigning a~{\algebra{$\kf$}} to the vector space~$V$.
          
          This formal explanation relates to the previous informal explanation in the following way:
          If~$A$ is any~{\algebra{$\kf$}} that contains~$V$ as a linear subspace then the inclusion map from~$V$ to~$A$ extend uniquely to an algebra homomorphism from~$\Tensor(V)$ to~$A$.
          Every relation between expressions built from the elements of~$V$ that holds in~$\Tensor(V)$ must then also hold in~$A$.
          The relations that hold in~$\Tensor(V)$ are the therefore precisely those relations which hold in \emph{every}~{\algebra{$\kf$}} containing~$V$.
      \end{description*}
      
    \item[Uniqueness]
      The above universal property determines the tensor algebra~$\Tensor(V)$ together with the inclusion map~$i$ from~$V$ to~$\Tensor(V)$ uniquely up to unique isomorphism, in the following sense.

      Let~$T$ be another~{\algebra{$\kf$}} and let~$j$ be a linear map from~$V$ to~$T$.
      Suppose that for every~{\algebra{$\kf$}}~$A$ and every~{\linear{$\kf$}} map~$f$ from~$V$ to~$A$ there exists a unique algebra homomorphism~$\Phi$ from~$T$ to~$A$ that makes the triangular diagram
      \[
        \begin{tikzcd}
          V
          \arrow{r}[above]{f}
          \arrow{d}[left]{j}
          &
          A
          \\
          T
          \arrow{ur}[below right]{\Phi}
          &
          {}
        \end{tikzcd}
      \]
      commute.
      Then there exist unique algebra homomorphisms~$\Phi$ from~$A$ to~$T$ and~$\Psi$ from~$T$ to~$A$ that make the triangular diagrams
      \[
        \begin{tikzcd}[column sep = {3em,between origins}]
          {}
          &
          V
          \arrow{dl}[above left]{i}
          \arrow{dr}[above right]{j}
          &
          {}
          \\
          \Tensor(V)
          \arrow[dashed]{rr}[below]{\Phi}
          &
          {}
          &
          T
        \end{tikzcd}
        \qquad\text{and}\qquad
        \begin{tikzcd}[column sep = {3em,between origins}]
          {}
          &
          V
          \arrow{dl}[above left]{j}
          \arrow{dr}[above right]{i}
          &
          {}
          \\
          T
          \arrow[dashed]{rr}[below]{\Psi}
          &
          {}
          &
          \Tensor(V)
        \end{tikzcd}
      \]
      commute.
      It then follows that the composites~$\Psi \circ \Phi$ and~$\Phi \circ \Psi$ make the triangular diagrams
      \[
        \begin{tikzcd}[column sep = small]
          {}
          &
          V
          \arrow{dl}[above left]{i}
          \arrow{dr}[above right]{i}
          &
          {}
          \\
          \Tensor(V)
          \arrow[dashed]{rr}[below]{\Psi \circ \Phi}
          &
          {}
          &
          \Tensor(V)
        \end{tikzcd}
        \qquad\text{and}\qquad
        \begin{tikzcd}[column sep = small]
          {}
          &
          V
          \arrow{dl}[above left]{j}
          \arrow{dr}[above right]{j}
          &
          {}
          \\
          T
          \arrow[dashed]{rr}[below]{\Phi \circ \Psi}
          &
          {}
          &
          T
        \end{tikzcd}
      \]
      commute.
      The algebra homomorphisms~$\Psi \circ \Phi$ and~$\Phi \circ \Psi$ are unique with this propert by the universal properties of~$(\Tensor(V), i)$ and~$(T, j)$.
      But the identities~$\id_{\Tensor(V)}$ and~$\id_T$ also make these diagrams commute.
      We therefore find that~$\Psi \circ \Phi = \id_{\Tensor(V)}$ and~$\Phi \circ \Psi = \id_{T}$.
      The homomorphisms~$\Phi$ and~$\Psi$ are therefore mutually inverse isomorphisms of algebras.
    
    \item[Functoriality]
      Let~$V$ and~$W$ be two vector spaces and let~$f$ be a~linear{$\kf$} map from~$V$ to~$W$.
      We can consider the following diagram.
      \[
        \begin{tikzcd}
          V
          \arrow{r}[above]{f}
          \arrow{d}[left]{i_V}
          &
          W
          \arrow{d}[right]{i_W}
          \\
          \Tensor(V)
          &
          \Tensor(W)
        \end{tikzcd}
      \]
      By applying the universal property of the tensor algebra~$\Tensor(V)$ to the composition~$i_W \circ f$ it follows that there exists a unique algebra homomorphism~$\Tensor(f)$ from~$\Tensor(V)$ to~$\Tensor(W)$ that makes the square diagram
      \[
        \begin{tikzcd}[sep = large]
          V
          \arrow{r}[above]{f}
          \arrow{d}[left]{i_V}
          \arrow[dashed]{dr}[above right]{i_W \circ f}
          &
          W
          \arrow{d}[right]{i_W}
          \\
          \Tensor(V)
          \arrow[dashed]{r}[below]{\Tensor(f)}
          &
          \Tensor(W)
        \end{tikzcd}
      \]
      commute.
      This induced algebra homorphism is functorial in the following sense:
      \begin{itemize}
        \item
          It holds that~$\Tensor(\id_V) = \id_{\Tensor(V)}$.
          Indeed, the commutativity of the square 
          \[
            \begin{tikzcd}[column sep = large]
              V
              \arrow{r}[above]{f}
              \arrow{d}
              &
              V
              \arrow{d}
              \\
              \Tensor(V)
              \arrow[dashed]{r}[below]{\Tensor(\id_V)}
              &
              \Tensor(V)
            \end{tikzcd}
          \]
          shows that the identity~$\id_{\Tensor(V)}$ satifies the defining property of the induced algebra homomorphism~$\Tensor(f)$.
        \item
          Let~$U$,~$V$,~$W$ be three vector spaces.
          Let~$f$ be a linear map from~$U$ to~$V$ and let~$g$ be a linear map from~$V$ to~$W$.
          Then
          \[
            \Tensor(g \circ f)
            =
            \Tensor(g) \circ \Tensor(f) \,.
          \]
          Indeed, it follows from the commutativity of the diagram
          \[
            \begin{tikzcd}[column sep = large]
              U
              \arrow[dashed, bend left=45]{rr}[above]{g \circ f}
              \arrow{r}[above]{f}
              \arrow{d}
              &
              V
              \arrow{r}[above]{g}
              \arrow{d}
              &
              W
              \arrow{d}
              \\
              \Tensor(U)
              \arrow{r}[below]{\Tensor(f)}
              \arrow[dashed, bend right=45]{rr}[below]{\Tensor(g) \circ \Tensor(f)}
              &
              \Tensor(V)
              \arrow{r}[below]{\Tensor(g)}
              &
              \Tensor(W)
            \end{tikzcd}
          \]
          that the subdiagram
          \[
            \begin{tikzcd}[column sep = huge]
              U
              \arrow{r}[above]{g \circ f}
              \arrow{d}
              &
              W
              \arrow{d}
              \\
              \Tensor(U)
              \arrow[dashed]{r}[below]{\Tensor(g) \circ \Tensor(f)}
              &
              \Tensor(W)
            \end{tikzcd}
          \]
          commutes.
          This shows that the composition~$\Tensor(g) \circ \Tensor(f)$ satisfies the defining property of the induced algebra homomorphism~$\Tensor(g \circ f)$.
      \end{itemize}
      
      We see from the above discussion that the assignment~$\Tensor$ defines a functor from~$\cVect{\kf}$ to~$\cAlg{\kf}$.
      The universal property of the tensor algebra states that the functor~$\Tensor$ is left adjoint to the forgetful functor from~$\cAlg{\kf}$ to~$\cVect{\kf}$ which assigns to each~{\algebra{$\kf$}} its underlying~{\vectorspace{$\kf$}}.
    
    \item[Description via a basis]
      Let~$(v_i)_{i \in I}$ be a basis of~$V$
      Then for every natural number~$d$ the tensor power~$V^{\tensor d}$ inherits a basis given by all simple tensors
      \[
        v_{i_1} \tensor \dotsb \tensor v_{i_d}
      \]
      with~$i_1, \dotsc, i_d \in I$.
      It follows that the tensor algebra has as a basis of all such simple tensors with~$d \geq 0$ and~$i_1, \dotsc, i_d \in I$.
      The product of two such basis vectors is again a basis vector.
      We may think about such a basis vector as finite word~$i_1 \dotso i_d$ in the alphabet~$I$, and about the multiplication of two basis vectors as the concatenation of words.
      
      If we think about the basis vector~$v_i$ of~$V$ as a formal variable~$X_i$ then we see that the tensor algebra~$\Tensor(V)$ is isomorphic to the noncommutative polynomial algebra~$\kf\gen{X_i \suchthat i \in I}$.

      This can also seen via adjunctions.
      Indeed, we have the following commutative diagram of forgetful functors.
      \[
        \begin{tikzcd}
          \cVect{\kf}
          \arrow{d}
          &
          \cAlg{\kf}
          \arrow{l}
          \arrow{dl}
          \\
          \cSet
          &
          {}
        \end{tikzcd}
      \]
      It follow that the resulting diagram of left-adjoint functors
      \[
        \begin{tikzcd}
          \cVect{\kf}
          \arrow{r}[above]{\Tensor}
          &
          \cAlg{\kf}
          \\
          \cSet
          \arrow{u}[left]{F}
          \arrow{ur}[below right]{\kf\gen{X_i \suchthat i \in (-)}}
          &
          {}
        \end{tikzcd}
      \]
      commutes up to natural isomorphism.
      The vector space~$V$ has a basis indexed by~$I$ and is therefore isomorphic to free vector space~$F(I)$.
      Hence
      \[
        \Tensor(V)
        \cong
        \Tensor(F(I))
        \cong
        \kf\gen{X_i \suchthat i \in I} \,.
      \]
  \end{description}
\end{recall}





\subsection{Reviewing the Symmetric Algebra}


\begin{recall}[Symmetric power]
  Let~$V$ be a vector space and let~$d$ be a natural number.
  The~{\howmanyth{$d$}} \defemph{symmetric power}\index{symmetric!power}~$\Symm^d(V)$ is the quotient vector space of the tensor power~$\Tensor^d(V)$ by the~{\linear{$\kf$}} subspace~$U_d$ which is generated by all all differences of the form
  \[
      v_1 \tensor \dotsb \tensor v_d
    - v_{\sigma(1)} \tensor \dotsb \tensor v_{\sigma(d)}
  \]
  where~$v_1, \dotsc, v_d$ are vectors in~$V$ and~$\sigma$ is a permutaiton.
  Hence
  \begin{align*}
    \Symm^d(V)
    &=
    V^{\tensor d} / U_d
    \\
    &=
    V^{\tensor d}
    /
    \gen{
        v_1 \tensor \dotsb \tensor v_d
      - v_{\sigma(1)} \tensor \dotsb \tensor v_{\sigma(d)} 
    \suchthat
      v_1, \dotsc, v_n \in V,
      \sigma \in \symm_n
    }_{\kf} \,.
  \end{align*}
  We can identify~$\Symm^0(V)$ with~$V^{\tensor 0}$ and thus with~$\kf$ because~$U_0$ vanishes.
  For all vectors~$v_1, \dotsc, v_n$ in~$V$ we denote the residue class of the simple tensor~$v_1 \tensor \dotsb \tensor v_d$ in~$\Symm^d(V)$ by~$v_1 \dotsm v_d$, and call this a \defemph{simple symmetric tensor}\index{simple symmetric tensor}.
  
  We have by construction of~$\Symm^d(V)$ that
  \[
    v_1 \dotsm v_d
    =
    v_{\sigma(1)} \dotsm v_{\sigma(d)}
  \]
  for all~$v_1, \dotsc, v_n \in V$,~$\sigma \in \symm_d$.

  The symmetric power~$\Symm^d(V)$ is in the following sense universal with this property.
  The map
  \[
    \alpha
    \colon
    V^{\times d}
    \to
    \Symm^d(V)  \,,
    \quad
    (v_1, \dotsc, v_d)
    \mapsto
    v_1 \dotsm v_d
  \]
  is symmetric and multilinear, and if~$\beta$ is any symmetric, multilinear map from~$V^{\times d}$ to another vector space~$W$ then there exists a unique linear map~$f$ from~$\Symm^d(V)$ to~$W$ that makes the triangular diagram
  \[
    \begin{tikzcd}
      V^{\times d}
      \arrow{d}[left]{\alpha}
      \arrow{dr}[above right]{\beta}
      &
      {}
      \\
      \Symm^d(V)
      \arrow[dashed]{r}[below]{f}
      &
      W
    \end{tikzcd}
  \]
  commute.
  A linear map from~$\Symm^d(V)$ to~$W$ is in this sense the same as a symmetric, multilinear map from~$V^{\times d}$ to~$W$.
  
  If~$(v_i)_{i \in I}$ is a basis of~$V$ such that~$(I, \leq)$ is a linearly ordered set then the ordered monomials
  \[
    v_{i_1} \dotsm v_{i_d}
    \qquad
    \text{with~$i_1 \leq \dotsb \leq i_d$}
  \]
  form a basis of the symmetric power~$\Symm^d(V)$.
  If~$V$ is of finite dimension~$n$ then it follows that
  \[
    \dim \Symm^d(V)
    =
    \binom{n+d-1}{d}  \,.
  \]
\end{recall}


\begin{recall}[Symmetric algebra]
  Let~$V$ be a vector space.
  Just as the tensor powers~$V^{\otimes d}$ can be used to construct the tensor algebra~$\Tensor(V)$, we can also use the symmetric powers~$\Symm^d(V)$ to construct the \defemph{symmetric algebra}\index{symmetric algebra}~$\Symm(V)$\glsadd{symmetric algebra}.
  Just as the tensor algebra~$\Tensor(V)$ is the free~\algebra{$\kf$} on~$V$, the symmetric algebra~$\Symm(V)$ is the free symmetric algebra on~$V$.

  The argumentation for the symmetric algebra is analogous to that for the tensor algebra, so we will skip some of the details.
  
  \begin{description}
    \item[Construction]
      For all vectors~$v_1, \dotsc, v_d$ in~$V$ we denote the corresponding simple symmetric tensor in~$\Symm^d(V)$ by~$v_1 \dotsm v_d$.
      We can define on the direct sum~$\Symm(V) \defined \bigoplus_{d \geq 0} \Symm^d(V)$ a multiplication such that
      \[
        (v_1 \dotsm v_p) \cdot (v_{p+1} \dotsm v_{p+q})
        =
        v_1 \dotsm v_p v_{p+1} \dotsm v_{p+q}
      \]
      for all~$p, q \geq 0$ and all~$v_1, \dotsc, v_{p+q} \in V$.
      By identifying the zeroth symmetric power~$\Symm^0(V)$ with the ground field~$\kf$ we make~$\Symm(V)$ into an associative~{\algebra{$\kf$}}.
      This also is commutative because
      \begin{align*}
        (v_1 \dotsm v_p) \cdot (v_{p+1} \dotsm v_{p+q})
        &=
        v_1 \dotsm v_p v_{p+1} \dotsm v_{p+q}
        \\
        &=
        v_{p+1} \dotsm v_{p+q} v_1 \dotsm v_p
        \\
        &=
        (v_{p+1} \dotsm v_{p+q}) \cdot (v_1 \dotsm v_p)
      \end{align*}
      for all~$p, q \geq 0$ and all~$v_1, \dotsc, v_{p+q} \in V$. 
      We can identify the vector space~$V$ with the first symmetric power~$\Symm^1(V)$ and then with the corresponding direct summand of~$\Symm(V)$.
      We can more generally idenfity every symmetric power~$\Symm^d(V)$ with the corresponding direct summand of~$\Symm(V)$.
      Every element of the symmetric algebra~$\Symm(V)$ is then a a linear combination of simple symmetric tensors.
      
    \item[Universal property]
      The symmetric algebra~$\Symm(V)$ is in the following sense the \enquote{free commutative~{\algebra{$\kf$}}} on the vector space~$V$.

      Let~$i$ be the inclusion map from~$V$ to~$\Symm(V)$.
      There exists for every symmetric~{\algebra{$\kf$}}~$A$ and every linear map~$f$ from~$V$ to~$A$ a unique algebra homomorphism~$f^+$ from~$\Symm(V)$ to~$A$ that makes the triangular diagram
      \[
        \begin{tikzcd}
          V
          \arrow{r}[above]{f}
          \arrow{d}[left]{i}
          &
          A
          \\
          \Symm(V)
          \arrow{ur}[below right]{f^+}
          &
          {}
        \end{tikzcd}
      \]
      commute.
      The algebra homomorphism~$f^+$ is given by
      \[
        f^+(v_1 \dotsm v_d)
        =
        f(v_1) \dotsm f(v_d)
      \]
      for all~$d \geq 0$,~$v_1, \dotsc, v_d \in V$.
      This construction results in a {\onetoonetext} correspondence
      \begin{align*}
        \left\{
          \begin{tabular}{@{}c@{}}
            \linear{$\kf$} maps \\
            $f \colon V \to A$
          \end{tabular}
      \right\}
        &\onetoone
        \left\{
          \begin{tabular}{@{}c@{}}
            algebra homomorphisms \\
            $\Phi \colon \Symm(V) \to A$
          \end{tabular}
        \right\} \,,
        \\
        f
        &\mapsto
        f^+ \,,
        \\
        \restrict{\Phi}{V}
        &\mapsfrom
        \Phi \,.
      \end{align*}
      
      It follows that a relations between elements of~$V$ holds in the symmetric algebra~$\Symm(V)$ if and only if it holds in every commutative algebra which contains~$V$.
      
    \item[Uniqueness]
      Let~$S$ be a commutative~{\algebra{$\kf$}} and let~$j$ be a linear map from~$V$ to~$S$.
      Suppose that the pair~$(S, j)$ satisfies the same universal property as the symmetric algebra~$(\Symm(V), i)$.
      Then there exists a unique algebra homomorphisms~$\Phi$ from~$\Symm(V)$ and a unique algebra homomorphism~$\Psi$ from~$S$ to~$\Symm(V)$ that make the triangular diagrams
      \[
        \begin{tikzcd}[column sep = {3em,between origins}]
          {}
          &
          V
          \arrow{dl}[above left]{i}
          \arrow{dr}[above right]{j}
          &
          {}
          \\
          \Symm(V)
          \arrow[dashed]{rr}[below]{\Phi}
          &
          {}
          &
          S
        \end{tikzcd}
        \qquad\text{and}\qquad
        \begin{tikzcd}[column sep = {3em,between origins}]
          {}
          &
          V
          \arrow{dl}[above left]{j}
          \arrow{dr}[above right]{i}
          &
          {}
          \\
          S
          \arrow[dashed]{rr}[below]{\Psi}
          &
          {}
          &
          \Symm(V)
        \end{tikzcd}
      \]
      commute.
      Thes homomorphisms~$\Phi$ and~$\Psi$ are mutually inverse isomorphisms of algebras.
      
    \item[Functoriality]
      Let~$V$ and~$W$ be two vector spaces and let~$f$ be a linear map from~$V$ to~$W$.
      Then there exists a unique homomorphism of algebras~$\Symm(f)$ from~$\Symm(V)$ to~$\Symm(W)$ which makes the square diagram
      \[
        \begin{tikzcd}
          V
          \arrow{r}[above]{f}
          \arrow{d}
          &
          W
          \arrow{d}
          \\
          \Symm(V)
          \arrow[dashed]{r}[below]{\Symm(f)}
          &
          \Symm(W)
        \end{tikzcd}
      \]
      commmute.
      It holds that~$\Symm(\id_V) = \id_{\Symm(V)}$ and it holds for all composable~{\linear{$\kf$}} maps~$f$ from~$U$ to~$V$ and~$g$ from~$V$ to~$W$ that~$\Symm(g \circ f) = \Symm(g) \circ \Symm(f)$.
      The assignment~$\Symm$ is thus a functor from~$\cVect{\kf}$ to~$\cCAlg{\kf}$, where~$\cCAlg{\kf}$ denotes the category of commutative~{\algebras{$\kf$}}.
      
    \item[Description via a basis]
      Let~$(v_i)_{i \in I}$ be a basis of$~V$ where~$(I, \leq)$ is a linearly ordered set.
      Then the symmetric power~$\Symm^d(V)$ inherits a basis given by all simple symmetric tensors
      \[
        v_{i_1} \dotsm v_{i_d}
        \qquad
        \text{where~$i_1 \leq \dotsb \leq i_d$} \,.
      \]
      It follows that the symmetric algebra~$\Symm(V)$ has as a basis all simple symmetric tensors~$v_{i_1} \dotsm v_{i_d}$ with~$d \geq 0$ and~$i_1, \dotsc, i_d \in I$ such that~$i_1 \leq \dotsb \leq i_d$.
      This basis may also be written as
      \[
        v_{i_1}^{n_1} \dotsm v_{i_r}^{n_r}
      \]
      with~$r \geq 0$,~$i_1, \dotsc, i_r \in I$ such that~$i_1 < \dotsb < i_r$ and~$n_1, \dotsc, n_r \geq 0$ (which is connected to the above description via~$d = n_1 + \dotsb + n_r$).
      
      We see from this description that the symmetric algebra~$\Symm(V)$ is isomorphic to the commutative polynomial algebra~$\kf[X_i \suchthat i \in I]$, which is the free commutative~{\algebra{$\kf$}} on the generators~$X_i$ with~$i$ in~$I$
      This can again be explained by considering the commutative diagram of forgetful functors
      \[
        \begin{tikzcd}
          \cVect{\kf}
          \arrow{d}
          &
          \cCAlg{\kf}
          \arrow{l}
          \arrow{dl}
          \\
          \cSet
          &
          {}
        \end{tikzcd}
      \]
      from which we see that the resulting diagram of left adjoints
      \[
        \begin{tikzcd}
          \cVect{\kf}
          \arrow{r}[above]{\Symm}
          &
          \cCAlg{\kf}
          \\
          \cSet
          \arrow{u}[left]{F}
          \arrow{ur}[below right]{\kf[X_i \suchthat i \in (-)]}
          &
          {}
        \end{tikzcd}
      \]
      commutes up to natural isomorphism.
      
    \item[Contruction via the tensor algebra]
      The symmetric algebra~$\Symm(V)$ can also be constructed as a quotient of the tensor algebra~$\Tensor(V)$.
      We give multiple ways how to see and think about this.
      Let in the following~$i$ denote the inclusion map from~$V$ to~$\Tensor(V)$ and let~$j$ denote the inclusion map from~$V$ to~$\Symm(V)$.
      \begin{itemize}
        \item
          Let~$I$ be the commutator ideal of~$\Tensor(V)$, i.e. the two-sided ideal generated by all elements of the form
          \[
            x \tensor y - y \tensor x
          \]
          where~$x$ and~$y$ are elements of~$\Tensor(V)$.
          Let~$\Pi$ denote the canonical quotient homomorphism from~$\Tensor(V)$ to~$\Tensor(V)/I$.
          The quotient algebra~$\Tensor(V)/I$ is commutative, whence there exists by the universal property of the symmetric algebra~$\Symm(V)$ a unique algebra homomorphism~$\Phi$ from~$\Symm(V)$ to~$\Tensor(V)/I$ that makes the diagram
          \[
            \begin{tikzcd}
              {}
              &
              V
              \arrow[bend right]{ddl}[above left]{i}
              \arrow[bend left]{dr}[above right]{j}
              &
              {}
              \\
              {}
              &
              {}
              &
              \Tensor(V)
              \arrow{d}[right]{\Pi}
              \\
              \Symm(V)
              \arrow[dashed]{rr}[above]{\Phi}
              &
              {}
              &
              \Tensor(V)/I
            \end{tikzcd}
          \]
          commute.
          The homomorphism~$\Phi$ is on the generating set~$V$ of~$\Symm(V)$ given by~$\Phi(v) = \class{v}$ for all~$v \in V$.

          We get from the universal property of the tensor algebra~$\Tensor(V)$ a unique algebra homomorphism~$\Psi'$ from~$\Tensor(V)$ to~$\Symm(V)$ that makes the diagram
          \[
            \begin{tikzcd}
              {}
              &
              V
              \arrow[bend right]{dl}[above left]{j}
              \arrow[bend left]{ddr}[above right]{i}
              &
              {}
              \\
              \Tensor(V)
              \arrow[bend left, dashed]{drr}[above right, pos=0.35]{\Psi'}
              \arrow{d}[left]{\Pi}
              &
              {}
              &
              {}
              \\
              \Tensor(V)/I
              &
              {}
              &
              \Symm(V)
            \end{tikzcd}
          \]
          commute.
          The commutator ideal~$I$ is contained in the kernel of~$\Psi'$ because the algebra~$\Symm(V)$ is commutative.
          There hence exists a unique algebra homomorphism~$\Psi$ from~$\Tensor(V)/I$ to~$\Symm(V)$ that makes the diagram
          \[
            \begin{tikzcd}
              {}
              &
              V
              \arrow[bend right]{dl}[above left]{j}
              \arrow[bend left]{ddr}[above right]{i}
              &
              {}
              \\
              \Tensor(V)
              \arrow[bend left, dashed]{drr}[above right, pos=0.37]{\Psi'}
              \arrow{d}[left]{\Pi}
              &
              {}
              &
              {}
              \\
              \Tensor(V)/I
              \arrow[dashed]{rr}[above]{\Psi}
              &
              {}
              &
              \Symm(V)
            \end{tikzcd}
          \]
          commute.
          The algebra homomorphism~$\Psi'$ is given on the algebra generating set~$\{ \class{v} \suchthat v \in V \}$ of~$\Tensor(V) / I$ by~$\Psi(\class{v}) = v$ for all~$v \in V$.
          
          It follows from the explicit descriptions of~$\Phi$ and~$\Psi$ that they are mutually inverse algebra isomorphisms.
          Thus~$\Symm(V) \cong \Tensor(V)/I$ via the isomorphism~$f$.
          
          We note that the commutator ideal~$I$ is already generated by the commutators~$v \tensor w - w \tensor v$ with~$v$,~$w$ in~$V$.
          Indeed, let~$J$ be the ideal generated by these elements.
          Then on the one hand~$J$ is contained in~$I$.
          But on the other hand the quotient~$\Tensor(V)/J$ is already commutative because it is generated by the residue classes~$\class{v}$ with~$v$ in~$V$, and these generators commute.
          The commutator ideal~$I$ is therefore contained in the kernel of the canonical quotient homomorphism from~$\Tensor(V)$ to~$\Tensor(V)/J$.
          But this kernel is the ideal~$J$, so~$I$ is contained in~$J$.
          
        \item
          The above argumentation is not surprising if we remember that the tensor algebra~$\Tensor(V)$ is the universal~{\algebra{$\kf$}} on~$V$ and that quotiening out the commutator ideal~$I$ is the universal way of making an algebra commutative.
          The quotient algebra~$\Tensor(V)/I$ thus ought to be the universal commutative~{\algebra{$\kf$}}.
          
          This motivation can be formalized by observing that the diagram of forgetful functors
          \[
            \begin{tikzcd}
              \cAlg{\kf}
              \arrow{d}
              &
              \cCAlg{\kf}
              \arrow{l}
              \arrow{dl}
              \\
              \cVect{\kf}
              &
              {}
            \end{tikzcd}
          \]
          commutes.
          It follows that the resulting diagram of left adjoints
          \[
            \begin{tikzcd}
              \cAlg{\kf}
              \arrow{r}[above]{C}
              &
              \cCAlg{\kf}
              \\
              \cVect{\kf}
              \arrow{u}[left]{\Tensor}
              \arrow{ur}[below right]{\Symm}
              &
              {}
            \end{tikzcd}
          \]
          commutes up to natural isomorphism.
          The adjoint~$C$ of the forgetful functor~$\cCAlg{\kf} \to \cAlg{\kf}$ is given by quotiening out the commutator ideal.
          Thus~$\Symm(V) \cong \Tensor(V)/I$.
        \item
          The above argumentation be also expressed via Yoneda’s lemma.
          We observing that for every commutative~{\algebra{$\kf$}}~$A$ there exist natural bijections
          \begin{align*}
            {}&
            \{ \textstyle\text{algebra homomorphisms~$\Symm(V) \to A$} \}
            \\
            \cong{}&
            \{ \textstyle\text{{\linear{$\kf$}} maps~$V \to A$} \}
            \\
            \cong{}&
            \{ \textstyle\text{algebra homomorphisms~$\Tensor(V) \to A$} \}
            \\
            \cong{}&
            \{ \textstyle\text{algebra homomorphisms~$\Tensor(V)/I \to A$} \} \,,
          \end{align*}
          It now follows from Yoneda’s~lemma that~$\Symm(V) \cong \Tensor(V)/I$.
      \end{itemize}
  \end{description}
\end{recall}


\begin{remark}
  One can similarly construct the \defemph{exterior algebra}~$\Exterior(V)$~$\glsadd{exterior algebra} = \bigoplus_{d \geq 0} \Exterior^d(V)$ of a vector space~$V$ by using the exterior powers~$\Exterior^d(V)$ instead of the the tensor powers~$V^{\otimes d}$ or symmetric powers~$\Symm^d(V)$.
  If~$A$ is another~\algebra{$\kf$} then a homomorphism of algebras~$\Phi$ from~$\Exterior(V)$ to~$A$ is then \enquote{the same} as a~\linear{$\kf$} map~$f$ from~$V$ to~$A$ such that~$f(v)^2 = 0$ for all~$v \in V$.
% TODO: Do we have to worry about char(k) = 2?
  It thus follows from a similar argumentation as for the symmetric algebra that~$\Exterior(V)$ is isomorphic to the quotient algebra~$\Tensor(V)/I$ where~$I$ is the two-sided ideal in~$\Tensor(V)$ generated by the elements~$v \tensor v$ with~$v$ in~$V$.
  
  If the vector space~$V$ is finite-dimensional then the exterior algebra~$\Exterior(V)$ is again finite-dimensional.
  Its dimension is given by~$\dim( \Exterior(V) ) = 2^{\dim(V)}$.
  This behavior is different to both that of the tensor algebra~$\Tensor(V)$ and that of the symmetric algebra~$\Symm(V)$, which are infinite-dimensional whenever the vector space~$V$ is nonzero.
\end{remark}





