\section{The Killing Form and Cartan’s Criterion}





\subsection{Associative Bilinear Forms}


\begin{definition}
  \label{associative bilinear form}
  Let~$\glie$ be a Lie~algebra over an arbitrary field~$\kf$.
  A bilinear form~$\beta \colon \glie \times \glie \to \kf$ is \defemph{associative}\index{associative bilinear form} if
  \[
    \beta(x,[y,z])
    =
    \beta([x,y],z)
  \]
  for all~$x, y, z \in \glie$.
\end{definition}


\begin{remark}
  Let~$V$ and~$W$ be two representations of a Lie~algebra~$\glie$.
  Then the actions of~$\glie$ on~$V$ and~$W$ induces an action of~$\glie$ on~$(V \tensor W)^*$ as described in \cref{new representations from old ones}:
  We have for all~$x \in \glie$,~$\phi \in (V \tensor W)^*$ and simple tensors~$v \tensor w \in V \tensor W$ that
  \begin{align*}
    (x.\phi)(v \tensor w)
    &=
    - \phi(x.(v \tensor w))
    \\
    &=
    - \phi((x.v) \tensor w + v \tensor (x.w))
    \\
    &=
    - \phi((x.v) \tensor w) - \phi(v \tensor (x.w)) \,.
  \end{align*}
  
  The vector space~$(V \tensor W)^*$ can be defined with~$\gls*{bilinear forms}$, the space of bilinear forms~$V \times W \to \kf$\index{bilinear forms} via the universal property of the tensor product.
  For any~$\phi \in (V \tensor W)^*$ the corresponding bilinear form~$\beta \in \BF(V,W)$ is given by
  \[
    \beta(v,w)
    =
    \phi(v \tensor w)
  \]
  for all~$v \in V$ and~$w \in W$.
  It follows that~$\BF(V,W)$ becomes a representation of~$\glie$ via
  \[
    (x.\beta)(v,w)
    =
    - \beta(x.v,w) - \beta(v,x.w)
  \]
  for all~$x \in \glie$,~$\beta \in \BF(V,W)$ and~$v \in V$,~$w \in W$.
  
  A special case occurs when both~$V$ and~$W$ are the adjoint representation.
  Then~$V = W = \glie$ and~$x.y = [x,y]$ for all~$x, y \in \glie$.
  Then the action of~$\glie$ on~$\BF(\glie,\glie)$ is given by
  \begin{align*}
    (y.\beta)(x,z)
    &=
    - \beta(y.x, z) - \beta(x, y.z)
    \\
    &=
    - \beta([y,x], z) - \beta(x, [y,z])
    \\
    &=
    \beta([x,y], z) - \beta(x, [y,z])
  \end{align*}
  for all~$y \in \glie$,~$\beta \in \BF(\glie, \glie)$ and~$x, z \in \glie$.
  
  We see that a bilinear form~$\beta \colon \glie \times \glie \to \kf$ is associative if and only if it is invariant under the action of~$\glie$ on~$\BF(\glie, \glie)$ (i.e.~$y.\beta = 0$ for every~$y \in \glie$).
  Because of this observation, associative bilinear forms on~$\glie$ are also called \defemph{invariant \textup(bilinear forms\textup)}\index{invariant!bilinear form}.
\end{remark}


\begin{lemma}
  \label{bilinear forms and hom identification}
  Let~$V$ and~$W$ be representations of a Lie~algebra~$\glie$.
  Then the natural isomorphims of vector spaces
% TODO: Align these
  \[
    \Phi_1
    \colon
    \BF(V,W)
    \to
    \Hom_\kf(V, W^*) \,,
    \quad
    \beta
    \mapsto
    (
      v
      \mapsto
      \beta(v,-)
    )
  \]
  and
  \[
    \Phi_2
    \colon
    \BF(V,W)
    \to
    \Hom_\kf(W, V^*) \,,
    \quad
    \beta
    \mapsto
    (
      w
      \mapsto
      \beta(-,w)
    )
  \]
  are already isomorphisms of representations.
\end{lemma}


\begin{proof}
  We have for all~$x \in \glie$,~$\beta \in \BF(V,W)$ and~$v \in V$,~$w \in W$ that
  \begin{align*}
    \Phi_1(x.\beta)(v)(w)
    &=
    ((x.\beta)(v,-))(w)
    \\
    &=
    (x.\beta)(v,w)
    \\
    &=
    - \beta(x.v, w) - \beta(v, x.w)
  \shortintertext{as well as}
    (x.\Phi_1(\beta))(v)(w)
    &=
    \bigl( x.\Phi_1(\beta)(v) - \Phi_1(\beta)(x.v) \bigr)(w)
    \\
    &=
    \bigl( x.\Phi_1(\beta)(v) - \Phi_1(\beta)(x.v) \bigr)(w)
    \\
    &=
    \bigl( x.\Phi_1(\beta)(v) - \Phi_1(\beta)(x.v) \bigr)(w)
    \\
    &=
    (x.\beta(v,-))(w) - \beta(x.v, w)
    \\
    &=
    - \beta(v,-)(x.w) - \beta(x.v, w)
    \\
    &=
    - \beta(v, x.w) - \beta(x.v, w) \,.
  \end{align*}
  This shows that~$\Phi_1$ is a homomorprhism, and hence isomorphism, of representations.
  For~$\Phi_2$ a similar caculation can be done.%
  \footnote{The author hasn’t actually done this but hopes that no sign problems occur.}
\end{proof}


\begin{corollary}
  \label{invariant bilinear form induces homomorphism of representations}
  Let~$V$ and~$W$ be representations of a Lie~algebra~$\glie$.
  Let~$\beta \colon V \times W \to \kf$ be a bilinear form with associated linear maps
  \begin{alignat*}{3}
    \phi_1
    \colon
    V
    \to
    W^* \,,
    \quad
    v
    \mapsto
    \beta(v, -) \,,
  \shortintertext{and}
    \phi_2
    \colon
    W
    \to
    V^* \,,
    \quad
    w
    \mapsto
    \beta(-, w) \,.
  \end{alignat*}
  Then the following conditions on~$\beta$ are equivalent:
  \begin{equivalenceslist}
    \item
      \label{beta invariant}
      $\beta$ is invariant.
    \item
      \label{phi1 a homomorphism}
      $\phi_1$ is a homomorphism of representations.
    \item
      \label{phi2 a homomorphism}
      $\phi_2$ is a homomorphism of representations.
  \end{equivalenceslist}
\end{corollary}


\begin{proof}
  The isomorphisms of representations~$\Phi_1$ from \cref{bilinear forms and hom identification} restrict to vector space isomorphisms between the invariants~$\BF(V, W)^\glie$ and~$\Hom_\kf(V,W)^\glie$.
  We know from \cref{homomorphisms of representations as invariants} that~$\Hom_\kf(V,W)^\glie = \Hom_\glie(V,W)$, whence the equivalence of~\ref*{beta invariant} and~\ref*{phi1 a homomorphism}.
  The equivalence of~\ref*{beta invariant} and~\ref*{phi2 a homomorphism} can be seen in the same way.
\end{proof}


\begin{corollary}
  \label{associative bilinear form induces homomorphism of representations}
  Let~$\glie$ be a Lie~algebra.
  Let~$\beta \colon \glie \times \glie \to \kf$ be a bilinear form with associated linear maps
  \begin{alignat*}{3}
    \phi_1
    \colon
    \glie
    \to
    \glie^* \,,
    \quad
    x
    \mapsto
    \beta(x, -) \,,
  \shortintertext{and}
    \phi_2
    \colon
    \glie
    \to
    \glie^* \,,
    \quad
    y
    \mapsto
    \beta(-, y) \,.
  \end{alignat*}
  Then the following conditions on~$\beta$ are equivalent:
  \begin{equivalenceslist}
    \item
      $\beta$ is associative.
    \item
      $\phi_1$ is a homomorphism of representations.
    \item
      $\phi_2$ is a homomorphism of representations.
  \end{equivalenceslist}
\end{corollary}


\begin{proof}
  Apply \cref{invariant bilinear form induces homomorphism of representations} with~$V$ and~$W$ the adjoint representation of~$\glie$.
\end{proof}


\begin{recall}
  Given two vector spaces~$V$ and~$W$ a bilinear form~$\beta \colon V \times W \to \kf$ is \defemph{non-degenerate}\index{non-degenerate bilinear form} if
  \begin{equivalenceslist}
    \item
      \label{nondegenerate in the left}
      for every nonzero~$v \in V$ there exists some~$w \in W$ with~$\beta(v,w) \neq 0$ and
    \item
      \label{nondegenerate in the right}
      for every nonzero~$w \in W$ there exists some~$v \in V$ with~$\beta(v,w) \neq 0$.
  \end{equivalenceslist}
  If more generally condition~\ref*{nondegenerate in the left} is satisfied then~$\beta$ is \defemph{non-degenerate in the left argument} and if condition~\ref*{nondegenerate in the right} is satisfied then~$\beta$ is \defemph{non-degenerate in the right argument}.
  The bilinear form~$\beta$ is therefore non-degenerate if and only if it is both non-degenerate in the left argument and non-degenerate in the right argument.
  
  If~$\phi_1 \colon V \to W^*$ and~$\phi_2 \colon W \to V^*$ are the linear maps associated to~$\beta$ then the conditions~\ref*{nondegenerate in the left} and~\ref*{nondegenerate in the right} may be rewritten, stating that
  \begin{equivalenceslist}[label = \roman*')]
    \item
      the linear map~$\phi_1 \colon V \to W^*$ is injective,
    \item
      the linear map~$\phi_2 \colon W \to V^*$ is injective.
  \end{equivalenceslist}
  
  Suppose now that both~$V$ and~$W$ are finite dimensional.
  For~$\beta$ to be non-degenerate in the left argument we need that~$\dim V \leq \dim W^* = \dim W$ because we need~$\phi_1$ to be injective.
  For~$\beta$ to be non-degenerate in the right argument we similarly need that~$\dim W \leq \dim V$.
  Hence for~$\beta$ to even have a chance at being non-degenerate we need that~$\dim V = \dim W$.%
  \footnote{
  This conclusion does not hold for infinite dimensional vector spaces:
  If~$V$ is any infinite dimensional then the evaluation map~$\beta \colon V \times V^* \to \kf$ given by~$\beta(v,\alpha) = \alpha(v)$ is a non-degenerate bilinear form.
  The linear map~$\phi_1 \colon V \to V^{**}$ is the canonical inclusion and the linear map~$\phi_2 \colon V^* \to V^*$ is just the identity, both of which are injective.
  But the dual space~$V^*$ has a strictly larger dimension than~$V$ itself.}
  
  So suppose now that both~$V$ and~$W$ have the same finite dimension~$n$.
  Then for basis~$v_1, \dotsc, v_n$ of~$V$ and a basis~$w_1, \dotsc, w_n$ of~$W$ we can represent the bilinear form~$\beta$ by a square matrix~$A \in \Mat_n(\kf)$ that is given by
  \[
    A_{ij}
    \defined
    \beta(v_i, w_j)
  \]
  for all~$i,j = 1, \dotsc, n$.
  Then with respect to the basis~$v_1, \dotsc, v_n$ of~$V$ and the (dual) basis~$w_1^*, \dotsc, w_n^*$ of~$W$ the linear map~$\phi_1 \colon V \to W^*$ is represented by the matrix~$A^T$;
  with respect to the basis~$w_1, \dotsc, w_n$ of~$W$ and the (dual) basis~$v_1^*, \dotsc, v_n^*$ of~$V^*$ the linear map~$\phi_2 \colon W \to V^*$ is given by the matrix~$A$.
  We therefore find that
  \begin{align*}
        {}& \text{$\beta$ is non-degenerate in the left argument} \\
    \iff{}& \text{$\phi_1$ is injective}  \\
    \iff{}& \text{$A^T$ is injective} \\
    \iff{}& \text{$A^T$ is invertible}
  \end{align*}
  and similarly that
  \begin{align*}
        {}& \text{$\beta$ is non-degenerate in the right argument} \\
    \iff{}& \text{$\phi_2$ is injective}  \\
    \iff{}& \text{$A$ is injective} \\
    \iff{}& \text{$A$ is invertible}  \,.
  \end{align*}
  But~$A$ is invertible if and only its transpose~$A^T$ is invertible.
  
  By using these observations, and that~$\dim V = \dim W = \dim V^* = \dim W^*$ we see that all of the following conditions are equivalent:
  \begin{equivalenceslist}
    \item
      The bilinear form~$\beta$ is non-degenerate.
    \item
      The bilinear form~$\beta$ is non-degenerate in the left argument.
    \item
      The bilinear form~$\beta$ is non-degenerate in the right argument.
    \item
      The linear map~$\phi_1$ is injective.
    \item
      The linear map~$\phi_1$ is bijective.
    \item
      The linear map~$\phi_2$ is injective.
    \item
      The linear map~$\phi_2$ is bijective.
    \item
      The matrix~$A$ is invertible.
  \end{equivalenceslist}
\end{recall}


\begin{corollary}
  \label{associative non-degenerate bilinear forms induce isomorphism to the dual}
  If~$\glie$ is a finite dimensional Lie~algebra and~$\beta \colon \glie \times \glie \to \kf$ is a bilinear form then for the associated bilinear maps
% TODO: Fix alignment.
  \begin{gather*}
    \phi_1
    \colon
    \glie
    \to
    \glie^*,
    \quad
    x
    \mapsto
    \beta(x,-)
  \shortintertext{and}
    \phi_2
    \colon
    \glie
    \to
    \glie^*,
    \quad
    y
    \mapsto
    \beta(-,y)
  \end{gather*}
  The following conditions are equivalent:
  \begin{equivalenceslist}
    \item
      $\beta$ is associative and non-degenerate.
    \item
      $\phi_1$ is an isomorphism of representations.
    \item
      $\phi_2$ is an isomorphism of representations.
    \qed
  \end{equivalenceslist}
\end{corollary}


\begin{definition}
 Let~$V$ be a vector space and let~$\beta \colon V \times V \to \kf$ be a symmetric bilinear form.
 Then
 \[
  \gls*{radical bilinear}
  \defined
  \{
    x \in V
  \suchthat
    \text{$\beta(x,y) = 0$ for every~$y \in V$}
  \}
 \]
 is the~\defemph{radical}\index{radical!of a bilinear form} of~$\beta$.
\end{definition}


\begin{lemma}
  \label{orthogonal complement of an ideal is again an ideal}
  Let~$\glie$ be a Lie~algebra over an arbitrary field~$\kf$ and~$\beta \colon \glie \times \glie \to \kf$ an associative symmetric bilinear form.
  Then for any ideal~$I$ in~$\glie$ the orthogonal complement
  \[
    I^\perp
    \defined
    \{
      x \in \glie
    \suchthat
      \text{$\beta(x,y) = 0$ for every $y \in I$}
    \}
  \]
  is again an ideal in~$\glie$.
\end{lemma}


\begin{proof}
  The associated linear map~$\phi \colon \glie \to \glie^*$ given by~$\phi(x) = \beta(x,-)$ is a homomorphism of representations because~$\beta$ is associative.
  The inclusion~$I \to \glie$ is a homomorphism of representions, and so its dual map
  \[
    \glie^*
    \to
    I^* \,,
    \quad
    \alpha
    \mapsto
    \restrict{\alpha}{I}
  \]
  is again a homomorphism of representations.
  It follows that the composition
  \[
    \psi
    \colon
    \glie
    \to
    I^* \,,
    \quad
    x
    \mapsto
    \restrict{\beta(x,-)}{I}
  \]
  is homomorphism of representations.
  Hence~$I^\perp = \ker \psi$ is a subrepresentation of~$\glie$, i.e.\ an ideal in~$\glie$.
\end{proof}


\begin{corollary}
  \label{radical of bilinear form is an ideal}
  If~$\beta$ is an associative symmetric bilinear form on a Lie~algebra~$\glie$ then its radical~$\rad \beta$ is an ideal in~$\glie$.
\end{corollary}


\begin{proof}
  Apply \cref{orthogonal complement of an ideal is again an ideal} to~$I = \glie$.
\end{proof}


\begin{remark}
 The proof of \ref{orthogonal complement of an ideal is again an ideal} did not use that~$\beta$ is symmetric.
 This artficial restraint is only there to simplify the situation and notation (we do not need to distinguish between orthogonal complements from the left and from the right).
 The main example of an associative bilinear form will be the Killing~form, which is symmetric, so this assumption will pose no problems to us.
\end{remark}





\subsection{The Killing~Form}


\begin{recall}
  \label{symmetric property of trace}
  Let~$V$ be a finite dimensional~{\vectorspace{$\kf$}}.
  Then
  \[
    \tr(f g)
    =
    \tr(g f)
  \]
  for any two endomorphisms~$f, g \in \End_\kf(V)$.
  It follows by induction that
  \[
    \tr(f_1 f_2 \dotsm f_n)
    =
    \tr(f_2 \dotsm f_n f_1)
  \]
  for all endomorphisms~$f_1, f_2, \dotsc, f_n \in \End_\kf(V)$.
\end{recall}


\begin{warning}
  If~$V$ is a finite dimensional~{\vectorspace{$\kf$}} with~$\dim V \geq 2$ then it \emph{does not} hold that
  \[
    \tr(f_1 \dotsm f_n)
    =
    \tr(f_{\sigma(1)} \dotsm f_{\sigma(n)})
  \]
  for all endomorphisms~$f_1, \dotsc, f_n \in \End_\kf(V)$ and every permuation~$\sigma \in S_n$.
  This can be seen by considering the matrices~$A_1, A_2, A_3 \in \Mat_2(\kf)$ given by
  \[
    A_1
    \defined
    \begin{pmatrix}
      1 & 0 \\
      0 & 0
    \end{pmatrix} \,,
    \qquad
    A_2
    \defined
    \begin{pmatrix}
      0 & 1 \\
      1 & 0
    \end{pmatrix} \,,
    \qquad
    A_3
    \defined
    \begin{pmatrix}
      0 & 0 \\
      1 & 0
    \end{pmatrix} \,.
  \]
  For these matrices
  \[
    \tr(A_1 A_2 A_3)
    =
    1
    \quad\text{but}\quad
    \tr(A_1 A_3 A_2)
    =
    0 \,.
  \]
\end{warning}


\begin{recall}
  \label{recalling the trace form}
  For any finite dimensional~{\vectorspace{$\kf$}}~$V$ the endomorphism vector space~$\End_\kf(V)$ admits a bilinear form~$(-,-)_{\tr}$ given by
  \[
    (f,g)_{\tr}
    \defined
    \tr(fg) \,,
  \]
  the so called~\defemph{trace form}\index{trace form} on~$\End_\kf(V)$.
  The trace form is symmetric by \cref{symmetric property of trace}, and it non degenerate:
  If with respect to a basis~$v_1, \dotsc, v_n$ of~$V$ the endomorphisms~$f, g \in \End_\kf(V)$ are represented by matrices~$A, B \in \Mat_n(\kf)$ then
  \[
    (f,g)_{\tr}
    =
    \tr(fg)
    =
    \tr(AB)
    =
    \sum_{i=1}^n (AB)_{ii}
    =
    \sum_{i,j=1}^n
    A_{ij} B_{ji} \,.
  \]
  If~$f \neq 0$ then~$A \neq 0$ and hence~$A_{ij} \neq 0$ for some indices~$i$,~$j$.
  For the endomorphism~$g$ with~$B = E_{ji}$ we then find that
  \[
    (f,g)_{\tr}
    =
    A_{ij}
    \neq
    0 \,.
  \]
  The trace form is also associative in the sense that
  \begin{equation}
    \label{associativity of the trace form wrt multiplication}
    (fg, h)_{\tr}
    =
    \tr(fgh)
    =
    (f, gh)_{\tr}
  \end{equation}
  for all~$f, g, h \in \End_\kf(V)$.
\end{recall}


\begin{example}
  \label{trace form is symmetric and associative}
  If~$V$ is a finite dimensional vector space then the trace form~$(-,-)_{\tr}$ on~$\gllie(V)$ is associative (in the sense of \cref{associative bilinear form}):
  By using that the trace form is symmetric and associative in the sense of~\eqref{associativity of the trace form wrt multiplication} we find that
  \begin{equation}
    \label{associativity of the trace form wrt bracket}
    \begin{aligned}
      ([f,g],h)_{\tr}
      &=
      (fg - gf, h)_{\tr}
      \\
      &=
      (fg, h)_{\tr} - (gf,h)_{\tr}
      \\
      &=
      (fg, h)_{\tr} - (h, gf)_{\tr}
      \\
      &=
      (f, gh)_{\tr} - (hg, f)_{\tr}
      \\
      &=
      (f, gh)_{\tr} - (f, hg)_{\tr}
      \\
      &=
      (f, gh - hg)_{\tr}
      \\
      &=
      (f, [g,h])_{\tr}
    \end{aligned}
  \end{equation}
  for all~$f, g, h \in \gllie(V)$.
  Hence~$(-,-)_{\tr}$ is an associative symmetric bilinear form on~$\gllie(V)$.
\end{example}


\begin{remark}
  Suppose more generally that~$A$ is a~{\algebra{$\kf$}} that is not necessarily associative nor unital.
  Then a bilinear form~$\beta \colon A \times A \to \kf$ is \defemph{associative}\index{associative bilinear form} if
  \[
    \beta(ab, c)
    =
    \beta(a, bc)
  \]
  for all~$a, b , c \in A$.
  If~$A$ is an associative algebra and~$\beta$ is an associative symmetric bilinear form on~$A$ then the calculation~\eqref{associativity of the trace form wrt bracket} shows that~$\beta$ is also associative with respect to the Lie bracket on~$A$, i.e.\ associative in the sense of \cref{associative bilinear form}.
\end{remark}


\begin{lemma}
  \label{pullback of associative bilinear form}
  Let~$f \colon \glie \to \hlie$ be a homorphism of Lie algebras.
  If~$\beta$ is an associative bilinear form on~$\hlie$ then the pullback
  \[
    \alpha(x,y)
    \defined
    \beta(f(x), f(y))
  \]
  is an associative bilinear form on~$\glie$.
  If~$\beta$ is symmetric then so is~$\alpha$.
  \qed
\end{lemma}


\begin{definition}[Killing form]
  \label{killing form}
  Let~$\glie$ be a finite dimensional Lie~algebra over an arbitrary field~$\kf$.
  The \defemph{Killing~form}~$\gls*{killing form}$\index{Killing form} of~$\glie$ is the pullback of the trace form of~$\gllie(V)$ along the Lie~algebra homomorphism~$\ad \colon \glie \to \gllie(V)$.
  More explicitely,
  \[
    \kappa(x,y)
    =
    \tr(\ad(x) \ad(y))
  \]
  for all~$x, y \in \glie$.
\end{definition}


\begin{lemma}
  \label{killing form is associative and symmetric}
  Let~$\glie$ be a finite dimensional Lie algebra over an arbitrary field~$\kf$.
  The Killing form of~$\glie$ is associative and symmetric.
\end{lemma}


\begin{proof}
  This follows by \cref{pullback of associative bilinear form} from \cref{trace form is symmetric and associative}.
\end{proof}


\begin{remark}
  \label{associative bilinear form of a representation}
  If~$(V,\rho)$ is any finite dimensional representation of a Lie algebra~$\glie$ then pulling back the trace form on~$\gllie(V)$ via the Lie~algebra homomorphism~$\rho \colon \glie \to \gllie(V)$ results in an associative symmetric bilinear form~$\beta_\rho$ on~$\glie$, given by
  \[
    \beta_\rho(x,y)
    \defined
    \tr(\rho(x) \rho(y))
  \]
  for all~$x, y \in \glie$.
  The Killing~form~$\kappa$ of~$\glie$ is the special case~$\kappa = \beta_{\ad}$.
\end{remark}


\begin{example}
  \label{killing form of gl_n}
  Let~$\glie \defined \glie_n(\kf)$ for some arbitrary~field $\kf$.
  Then
  \[
    \kappa(x,y)
    =
    2n\tr(xy) - 2\tr(x)\tr(y)
    \defines
    \beta(x,y)
  \]
  for all~$x, y \in \glie$.
  
  To see this let~$E_{ij}$ with~$i,j = 1, \dotsc, n$ be the standard basis of~$\gllie_n(\kf)$.
  Then
  \begin{equation}
    \label{product of elementary matrices}
    E_{ij} E_{kl}
    =
    \delta_{jk} E_{il}
  \end{equation}
  for all~$i,j,k,l = 1, \dotsc, n$.
  To show that~$\kappa = \beta$ is sufficies to show that
  \[
    \kappa(E_{ij}, E_{kl})
    =
    \beta(E_{ij}, E_{kl})
  \]
  for all~$i,j,k,l = 1, \dotsc, n$ as both~$\kappa$ and~$\beta$ are bilinear.
  
  For all~$k,l,g,h = 1, \dotsc, n$ it follows from the relation~\eqref{product of elementary matrices} that
  \[
    \ad(E_{kl})(E_{gh})
    =
    [E_{kl}, E_{gh}]
    =
    E_{kl} E_{gh} - E_{gh} E_{kl}
    =
    \delta_{lg} E_{kh} - \delta_{kh} E_{gl} \,.
  \]
  It follows for all~$i,j,k,l,g,h = 1, \dotsc, n$ that
  \begin{align*}
    \ad(E_{ij})\ad(E_{kl})(E_{gh})
    &=
    \ad(E_{ij})(\delta_{lg} E_{kh}
    - \delta_{kh} E_{gl})
    \\
    &=
    \delta_{lg} \ad(E_{ij})(E_{kh})
    - \delta_{kh} \ad(E_{ij})(E_{gl})
    \\
    &=
      \delta_{lg}(\delta_{jk} E_{ih}
    - \delta_{ih} E_{kj})
    - \delta_{kh}(\delta_{jg} E_{il}
    - \delta_{il} E_{gj})
    \\
    &=
      \delta_{jk} \delta_{lg} E_{ih}
    - \delta_{ih} \delta_{lg} E_{kj}
    - \delta_{jg} \delta_{kh} E_{il}
    + \delta_{il} \delta_{kh} E_{gj}
  \end{align*}
  and the coefficient of~$E_{gh}$ in this expression is
  \[
    a_{gh}
    =
      \delta_{jk} \delta_{lg} \delta_{ig}
    + \delta_{il} \delta_{kh} \delta_{jh}
    - \delta_{ih} \delta_{lg} \delta_{kg} \delta_{jh}
    - \delta_{jg} \delta_{kh} \delta_{ig} \delta_{hl} \,.
  \]
  It follows for all~$i,j,k,l = 1, \dotsc, n$ that
  \begin{align*}
    \kappa(E_{ij}, E_{kl})
    &=
    \sum_{g,h=1}^n a_{gh}
    \\
    &=
    \sum_{g,h=1}^n
    (
        \delta_{jk} \delta_{lg} \delta_{ig}
      + \delta_{il} \delta_{kh} \delta_{jh}
      - \delta_{ih} \delta_{lg} \delta_{kg} \delta_{jh}
      - \delta_{jg} \delta_{kh} \delta_{ig} \delta_{hl}
    )
    \\
    &=
      \sum_{g,h=1}^n \delta_{jk} \delta_{lg} \delta_{ig}
    + \sum_{g,h=1}^n \delta_{il} \delta_{kh} \delta_{jh}
    - \sum_{g,h=1}^n \delta_{ih} \delta_{lg} \delta_{kg} \delta_{jh}
    - \sum_{g,h=1}^n \delta_{jg} \delta_{kh} \delta_{ig} \delta_{hl}
    \\
    &=
      n \delta_{jk} \delta_{il}
    + n \delta_{il} \delta_{jk}
    - \delta_{ij} \delta_{kl}
    - \delta_{ij} \delta_{kl}
    \\
    &=
      2 n \delta_{il} \delta_{jk}
    - 2 \delta_{ij} \delta_{kl}
    \\
    &=
    2n \delta_{jk} \tr(E_{il})  - 2 \tr(E_{ij}) \tr(E_{kl})
    \\
    &=
    2n \tr(E_{ij} E_{kl}) - 2 \tr(E_{ij}) \tr(E_{kl})
    \\
    &=
    \beta(E_{ij}, E_{kl})  \,.
  \end{align*}
\end{example}


\begin{lemma}
  \label{restriction of the killing form to an ideal}
  Let~$\glie$ be a finite-dimensional Lie~algebra over an arbitrary field~$\kf$.
  If~$I$ is an ideal in~$\glie$ then the Killing~form~$\kappa_I$ of~$I$ is given by restriction of the Killing~form~$\kappa_\glie$ of~$\glie$ to~$I$, i.e.~$\kappa_I = \restrict{\kappa_\glie}{I \times I}$.
\end{lemma}


\begin{proof}
  Let~$x, y \in I$.
  Let~$(x_1, \dotsc, x_n)$ be a basis of~$I$ and extend it to basis~$(x_1, \dotsc, x_n, x_{n+1}, \dotsc, x_m)$ of~$\glie$.
  With respect to the basis~$(x_1, \dotsc, x_n)$ of~$I$ the endomorphism~$\ad_I(x)$ is represented by a matrix~$A_x \in \Mat_n(\kf)$ and the endomorphism~$\ad_I(y)$ is represented by a matrix~$A_y \in \Mat_n(\kf)$.
  
  The images of the endomorphisms~$\ad_\glie(x)$ and~$\ad_\glie(y)$ are contained in~$I$ because~$I$ is an ideal in~$\glie$.
  With respect to the basis~$(x_1, \dotsc, x_m)$ of~$\glie$ the endomorphisms $\ad_\glie(x)$ and $\ad_\glie(y)$ are therefore represented by matrices of the form
  \[
    C_x
    =
    \begin{pmatrix}
      A_x & B_x \\
      0   & 0
    \end{pmatrix}
    \in
    \Mat_m(\kf)
    \qquad\text{and}\qquad
    C_y
    =
    \begin{pmatrix}
      A_y & B_y \\
      0   & 0
    \end{pmatrix}
    \in
    \Mat_m(\kf)
  \]
  for some matrices~$B_x, B_y \in \Mat_{n,m-n}(\kf)$.
  (Here we used that the endomorphisms~$\ad_I(x)$ and~$\ad_I(y)$ of~$I$ are the restrictions of the endorphisms~$\ad_\glie(x)$ and~$\ad_\glie(y)$ of~$\glie$ to~$I$.)
  From these explicit matrix descriptions we find that
  \begin{align*}
    \kappa_\glie(x,y)
    &=
    \tr(\ad_\glie(x) \ad_\glie(y))
    \\
    &=
    \tr\left(
      \begin{pmatrix}
        A_x & B_x \\
        0   & 0
      \end{pmatrix}
      \begin{pmatrix}
        A_y & B_y \\
        0   & 0
      \end{pmatrix}
    \right)
    \\
    &=
    \tr
    \begin{pmatrix}
      A_x A_y & A_x B_y \\
      0       & 0
    \end{pmatrix}
    \\
    &=
    \tr(A_x A_y)
    \\
    &=
    \tr(\ad_I(x) \ad_I(y))
    \\
    &=
    \kappa_I(x,y) \,,
  \end{align*}
  as desired.
\end{proof}


% TODO: Lemma does not hold for the restriction to subalgebras


\begin{example}
  We have seen in \cref{killing form of gl_n} that the Killing~form of~$\gllie_n(\kf)$ is given by
  \[
    \kappa_{\gllie_n(\kf)}(x,y)
    =
    2n \tr(xy) - 2\tr(x)\tr(y)
  \]
  for all~$x, y \in \gllie_n(\kf)$.
  It follows that the Killing~form of~$\sllie_n(\kf) = [\gllie_n(\kf), \gllie_n(\kf)]$ is given by
  \[
    \kappa_{\sllie_n(\kf)}(x,y)
    =
    2n \tr(xy)
  \]
  for all~$x, y \in \sllie_n(\kf)$ (recall that~$\sllie_n(\kf) = \{x \in \gllie_n(\kf) \suchthat \tr(x) = 0\}$.
  We observe that the Killing~form of~$\sllie_n(\kf)$ is a scalar multiple of the trace form.
\end{example}


\begin{lemma}
  \label{orthogonal ideals with respect to the killing form}
  Let~$\glie$ be a finite dimensional Lie~algebra and suppose that~$\glie = I \oplus J$ for two ideals~$I$ and~$J$.
  \begin{enumerate}
    \item
      The ideals~$I$ and~$J$ are mutually orthogonal with respect to the Killing form of~$\glie$.
    \item
      If~$x,y \in \glie$ are given by~$x = x_1 + x_2$ and~$y = y_1 + y_2$ with respect to the decomposition~$\glie = I \oplus J$ then
      \[
        \kappa_\glie(x,y)
        =
        \kappa_I(x_1, y_1) + \kappa_J(x_2, y_2) \,.
      \]
  \end{enumerate}
\end{lemma}


\begin{proof}
  \leavevmode
  \begin{enumerate}
    \item
      We have for all~$x \in I$ and~$y \in J$ that~$\ad(x) \ad(y) = 0$ because
      \[
        \ad(x)\ad(y)(\glie)
        =
        [x,[y,\glie]]
        \subseteq
        [I,[J,\glie]]
        \subseteq
        [I,J]
        \subseteq
        I \cap J
        =
        0 \,.
      \]
      Is follows that
      \[
        \kappa(x,y)
        =
        \tr(\ad(x)\ad(y))
        =
        \tr(0)
        =
        0 \,.
      \]
    \item
      We know from \cref{restriction of the killing form to an ideal} that~$\kappa(x_1, y_1) = \kappa_I(x_1, y_1)$ and~$\kappa(x_2, y_2) = \kappa_J(x_2, y_2)$.
      Therefore
      \begin{align*}
        \kappa(x,y)
        &=
        \kappa(x_1 + x_2, y_1 + y_2)
        \\
        &=
          \kappa(x_1, y_1)
        + \underbrace{\kappa(x_1, y_2)}_{=0}
        + \underbrace{\kappa(x_2, y_1)}_{=0}
        + \kappa(x_2, y_2)
        \\
        &=
        \kappa_I(x_1, y_1) + \kappa_J(x_2, y_2) \,,
      \end{align*}
      as desired.
    \qedhere
  \end{enumerate}
\end{proof}





\subsection{Concrete Jordan Decomposition}


\begin{definition}
  An endomorphism of a finite dimensional vector space (resp.\ a square matrix) is \defemph{semisimple}\index{semisimple!endomorphism}\index{semisimple!matrix} if it is diagonalizable.
\end{definition}


\begin{remark}
  An endomorphism~$x$ of a finite dimensional~{\vectorspace{$\kf$}}~$V$ is semisimple if and only if every~{\invariant{$x$}} subspace of~$V$ admits a direct complement that is again~{\invariant{$x$}}.
  (Every diagonalizable endomorphism has this property;
  the converse depends on~$\kf$ being algebraically closed.)
\end{remark}


\begin{theorem}[Concrete Jordan decomposition]
\label{concrete jordan decomposition}
  Let~$x$ be an endomorphism of a finite dimensional~{\vectorspace{$\kf$}}~$V$.
  \begin{enumerate}
    \item
      \label{existence and uniqueness of concrete jcd}
      There exist unique endomorphisms~$x_s$ and~$x_n$ of~$V$ such that
      \begin{enumerate}
        \item
          $x = x_s + x_n$,
        \item
          $x_s$ is semisimple and~$x_n$ is nilpotent,
        \item
          $x_s$ and~$x_n$ commute.
      \end{enumerate}
    \item
      \label{concrete jcd are polynomial}
      The endomorphisms~$x_s$ and~$x_n$ can be expressed as polynomials in~$x$ without constant term, i.e.\ there exist polynomials~$P,Q \in \kf[T]$ with~$P(0) = Q(0) = 0$ such that~$x_s = P(x)$ and~$x_n = Q(x)$.
    \item
      \label{commuting via concrete jcd}
      Any endomorphism of~$V$ commutes with~$x$ if and only if it commutes with both~$x_s$ and~$x_n$.
    \item
      \label{concrete jcd compatible with mapping of subspaces}
      If~$U \subseteq W$ are linear subspaces of~$V$ with~$x(W) \subseteq U$ then also~$x_s(W) \subseteq U$ and~$x_n(W) \subseteq U$.
  \end{enumerate}
\end{theorem}


\begin{proof}
  Let~$\chi(T)$ be the characteristic polynomial of~$x$, given by
  \[
    \chi(T)
    =
    \prod_{i=1}^n (T-\lambda_i)^{m_i} \,,
  \]
  where~$\lambda_1, \dotsc, \lambda_n \in \kf$ are the pairwise different eigenvalues of~$x$.
  By the chinese remainder theorem there exists some polynomial~$P \in \kf[T]$ with
  \begin{equation}
  \label{congruences of P}
    P(T)
    \equiv
    \lambda_i
    \mod
    (T-\lambda_i)^{m_i}
  \end{equation}
  for every~$i = 1, \dotsc, n$.
  
  We may also assume that the polynomial~$P(T)$ has no constant term.
  If~$\lambda_i = 0$ for some~$i$ then this is entailed in~\eqref{congruences of P}.
  Otherwise we may replace the characteristic polynomial~$\chi(T)$ by the polynomial~$\chi(T) T$ before applying the chinese remainder theorem to ensure that, in addition to the conditions~\eqref{congruences of P}, also
  \[
    P(T)
    \equiv
    0
    \mod
    T \,.
  \]
  This means precisely that~$P(T)$ has no constant term.
  
  Let~$Q(T) \defined T - P(T)$.
  We construct the desired endomorphism~$x_s$ and~$x_n$ as
  \[
    x_s
    \defined
    P(x)
    \quad\text{and}\quad
    x_n
    \defined
    Q(x)  \,.
  \]
  Then~$x = x_s + x_n$ by choice of~$Q$, and the endomorphisms~$x_s$ and~$x_n$ commute because they are both polynomials in~$x$.
  
  To show that~$x_s$ is semisimple and~$x_n$ is nilpotent we use the generalized eigenspace decomposition
  \begin{gather*}
    V
    =
    \bigoplus_{i=1}^n V_i
  \shortintertext{where}
    V_i
    \defined
    \ker (x - \lambda_i \id_V)^{m_i}
  \end{gather*}
  is the generalized eigenspace of~$x$ with respect to the eigenvalue~$\lambda_i$.
  
  It follows from~\eqref{congruences of P} that the polynomial~$P(T)$ can for every~$i = 1, \dotsc, n$ be expressed as
  \[
    P(T)
    =
    \lambda_i + R(T) (T - \lambda_i)^{m_i}
  \]
  for some polynomial~$R(T) \in \kf[T]$.
  We plug in the endomorphism~$x$ to find that
  \[
    x_s
    =
    P(x)
    =
    \lambda_i \id_V + R(x) (x - \lambda_i \id_V)^{m_i}  \,.
  \]
  The endomorphism~$(x - \lambda_i \id_V)^{m_i}$ vanishes on the generalized eigenspace~$V_i$, hence we find that the endomorphisms~$x_s$ and~$\lambda_i \id_V$ act the same on this generalized eigenspace.
  The generalized eigenspace decomposition~$V = \bigoplus_{i=1}^n V_i$ is therefore a decomposition of~$V$ into eigenspaces with respect to~$x_s$.
  This shows that~$x_s$ is indecomposable.
  
  To show that~$x_n$ is nilpotent we show that~$x_n$ acts nilpotently on each subspace~$V_i$.
  For this we observe that
  \[
    Q(T)
    =
    T - P(T)
    =
    T - \lambda_i - R(T) (T - \lambda_i)^{m_i}
    =
    (T - \lambda_i) - R(T) (T - \lambda_i)^{m_i}
  \]
  and hence
  \[
    x_n
    =
    Q(x)
    =
    (x - \lambda_i \id_V) - R(x) (x - \lambda_i \id_V)^{m_i}  \,.
  \]
  We find as before that the endomorphisms~$x_n$ and~$x - \lambda_i \id_V$ act in the same way on the subspace~$V_i$.
  The endomorphism~$x - \lambda_i \id_V$ is nilpotent on the subspace~$V_i$ (by choice of~$V_i$), hence~$x_n$ is nilpotent on the subspace~$V_i$.
  As this holds for every~$i = 1, \dotsc, n$ we find altogether that~$x_n$ is nilpotent.
  
  We have now shown the existence for part~\ref*{existence and uniqueness of concrete jcd} and the existence of the polynomials~$P$ and~$Q$ for part~\ref*{concrete jcd are polynomial} (for the constructed decomposition).
  Part~\ref*{commuting via concrete jcd} follows from the equality~$x = x_s + x_n$ and part~\ref*{concrete jcd are polynomial}
  Part~\ref*{concrete jcd compatible with mapping of subspaces} follows from part~\ref*{concrete jcd are polynomial} because inductively~$x^i(W) \subseteq U$ for every~$i \geq 1$.
  
  It remains to show that every other decomposition~$x = y_s + y_n$ coincides with the constructed one.
  The parts~$y_s$ and~$y_n$ commute with~$x = y_s + y_n$ because~$y_s$ and~$y_n$ commute.
  It follows from part~\ref*{concrete jcd are polynomial} (for the given decomposition) that both~$y_s$ and~$y_n$ commute with both~$x_s$ and~$x_n$.
  We find in particular that the endomorphisms~$x_s$ and~$y_s$ as well as~$x_n$ and~$y_n$ commute.
  It follows that the differences~$x_s - y_s$ is again semisimple\footnote{Because~$x_s$ and~$y_s$ commute and are both diagonalizable, and are thus simultaneously diagonalizable.} and that the difference~$y_n - x_n$ is nilpotent.
  As
  \[
    x_s + x_n
    =
    x
    =
    y_s + y_n
  \]
  we have that
  \[
    x_s - y_s
    =
    y_n - x_n \,.
  \]
  This last endomorphism is both semisimple and nilpotent and hence the zero endomorphism, i.e.
  \[
    x_s - y_s
    =
    y_n - x_n
    =
    0 \,.
  \]
  This shows that~$x_s = y_s$ and~$x_n = y_n$.
\end{proof}


\begin{corollary}[Concrete Jordan decomposition for matrices]
  \label{concrete jordan decomposition for matrices}
  Let~$x \in \Mat_n(\kf)$.
  \begin{enumerate}
    \item
      There exist unique matrices~$x_s, x_n \in \Mat_n(\kf)$ such that
      \begin{enumerate}
        \item
          $x = x_s + x_n$,
        \item
          $x_s$ is semisimple and~$x_n$ is nilpotent,
        \item
          $x_s$ and~$x_n$ commute.
      \end{enumerate}
    \item
      The matrices~$x_s$ and~$x_n$ can be expressed as polynomials in~$x$ without constant term, i.e.\ there exist polynomials~$P,Q \in \kf[T]$ with~$P(0) = Q(0) = 0$ such that~$x_s = P(x)$ and~$x_n = Q(x)$.
    \item
      Any matrix~$y \in \Mat_n(\kf)$ commutes with~$x$ if and only if it commutes with both~$x_s$ and~$x_n$.
    \item
      If~$U \subseteq W$ are linear subspaces of~$\kf^n$ with~$x W \subseteq U$ then also~$x_s W \subseteq U$ and~$x_n W \subseteq U$.
    \qed
  \end{enumerate}
\end{corollary}


\begin{definition}
  Let~$x$ be an endomorphism of a finite dimensional vector space~$V$ (resp.\ a square matrix~$x \in \Mat_n(\kf)$).
  Then the decomposition~$x = x_s + x_n$ from \cref{concrete jordan decomposition} (resp.\ \cref{concrete jordan decomposition for matrices}) is the \defemph{\textup(concrete\textup) Jordan decomposition}\index{concrete Jordan decomposition}\index{Jordan decomposition!concrete} of~$x$.
  The element~\gls*{semisimple part} is the \defemph{\textup(concrete\textup) semisimple part}\index{semisimple!part} of~$x$ and the element~\gls*{nilpotent part} is the \defemph{\textup(concrete\textup) nilpotent part}\index{nilpotent!part} of~$x$.
\end{definition}


\begin{example}
  \label{computation of jd via jnf}
  Let~$x$ be an endomorphism of a finite dimensional vector space and let~$v_1, \dotsc, v_n$ be a basis of~$V$ with respect to which~$x$ is in Jordan normal form~$J \in \Mat_n(\kf)$.
  Then
  \[
    J
    =
    D + N
  \]
  for a diagonal matrix~$D$ and strictly upper triangular (and hence nilpotent) matrix~$N$, and the two matrices~$D$ and~$N$ commute.
  We find that~$D$ is the semisimple part of~$J$ and that~$N$ is the nilpotent part of~$J$.
  With respect to the choosen basis of~$V$ the semisimple part of~$x$ is now given by~$D$, and the nilpotent part of~$x$ is given by~$N$.
  
  This shows that in coordinates the concrete Jordan decomposition of an endomorphism~$x$ corresponds to the decomposition of its Jordan normal form~$J$ into a diagonal part~$D$ and a strictly upper triangular part~$N$.
\end{example}


\begin{warning}
  Every upper triangular matrix~$x$ can be written as the sum of a diagonal matrix~$D$ and strictly upper triangular matrix~$N$ in a unique way.
  But decomposition will in general not coincide with the Jordan decomposition of~$x$.
  Indeed, this holds if and only if~$D$ and~$N$ commute.
\end{warning}


\begin{definition}
  An element~$x \in \glie$ of a finite dimensional Lie~algebra~$\glie$ is~\defemph{\adsemisimple} if the endomorphism~$\ad(x)$ of~$\glie$ is semisimple.
\end{definition}


\begin{lemma}
  \label{ss and nilpotent implies ad-ss and ad-nilpotent}
  Let~$\glie$ be a Lie subalgebra of~$\gllie(V)$ for some finite dimensional vector space~$V$, or let~$\glie$ be a subalgebra of~$\gllie_n(\kf)$ for some~$n \geq 0$.
  \begin{enumerate}
    \item
      If~$x \in \glie$ is semisimple then~$x$ is also~$\ad$-semisimple.
      
      Suppose more specifically that~$\glie$ is a a subalgebra of~$\gllie(V)$.
      Let~$v_1, \dotsc, v_n$ be a basis of~$V$ consisting of eigenvectors of~$x$ with corresponding eigenvalues~$\lambda_1, \dotsc, \lambda_n \in \kf$, and let~$E_{ij}$ with~$i,j = 1, \dotsc, n$ be the associated basis of~$\gllie(V)$ given by~$E_{ij}(v_k) = \delta_{jk} v_i$ for all~$i,j,k = 1, \dotsc, n$.
      Then~$E_{ij}$ is an eigenvector of~$\ad_{\gllie(V)}$ with eigenvalues~$\lambda_i - \lambda_j$.
    \item
      If~$x \in \glie$ is nilpotent then~$x$ is also~$\ad$-nilpotent.
  \end{enumerate}
\end{lemma}


\begin{proof}
  It sufficies to consider~$\glie \subseteq \gllie(V)$.
  \begin{enumerate}[leftmargin=*]
    \item
      We have for all~$i,j,k = 1, \dotsc, n$ that
      \begin{align*}
        [x,E_{ij}](e_k)
        &=
        (x E_{ij} - E_{ij} x)(e_k)
        \\
        &=
        x E_{ij}(e_k) - E_{ij} x(e_k)
        \\
        &=
        \delta_{jk} x(e_i) - \lambda_k E_{ij}(e_k)
        \\
        &=
        \delta_{jk} \lambda_i e_i - \delta_{jk} \lambda_k e_i
        \\
        &=
        (\lambda_i - \lambda_k) \delta_{jk} e_i
        \\
        &=
        (\lambda_i - \lambda_j) \delta_{jk} e_i
        \\
        &=
        (\lambda_i - \lambda_j) E_{ij} e_k  \,.
      \end{align*}
      This calculation shows that~$E_{ij}$ is indeed an eigenvector for~$\ad(x) = [x,-]$ with eigenvalue~$\lambda_i - \lambda_j$.
      This shows that~$\ad_{\gllie(V)}(x)$ is semisimple.
      The restriction~$\restrict{\ad_\gllie(V)(x)}{\glie} = \ad_\glie(x)$ is therefore again semisimple.
    \item
      It was shown in \cref{nilpotent implies ad-nilpotent} that~$\ad_{\gllie(V)}(x)$ is nilpotent.
      The restrition~$\ad_\glie(x) = \restrict{\ad_{\gllie(V)}(x)}{\glie}$ is again nilpotent.
    \qedhere
  \end{enumerate}
\end{proof}


\begin{corollary}
  \label{concrete jordan decomposition compatible with adjoint representation}
  Let~$\glie$ be a Lie~subalgebra of~$\gllie(V)$ for some finite dimensional vector space $V$, or a Lie~subalgebra of~$\gllie_n(\kf)$ for some~$n \geq 0$.
  If~$x \in \glie$ has the Jordan decomposition~$x = x_s + x_n$ then~$\ad(x) = \ad(x_s) + \ad(x_n)$ is the Jordan decomposition of~$\ad(x)$.
\end{corollary}


\begin{proof}
  It follows from \cref{ss and nilpotent implies ad-ss and ad-nilpotent} that~$\ad(x_s)$ is semisimple and~$\ad(x_n)$ is nilpotent.
  The endomorphisms~$\ad(x_s)$ and~$\ad(x_n)$ commute because~$x_s$ and~$x_n$ commute and~$\ad$ is a homomorphism of Lie~algebras.
  The decomposition~$\ad(x) = \ad(x_s) + \ad(x_n)$ therefore satisfy the defining properties of the Jordan decomposition for~$\ad(x)$.
\end{proof}





\subsection{Cartan’s Criterion}


\begin{lemma}
  \label{cartans criterion technical lemma}
  Let~$A \subseteq B$ be linear subspaces of the Lie~algebra~$\gllie(V)$, where~$V$ is some finite dimensional vector space, and let
  \[
    T
    \defined
    \{
      z \in \gllie(V)
    \suchthat
      \ad(z)(B) \subseteq A
    \}  \,.
  \]
  If~$x \in T$ with~$\tr(xz) = 0$ for every~$z \in T$ then~$x$ is nilpotent.
\end{lemma}
% 
\begin{proof}
  Let $x = x_s + x_n$ be the concrete Jordan decomposition of $x$.
  We need to show that~$x_s = 0$.
  
  The concrete Jordan decomposition of~$\ad(x)$ is given by~$\ad(x) = \ad(x_s) + \ad(x_n)$ by \cref{concrete jordan decomposition compatible with adjoint representation}.
  It follows from~$\ad(x)(B) \subseteq A$ and \cref{concrete jordan decomposition} that also $\ad(x_s)(B) = \ad(x)_s(B) \subseteq A$.
  Hence also~$x_s \in T$.
  
  Let~$v_1, \dotsc, v_n$ be a basis of~$V$ with respect to which~$x_s$ is given by a diagonal matrix and~$x_n$ is given by a strictly upper triangular matrix (e.g.\ a basis such that~$x$ is in Jordan normal form, see \cref{computation of jd via jnf}).
  This is in particular a basis consisting of eigenvectors for~$x_s$ with corresponding eigenvalues~$\lambda_1, \dotsc, \lambda_n \in \kf$.
  We say that an endomorphism~$y \in \gllie(V)$ is \defemph{diagonal} if~$v_1, \dotsc, v_n$ are eigenvectors of~$y$, i.e.\ if~$y$ is represented by a diagonal matrix with respect to this basis.
  We write~$y = (\mu_1, \dotsc, \mu_n)$ to mean that~$y$ is the diagonal endomorphism with~$y(v_i) = \mu_i$.
  (We encourage the reader
  We observe that if~$y = (\mu_1, \dotsc, \mu_n)$ is a diagonal endomorphism contained in~$T$ then
  \[
    0
    =
    \tr(xy)
    =
    \tr((x_s + x_n)y)
    =
    \tr(x_s y) + \underbrace{\tr(x_n y)}_{=0}
    =
    \tr(x_s y)
    =
    \lambda_1 \mu_1 + \dotsb + \lambda_n \mu_n  \,.
  \]
  We used that~$\tr(x_n y) = 0$ because~$x_n$ is represented by a strictly upper triangular matrix with respect to the basis~$v_1, \dotsc, v_n$, and the same then also goes for~$x_n y$.
  We also used that~$x_s y$ is again diagonal with~$x_s y = (\lambda_1, \dotsc, \lambda_n)(\mu_1, \dotsc, \mu_n) = (\lambda_1 \mu_1, \dotsc, \lambda_n \mu_n)$.
  
  Note that in the case of~$\kf = \Real$ or even~$\kf = \Rational$ we could choose~$y = x_s$ to find that
  \begin{equation}
    \label{sum of squares is zero}
    0
    =
    \sum_{i=1}^n \lambda_i^2
  \end{equation}
  Then~$\lambda_1 = \dotsb = \lambda_n = 0$ and hence~$x_s = 0$.
  But in our general field~$\kf$ the choice~$y = x_s$ will not suffice, because it does not follow from~$\sum_{i=1}^n \lambda_i^2 = 0$ that already~$\lambda_i = 0$ for all (or even any)~$i$.
  
  Instead we will show that~$T$ contains sufficiently many diagonal endomorphisms~$y$ such that it follows from~$\tr(x_s y) = 0$ (for all such~$y$) that already~$x_s = 0$.
  The main problem will be to ensure that the needed diagonal endomorphisms~$y$ are contained in~$T$.
  We will do so by constructing~$y$ in such a way that~$\ad(y)$ is a polynomial without constant term in~$\ad(x_s)$.
  It then follows from~$\ad(x_s)(B) \subseteq A$ that also~$\ad(y)(B) \subseteq A$ and hence~$y \in T$.
  We will now establish two conditions that are necessary and sufficient to ensure the existence of such a polynomial~$p$.
  
  Let~$E_{ij}$ with~$i,j = 1, \dotsc, n$ be the basis of~$\gllie(V)$ induced by the basis~$v_1, \dotsc, v_n$ of~$V$, i.e.\ we have that~$E_{ij}(v_k) = \delta_{jk} v_i$ for all~$i,j,k = 1, \dotsc, n$.
  We have seen in \cref{ss and nilpotent implies ad-ss and ad-nilpotent} that~$E_{ij}$ is an eigenvector for~$\ad(x_s)$ with eigenvalue~$\lambda_i - \lambda_j$.
  If more generally~$y = (\mu_1, \dotsc, \mu_n)$ is any diagonal endomorphism then we see that~$E_{ij}$ is an eigenvector for~$\ad(y)$ with eigenvalue~$\mu_i - \mu_j$.
  We find that~$\ad(y) = p(\ad(x))$ for some polynomial~$p \in \kf[t]$ if and only if
  \[
    p(\lambda_i - \lambda_j)
    =
    \mu_i - \mu_j
  \]
  for all~$i, j = 1, \dotsc, n$.
  Recall that there exists for all pairwise different inputs~$a_1, \dotsc, a_n \in \kf$ and all desired outputs~$b_1, \dotsc, b_n \in \kf$ some polynomial~$p \in \kf[t]$ with~$p(a_i) = b_i$ for every~$i = 1, \dotsc, n$.
  The existence of~$p$ is hence equivalent to the condition
  \begin{equation}
    \label{difference condition}
    \lambda_i - \lambda_j
    =
    \lambda_k - \lambda_l
    \implies
    \mu_i - \mu_j
    =
    \mu_k - \mu_l
  \end{equation}
  holding for all~$i,j,k,l = 1, \dotsc, n$.
  But we also want~$p$ to have no constant term, i.e.\ that~$p(0) = 0$.
  This means that
  \[
    \lambda_i - \lambda_j = 0
    \implies
    \mu_i - \mu_j = 0
  \]
  has to hold for all~$i, j = 1, \dotsc, n$, i.e.\ that
  \begin{equation}
    \label{function condition}
    \lambda_i = \lambda_j
    \implies
    \mu_i = \mu_j
  \end{equation}
  for all~$i, j = 1, \dotsc, n$.
  
  Condition~\eqref{function condition} means precisely that~$\mu_i$ is a function in~$\lambda_i$ in the sense that there exists a function~$f \colon \{\lambda_1, \dotsc, \lambda_n\} \to \kf$ with~$\mu_i = f(\lambda_i)$ for every~$i = 1, \dotsc, n$.
  To ensure condition~\eqref{difference condition} we want this function~$f$ be be \enquote{additive}.
  For this to make sense let~$E$ be the additive subgroup of~$\kf$ generated by~$\lambda_1, \dotsc, \lambda_n$.
  Then for every additive function~$f \colon E \to \kf$ we get a diagonal endomorphism
  \[
    y_f
    \defined
    (f(\lambda_1), \dotsc, f(\lambda_n))  \,.
  \]
  We have seen that the construction of~$y_f$ ensures condition~\eqref{function condition}.
  Condition~\eqref{difference condition} holds by the additivity of~$f$ because
  \[
    \lambda_i - \lambda_j
    =
    \lambda_k - \lambda_l
    \implies
    f(\lambda_i - \lambda_j)
    =
    f(\lambda_k - \lambda_l)
    \implies
    f(\lambda_i) - f(\lambda_j)
    =
    f(\lambda_k) - f(\lambda_l)
  \]
  
  We have altogether figured out that any additive function~$f \colon E \to \kf$ results in a diagonal endomorphism~$y_f$ such that~$\ad(y_f)$ can be written as a polynomial without constant coefficient in~$\ad(x)$, so that~$y_f$ is again contained in~$T$, and thus
  \[
    0
    =
    \lambda_1 f(\lambda_1) + \dotsb + \lambda_n f(\lambda_n)  \,.
  \]
  
  We now return to our trick from~\eqref{sum of squares is zero}:
  If we restrict our attention to additive functions~$f \colon E \to \Integer$ then the term~$\sum_{i=1}^n \lambda_i f(\lambda_i)$ is again contained in~$E$, and it hence it makes sense to apply~$f$ to it.
  It then follows that
  \[
    0
    =
    \sum_{i=1}^n f(\lambda_i)^2 \,.
  \]
  The field~$\kf$ contains~$\Integer$ is a subring because~$\ringchar \kf = 0$, hence this is already an identity in~$\Integer$.%
  \footnote{For~$\ringchar \kf = p > 0$ this would only be an identity modulo~$p$.}
  We can therefore conclude that already~$f(\lambda_i) = 0$ for all~$i$ and hence that~$f = 0$ since~$E$ is generated by the~$\lambda_i$.
  We now observe that~$E \cong \Integer^m$ for some~$m \leq n$ since~$\ringchar \kf = 0$ (this follows for example from the classification of finiteley generated abelian groups because~$E$ is torsion-free since~$(\kf,+)$ is torsion-free).
  But we have shown that every additive function~$E \to \Integer$ is already zero.
  Hence~$E = 0$ and therefore~$\lambda_1 = \dotsb = \lambda_n = 0$.
\end{proof}


\begin{remark}
  One can also take~$E$ to be the rational span of~$\lambda_1, \dotsc, \lambda_n$ in~$\kf$, which makes sense because~$\Rational$ is the prime field of~$\kf$.
  One then considers for~$f$ not only additive functions~$E \to \Integer$ but all~{\linear{$\Rational$}} functions~$E \to \Rational$, i.e.\ elements of the dual space~$E^*$.
  The above argumentation then shows that~$E^* = 0$ and hence~$E = 0$.
\end{remark}


\begin{lemma}[Cartan’s criterion for~$\gllie(V)$]
  \label{cartans criterion for linear lie algebra}
  Let~$\glie$ be a Lie~subalgebra of~$\gllie(V)$ for some finite dimensional vector space~$V$.
  Then~$\glie$ is solvable if and only if~$\tr(xy) = 0$ for every~$x \in \glie$ and~$y \in [\glie, \glie]$.
\end{lemma}


\begin{proof}
  Suppose that $\glie$ is solvable.
  Then by Lie’s theorem there exists a basis of~$V$ with respect to which every~$x \in \glie$ is represented by an upper triangular matrix.
  Then every~$y \in [\glie,\glie]$ is represented by a strictly upper triangular matrix with respect to this basis.
  The product~$xy$ is then also represented by a stricly upper triangular matrix. 
  Hence~$\tr(xy) = 0$.
  
  Suppose that~$\tr(xy) = 0$ for every~$x \in \glie$ and~$y \in [\glie, \glie]$.
  To show that~$\glie$ is solvable we invoke \cref{solvable iff derived is nilpotent} and show that every~$x \in [\glie, \glie]$ is nilpotent.
  For this we will use \cref{cartans criterion technical lemma} with~$A \defined [\glie, \glie]$,~$B \defined \glie$ and
  \[
    T
    \defined
    \{
      x
      \in
      \gllie(V)
    \suchthat
      \ad(x)(B)
      \subseteq
      A
    \} \,.
  \]
  Observe that~$T$ contains~$\glie$ because~$\glie$ is a Lie~subalgebra of~$\gllie(V)$.
  
  It now sufficies to show that~$\tr(xz) = 0$ for all~$x \in [\glie, \glie]$ and~$z \in T$.
  We can then apply \cref{cartans criterion technical lemma} to see that every~$x \in [\glie, \glie]$ is nilpotent.
  It then follows from \cref{linear lie algebras consisting of nilpotent endomorphisms are nilpotent} that~$[\glie, \glie]$ is nilpotent.
  
  It sufficies to consider the case~$x = [a,b]$.
  Then
  \[
    \tr(x z)
    =
    \tr([a,b] z)
    =
    ([a,b], z)_{\tr}
    =
    (a, [b,z])_{\tr}
    =
    \tr(a [b,z])
    =
    0
  \]
  as desired.
  We used the associativity of the trace form~$(-,-)_{\tr}$ to shorten the calculation, and for the last equality the assumption that~$\tr(xy) = 0$ for all~$x \in \glie$ and~$y \in [\glie, \glie]$.
\end{proof}


\begin{theorem}[Cartan’s criterion for solvability]
  A finite dimensional Lie~algebra~$\glie$ is solvable if and only if~$\kappa(x,y) = 0$ for all~$x \in \glie$ and~$y \in [\glie, \glie]$.
\end{theorem}


\begin{proof}
  The Lie algebra~$\glie$ is solvable if and only if~$\glie' \defined \ad(\glie) = \glie/\centerlie(\glie)$ is solvable.
  By Cartan’s criterion for~$\gllie(\glie)$ this is the case if and only if~$\tr(x' y') = 0$ for all~$x' \in \glie'$ and~$y' \in [\glie', \glie']$.
  This is equivalent to~$\kappa(xy) = 0$ for all~$x, y \in [\glie, \glie]$ because~$[\glie', \glie'] = [\ad(\glie), \ad(\glie)] = \ad([\glie, \glie])$.
\end{proof}


\begin{corollary}
  \label{rad kappa is a solvable ideal}
  Let~$\glie$ be a finite dimensional Lie~algebra with Killing~form~$\kappa$.
  Then~$\rad \kappa$ is a solvable ideal in~$\glie$
\end{corollary}


\begin{proof}
  \Cref{radical of bilinear form is an ideal} shows that~$\rad \kappa$ is an ideal in~$\glie$.
  It follows from \cref{restriction of the killing form to an ideal} that
  \[
    \kappa_{\rad \glie}(x,y)
    =
    \kappa(x,y)
    =
    0
  \]
  for all~$x, y \in \rad \kappa$.
  Hence~$\rad \kappa$ is solvable by Cartan’s criterion.
\end{proof}


\begin{corollary}
  If~$\glie$ is a finite dimensional Lie~algebra with Killing form~$\kappa$ then~$\rad \kappa \subseteq \rad \glie$.
  \qed
\end{corollary}




