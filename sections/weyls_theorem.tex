\section{Weyl’s Theorem on complete reducibility}


\begin{lem}\label{lem: Casimirs endomorphism for Weyl}
 Let $\g$ be an $n$-dimensional semisimple Lie algebra, $V$ a finite dimensional vector space and $\rho \colon \g \to \gl(V)$ a representation of $\g$.
 \begin{enumerate}
  \item
   The map $\beta \colon \g \times \g \to k$ with $\beta(x,y) \coloneqq \tr(\rho(x)\rho(y))$ for all $x,y \in \g$ is an associative and symmetric bilinear form.
  \item
   If $\rho$ is a faithful representation then $\beta$ is non-degenerate.
  \item
   Suppose $\rho$ is faithful. Let $x_1, \dotsc, x_n$ ba a basis of $\g$ and $x^1, \dotsc, x^n$ the dual basis with respect to $\beta$. Then the map
   \[
    C_\beta^V \colon V \to V,
    \quad
    v \mapsto C_\beta \cdot v = \sum_{i=1}^n x_i.x^i.v = \sum_{i=1}^n \rho(x_i)\rho(x^i)(v)
   \]
   is a homomorphism of representations with $\tr(C_\beta^V) = \dim_k \g$.
 \end{enumerate}
\end{lem}
\begin{proof}
 \begin{enumerate}[leftmargin=*]
  \item
   This was already remarked in Remark~\ref{rem: associative bilinear form of a representation}.
  \item
   Because $\rad \beta \subideal \g$ is and $\rho$ is an isomorphism onto its image it follows that $\rho(\rad \beta) \subideal \rho(\g)$ is an ideal with $\tr(xy) = 0$ for all $x,y \in \rho(\g)$. By Cartan’s criterion for linear Lie algebras (see Lemma~\ref{lem: Cartans criterion for linear Lie algebra}) the ideal $\rho(\rad \beta)$ is solvable. Because $\rho(\g)$ is semisimple it follows that $\rho(\rad \beta) = 0$ and hence $\rad \beta = 0$ because $\rho$ is faithful.
  \item
   That $C_\beta^V$ is a homomorphism of representations was follows from Corollary~\ref{cor: Casimir homomorphism of a representation} and notice that
   \[
    \tr(C_\beta^V)
    = \tr\left( \sum_{i=1}^n \rho(x_i)\rho(x^i) \right)
    = \sum_{i=1}^n \tr(\rho(x_i)\rho(x^i))
    = \sum_{i=1}^n \beta(x_i, x^i)
    = n
    = \dim \g.
    \qedhere
   \]
 \end{enumerate}
\end{proof}


\begin{defi}\label{defi: casimir endomorphism of a faithful representation}
 Let $V$ be a faithful representation of a finite dimensional semisimple Lie algebra $\g$ and $\beta \colon \g \times \g \to k$ the associative, symmetric and non-degenerate bilinear form defined by $\beta(x,y) = \tr(\rho(x)\rho(y))$ for all $x,y \in \g$. Then define
 \[
  C^V_\g \colon V \to V, \quad v \mapsto C_\beta \cdot v.
 \]
\end{defi}


\begin{rem}
 Notice for the situation of Definition~\ref{defi: casimir endomorphism of a faithful representation} that if $x_1, \dotsc, x_n$ is a basis of $\g$ and $x^1, \dotsc, x^n$ is the dual basis with respect to $\beta$ then $C^V_\g$ can also be expressed as
 \[
  C^V_\g = \sum_{i=1}^n \rho(x_i) \rho(x^i).
 \]
\end{rem}


\begin{rem}
 Definition~\ref{defi: casimir endomorphism of a faithful representation} was not given in the lecture, but helps to bake the following easier to read.
\end{rem}


\begin{lem}\label{lem: decomposition of representations of semisimple Lie algebras}
 Let $g$ be a finite dimensional semisimple $k$-Lie algebra and let $V$ be a finite dimensional representation of $\g$. Then $V = \g V \oplus V^\g$ for the subrepresentations
 \[
  \g V \coloneqq \{x.v \mid x \in \g, v \in V\}
  \quad\text{and}\quad
  V^\g \coloneqq \{v \in V \mid \text{$x.v = 0$ for every $x \in \g$}\}
 \]
\end{lem}
\begin{proof}
 If $V = U_1 \oplus \dotsb \oplus U_n$ is a decomposition into subrepresentations of $\g$ then $\g V = \g U_1 \oplus \dotsb \oplus \g U_n$ and $V^\g = U_1^\g \oplus \dotsb \oplus U_n^\g$. Hence it sufficies to show the statement for indecomposable representations. Hence suppose that $V$ is indecomposable. It can also be assumed that $V \neq 0$.
 
 By replacing $\g$ with $\rho(\g)$ it can be assumed w.l.o.g.\ that $V$ is a faithful representation of $\g$. Notice that $\rho(\g)$ is also semisimple by Corollary~\ref{cor: ideals and quotients of semisimple again semisimple}.
 
 Now let $C \coloneqq C^V_\g \colon V \to V$ be the endomorphism of $V$ defined as in Lemma~\ref{lem: Casimirs endomorphism for Weyl}. Because $C$ is a homomorphism of representations of $\g$ and $V$ is finite dimensional it follows that $V$ decomposes into the generalized eigenspaces of $C$. Because $V$ is indecomposable by assumption it follows that $V$ is the generalized eigenspace of $C$ with respect to some $\lambda \in k$ (here it is used that $V \neq 0$ by assumption). Notice that by Lemma~\ref{lem: Casimirs endomorphism for Weyl}
 \[
  \dim_k \g = \tr(C) = (\dim_k V) \cdot \lambda.
 \]
 
 If $\lambda = 0$ then it follows that $\g = 0$ (this can also be seen directly because $\g$ acts faithful by assumption). Then $V = V^\g$ and the statement holds.
 
 Suppose otherwise that $\lambda \neq 0$. Then $\det C = \lambda^{\dim_k V} \neq 0$, hence $C$ is actully an automorphism of $V$, hence $C(V) = V$. As $C$ is given by multiplication with a Casimir element $C_\beta \in \Ue(\g)$ it follows that
 \[
  \g V
  = \{x.v \mid x \in \g, v \in V\}
  = \{x \cdot v \mid x \in \Ue(\g) \mid V\}
  \supseteq \{ C_\beta \cdot v \mid v \in V\}
  = C(V)
  = V,
 \]
 hence $\g V = V$.
\end{proof}


\begin{thrm}[Weyl]
 Let $V$ be finite dimensional representation of a finite dimensional semisimple Lie algebra $\g$. Then $V$ is completely reducible.
\end{thrm}
\begin{proof}
 It sufficies to show that any subrepresentation $U \subseteq V$ has a direct complement which is again a subrepresentation, i.e.\ that there exists a subrepresentation $W \subseteq V$ with $V = U \oplus W$.
 
 By Lemma~\ref{lem: decomposition of representations of semisimple Lie algebras} there exists a short exact sequence
 \[
  0 \to \g \Hom_k(V, U) \to \Hom_k(V,U) \xrightarrow{\pi_V} \Hom_k(V,U)^\g \to 0
 \]
 as well as a short exact sequence
 \[
  0 \to \g \Hom_k(U, U) \to \Hom_k(U,U) \xrightarrow{\pi_U} \Hom_k(U,U)^\g \to 0.
 \]
 The restriction map $\rho \colon \Hom_k(V,U) \to \Hom_k(U,U), f \mapsto f|_{U} = f \circ \iota$ is surjective, where $\iota \colon U \inc V$ denotes the canonical inclusion (notice that it a map between the $k$-linear maps). It is a homomorphism of representations of $\g$, hence induces a map $\bar{\rho} \colon \Hom_k(V,U)^\g \to \Hom_k(U,U)^\g$ by restriction. The maps so far fit in the following commutative diagram:
 \[
   \begin{tikzcd}
     \Hom_k(V,U)
     \arrow{r}[above]{\rho}
     \arrow{d}[left]{\pi_V}
     &
     \Hom_k(U,U)
     \arrow{d}[right]{\pi_U}
     \\
     \Hom_k(V,U)^\g
     \arrow{r}[above]{\tilde{\rho}}
     &
     \Hom_k(U,U)^\g
   \end{tikzcd}
 \]
 Because $\pi_U$ and $\rho$ are surjective it follows that $\bar{\rho}$ is also surjective. Thus there exists some $\pi \in \Hom_k(V,U)^\g$ with $\id_U = \bar{\rho}(\pi) = \pi \circ \iota$.
 
 Notice that $\Hom_k(V,U)^\g = \Hom_\g(V,U)$ and $\Hom_k(U,U)^\g = \Hom_\g(U,U)$ as already seen in Remark~\ref{rem: homomorphisms of representations as invariants}. Hence $\id_U = \pi \circ \iota$ is a statement about homomorphism of representations of $\g$, i.e.\ homomorphisms of $\Ue(\g)$-modules. It follows that $\pi$ is a retraction in the category of $\Ue(\g)$-modules, which is why $V = U \oplus \ker \pi$ with $\ker \pi \subseteq V$ being a subrepresentation of $\g$.
\end{proof}


\begin{rem}
 As a representation $V$ of a Lie algebra $\g$ is completely reducible if and only if it is semisimple as an $\Ue(\g)$-module one has the usual equivalent definitions of complete reducibility, i.e.\ the following are equivalent:
 \begin{enumerate}
  \item
   $V$ is a completely reducible, i.e.\ $V$ is the direct sum of irreducible subrepresentations.
  \item
   $V$ is the direct sum of irreducible subrepresentations.
  \item
   Every submodule $U \subseteq V$ has a direct summand which is also a subreprentation.
 \end{enumerate}
\end{rem}


\begin{rem}
 Weyl’s theorem does not necessarily hold for infine dimensional representations. To see this take $k[X]$ as a representation of $\sll_2(k)$ via
 \[
  \sll_2(k) \to \gl(k[X]), \quad
  e \mapsto \dd{X}, \quad
  h \mapsto -2X\dd{X}, \quad
  f \mapsto -x^2\dd{X}
 \]
 which was already shown in Examples~\ref{expls: representations of Lie algebras} to define a representation of $\sll_2(k)$. Notice that $k \subseteq k[X]$ is a subrepresentation. If $U \subseteq k[X]$ is a subrepresentation and $P \in U$ a polynomial of degree $n$ then by applying $e$ it turns out that $U$ contains of polynomial of degree $d$ for every $d \leq n$, which is why $U$ contains $1, X, \dotsc, X^n$ (here it is used that $\chara k = 0$). Moreover, if $X \in U$ then by applying $f$ it turns out that $X^d \in U$ for every $d \geq 1$. Hence if $U$ contains a non constant polynomial then already $X^d \in U$ for every $d \in \N$ and therefore $U = k[X]$.
 
 So $0,$ $k$ and $k[X]$ are the only subrepresentations of $k[X]$. in particular $k[X]$ is not completely reducible.
\end{rem}




