\section{Weyl’s Theorem on Complete Reducibility}


% TODO: Converse to Weyls theorem





\subsection{Casimir Elements}


\begin{convention}
  For this subsection let~$\glie$ be a finite dimensional Lie~algebra and let~$\beta$ be a non-degenerate associative symmetric bilinear form on~$\glie$.
\end{convention}


\begin{definition}
  \label{definition of casimir}
  Let $\varphi \colon \glie \tensor \glie^* \to \End_{\kf}(\glie)$ and~$\psi \colon \glie \to \glie^*$ be the isomorphisms of vector spaces given by
  \[
    \varphi(x \tensor \phi)(y) = \phi(y) x
    \qquad\text{and}\qquad
    \psi(x) = \beta(x, -)
  \]  
  for all~$x,y \in \glie$ and~$\phi \in \glie^*$.
  Then the image of~$1 \in \kf$ under the composition
  \begin{equation}
    \label{casimir without coordinates}
    \kf
    \xlongto{\lambda \mapsto \lambda \id_{\glie}}
    \End_{\kf}(\glie)
    \xlongto{\varphi^{-1}}
    \glie \tensor \glie^*
    \xlongto{\id_{\glie} \tensor \psi^{-1}}
    \glie \tensor \glie
    \xlongto{\text{multiplication}}
    \Univ(\glie)
  \end{equation}
  is the \defemph{Casimir~element of~$\beta$}\index{Casimir!element}, and it is denoted by~\gls*{bilinear casimir}.
\end{definition}


\begin{lemma}
  The Casimir~element~$C_\beta$ in central in the universal enveloping algebra~$\Univ(\glie)$.
\end{lemma}


\begin{proof}
  Let~$\varphi$ and~$\psi$ be as in \cref{definition of casimir}.
  
  It sufficies to show~$C_\beta$ commutes with every~$x \in \glie$ because~$\Univ(\glie)$ is generated by~$\glie$ as a~\algebra{$\kf$}.
  We thus need to show that~$[x, C_\beta] = 0$ for every~$x \in \glie$ (where the commutator bracket~$[-,-]$ is taken in~$\Univ(\glie)$).
  We let~$\glie$ act on~$\Univ(\glie)$ via~$x.a = [x,a]$ for all~$x \in \glie$ and~$a \in A$ (note that this is just the restriction of the adjoint action of~$\Univ(\glie)$ to an action of~$\glie$).
  This allows us to observe that every map in~\eqref{casimir without coordinates} is a homomorphism of~{\representations{$\glie$}}.
  \begin{itemize}
    \item
      The first map~$\kf \to \End_{\kf}(\glie)$ is a homomorphism of representations because~$\glie$ acts trivially on both~$\kf$ and on the {\onedimensional} linear subspace~$\kf \id_{\glie}$ of~$\End_{\kf}(\glie)$ (since~$\id_{\glie} \in \End_{\glie}(\glie) = \End_{\kf}(\glie)^{\glie}$).
    \item
      We known from \cref{list of homomorphism of representations} that~$\varphi$ is an isomorphism of representations.
    \item
      The map~$\psi \colon \glie \to \glie^*$ is an isomorphism of representations by \cref{associative non-degenerate bilinear forms induce isomorphism to the dual}.
      It follows from \cref{list of homomorphism of representations} that~$\id_{\glie} \tensor \psi^{-1}$ is again a homomorphism of representations.
    \item
      That the multiplication map~$\mu \colon \glie \tensor \glie \to \Univ(\glie)$ given by~$x \tensor y \mapsto xy$ is a homomorphism of representations can be seen from direct calculation, as we have for all~$x, y, z \in \glie$ that
      \begin{align*}
        \mu(x.(y \tensor z))
        &=
        \mu((x.y) \tensor z + y \tensor (x.z))
        \\
        &=
        (x.y)z + y(x.z)
        \\
        &=
        [x,y]z + y[x,z]
        \\
        &=
        xyz - yxz + yxz - yzx
        \\
        &=
        xyz - yzx
        \\
        &= [x,yz]
        \\
        &= x.(yz)
        \\
        &= x.\mu(y \tensor z) \,.
      \end{align*}
  \end{itemize}
  
  It now follows that the element~$1 \in \kf$ on which~$\glie$ acts trivially is mapped to the element~$C_\beta \in \Univ(\glie)$ on which~$\glie$ then also has to act trivially.
  By choice of the action of~$\glie$ on~$\Univ(\glie)$ this means that
  \[
    0
    =
    x.C_\beta
    =
    [x, C_\beta]
  \]
  for every~$x \in \glie$, as desired.
\end{proof}


\begin{corollary}
  \label{casimir homomorphism of a representation}
  Let~$V$ be any~{\module{$\Univ(\glie)$}} (i.e.~{\representation{$\glie$}}).
  Then the map
  \[
    C_\beta^V
    \colon
    V
    \to
    V \,,
    \quad
    v
    \mapsto
    C_\beta \cdot v
  \]
  is an~{\module{$\Univ(\glie)$}} endomorphism (and equivalently an endomorphism of~$V$ as a~{\representation{$\glie$}}).
\end{corollary}


\begin{proof}
  This holds because~$C_\beta$ is central in~$\Univ(\glie)$.
\end{proof}


\begin{lemma}[Casimir in coordinates]
  \label{casimir in coordinates}
  Let~$x_1, \dotsc, x_n$ be a basis of~$\glie$ and let~$x^1, \dotsc, x^n$ be the dual basis of~$\glie$ with respect to~$\beta$, i.e.\ $\beta(x_i, x^{\spacing j}) = \delta_{ij}$ for all $i,j = 1, \dotsc, n$.
  Then the Casimir~element~$C_\beta$ is given by
  \[
    C_\beta = \sum_{i=1}^n x_i x^i  \,.
  \]
\end{lemma}


\begin{proof}
  Let~$\psi$ be as in \cref{definition of casimir}.
  If~$x_1^*, \dotsc, x_n^*$ denotes the dual basis of~$\glie^*$ then~$x^i = \psi^{-1}(x_i^*)$ for every~$i = 1, \dotsc, n$.
  With this information we can follow the element~$1 \in \kf$ through its travel in~\eqref{casimir without coordinates} to see that
  \[
    1
    \longmapsto
    \id_{\glie}
    \longmapsto
    \sum_{i=1}^n x_i \tensor x_i^*
    \longmapsto
    \sum_{i=1}^n x_i \tensor x^i
    \longmapsto
    \sum_{i=1}^n x_i x^i  \,.
  \]
  The last term of this chain of mappings is by definition the Casimir~element~$C_\beta$.
\end{proof}


\begin{example}
  We have seen in \cref{killing form of sln} that the Killing~form of~$\sllie_2(\kf)$ is given by
  \[
    \kappa(x,y)
    =
    4\tr(x,y)
  \]
  for all~$x, y \in \glie$.
  On the standard basis
  \[
    e
    \defined
    \begin{pmatrix}
      0 & 1 \\
      0 & 0
    \end{pmatrix} \,,
    \qquad
    h
    \defined
    \begin{pmatrix*}[r]
      1 &  0  \\
      0 & -1
    \end{pmatrix*}  \,,
    \qquad
    f
    \defined
    \begin{pmatrix}
      0 & 0 \\
      1 & 0
    \end{pmatrix}
  \]
  of~$\sllie_2(\kf)$ the Killing~form~$\kappa$ is therefore represented by the matrix
  \[
    \begin{pmatrix}
      0
      &
      0
      &
      4
      \\
      0
      &
      8
      & 
      0
      \\
      4
      &
      0
      &
      0
    \end{pmatrix} \,.
  \]
  The dual basis of the standard basis~$e$,~$h$,~$f$ is therefore given by
  \[
    e^*
    =
    \frac{1}{4} \spacing f \,,
    \qquad
    h^*
    =
    \frac{1}{8} h \,,
    \qquad
    f^*
    =
    \frac{1}{4} e \,.
  \]
  The Casimir~element~$C_\kappa$ is therefore given by
  \[
    C_\kappa
    =
    e e^* + h h^* + f {} f^*
    =
    \frac{1}{4} ef + \frac{1}{8} h^2 + \frac{1}{4} \spacing fe \,.
  \]
  By writing~$fe = ef + [\spacing f,e] = ef - h$ we may also write~$C_\kappa$ as
  \[
    C_\kappa
    =
    \frac{1}{4} ef + \frac{1}{8} h^2 + \frac{1}{4} (ef - h)
    =
    \frac{1}{2} ef + \frac{1}{8} h^2 - \frac{1}{4} h
  \]
  but we could also use the relation~$ef = fe + [e,f] = fe + h$ to write~$C_\beta$ as
  \[
    C_\kappa
    =
    \frac{1}{4} ({\spacing}fe + h) + \frac{1}{8} h^2 + \frac{1}{4} \spacing fe
    =
    \frac{1}{8} h^2 + \frac{1}{4} h + \frac{1}{2} \spacing fe  \,.
  \]
\end{example}


% TODO: (How) does the Casimir~element change if we use (\spacing g^* \tensor g) instead of (\spacing g \tensor g^*)?


\begin{remark}
  That the Casimir~element~$C_\beta$ is central in~$\glie$ can also be shown with the above coordinate description of~$C_\beta$:
  
  Let~$x_1, \dotsc, x_n \in \glie$ be basis of~$\glie$ and let~$x \in \glie$.
  We need to show that~$[x, C_\beta] = 0$.
  We can for every~$i \in I$ write the commutators~$[x, x_i]$ and~$[x, x^i]$ as linear combinations
  \[
    [x, x_i]
    =
    \sum_{j=1}^n a_{ij} x_j
    \qquad\text{and}\qquad
    [x, x^i]
    =
    \sum_{j=1}^n b_{ij} x^{\spacing j} \,.
  \]
  Then on the one hand
  \[
    \beta([x, x_i], x^{\spacing j})
    =
    \sum_{k=1}^n a_{ik} \beta(x_k, x^{\spacing j})
    =
    a_{ij}
  \]
  but on the other hand also
  \[
    \beta([x, x_i], x^{\spacing j})
    =
    -\beta([x_i, x], x^{\spacing j})
    =
    -\beta(x_i, [x, x^{\spacing j}])
    =
    -\sum_{k=1}^n b_{jk} \beta(x_i, x^k)
    =
    -b_{ji}
  \]
  for all~$i, j = 1, \dotsc, n$.
  Hence~$a_{ij} = -b_{ji}$ for all~$i, j = 1, \dotsc, n$.
  It follows that
  \begin{align*}
    x C_\beta - C_\beta x
    &=
    \sum_{i=1}^n x x_i x^i - \sum_{i=1}^n x_i x^i x
    \\
    &=
      \sum_{i=1}^n x x_i x^i
    - \sum_{i=1}^n x_i x x^i
    + \sum_{i=1}^n x_i x x^i
    - \sum_{i=1}^n x_i x^i x
    \\
    &=
    \sum_{i=1}^n [x, x_i] x^i - \sum_{i=1}^n x_i [x^i, x]
    \\
    &=
    \sum_{i,j=1}^n a_{ij} x_j x^i + \sum_{i,j=1}^n b_{ij} x_i x^{\spacing j}
    \\
    &=
    \sum_{i,j=1}^n a_{ij} x_j x^i - \sum_{i,j=1}^n a_{ij} x_j x^i
    \\
    &=
    0 \,.
  \end{align*}
\end{remark}





\subsection{Weyl’s Theorem}


\begin{lemma}
  \label{construction of casimir endomorphism}
  Let~$\glie$ be Lie~algebra,~$V$ a finite dimensional vector space and~$\rho \colon \glie \to \gllie(V)$ a representation of~$\glie$.
  \begin{enumerate}
    \item
      The map~$\beta \colon \glie \times \glie \to \kf$ given by~$\beta(x,y) \defined \tr(\rho(x)\rho(y))$ for all~$x,y \in \glie$ is an associative and symmetric bilinear form on~$\glie$.
  \end{enumerate}
  Suppose now that the representation~$(V,\rho)$ is faithful and that~$\glie$ is~{\dimensional{$n$}} and semisimple.
  \begin{enumerate}[resume]
    \item
      If the representation~$V$ is faithful then~$\beta$ is non-degenerate.
    \item
      The homomorphism of representations~$C^V \colon V \to V$,~$v \mapsto C_\beta \cdot v$ has trace~$\tr(C^V) = \dim_{\kf} \glie$.
  \end{enumerate}
\end{lemma}


\begin{proof}
  \leavevmode
  \begin{enumerate}
    \item
      See \cref{associative bilinear form of a representation}.
    \item
      The radical of~$\beta$ is an ideal in~$\glie$ and~$\rho$ is an isomorphism onto its image.
      It follows that~$\rho(\rad \beta)$ is an ideal in~$\rho(\glie)$ with~$\tr(xy) = 0$ for all~$x,y \in \rho(\rad \beta)$.
      By Cartan’s criterion for linear Lie~algebras (\cref{cartans criterion for linear lie algebra}) the ideal~$\rho(\rad \beta)$ is solvable.
      The Lie~algebra~$\rho(\glie) \cong \glie$ is semisimple whence~$\rho(\rad \beta) = 0$.
      Then~$\rad \beta = 0$ because~$\rho$ is injective.
    \item
      \Cref{casimir homomorphism of a representation} shows that~$C^V$ is a homomorphism of representations.
      If~$x_1, \dotsc, x_n$ is a basis of~$\glie$ and~$x^1, \dotsc, x^n$ is the dual basis with respect to~$\beta$ then~$C_\beta = \sum_{i=1}^n x_i x^i$ by \cref{casimir in coordinates} and thus
      \[
        \tr(C^V)
        =
        \tr\left( \sum_{i=1}^n \rho(x_i)\rho(x^i) \right)
        =
        \sum_{i=1}^n \tr(\rho(x_i)\rho(x^i))
        =
        \sum_{i=1}^n \beta(x_i, x^i)
        =
        n
      \]
      as claimed.
    \qedhere
  \end{enumerate}
\end{proof}


\begin{definition}
  \label{casimir endomorphism of a faithful representation}
  Let~$V$ be a faithful representation of a finite dimensional semisimple Lie~algebra~$\glie$ and let~$\beta \colon \glie \times \glie \to \kf$ the associative, symmetric, non-degenerate bilinear form defined by~$\beta(x,y) = \tr(\rho(x)\rho(y))$ for all~$x,y \in \glie$.
  Then the endomorphism~$C^V$ of~$V$ given by
  \[
    C^V
    \colon
    V
    \to
    V \,,
    \quad
    v
    \mapsto
    C_\beta \cdot v
  \]
  is the \defemph{Casimir endomorphism}\index{Casimir!endomorphism} of~$V$.
\end{definition}
 

\begin{proposition}[Canonical decomposition]
  \label{decomposition of representations of semisimple lie algebras}
  Let~$\glie$ be a finite dimensional semisimple Lie~algebra and let~$V$ be a finite dimensional representation of~$\glie$.
  Then~$V = \glie V \oplus V^{\glie}$.
\end{proposition}


\begin{proof}
  There exists a decomposition~$V = U_1 \oplus \dotsb U_n$ into indecomposable subrepresentations.
  We have
  \begin{align*}
    \glie V
    &=
    \glie U_1 \oplus \dotsb \oplus \glie U_n
  \shortintertext{and}
    V^{\glie}
    &=
    U_1^{\glie} \oplus \dotsb \oplus U_n^{\glie}
  \end{align*}
  It therefore sufficies to consider the case that~$V$ is indecomposable.
  
  If~$\rho \colon \glie \to \gllie(V)$ is the Lie~algebra homomorphism that corresponds to the action of~$\glie$ on~$V$ then we may replace~$\glie$ with~$\rho(\glie)$ to assume that the representation~$V$ is faithful.
  Note that~$\rho(\glie)$ again semisimple by Corollary~\ref{ideals and quotients of semisimple again semisimple}.
  
  We can now consider the Casimir~endomorphism~$C \defined C^V \colon V \to V$. 
  We can decompose~$V$ into the generalized eigenspaces of~$C$, each of which is a subrepresentation of~$V$ because~$C$ is an endomorphism of~$V$ as a representation.
  It follows that~$V$ is the generalized eigenspace of~$C$ with respect to some eigenvalue~$\lambda \in \kf$ because~$V$ is indecomposable.
  We know from \cref{construction of casimir endomorphism} that
  \[
    \dim_\kf \glie
    =
    \tr(C)
    =
    \dim_{\kf}(V) \cdot \lambda  \,.
  \]
  
  We now distinguish between the two cases~$\lambda \neq 0$ and~$\lambda \neq 0$:
  If~$\lambda = 0$ then it follows that~$\dim \glie = 0$ and hence~$\glie = 0$.
  Then $V = V^{\glie}$.
  If~$\lambda \neq 0$ then~$\det C = \lambda^{\dim_{\kf} V} \neq 0$, hence~$C$ is actully an automorphism of~$V$, whence~$C(V) = V$.
  The Casimir endomorphism~$C$ is given by multiplication with the Casimir~element~$C_\beta \in \Univ(\glie)$ (where~$\beta$ is the associative, symmetric, non-degenerate bilinear form on~$V$ as described in \cref{construction of casimir endomorphism}).
  Therefore
  \[
    V
    =
    C(V)
    =
    C_\beta \cdot V
    \subseteq
    \glie V
  \]
  and hence~$V = \glie V$.
\end{proof}


\begin{corollary}
  \label{everything is projective and injective}
  Let~$V$ and~$W$ be two finite dimensional representations of a finite dimensional semisimple Lie~algebra~$\glie$.
  Let~$f \colon V \to W$ be a homomorphism of representations.
  Suppose that~$f$ is surjective (resp.\ injective).
  \begin{enumerate}
    \item
      \label{restrictions again sur or inj}
      The restrictions~$\glie V \to \glie W$ and~$V^{\glie} \to W^{\glie}$ are again surjective (resp.\ injective).
  \end{enumerate}
  Suppose now that~$U$ is another finite dimensional representation of~$\glie$.
  \begin{enumerate}[resume]
    \item
      \label{everything projective}
      For every finite dimensional~{\representation{$\glie$}}~$U$ the induced linear map
      \[
        f_*
        \colon
        \Hom_{\glie}(U,V)
        \to
        \Hom_{\glie}(U,W)
      \]
      is again surjective (resp.\ injective).
    \item
      \label{everything injective}
      For every finite dimensional~{\representation{$\glie$}}~$U$ of~$\glie$ the induced linear map
      \[
        f^*
        \colon
        \Hom_{\glie}(W,U)
        \to
        \Hom_{\glie}(V,U)
      \]
      is injective (resp.\ surjective).
  \end{enumerate}
\end{corollary}


\begin{proof}
  \leavevmode
  \begin{enumerate}
    \item
      If~$f_1$ denotes the restriction~$\glie V \to \glie W$ and~$f_2$ denotes the restrictions~$V^{\glie} \to W^{\glie}$ then with respect to the decompositions~$V = \glie V \oplus V^{\glie}$ and~$W = \glie W \oplus W^{\glie}$ we have~$f = f_1 \oplus f_2$.
      Thus~$f$ is surjective (resp.\ injective) if and only if both~$f_1$ and~$f_2$ are.
    \item
      The induced linear map~$f_* \colon \Hom_{\kf}(U,V) \to \Hom_{\kf}(U,W)$ is a homomorphism of representations by \cref{list of homomorphism of representations}.
      We can now apply part~\ref*{restrictions again sur or inj} with~$\Hom_{\glie}(U,V) = \Hom_{\kf}(U,V)^{\glie}$ and~$\Hom_{\glie}(U,W) = \Hom_{\kf}(U,W)^{\glie}$.
    \item
      The induced linear map~$f^* \colon \Hom_{\kf}(W,U) \to \Hom_{\kf}(V,U)$ is a homomorphism of representations by \cref{list of homomorphism of representations}.
      We can now again apply part~\ref*{restrictions again sur or inj}.
    \qedhere
  \end{enumerate}
\end{proof}


\begin{theorem}[Weyl]
  Let~$\glie$ be a finite dimensional semisimple Lie~algebra.
  Then every finite dimensional representation~$V$ of~$\glie$ is completely reducible.
\end{theorem}


\begin{proof}
  It sufficies to show that any subrepresentation~$U$ of~$V$ admits a direct complement that is again a subrepresentation.
  The inclusion~$\iota \colon U \to V$ is injective whence the induced linear map~$\iota_* \colon \Hom_{\glie}(V, U) \to \Hom_{\glie}(U, U)$ is surjective by \cref{everything is projective and injective}.
  It follows that there exists a homomorphism of representations~$\sigma \colon V \to U$ with~$\id_U = \iota^*(\sigma) = \sigma \circ \iota$, i.e.\ a split for~$\iota$.
  Now
  \[
    V
    =
    \im \iota \oplus \ker \sigma
    =
    U \oplus \ker \sigma
  \]
  with~$\ker \sigma$ being a subrepresentation of~$V$.
\end{proof}


\begin{remark}
  Let~$\glie$ be a Lie~algebra and let~$\catA$ be the category of finite dimensional representation of~$\glie$.  
  Part~\ref*{everything projective} of \cref{everything is projective and injective} shows that for every representation~$V \in \catA$ the left-exact functor
  \[
    \Hom_{\glie}(V, -)
    \colon
    \catA
    \to
    \cVect{\kf}
  \]
  is already exact, which means that every~$V \in \catA$ is projective.
  Part~\ref*{everything injective} of \cref{everything is projective and injective} also shows that for every representation~$V \in \catA$ the left-exact functor
  \[
    \Hom_{\glie}(-, V)
    \colon
    \catA^\op
    \to
    \cVect{\kf}
  \]
  is already exact, which means that every~$V \in \catA$ is injective.
  Each of these conditions is equivalent to~$\catA$ being semisimple, which is a reformulation of Weyl’s~theorem.
\end{remark}


\begin{warning}
  Weyl’s theorem does not generalize to infinite dimensional representations.
  We can more explicitly answer the following question:
  \begin{center}
    \emph{Question}:
    For which Lie~algebras is every representation completely reducible?
  \end{center}
  For this we reformulate this question:
  \begin{center}
    \emph{Question}:
    For which Lie~algebras~$\glie$ is every~{\module{$\Univ(\glie)$}} semisimple?
  \end{center}
  The answer turns out to be surprisingly easy:
  \begin{center}
    \emph{Answer}:
    Only~$\glie = 0$.
  \end{center}
  Indeed, if~$\glie = 0$ then~$\Univ(\glie) = \kf$ so that~{\modules{$\Univ(\glie)$}} are the same as vector spaces.
  
  If~$\glie \neq 0$ then~$\Univ(\glie)$ is infinite dimensional by the PBW~theorem.
  The~{\module{$\Univ(\glie)$}} module structure of~$M \defined \Univ(\glie)$ makes~$M$ into a~{\representation{$\glie$}} whose subrepresentations are precisely the left ideals of~$\Univ(\glie)$.
  We now observe that every nonzero left ideal in~$\Univ(\glie)$ is infinite dimensional as for every nonzero~$x \in \Univ(\glie)$ the cyclic left ideal~$\Univ(\glie) x = \gen{x}_{\Univ(\glie)}$ is infinite dimensional by \cref{uea contains no zero divisors}.
  
  It follows that in~$\Univ(\glie)$ no proper ideal of finite codimension admits a direct complement that is again an ideal.
  But the zero Lie~algebra homomorphism~$\glie \to \kf$ induces an algebra homomorphism~$\varepsilon \colon \Univ(\glie) \to \kf$ --- which is necessarily surjective because~$1_{\Univ(\glie)} \mapsto 1_{\kf}$ --- whose kernel is an ideal in~$\Univ(\glie)$ of codimension~$1$. 
  This shows that~$\ker(\varepsilon)$ is a~{\subrepresentation{$\glie$}} of~$M$ that admits no direct complement.
  
%     \item
%       Weyl’s theorem does not necessarily hold for infine dimensional representations. To see this take $k[X]$ as a representation of $\sll_2(k)$ via
%       \[
%         \sll_2(k) \to \gl(k[X]), \quad
%         e \mapsto \dd{X}, \quad
%         h \mapsto -2X\dd{X}, \quad
%         f \mapsto -x^2\dd{X}
%       \]
%       which was already shown in Examples~\ref{expls: representations of Lie algebras} to define a representation of $\sll_2(k)$. Notice that $k \subseteq k[X]$ is a subrepresentation. If $U \subseteq k[X]$ is a subrepresentation and $P \in U$ a polynomial of degree $n$ then by applying $e$ it turns out that $U$ contains of polynomial of degree $d$ for every $d \leq n$, which is why $U$ contains $1, X, \dotsc, X^n$ (here it is used that $\chara k = 0$). Moreover, if $X \in U$ then by applying $f$ it turns out that $X^d \in U$ for every $d \geq 1$. Hence if $U$ contains a non constant polynomial then already $X^d \in U$ for every $d \in \N$ and therefore $U = k[X]$.
%       
%       So $0,$ $k$ and $k[X]$ are the only subrepresentations of $k[X]$. in particular $k[X]$ is not completely reducible.
\end{warning}


\begin{remark}[Semisimplicity for Hopf~algebras]
  The above results falls into a more general picture:
  The universal enveloping algebra~$\Univ(\glie)$ of a Lie~algebra~$\glie$ inherits from the Lie~algebra structure of~$\glie$ the additional structure of a \defemph{Hopf~algebra}\index{Hopf algebra}.
  Part of the structure of a Hopf~algebra~$H$ is an algebra homomorphism~$\varepsilon \colon H \to \kf$.
  One can then show that the following conditions are equivalent:
  \begin{equivalenceslist}
    \item
      \label{hopf is semisimple}
      $H$ is semisimple, i.e.~every~{\module{$H$}} is semisimple.
    \item
      \label{counit kernel has complement}
      The ideal~$\ker(\varepsilon)$ admits a direct complement in~$H$ that is again a left ideal.
    \item
      \label{hopf admits unnormed integral}
      There exists some~$z \in H$ with~$xz = \varepsilon(x)z$ for every~$x \in H$ such that~$\varepsilon(z) \neq 0$.
    \item
      \label{hopf admits normed integral}
      There exists some~$z \in H$ with~$xz = \varepsilon(x)z$ for every~$x \in H$ such that~$\varepsilon(z) = 1$.
  \end{equivalenceslist}
  The implications between these conditions are as follows:
  \begin{implicationlist}
    \item[\ref*{hopf is semisimple}~$\implies$~\ref*{counit kernel has complement}]
      The kernel~$\ker(\varepsilon)$ is an~{\submodule{$H$}} of~$H$ and thus admits a direct complement.
    \item[\ref*{counit kernel has complement}~$\implies$~\ref*{hopf admits unnormed integral}]
      A complement~$I$ of~$\ker(\varepsilon)$ is necessarily {\onedimensional} and thus spanned by a single nonzero element~$z$.
      We find that~$\varepsilon(z) \neq 0$ because otherwise~$\varepsilon = 0$, which would contradicts~$\varepsilon$ being an algebra homomorphism (as~$\varepsilon(1) = 1$).
      It follows for every~$x \in H$ from~$z \in I$ that~$xz \in I = \gen{z}_\kf$ so that~$xz = \lambda z$ for some~$\lambda \in \kf$.
      Then~$\varepsilon(xz) = \varepsilon(x)\varepsilon(z)$ but also~$\varepsilon(xz) = \varepsilon(\lambda z) = \lambda \varepsilon(z)$ so that~$\lambda = \varepsilon(x)$.
    \item[\ref*{hopf admits unnormed integral}~$\implies$~\ref*{hopf admits normed integral}]
      We may replace~$z$ by~$z/\varepsilon(z)$.
    \item[\ref*{hopf admits normed integral}~$\implies$~\ref*{hopf is semisimple}]
      For this implications one needs some basic notions about Hopf~algebras and their representations.
      The actual proof is a direct copy of the proof of Maschke’s theorem about representations of a finite group~$G$ with~$\ringchar(\kf) \mid \abs{G}$.
      The element~$z$ plays the role of the averaging element~$\frac{1}{\abs{G}} \sum_{g \in G} g$.%
      \footnote{The group algebra~$\kf[G]$ is actually also a Hopf~algebra and the algebra homomorphism~$\varepsilon \colon \kf[G] \to \kf$ is given by~$\varepsilon(\spacing g) = 1$ for every~$g \in g$, so that~$\varepsilon(\sum_{g \in G} a_g g) = \sum_{g \in G} a_g$.
      We can therefore apply the equivalence of the given conditions to the group algebra~$\kf[G]$:
      
      An element~$z \in \kf[G]$ satisfies~$xz = \varepsilon(x)z$ for all~$x \in \kf[G]$ if and only if~$gz = z$ for every~$g \in G$.
      This means that~$z = c \sum_{g \in G} g$ for some~$c \in \kf$ if~$G$ is finite, and~$z = 0$ if~$G$ is infinite.
      Hence~$\kf[G]$ is not semisimple if~$G$ is infinite.
      If~$G$ is finite then~$\varepsilon(z) = c \abs{G}$.
      Therefore~$\varepsilon(z) \neq 0$ is possible if and only if~$\abs{G} \neq 0$ in~$\kf$, i.e.\ if and only if~$\ringchar(\kf) \nmid \abs{G}$.
      The conditions~$\varepsilon(z) = 1$ then means that~$z = \frac{1}{|G|} \sum_{g \in G}$.
      }
  \end{implicationlist}
  It can moreover be shown that if an element~$z$ as in condition~\ref*{hopf admits normed integral} exists then it is unique.
  
  It can also be shown that if a Hopf~algebra~$H$ contains a finite dimensional nonzero one-sided ideal then~$H$ is already finite dimensional itself.
  Whence an infinite dimensional Hopf~algebra can never be semisimple, as~$\ker(\varepsilon)$ cannot have a complement that is again a left ideal.
  
  In the light of these general facts about Hopf~algebras it is clear how to proceed for the universal enveloping algebra~$\Univ(\glie)$ of a Lie~algebra~$\glie$:
  If~$\glie \neq 0$ then~$\Univ(\glie)$ is infinite dimensional by the PBW~theorem and thus cannot be semisimple by the general theory of Hopf~algebras.
  Moreover, the kernel of~$\varepsilon$ is then an ideal in~$\Univ(\glie)$ that won’t admit a direct complement.
\end{remark}




