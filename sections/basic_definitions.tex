\section{Basic Definitions}





\subsection{Definition and Examples of Lie~Algebras}


\begin{definition}
  Let~$\glie$ be a vector space over some field~$\kf$.
  A~{\bilinear{$\kf$}} map
  \[
    \gls*{lie bracket}
    \colon
    \glie \times \glie
    \to
    \glie
  \]
  is called a \defemph{Lie~bracket}\index{Lie!bracket} if it satisfies the following two conditions:
  \begin{enumerate}
    \item
    $[-, -]$ is \defemph{alternating}\index{alternating}, i.e.~$[x,x] = 0$ for every~$x \in \glie$.
    \item
    $[-, -]$ satisfies the \defemph{Jacobi identity}\index{Jacobi identity}
    \[
      [x,[y,z]] + [y,[z,x]] + [z,[x,y]]
      =
      0
    \]
    for all~$x,y,z \in \glie$.
  \end{enumerate}
  A~{\vectorspace{$\kf$}}~$\glie$ together with a Lie~bracket~$[-,-]$ is called a~\defemph{\liealgebra{$\kf$}}\index{Lie!algebra}.
\end{definition}


\begin{remark}
  A Lie~bracket~$[-, -]$ on a vector space~$\glie$ is always antisymmetric, i.e.~$[y,x] = -[x,y]$ for all~$x,y \in \glie$, because
  \[
    0
    =
    [x+y, x+y]
    =
    [x,x] + [x,y] + [y,x] + [y,y]
    =
    [x,y] + [y,x] \,.
  \]
\end{remark}


\begin{remark}
  By using the antisymmetry of the Lie~bracket the Jacobi identity\index{Jacobi identity} can be rewritten as
  \[
    [x,[y,z]]
    =
    [[x,y],z] + [y,[x,z]]
  \]
  for all~$x, y, z \in \glie$.
\end{remark}


\begin{examples}
  \leavevmode
  \begin{enumerate}
    \item
      Any vector space~$\glie$ becomes a Lie~algebra via~$[x,y] = 0$ for all~$x,y \in \glie$.
    \item
      Any \emph{associative}~{\algebra{$\kf$}}~$A$ becomes a~{\liealgebra{$k$}} via~$[a,b] \defined ab - ba$ for all~$a, b \in A$.
      Then~$[-, -]$ is alternating and it follows from the associativity of the multiplication of~$A$ that
      \begin{align*}
         {}&  [a,[b,c]] + [b,[c,a]] + [c,[a,b]] \\
        ={}&  [a, (bc-cb)] + [b, (ca-ac)] + [c, (ab-ba)] \\
        ={}&  a(bc-cb)-(bc-cb)a + b(ca-ac) - (ca-ac)b + c(ab-ba) - (ab-ba)c \\
        ={}&  abc - acb - bca + cba + bca - bac - cab + acb + cab - cba - abc + bac \\
        ={}&  0
    \end{align*}
    for all~$a, b, c \in A$.
    The element~$[a,b]$ is the \defemph{commutator}\index{commutator} of~$a$ and~$b$.
   
    The following two are important examples of this construction:
    \begin{enumerate}
      \item
        The~{\algebra{$\kf$}} of~($n \times n$)-matrices~$\Mat_n(\kf)$ becomes a Lie~algebra via
        \[
          [A,B]
          \defined
          AB - BA
        \]
        for all~$A, B \in \Mat_n(\kf)$.
        This Lie~algebra is called the \defemph{general linear Lie~algebra}\index{general linear Lie algebra} and is denoted by~\gls*{general lie matrix}.
      \item
        More generally for any~{\vectorspace{$\kf$}} the~{\algebra{$\kf$}}~$\End_k(V)$ becomes a Lie~algebra via
        \[
          [\phi_1, \phi_2]
          \defined
          \phi_1 \circ \phi_2 - \phi_2 \circ \phi_1
        \]
        for all~$\phi_1, \phi_2 \in \End_k$.
        This Lie~algebra is called the \defemph{general linear Lie~algebra of~$V$}\index{general linear Lie algebra} and is denoted by~\gls*{general lie endomorphism}.
    \end{enumerate}
  \end{enumerate}
\end{examples}


\begin{recall}
  For calculations in~$\gllie_n(\kf)$ it is often useful to remember that
  \[
    E_{ij} E_{kl}
    =
    \begin{cases}
      E_{il}  & \text{if~$j = k$} \,, \\
      0       & \text{otherwise}  \,,
    \end{cases}
  \]
  where~$E_{ij}$ with~$1 \leq i,j \leq n$ denotes the standard basis of~$\gllie_n(\kf)$.
  One may think about the matrix~$E_{ij}$ as \enquote{going from~$j$ to~$i$}.
  The composition~$E_{ij} E_{kl}$ does then \enquote{go from~$l$ to~$i$} if the positions~$j$ and~$k$ fit together;
  if they do not fit, then the composition is simply zero.
  
  This intuition can be formalized by observing that
  \[
    E_{ij} e_k
    =
    \begin{cases}
      e_i & \text{if~$j = k$} \,, \\
      0   & \text{otherwise}  \,.
    \end{cases}
  \]
  This means that the matrix~$E_{ij}$ maps one of the standard basis vectors~$e_1, \dotsc, e_n$ (namely~$e_j$) to another standard basis vector (namely~$e_i$), but filters out all other standard basis vectors.
\end{recall}



\begin{definition}
  Let~$\glie$ be a~{\liealgebra{$\kf$}}.
  \begin{enumerate}
    \item
      A \defemph{Lie~subalgebra}\index{Lie!subalgebra} of~$\glie$ is a~{\linear{$\kf$}} subspace~$\hlie \subseteq \glie$ such that~$[x,y] \in \hlie$ for all~$x, y \in \hlie$.
    \item
      A \defemph{Lie~ideal}\index{Lie!ideal}, or simply~\emph{ideal} in~$\glie$ is a~{\linear{$\kf$}} subspace~$I \subseteq \glie$ such that~$[x,y] \in I$ for all~$x \in \glie$ and~$y \in I$.
      That~$I$ is an ideal in~$\glie$ is denoted by~$I \mathrel{\gls*{lie ideal}} \glie$.
  \end{enumerate}
\end{definition}


\begin{remark}
  It is not necessary to distinguish between left ideals or right ideals in a Lie~algebra because the Lie~bracket is antisymmetric.
\end{remark}


\begin{remark}
  \leavevmode
  \begin{enumerate}
    \item
      For a Lie~algebra~$\glie$ and a subalgebra~$\hlie \subseteq \glie$ it follows that~$\hlie$ becomes a Lie~algebra by restricting the Lie~bracket of~$\glie$ to~$\hlie$.
    \item
      Every ideal in~$\glie$ is also a subalgebra of~$\glie$.
  \end{enumerate}
\end{remark}


\begin{definition}
  Let~$\glie$ be a Lie~algebra.
  The \defemph{center}\index{center} of~$\glie$ is
  \[
    \gls*{center}
    \defined
    \{
      x \in \glie
    \suchthat
      \text{$[x,y] = 0$ for every~$y \in \glie$}
    \}  \,.
  \]
\end{definition}


\begin{definition}
  For a Lie-algebra~$\glie$ over some field~$\kf$ and all subsets~$X, Y \subseteq \glie$ the subspace
  \[
    \gls*{commutator space}
    \defined
    \vspan_k
    \{
      [x,y]
    \suchthat
      x \in X,
      y \in Y
    \}
  \]
  is the \defemph{commutator}\index{commutator} of~$X$ and~$Y$.
\end{definition}


\begin{definition}
  A Lie~algebra~$\glie$ is \defemph{abelian}\index{abelian} if~$[x,y] = 0$ for all~$x, y \in \glie$.
\end{definition}


\begin{remark}
  Note that a Lie~algebra~$\glie$ is abelian if and only if~$\centerlie(\glie) = 0$, if and only if~$[\glie, \glie] = 0$.
\end{remark}


\begin{example}
  A~{\algebra{$\kf$}}~$A$ is commutative if and only if it is abelian as a Lie~algebra.
\end{example}


\begin{lemma}
  Let~$\glie$ be a Lie~algebra.
  \begin{enumerate}
    \item
    If~$I_\lambda$,~$\lambda \in \Lambda$ is a collection of ideals of~$\glie$ then their intersection~$\bigcap_{\lambda \in \Lambda} I_\lambda$ and their sum~$\sum_{\lambda \in \Lambda} I_\lambda$ are again ideals in~$\glie$.
    \item
    If~$I$ and~$J$ are ideals in~$\glie$ then their commutator~$[I,J]$ is again an ideal in~$\glie$.
  \end{enumerate}
\end{lemma}


\begin{proof}
  \leavevmode
  \begin{enumerate}
    \item
      This follows from direct calculation.
    \item
      As the commutator~$[I,J]$ is spanned by the elements~$[y,z]$ with~$y \in I$ and~$z \in J$ it sufficies to show that~$[x,[y,z]] \in [I,J]$ for all~$x \in \glie$,~$y \in I$ and~$z \in J$.
      This follows from~$I, J \ideal \glie$ and the Jacobi identity because
      \[
        [x,[y,z]]
        =
        [[x,y], z] + [y, [x,z]]
        \subseteq
        [I, z] + [y, J]
        \subseteq
        [I, J] + [J, I]
        =
        [I,J] \,.
      \]
      Here we use that~$[J,I] = -[I,J] = [I,J]$ by the antisymmetry of the Lie bracket~$[-,-]$.
   \qedhere
 \end{enumerate}
\end{proof}


\begin{definition}
  A Lie~algebra~$\glie$ is~\emph{linear}\index{linear} if~$\glie$ is a Lie~subalgebra of~$\gllie(V)$ for some finite dimensional vector space~$V$, or a Lie~subalgebra of~$\gllie_n(\kf)$ for some~$n$.
\end{definition}


% TODO: Fix this window.
\begin{examples}[Linear Lie~Algebras]
  \leavevmode
  \begin{enumerate}
  \item
    Let $\glie \defined \gllie_n(\kf)$.
    Then
    \[
      \gls*{special lie matrix}
      \defined
      \{
        A \in \glie
      \suchthat
        \tr A = 0
      \}
    \]
    is an ideal in~$\glie$, namely
    \[
      \sllie_n(\kf)
      =
      [\glie,\glie]  \,.
    \]
      
    Indeed, we have for all~$A, B \in \glie$ that~$\tr(AB) = \tr(BA)$ and hence
    \[
        \tr( [A,B] )
      = \tr(AB-BA)
      = \tr(AB) - \tr(BA)
      = \tr(AB) - \tr(AB)
      = 0  \,.
    \]
    This shows that~$[\glie, \glie] \subseteq \sllie_n(\kf)$.
    Observe on the other hand that~$\sllie_n(\kf)$ has a basis given by the matrices $E_{ij}$ with $1 \leq i \neq j \leq n$ together with the matrices $E_{11} - E_{ii}$ with $i = 2, \dotsc, n$.
    Each of these matrices is given as a commutator, namely
    \[
        E_{ij}
        =
        E_{ij} E_{jj} - \underbrace{ E_{jj} E_{ij} }_{=0}
        =
        [E_{ij}, E_{jj}]
    \]
    for $1 \leq i \neq j \leq n$ and
    \[
      E_{11} - E_{ii}
      =
      E_{1i} E_{i1} - E_{i1} E_{1i}
      =
      [E_{1i}, E_{i1}]
    \]
    for $i = 2, \dotsc, n$.
    Therefore $\sllie_n(\kf) \subseteq [\glie, \glie]$.
    
    The Lie~algebra~$\sllie_n(\kf)$ is the \defemph{special linear Lie~algebra}\index{special linear Lie algebra}.
    
    If~$V$ is any finite dimensional~{\vectorspace{$\kf$}} then
    \[
      \gls*{special lie endomorphism}
      \defined
      [\gllie(V), \gllie(V)]
      =
      \{
        f \in \gllie(V)
      \suchthat
        \tr(f) = 0
      \}
    \]
    is the \defemph{special linear Lie~algebra}\index{special linear Lie algebra} of~$V$.
   
  \item
    The upper triangular matrices\index{upper triangular matrices}\index{triangular matrices!upper}
    \[
      \gls*{triangular lie matrix}
      \defined
      \left\{
        \begin{pmatrix}
            a_{11}
          & \cdots
          & \cdots
          & a_{1n}
          \\
            0
          & \ddots
          & {}
          & \vdots
          \\
            \vdots
          & \ddots
          & \ddots
          & \vdots
          \\
            0
          & \cdots
          & 0
          & a_{nn}
        \end{pmatrix}
      \suchthat*
        \text{$a_{ij} \in \kf$ for all~$1 \leq i \leq j \leq n$}
      \right\}
    \]
    form a Lie~subalgebra of~$\gllie_n(\kf)$.
    This holds because the space of upper triangular matrices is a~{\subalgebra{$\kf$}} of~$\Mat_n(\kf)$, and are hence closed under the commutator~$[A,B] = AB - BA$.
   
  \item
    The strictly upper triangular matrices\index{stricly upper triangular matrices}\index{triangular matrices!strictly upper}
    \[
      \gls*{strictly triangular lie matrix}
      \defined
      \left\{
        \begin{pmatrix}
            0
          & a_{12}
          & \cdots
          & a_{1n}
          \\
            \vdots
          & \ddots
          & \ddots
          & \vdots
          \\
            \vdots
          & {}
          & \ddots
          & a_{n-1,n}
          \\
            0
          & \cdots
          & \cdots
          & 0
        \end{pmatrix}
      \suchthat*
        \text{$a_{ij} \in \kf$ for all~$1 \leq i < j \leq n$}
      \right\}
    \]
    form a Lie~subalgebra of~$\tlie_n(\kf)$ and therefore also of~$\gllie_n(\kf)$.
    The space~$\nlie_n(\kf)$ is even an ideal in~$\tlie_n(\kf)$, namely~$\nlie_n(\kf) = [\tlie_n(\kf), \tlie_n(\kf)]$:
   
    For any two matrices~$A, B \in \tlie_n(\kf)$ their products~$AB$ and~$BA$ are again upper triangular and both products have the same diagonal entries.
    The commutator~$[A,B] = AB - BA$ is therefore a strictly upper triangular matrix.
    Hence~$[\tlie_n(\kf), \tlie_n(\kf)] \subseteq \nlie_n(\kf)$.
    We know on the other hand that~$\nlie_n(\kf)$ has as a basis the matrices~$E_{ij}$ with~$1 \leq i < j \leq n$, and each of those matrices can be written as a commutator
    \[
      E_{ij}
      =
      E_{ii} E_{ij} - \underbrace{ E_{ij} E_{ii} }_{= 0}
    \]
    with~$E_{ii}, E_{ij} \in \tlie_n(\kf)$.
    Therefore~$\nlie_n(\kf) \subseteq \tlie_n(\kf)$.
  \end{enumerate}
\end{examples}


\begin{definition}
  If~$\glie$ is a Lie~algebra and~$U \subseteq \glie$ any linear subspace then
  \[
    \gls*{normalizer}
    \defined
    \{
      x \in \glie
    \suchthat
      \text{$[x,y] \in \glie$ for every~$y \in U$}
    \}
  \]
  is the \defemph{normalizer}\index{normalizer} of~$U$ in~$\glie$, and
  \[
    \gls*{centralizer of space}
    \defined
    \{
      x \in \glie
    \suchthat
      \text{$[x,y] = 0$ for every $y \in U$}
    \}
  \]
  is the \defemph{centralizer}\index{centralizer} of~$U$ in~$\glie$.
  For a single element~$x \in \glie$ the \defemph{centralizer}\index{centralizer} of~$x$ in~$\glie$ is
  \[
    \gls*{centralizer of element}
    \defined
    \{
      y \in \glie
    \suchthat
      [x,y] = 0
    \} \,.
  \]
\end{definition}


\begin{lemma}
 Let~$\glie$ be a Lie~algebra and~$U \subseteq \glie$ a linear subspace.
 Then both the normalizer~$\normallie_\glie(U)$ and the centralizer~$\centerlie_\glie(U)$ are Lie~subalgebras of~$\glie$.
 If~$x \in \glie$ is any element then the centralizer~$\centerlie_\glie(x)$ is a Lie~subalgebra of~$\glie$.
\end{lemma}


\begin{proof}
 If~$x, y \in \normallie_\glie(U)$ then it follows for every $z \in U$ from the Jacobi~identity that
 \[
  [[x,y], z]
  =
  - [z, [x,y]]
  =
  - [[z,x], y] - [x, [z,y]
  \in
  - [U, y] - [x, U]
  \subseteq
  - U - U
  =
  U
 \]
 and therefore~$[x,y] \in \normallie_\glie(U)$.
 In the same way it follows for all~$x, y \in \glie$ and~$z \in U$ that
 \[
  [[x,y], z]
  =
  - [[z,x], y] - [x, [z,y]]
  =
  - [0, y] - [x, 0]
  =
  0
 \]
 and therefore~$[x,y] \in Z_\glie(U)$.
 For any~$x \in \glie$ the span~$\kf x$ is a linear subspace of~$\glie$ with $\centerlie_\glie(x) = \centerlie_\glie(\kf x)$.
 Hence~$\centerlie_\glie(x)$ is a Lie~subalgebra of $\glie$, as shown above.
\end{proof}


\begin{remark}
 Let~$\glie$ be a Lie~algebra and~$L \subseteq \glie$ a linear subspace.
 Then~$L$ is a Lie~subalgebra of~$\glie$ if and only if~$L \subseteq \normallie_\glie(L)$.
 Then~$L$ is not only contained in~$\normallie_\glie(L)$ but~$\normallie_\glie(L)$ is the maximal Lie~subalgebra of~$\glie$ that contains~$L$ as an ideal.
 As a special case of this, we find that~$L$ is an ideal in~$\glie$ if and only if~$\normallie_\glie(L) = \glie$.
\end{remark}
% 
% 
% 
% 
% 
% 
% 
% 
% 
% \subsection{Homomorphisms of Lie~algebras}
% 
% 
% \begin{definition}
%  Given Lie~algebras $\g_1$ and $\g_2$ over the same field $\kf$ a \emph{homomorphism of Lie~algebras} $\g_1 \to \g_2$ is a $\kf$-linear map $f \colon \g_1 \to \g_2$ such that
%  \[
%   f([x,y]) =[f(x),f(y)] \quad \text{for all $x,y \in \g_1$}.
%  \]
% \end{definition}
% 
% 
% \begin{examples}\label{expls: homomorphisms of lie algebras}
%  \begin{enumerate}[leftmargin=*]
%   \item
%    For any Lie~algebra $\glie$ the identity $\id_\g \colon \g \to \glie$ is a Lie~algebra homomorphism.
%   \item
%    Given Lie~algebras $\g_1$, $\g_2$ and $\g_3$ and Lie~algebra homomorphisms $f_1 \colon \g_1 \to \g_2$ and $f_2 \colon \g_2 \to \g_3$ the composition $f_2 \circ f_1 \colon \g_1 \to \g_3$ is also a homomorphism of Lie~algebras.
%   \item
%    If $\glie$ is a Lie~algebra and $\hlie \subseteq \glie$ a Lie subalgebra then the inclusion $\hlie \inc \glie$ is a homomorphism of Lie~algebras.
%   \item
%    Given two abelian Lie~algebras $\g_1$ and $\g_2$ any linear map $\g_1 \to \g_2$ is already a homomorphism of Lie~algebras.
%   \item
%    Let $\glie$ ba a Lie~algebra over an arbitrary fielid $\kf$. Then for every $x \in \glie$ let
%    \[
%     \ad(x) \colon \g \to \g \quad \text{mit} \quad \ad(x)(y) = [x,y]
%     \quad \text{for every $y \in \glie$}.
%    \]
%    Then the map $\ad \colon \g \to \gl(\g)$ is a homomorphism of Lie~algebras. This follows from the Jacobi identity because for all $x,y,z \in \glie$
%    \begin{align*}
%     \ad([x,y])(z)
%     &= [[x,y],z]
%     = -[z,[x,y]]
%     = -[[z,x],y] -[x,[z,y]]
%     = [x,[y,z]] - [y,[x,z]] \\
%     &= \ad(x)\ad(y)(z) - \ad(y)\ad(x)(z)
%     = [\ad(x),\ad(y)](z).
%    \end{align*}
%   \item
%    If $A_1$ and $A_2$ are associative $\kf$-algebras and $f \colon A_1 \to A_2$ a homomorphism of $\kf$-algebras then it is also a homomorphism of Lie~algebras because
%    \[
%     f([a,b]) = f(ab-ba) = f(a)f(b)-f(b)f(a) = [f(a),f(b)]
%     \quad \text{for all $a,b \in A_1$}.
%    \]
%   \item
%    Let $\glie$ be a Lie~algebra over an arbitary field $\kf$. If $\phi \colon \sll_2(k) \to \glie$ is a homomorphism of Lie~algebras then the images
%    \[
%     E \coloneqq \phi(e), \quad H \coloneqq \phi(h), \quad F \coloneqq \phi(f)
%    \]
%    satisfy the relations
%    \[
%     [H,E] = 2E, \quad [H,F] = 2F, \quad [E,F] = H.
%    \]
%    On the other hand every triple $(E', H', F')$ of elements satisfying the relations above (with $X$ replaced by $X'$ for $X \in \{E,H,F\}$) gives rise to a unique homomorphism of Lie~algebras $\phi' \colon \sll_2(k) \to \glie$ with
%    \[
%     \phi'(E) = E', \quad \phi'(H) = H', \quad \phi'(F) = F'.
%    \]
%    
%    Hence there is a bijection between Lie~algebra homomorphisms $\sll_2(k) \to \glie$ and triples as above. Such triples will play an important part later on.
%  \end{enumerate}
% \end{examples}
% 
% 
% \begin{definition}
%  Let $\g_1, \g_2$ be Lie~algebras over the same field $\kf$. A homomorphism of Lie~algebras $f \colon \g_1 \to \g_2$ is called an \emph{isomorphism of $\kf$-Lie~algebras} if $f$ is bijective.
% \end{definition}
% 
% 
% \begin{lemma}
%  If $f \colon \g_1 \to \g_2$ is an isomorphism of $\kf$-Lie~algebras, then the $\kf$-linear map $f^{-1} \colon \g_2 \to \g_1$ is also a homomorphism of Lie-algebras and therefore also an isomorphism.
% \end{lemma}
% \begin{proof}
%  For all $x,y \in \g_2$
%  \begin{align*}
%   f^{-1}( [x,y] )
%   &= f^{-1}( [ f(f^{-1}(x)), f(f^{-1}(y)) ] ) \\
%   &= f^{-1}(f( [f^{-1}(x) , f^{-1}(y)] )) \\
%   &= [f^{-1}(x), f^{-1}(y)].
%  \qedhere
%  \end{align*}
% \end{proof}
% 
% 
% \begin{remark}
%  It follows that Lie~algebras over the same field $\kf$ together with homomorphisms of Lie~algebras and their usual composition form a category, which will be denoted by $\cLie{k}$. An \emph{isomorphism of $\kf$-Lie~algebras} is the same as an isomorphism in $\cLie{k}$.
% \end{remark}
% 
% 
% \begin{example}[Classification of one- and two-dimensional Lie~algebras]
%  Let $\kf$ be any field.
%  
%  As every linear map between abelian Lie~algebras is already a homomorphism of Lie~algebras it follows that there exists up to isomorphism exactly one $n$-dimensional abelian Lie~algebra over $\kf$ for every $n \in \N$.
%  
%  If $\glie$ is a one-dimensional Lie~algebra over $\kf$ then the Lie~bracket of $\glie$ is zero because it is alternating, which is why $\glie$ is abelian. Hence there is up to isomorphism exactly one one-dimensional Lie~algebra over $\kf$.
%  
%  Up to isomorphism there exists exactly one two-dimensional abelian $\kf$-Lie~algebra
%  
%  Suppose that $\glie$ is a two-dimensional, non-abelian Lie~algebra over $\kf$. Let $\tilde{x}$,$\tilde{y}$ be a basis of $\glie$. Because $[\g,\g]$ is non-abelian it follows that $[\g,\g] \neq 0$ and on the other hand $[\g,\g]$ is spanned by $[\tilde{x},\tilde{y}]$ because the Lie~bracket is alternating, so $[\g,\g] = k [\tilde{x},\tilde{y}]$ with $[\tilde{x},\tilde{y}] \neq 0$. Let $x \coloneqq [\tilde{x},\tilde{y}]$ and $y \in \glie$ such that $x,y$ is a basis of $\glie$. Then $[x,y] \neq 0$ and $[x,y] \in k x$. By rescaling $y$ it can be assumed that $[x,y] = x$. This that up to isomorphism there is at most one two-dimensional, non-abelian Lie~algebra $\glie$ over $\kf$.
%  
%  Such an Lie~algebra also exists. It can be realized as a subalgebra of $\gl_2(k)$ by choosing
%  \begin{gather*}
%   x \coloneqq \begin{pmatrix} 0 & 1 \\ 0 & 0 \end{pmatrix} = e_{12}
%   \quad\text{and}\quad
%   y \coloneqq \begin{pmatrix} 0 & 0 \\ 0 & 1 \end{pmatrix} = e_{22}
%  \shortintertext{because}
%   [x,y] = [e_{12}, e_{22}] = e_{12} e_{22} - e_{22} e_{12} = e_{12} = x.
%  \end{gather*}
%  Hence there are up to isomorphism exactly two two-dimensional Lie~algebras over $\kf$.
% \end{example}
% 
% 
% \begin{proposition}[Homomorphism theorem]
%  Let $\g_1$ and $\g_2$ be Lie~algebras and $f \colon \g_1 \to \g_2$ a homomorphism of Lie~algebras.
%  \begin{enumerate}[leftmargin=*]
%   \item
%    $\ker f \subideal \g_1$ is an ideal.
%   \item
%    $\im f \subseteq \g_2$ is a Lie subalgebra.
%   \item
%    If $I \subideal \g_1$ is an ideal with $\ker f \subseteq I$ then there exists a unique homomorphism of Lie~algebras $\tilde{f} \colon \g_1/I \to \g_2$ with $f = \tilde{f} \circ \pi$ where $\pi \colon \g_1 \to \g_1/I$ is the canonical projection.
%    \[
%      \begin{tikzcd}
%        \g_1
%        \arrow{dr}[above right]{f}
%        \arrow{d}[left]{d}
%        &
%        {}
%        \\
%        \g_1/I
%        \arrow[dashed]{r}[below]{\exists! f}
%        &
%        \g_2
%      \end{tikzcd}
%    \]
% 
%   \item
%    If $I, J \subideal \glie$ are subideals with $I \subseteq J$ then $J/I \subideal \g/I$ and the natural isomorphism of vector spaces
%    \[
%     (\g/I)/(J/I) \to \g/I, \quad (x+I)+(J/I) \mapsto x+I
%    \]
%    is already a natural isomorphism of Lie~algebras.
%   \item
%    If $I, J \subideal \glie$ are subideals then the natural isomorphism of vector spaces
%    \begin{gather*}
%     (I + J)/J \to I/(I \cap J)
%    \shortintertext{defined by}
%     (x+J)+I \mapsto x + (I \cap J) \quad \text{for every $x \in I$}
%    \end{gather*}
%    is already a natural isomorphism of Lie~algebras.
%  \end{enumerate}
% \end{proposition}
% 
% 
% \begin{remark}
%  For a Lie~algebra $\glie$ the ideal $[\g,\g]$ is the minimal ideal inside $\glie$ such that $\g/[\g,\g]$ is abelian. Furthermore given any abelian Lie~algebra $\hlie$ any homomorphism of Lie~algebras $\g \to \hlie$ factorizes through a unique homomorphism of Lie~algebras $\g/[\g,\g] \to \hlie$.
%  \[
%    \begin{tikzcd}
%      \g
%      \arrow{rr}
%      \arrow{dr}
%      &
%      {}
%      &
%      \h
%      \\
%      {}
%      &
%      \g / [\g,\g]
%      \arrow[dashed]{ur}[below right]{\exists!}
%      &
%      {}
%    \end{tikzcd}
%  \]
% \end{remark}
% 
% 
% 
% 
% \subsection{New Lie~algebras from old ones}
% 
% 
% \begin{definition}
%  Let $\g_1$ and $\g_2$ be Lie~algebras over the same field $\kf$. Then the \emph{product} of $\g_1$ and $\g_2$ is defined as the $\kf$-vector space $\g_1 \times \g_2$ together with the Lie~bracket
%  \[
%   [(x_1, y_1), (x_2, y_2)]
%   = ([x_1, x_2], [y_1, y_2])
%   \quad
%   \text{for all $(x_1, y_1), (x_2, y_2) \in \g_1 \times \g_2$}.
%  \]
% \end{definition}
% 
% 
% \begin{definition}
%  Let $\glie$ be a Lie~algebra and $I \subideal \glie$. Then the induced Lie~algebra structure on the quotient vector space $\g/I$ is given by
%  \[
%   [x+I, y+I] = [x,y] + I \quad \text{for all $x,y \in \glie$}.
%  \]
% \end{definition}
% 
% 
% \begin{remark}
%  The Lie~algebra structure on the quotient $\g/I$ is well-defined: If $x,y, x',y' \in \glie$ with $x+I = x'+I$ and $y+I = y'+I$ then $x-x' \in I$ and $y-y' \in I$ and thus
%  \begin{align*}
%   [x,y] + I
%   &= [x' + x-x', y' + y-y'] + I \\
%   &= [x',y'] + \underbrace{[x', y-y']}_{\in I} + \underbrace{[x-x', y']}_{\in I} + \underbrace{[x-x', y-y']}_{\in I} + I
%   = [x', y'] + I.
%  \end{align*}
%  The additional properties of a Lie~bracket follows from the Lie~bracket of $\glie$ safisfying them.
% \end{remark}
% 
% 
% \begin{lemma}
%  \begin{enumerate}[leftmargin=*]
%   \item
%    If $\g_1$ and $\g_2$ are Lie~algebras then the canonical projections
%    \[
%     \pi_i \colon \g_1 \times \g_2 \to \g_i, \quad (x_1, x_2) \mapsto x_i
%     \quad \text{for $i = 1, 2$}
%    \]
%    are homomorphisms of Lie-algebras.
%   \item
%    If $\glie$ is a Lie~algebra and $I \subideal \glie$ then the canonical projection $\pi \colon \g \to \g/I, x \mapsto [x]$ is a homomorphism of Lie~algebras.
%  \end{enumerate}
% \end{lemma}
% 
% 
% 
% \begin{lemma}\label{lem: quasi extension of scalars for lie algebras}
% Let $\glie$ be a Lie~algebra over $\kf$ and $A$ an associative, commutative $\kf$-algebra. Then $A \otimes_k \glie$ is a Lie~algebra over $\kf$ via
% \[
%  [a \otimes x, b \otimes y] = (ab) \otimes [x,y]
%  \quad
%  \text{for all $a,b \in A$ and $x,y \in \glie$}.
% \]
% Similarly $\g \otimes_k A$ carries the structure of a Lie~algebra over $\kf$ via
% \[
%  [x \otimes a, y \otimes b] = [x,y] \otimes (ab)
%  \quad
%  \text{for all $x,y \in \glie$ and $a,b \in A$}.
% \]
% \end{lemma}
% 
% 
% \begin{example}
%  If $L/k$ is a field extension and $\glie$ a Lie~algebra over $\kf$, then $L \otimes_k \glie$ is a Lie~algebra over $\kf$ via
%  \[
%   [\lambda \otimes x, \mu \otimes y] = (\lambda \mu) \otimes [x,y]
%   \quad
%   \text{for alle $\lambda, \mu \in k$ and $x,y \in \glie$}.
%  \]
%  $L \otimes_k \glie$ also carries the structure of an $L$-vector space via extension of scalars, i.e.
%  \[
%   \lambda \cdot (\mu \otimes x) = (\lambda \mu) \otimes x
%   \quad
%   \text{for alle $\lambda, \mu \in k$ and $x \in \glie$},
%  \]
%  and the Lie~bracket is not only $\kf$-bilinear, but also $L$-bilinear. Hence the structure of a $\kf$-Lie~algebra on $L \otimes_k \glie$ can be extended to the structure of a Lie~algebra over $L$. (Notice that the Jacobi-Identity is independent of the ground field.)
% \end{example}
% 
% 
% \begin{definition}
%  Let $\glie$ be a Lie~algebra and $A = k[t,t^{-1}]$ be the algebra of Laurent polynomials over $\kf$. Then
%  \[
%   \Loop(\g) \coloneqq \g \otimes_k A
%  \]
%  with the Lie~bracket as in Lemma~\ref{lem: quasi extension of scalars for lie algebras} is called the \emph{loop (Lie) algebra} of $\glie$.
% \end{definition}
% 
% 
% Another example for constructing new Lie~algebras out of old ones are \emph{central extensions}: Let $\glie$ be any $\kf$-Lie~algebra.
% \[
%  \tilde{\g}
%  \coloneqq \g \oplus k
%  = \{x + \lambda c \mid x \in \g, \lambda \in k\},
% \]
% where we understand $c$ as a formal variable. Suppose that $\kappa \colon \g \times \g \to k$ is a $\kf$-bilinear map satisfying the following properties:
% \begin{enumerate}
%  \item
%   $\kappa$ is antisymmetric, i.e.\ $\kappa(x,y) = -\kappa(y,x)$ for all $x,y \in \glie$.
%  \item
%   $\kappa$ satisfies the $2$-cocycle condition
%   \[
%    \kappa([x,y],z) + \kappa([y,z],x) + \kappa([z,x],y) = 0
%    \quad
%    \text{for all $x,y,z \in \glie$}.
%   \]
% \end{enumerate}
% Then $\tilde{\g}$ becomes a Lie~algebra via
% \[
%  [x + \lambda c, y + \mu c] \coloneqq [x,y] + \kappa(x,y) \lambda \mu c
%  \quad \text{for all $x,y \in \glie$ and $\lambda, \mu \in k$.}
% \]
% Note that $c$ is central in $\tilde{\g}$ in the sense that $[x,c] = 0$ for all $x \in \glie$.
% 
% 
% \begin{example}
%  Let $\g = \gllie_n(\kf)$. We define a symmetric bilinear form on $\glie$ via
%  \[
%   (A,B)_{\tr} = \tr(AB) \quad \text{for all $A,B \in \glie$}.
%  \]
%  We define a bilinear form
%  \[
%   \Loop(\g) \times \Loop(\g) \to k[t,t^{-1}], \quad
%   (x \otimes p, y \otimes q) \mapsto (x,y)_{\tr}\ pq
%  \]
%  We now get a $2$-cocycle $\kappa \colon \Loop(\g) \times \Loop(\g) \to k$ via
%  \[
%   \kappa(a,b) \coloneqq \mathrm{Res}\left(\frac{\partial a}{\partial t}, b\right).
%  \]
%  $\kappa$ is also antisymmetric: Let $a = x \otimes t^i$ and $b = y \otimes t^j$ with $x,y \in \glie$ and $i,j \in \Z$. Then
%  \begin{align*}
%   \kappa(x \otimes t^i, y \otimes t^j)
%   = \mathrm{Res}(i x \otimes t^{i-1}, y \otimes t^j)
%   &= \mathrm{Res}(i t^{i+j-1} (x,y)_{\tr}) \\
%   &=
%   \begin{cases}
%    i (x,y)_{\tr} & \text{if $i+j = 0$}, \\
%                0 & \text{otherwise}.
%   \end{cases}
%  \end{align*}
%  In the same way we find that
%  \[
%   \kappa(y \otimes t^j, x \otimes t^i) =
%   \begin{cases}
%    j (x,y)_{\tr} & \text{if $i+j = 0$}, \\
%                0 & \text{otherwise}.
%   \end{cases}
%  \]
%  Since $(\cdot,\cdot)_{\tr}$ is symmetric we find that
%  \begin{align*}
%   \kappa(x \otimes t^i, y \otimes t^j)
%   &=
%   \begin{cases}
%    i (x,y)_{\tr} & \text{if $i+j = 0$}, \\
%                0 & \text{otherwise},
%   \end{cases} \\
%   &=
%   \begin{cases}
%    -j (x,y)_{\tr} & \text{if $i+j = 0$}, \\
%                 0 & \text{otherwise},
%   \end{cases} \\
%   &=
%   -\kappa(y \otimes t^j, x \otimes t^i).
%  \end{align*}
% \end{example}
% 
% 
% 
% 
% 
% \subsection{Derivations}
% 
% 
% \begin{definition}
%  Let $A$ be a $\kf$-algebra (not necessarily unitary of even associative). A \emph{derivation of $A$} is a $\kf$-linear map $d \colon A \to A$ such that
%  \[
%   d(ab) = d(a)b + ad(b) \quad \text{for all $a,b \in A$}.
%  \]
%  We set
%  \[
%   \Der(A) \coloneqq \{d \colon A \to A \mid \text{$d$ is a derivation of $A$} \}.
%  \]
% \end{definition}
% 
% 
% \begin{remark}
%  $\Der(A)$ is a $\kf$-linear subspace of $\End_k(A)$.
% \end{remark}
% 
% 
% \begin{example}
%   Let $A$ be a $\kf$-algebra. It follows from direct calculation that for all $d, d' \in \Der(A)$ the commutator $[d,d'] = d \circ d' - d' \circ d$ is again a derivation $\Der(A)$. Hence $\Der(A)$ is a Lie subalgebra of $\gl(A)$.
% \end{example}
% 
% 
% \begin{lemma}\label{lem: Lie algebras act adjoint by derivations}
%  Let $\glie$ be a Lie~algebra. Then for any $x \in \glie$ the map
%  \[
%   \ad(x) \colon \g \to \g, \quad y \mapsto [x,y]
%  \]
%  is a derivation of $\glie$.
% \end{lemma}
% \begin{proof}
%  By the Jacobi identity
%  \begin{align*}
%   \ad(x)([y,z])
%   &= [x,[y,z]]
%   = [[x,y],z] + [y,[x,z]] \\
%   &= [\ad(x)(y),z] + [y,\ad(x)(z)]
%  \end{align*}
%  for all $y,z \in \glie$.
% \end{proof}
% 
% 
% \begin{definition}
%  Let $\glie$ be a Lie~algebra. A derivation of $\glie$ is called \emph{inner} if it is of the form $\ad(x)$ for some $x \in \glie$.
% \end{definition}
% 
% 
% \begin{lemma}\label{lem: inner derivations are in ideal}
%  If $\glie$ is a Lie~algebra then the inner derivations form an ideal inside of $\Der(\g)$.
% \end{lemma}
% \begin{proof}
%  Let $I \coloneqq \im \ad \subseteq \Der(\g)$ be the linear subspace of inner derivations. For any $\delta \in \Der$ and $x \in \glie$ it follows that for any $y \in \glie$
%  \begin{align*}
%   &\,[\delta, \ad(x)](y)
%   = (\delta \ad(x) - \ad(x) \delta(x))(y) \\
%   &= \delta([x,y]) - [x,\delta(y)]
%   = [\delta(x),y] + [x,\delta(y)] - [x,\delta(y)] \\
%   &= [\delta(x),y]
%   = \ad(\delta(x))(y).
%  \end{align*}
%  Hence $[\delta, \ad(x)] = \ad(\delta(x)) \in I$.
% \end{proof}
% 
% 
% 
% 
% 
% \subsection{Simple Lie~algebras}
% 
% 
% \begin{definition}
%  A Lie~algebra $\glie$ is \emph{simple} if $0$ and $\glie$ are the only ideals inside $\glie$ and $\glie$ is not abelian.
% \end{definition}
% 
% 
% \begin{lemma}
%  Let $\glie$ be a simple Lie~algebra. Then $[\g,\g] = \glie$ and $Z(\g)=0$.
% \end{lemma}
% \begin{proof}
%  Because $\glie$ is simple it is not abelian. Therefore $[\g,\g] \neq 0$ and $Z(\g) \neq \glie$. Since $[\g,\g]$ and $Z(\g)$ are ideals inside $\glie$ it follows that $[\g,\g] = \glie$ and $Z(\g) = 0$. By the homomorphism theorem $\glie$ is isomorphic to its image $\ad \glie$ and hence to a linear Lie~algebra.
% \end{proof}
% 
% 
% \begin{corollary}
%  Let $\glie$ be simple. Then the homomorphism $\ad \colon \g \to \gl(\g), x \mapsto \ad(x)$ is injective. In particular $\glie$ can be realized as a linear Lie~algebra.
% \end{corollary}
% \begin{proof}
%  As part of Examples~\ref{expls: homomorphisms of lie algebras} has already been shown that $\ad$ is a homomorphism of Lie~algebras. That it is injective follows directly from $\ker \ad = Z(\g) = 0$.
% \end{proof}
% 
% 
% It can be shown that every finite dimensional Lie~algebra can be realized as a linear Lie~algebra. This will not be proven in this lecture and is by far not trivial.
% 
% 
% \begin{theorem}[Ado]
%  Every finite dimensional Lie~algebra $\glie$ is isomorphic to a linear Lie~algebra.
% \end{theorem}
% 
% 
% \begin{examples}
%  \begin{enumerate}[leftmargin=*]
%   \item
%    Since $[\gllie_n(\kf),\gllie_n(\kf)] = \sllie_n(\kf) \neq \gllie_n(\kf)$ we find that $\gllie_n(\kf)$ is not simple.
%   \item
%    Let $\g = \sll_2(k)$. Then $\glie$ is simple if and only if $\chara k \neq 2$. To see this consider the basis $(e,h,f)$ of $\sll_2(k)$ consisting of the matrices
%    \[
%     e = \begin{pmatrix}0 & 1 \\ 0 & 0\end{pmatrix}, \quad
%     h = \vect{1 & 0 \\ 0 & -1}, \quad
%     f = \vect{0 & 0 \\ 1 & 0}.
%    \]
%    of $\sll_2(k)$. Then
%    \[
%     [h,e] = 2e, \quad
%     [h,f] = -2f, \quad
%     [e,f] = h.
%    \]
%    If $\chara k = 2$ then $h$ spans a $1$-dimensional ideal, thus $\sll_2(k)$ is not simple. Suppose that $\chara k \neq 2$ and let $I \subseteq \sll_2(k)$ be an ideal with $I \neq 0$. From the above relations it follows that if $I$ contains one of the basis vectors $e$, $h$ or $f$ then already $I = \sll_2(k)$. Let $x \in I$ with $x \neq 0$ and write $x = \alpha e + \beta h + \gamma f$. Then
%    \[
%     [e,x] = -2 \beta e + \gamma h \quad \text{and} \quad [e,[e,x]] = -2 \gamma e.
%    \]
%    Since $\gamma = 0$ or $\gamma \neq 0$ we find that $e \in I$.
%  \end{enumerate}
% \end{examples}
% 
% 
% \begin{definition}
%  Let $\kf$ be any field. The basis
%  \[
%   e = \begin{pmatrix}0 & 1 \\ 0 & 0\end{pmatrix}, \quad
%   h = \vect{1 & 0 \\ 0 & -1}, \quad
%   f = \vect{0 & 0 \\ 1 & 0}.
%  \]
%  of $\sll_2(k)$ is called the \emph{standard basis} of $\sll_2(k)$.
% \end{definition}
% 
% 
% \begin{remark}
%  If $\chara k = 0$ then $\sllie_n(\kf)$ is simple for all $n \geq 2$.
% \end{remark}
