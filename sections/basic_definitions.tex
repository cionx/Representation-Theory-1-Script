\section{Lie Algebras}





\subsection{Definition and Examples}


\begin{definition}
  Let~$\glie$ be a vector space over some field~$\kf$.
  A~{\bilinear{$\kf$}} map
  \[
    \gls*{lie bracket}
    \colon
    \glie \times \glie
    \to
    \glie
  \]
  is a \defemph{Lie~bracket}\index{Lie!bracket} if it satisfies the following two conditions:
  \begin{enumerate}
    \item
    $[-, -]$ is \defemph{alternating}\index{alternating}, i.e.~$[x,x] = 0$ for every~$x \in \glie$.
    \item
    $[-, -]$ satisfies the \defemph{Jacobi identity}\index{Jacobi identity}
    \[
      [x,[y,z]] + [y,[z,x]] + [z,[x,y]]
      =
      0
    \]
    for all~$x,y,z \in \glie$.
  \end{enumerate}
  A~{\vectorspace{$\kf$}}~$\glie$ together with a Lie~bracket~$[-,-]$ is a~\defemph{\liealgebra{$\kf$}}\index{Lie!algebra}.
\end{definition}


\begin{remark}
  A Lie~bracket~$[-, -]$ on a vector space~$\glie$ is always antisymmetric, i.e.~$[y,x] = -[x,y]$ for all~$x,y \in \glie$, because
  \[
    0
    =
    [x+y, x+y]
    =
    [x,x] + [x,y] + [y,x] + [y,y]
    =
    [x,y] + [y,x] \,.
  \]
\end{remark}


\begin{remark}
  By using the antisymmetry of the Lie~bracket the Jacobi identity\index{Jacobi identity} can be rewritten as
  \[
    [x,[y,z]]
    =
    [[x,y],z] + [y,[x,z]]
  \]
  for all~$x, y, z \in \glie$.
\end{remark}


\begin{examples}
  \leavevmode
  \begin{enumerate}
    \item
      Any vector space~$\glie$ becomes a Lie~algebra via~$[x,y] = 0$ for all~$x,y \in \glie$.
    \item
      Any \emph{associative}~{\algebra{$\kf$}}~$A$ becomes a~{\liealgebra{$\kf$}} via~$[a,b] \defined ab - ba$ for all~$a, b \in A$.
      Then~$[-, -]$ is alternating and it follows from the associativity of the multiplication of~$A$ that
      \begin{align*}
         {}&  [a,[b,c]] + [b,[c,a]] + [c,[a,b]] \\
        ={}&  [a, (bc-cb)] + [b, (ca-ac)] + [c, (ab-ba)] \\
        ={}&  a(bc-cb)-(bc-cb)a + b(ca-ac) - (ca-ac)b + c(ab-ba) - (ab-ba)c \\
        ={}&  abc - acb - bca + cba + bca - bac - cab + acb + cab - cba - abc + bac \\
        ={}&  0
    \end{align*}
    for all~$a, b, c \in A$.
    The element~$[a,b]$ is the \defemph{commutator}\index{commutator} of~$a$ and~$b$.
   
    The following two are important examples of this construction:
    \begin{enumerate}
      \item
        The~{\algebra{$\kf$}} of~($n \times n$)-matrices~$\Mat_n(\kf)$ becomes a Lie~algebra via
        \[
          [A,B]
          \defined
          AB - BA
        \]
        for all~$A, B \in \Mat_n(\kf)$.
        This Lie~algebra is called the \defemph{general linear Lie~algebra}\index{general linear Lie algebra} and is denoted by~\gls*{general lie matrix}.
      \item
        More generally for any~{\vectorspace{$\kf$}} the~{\algebra{$\kf$}}~$\End_k(V)$ becomes a Lie~algebra via
        \[
          [\phi_1, \phi_2]
          \defined
          \phi_1 \circ \phi_2 - \phi_2 \circ \phi_1
        \]
        for all~$\phi_1, \phi_2 \in \End_k$.
        This Lie~algebra is called the \defemph{general linear Lie~algebra of~$V$}\index{general linear Lie algebra} and is denoted by~\gls*{general lie endomorphism}.
    \end{enumerate}
  \end{enumerate}
\end{examples}


\begin{recall}
  For calculations in~$\gllie_n(\kf)$ it is often useful to remember that
  \[
    E_{ij} E_{kl}
    =
    \begin{cases}
      E_{il}  & \text{if~$j = k$} \,, \\
      0       & \text{otherwise}  \,,
    \end{cases}
  \]
  where~$E_{ij}$ with~$1 \leq i,j \leq n$ denotes the standard basis of~$\gllie_n(\kf)$.
  One may think about the matrix~$E_{ij}$ as \enquote{going from~$j$ to~$i$}.
  The composition~$E_{ij} E_{kl}$ does then \enquote{go from~$l$ to~$i$} if the positions~$j$ and~$k$ fit together;
  if they do not fit, then the composition is simply zero.
  
  This intuition can be formalized by observing that
  \[
    E_{ij} e_k
    =
    \begin{cases}
      e_i & \text{if~$j = k$} \,, \\
      0   & \text{otherwise}  \,.
    \end{cases}
  \]
  This means that the matrix~$E_{ij}$ maps one of the standard basis vectors~$e_1, \dotsc, e_n$ (namely~$e_j$) to another standard basis vector (namely~$e_i$), but filters out all other standard basis vectors.
\end{recall}



\begin{definition}
  Let~$\glie$ be a~{\liealgebra{$\kf$}}.
  \begin{enumerate}
    \item
      A \defemph{Lie~subalgebra}\index{Lie!subalgebra} of~$\glie$ is a~{\linear{$\kf$}} subspace~$\hlie \subseteq \glie$ such that~$[x,y] \in \hlie$ for all~$x, y \in \hlie$.
    \item
      A \defemph{Lie~ideal}\index{Lie!ideal}, or simply~\defemph{ideal} in~$\glie$ is a~{\linear{$\kf$}} subspace~$I \subseteq \glie$ such that~$[x,y] \in I$ for all~$x \in \glie$ and~$y \in I$.
      That~$I$ is an ideal in~$\glie$ is denoted by~$I \mathrel{\gls*{lie ideal}} \glie$.
  \end{enumerate}
\end{definition}


\begin{remark}
  It is not necessary to distinguish between left ideals or right ideals in a Lie~algebra because the Lie~bracket is antisymmetric.
\end{remark}


\begin{remark}
  \leavevmode
  \begin{enumerate}
    \item
      For a Lie~algebra~$\glie$ and a subalgebra~$\hlie \subseteq \glie$ it follows that~$\hlie$ becomes a Lie~algebra by restricting the Lie~bracket of~$\glie$ to~$\hlie$.
    \item
      Every ideal in~$\glie$ is also a subalgebra of~$\glie$.
  \end{enumerate}
\end{remark}


\begin{definition}
  Let~$\glie$ be a Lie~algebra.
  The \defemph{center}\index{center} of~$\glie$ is
  \[
    \gls*{center}
    \defined
    \{
      x \in \glie
    \suchthat
      \text{$[x,y] = 0$ for every~$y \in \glie$}
    \}  \,.
  \]
\end{definition}


\begin{definition}
  For a Lie-algebra~$\glie$ over some field~$\kf$ and all subsets~$X, Y \subseteq \glie$ the subspace
  \[
    \gls*{commutator space}
    \defined
    \vspan_k
    \{
      [x,y]
    \suchthat
      x \in X,
      y \in Y
    \}
  \]
  is the \defemph{commutator}\index{commutator} of~$X$ and~$Y$.
\end{definition}


\begin{definition}
  A Lie~algebra~$\glie$ is \defemph{abelian}\index{abelian} if~$[x,y] = 0$ for all~$x, y \in \glie$.
\end{definition}


\begin{remark}
  Note that a Lie~algebra~$\glie$ is abelian if and only if~$\centerlie(\glie) = 0$, if and only if~$[\glie, \glie] = 0$.
\end{remark}


\begin{example}
  A~{\algebra{$\kf$}}~$A$ is commutative if and only if it is abelian as a Lie~algebra.
\end{example}


\begin{lemma}
  Let~$\glie$ be a Lie~algebra.
  \begin{enumerate}
    \item
    If~$I_\lambda$,~$\lambda \in \Lambda$ is a collection of ideals of~$\glie$ then their intersection~$\bigcap_{\lambda \in \Lambda} I_\lambda$ and their sum~$\sum_{\lambda \in \Lambda} I_\lambda$ are again ideals in~$\glie$.
    \item
    If~$I$ and~$J$ are ideals in~$\glie$ then their commutator~$[I,J]$ is again an ideal in~$\glie$.
  \end{enumerate}
\end{lemma}


\begin{proof}
  \leavevmode
  \begin{enumerate}
    \item
      This follows from direct calculation.
    \item
      As the commutator~$[I,J]$ is spanned by the elements~$[y,z]$ with~$y \in I$ and~$z \in J$ it sufficies to show that~$[x,[y,z]] \in [I,J]$ for all~$x \in \glie$,~$y \in I$ and~$z \in J$.
      This follows from~$I, J \ideal \glie$ and the Jacobi identity because
      \[
        [x,[y,z]]
        =
        [[x,y], z] + [y, [x,z]]
        \subseteq
        [I, z] + [y, J]
        \subseteq
        [I, J] + [J, I]
        =
        [I,J] \,.
      \]
      Here we use that~$[J,I] = -[I,J] = [I,J]$ by the antisymmetry of the Lie bracket~$[-,-]$.
   \qedhere
 \end{enumerate}
\end{proof}


\begin{definition}
  A Lie~algebra~$\glie$ is~\defemph{linear}\index{linear} if~$\glie$ is a Lie~subalgebra of~$\gllie(V)$ for some finite dimensional vector space~$V$, or a Lie~subalgebra of~$\gllie_n(\kf)$ for some~$n$.
\end{definition}


% TODO: Fix this window.
\begin{examples}[Linear Lie~algebras]
  \leavevmode
  \begin{enumerate}
  \item
    Let~$\glie \defined \gllie_n(\kf)$.
    Then
    \[
      \gls*{special lie matrix}
      \defined
      \{
        A \in \glie
      \suchthat
        \tr A = 0
      \}
    \]
    is an ideal in~$\glie$, namely
    \[
      \sllie_n(\kf)
      =
      [\glie,\glie]  \,.
    \]
      
    Indeed, we have for all~$A, B \in \glie$ that~$\tr(AB) = \tr(BA)$ and hence
    \[
        \tr( [A,B] )
      = \tr(AB-BA)
      = \tr(AB) - \tr(BA)
      = \tr(AB) - \tr(AB)
      = 0  \,.
    \]
    This shows that~$[\glie, \glie] \subseteq \sllie_n(\kf)$.
    Observe on the other hand that~$\sllie_n(\kf)$ has a basis given by the matrices~$E_{ij}$ with~$1 \leq i \neq j \leq n$ together with the matrices~$E_{11} - E_{ii}$ with~$i = 2, \dotsc, n$.
    Each of these matrices is given as a commutator, namely
    \[
        E_{ij}
        =
        E_{ij} E_{jj} - \underbrace{ E_{jj} E_{ij} }_{=0}
        =
        [E_{ij}, E_{jj}]
    \]
    for~$1 \leq i \neq j \leq n$ and
    \[
      E_{11} - E_{ii}
      =
      E_{1i} E_{i1} - E_{i1} E_{1i}
      =
      [E_{1i}, E_{i1}]
    \]
    for~$i = 2, \dotsc, n$.
    Therefore~$\sllie_n(\kf) \subseteq [\glie, \glie]$.
    
    The Lie~algebra~$\sllie_n(\kf)$ is the \defemph{special linear Lie~algebra}\index{special linear Lie algebra}.
    
    If~$V$ is any finite dimensional~{\vectorspace{$\kf$}} then
    \[
      \gls*{special lie endomorphism}
      \defined
      [\gllie(V), \gllie(V)]
      =
      \{
        f \in \gllie(V)
      \suchthat
        \tr(f) = 0
      \}
    \]
    is the \defemph{special linear Lie~algebra}\index{special linear Lie algebra} of~$V$.
   
  \item
    The upper triangular matrices\index{upper triangular matrices}\index{triangular matrices!upper}
    \[
      \gls*{triangular lie matrix}
      \defined
      \left\{
        \begin{pmatrix}
            a_{11}
          & \cdots
          & \cdots
          & a_{1n}
          \\
            0
          & \ddots
          & {}
          & \vdots
          \\
            \vdots
          & \ddots
          & \ddots
          & \vdots
          \\
            0
          & \cdots
          & 0
          & a_{nn}
        \end{pmatrix}
      \suchthat*
        \text{$a_{ij} \in \kf$ for all~$1 \leq i \leq j \leq n$}
      \right\}
    \]
    form a Lie~subalgebra of~$\gllie_n(\kf)$.
    This holds because the space of upper triangular matrices is a~{\subalgebra{$\kf$}} of~$\Mat_n(\kf)$, and are hence closed under the commutator~$[A,B] = AB - BA$.
   
  \item
    The strictly upper triangular matrices\index{stricly upper triangular matrices}\index{triangular matrices!strictly upper}
    \[
      \gls*{strictly triangular lie matrix}
      \defined
      \left\{
        \begin{pmatrix}
            0
          & a_{12}
          & \cdots
          & a_{1n}
          \\
            \vdots
          & \ddots
          & \ddots
          & \vdots
          \\
            \vdots
          & {}
          & \ddots
          & a_{n-1,n}
          \\
            0
          & \cdots
          & \cdots
          & 0
        \end{pmatrix}
      \suchthat*
        \text{$a_{ij} \in \kf$ for all~$1 \leq i < j \leq n$}
      \right\}
    \]
    form a Lie~subalgebra of~$\tlie_n(\kf)$ and therefore also of~$\gllie_n(\kf)$.
    The space~$\nlie_n(\kf)$ is even an ideal in~$\tlie_n(\kf)$, namely~$\nlie_n(\kf) = [\tlie_n(\kf), \tlie_n(\kf)]$:
   
    For any two matrices~$A, B \in \tlie_n(\kf)$ their products~$AB$ and~$BA$ are again upper triangular and both products have the same diagonal entries.
    The commutator~$[A,B] = AB - BA$ is therefore a strictly upper triangular matrix.
    Hence~$[\tlie_n(\kf), \tlie_n(\kf)] \subseteq \nlie_n(\kf)$.
    We know on the other hand that~$\nlie_n(\kf)$ has as a basis the matrices~$E_{ij}$ with~$1 \leq i < j \leq n$, and each of those matrices can be written as a commutator
    \[
      E_{ij}
      =
      E_{ii} E_{ij} - \underbrace{ E_{ij} E_{ii} }_{= 0}
    \]
    with~$E_{ii}, E_{ij} \in \tlie_n(\kf)$.
    Therefore~$\nlie_n(\kf) \subseteq \tlie_n(\kf)$.
  \end{enumerate}
\end{examples}


\begin{remark}
  The Lie~algebra~$\sllie_2(\kf)$ will play a crucial role for our understanding of semisimple Lie~algebras.
  We already observe from the above discussion that~$\sllie_2(\kf)$ has a basis given by
  \[
    e
    \defined
    \begin{pmatrix}
      0 & 1 \\
      0 & 0
    \end{pmatrix} \,,
    \qquad
    h
    \defined
    \begin{pmatrix*}[r]
      1 &  0  \\
      0 & -1
    \end{pmatrix*}  \,,
    \qquad
    f
    \defined
    \begin{pmatrix}
      0 & 0 \\
      1 & 0
    \end{pmatrix} \,.
  \]
  A straightforward calculation shows that the Lie~bracket~$[-,-]$ of~$\sllie_2(\kf)$ is on these basis elements given by
  \[
    [h, e]
    =
    2e  \,,
    \qquad
    [h, f]
    =
    -2f \,,
    \qquad
    [e,f]
    =
    h \,.
  \]
  We will come back to~$\sllie_2(\kf)$ and the properties of this basis later on.
\end{remark}


\begin{definition}
  If~$\glie$ is a Lie~algebra and~$U \subseteq \glie$ any linear subspace then
  \[
    \gls*{normalizer}
    \defined
    \{
      x \in \glie
    \suchthat
      \text{$[x,y] \in \glie$ for every~$y \in U$}
    \}
  \]
  is the \defemph{normalizer}\index{normalizer} of~$U$ in~$\glie$, and
  \[
    \gls*{centralizer of space}
    \defined
    \{
      x \in \glie
    \suchthat
      \text{$[x,y] = 0$ for every~$y \in U$}
    \}
  \]
  is the \defemph{centralizer}\index{centralizer} of~$U$ in~$\glie$.
  For a single element~$x \in \glie$ the \defemph{centralizer}\index{centralizer} of~$x$ in~$\glie$ is
  \[
    \gls*{centralizer of element}
    \defined
    \{
      y \in \glie
    \suchthat
      [x,y] = 0
    \} \,.
  \]
\end{definition}


\begin{lemma}
 Let~$\glie$ be a Lie~algebra and~$U \subseteq \glie$ a linear subspace.
 Then both the normalizer~$\normallie_{\glie}(U)$ and the centralizer~$\centerlie_{\glie}(U)$ are Lie~subalgebras of~$\glie$.
 If~$x \in \glie$ is any element then the centralizer~$\centerlie_{\glie}(x)$ is a Lie~subalgebra of~$\glie$.
\end{lemma}


\begin{proof}
 If~$x, y \in \normallie_{\glie}(U)$ then it follows for every~$z \in U$ from the Jacobi~identity that
 \[
  [[x,y], z]
  =
  - [z, [x,y]]
  =
  - [[z,x], y] - [x, [z,y]
  \in
  - [U, y] - [x, U]
  \subseteq
  - U - U
  =
  U
 \]
 and therefore~$[x,y] \in \normallie_{\glie}(U)$.
 In the same way it follows for all~$x, y \in \glie$ and~$z \in U$ that
 \[
  [[x,y], z]
  =
  - [[z,x], y] - [x, [z,y]]
  =
  - [0, y] - [x, 0]
  =
  0
 \]
 and therefore~$[x,y] \in \centerlie_{\glie}(U)$.
 
 The span~$\kf x$ of any~$x \in \glie$ is a linear subspace of~$\glie$ with~$\centerlie_{\glie}(x) = \centerlie_{\glie}(\kf x)$.
 Hence~$\centerlie_{\glie}(x)$ is a Lie~subalgebra of~$\glie$, as shown above.
\end{proof}


\begin{remark}
 Let~$\glie$ be a Lie~algebra and~$L \subseteq \glie$ a linear subspace.
 Then~$L$ is a Lie~subalgebra of~$\glie$ if and only if~$L \subseteq \normallie_{\glie}(L)$.
 Then~$L$ is not only contained in~$\normallie_{\glie}(L)$ but~$\normallie_{\glie}(L)$ is the maximal Lie~subalgebra of~$\glie$ that contains~$L$ as an ideal.
 As a special case of this, we find that~$L$ is an ideal in~$\glie$ if and only if~$\normallie_{\glie}(L) = \glie$.
\end{remark}





\subsection{New Lie~Algebras From Old Ones}


\begin{definition}
  Let~$\glie$ and~$\hlie$ be Lie~algebras over the same field~$\kf$.
  Then the \,\defemph{product}\index{product} of~$\glie$ and~$\hlie$ is the~{\vectorspace{$\kf$}}~$\gls*{product of lie algebras}$ together with the Lie~bracket
  \[
    [(x,y), (x',y')]
    \defined
    ([x, x'], [y, y'])
  \]
  for all~$(x,y), (x',y') \in \glie \times \hlie$.
\end{definition}


\begin{lemma}
  \label{construction of quotient lie algebra}
  Let~$\glie$ be a Lie~algebra and let~$I$ an ideal in~$\glie$.
  Then the quotient vector space~$\glie/I$ inherits from~$\glie$ the structure of a Lie~algebra via
  \[
    [x+I, y+I]
    \defined
    [x,y] + I
  \]
  for all~$x, y \in \glie$.
\end{lemma}


\begin{definition}
  The Lie algebra~$\gls*{quotient lie algebra}$ from \cref{construction of quotient lie algebra} is the \defemph{quotient}\index{quotient} of~$\glie$ by~$I$.
\end{definition}



\begin{proof}[Proof of \cref*{construction of quotient lie algebra}]
  The bilinear map~$[-,-]$ on~$\glie/I$ is well-defined:
  If~$x, y, x', y' \in \glie$ with both~$x+I = x'+I$ and~$y+I = y'+I$ then~$x-x' \in I$ and~$y-y' \in I$ and thus
  \begin{align*}
    [x,y] + I
    &=
    [x' + (x-x'), y' + (y-y')] + I \\
    &=
    [x',y']
    + \underbrace{[x', y-y']}_{\in I}
    + \underbrace{[x-x', y']}_{\in I}
    + \underbrace{[x-x', y-y']}_{\in I}
    + I
    \\
    &=
    [x', y'] + I  \,.
  \end{align*}
  The additional requirements for a Lie~bracket can be checked on representatives, for which they hold because~$\glie$ is a Lie~algebra.
\end{proof}


\begin{lemma}
  \label{quasi extension of scalars for lie algebras}
  Let~$\glie$ be a Lie~algebra over a field~$\kf$ and let~$A$ be an associative, commutative~{\algebra{$\kf$}}.
  Then~$A \tensor_\kf \glie$ is again a Lie~algebra over~$\kf$ via
  \[
    [a \tensor x, b \tensor y]
    =
    (ab) \tensor [x,y]
  \]
  for all simple tensors~$a \tensor x, b \tensor y \in A \tensor_\kf \glie$.
  Similarly~$\glie \tensor_k A$ carries the structure of a~{\liealgebra{$\kf$}} via
  \[
    [x \tensor a, y \tensor b]
    =
    [x,y] \tensor (ab)
  \]
  for all simple tensors~$x \tensor a, y \tensor b \in \glie \tensor A$.
\end{lemma}


\begin{example}
  \index{extension of scalars}
  If~$\Lf/\kf$ is a field extension and~$\glie$ a Lie~algebra over~$\kf$, then~$\Lf \tensor_\kf \glie$ is a Lie~algebra over~$\kf$ via
  \[
    [\lambda \tensor x, \mu \tensor y]
    = 
    (\lambda \mu) \tensor [x,y]
  \]
  for all simple tensor~$\lambda \tensor x, \mu \tensor y \in \Lf \tensor_\kf \glie$.
  But~$\Lf \tensor_\kf \glie$ also carries the structure of an {\vectorspace{$\Lf$}} via extension of scalars, i.e.\ via
  \[
    \lambda \cdot (\mu \tensor x)
    =
    (\lambda \mu) \tensor x
  \]
  for all scalars~$\lambda \in \Lf$ and simple tensors~$\mu \tensor x \in \Lf \tensor_\kf \glie$.
  The Lie~bracket~$[-,-]$ is not only~{\bilinear{$\kf$}} but already~{\bilinear{$\Lf$}}, so that the {\liealgebra{$\kf$}} structure of~$\Lf \tensor_\kf \glie$ can be extended to an~{\liealgebra{$\Lf$}} structure.
  (Note that the validity of the Jacobi identity does not depend on the ground field.)
\end{example}


\begin{definition}
  Let~$\glie$ be a Lie~algebra and let~$\kf[t, t^{-1}]$ be the algebra of Laurant polynomials over~$\kf$.
  Then
  \[
    \gls*{loop lie algebra}
    \defined
    \glie \tensor_\kf \kf[t, t^{-1}]
  \]
  with the Lie~bracket as in Lemma~\ref{quasi extension of scalars for lie algebras} is the \defemph{loop \textup(Lie\textup)~algebra}\index{loop Lie algebra} of~$\glie$.
\end{definition}
 
 
\begin{example}
  Another example for constructing new Lie~algebras out of old ones are \defemph{central extensions}:
  Let~$\glie$ be any~$\kf$-Lie~algebra.
  Then let
  \[
    \widetilde{\glie}
    \defined
    \glie \oplus \kf
    =
    \{
      x + \lambda c
    \suchthat
      x \in \glie,
      \lambda \in \kf
    \},
  \]
  where we understand~$c$ as a formal variable.
  Suppose that~$\kappa \colon \glie \times \glie \to \kf$ is a~{\bilinear{$\kf$}} map satisfying the following properties:
  \begin{enumerate}
  \item
    $\kappa$ is antisymmetric, i.e.~$\kappa(x,y) = -\kappa(y,x)$ for all~$x,y \in \glie$.
  \item
    $\kappa$ satisfies the \defemph{$2$-cocycle condition}\index{$2$-cocycle condition}
    \[
      \kappa([x,y],z) + \kappa([y,z],x) + \kappa([z,x],y) = 0
    \]
    for all~$x, y, z \in \glie$.
  \end{enumerate}
  Then~$\widetilde{\glie}$ becomes a Lie~algebra via
  \[
    [x + \lambda c, y + \mu c]
    \defined
    [x,y] + \kappa(x,y) \lambda \mu c
  \]
  for all~$x, y \in \glie$ and~$\lambda, \mu \in \kf$.
  Note that the element~$c$ is central in~$\widetilde{\glie}$ in the sense that~$[x,c] = 0$ for all~$x \in \glie$.
  
  Take for example~$\glie \defined \gllie_n(\kf)$.
  We can then define a symmetric bilinear form~$(-,-)_{\tr}$ on~$\glie$ via
  \[
    (A,B)_{\tr}
    \defined
    \tr(AB)
  \]
  for all~$A, B \in \glie$.
  We can use~$(-,-)_{\tr}$ to define on the Loop~algebra~$\looplie(\glie)$ a~{\bilinear{$k[t,t^{-1}]$}} form~$(-,-)$ via
  \[
    \looplie(\glie) \times \looplie(\glie)
    \to
    \kf[t,t^{-1}] \,,
    \quad
    (x \tensor p, y \tensor q)
    \mapsto
    (x,y)_{\tr} \, pq \,.
  \]
  We get from this a bilinear form a~{\twococycle}~$\kappa \colon \looplie(\glie) \times \looplie(\glie) \to \kf$ via
  \[
    \kappa(a,b)
    \defined
    \Res\left( \frac{\partial a}{\partial t}, b \right) \,.
  \]
  The bilinear form~$\kappa$ is also antisymmetric:
  Let~$a = x \tensor t^i$ and~$b = y \tensor t^j$ with~$x,y \in \glie$ and~$i,j \in \Integer$.
  Then
  \begin{align*}
    \kappa(x \tensor t^i, y \tensor t^j)
    &=
    \Res(i x \tensor t^{i-1}, y \tensor t^j)
    \\
    &= 
    \Res(i t^{i+j-1} (x,y)_{\tr})
    \\
    &=
    \begin{cases}
      i (x,y)_{\tr} & \text{if~$i+j = 0$} \,,\\
                  0 & \text{otherwise}  \,.
    \end{cases}
  \end{align*}
  In the same way we find that
  \[
    \kappa(y \tensor t^j, x \tensor t^i)
    =
    \begin{cases}
      j (x,y)_{\tr} & \text{if~$i+j = 0$} \,, \\
                  0 & \text{otherwise}  \,.
    \end{cases}
  \]
  Since~$(\cdot,\cdot)_{\tr}$ is symmetric we find that
  \begin{align*}
    \kappa(x \tensor t^i, y \tensor t^j)
    &=
    \begin{cases}
    i (x,y)_{\tr} & \text{if~$i+j = 0$} \,, \\
                0 & \text{otherwise}  \,,
    \end{cases} \\
    &=
    \begin{cases}
    -j (x,y)_{\tr} & \text{if~$i+j = 0$}  \,, \\
                  0 & \text{otherwise}  \,,
    \end{cases} \\
    &=
    -\kappa(y \tensor t^j, x \tensor t^i) \,.
  \end{align*}
\end{example}


\begin{remark}
  During the rest of these notes we will never see the Loop algebra again.
\end{remark}





\subsection{Homomorphisms of Lie~Algebras}


\begin{definition}
 Given Lie~algebras~$\glie$ and~$\hlie$ over the same field~$\kf$ a \defemph{homomorphism of Lie~algebras}~$\glie \to \hlie$\index{homomorphism!of Lie algebras} is a~{\linear{$\kf$}} map~$f \colon \glie \to \hlie$ such that
 \[
  f([x,y])
  =
  [f(x),f(y)]
 \]
 for all~$x, y \in \glie$.
\end{definition}


\begin{examples}
  \label{homomorphisms of lie algebras}
  \leavevmode
  \begin{enumerate}
    \item
      For any Lie~algebra~$\glie$ the identity~$\id_{\glie} \colon \glie \to \glie$ is a Lie~algebra homomorphism.
    \item
      Given Lie~algebras~$\glie$,~$\hlie$ and~$\klie$ and Lie~algebra homomorphisms~$f \colon \glie \to \hlie$ and~$g \colon \hlie \to \klie$ the composition~$g \circ f \colon \glie \to \klie$ is again a homomorphism of Lie~algebras.
    \item
      If~$\glie$ is a Lie~algebra and~$\hlie \subseteq \glie$ a Lie~subalgebra then the inclusion~$\hlie \inclusion \glie$ is a homomorphism of Lie~algebras.
    \item
      Given two abelian Lie~algebras~$\glie$ and~$\hlie$ any linear map~$\glie \to \hlie$ is already a homomorphism of Lie~algebras.
    \item
      Let~$\glie$ be a Lie~algebra.
      Then for every~$x \in \glie$ let
      \[
        \gls*{adjoint representation}(x)
        \colon
        \glie
        \to
        \glie \,,
        \qquad
        \ad(x)(y)
        \defined
        [x,y]
      \]
      for every~$y \in \glie$.\index{adjoint representation}
      Then the map~$\ad \colon \glie \to \gllie(\glie)$ is a homomorphism of Lie~algebras.
      This follows from the Jacobi~identity because for all~$x,y,z \in \glie$
      \begin{align*}
          \ad([x,y])(z)
          &=
          [[x,y], z]
          \\
          &=
          - [z, [x,y]]
          \\
          &=
          - [[z,x], y] - [x, [z,y]]
          \\
          &=
          [x, [y,z]] - [y, [x,z]]
          \\
          &=
          \ad(x)\ad(y)(z) - \ad(y)\ad(x)(z)
          \\
          &=
          [\ad(x), \ad(y)](z) \,.
      \end{align*}
    \item
    If~$A$ and~$B$ are associative~{\algebras{$\kf$}} then any homomorphism of~{\algebras{$\kf$}}~$f \colon A \to B$ is also a homomorphism of Lie~algebras because
    \[
      f([a,b])
      =
      f(ab - ba)
      =
      f(a)f(b) - f(b)f(a)
      =
      [f(a), f(b)]
    \]
    for all~$a, b \in A$.
  \item
    Let~$\glie$ be a Lie~algebra over an arbitary field~$\kf$.
    If~$\phi \colon \sllie_2(\kf) \to \glie$ is a homomorphism of Lie~algebras then the images
    \[
      E \defined \phi(e)  \,,
      \qquad
      H \defined \phi(h)  \,,
      \qquad
      F \defined \phi(f)
    \]
    satisfy the relations
    \[
      [H, E] = 2E  \,,
      \qquad
      [H, F] = -2F  \,,
      \qquad
      [E, F] = H \,.
    \]
    On the other hand every such triple~$(E', H', F')$ of elements satisfying the above relations (with~$X$ replaced by~$X'$ for~$X \in \{ E, H, F \}$) gives rise to a unique homomorphism of Lie~algebras~$\phi' \colon \sllie_2(k) \to \glie$ that is given by
    \[
      \phi'(E) = E' \,,
      \qquad
      \phi'(H) = H' \,,
      \qquad
      \phi'(F) = F' \,.
    \]
   
    Hence there is a bijection between Lie~algebra homomorphisms~$\sllie_2(\kf) \to \glie$ and triples as above.
    Such triples will play an important part later on.
%   TODO: Do they?
  \end{enumerate}
\end{examples}


\begin{lemma}
  \leavevmode
  \begin{enumerate}
    \item
      If~$\glie$ and~$\hlie$ are Lie~algebras over the same field~$\kf$ then the canonical projections
      \begin{align*}
        p
        \colon
        \glie \times \hlie
        \to
        \glie,
        \quad
        (x,y)
        \mapsto
        x
      \shortintertext{and}
        q
        \colon
        \glie \times \hlie
        \to
        \hlie,
        \quad
        (x,y)
        \mapsto
        y
      \end{align*}
      are homomorphisms of Lie-algebras.
    \item
      If~$\glie$ is a Lie~algebra and~$I$ an ideal in~$\glie$ then the canonical projection~$\pi \colon \glie \to \glie/I$ given by~$x \mapsto x+I$ is a homomorphism of Lie~algebras.
    \qed
  \end{enumerate}
\end{lemma}


\begin{definition}
 Let~$\glie, \hlie$ be Lie~algebras over the same field~$\kf$.
 A homomorphism of Lie~algebras~$f \colon \glie \to \hlie$ is an~\defemph{isomorphism of Lie~algebras}\index{isomorphism!of Lie algebras} if~$f$~is bijective.
\end{definition}


\begin{lemma}
 If~$f \colon \glie \to \hlie$ is an isomorphism of~{\liealgebras{$\kf$}}, then the~{\linear{$\kf$}} map~$f^{-1} \colon \hlie \to \glie$ is again a homomorphism of Lie~algebras and therefore also an isomorphism.
\end{lemma}


\begin{proof}
 We find that
 \begin{align*}
  f^{-1}( [x,y] )
  &=
  f^{-1}( [ f(f^{-1}(x)), f(f^{-1}(y)) ] ) \\
  &=
  f^{-1}(f( [f^{-1}(x), f^{-1}(y)] )) \\
  &=
  [f^{-1}(x), f^{-1}(y)]
 \end{align*}
 for all~$x, y \in \hlie$.
\end{proof}


\begin{remark}
 It follows that Lie~algebras over the same field~$\kf$ together with homomorphisms of Lie~algebras and their usual composition form a category, that will be denoted by~$\cLie{\kf}$.
 An isomorphism of~{\liealgebras{$\kf$}} is then the same as an isomorphism in the category~$\cLie{\kf}$.
\end{remark}


\begin{example}[Classification of one- and two-dimensional Lie~algebras]
  Let~$\kf$ be any field.
  
  Two finite dimensional abelian~{\liealgebras{$\kf$}} are isomorphic (as Lie~algebras) if and only if they are isomorphic as~{\vectorspaces{$\kf$}}, because every vector space isomorphism between them is already a Lie~algebra isomorphism.
  There hence exists for every~$n \geq 0$ up to isomorphism precisely one abelian~{\liealgebra{$\kf$}} of dimension~$n$.
  
  If~$\glie$ is a~{\onedimensional} Lie~algebra over~$\kf$ then the Lie~bracket of~$\glie$ is zero because it is alternating, so that~$\glie$ is abelian.
  Hence there is up to isomorphism precisely one {\onedimensional} Lie~algebra over~$\kf$, as seen above.
  
  We have also seen that there exist precisely one {\twodimensional} abelian~{\liealgebra{$\kf$}} up to isomorphism.
  We claim that there also exists precisely none non-abelian {\twodimensional}~{\liealgebra{$\kf$}} up to isomorphism:
  
  Suppose that~$\glie$ is a {\twodimensional}, non-abelian Lie~algebra over~$\kf$.
  Let~$\tilde{x}$,~$\tilde{y}$ be a basis of~$\glie$.
  We have~$[\glie,\glie] \neq 0$ because~$\glie$ is non-abelian, and~$[\glie, \glie]$ is spanned by the single element~$[\tilde{x}, \tilde{y}]$ because the Lie~bracket is alternating.
  Hence~$[\glie, \glie] = \kf [\tilde{x}, \tilde{y}]$ with~$[\tilde{x}, \tilde{y}] \neq 0$.
  Let~$x \defined [\tilde{x},\tilde{y}]$ and~$y \in \glie$ such that~$x$,~$y$ is a basis of~$\glie$.
  Then~$[x,y] \neq 0$ because~$\glie$ is non-abelian, and~$[x,y] \in \kf x$.
  By rescaling~$y$ it can be assumed that~$[x,y] = x$.
 This shows that~$\glie$ is up to isomorphism the unique {\twodimensional} non-abelian Lie~algebra over~$\kf$.
 
  Such a Lie~algebra also exists.
  It can be realized as a subalgebra of~$\gllie_2(\kf)$ by choosing
  \[
    x
    \defined
    \begin{pmatrix}
      0 & 1 \\
      0 & 0
    \end{pmatrix}
    =
    E_{12}  \,,
    \qquad
    y
    \defined
    \begin{pmatrix}
      0 & 0 \\
      0 & 1
    \end{pmatrix}
    =
    E_{22}  \,.
  \]
  A direct calculation shows that indeed
  \[
    [x,y]
    =
    [E_{12}, E_{22}]
    =
    E_{12} E_{22} - E_{22} E_{12}
    =
    E_{12} - 0
    =
    E_{12}
    =
    x \,.
  \]
  Hence there exist up to isomorphism precisely two {\twodimensional} Lie~algebras over~$\kf$, one of which is abelian and the other non-abelian.
\end{example}


\begin{proposition}[Homomorphism theorem]
  Let~$\glie$ and~$\hlie$ be Lie~algebras and let~$f \colon \glie \to \hlie$ be a homomorphism of Lie~algebras.
  \begin{enumerate}
    \item
      The kernel~$\ker f$ is an ideal in~$\glie$.
    \item
      The image~$\im f$ is a Lie~subalgebra of~$\hlie$.
    \item
      If~$I$ is any ideal in~$\glie$ with~$\ker f \subseteq I$ then there exists a unique homomorphism of Lie~algebras~$\tilde{f} \colon \glie/I \to \hlie$ with~$f = \tilde{f} \circ \pi$ where~$\pi \colon \glie \to \hlie/I$ is the canonical projection.
      \[
        \begin{tikzcd}
          \glie
          \arrow{r}[above]{f}
          \arrow{d}[left]{\pi}
          &
          \hlie
          \\
          \glie/I
          \arrow[dashed]{ur}[below right]{\tilde{f}}
          &
          {}
        \end{tikzcd}
      \]
    \item
      If~$I$ and~$J$ are two ideals in~$\glie$ are two ideals with~$I \subseteq J$ then~$J/I$ is an ideal in~$\glie/I$ and the natural isomorphism of vector spaces
      \[
        (\glie/I) / (J/I)
        \to
        \glie/I \,,
        \quad
        (x+I) + (J/I)
        \mapsto
        x+I
      \]
      is already a natural isomorphism of Lie~algebras.
    \item
      For any two ideals~$I$ and~$J$ in~$\glie$ the natural isomorphism of vector spaces
      \begin{gather*}
        (I + J)/J
        \to
        I/(I \cap J)  \,,
        \quad
        (x+J)+I
        \mapsto
        x + (I \cap J)
      \end{gather*}
      is already a natural isomorphism of Lie~algebras.
  \end{enumerate}
\end{proposition}


\begin{corollary}
  The center~$\centerlie(\glie)$ of a Lie~algebra~$\glie$ is an ideal in~$\glie$.
\end{corollary}


\begin{proof}
  The center~$\centerlie(\glie)$ is precisely the kernel of the Lie~algebra homomorphism~$\ad \colon \glie \to \gllie(\glie)$.
\end{proof}


\begin{remark}
  For a Lie~algebra~$\glie$ the ideal~$[\glie,\glie]$ is the minimal ideal inside~$\glie$ such that~$\glie/[\glie,\glie]$ is abelian.
  Any homomorphism of Lie~algebras~$\glie \to \hlie$ into an abelian Lie~algebra~$\hlie$ factorizes through a unique homomorphism of Lie~algebras~$\glie/[\glie, \glie] \to \hlie$.
  \[
    \begin{tikzcd}[column sep = small]
      \glie
      \arrow{rr}
      \arrow{dr}
      &
      {}
      &
      \hlie
      \\
      {}
      &
      \glie/[\glie,\glie]
      \arrow[dashed]{ur}
      &
      {}
    \end{tikzcd}
  \]
\end{remark}


\begin{lemma}
  \label{homomorphisms respect commutators of sets}
  If~$f \colon \glie \to \hlie$ is a homomorphism of Lie~algebras then
  \[
    f([X,Y])
    =
    [f(X), f(Y)]
  \]
  for any two subsets~$X, Y \subseteq \glie$.
  \qed
\end{lemma}




\subsection{Derivations}


\begin{definition}
  Let~$A$ be a~{\algebra{$\kf$}} that is not necessarily unitary of even associative.
  A \defemph{derivation of~$A$}\index{derivation} is a {\linear{$\kf$}} map~$d \colon A \to A$ such that
  \[
    d(ab)
    =
    d(a)b + ad(b)
  \]
  for all~$a, b \in A$.
  We set
  \[
    \gls*{derivations}
    \defined
    \{
      d
      \colon
      A
      \to
      A
    \suchthat
      \text{$d$ is a derivation of~$A$}
    \}  \,.
  \]
\end{definition}


\begin{remark}
  The space of derivations~$\Der(A)$ is a~{\linear{$\kf$}} subspace of~$\End_\kf(A)$.
\end{remark}


\begin{example}
  Let~$A$ be a~{\algebra{$\kf$}} (as above).
  It follows from direct calculation that for all derivations~$d, d' \in \Der(A)$ their commutator~$[d,d'] = d \circ d' - d' \circ d$ is again a derivation.
  Hence~$\Der(A)$ is a Lie~subalgebra of~$\gllie(A)$.
\end{example}


\begin{lemma}
\label{lie algebras act adjoint by derivations}
  Let~$\glie$ be a Lie~algebra.
  Then for any~$x \in \glie$ the map
  \[
    \gls*{adjoint representation}(x)
    \colon
    \glie
    \to
    \glie,
    \quad
    y
    \mapsto
    [x,y]
  \]
  is a derivation of~$\glie$.\index{adjoint representation}
\end{lemma}


\begin{proof}
  By the Jacobi identity
  \[
    \ad(x)([y,z])
    =
    [x,[y,z]]
    =
    [[x,y],z] + [y,[x,z]] \\
    =
    [\ad(x)(y), z] + [y, \ad(x)(z)]
  \]
  for all~$y,z \in \glie$.
\end{proof}


\begin{definition}
 Let~$\glie$ be a Lie~algebra.
 A derivation of~$\glie$ is \defemph{inner}\index{inner derivation}\index{derivation!inner} if it is of the form~$\ad(x)$ for some~$x \in \glie$.
\end{definition}


\begin{lemma}
  \label{commutator of any derivation and inner derivation}
  Let~$\glie$ be a Lie~algebra.
  If~$x \in \glie$ and~$\delta \in \Der(\glie)$ then~$[\delta, \ad(x)] = \ad(\delta(x))$.
\end{lemma}


\begin{proof}
 We have that
 \begin{align*}
  [\delta, \ad(x)](y)
  &= 
  (\delta \ad(x) - \ad(x) \delta(x))(y)
  \\
  &=
  \delta([x,y]) - [x,\delta(y)]
  \\
  &=
  [\delta(x),y] + [x,\delta(y)] - [x,\delta(y)]
  \\
  &=
  [\delta(x),y]
  \\
  &=
  \ad(\delta(x))(y)
 \end{align*}
 for every~$y \in \glie$.
\end{proof}


\begin{corollary}
  \label{inner derivations are an ideal}
  The inner derivations of a Lie~algebra~$\glie$ form an ideal inside the Lie~algebra~$\Der(\glie)$.
\end{corollary}

\begin{proof}
  That the image~$\ad(\glie)$ is a~{\linear{$\kf$}} subspace of~$\Der(A)$ follows from the linearity of the map~$\ad$.
  That~$\ad(\glie)$ is already an ideal in~$\Der(A)$ follows from \cref{commutator of any derivation and inner derivation}.
\end{proof}







\subsection{Simple Lie~algebras}


\begin{definition}
 A Lie~algebra~$\glie$ is \defemph{simple}\index{simple!Lie algebra}\index{Lie!algebra!simple} if it is non-abelian, nonzero and contains no ideals apart from~$0$ and~$\glie$ itself.
\end{definition}


\begin{remark}
  A Lie~algebra is simple if and only if it is non-abelian and contains precisely two ideals.
\end{remark}


\begin{lemma}
 Let~$\glie$ be a simple Lie~algebra.
 Then~$[\glie, \glie] = \glie$ and~$\centerlie(\glie) = 0$.
\end{lemma}


\begin{proof}
 The Lie~algebra~$\glie$ is non-abelian because it is simple.
 Hence~$[\glie, \glie] \neq 0$ and~$\centerlie(\glie) \neq \glie$.
 As both~$[\glie, \glie]$ and~$\centerlie(\glie)$ are ideals in~$\glie$ it follows that~$[\glie, \glie] = \glie$ and~$\centerlie(\glie) = 0$ because~$\glie$ is simple.
\end{proof}


\begin{remark}
  If~$\glie$ is a finite dimensional simple Lie~algebra then~$\glie$ is isomorphic to the Lie~subalgebra~$\im \ad = \ad(\glie)$ of~$\gllie(\glie)$ because~$\ker \ad = \centerlie(\ad) = 0$.
  (We have seen in \cref{homomorphisms of lie algebras} that~$\ad$ is a Lie algebra homomorphism.)
  The Lie~algebra~$\glie$ is hence isomorphic to a linear Lie~algebra.
  
  A famous theorem by Ado --- which we will attempt to prove in this lecture --- states that this conclusion does actually hold for every finite dimensional Lie algebra:
\end{remark}


\begin{theorem}[Ado]
  \index{Ado’s theorem}
  Every finite dimensional Lie~algebra~$\glie$ is isomorphic to a linear Lie~algebra.
\end{theorem}


\begin{examples}
  \leavevmode
  \begin{enumerate}
    \item
      The Lie~algebra~$\gllie_n(\kf)$ is never simple because~$[\gllie_n(\kf), \gllie_n(\kf)] = \sllie_n(\kf) \neq \gllie_n(\kf)$ for~$n \geq 1$, and~$\gllie_n(\kf) = 0$ for~$n = 0$.
    \item
      The Lie~algebra~$\glie \defined \sllie_2(\kf)$ is simple if and only if~$\ringchar \kf \neq 2$:
      
      Consider the basis~$(e,h,f)$ of~$\sllie_2(\kf)$ consisting of the matrices
      \[
        e
        =
        \begin{pmatrix}
          0 & 1 \\
          0 & 0
        \end{pmatrix} \,,
        \qquad
        h
        =
        \begin{pmatrix*}[r]
          1 &  0  \\
          0 & -1
        \end{pmatrix*},
        \qquad
        f
        =
        \begin{pmatrix}
          0 & 0 \\
          1 & 0
        \end{pmatrix} \,.
      \]
      Then
      \[
        [h,e] = 2e  \,,
        \qquad
        [h,f] = -2f \,,
        \qquad
        [e,f] = h \,.
      \]
      
      If~$\ringchar \kf = 2$ then the element~$h$ spans a {\onedimensional} ideal whence~$\sllie_2(\kf)$ is not simple.
      
      For the case~$\ringchar \kf \neq 2$ let~$I$ be a nonzero ideal in~$\sllie_2(\kf)$ and let~$x \in I$ with~$x \neq 0$.
      Then we may write~$x = \alpha e + \beta h + \gamma f$ for some scalar~$\alpha, \beta, \gamma \in \kf$.
      It follows that
      \[
        [e,x]
        =
        -2 \beta e + \gamma h
        \qquad \text{and thus}\qquad
        [e,[e,x]]
        =
        -2 \gamma e \,.
      \]
      with~$[e,x] \in I$ and~$[e,[e,x]] \in I$.
      
      It follows that the ideal~$I$ contains the basis vector~$e$:
      If~$\gamma \neq 0$ then~$e \in I$ since~$[e,[e,x]] = -2 \gamma e$;
      if~$\gamma = 0$ but~$\beta \neq 0$ then~$e \in I$ because~$[e,x] = -2 \beta e$;
      and if~$\beta = \gamma = 0$ then it follows from~$x \neq 0$ that~$\alpha \neq 0$ and hence~$e \in I$ because~$x = \alpha e$.
      
      It now follows from the above relations that not only~$h = [e,f] \in I$ but also~$f = -[h,f]/2 \in I$.
      We have found altogether that the ideals~$I$ contains all three basis vectors, and thus~$I = \sllie_2(\kf)$
  \end{enumerate}
\end{examples}


\begin{definition}
 Let~$\kf$ be any field.
 The basis
 \[
    \gls*{standard basis e}
    =
    \begin{pmatrix}
      0 & 1 \\
      0 & 0
    \end{pmatrix} \,,
    \qquad
    \gls*{standard basis h}
    =
    \begin{pmatrix*}[r]
      1 &  0  \\
      0 & -1
    \end{pmatrix*}  \,,
    \qquad
    \gls*{standard basis f}
    =
    \begin{pmatrix}
      0 & 0 \\
      1 & 0
    \end{pmatrix} \,.
  \]
  of the Lie~algebra~$\sllie_2(\kf)$ is its \defemph{standard basis}\index{standard basis}.
\end{definition}


\begin{remark}
 If~$\ringchar \kf = 0$ then the Lie algebras~$\sllie_n(\kf)$ are simple for all~$n \geq 2$.
\end{remark}




