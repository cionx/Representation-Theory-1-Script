
\section{Computation in Small Degrees}


\begin{fluff}
	We will now compute Lie algebra cohomology and Lie algebra homology in certain degrees.
\end{fluff}



\subsection{Cohomology and Homology in Negative Degrees}
\index{Lie algebra homology!in negative degrees}
\index{Lie algebra cohomology!in negative degrees}

\begin{fluff}
	Let~$\glie$ be a Lie~algebra and let~$M$ be a representation of~$\glie$.
	The chain complex~$\Chain_\bullet(\glie, M)$ vanishes in negative degrees, and the same goes for the cochain complex~$\Chain^\bullet(\glie, M)$.
	It follows that both the Lie algebra homology~$\Homology_\bullet(\glie, M)$ and the Lie algebra cohomology~$\Homology^\bullet(\glie, M)$ vanish in negative degrees.
\end{fluff}



\subsection{Cohomology and Homology in Large Degrees}
\index{Lie algebra homology!in large degrees}
\index{Lie algebra cohomology!in large degrees}

\begin{fluff}
	Let~$\glie$ be a finite-dimensional Lie~algebra.
	Then~$\Exterior^n(\glie) = 0$ whenever~$n$ exceeds the dimension of~$\glie$.
	It follows for every representation~$M$ of~$\glie$ that
	\[
		\Chain_n(\glie, M)
		=
		\Exterior^n(\glie) \tensor M
		=
		0
	\]
	and
	\[
		\Chain^n(\glie, M)
		=
		\Alt^n(\glie, M)
		\cong
		\Hom_{\kf}\Bigl( \Exterior^n(\glie), M \Bigr)
		=
		0
	\]
	for all~$n > \dim(\glie)$, and thus
	\[
		\Homology_n(\glie, M) = 0
		\quad\text{and}\quad
		\Homology^n(\glie, M) = 0
	\]
	for all~$n > \dim(\glie)$.
\end{fluff}



\subsection{Zeroth Cohomology}
\index{zeroth Lie algebra cohomology}
\index{Lie algebra cohomology!zeroth}

\begin{fluff}
	\label{zeroth cohomology}
	Let~$\glie$ be a Lie~algebra and let~$M$ be a representation of~$\glie$.
	Then
	\[
		\Chain^0(\glie, M)
		=
		\Hom\Bigl( \Exterior^0(\glie), M \Bigr)
		\cong
		\Hom_{\kf}( \kf, M )
		\cong
		M
	\]
	and
	\[
		\Chain^1(\glie, M)
		=
		\Hom\Bigl( \Exterior^1(\glie), M \Bigr)
		\cong
		\Hom_{\kf}( \glie, M ) \,.
	\]
	The differential~$d^1$ is under these identifications given by the linear map
	\[
		M
		\to
		\Hom_{\kf}(\glie, M) \,,
		\quad
		m
		\mapsto
		( x \mapsto x \act m ) \,.
	\]
	It follows that under these identifications we have
	\[
		\Cycle^0(\glie, M)
		=
		\ker(d^1)
		\cong
		\{
			m \in M
		\suchthat
			x \cdot m
		\}
		=
		M^{\glie} \,.
	\]
	We also have~$\Boundary^0(\glie, M) = 0$ (because~$\Chain^{-1}(\glie, M) = 0$) and thus
	\[
		\Homology^0(\glie, M)
		=
		\Cycle^0(\glie, M) / { \Boundary^0(\glie, M) }
		=
		\Cycle^0(\glie, M) / 0
		\cong
		\Cycle^0(\glie, M)
		\cong
		M^{\glie} \,.
	\]
	This isomorphism is moreover natural in~$M$, whence
	\[
		\Homology^0(\glie, \ph)
		\cong
		(\ph)^{\glie} \,.
		\index{invariants}
	\]
\end{fluff}



\subsection{Zeroth Homology}
\index{zeroth Lie algebra homology}
\index{Lie algebra homology!zeroth}

\begin{definition}
	Let~$\glie$ be a Lie~algebra and let~$M$ be a representation of~$\glie$.
	The quotient representation
	\[
		M / \glie M
	\]
	is the representation of~\defemph{coinvariants}\index{coinvariants} of~$M$.
	It is denoted by~$M_{\glie}$\glsadd{coinvariants}.
\end{definition}


\begin{proposition}[Functoriality of Coinvariants]
	\label{functoriality of coinvariants}
	Let~$\glie$ be a Lie~algebra and let~$M$ and~$N$ be two representations of~$\glie$.
	Let~$f$ be a homomorphism of representations from~$M$ to~$N$.
	Then~$f$ induces a homomorphism of representations
	\[
		M_{\glie}
		\to
		N_{\glie} \,,
		\quad
		\class{m}
		\mapsto
		\class{f(m)} \,.
	\]
\end{proposition}


\begin{proof}
	This holds because the homomorphisms~$f$ maps the subrepresentation~$\glie M$ of~$M$ into the subrepresentation~$\glie N$ of~$N$.
\end{proof}


\begin{remark}
	Let~$\glie$ be a Lie~algebra.
	\begin{enumerate}
		\item
			The representation~$M_{\glie}$ is the largest quotient representation of~$M$ on which~$\glie$ acts trivially.
		\item
			We can extend our discussion from \cref{invariants are right adjoint} as follows.

			We have a functor~$(\ph)_{\glie}$ from~$\cRep{\glie}$ to~$\cReptriv{\glie}$\index{category!of trivial representations}.
			This functor assigns to each~\representation{$\glie$}~$M$ its space of coinvariants~$M_{\glie}$.
			To each homomorphism of representations~$f$ from~$M$ to~$N$ it assigns the induced homomorphism of representations from~$M_{\glie}$ and~$N_{\glie}$ as explained in \cref{functoriality of coinvariants}.

			If~$M$ is any~\representation{$\glie$} and~$N$ is any trivial~\representation{$\glie$}, then the quotient map~$p$ from~$M$ to~$M_{\glie}$ induces an isomorphism of vector spaces
			\[
				p^*
				\colon
				\Hom_{\glie}(M_{\glie}, N)
				\to
				\Hom_{\glie}(M, N) \,,
				\quad
				f
				\mapsto
				f \circ p \,.
			\]
			This isomorphism is natural in both~$M$ and~$N$.
			The functor~$(\ph)_{\glie}$ is therefore left adjoint\index{adjunction} to the inclusion functor~$T$ from~$\cReptriv{\glie}$ to~$\cRep{\glie}$.
			We have overall the adjunctions\index{adjunction}
			\[
				(\ph)_{\glie}
				\leftadjoint
				T
				\leftadjoint
				(\ph)^{\glie} \,.
			\]
		\item
			The category~$\cReptriv{\glie}$ is isomorphic to~$\cVect{\kf}$ via the restriction of the forgetful functor from~$\cRep{\glie}$ to~$\cVect{\kf}$.
			We will therefore often regard both~$(\ph)^{\glie}$ and~$(\ph)_{\glie}$ as functors from~$\cRep{\glie}$ to~$\cVect{\kf}$.
	\end{enumerate}
\end{remark}


\begin{fluff}
	Let~$\glie$ be a Lie~algebra and let~$M$ be a representation of~$\glie$.
	We have
	\[
		\Chain_0(\glie, M)
		=
		\Exterior^0(\glie) \tensor M
		\cong
		\kf \tensor M
		\cong
		M
	\]
	and
	\[
		\Chain_1(\glie, M)
		=
		\Exterior^1(\glie) \tensor M
		\cong
		\glie \tensor M \,.
	\]
	Under these isomorphisms, the differential~$d_1$ corresponds to the linear map
	\[
		\glie \tensor M
		\to
		M \,,
		\quad
		x \tensor m
		\mapsto
		- x \act m \,.
	\]
	It follows that
	\[
		\Boundary_0(\glie, M)
		=
		\im(d_1)
		\cong
		\glie M
	\]
	under the above identification of~$\Chain_0(\glie, M)$ with~$M$.
	We also have
	\[
		\Cycle_0(\glie, M)
		=
		\Chain_0(\glie, M)
	\]
	because~$d_{-1} = 0$.
	Under the above identification of~$\Chain_0(\glie, M)$ with~$M$ we hence find that~$\Cycle_0(\glie, M)$ corresponds to~$M$.
	It follows that
	\[
		\Homology_0(\glie, M)
		=
		\Cycle_0(\glie, M) / { \Boundary_0(\glie, M) }
		\cong
		M / \glie M
		=
		M_{\glie} \,.
		\index{coinvariants}
	\]
	This isomorphism is moreover natural in~$M$, whence
	\[
		\Homology_0(\glie, \ph)
		\cong
		(\ph)_{\glie} \,.
	\]
\end{fluff}



\subsection{First Cohomology}
\index{first Lie algebra cohomology}
\index{Lie algebra cohomology!first}

\begin{definition}
	Let~$\glie$ be a Lie~algebra and let~$M$ be a representation of~$\glie$.
	A map~$\delta$ from~$\glie$ to~$M$ is a \defemph{derivation}\index{derivation!of a representation} if it is linear and satisfies the condition
	\[
		\delta( [x,y] )
		=
		x \act \delta(y) - y \act \delta(x)
	\]
	for all~$x, y \in \glie$.
	The set of derivations of~$M$ is denoted by~$\Der(\glie, M)$\glsadd{derivations of module}.
\end{definition}


% TODO: Explain this in the general context of derivations of a module.


\begin{proposition}
	\label{inner derivations of representations}
	Let~$\glie$ be a Lie~algebra and let~$M$ be a representation of~$\glie$.
	The map
	\[
		\glie
		\to
		M \,,
		\quad
		x
		\mapsto
		x \act m
	\]
	is for every element~$m$ of~$M$ a derivation of~$M$.
\end{proposition}


\begin{proof}
	We denote this map by~$\delta$ and compute that
	\[
		\delta([x,y])
		=
		[x,y] \act m
		=
		x \act y \act m - y \act x \act m
		=
		x \act \delta(y) - y \act \delta(x)
	\]
	for all~$x, y \in \glie$.
\end{proof}


\begin{definition}
	Let~$\glie$ be a Lie~algebra and let~$M$ be a representation of~$\glie$.
	A derivation~$\delta$ of~$M$ is \defemph{inner}\index{inner derivation!of a representation} if there exists an element~$m$ of~$M$ with
	\[
		\delta(x) = x \act m
		\qquad
		\text{for all~$x \in \glie$.}
	\]
	The set of inner dervations of~$M$ is denoted by~$\InnDer(\glie, M)$\glsadd{inner derivations of module}.
\end{definition}


\begin{remark}
	Let~$\glie$ be a Lie~algebra.
	For the adjoint representation of~$\glie$ the above notions of derivations and inner derivations coincide with the notions from \cref{definition of derivations} and \cref{definition of inner derivations}.
\end{remark}


\begin{proposition}
	Let~$\glie$ be Lie~algebra and let~$M$ be a representation of~$\glie$.
	\begin{enumerate}
		\item
			The set~$\Der(\glie, M)$ is a linear subspace of~$\Hom_{\kf}(\glie, M)$.
		\item
			The set~$\InnDer(\glie, M)$ is a linear subspace of~$\Der(\glie, M)$.
		\qed
	\end{enumerate}
\end{proposition}


\begin{definition}
	Let~$\glie$ be a Lie~algebra and let~$M$ be a representation of~$\glie$.
	The quotient vector space~$\Der(\glie, M) / { \InnDer(\glie, M) }$ is the space of \defemph{outer derivations}\index{outer derivation!of a representation} of~$M$.
	It is denoted by~$\OutDer(\glie, M)$\glsadd{outer derivations of module}.
\end{definition}


\begin{proposition}
	\label{functiorality of outer derivations}
	Let~$\glie$ be a Lie~algebra and let~$M$ and~$N$ be two representations of~$\glie$.
	Let~$f$ be a homomorphism of representations from~$M$ to~$N$.
	\begin{enumerate}
		\item
			The linear map~$f_* \colon \Der(\glie, M) \to \Der(\glie, N)$ given by~$\delta \mapsto f \circ \delta$ is a well-defined.
		\item
			The map~$f_*$ maps the inner derivations of~$M$ to inner derivations of~$N$.
		\item
			The map~$f_*$ induces a linear map from~$\OutDer(\glie, M)$ to~$\OutDer(\glie, N)$ given by~$\class{\delta} \mapsto \class{f \circ \delta}$.
	\end{enumerate}
\end{proposition}


\begin{proof}
	\leavevmode
	\begin{enumerate}
		\item
			Let~$\delta$ be a derivation of~$M$.
			Then
			\begin{align*}
				(f \circ \delta)([x,y])
				&=
				f( \delta([x,y]) )
				\\
				&=
				f( x \act \delta(y) - y \act \delta(x) )
				\\
				&=
				f( x \act \delta(y) ) - f( y \act \delta(x) )
				\\
				&=
				x \act f( \delta(y) ) - y \act f( \delta(x) )
				\\
				&=
				x \act (f \circ \delta)(y) - y \act (f \circ \delta)(x)
			\end{align*}
			for all~$x, y \in \glie$.
			This shows that the composite~$f \circ \delta$ is again a derivation.
		\item
			For every element~$m$ of~$M$ let~$\delta_m$ denote the resulting inner derivation of~$M$.
			Then
			\[
				f_*(\delta_m)(x)
				=
				(f \circ \delta_m)(x)
				=
				f( \delta_m(x) )
				=
				f( x \act m )
				=
				x \act f(m)
				=
				\delta_{f(m)}(x)
			\]
			for all~$x \in \glie$ and thus~$f_*( \delta_m ) = \delta_{f(m)}$.
		\item
			The linear map~$f_*$ from~$\Der(\glie, M)$ to~$\Der(\glie, N)$ maps the linear subspace~$\InnDer(\glie, M) )$ of~$\Der(\glie, M)$ into the linear subspace~$\InnDer(\glie, N)$ of~$\Der(\glie, N)$.
			It thus induces a well-defined linear map from~$\Der(\glie, M) / {\InnDer(\glie, M)}$ to~$\Der(\glie, N) / {\InnDer(\glie, N)}$ which is given by~$\class{\delta} \mapsto \class{f_*(\delta)}$.
			This is the desired induced map.
		\qedhere
	\end{enumerate}
\end{proof}


\begin{remark}
	It follows with the help of~\cref{functiorality of outer derivations} that we have functors
	\[
		\Der(\glie, \ph),
		\,
		\InnDer(\glie, \ph),
		\,
		\OutDer(\glie, \ph)
		\colon
		\cRep{\glie}
		\to
		\cVect{\kf} \,.
	\]
\end{remark}


\begin{fluff}
	Let~$\glie$ be a Lie~algebra and let~$M$ be a representation of~$\glie$.
	We will now compute~$\Homology^1(\glie, M)$.
	
	As in \cref{zeroth cohomology}, we will identify~$\Chain^0(\glie, M)$ with~$M$ and~$\Chain^1(\glie, M)$ with~$\Hom_{\kf}(\glie, M)$.
	The differential~$d^0$ is under these identifications given by the linear map
	\[
		d^0
		\colon
		M
		\to
		\Hom_{\kf}(\glie, M) \,,
		\quad
		d^1(m)(x)
		=
		x \act m \,.
	\]
	The differential~$d^1$ is under these identifications given by the linear map
	\[
		d^1
		\colon
		\Hom_{\kf}(\glie, M)
		\to
		\Alt^2(\glie, M) \,,
	\]
	given by
	\[
		d^1(\varphi)(x_1, x_2)
		=
		- \varphi( [x_1, x_2] )
		- x_1 \act \varphi(x_2)
		+ x_2 \act \varphi(x_1)
	\]
	for all~$\varphi \in \Hom_{\kf}(\glie, M)$ and~$x_1, x_2 \in \glie$.

	We see from these explicit descriptions that an element~$\delta$ of~$\Hom_{\kf}(\glie, M)$ is a~\cocycle{$1$} if and only if is satisfies the identity
	\[
		\delta( [x_1, x_2] )
		=
		x_1 \act \delta(x_2) - x_2 \act \delta(x_1) \,,
	\]
	for all~$x_1, x_2 \in \glie$, i.e. if and only if~$\delta$ is a derivation.
	We also see that an element~$\delta$ of~$\Hom_{\kf}(\glie, M)$ is a~\coboundary{$1$} if and only if there exists an element~$m$ of~$M$ such that
	\[
		\delta(x)
		=
		x \act m
	\]
	for all~$x \in \glie$, i.e. if and only if~$\delta$ is an inner derivation.
	
	We see from these computations that
	\[
		\Homology^1(\glie, M)
		=
		\Cycle^1(\glie, M) / {\Boundary^1(\glie, M)}
		\cong
		\Der(\glie, M) / {\InnDer(\glie, M)}
		=
		\OutDer(\glie, M) \,.
	\]
	The above isomorphism~$\Chain^1(\glie, M) \cong \Hom_{\kf}(\glie, M)$ is natural in~$M$, and so the induced isomorphism~$\Homology^1(\glie, M) \cong \OutDer(\glie, M)$ is again natural in~$M$.
	We have thus shown that
	\[
		\Homology^1(\glie, \ph)
		\cong
		\OutDer(\glie, \ph) \,.
	\]
\end{fluff}


\begin{example}
	Let~$\glie$ be a Lie~algebra and let~$M$ be a trivial representation of~$\glie$.
	A linear map~$\delta$ from~$\glie$ to~$M$ is a derivation if and only if it vanishes on the commutator~$[\glie, \glie]$.
	The only inner derivation of~$M$ is the zero map.
	It follows that
	\[
		\OutDer(\glie, M)
		\cong
		\Der(\glie, M)
		\cong
		\Hom_{\kf}( \glie / [\glie, \glie], M )
		=
		\Hom_{\kf}( \glie^{\ab},  M ) \,.
	\]
\end{example}



\subsection{First Homology}
\index{first Lie algebra homology}
\index{Lie algebra homology!first}

\begin{fluff}
	Let~$\glie$ be a Lie~algebra.
	We will compute~$\Homology_1(\glie, \kf)$, where we denote by~$\kf$ the trivial representation of~$\glie$.

	We have
	\[
		\Chain_n(\glie, \kf)
		=
		\Exterior^n(\glie) \tensor \kf
		\cong
		\Exterior^n(\glie)
	\]
	for every natural number~$n$.
	Under these identifications the differential~$d_n$ for~$n \geq 1$ is given by
	\[
		d_n(x_1 \wedge \dotsb \wedge x_n)
		=
		\sum_{1 \leq i < j \leq n}
		(-1)^{i+j}
		[x_i, x_j] \wedge x_1 \wedge \dotsb \wedge \widehat{x_i} \wedge \dotsb \wedge \widehat{x_j} \wedge \dotsb \wedge x_n
	\]
	for all~$x_1, \dotsc, x_n \in \glie$.

	For the computation of~$\Homology_1(\glie, \kf)$ we also identify~$\Exterior^0(\glie)$ with~$\kf$ and identify~$\Exterior^1(\glie)$ with~$\glie$.
	We have thus identified the chain complex~$\Chain_\bullet(\glie, \kf)$ with a chain complex of the form
	\[
		\dotsb
		\to
		\Exterior^2(\glie)
		\xto{- [\ph, \ph]}
		\glie
		\xto{0}
		\kf
		\to
		0
		\to
		0
		\to
		\dotsb \,,
	\]
	where the term~$\glie$ sits in degree~$1$.
	We now see that
	\[
		\Homology_1(\glie, \kf)
		=
		\Cycle_1(\glie, \kf) / { \Boundary_1(\glie, \kf) }
		\cong
		\glie / [\glie, \glie]
		=
		\glie^{\ab} \,.
		\index{abelianization}
	\]
\end{fluff}


\begin{recall}
	Let~$Y$ be a vector space.
	To any chain complex
	\[
		\dotsb
		\to
		X_{n+1}
		\xto{d_{n+1}}
		X_n
		\xto{d_n}
		X_{n-1}
		\to
		\dotsb
	\]
	we can apply the functor~$(\ph) \tensor Y$ to get a new chain complex
	\[
		\dotsb
		\to
		X_{n+1} \tensor Y
		\xto{d_{n+1} \tensor \id}
		X_n \tensor Y
		\xto{d_n \tensor \id}
		X_{n-1} \tensor Y
		\to
		\dotsb
	\]
	This chain complex is denoted by~$X_\bullet \tensor Y$\glsadd{tensor product of chain complex and module}.
	It holds that
	\[
		\Boundary_n( X_\bullet \tensor Y )
		=
		\im( d_{n+1} \tensor \id_Y )
		=
		\im( d_{n+1} ) \tensor \im( \id_Y )
		=
		\Boundary_n( X_\bullet ) \tensor Y
	\]
	for every integer~$n$, as well as
	\[
		\Cycle_n( X_\bullet \tensor Y )
		=
		\ker( d_n \tensor \id_Y )
		=
		\ker( d_n ) \tensor Y + X_n \tensor \ker( \id_Y )
		=
		\Cycle_n( X_\bullet ) \tensor Y
	\]
	for every integer~$n$.
	It follows that
	\begin{align*}
		\Homology_n( X_\bullet \tensor Y )
		&=
		\Cycle_n( X_\bullet \tensor Y) / { \Boundary_n( X_\bullet \tensor Y ) }
		\\
		&\cong
		\Cycle_n( X_\bullet ) \tensor Y / { \Boundary_n( X_\bullet) \tensor Y }
		\\
		&\cong
		( \Cycle_n( X_\bullet ) / { \Boundary_n( X_\bullet ) } ) \tensor Y
		\\
		&=
		\Homology_n( X_\bullet ) \tensor Y
	\end{align*}
	for every integer~$n$.
\end{recall}


\begin{example}
	Let~$\glie$ be a Lie~algebra and let~$M$ be a representation of~$\glie$.
	Let~$N$ be a trivial representation of~$\glie$.
	Then
	\[
		\Chain_\bullet(\glie, M \tensor N)
		\cong
		\Chain_\bullet(\glie, M) \tensor N
	\]
	and therefore
	\begin{align*}
		\SwapAboveDisplaySkip
		\Homology_n(\glie, M \tensor N)
		&=
		\Homology_n( \Chain_\bullet(\glie, M \tensor N) )
		\\
		&\cong
		\Homology_n( \Chain_\bullet(\glie, M) \tensor N )
		\\
		&\cong
		\Homology_n( \Chain_\bullet(\glie, M) ) \tensor N
		\\
		&\cong
		\Homology_n(\glie, M) \tensor N \,.
	\end{align*}
\end{example}


\begin{fluff}
	Let~$\glie$ be a Lie~algebra and let~$M$ be a trivial representation of~$\glie$.
	Then
	\[
		\Homology_1(\glie, M)
		\cong
		\Homology_1(\glie, \kf \tensor M)
		\cong
		\Homology_1(\glie, \kf) \tensor M
		\cong
		\glie^{\ab} \tensor M \,.
	\]
\end{fluff}





\subsection{Second Cohomology}
\index{second Lie algebra cohomology}
\index{Lie algebra cohomology!second}

%\begin{fluff}
%  Let~$\hlie$ be a Lie~algebra and let~$I$ be an abelian Lie~algebra.
%  We may equivalently regard~$I$ simply as a vector space, or equivalently as a trivial representation of~$\hlie$.
%
%  An element~$\kappa$ of~$\Chain^2(\hlie, I) = \Alt^2(\hlie, I)$ is a cocycle if and only if it satisfies the identity
%  \[
%    \kappa([x_1, x_2], x_3) - \kappa([x_1, x_3], x_2) + \kappa(x_1, [x_2, x_3])
%    =
%    0
%  \]
%  for all~$x_1, x_2, x_3 \in \hlie$.
%  We have seen in \cref{structure of central extensions} that this means that~$\kappa$ defines a central extension of~$\hlie$ by~$I$.
%
%  A two-coycle~$\omega$ is a coboundary if and only if there exists an element~$\varphi$ of~$\Alt^1(\hlie, I)$, i.e. of~$\Hom_{\kf}(\hlie, I)$, such that
%  \[
%    \omega(x_1, x_2)
%    =
%    -\varphi([x_1, x_2])
%  \]
%  for all~$x_1, x_2 \in \hlie$.
%  It follows from \cref{structure of central extensions} that two two-coycles~$\kappa_1$ und~$\kappa_2$ define equivalent central extensions if and only if their difference~$\kappa_1 - \kappa_2$ is a two-coboundary.
%  
%  We find overall that~$\Homology^2(\hlie, I)$ is in {\onetoonetext} correspondence with the equivalence classes of central extensions of~$\hlie$ by~$I$.
%\end{fluff}


\begin{fluff}
	Let~$\glie$ be a Lie~algebra and let~$M$ be a representation of~$\glie$.
	Let~$\AbEx(\glie, M)$ be the corresponding class of abelian extensions of~$\glie$ as in \cref{structure of abelian extensions}.

	An element~$\kappa$ of~$\Chain^2(\glie, M) = \Alt^2(\glie, M)$ is a cocycle if and only if it satisfies the identity
	\begin{align*}
		0
		={}&
		- \kappa([x_1, x_2], x_3)
		+ \kappa([x_1, x_3], x_2)
		- \kappa([x_2, x_3], x_1)
		\\
		{}&
		+ x_1 \act \kappa(x_2, x_3)
		- x_2 \act \kappa(x_1, x_3)
		+ x_3 \act \kappa(x_1, x_2)
	\end{align*}
	for all~$x_1, x_2, x_3 \in \glie$, or equivalently
	\begin{align*}
		{}&
		\kappa([x_1, x_2], x_3)
		+ \kappa([x_2, x_3], x_1)
		+ \kappa([x_3, x_1], x_2)
		\\
		={}&
		x_1 \act \kappa(x_2, x_3)
		+ x_2 \act \kappa(x_3, x_1)
		+ x_3 \act \kappa(x_1, x_2)
	\end{align*}
	for all~$x_1, x_2, x_3 \in \glie$.
	It follows from \cref{structure of abelian extensions} that every such~\cocycle{$2$}~$\kappa$ defines an abelian extension\index{abelian extension of Lie algebras}\index{extension of Lie algebras!abelian} of~$\glie$ contained in~$\AbEx(\glie, M)$, and that every extension in~$\AbEx(\glie, M)$ is of this form up to equivalence.
	For a two-coycle~$\kappa$ the associated abelian extension is more explicitely given by the vector space~$\glie \oplus M$ together with the Lie bracket given by
	\[
		[(x, m), (y, n)]
		=
		( [x,y], x \act n - y \act m + \kappa(x,y) )
	\]
	for all~$(x,m), (y,n) \in \glie \oplus M$.

	A~\cocycle{$2$}~$\omega$ is a coboundary if and only if there exists an element~$\varphi$ of~$\Alt^1(\glie, M)$, i.e a linear map~$\varphi$ from~$\glie$ to~$M$, such that
	\[
		\omega(x_1, x_2)
		=
		x_1 \act \varphi(x_2)
		- x_2 \act \varphi(x_1)
		- \varphi( [x_1, x_2] )
	\]
	for all~$x_1, x_2 \in \glie$.
	It follows from \cref{structure of abelian extensions} that two~\cocycles{$2$}~$\kappa_1$ and~$\kappa_2$ define equivalent abelian extensions if and only if their difference~$\kappa_1 - \kappa_2$ is a~\coboundary{$2$}.

	We have overall constructed a {\onetoonetext} correspondence between~$\Homology^2(\glie, M)$ and the set of equivalence classes of extensionse in~$\AbEx(\glie, M)$\glsadd{abelian extensions associated to M}.

	We have also seen in \cref{each class of abelian extensions contains precisely one split extension} that the zero element of~$\Homology^2(\glie, M)$ corresponds to the unique equivalence classes of split extensions in~$\AbEx(\glie, M)$.
\end{fluff}


\begin{fluff}
	As a special case of the above discussion let~$\glie$ be a Lie~algebra and let~$M$ be an abelian Lie~algebra.
	We may regard~$M$ as a trivial representation of~$\glie$.
	We then have a {\onetoonetext} correspondence between~$\Homology^2(\glie, M)$ and the set of equivalence classes of central extensions of~$\glie$ by~$M$.
\end{fluff}





