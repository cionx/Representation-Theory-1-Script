\section{Nilpotent and Solvable Lie~Algebras}





\subsection{Definition, Examples and Properties}


\begin{definition}
  \leavevmode
  \begin{enumerate}
    \item
      An element~$a$ of an associative~{\algebra{$\kf$}}~$A$ is \defemph{nilpotent}\index{nilpotent} if~$a^n = 0$ for some~$n \geq 0$.
    \item
      An element~$x$ of a Lie~algebra~$\glie$ is~\defemph{\adnilpotent} if the endomorphism~$\ad(x) \in \End_{\kf}(\glie)$ is nilpotent.
  \end{enumerate}
\end{definition}


\begin{lemma}
  \label{nilpotent implies ad-nilpotent}
  If~$A$ is an associative~{\algebra{$\kf$}} and~$x \in A$ is nilpotent then~$x$ is also {\adnilpotent}.
\end{lemma}

\begin{proof}
  Let~$\lambda_x \colon A \to A$,~$a \mapsto xa$ denote the left multiplication by~$x$ and~$\rho_x \colon A \to A$,~$a \mapsto ax$ the right multiplication by~$x$.
  Both~$\lambda_x$ and~$\rho_x$ are nilpotent because~$x$ is nilpotent.
  The endomorphisms~$\lambda_x$ and~$\rho_x$ commute because~$A$ is associative.
  Hence~$\ad(x) = \lambda_x - \rho_x$ is the sum (or rather difference) of two commuting nilpotent endomorphisms, and therefore again nilpotent.
\end{proof}


\begin{definition}
  Let~$\glie$ be a Lie~algebra.
  \begin{enumerate}
    \item
      Let~$\glie^0 \defined \glie$ and inductively~$\glie^{i+1} \defined [\glie, \gls*{central series}]$ for every~$i \geq 0$.
      The decreasing sequence
      \[
        \glie
        =
        \glie^0
        \supseteq
        \glie^1
        \supseteq
        \glie^2
        \supseteq
        \glie^3
        \supseteq
        \dotsb
      \]
      is the \defemph{central series}\index{central series}\index{series!central} of~$\glie$.
      The Lie~algebra~$\glie$ is \defemph{nilpotent}\index{nilpotent!Lie algebra}\index{Lie algebra!nilpotent} if~$\glie^n = 0$ for~$n$ sufficiently large.
    \item
      Let~$\glie^{(0)} \defined \glie$ and~$\glie^{(i+1)} \defined [\glie^{(i)}, \gls*{derived series}]$ for every~$i \geq 0$.
      The decreasing sequence
      \[
        \glie
        =
        \glie^{(0)}
        \supseteq
        \glie^{(1)}
        \supseteq
        \glie^{(2)}
        \supseteq
        \glie^{(3)}
        \supseteq
        \dotsb
      \]
      is the \defemph{derived series}\index{derived series}\index{series!derived} of~$\glie$.
      The Lie~algebra~$\glie$ is \defemph{solvable}\index{solvable}\index{Lie algebra!solvable} if~$\glie^{(n)} = 0$ for~$n$ sufficiently large.
  \end{enumerate}
\end{definition}


\begin{definition}
  If~$\glie$ is a Lie~algebra then~$\glie^{(1)} = g^1 = [\glie, \glie]$ is the \defemph{derived Lie~algebra}\index{derived Lie algebra}\index{Lie algebra!derived} of~$\glie$.
\end{definition}


\begin{remark}
  Let~$\glie$ be a Lie~algebra.
  \begin{enumerate}
    \item
      Both the central series~$(\glie^i)_{i \geq 0}$ and the derived series~$(\glie^{(i)})_{i \geq 0}$ consist of ideals of~$\glie$.
    \item
      If~$\glie$ is nilpotent then it is also solvable because~$\glie^{(i)} \subseteq \glie^i$ for every~$i \geq 0$.
    \item
      Our indexing convention is choosen so that~$(\glie^{(i)})^{(j)} = \glie^{(i+j)}$ for all~$i, j \geq 0$.
    \item
      It follows that~$\glie$ is solvable if and only if~$\glie^{(i)}$ is solvable for some~$i \geq 0$, if and only if~$\glie^{(i)}$ is solvable for every~$i \geq 0$.
%     \item
%       One can define more generally for every ideal~$I$ in~$\glie$ the central series series~$(I^i)_{i \geq 0}$ of~$I$ in~$\glie$ by~$I^0 \defined \glie$ and~$I^{i+1} \defined [\glie, I^i]$ for every~$i \geq 0$.
%       Then
%       \[
%         \glie
%         =
%         I^0
%         \supseteq
%         I^1
%         \supseteq
%         I^2
%         \supseteq
%         I^3
%         \supseteq
%         \dotsb
%       \]
%       is a decreasing sequence of ideals in~$\glie$.
%       Note that the ideal~$I$ itself does not have to appear in this sequence;
%       the term~$I^1$ is contained in~$I$ (as are all following terms) but does not have to coincide with~$I$.
%       (Suppose for example that~$\glie$ is nonabelian and that~$I$ is nonzero but contained in the center~$\centerlie(\glie)$.
%       Then~$I \neq \glie = I^0$ but already~$I^1 = [I, \glie] = 0$.)
  \end{enumerate}
\end{remark}


\begin{examples}
  \label{examples for solvable and nilpotent}
  \leavevmode
  \begin{enumerate}
    \item
      The Lie~algebra of upper triangular matrices~$\tlie_n(\kf)$ is solvable but not nilpotent.
    \item
      The Lie~algebra of strictly upper triangular matrices~$\nlie_n(\kf)$ is nilpotent, and hence also solvable.
    \item
      If~$n \geq 2$ then the special linear Lie~algebra~$\sllie_n(\Complex)$ is simple, and hence~$[\sllie_n(\Complex), \sllie_n(\Complex)] = \sllie_n(\Complex)$.
      Thus~$\sllie_n(\Complex)$ is neither solvable nor nilpotent.
    \item
      The general linear Lie~algebra~$\gllie_n(\Complex)$ is not solvable because
      \[
        \gllie_n(\Complex)^{(1)}
        =
        [\gllie_n(\Complex), \gllie_n(\Complex)]
        =
        \sllie_n(\Complex)
      \]
      is not solvable.
    \item
      If~$\glie$ is abelian then~$\glie$ is nilpotent (with~$\glie^1 = 0$) and therefore also solvable.
    \item
      Every {\onedimensional} Lie~algebra is abelian, thus nilpotent and hence solvable.
      The same goes for the {\twodimensional} abelian Lie~algebra.
      The {\twodimensional} non-abelian Lie~algebra~$\glie$ admits a basis~$x$,~$y$ with~$[x,y] = x$.
      Then~$\glie^{(i)} = 0$ for every~$i \geq 2$ but~$\glie^i = \kf x$ for every~$i \geq 1$.
      The Lie~algebra~$\glie$ is thus solvable but not nilpotent.
    \item
      The \defemph{Heisenberg Lie~algebra}\index{Heisenberg Lie algebra}\index{Lie algebra!Heisenberg} consists of a real vector space with basis~$P_1, \dotsc, P_n, Q_1, \dotsc, Q_n, C$ together with the Lie~bracket satisfying the relations
      \begin{itemize}
        \item
          $[P_i, P_j] = 0$ and~$[Q_i, Q_j] = 0$ for all~$i, j = 1, \dotsc, n$,
        \item
          $[P_i, C] = [Q_i, C] = 0$ for all~$i = 1, \dotsc, n$,
        \item
          $[P_i, Q_j] = \delta_{ij} C$ for all~$i,j = 1, \dotsc, n$.
      \end{itemize}
      This defines a nilpotent Lie~algebra.
      
      As a faithful representation of the Heisenberg Lie~algebra (and thus a realization of the Heisenberg Lie~algebra via endomorphisms) we can consider the polynomial ring~$V = \Real[x_1, \dotsc, x_n]$.
      Let~$p_1, \dotsc, p_n, q_1, \dotsc, q_n, c \in \gllie(V)$ so that~$p_i = \partial_i$ is the~{\howmanyth{$i$}} partial derivate,~$q_i = \partial_i$ is the multiplication with~$x_i$ and~$c = \id_V$.
      These endomorphisms are linearly independent and satisfy the desired relations.
      It follows that~$\hlie = \vspan_\kf(p_1, \dotsc, p_n, q_1, \dotsc, q_n, c)$ is a Lie subalgebra of~$\gllie(V)$ that is isomorphic to the Heisenberg Lie~algebra.
  \end{enumerate}
\end{examples}


\begin{proposition}[Properites of solvability]
  \label{properties of solvable and nilpotent}
  Let~$\glie$ be a Lie~algebra.
  \begin{enumerate}
    \item
      If~$\hlie$ is another Lie~algebra and~$f \colon \glie \to \hlie$ is a homomorphism of Lie~algebras then
      \[
        f(\glie)^{(i)}
        =
        f(\glie^{(i)})
      \]
      for every~$i \geq 0$.
    \item
      If the Lie~algebra~$\glie$ is solvable then any Lie~subalgebra~$\hlie$ of~$\glie$ is again solvable.
    \item
      If~$\glie$ is solvable and~$I$ is an ideal in~$\glie$ then the quotient Lie~algebra~$\glie/I$ is again solvable.
    \item
      If~$I$ is an ideal in~$\glie$ such that both~$I$ and~$\glie/I$ is solvable then~$\glie$ is solvable.
  \end{enumerate}
\end{proposition}


\begin{proof}
  \leavevmode
  \begin{enumerate}
    \item
      This can be shown by induction because~$f([X,Y]) = [f(X), f(Y)]$ for all subsets~$X, Y \subseteq \glie$.
    \item
      The inclusion~$\hlie \inclusion \glie$ is a Lie algebra homomorphism, hence~$\hlie^{(i)} \subseteq \glie^{(i)}$ for every~$i \geq 0$.
      It thus follows from~$\glie^{(n)} = 0$ for~$n$ sufficiently large that also~$\hlie^{(n)} = 0$.
    \item
      The canonical projection~$\pi \colon \glie \to \glie/I$ is a Lie algebra homomorphism, and hence
      \[
        (\glie/I)^{(i)}
        =
        \pi(\glie)^{(i)}
        =
        \pi(\glie^{(i)})
      \]
      for every~$i \geq 0$.
      It thus follows from~$\glie^{(n)} = 0$ for~$n$ sufficietly large that also~$(\glie/I)^{(n)} = 0$.
    \item
      It follows from the solvability of~$\glie/I$ that~$(\glie/I)^{(n)} = 0$ for~$n$ sufficiently large.
      This means that
      \[
        0
        =
        (\glie/I)^{(n)}
        =
        \pi(\glie)^{(n)}
        =
        \pi(\glie^{(n)})  \,,
      \]
      so that~$\glie^{(n)} \subseteq I$.
      It follows from the solvability of~$I$ that~$I^{(m)} = 0$ for~$m$ sufficiently large.
      Then
      \[
        \glie^{(n+m)}
        =
        (\glie^{(n)})^{(m)}
        \subseteq
        I^{(m)}
        =
        0
      \]
      and hence~$\glie^{(n+m)} = 0$.
    \qedhere
  \end{enumerate}
\end{proof}


\begin{corollary}
  \label{solvable via ses}
  If~$I$ is an ideal in a Lie~algebra~$\glie$ then~$\glie$ is solvable if and only if both~$I$ and~$\glie/I$ are solvable
  Hence in a short exact sequence
  \[
    0
    \to
    I
    \to
    \glie
    \to
    \glie/I
    \to
    0
  \]
  the middle term is solvable if and only if the two outer terms are solvable.
  \qed
\end{corollary}


\begin{corollary}
  \label{sum of solvable ideals is solvable}
  Let~$\glie$ be a Lie~algebra and let~$I$ and~$J$ be two solvable ideals in~$\glie$.
  Then their sum~$I + J$ is again solvable.
\end{corollary}


\begin{proof}
  We consider the short exact sequence
  \[
    0
    \to
    I
    \to
    I+J
    \to
    (I+J)/I
    \to
    0 \,.
  \]
  The terms~$I$ and~$(I+J)/I \cong J/(I \cap J)$ are both solvable (the second one because~$J$ is solvable) and hence the middle term~$I+J$ is solvable.
\end{proof}


\begin{definition}
 Let~$\glie$ be a finite dimensional Lie~algebra.
 It follows from \cref{sum of solvable ideals is solvable} that~$\glie$ contains a unique maximal solvable ideal. This maximal solvable ideal is the \defemph{radical}\index{radical!of a Lie algebra} of~$\glie$ and is denoted by~\gls*{radical}.
\end{definition}


\begin{proposition}[Properties of nilpotence]
  \label{properties of nilpotence}
  Let~$\glie$ be a Lie~algebra.
  \begin{enumerate}
    \item
      If~$\hlie$ is another Lie~algebra and~$f \colon \glie \to \hlie$ is a homomorphism of Lie~algebras then
      \[
        f(\glie)^i
        =
        f(\glie^i)
      \]
      for every~$i \geq 0$.
    \item
      If the Lie~algebra~$\glie$ is nilpotent then any Lie~subalgebra~$\hlie$ of~$\glie$ is again nilpotent.
    \item
      If~$\glie$ is nilpotent and~$I$ is an ideal in~$\glie$ then the quotient Lie~algebra~$\glie/I$ is again nilpotent.
    \item
      If~$I$ is an ideal in~$\glie$ such that~$I$ is contained in the center~$\centerlie(\glie)$ and such that~$\glie/I$ is nilpotent then already~$\glie$ is nilpotent.
    \item
      If~$\glie$ is nonzero and nilpotent then~$\centerlie(\glie)$ is nonzero.
    \item
      If~$\glie$ is nilpotent then every~$x \in \glie$ is~{\adnilpotent}.
  \end{enumerate}
\end{proposition}


\begin{proof}
  \leavevmode
  \begin{enumerate}
    \item
      This can be shown inductively because~$f([X,Y]) = [f(X), f(Y)]$ for any two subsets~$X, Y \subseteq \glie$.
    \item
      The inclusion~$\hlie \to \glie$ is a homomorphism of Lie~algebras, hence~$\hlie^i \subseteq \glie^i$ for all~$i \geq 0$.
      It thus follows for~$n$ sufficiently large from~$\glie^i = 0$ that also~$\hlie^i = 0$.
    \item
      The canonical projection~$\pi \colon \glie \to \glie/I$ is a homorphism of Lie~algebras, hence
      \[
        (\glie/I)^i
        =
        \pi(\glie)^i
        =
        \pi(\glie^i)
      \]
      for every~$i \geq 0$.
      It thus follows for~$n$ sufficiently large from~$\glie^n = 0$ that also~$(\glie/I)^n = 0$.
    \item
      For~$n$ sufficiently large we have that~$(\glie/I)^n = 0$ and hence
      \[
        0
        =
        (\glie/I)^n
        =
        \pi(\glie)^n
        =
        \pi(\glie^n)  \,.
      \]
      This means that~$\glie^n \subseteq I \subseteq \centerlie(\glie)$.
      Then
      \[
        \glie^{n+1}
        =
        [\glie, \glie^n]
        \subseteq
        [\glie, \centerlie(\glie)]
        =
        0
      \]
      and hence~$\glie^{n+1} = 0$.
    \item
      Let~$n \geq 0$ be minimal with~$\glie^n = 0$.
      That~$\glie$ is nonzero means that~$n \geq 1$.
      Then~$\glie^{n-1} \neq 0$, but also
      \[
        [\glie, \glie^{n-1}]
        =
        \glie^n
        =
        0
      \]
      and hence~$\glie^{n-1} \subseteq \centerlie(\glie)$.
      It follows that~$\centerlie(\glie) \neq 0$.
    \item
      Let~$x \in \glie$.
      We find by induction that
      \[
        \ad(x)^n(\glie)
        \subseteq
        \glie^n
      \]
      for all~$n \geq 0$.
      It hence follows for~$n$ sufficiently large from~$\glie^n = 0$ that~$\ad(x)^n = 0$.
    \qedhere
  \end{enumerate}
\end{proof}


\begin{remark}
  The analogeous statement for \cref{solvable via ses} about nilpotency does not necessarily hold.
  Take for example the {\twodimensional} non-abelian Lie~algebra~$\glie$, that has a basis~$x$,~$y$ with~$[x,y] = x$.
  Then the {\onedimensional} linear subspace~$I \defined \kf x$ is an abelian ideal in~$\glie$ and in particular nilpotent.
  The quotient~$\glie/I$ is one-dimensional and therefore also nilpotent.
  But~$\glie$ itself is not nilpotent as seen in \cref{examples for solvable and nilpotent}.
  
  But we have the following result:
\end{remark}


\begin{corollary}
  \label{nilpotent iff nilpotent moudulo center}
  A Lie algebra~$\glie$ is nilpotent if and only if its quotient~$\glie/\centerlie(\glie)$ is nilpotent.
  \qed
\end{corollary}


\begin{remark}
  Observe that the quotient~$\glie/\centerlie(\glie)$ can be identified with the Lie~subalgebra~$\ad(\glie)$ of~$\gllie(\glie)$.
  We can therefore use \cref{solvable via ses} and \cref{nilpotent iff nilpotent moudulo center} to reduce questions about solvability or nilpotence of Lie~algebras to the case of linear Lie~algebras (observe for the application of \cref{solvable via ses} that the ideal~$\centerlie(\glie)$ is abelian and hence solvable).
\end{remark}


\begin{remark}
  If~$\glie$ is a Lie~algebra and~$I$ and~$J$ are two nilpotent ideals in~$\glie$ then it can be shown that their sum~$I+J$ is again nilpotent.
  Indeed, we have for all~$n \geq 0$ that
  \[
    (I + J)^{2n}
    \subseteq
    I^n + J^n \,.
  \]
  To see this we write
  \begin{equation}
    \label{power of sum}
    \begin{aligned}
      (I + J)^{2n}
      &=
      [I+J, [I+J, [\dotsc, [I+J, I+J] \dotsc] ] ]
      \\
      &=
      \sum_{K_i \in \{I, J\}}
      [K_1, [K_2, [\dotsc, [K_{2n}, K_{2n+1}] \dotsc] ] ] \,.
    \end{aligned}
  \end{equation}
  In each summand~$[K_1, [K_2, [\dotsc, [K_{2n}, K_{2n+1}] \dotsc] ] ]$ at least~{\many{$(n+1)$}} ideals~$K_i$ are equal to the same ideal~$K \in \{I, J\}$ (the choice of which depends on the summand), say
  \[
    K_{i_1}
    =
    \dotsb
    =
    K_{i_{n+1}}
    =
    K
  \]
  with~$1_1 < i_2 < \dotsb < i_{n+1}$.
  We have that~$[L_1, L_2] \subseteq L_1, L_2$ for all ideals~$L_1, L_2 \subseteq \glie$, so we may leave out all terms~$K_i$ with~$i \notin \{i_1, \dotsc, i_{n+1}\}$ to get
  \begin{align*}
    {}&
    [K_1, [K_2, [\dotsc, [K_{2n}, K_{2n+1}] \dotsc] ] ]
    \\
    \subseteq{}&
    [K_{i_1}, [K_{i_n}, [\dotsc, [K_{i_n}, K_{i_{n+1}}] \dotsc] ] ]
    \\
    ={}&
    [K, [K, [\dotsc, [K, K] \dotsc] ] ]
    \\
    ={}&
    K^n
    \\
    \in{}&
    \{I^n, J^n\}  \,.
  \end{align*}
  This holds for every summand in~\eqref{power of sum}, so overall~$(I + J)^{2n} \subseteq I^n + J^n$ .
  If the ideals~$I$ and~$J$ are both nilpotent then~$I^n = J^n = 0$ for~$n$ sufficiently large, and hence~$(I + J)^{2n} = 0$.
%   WARNING: The following uses the wrong convention.
%   \begin{enumerate}
%     \item
%       For every~$n \geq 0$ we have that
%       \[
%         (I + J)^n
%         \subseteq
%         \sum_{k=0}^n I^k \cap J^{n-k}
%       \]
%       when we use the convention~$I^0 = J^0 = \glie$ (instead of~$I^0 = I$ and~$J^0 = J$).
%       This holds for~$n = 0$ because
%       \[
%         (I + J)^0
%         =
%         \glie
%         =
%         \glie \cap \glie
%         =
%         I^0 \cap J^0
%         =
%         \sum_{k=0}^0 I^k \cap J^{n-k} \,.
%       \]
%       It then follows inductively that
%       \begin{align*}
%         (I + J)^{n+1}
%         &=
%         [I + J, (I + J)^n]
%         \\
%         &\subseteq
%         \left[ I + J, \sum_{i=0}^n I^k \cap J^{n-k} \right]
%         \\
%         &=
%         \sum_{k=0}^n [I + J, I^k \cap J^{n-k}]
%         \\
%         &=
%         \sum_{k=0}^n \biggl( [I, I^k \cap J^{n-k}] + [J, I^k \cap J^{n-k}] \biggr)
%         \\
%         &\subseteq 
%         \sum_{k=0}^n \biggl( [I, I^k] \cap [I, J^{n-k}] \biggr) + \biggl( [J, I^k] \cap [J, J^{n-k}] \biggr)
%         \\
%         &=
%           \sum_{k=0}^n \biggl( [I, I^k] \cap [I, J^{n-k}] \biggr)
%         + \sum_{k=0}^n \biggl( [J, I^k] \cap [J, J^{n-k}] \biggr)
%         \\
%         &\subseteq
%           \sum_{k=0}^n \biggl( I^{k+1} \cap J^{n-k} \biggr)
%         + \sum_{k=0}^n \biggl( I^k \cap J^{n+1-k} \biggr)
%         \\
%         &\subseteq
%           \sum_{k=0}^{n+1} I^k \cap J^{n+1-k} \,.
%       \end{align*}
%     \item
%       It follows for all~$n, m \geq 0$ that
%       \[
%         (I + J)^{n+m}
%         \subseteq
%         \sum_{k=0}^n I^k \cap J^{n+m-k}
%         \subseteq
%         I^n + J^m
%       \]
%       because~$I^k \cap J^{n+m-k} \subseteq J^{n+m-k} \subseteq J^m$ whenever~$0 \leq k \leq n$ and~$I^k \cap J^{n+m-k} \subseteq I^k \subseteq I^n$ whenever~$n \leq k \leq n+m$.
%     \item
%       If the ideals~$I$ and~$J$ are both nilpotent with~$I^n = 0$ and~$J^m = 0$ then it therefore follows that~$(I + J)^{n+m} = 0$.
%   \end{enumerate}

  It follows that every finite dimensional Lie~algebra~$\glie$ admits a unique maxmial nilpotent ideal, the \defemph{nilradical}\index{nilradical} of~$\glie$.
\end{remark}





\subsection{Engel’s Theorem}


\begin{remark}
  If~$V$ is an~{\dimensional{$n$}} vector space over~$\kf$ and~$x \colon V \to V$ a nilpotent endomorphism then~$0$ is the only eigenvalue of~$x$ (and occurs with algebraic multiplicity~$n$).
  Hence there exists an eigenvector~$v \in V$ (which entails that~$v \neq 0$) with~$x(v) = 0$.
  The following proposition generalizes this observations to linear Lie~algebras consisting of nilpotent endomorphisms.
\end{remark}


\begin{proposition}
  \label{common eigenvector for nilpotent Lie algebras}
  Let~$V$ be a nonzero finite dimensional vector space and let~$\glie$ be a Lie~subalgebra of~$\gllie(V)$ consisting of nilpotent endomorphisms.
  Then there exists some nonzero~$v \in V$ with~$x(v) = 0$ for every~$x \in \glie$, i.e.~$v$ is a common eigenvector for all~$x \in \glie$ (all of which are nilpotent and thus have~$0$ as their only eigenvalue).
\end{proposition}


\begin{proof}
  We show the statement by induction over the dimension of~$\glie$.
  If~$\dim \glie = 0$ then~$\glie = 0$ and we may choose any nonzero~$v \in V$.
  If~$\dim \glie = 1$ then~$\glie = \kf x$ for any nonzero~$x \in \glie$ and we may choose any nonzero~$v \in \ker x$ (which exists because~$x$ is a nilpotent endomorphism of~$V$).
  Let now~$\dim \glie \geq 2$ and suppose that the \lcnamecref{common eigenvector for nilpotent Lie algebras} holds for all strictly smaller dimensions.
  
  We now proceed in two steps:
  In the first step we show that the \lcnamecref{common eigenvector for nilpotent Lie algebras} holds if~$\glie$ admits a nonzero proper ideal~$I$.
  We do so by applying the induction hypothesis to~$I$ and~$\glie/I$ (or rather~$\glie'$, a further quotient of~$\glie/I$).
  In the second step we show that~$\glie$ does indeed admit such an ideal.
  We do so by showing that every maximal proper Lie~subalgebra~$I$ of~$\glie$ is already an ideal, which we do by showing that~$\normallie_{\glie}(I) = \glie$ because~$I$ acts nilpotent on the quotient vector space~$\glie/I$.
  
  If~$\hlie$ is a proper Lie~subalgebra of~$\glie$ then the linear subspace
  \[
    W_{\hlie}
    \defined
    \{
      v \in V
    \suchthat
      \text{$x.v = 0$ for every~$x \in \hlie$}
    \}
  \]
  is by induction hypothesis nonzero.
  We observe that if~$I$ is an ideal in~$\glie$ then~$W_I$ is already a {\subrepresentation{$\glie$}} of~$V$:
  Indeed, we have for all~$x \in \glie$ and~$w \in W_I$ that
  \[
    z.(x.w)
    =
    x.(z.w) - [x,z].w
  \]
  for every~$z \in I$, with~$z.w = 0$ and~$[x,z].w = 0$ because~$z, [x,z] \in I$.
  
  Suppose now that~$\glie$ admits a nonzero proper ideal~$I$.
  Seen we find that~$W \defined W_I$ is a nonzero {\subrepresentation{$\glie$}} of~$V$.
  We hence get a homomorphism of Lie~algebras
  \[
    f
    \colon
    \glie
    \to
    \gllie(W) \,,
    \quad
    x
    \mapsto
    \restrict{x}{W} \,.
  \]
  The image~$\glie' \defined \im f = f(\glie)$ is a Lie~subalgebra of~$\gllie(W)$ that again consists of nilpotent endomorphisms.
  But the nonzero ideal~$I$ is contained in the kernel of~$f$, hence~$\glie'$ has strictly smaller dimension than~$\glie$.
  We hence find by induction hypothesis that there exists some nonzero~$w \in W$ with~$y.w = 0$ for every~$y \in \glie'$.
  By definition of~$\glie'$ this means that~$x.w = 0$ for every~$x \in \glie$, so we may choose~$v = w$.
  
  It remains to show that~$\glie$ admits a nonzero proper ideal.
  Let~$I$ be a proper Lie~subalgebra of~$\glie$ of maximal dimension.
  Then~$I$ is nonzero, because~$\glie$ contains nonzero proper Lie~subalgebras (e.g.~$\kf x$ with~$x \in \glie$, where we use that~$\dim \glie \geq 2$), and maximal among all Lie~subalgebras of~$\glie$ with respect to inclusion.
  We claim that~$I$ is already an ideal in~$\glie$, which is then the desired nonzero proper ideal.
  
  Indeed, for every~$x \in I$ the adjoint action~$\ad(x) = [x,-]$ maps~$I$ into itself, and hence descends to an endomorphism~$\induced{x} \colon \glie/I \to \glie/I$.
  We hence get a homomorphism of Lie~algebras
  \[
    g
    \colon
    I
    \to
    \gllie(\glie/I)  \,,
    \quad
    z
    \mapsto
    \induced{z} \,.
  \]
  The image of~$g$ is a Lie subalgebra of~$\gllie(\glie/I)$ of dimension strictly smaller than~$\glie$.
  So by induction hypothesis there exists some nonzero~$y \in \glie/I$ with~$\induced{z}.y = 0$ for every~$x \in I$.
  If~$x \in \glie$ is a representation of~$y$ then this means that~$x \notin I$ (as~$y$ is nonzero) but~$[z,x] = z.x \in I$ for every~$z \in I$ (because~$\induced{z}.y = 0$ for every~$z \in I$.
  Hence~$x$ is contained in the normalizer~$\normallie_{\glie}(I)$, but not contained in~$I$.
  This shows that~$I$ is properly contained in its normalizer.
  This normalizer is again a Lie~subalgebra of~$\glie$, and hence it follows from the maximality of~$\glie$ (among all proper Lie~subalgebras of~$\glie$) that~$\normallie_{\glie}(I) = \glie$.
  This means that~$I$ is an ideal in~$\glie$.
\end{proof}


\begin{recall}[Triangularization and invariant flags]
  \label{triangular recall}
  Let~$V$ be a finite dimensional vector space.
  \begin{itemize}
    \item
      A \defemph{flag}\index{flag} in~$V$ is an increasing sequence
      \[
        0
        =
        V_0
        \subsetneq
        V_1
        \subsetneq
        V_2
        \subsetneq
        \dotsb
        \subsetneq
        V_{n-1}
        \subsetneq
        V_n
        =
        V
      \]
      of linear subspaces~$V_i$ of~$V$.
      Note that by (our) convention every flag starts with~$0$ and ends with~$V$.
      If~$n = \dim(V)$ then the following conditions on a flag~$(V_i)_{i=0}^m$ are equivalent:
      \begin{equivalenceslist}
        \item
          $m = n$,
        \item
          $\dim V_i = i$ for every~$i = 0, \dotsc, m$,
        \item
          $(V_i)_{i=0}^m$ is a flag of maximal length,
        \item
          every quotient~$V_i/V_{i-1}$ with~$i = 1, \dotsc, m$ is {\onedimensional}.
      \end{equivalenceslist}
      The flag~$(V_i)_{i=0}^m$ is \defemph{complete}\index{complete flag}\index{flag!complete} if it satisfies these equivalent properties.
      
      Every basis~$v_1, \dotsc, v_n$ of~$V$ brings with it an \defemph{associated complete flag}~$(V_i)_{i=0}^n$\index{associated flag} of~$V$ that is given by~$V_i = \vspan_\kf(v_1, \dotsc, v_i)$ for every~$i = 0, \dotsc, n$.
      It follows by extending bases that every complete flag of~$V$ is associated to some basis of~$V$.
      
    \item
      Let~$f$ be an endomorphism of~$V$.
      A flag~$(V_i)_{i=0}^n$ is~\defemph{\invariant{$f$}}\index{invariant flag}\index{flag!invariant} if every term~$V_i$ is~{\invariant{$f$}} in the sense that~$f(V_i) \subseteq V_i$.
      
      The endomorphism~$f$ is represented by an upper triangular matrix with respect to a basis~$v_1, \dotsc, v_n$ of~$V$ if and only if the associated complete flag~$(V_i)_{i=0}^n$ is~{\invariant{$f$}}.
      It follows in particular that~$f$ is triangularizable if and only if~$V$ admits a complete~{\invariant{$f$}} flag.
      
      Observe that if the endomorphism~$f$ is nilpotent and and represented by an upper triangular matrix with respect to some basis of~$V$, then this representing matrix is nilpotent and must therefore already be strictly upper triangular.
      If~$(V_i)_{i=0}^n$ is a complete~{\invariant{$f$}} flag then it follows that already~$f(V_i) \subseteq V_{i-1}$ for every~$i \geq 1$.
    \item
      Let~$\mathcal{F}$ be any family of endomorphisms of~$V$.
      A flag~$(V_i)_{i=0}^n$ of~$V$ is~\defemph{\invariant{$\mathcal{F}$}}\index{invariant flag}\index{flag!invariant} if every term~$V_i$ is~{\invariant{$\mathcal{F}$}} in the sense that~$V_i$ is~{\invariant{$f$}} for every~$f \in \mathcal{F}$.
      Then for a basis~$v_1, \dotsc, v_n$ of~$V$ the associated complete flag~$(V_i)_{i=0}^n$ is~{\invariant{$\mathcal{F}$}} if and only if every~$f \in \mathcal{F}$ is given by an upper triangular matrix with respect to this basis.
      It follows in particular that the endomorphisms~$f \in \mathcal{F}$ are simultaneous triangularizable if and only if~$V$ admits a complete~{\invariant{$\mathcal{F}$}} flag.
  \end{itemize}
\end{recall}


\begin{lemma}[Triangularization lemma\footnote{The author is to blame for this \lcnamecref{triangularization lemma}.}]
  \label{triangularization lemma}
  Let~$\mathcal{V}$ be a class of finite dimensional representations of a Lie algebra~$\glie$ such that
  \begin{itemize}
    \item
      $\mathcal{V}$ is closed under taking quotients,
    \item
      every nonzero representation~$V \in \mathcal{V}$ admits a {\onedimensional} subrepresentation.
  \end{itemize}
  Then every representation~$V \in \mathcal{V}$ admits a complete flag of subrepresentations, and hence a basis with respect to which the action of every~$x \in \glie$ is given by an upper triangular matrix.
\end{lemma}


\begin{proof}
  We fix~$V \in \mathcal{V}$ and show the claim by induction of the dimension~$n$ of~$V$.
  For~$n = 0$ and~$n = 1$ any flag sufficies.
  If~$n \geq 2$ then~$V$ admits a {\onedimensional} subrepresentation~$U$.
  The quotient~$W \defined V/U$ is again contained in~$\mathcal{V}$ but of dimension~$n-1$ and hence admits a complete flag of subrepresentations
  \[
    0
    =
    W_0
    \subsetneq
    W_1
    \subsetneq
    W_2
    \subsetneq
    \dotsb
    \subsetneq
    W_{n-2}
    \subsetneq
    W_{n-1}
    =
    W \,.
  \]
  Let~$\pi \colon V \to V/U = W$ denote the canonical projection and for every~$i \geq 1$ let~$V_i \defined \pi^{-1}(W_{i-1})$.
  Note that~$U = V_1$, so that with~$V_0 \defined 0$ the sequence
  \[
    0
    =
    V_0
    \subsetneq
    V_1
    \subsetneq
    V_2
    \subsetneq
    \dotsb
    \subsetneq
    V_{n-1}
    \subsetneq
    V_n
    =
    V
  \]
  is a complete flag of subrepresentations for~$V$.
  This shows the first assertion.
  
  It follows from \cref{triangular recall} that any such complete flag of subrepresentations of~$V$ is associated to a basis for~$V$ with respect to which the action of every~$x \in \glie$ is given by an upper triangular matrix.
  This shows the second assertion.
\end{proof}


\begin{proposition}
  \label{characterizations of linear lie algebras consisting of nilpotent endomorphisms}
  Let~$V$ be a vector space of finite dimension~$n \defined \dim V$ and let~$\glie$ be a Lie~subalgebra of~$\gllie(V)$.
  Then the following conditions on~$\glie$ are equivalent:
  \begin{equivalenceslist}
    \item
      \label{engels g consists of nilpotent endomorphisms}
      The Lie~algebra~$\glie$ consists of nilpotent endomorphisms.
    \item
      \label{engels there exists a complete flag shifted by g}
      There exists a complete flag~$(V_i)_{i=0}^n$ of~$V$ such that~$x(V_i) \subseteq V_{i-1}$ for all~$x \in \glie$ and~$i = 1, \dotsc, n$.
    \item
      \label{engels represented by strictly upper triangular matrices}
      There exists a basis of~$V$ with respect to which every~$x \in \glie$ is represented by a strictly upper triangular matrix.
      In particular,~$\glie$ is isomorphic to a Lie~subalgebra of~$\nlie_n(\kf)$ for~$n = \dim(V)$.
  \end{equivalenceslist}
\end{proposition}


\begin{proof}
  \leavevmode
  \begin{implicationlist}
    \item[\ref*{engels g consists of nilpotent endomorphisms}~$\implies$~\ref*{engels represented by strictly upper triangular matrices}]
      The vector space~$V$ is the natural representation of the Lie~algebra~$\glie$.
      Let~$\mathcal{V}$ be the class of all quotient representations of~$V$ and observe that~$V \in \mathcal{V}$.
      Every~$x \in \glie$ acts nilpotently on~$V$ and hence also on every quotient of~$V$.
      We can therefore apply Engel’s theorem to the image of the associated Lie~algebra homomorphism~$\rho \colon \glie \to \gllie(W)$.
      It follows that for every nonzero~$W \in \mathcal{V}$ there exists some nonzero~$x \in W$ with~$x.w = 0$ for every~$x \in \glie$;
      then~$\vspan_\kf(v)$ is a {\onedimensional} subrepresentation of~$W$.
      
      This show that we can apply \cref{triangularization lemma} ot~$\mathcal{V}$.
      There hence exists a basis of~$V$ with respect to which the action of every~$x \in \glie$ is given by an upper triangular matrix.
      Every such action is nilpotent, and hence must already be represented by a strictly triangular matrix.
    \item[\ref*{engels represented by strictly upper triangular matrices}~$\implies$~\ref*{engels there exists a complete flag shifted by g}]
      This follows from \cref{triangular recall}.
    \item[\ref*{engels there exists a complete flag shifted by g}~$\implies$~\ref*{engels g consists of nilpotent endomorphisms}]
      It follows that~$x^n = 0$.
    \qedhere
  \end{implicationlist}
\end{proof}


\begin{corollary}
  \label{linear lie algebras consisting of nilpotent endomorphisms are nilpotent}
  If a Lie~subalgebra~$\glie$ of~$\gllie(V)$ for some finite dimensional vector space~$V$ consists of nilpotent endomorphisms then~$\glie$ is nilpotent.
\end{corollary}


\begin{proof}
  It follows from \cref{characterizations of linear lie algebras consisting of nilpotent endomorphisms} that~$\glie$ is for~$n = \dim V$ isomorphic to a Lie~subalgebra of the nilpotent Lie~algebra~$\nlie_n(\kf)$.
\end{proof}


\begin{theorem}[Engel]
  \index{Engel’s theorem}
  A finite dimensional Lie~algebra~$\glie$ is nilpotent if and only if all its elements are~{\adnilpotent}.
\end{theorem}


\begin{proof}
  If~$\glie$ is nilpotent then every~$x \in \glie$ is~{\adnilpotent} by \cref{properties of nilpotence}.
 
  Suppose on the other hand that~$\glie$ consists of~{\adnilpotent} elemenents.
  Then~$\glie' \defined \ad(\glie)$ is a Lie subalgebra of~$\gllie(\glie)$ consisting of nilpotent endomorphisms.
  According to \cref{characterizations of linear lie algebras consisting of nilpotent endomorphisms} there exists a basis of~$V$ with respect to which every~$x \in \glie'$ is given by a strictly upper triangular matrix.
  This shows that~$\glie'$ is isomorphic to a Lie~subalgebra of some~$\nlie_n(\kf)$ (for~$n = \dim \glie$) and hence nilpotent.
  
  This shows that~$\glie/{\centerlie(\glie)} \cong \glie'$ is nilpotent.
  It follows by \cref{characterizations of linear lie algebras consisting of nilpotent endomorphisms} that~$\glie$ is nilpotent.
\end{proof}


\begin{remark}
  Engel’s theorem gives an alternative proof of \cref{linear lie algebras consisting of nilpotent endomorphisms are nilpotent}:
  Every element of~$\glie$ is~{\adnilpotent} by \cref{nilpotent implies ad-nilpotent} and~$\glie$ is therefore nilpotent by Engel’s theorem.
\end{remark}


\begin{remark}
  It is not true that every nilpotent Lie~subalgebra~$\glie$ of~$\gllie(V)$, where~$V$ is a finite dimensional vector space, can be represented by strictly upper triangular matrices with respect to some suitable basis of~$V$.
  The Lie~algebra~$\glie$ does not even have to contain any nonzery nilpotent endomorphisms.
  An example for this is the {\onedimensional} Lie~subalgebra~$\glie \defined \kf \, {\id_V}$ of~$\gllie(V)$.
\end{remark}


\begin{corollary}
  \label{ideals in nilpotent does intersection center}
  If~$\glie$ is a finite dimensional nilpotent ideal then~$I \cap \centerlie(\glie) \neq 0$ for every nonzero ideal~$I$ in~$\glie$.
\end{corollary}


\begin{proof}
  The Lie~algebra~$\glie$ acts on itself via the adjoint representation and~$\ad(x)$ is nilpotent for every~$x \in \glie$ by Engel’s theorem.
  The ideal~$I$ is a subrepresentation of~$\glie$ and the restriction~$\restrict{\ad(x)}{I}$ is again nilpotent for every~$x \in \glie$.
  The image of the resulting Lie~algebra homomorphism
  \[
    \glie
    \to
    \gllie(I) \,,
    \quad
    x
    \mapsto
    \restrict{\ad(x)}{I}
  \]
  does therefore consists of nilpotent endomorphisms.
  By \cref{common eigenvector for nilpotent Lie algebras} there exists some non-zero~$y \in I$ with~$\restrict{\ad(x)}{I}(y) = 0$ for every~$y \in \glie$, i.e.~$[\glie, y] = 0$.
  Then~$y \in \centerlie(\glie) \cap I$.
\end{proof}


\begin{remark}
  We have in this section introduces various notions of nilpotence for Lie~algebras and their elements.
  Before we move on the solvable Lie~algebras we will give a short overview our results.
  For this we pose the following question:
  \begin{center}
    What does it mean for a Lie~algebra to be nilpotent?
  \end{center}
  We have in this section seen four possible answers to this question.
  
  If~$\glie$ is an abstract Lie~algebra (i.e.\ any vector space with a Lie~bracket satisfying certain axioms) then we have introduced two notions of nilpotence for~$\glie$:
  \begin{equivalenceslist}[label=N\arabic*)]
    \item
      The Lie~algebra~$\glie$ is nilpotent, in the sense that~$\glie^n = 0$ for~$n$ sufficiently large.
    \item
      Every element~$x \in \glie$ is {\adnilpotent}, in the sense that the endomorphism~$\ad(x)$ of~$\glie$ is nilpotent for every~$x \in \glie$.
  \end{equivalenceslist}
  The first definition can be thought of as a global definition of nilpotence, whereas the second definition is a local one.
  The global notion states that for~$n$ sufficently large the compositions~$\ad(x_1) \dotsm \ad(x_n)$ vanish for all~$x_1, \dotsc, x_n \in \glie$.
  The local notion states that for every~$x \in \glie$ there exists some~$n$, possibly depending on~$x$, such that~$\ad(x)^n = 0$.
  The global notion is therefore stonger (but not necessarily strictly stronger) than the local one.
  Engel’s theorem asserts that for a finite dimensional Lie~algebra these two concepts of nilpotence actually coincide.
  So there is actually only one notion of abstract nilpotence for finite dimensional Lie~algebras.
  
  If~$\glie$ is a linear Lie~algebra, say a Lie~subalgebra of~$\gllie(V)$ with~$V$ some finite dimensional vector space, then we have two additional concepts of nilpotence:
  \begin{equivalenceslist}[label=N\arabic*), start=3]
    \item
      Every~$x \in \glie$ is nilpotent as an endomorphism of~$V$.
    \item
      There exists a basis of~$V$ with respect to which every~$x \in \glie$ is given by a strictly upper triangular matrix.
  \end{equivalenceslist}
  Observe that the two notions do not depend on the Lie bracket of~$\glie$;
  but they depend on the multiplication in~$\End_\kf(V)$, which in turn determines the commutator of~$\glie$.
  The first of these two notions can again be thought of as a local one, and the second notion as a global one.
  Indeed, we know from linear algebra that an endomorphism~$x$ of~$V$ is nilpotent if and only if there exists a basis of~$V$ with respect to which~$x$ is given by a strictly upper triangular matrix.
  The global notion therefore implies the local one.
  \Cref{characterizations of linear lie algebras consisting of nilpotent endomorphisms} shows that the converse is also true, so that these two concrete notions of nilpotence are actually equivalent.
  (This equivalence depends on the fact that~$\glie$ is not just any collection of endomorphisms, but a Lie~subalgebra of~$\gllie(V)$).
  So there is actually only one notion of concrete nilpotence.%
  \footnote{Recall that in these notes linear Lie~algebras are Lie~subalgebras of~$\gllie(V)$ for some \emph{finite dimensional} vector space~$V$.
  If one drops this finite dimensionality requirement then the statement no longer holds.}
  
  For such a linear Lie~algebra~$\glie$ we also need to compare the abstract notion of nilpotence to the concrete notion of nilpotence.
  This is done by \cref{nilpotent implies ad-nilpotent} which states that (local) concrete nilpotence implies (local) abstract nilpotence.
  The converse however is not true, as can be seen from the example~$\glie = \kf {\id_V}$.
  We can summarize our findings as follows:
  \[
    \begin{tikzcd}[column sep = 7em, row sep = huge]
      \begin{tabular}{c}
        abstract global nilpotence: \\
        $\glie^n = 0$
      \end{tabular}
      \arrow[Leftrightarrow]{r}[above]{\text{Engel’s theorem}}
      &
      \begin{tabular}{c}
        abstract local nilpotence:  \\
        every~$\ad(x)$ nilpotent
      \end{tabular}
      \\
      \begin{tabular}{c}
        concrete global nilpotence: \\
        given by~$\nlie_n(\kf)$
      \end{tabular}
      \arrow[Rightarrow]{u}[left]{\text{$\nlie_n(\kf)$ is nilpotent}\;}
      \arrow[Leftrightarrow]{r}[above]{\text{\cref{characterizations of linear lie algebras consisting of nilpotent endomorphisms}}}
      &
      \begin{tabular}{c}
        concrete local nilpotence: \\
        every~$x$ nilpotent
      \end{tabular}
      \arrow[Rightarrow]{u}[right]{\;\text{\cref{nilpotent implies ad-nilpotent}}}
    \end{tikzcd}
  \]
  Note that \cref{linear lie algebras consisting of nilpotent endomorphisms are nilpotent} can be found in this diagram as the composition of the equivalence on the bottom and the implication on the left.
  
  We also need to consider that to any finite dimensional abstract Lie~algebra~$\glie$ we can associate a linear Lie~algebra, namely~$\glie' \defined \im \ad = \ad(\glie)$.
  That~$\glie$ consists of {\adnilpotent} elements means precisely that~$\glie'$ consists of nilpotent endomorphisms.
  This may be summarized as follows:
  \[
    \begin{tikzcd}
      \text{concrete nilpotence for~$\glie'$}
      \\
      \text{abstract nilpotence for~$\glie$}
      \arrow[Leftrightarrow]{u}
    \end{tikzcd}
  \]
  The hard part of the proof of Engel’s theorem can be summarized by the following equivalences and implications:
  \[
    \begin{tikzcd}[column sep = 7em, row sep = large]
      \begin{tabular}{c}
        concrete local nilpotence \\
        for~$\glie'$
      \end{tabular}
      \arrow[Leftrightarrow]{r}[above]{\text{\cref{characterizations of linear lie algebras consisting of nilpotent endomorphisms}}}
      &
      \begin{tabular}{c}
        concrete global nilpotence  \\
        for~$\glie'$
      \end{tabular}
      \arrow[Rightarrow]{d}[right]{\;\text{$\nlie_n(\kf)$ is nilpotent}}
      \\
      {}
      &
      \begin{tabular}{c}
        abstract global nilpotence  \\
        for~$\glie'$
      \end{tabular}
      \arrow[Leftrightarrow]{d}[right]{\;\text{$\glie$ nilpotent iff~$\glie'$ nilpotent}}
      \\
      \begin{tabular}{c}
        abstract local nilpotence \\
        for~$\glie$
      \end{tabular}
      \arrow[Leftrightarrow]{uu}
      \arrow[Rightarrow, dashed]{r}[above]{\text{Engel’s  theorem}}
      &
      \begin{tabular}{c}
        abstract global nilpotence \\
        for~$\glie$
      \end{tabular}
    \end{tikzcd}
  \]
  
  If~$\glie$ is already a linear Lie~algebra itself then we now have the following relations between the notions of nilpotence for~$\glie$ and~$\glie'$.
  \[
    \begin{tikzcd}[column sep = 7em, row sep = huge]
      \begin{tabular}{c}
        concrete global nilpotence  \\
        for~$\glie'$
      \end{tabular}
      \arrow[Leftrightarrow]{r}[above]{\text{\cref{characterizations of linear lie algebras consisting of nilpotent endomorphisms}}}
      &
      \begin{tabular}{c}
        concrete local nilpotence \\
        for~$\glie'$
      \end{tabular}
    \\
      \begin{tabular}{c}
        abstract global nilpotence \\
        for~$\glie$
      \end{tabular}
      \arrow[Leftrightarrow, dashed]{u}
      \arrow[Leftrightarrow]{r}[above]{\text{Engel’s theorem}}
      &
      \begin{tabular}{c}
        abstract local nilpotence \\
        for~$\glie$
      \end{tabular}
      \arrow[Leftrightarrow]{u}
      \\
      \begin{tabular}{c}
        concrete global nilpotence  \\
        for~$\glie$
      \end{tabular}
      \arrow[Rightarrow]{u}[left]{\text{$\nlie_n(\kf)$ is nilpotent}\;}
      \arrow[Leftrightarrow]{r}[above]{\text{\cref{characterizations of linear lie algebras consisting of nilpotent endomorphisms}}}
      &
      \begin{tabular}{c}
        concrete local nilpotence \\
        for~$\glie$
      \end{tabular}
      \arrow[Rightarrow]{u}[right]{\;\text{\cref{nilpotent implies ad-nilpotent}}}
    \end{tikzcd}
  \]
  The Lie~algebra~$\glie'$ is again a linear Lie~algebra just as~$\glie$ itself.
  Therefore one could be attempted to continue to diagram, by taking the same diagram for~$\glie'$ (then involving~$\glie''$) and pasting it together with the abvove one for~$\glie'$.
  
  But we can observe that if~$\glie$ is any finite dimensional Lie~algebra then for~$\glie'$ the notions of concrete nilpotence and abstract nilpotence are already equivalent.
  Indeed, the missing implication follows from
  \begin{align*}
    {}&
    \text{$\glie'$ is abstract nilpotent}
    \\
    \implies{}&
    \text{$\glie$ is abstract nilpotent}
    \\
    \implies{}&
    \text{$\glie'$ is concrete nilpotent}
  \end{align*}
  Hence for linear Lie~algebras of the form~$\glie'$ the notions of concrete and abstract nilpotence coincide.
  
  We may further summarize our findings as follows:
  \begin{itemize}
    \item
      Gobal nilpotence and local nilpotence are equivalent.
    \item
      Concrete nilpotence implies abstract nilpotence.
    \item
      Abstract nilpotence for~$\glie$ is equivalent to concrete nilpotence for~$\glie'$.
    \item
      For~$\glie'$ both concrete and abstract nilpotence are equivalent.
  \end{itemize}
\end{remark}





\subsection{Lie’s Theorem}


\begin{convention}
  From now on \emph{all} occuring fields are required to be algebraically closed and of characteristic~$0$, unless otherwise specified.
  This convention holds for the rest of these notes.
  In particular all occuring Lie~algebras and vector spaces will have such a field as their ground field, even if not explicitely stated.
\end{convention}


\begin{definition}
  Let~$V$ be a representation of a Lie~algebra~$\glie$.
  For every linear functional~$\lambda \in \glie^*$ the linear subspace~$\gls*{weight space}$ of~$V$ given by
  \[
    V_\lambda
    \defined
    \{
      v \in V
    \suchthat
      \text{$x.v = \lambda(x) v$ for every~$x \in \glie$}
    \}
  \]
  is the \defemph{weight space}\index{weight!space} of~$V$ with respect to~$\lambda$.
  An element~$\lambda \in \glie^*$ with~$V_\lambda \neq 0$ is a \defemph{weight}\index{weight} of~$V$.
\end{definition}


\begin{lemma}[Invariance lemma]
  \index{invariance lemma}
  \index{lemma!invariance}
  Let~$V$ be a finite dimensional representation of a Lie~algebra~$\glie$ and let~$I$ be an ideal in~$\glie$.
  Then~$V$ is also a representation of~$I$ by restricting the action of~$\glie$ to~$I$.
  For~$\lambda \in I^*$ let~$V_\lambda$ be the~{\weightspace{$I$}} of~$V$ with respect to~$\lambda$.
  Then~$V_\lambda$ is already a~{\subrepresentation{$\glie$}} of~$V$.
\end{lemma}


\begin{proof}
  For~$v \in V$ and~$x_1, \dotsc, x_n \in \glie$ we will write
  \[
    x_1 \dotsm x_n v
    \defined
    x_1.(x_2.( \dotsc x_{n-1}.(x_n.v) \dotsc )) \,.
  \]
  The case~$V_\lambda = 0$ is welll understood, so for the rest of this proof we fix some~$\lambda \in I^*$ with~$V_\lambda \neq 0$.
 
  That~$V_\lambda$ is a {\subrepresentation{$\glie$}} of~$V$ means that~$yv \in V_\lambda$ for all~$y \in \glie$ and~$v \in V_\lambda$, which  means that~$xyv = \lambda(x)yv$ for all~$x \in I$,~$y \in \glie$ and~$v \in V_\lambda$.
  We see that
  \[
    xyv
    =
    [x,y]v + yxv
    =
    \lambda([x,y])v + \lambda(x)yv
  \]
  for all~$x \in I$,~$y \in \glie$ and~$v \in V_\lambda$.
  We hence need to show that~$\lambda([x,y]) = 0$ for all~$x \in I$ and~$y \in \glie$.
 
  Until further notice we fix some~$y \in \glie$ and some nonzero~$v \in V_\lambda$.
  As~$V$ is finite dimensional there exists some maximal~$n \geq 1$ such that the vectors~$v, yv, \dotsc, y^n v$ are linearly independent.
  Let
  \[
    W_i
    \defined
    \vspan_k(v, yv, \dotsc, y^i v)
  \]
  for every~$i = 0, \dotsc, n$.
  The linear space~$W_n$ is invariant under the action of~$y$ because the vector~$y^{n+1} v$ is linearly dependent on the vectors~$v, \dotsc, y^n v$.
 
  \begin{claim*}
    The linear subspaces~$W_1, \dotsc, W_n$ are~{\subrepresentations{$I$}} of~$V$.
    With respect to the basis~$w, y w, \dotsc, y^i w$ of~$W_i$ the action of any~$x \in I$ is represented by an upper triangular matrix whose every diagonal entry is~$\lambda(x)$.
  \end{claim*}

 \begin{proof}[Proof of the claim]
    We prove the claim by induction over~$i$.
    The claim holds for~$i = 0$ since~$W_0 = \kf v$ is spanned by the vector~$v$ which is a~{\weightvector{$\lambda$}} for the action of~$I$.
    Suppose now that~$i < n$ and that the claim holds for the linear subspaces~$W_0, \dotsc, W_i$.
    If~$x \in I$ then also~$[x,y] \in I$ and therefore
    \[
      x y^{i+1} v
      =
      \underbrace{[x,y] y^i v}_{\mathclap{\substack{\in W_i \\ \text{by induction}}}} + y x y^i v
      \equiv
      y x y^i v
      \mod
      W_i \,.
    \]
    We find by the induction hypothesis that~$x y^i v \in W_i$ because~$W_i$ is a subrepresentation, and also that this vector is of the form
    \[
      x y^i v
      \equiv
      \lambda(x) y^i v
      \mod
      W_{i-1} \,.
    \]
    It follows that
    \[
      y x y^i v
      \equiv
      \lambda(x) y^{i+1} v
      \mod W_i  \,,
    \]
    because~$y W_{i-1} \subseteq W_i$.
    This shows the claim for~$W_{i+1}$.
 \end{proof}
  
  Let~$x \in I$.
  It follows from the above claim that the~{\dimensional{$(n+1)$}} linear subspace~$W_n$ is invariant under the action of~$[x,y]$ because~$[x,y] \in I$.
  This action is given by an endomorphism~$\phi_{[x,y]} \in W_n \to W_n$, that is represented with respect to some suitable basis of~$W_n$ by an upper triangular matrix whose every diagonal entry is~$\lambda([x,y])$.
  It follows that in particular
  \begin{equation}
    \label{invariance lemma zero trace}
    \tr \phi_{[x,y]}
    =
    (n+1) \lambda([x,y])  \,.
  \end{equation}
  On the other hand the linear subspace~$W_n$ is invariant under the action of both~$x$ (by the claim because~$x \in I$) and under the action of~$y$ (as seen above before the claim).
  These two elements act by endomorphisms~$\phi_x$ and~$\phi_y$ on~$W_n$.
  Because~$V$ is a representation of the Lie~algebra~$\glie$ it follows that~$\phi_{[x,y]} = [\phi_x, \phi_y]$ and thus~$\tr \phi_{[x,y]} = 0$.
  Together with \eqref{invariance lemma zero trace} it follows that~$\lambda([x,y]) = 0$.
\end{proof}


\begin{theorem}[Lie]
  \index{Lie’s theorem}
  Let~$\glie$ be a solvable Lie~subalgebra of~$\gllie(V)$ for some nonzero finite dimensional~{\vectorspace{$\kf$}}. Then there exists a common eigenvector for~$\glie$, i.e.\ some nonzero~$v \in V$ with~$x(v) \in \kf v$ for every~$x \in \glie$.
\end{theorem}


\begin{proof}
  We show the statement by induction over the dimension~$n$ of~$\glie$.
  If~$n = 0$ then~$\glie = 0$ and any nonzero~$v \in V$ does the job.
  If~$n = 1$ then~$\glie = \kf x$ for some nonzero~$x \in \gllie(V)$.
  Then any eigenvector of~$x$ does the job (and such an eigenvector exists because the field~$\kf$ is assumed to be algebraically closed and~$V \neq 0$).
 
  Suppose that~$n \geq 2$ and that the statement holds for every smaller dimension.
  The proof proceeds in four steps:
  \begin{itemize}
    \item
      Finding an ideal~$I$ in~$\glie$ of codimension~$1$.
    \item
      Finding a common eigenvectors for~$I$ by induction hypothesis.
    \item
      Showing that~$\glie$ stabilizes as nonzero subspace~$U \subseteq V$ of such eigenvectors.
    \item
      Writing~$\glie = I \oplus \kf y$ (as vector spaces) and finding an eigenvector of~$y$ in~$U$.
  \end{itemize}

  For the first step we note that~$[\glie,\glie]$ in~$\glie$ because~$\glie$ is nonzero and solvable.
  Hence~$\glie/[\glie,\glie]$ is a nonzero abelian Lie~algebra.
  Any linear subspace~$J \subseteq \glie/[\glie,\glie]$ of codimension~$1$ is then an ideal in~$\glie/[\glie,\glie]$.
  Hence the preimage~$I = \pi^{-1}(J)$ under the canonical projection~$\glie \to \glie/[\glie,\glie]$ is an ideal in~$\glie$ of codimension~$1$.
 
  For the second step we note~$I$ is solvable because it is a Lie~subalgebra of~$\glie$.
  So by induction hypothesis there exists a common eigenvector for~$I$.
  Hence there exists some~$\lambda \in I^*$ with~$U \defined V_\lambda \neq 0$, where~$V_\lambda$ denotes the~{\weightspace{$I$}} with respect to~$\lambda$.
  
  For the third step we apply the invariance lemma to see that~$U$ is a~{\subrepresentation{$\glie$}} of~$V$.
 
  For the fourth step we may write~$\glie = I \oplus \kf y$ for some~$y \in \glie$ because~$I$ has codimension~$1$.
  Then~$y$ stabilizs~$U$ and hence admits an eigenvector~$v \in U$, as~$\kf$ is algebraically closed.
  This vector~$v$ is then a common eigenvector for~$\glie$.
\end{proof}


\begin{remark}
 The idea of organizing the proof into four steps is taken from~\cite[\S 4.1]{Humphreys}.
%  This organizational structure also serves to emphasize the similarities with the proof of Proposition~\ref{common eigenvector for nilpotent Lie algebras}.
\end{remark}


\begin{remark}
  Observe that for a Lie~subalgebra~$\glie$ of some~$\gllie(V)$, where~$V$ is some vector space, a vector~$v \in V$ is a common eigenvector for~$\glie$ if and only if the the subspace~$\vspan_\kf(v)$ is a {\onedimensional}~{\subrepresentation{$\glie$}} of~$V$.
\end{remark}


\begin{proposition}
  \label{triangularization of solvable linear lie algebras}
  Let~$\glie$ be a Lie~subalgebra of~$\gllie(V)$ where~$V$ is a finite dimensional~{\vectorspace{$\kf$}}.
  Then the following conditions on~$\glie$ are equivalent:
  \begin{equivalenceslist}
    \item
      \label{is solvable}
      $\glie$ is solvable.
    \item
      \label{stabilizes a complete flag}
      $\glie$ stabilizes some complete flag of~$V$, i.e.\ there exists a complete flag~$(V_i)_{i=0}^n$ with~$x(V_i) \subseteq V_i$ for all~$x \in \glie$ and~$i = 0, \dotsc, n$.
    \item
      \label{is triangularizable}
      There exists a basis of~$V$ with respect to which every~$x \in \glie$ is represented by an upper triangular matrix.
      In particular,~$\glie$ is isomorphic to a Lie~subalgebra of~$\tlie_n(\kf)$ for~$n = \dim V$.
  \end{equivalenceslist}
\end{proposition}


\begin{proof}
  \leavevmode
  \begin{implicationlist}
    \item[\ref*{is solvable}~$\implies$~\ref*{is triangularizable}]
      The vector space~$V$ is the natural representation of the Lie~algebra~$\glie$.
      Let~$\mathcal{V}$ be the class of all quotient representations of~$V$.
      For every~$W \in \mathcal{V}$ the image of the Lie~algebra homomorphism~$\rho_W \colon\glie \to \gllie(W)$,~$x \mapsto \restrict{x}{W}$ is again solvable.
      It therefore follows from Lie’s theorem that there exists for every nonzero~$W \in \mathcal{V}$ some nonzero~$w \in W$ such that~$w$ is a common eigenvector vor~$\rho_W(\glie)$;
      then~$\vspan_\kf(v)$ is a {\onedimensional} subrepresentation of~$W$.
      
      This shows that we can apply \cref{triangularization lemma} to get the desired result.
    \item[\ref*{is triangularizable}~$\implies$~\ref*{is solvable}]
      Every Lie~subalgebra of~$\tlie_n(\kf)$ is again solvable.
    \item[\ref*{is triangularizable}~$\iff$~\ref*{stabilizes a complete flag}]
      This follows from \cref{triangular recall}.
    \qedhere
  \end{implicationlist}
\end{proof}


\begin{corollary}
  \label{solvable linear lie algebra has commutator consisting of nilpotent endomorphisms}
  If~$\glie$ is a solvable Lie~subalgebra of~$\gllie(V)$ for some finite dimensional vector space~$V$ then~$[\glie, \glie]$ consists of nilpotent endomorphisms.
\end{corollary}


\begin{proof}
  There exists some basis of~$V$ with respect to which every~$x \in \glie$ is represented by an upper triangular matrix.
  Every~$x \in [\glie, \glie]$ is then represented by a strictly upper triangular matrix and hence nilpotent.
\end{proof}


\begin{corollary}
  \label{solvable iff derived is nilpotent}
  A finite dimensional Lie~algebra~$\glie$ is solvable if and only if~$[\glie, \glie]$ is nilpotent.
\end{corollary}


\begin{proof}
  If~$[\glie, \glie] = \glie^{(1)}$ is nilpotent then it is solvable, and hence~$\glie$ is solvable.
  
  For the converse we observe that~$\glie' \defined \ad(\glie)$ is a Lie~subalgebra of~$\gllie(\glie)$ that is again solvable.
  It follows that~$[\glie, \glie]$ consists of~{\adnilpotent} elements because
  \[
    \ad([\glie, \glie])
    =
    [\ad(\glie), \ad(\glie)]
    =
    [\glie', \glie']
  \]
  (since~$\ad$ is a homomorphism of Lie~algebras) consists of nilpotent endomorphisms of~$\glie$ by \cref{solvable linear lie algebra has commutator consisting of nilpotent endomorphisms}.
  The Lie~algebra~$[\glie, \glie]$ is hence nilpotent by Engel’s theorem.
\end{proof}


\begin{corollary}
  \label{characterization of fd solvable Lie algebras}
  For a finite dimensional Lie~algebra~$\glie$ the following conditions are equivalent:
  \begin{equivalenceslist}
    \item
      \label{g is solvable}
      The Lie~algebra~$\glie$ is solvable.
    \item
      \label{every fd rep is triangularizable}
      Every finite dimensional representation of~$\glie$ admits a basis with respect to which the action of every~$x \in \glie$ is given by an upper triangular matrix.
    \item
      \label{every fd rep admits a complete flag of subreps}
      Every finite dimensional representation of~$\glie$ admits a complete flag~$(V_i)_{i=0}^n$ consisting of subrepresentations.
    \item
      \label{every fd rep contains a one dimensional subrep}
      Every nonzero finite dimensional representation of~$\glie$ contains a {\onedimensional} subrepresentation.
    \item
      \label{every fd irrep is one dimensional}
      Every finite dimensional irreducible representation of~$\glie$ is {\onedimensional}.
  \end{equivalenceslist}
\end{corollary}


\begin{proof}
  \leavevmode
  \begin{implicationlist}
    \item[\ref*{g is solvable}~$\implies$~\ref*{every fd rep is triangularizable}]
      If~$(V, \rho)$ is any finite dimensional representation of~$\glie$ then~$\glie' \defined \rho(\glie)$ is a solvable Lie~subalgebra of~$\gllie(V)$.
      It follows from \cref{triangularization of solvable linear lie algebras} that there exists a basis of~$V$ with respect to which every~$x' \in \glie'$ is given by an upper triangular matrix.
      This means that the action of every~$x \in \glie'$ on~$V$ is given by an upper triangular matrix with respect to this basis.
    \item[\ref*{every fd rep is triangularizable}~$\iff~$\ref*{every fd rep admits a complete flag of subreps}]
      If~$(V, \rho)$ is any finite dimensional representation of~$\glie$ then~$\glie' \defined \rho(\glie)$ is a Lie subalgebra of~$\gllie(V)$.
      The equivalence follows by applying \cref{triangularization of solvable linear lie algebras} to~$\glie'$.
    \item[\ref*{every fd rep admits a complete flag of subreps}~$\implies$~\ref*{every fd rep contains a one dimensional subrep}]
      Every complete flag of subrepresentations contains a {\onedimensional} term.
    \item[\ref*{every fd rep contains a one dimensional subrep}~$\implies$~\ref*{every fd irrep is one dimensional}]
      Every finite dimensional irreducible representation~$V$ contains a {\onedimensional} subrepresentation~$U$, for which necessarily~$V = U$ by the irreduciblity of~$V$.
    \item[\ref*{every fd irrep is one dimensional}~$\implies$~\ref*{every fd rep contains a one dimensional subrep}]
      Every finite dimensional representation contains an irreducible subrepresentation (e.g.\ a subrepresentation of minimal nonzero dimension) that is then {\onedimensional}.
    \item[\ref*{every fd rep contains a one dimensional subrep}~$\implies$~\ref*{every fd rep is triangularizable}]
      We can apply \cref{triangularization lemma} to the class of all finite dimensional representations of~$\glie$.
    \item[\ref*{every fd rep is triangularizable}~$\implies$~\ref*{g is solvable}]
      We find that for the adjoint representation of~$\glie$ there exists a basis of~$\glie$ with respect to which the action of every~$x \in \glie$ (i.e.~$[x,-]$) is given by an upper triangular matrix.
      This means by \cref{triangularization of solvable linear lie algebras} that the Lie~subalgebra~$\glie' \defined \ad(\glie)$ of~$\gllie(V)$ is solvable.
      Therefore~$\glie$ is solvable because~$\glie' \cong \glie/\centerlie(\glie)$ with~$\centerlie(\glie)$ being an abelian and hence solvable ideal in~$\glie$.
    \qedhere
  \end{implicationlist}
\end{proof}


\begin{remark}
  We used the finite dimensionality of~$\glie$ only for the implication~\ref*{every fd rep is triangularizable}~$\implies$~\ref*{g is solvable}.
  The author does not know what happens for this implication when~$\glie$ is infinite dimensional.
  (He suspects that it is false, but does not know a counterexample.)
\end{remark}


\begin{remark}
  \Cref{characterization of fd solvable Lie algebras} does not hold for general fields~$\kf$, even if algebraically closed.
  To see this let~$\kf$ be an algebraically closed field with~$\ringchar \kf = 2$ and let~$\glie \defined \sllie_2(\kf)$.
  Recall that the standard basis of~$\glie$ is given by the matrices
  \begin{gather*}
    e
    =
    \begin{pmatrix}
      0 & 1 \\
      0 & 0
    \end{pmatrix} \,,
    \qquad
    h
    =
    \begin{pmatrix*}[r]
      1 &  0  \\
      0 & -1
    \end{pmatrix*},
    \qquad
    f
    =
    \begin{pmatrix}
      0 & 0 \\
      1 & 0
    \end{pmatrix}
  \shortintertext{with}
    [h,e] = 0 \,,
    \qquad
    [h,f] = 0 \,,
    \qquad
    [e,f] = h \,.
  \end{gather*}
  We find that the Lie algebra~$\glie$ is nilpotent (with~$\glie^2 = 0$) and in particular also solvable.
  Let~$V \defined \kf^2$ be the natural representation of~$\glie$, i.e.~$\glie$ acts on~$V$ by~$x.v = x(v)$ for all~$x \in \glie$ and~$v \in V$.
  Then
  \[
    e.\vect{x \\ y}
    =
    \vect{y \\ 0}
    \qquad\text{and}\qquad
    f.\vect{x \\ y}
    =
    \vect{0 \\ x} \,.
  \]
  It follows that if~$U \subseteq V$ is a nonzero subrepresentation then~$U$ contains one of the standard basis vectors~$e_1$,~$e_2$ of~$V$, and then also the other one.
  Hence~$U = V$, which shows that~$V$ is an irreducible representation of~$\glie$.
  But~$V$ is not {\onedimensional}.
\end{remark}
