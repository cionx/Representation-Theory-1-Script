\section{Nilpotent and Solvable Lie~Algebras}





\subsection{Definition, Examples and Properties}


\begin{definition}
  Let~$A$ be an associative~{\algebra{$\kf$}}.
  An element~$a \in A$ is \defemph{nilpotent}\index{nilpotent} if~$a^n = 0$ for some~$n \geq 1$.
  Given a Lie~algebra~$\glie$ an element~$x \in \glie$ is called \defemph{\adnilpotent} if the endomorphism~$\ad(x) \in \End_{\kf}(\glie)$ is nilpotent.
\end{definition}


\begin{lemma}
  \label{nilpotent implies ad-nilpotent}
  If~$A$ is an associative~{\algebra{$\kf$}} and~$x \in A$ is nilpotent then~$x$ is also {\adnilpotent}.
\end{lemma}

\begin{proof}
  Let~$\lambda_x \colon A \to A$,~$a \mapsto xa$ denote the left multiplication by~$x$ and~$\rho_x \colon A \to A$,~$a \mapsto ax$ the right multiplication by~$x$.
  Because~$x$ is nilpotent both~$\lambda_x$ and~$\rho_x$ are nilpotent.
  The endomorphisms~$\lambda_x$ and~$\rho_x$ commute because~$A$ is associative.
  Hence~$\ad(x) = \lambda_x - \rho_x$ is the sum (or rather difference) of two commuting nilpotent endomorphisms, and therefore again nilpotent.
\end{proof}


\begin{definition}
  Let~$\glie$ be a Lie~algebra.
  \begin{enumerate}
    \item
      Let~$\glie^0 \defined \glie$ and inductively~$\glie^{i+1} \defined [\glie, \gls*{central series}]$ for every~$i \geq 0$.
      The decreasing sequence
      \[
        \glie
        =
        \glie^0
        \supseteq
        \glie^1
        \supseteq
        \glie^2
        \supseteq
        \glie^3
        \supseteq
        \dotsb
      \]
      is the \defemph{central series}\index{central series}\index{series!central} of~$\glie$.
      The Lie~algebra~$\glie$ is \defemph{nilpotent}\index{nilpotent!Lie algebra}\index{Lie algebra!nilpotent} if~$\glie^n = 0$ for~$n$ sufficiently large.
    \item
      Let~$\glie^{(0)} \defined \glie$ and~$\glie^{(i+1)} \defined [\glie^{(i)}, \gls*{derived series}]$ for every~$i \geq 0$.
      The decreasing sequence
      \[
        \glie
        =
        \glie^{(0)}
        \supseteq
        \glie^{(1)}
        \supseteq
        \glie^{(2)}
        \supseteq
        \glie^{(3)}
        \supseteq
        \dotsb
      \]
      is the \defemph{derived series}\index{derived series}\index{series!derived} of~$\glie$.
      The Lie~algebra~$\glie$ is \defemph{solvable}\index{solvable}\index{Lie algebra!solvable} if~$\glie^{(n)} = 0$ for~$n$ sufficiently large.
  \end{enumerate}
\end{definition}


\begin{remark}
  Let~$\glie$ be a Lie~algebra.
  \begin{enumerate}
    \item
      Both the central series~$(\glie^i)_{i \geq 0}$ and the derived series~$(\glie^{(i)})_{i \geq 0}$ consist of ideals of~$\glie$.
    \item
      If~$\glie$ is nilpotent then it is also solvable because~$\glie^{(i)} \subseteq \glie^i$ for every~$i \geq 0$.
    \item
      Our indexing convention is choosen so that~$\glie^{(i+j)} = (\glie^{(i)})^{(j)}$ for all~$i, j \geq 0$.
    \item
      It follows that~$\glie$ is solvable if and only if~$\glie^{(i)}$ is solvable for some~$i \geq 0$, if and only if~$\glie^{(i)}$ is solvable for every~$i \geq 0$.
  \end{enumerate}
\end{remark}


\begin{examples}
  \label{examples for solvable and nilpotent}
  \leavevmode
  \begin{enumerate}
    \item
      The Lie~algebra of upper triangular matrices~$\tlie_n(\kf)$ is solvable but not nilpotent.
    \item
      The Lie~algebra of strictly upper triangular matrices~$\nlie_n(\kf)$ is nilpotent, and hence also solvable.
    \item
      If~$n \geq 2$ then the special linear Lie~algebra~$\sllie_2(\Complex)$ is simple, and hence~$[\sllie_n(\Complex), \sllie_n(\Complex)] = \sllie_n(\Complex)$.
      Thus~$\sllie_n(\Complex)$ is neither solvable nor nilpotent.
    \item
      The general linear Lie~algebra~$\gllie_n(\Complex)$ is not solvable because
      \[
        \gllie_n(\Complex)^{(1)}
        =
        [\gllie_n(\Complex), \gllie_n(\Complex)]
        =
        \sllie_n(\Complex)
      \]
      is not solvable.
    \item
      If~$\glie$ is abelian then~$\glie$ is nilpotent (with~$\glie^1 = 0$) and therefore also solvable.
    \item
      Every {\onedimensional} Lie~algebra is abelian, thus nilpotent and hence solvable.
      The same goes for the {\twodimensional} abelian Lie~algebra.
      The {\twodimensional} non-abelian Lie~algebra~$\glie$ admits a basis~$x$,~$y$ with~$[x,y] = x$.
      Then~$\glie^i = 0$ for every~$i \geq 2$ but~$\glie^{(i)} = \kf x$ for every~$i \geq 1$.
      The Lie~algebra~$\glie$ is thus solvable but not nilpotent.
    \item
      A \defemph{Heisenberg Lie~algebra}\index{Heisenberg Lie algebra}\index{Lie algebra!Heisenberg} consists of a real vector space with basis~$P_1, \dotsc, P_n, Q_1, \dotsc, Q_n, C$ together with the Lie~bracket satisfying the relations
      \begin{itemize}
        \item
          $[P_i, P_j] = [Q_i, Q_j] = 0$ whenever~$i \neq j$,
        \item
          $[P_i, C] = [Q_i, C] = 0$ for all~$i = 1, \dotsc, n$,
        \item
          $[P_i, Q_j] = \delta_{ij} C$ for all~$i,j = 1, \dotsc, n$.
      \end{itemize}
      This defines a nilpotent Lie~algebra.
  \end{enumerate}
\end{examples}


\begin{proposition}[Properites of solvability]
  \label{properties of solvable and nilpotent}
  Let~$\glie$ be a Lie~algebra.
  \begin{enumerate}
    \item
      If~$\hlie$ is another Lie~algebra and~$f \colon \glie \to \hlie$ is a homomorphism of Lie~algebras then
      \[
        f(\glie)^{(i)}
        =
        f(\glie^{(i)})
      \]
      for every~$i \geq 0$.
    \item
      If the Lie~algebra~$\glie$ is solvable then any Lie~subalgebra~$\hlie$ of~$\glie$ is again solvable.
    \item
      If~$\glie$ is solvable and~$I$ is an ideal in~$\glie$ then the quotient Lie~algebra~$\glie/I$ is again solvable.
    \item
      If~$I$ is an ideal in~$\glie$ such that both~$I$ and~$\glie/I$ is solvable then~$\glie$ is solvable.
  \end{enumerate}
\end{proposition}


\begin{proof}
  \leavevmode
  \begin{enumerate}
    \item
      This can be shown by induction because~$f([X,Y]) = [f(X), f(Y)]$ for any two subsets~$X, Y \subseteq \glie$.
    \item
      The inclusion~$\hlie \to \glie$ is a Lie algebra homomorphism hence~$\hlie^{(i)} \subseteq \glie^{(i)}$ for every~$i \geq 0$.
      It thus follows from~$\glie^{(n)} = 0$ for~$n$ sufficiently large that also~$\hlie^{(n)} = 0$.
    \item
      The canonical projection~$\pi \colon \glie \to \glie/I$ is a Lie algebra homomorphism, and hence
      \[
        (\glie/I)^{(i)}
        =
        \pi(\glie)^{(i)}
        =
        \pi(\glie^{(i)})
      \]
      for every~$i \geq 0$.
      It thus follows from~$\glie^{(n)} = 0$ for~$n$ sufficietly large that also~$(\glie/I)^{(n)} = 0$.
    \item
      It follows from the solvability from~$\glie/I$ that~$(\glie/I)^{(n)} = 0$ for~$n$ sufficiently large.
      This means that
      \[
        0
        =
        (\glie/I)^{(n)}
        =
        \pi(\glie)^{(n)}
        =
        \pi(\glie^{(n)})  \,,
      \]
      so that~$\glie^{(n)} \subseteq I$.
      It follows from the solvability of~$I$ that~$I^{(m)} = 0$ for~$m$ sufficiently large.
      Then
      \[
        \glie^{(n+m)}
        =
        (\glie^{(n)})^{(m)}
        \subseteq
        I^{(m)}
        =
        0
      \]
      and hence~$\glie^{(n+m)} = 0$.
    \qedhere
  \end{enumerate}
\end{proof}


\begin{proposition}[Properties of nilpotence]
  \label{properties of nilpotence}
  Let~$\glie$ be a Lie~algebra.
  \begin{enumerate}
    \item
      If~$\hlie$ is another Lie~algebra and~$f \colon \glie \to \hlie$ is a homomorphism of Lie~algebras then
      \[
        f(\glie)^i
        =
        f(\glie^i)
      \]
      for every~$i \geq 0$.
    \item
      If the Lie~algebra~$\glie$ is nilpotent then any Lie~subalgebra~$\hlie$ of~$\glie$ is again nilpotent.
    \item
      If~$\glie$ is nilpotent and~$I$ is an ideal in~$\glie$ then the quotient Lie~algebra~$\glie/I$ is again nilpotent.
    \item
      If~$I$ is an ideal in~$\glie$ such that~$I$ is contained in the center~$\centerlie(\glie)$ and such that~$\glie/I$ is nilpotent then~$\glie$ is nilpotent.
    \item
      If~$\glie$ is nonzero and nilpotent then~$\centerlie(\glie)$ is nonzero.
    \item
      If~$\glie$ is nilpotent then every~$x \in \glie$ is~{\adnilpotent}.
  \end{enumerate}
\end{proposition}


\begin{proof}
  \leavevmode
  \begin{enumerate}
    \item
      This can be shown inductively because~$f([X,Y]) = [f(X), f(Y)]$ for any two subsets~$X, Y \subseteq \glie$.
    \item
      The inclusion~$\hlie \to \glie$ is a homomorphism of Lie~algebras, hence~$\hlie^i \subseteq \glie^i$ for all~$i \geq 0$.
      It thus follows for~$n$ sufficiently large from~$\glie^i = 0$ that also~$\hlie^i = 0$.
    \item
      The canonical projection~$\pi \colon \glie \to \glie/I$ is a homorphism of Lie~algebras, hence
      \[
        (\glie/I)^i
        =
        \pi(\glie)^i
        =
        \pi(\glie^i)
      \]
      for every~$i \geq 0$.
      It thus follows for~$n$ sufficiently large from~$\glie^n = 0$ that also~$(\glie/I)^n = 0$.
    \item
      For~$n$ sufficiently large we have that~$(\glie/I)^n = 0$ and hence
      \[
        0
        =
        (\glie/I)^n
        =
        \pi(\glie)^n
        =
        \pi(\glie^n)  \,.
      \]
      This means that~$\glie^n \subseteq I \subseteq \centerlie(\glie)$.
      Then
      \[
        \glie^{n+1}
        =
        [\glie, \glie^n]
        \subseteq
        [\glie, \centerlie(\glie)]
        =
        0
      \]
      and hence~$\glie^{n+1} = 0$.
    \item
      Let~$n \geq 0$ be minimal with~$\glie^n = 0$.
      That~$\glie$ is nonzero means that~$n \geq 1$.
      Then~$\glie^{n-1} \neq 0$, but also
      \[
        [\glie, \glie^{n-1}]
        =
        \glie^n
        =
        0
      \]
      and hence~$\glie^{n-1} \subseteq \centerlie(\glie)$.
      It follows that~$\centerlie(\glie) \neq 0$.
    \item
      Let~$x \in \glie$.
      We find by induction that
      \[
        \ad(x)^n(\glie)
        \subseteq
        \glie^n
      \]
      for all~$n \geq 0$.
      It hence follows for~$n$ sufficiently large from~$\glie^n = 0$ that~$\ad(x)^n = 0$.
    \qedhere
  \end{enumerate}
\end{proof}

\begin{corollary}
  \label{solvable via ses}
  If~$I$ is an ideal in a Lie~algebra~$\glie$ then~$\glie$ is solvable if and only if both~$I$ and~$\glie/I$ are solvable
  Hence in a short exact sequence
  \[
    0
    \to
    I
    \to
    \glie
    \to
    \glie/I
    \to
    0
  \]
  the middle term is solvable if and only if the two outer terms are solvable.
  \qed
\end{corollary}


\begin{remark}
  The analogeous statement about nilpotency does not necessarily hold.
  Take for example the {\twodimensional} non-abelian Lie~algebra~$\glie$, that has a basis~$x$,~$y$ with~$[x,y] = x$.
  Then the {\onedimensional} linear subspace~$I \defined kx$ is an abelian ideal in~$\glie$ and in particular nilpotent.
  The quotient~$\glie/I$ is one-dimensional and therefore also nilpotent.
  But~$\glie$ itself is not nilpotent as seen in \cref{examples for solvable and nilpotent}.
\end{remark}


\begin{corollary}
  \label{sum of solvable ideals is solvable}
  Let~$\glie$ be a Lie~algebra and let~$I$ and~$J$ be two solvable ideals in~$\glie$.
  Then their sum~$I + J$ is again solvable.
\end{corollary}


\begin{proof}
  We consider the short exact sequence
  \[
    0
    \to
    I
    \to
    I+J
    \to
    (I+J)/I
    \to
    0 \,.
  \]
  The terms~$I$ and~$(I+J)/I \cong J/(I \cap J)$ are both solvable (the second one because~$J$ is solvable) and hence the middle term~$I+J$ is solvable.
\end{proof}


\begin{definition}
 Let~$\glie$ be a finite dimenisonal Lie~algebra.
 It follows from \cref{sum of solvable ideals is solvable} that~$\glie$ contains a unique maximal solvable ideal. This ideal is called the \defemph{radical}\index{radical} of~$\glie$ and is denoted by~\gls*{radical}.
\end{definition}


\begin{remark}
  If~$\glie$ is a Lie~algebra and~$I$ and~$J$ are two nilpotent ideals in~$\glie$ then it can be shown that their sum~$I+J$ is again nilpotent.
  It follows that every finite dimensional Lie~algebra~$\glie$ admits a unique maxmial nilpotent ideal, that is then called the \defemph{nilradical}\index{nilradical} of~$\glie$.
\end{remark}





\subsection{Engel’s Theorem}


\begin{remark}
  If~$V$ is an~{\dimensional{$n$}} vector space over~$\kf$ and~$x \colon V \to V$ a nilpotent endomorphism then~$0$ is the only eigenvalue of~$x$ (and occurs with algebraic multiplicity~$n$).
  Hence there exists an eigenvector~$v \in V$ (which entails that~$v \neq 0$) with~$x(v) = 0$.
  The following proposition generalizes this observations to linear Lie~algebras consisting of nilpotent endomorphisms.
\end{remark}


\begin{proposition}
  \label{common eigenvector for nilpotent Lie algebras}
  Let~$V$ be a nonzero finite dimensional vector space and let~$\glie$ be a Lie~subalgebra of~$\gllie(V)$ consisting of nilpotent endomorphisms.
  Then there exists some nonzero~$v \in V$ with~$x(v) = 0$ for every~$x \in \glie$, i.e.~$v$ is a common eigenvector for all~$x \in \glie$ (all of which are nilpotent and thus have~$0$ as their only eigenvalue).
\end{proposition}


\begin{proof}
  We show the statement by induction over the dimension of~$\glie$.
  If~$\dim \glie = 0$ then~$\glie = 0$ and we may choose any nonzero~$v \in V$.
  If~$\dim \glie = 1$ then~$\glie = \kf x$ for any nonzero~$x \in \glie$ and we may choose any nonzero~$v \in \ker x$ (which exists because~$x$ is a nilpotent endomorphism of~$V$).
  Let now~$\dim \glie \geq 2$ and suppose that the \lcnamecref{common eigenvector for nilpotent Lie algebras} holds for all strictly smaller dimensions.
  
  We now proceed in two steps:
  In the first step we show that the \lcnamecref{common eigenvector for nilpotent Lie algebras} holds if~$\glie$ admits a nonzero proper ideal~$I$.
  We do so by applying the induction hypothesis to~$I$ and~$\glie/I$ (or rather~$\glie'$, a further quotient of~$\glie/I$).
  In the second step we show that~$\glie$ does indeed admit such an ideal.
  We do so by showing that every maximal proper Lie~subalgebra~$I$ of~$\glie$ is already an ideal, which we do by showing that~$\normallie_\glie(I) = \glie$ because~$I$ acts nilpotent on the quotient vector space~$\glie/I$.
  
  If~$\hlie$ is a proper Lie~subalgebra of~$\glie$ then the linear subspace
  \[
    W_\hlie
    \defined
    \{
      v \in V
    \suchthat
      \text{$x.v = 0$ for every~$x \in \hlie$}
    \}
  \]
  is by induction hypothesis nonzero.
  We observe that if~$I$ is an ideal in~$\glie$ then~$W_I$ is already a {\subrepresentation{$\glie$}} of~$V$:
  Indeed, we have for all~$x \in \glie$ and~$w \in W_I$ that
  \[
    z.(x.w)
    =
    x.(z.w) - [x,z].w
  \]
  for every~$z \in I$, with~$z.w = 0$ and~$[x,z].w = 0$ because~$z, [x,z] \in I$.
  
  Suppose now that~$\glie$ admits a nonzero proper ideal~$I$.
  Seen we find that~$W \defined W_I$ is a nonzero {\subrepresentation{$\glie$}} of~$V$.
  We hence get a homomorphism of Lie~algebras
  \[
    f
    \colon
    \glie
    \to
    \gllie(W) \,,
    \quad
    x
    \mapsto
    \restrict{x}{W} \,.
  \]
  The image~$\glie' \defined \im f = f(\glie)$ is a Lie~subalgebra of~$\gllie(W)$ that again consists of nilpotent endomorphisms.
  But the nonzero ideal~$I$ is contained in the kernel of~$f$, hence~$\glie'$ has strictly smaller dimension than~$\glie$.
  We hence find by induction hypothesis that there exists some nonzero~$w \in W$ with~$y.w = 0$ for every~$y \in \glie'$.
  By definition of~$\glie'$ this means that~$x.w = 0$ for every~$x \in \glie$, so we may choose~$v = w$.
  
  It remains to show that~$\glie$ admits a nonzero proper ideal.
  Let~$I$ be a proper Lie~subalgebra of~$\glie$ of maximal dimension.
  Then~$I$ is nonzero, because~$\glie$ contains nonzero proper Lie~subalgebras (e.g.~$\kf x$ with~$x \in \glie$, where we use that~$\dim \glie \geq 2$), and maximal among all Lie~subalgebras of~$\glie$ with respect to inclusion.
  We claim that~$I$ is already an ideal in~$\glie$, which is then the desired nonzero proper ideal.
  
  Indeed, for every~$x \in I$ the adjoint action~$\ad(x) = [x,-]$ maps~$I$ into itself, and hence descends to an endomorphism~$\induced{x} \colon \glie/I \to \glie/I$.
  We hence get a homomorphism of Lie~algebras
  \[
    g
    \colon
    I
    \to
    \gllie(\glie/I)  \,,
    \quad
    z
    \mapsto
    \induced{z} \,.
  \]
  The image of~$g$ is a Lie subalgebra of~$\gllie(\glie/I)$ of dimension strictly smaller than~$\glie$.
  So by induction hypothesis there exists some nonzero~$y \in \glie/I$ with~$\induced{z}.y = 0$ for every~$x \in I$.
  If~$x \in \glie$ is a representation of~$y$ then this means that~$x \notin I$ (as~$y$ is nonzero) but~$[z,x] = z.x \in I$ for every~$z \in I$ (because~$\induced{z}.y = 0$ for every~$z \in I$.
  Hence~$x$ is contained in the normalizer~$\normallie_\glie(I)$, but not contained in~$I$.
  This shows that~$I$ is properly contained in its normalizer.
  This normalizer is again a Lie~subalgebra of~$\glie$, and hence it follows from the maximality of~$\glie$ (among all proper Lie~subalgebras of~$\glie$) that~$\normallie_\glie(I) = \glie$.
  This means that~$I$ is an ideal in~$\glie$.
\end{proof}


\begin{proposition}
  \label{characterizations of linear lie algebras consisting of nilpotent endomorphisms}
  Let~$V$ be a vector space of finite dimension~$n \defined \dim V$ and let~$\glie$ be a Lie~subalgebra of~$\gllie(V)$.
  Then the following conditions on~$\glie$ are equivalent:
  \begin{enumerate}
    \item
      \label{engels g consists of nilpotent endomorphisms}
      The Lie~algebra~$\glie$ consists of nilpotent endomorphisms.
    \item
      \label{engels there exists a complete flag shifted by g}
      There exists a complete flag of~$V$,
      \[
        V
        =
        V_n
        \supsetneq
        V_{n-1}
        \supsetneq
        V_{n-2}
        \supsetneq
        \dotsb
        \supsetneq
        V_1
        \supsetneq
        V_0
        =
        0 \,,
      \]
      with~$x(V_i) \subseteq V_{i-1}$ for all~$x \in \glie$ and every~$i = 1, \dotsc, n$.
    \item
      \label{engels represented by strictly upper triangular matrices}
      There exists a basis of~$V$ with respect to which every~$x \in \glie$ is represented by an strictly upper triangular matrix.
  \end{enumerate}
\end{proposition}


\begin{proof}
  \leavevmode
  \begin{description}
    \item[\ref*{engels g consists of nilpotent endomorphisms}~$\implies$~\ref*{engels there exists a complete flag shifted by g}]
      This implication can be shown by induction over~$n = \dim V$.
      For~$n = 1$ let~$V_0 \defined 0$ and~$V_1 \defined V$.
      By assumption every~$x \in \glie$ acts nilpotent on~$V$, so~$x(V) = 0$ because~$V$ is {\onedimensional}.
      Thus
      \[
        V
        = 
        V_1
        \supsetneq
        V_0
        =
        0
      \]
      is a complete flag for~$V$ satisfying the condititons.
    
      Let now~$n \geq 2$ and suppose the statement holds for all smaller dimensions.
      Let~$v \in V$ be nonzero with~$x(v) = 0$ for every~$x \in \glie$.
      For~$W \defined V / \kf v$ every~$x \in \glie$ induces an endomorphism
      \[
        \induced{x}
        \colon
        W
        \to W \,,
        \quad
        v + \kf v
        \mapsto
        x(v) + \kf v \,.
      \]
      By induction assumption there exists a complete flag
      \[
        W
        =
        W_{n-1}
        \supsetneq
        W_{n-2}
        \supsetneq
        W_{n-3}
        \supsetneq
        \dotsb
        \supsetneq
        W_1
        \supsetneq
        W_0
        =
        0
      \]
      with~$\induced{x}(W_i) \subseteq W_{i-1}$ for every~$x \in \glie$ and~$i = 1, \dotsc, n-1$.
      By setting~$V_i \defined \pi^{-1}(W_{i-1})$ for every~$i = 1, \dotsc, n$ together with~$V_0 \defined 0$ it follows that
      \[
        V
        =
        V_n
        \supsetneq
        V_{n-1}
        \supsetneq
        V_{n-2}
        \supsetneq
        \dotsb
        \supsetneq 
        V_1
        \supsetneq
        V_0
        =
        0
      \]
      is a complete flag of~$V$.
      On the one hand~$x(V_1) = x(\kf v) = 0 = V_0$ for every~$x \in \glie$ and on the other hand
      \[
        \pi(x(V_i))
        =
        \overline{x}(\pi(V_i))
        =
        \overline{x}(W_{i-1})
        \subseteq
        W_{i-2} \,,
      \]
      and therefore~$x(V_i) \subseteq \pi^{-1}(W_{i-2}) = V_{i-1}$ for every~$i = 2, \dotsc, n$.
    \item[\ref*{engels there exists a complete flag shifted by g}~$\implies$~\ref*{engels represented by strictly upper triangular matrices}]
      This is linear algebra
    \item[\ref*{engels represented by strictly upper triangular matrices}~$\implies$~\ref*{engels g consists of nilpotent endomorphisms}]
      This is again linear algebra.
    \qedhere
  \end{description}
\end{proof}


\begin{theorem}[Engel]
  \index{Engel’s theorem}
  A finite dimensional Lie~algebra~$\glie$ is nilpotent if and only if all its elements are~{\adnilpotent}.
\end{theorem}


\begin{proof}
  If~$\glie$ is nilpotent then every~$x \in \glie$ is~{\adnilpotent} by \cref{properties of nilpotence}.
 
  Suppose on the other hand that~$\glie$ consists of~{\adnilpotent} elemenents.
  Then~$\glie' \defined \ad(\glie)$ is a Lie subalgebra of~$\gllie(\glie)$ consisting of nilpotent endomorphisms.
  The Lie~algebra~$\glie'$ can therefore, with respect to a suitable basis of~$\glie$, be represented by upper triangular matrices by \cref{characterizations of linear lie algebras consisting of nilpotent endomorphisms}
  This showws that~$\glie'$ is isomorphic to a Lie~subalgebra of some~$\nlie_n(\kf)$ (for~$n = \dim \glie$) and hence nilpotent.
  
  This shows that~$\glie/\centerlie(\glie) \cong \ad(\glie)$ is nilpotent.
  It follows that~$\glie$ is nilpotent by \cref{characterizations of linear lie algebras consisting of nilpotent endomorphisms}.
\end{proof}


\begin{remark}
  It is not true that every nilpotent Lie~subalgebra~$\glie$ of~$\gllie(V)$, where~$V$ is a finite dimensional vector space, can be represented by strictly upper triangular matrices with respect to some suitable basis of~$V$.
  An example for this is the {\onedimensional} Lie~subalgebra~$\glie \defined \kf {\id_V}$ of~$\gllie(V)$.
  This Lie algebra is abelian and hence nilpotent, but is with respect to every basis of~$V$ represented by the scalar matrices~$\kf I$.
\end{remark}


\begin{corollary}
  \label{ideals in nilpotent does intersection center}
  If~$I$ is a nonzero ideal in a finite dimensional nilpotent Lie~algebra then~$I \cap \centerlie(\glie) \neq 0$.
\end{corollary}


\begin{proof}
  The Lie~algebra~$\glie$ acts on itself via the adjoint representation and~$\ad(x)$ is nilpotent for every~$x \in \glie$ by Engel’s theorem.
  The ideal~$I$ is a subrepresentation of~$\glie$ and the restriction~$\restrict{\ad(x)}{I}$ is again nilpotent for every~$x \in \glie$.
  The image of the resulting Lie~algebra homomorphism
  \[
    \glie
    \to
    \gllie(I) \,,
    \quad
    x
    \mapsto
    \restrict{\ad(x)}{I}
  \]
  does therefore consists of nilpotent endomorphisms.
  By \cref{common eigenvector for nilpotent Lie algebras} there exists some non-zero~$y \in I$ with~$\restrict{\ad(x)}{I}(y) = 0$ for every~$y \in \glie$, i.e.~$[\glie, y] = 0$.
  Then~$y \in \centerlie(\glie) \cap I$.
\end{proof}





\subsection{Lie’s Theorem}


\begin{convention}
  From now on \emph{all} occuring fields are required to be algebraically closed and of characteristic~$0$, unless otherwise specified.
  This convention holds for the rest of these notes.
  In particular all occuring Lie~algebras and vector spaces will have such a field as their ground field, even if not explicitely stated.
\end{convention}


\begin{definition}
  Let~$V$ be a representation of a Lie~algebra~$\glie$.
  For every linear functional~$\lambda \in \glie^*$ the linear subspace~$\gls*{weight space}$ of~$V$ given by
  \[
    V_\lambda
    \defined
    \{
      v \in V
    \suchthat
      \text{$x.v = \lambda(x) v$ for every~$x \in \glie$}
    \}
  \]
  is the \defemph{weight space}\index{weight!space} of~$V$ with respect to~$\lambda$.
  An element~$\lambda \in \glie^*$ with~$V_\lambda \neq 0$ is a \defemph{weight}\index{weight} of~$V$.
\end{definition}


\begin{lemma}[Invariance lemma]
  Let~$V$ be a finite dimensional representation of a Lie~algebra~$\glie$ and let~$I$ be an ideal in~$\glie$.
  Then~$V$ is also a representation of~$I$ by restricting the action of~$\glie$ to~$I$.
  For~$\lambda \in I^*$ let~$V_\lambda$ be the~{\weightspace{$I$}} of~$V$ with respect to~$\lambda$.
  Then~$V_\lambda$ is already a~{\subrepresentation{$\glie$}} of~$V$.
\end{lemma}


\begin{proof}
  For~$v \in V$ and~$x_1, \dotsc, x_n \in \glie$ we will write
  \[
    x_1 \dotsm x_n v
    \defined
    x_1.(x_2.( \dotsc x_{n-1}.(x_n.v) \dotsc )) \,.
  \]
  The case~$V_\lambda = 0$ is welll understood, so for the rest of this proof we fix some~$\lambda \in I^*$ with~$V_\lambda \neq 0$.
 
  That~$V_\lambda$ is a {\subrepresentation{$\glie$}} of~$V$ means that~$yv \in V_\lambda$ for all~$y \in \glie$ and~$v \in V_\lambda$, which  means that~$xyv = \lambda(x)yv$ for all~$x \in I$,~$y \in \glie$ and~$v \in V_\lambda$.
  We see that
  \[
    xyv
    =
    [x,y]v + yxv
    =
    \lambda([x,y])v + \lambda(x)yv
  \]
  for all~$x \in I$,~$y \in \glie$ and~$v \in V_\lambda$.
  We hence need to show that~$\lambda([x,y]) = 0$ for all~$x \in I$ and~$y \in \glie$.
 
  Until further notice we fix some~$y \in \glie$ and some nonzero~$v \in V_\lambda$.
  As~$V$ is finite dimensional there exists some maximal~$n \geq 1$ such that the vectors~$v, yv, \dotsc, y^n v$ are linearly independent.
  Let
  \[
    W_i
    \defined
    \vspan_k(v, yv, \dotsc, y^i v)
  \]
  for every~$i = 0, \dotsc, n$.
  The linear space~$W_n$ is invariant under the action of~$y$ because the vector~$y^{n+1} v$ is linearly dependent on the vectors~$v, \dotsc, y^n v$.
 
  \begin{claim*}
    The linear subspaces~$W_1, \dotsc, W_n$ are~{\subrepresentations{$I$}} of~$V$.
    With respect to the basis~$w, y w, \dotsc, y^i w$ of~$W_i$ the action of any~$x \in I$ is represented by an upper triangular matrix whose every diagonal entry is~$\lambda(x)$.
  \end{claim*}

 \begin{proof}[Proof of the claim]
    We prove the claim by induction over~$i$.
    The claim holds for~$i = 0$ since~$W_0 = \kf v$ is spanned by the vector~$v$ which is a~{\weightvector{$\lambda$}} for the action of~$I$.
    Suppose now that~$i < n$ and that the claim holds for the linear subspaces~$W_0, \dotsc, W_i$.
    If~$x \in I$ then also~$[x,y] \in I$ and therefore
    \[
      x y^{i+1} v
      =
      \underbrace{[x,y] y^i v}_{\mathclap{\substack{\in W_i \\ \text{by induction}}}} + y x y^i v
      \equiv
      y x y^i v
      \mod
      W_i \,.
    \]
    We find by the induction hypothesis that~$x y^i v \in W_i$ because~$W_i$ is a subrepresentation, and also that this vector is of the form
    \[
      x y^i v
      \equiv
      \lambda(x) y^i v
      \mod
      W_{i-1} \,.
    \]
    It follows that
    \[
      y x y^i v
      \equiv
      \lambda(x) y^{i+1} v
      \mod W_i  \,,
    \]
    because~$y W_{i-1} \subseteq W_i$.
    This shows the claim for~$W_{i+1}$.
 \end{proof}
  
  Let~$x \in I$.
  It follows from the above claim that the~{\dimensional{$(n+1)$}} linear subspace~$W_n$ is invariant under the action of~$[x,y]$ because~$[x,y] \in I$.
  This action is given by an endomorphism~$\phi_{[x,y]} \in W_n \to W_n$, that is represented with respect to some suitable basis of~$W_n$ by an upper triangular matrix whose every diagonal entry is~$\lambda([x,y])$.
  It follows that in particular
  \begin{equation}
    \label{invariance lemma zero trace}
    \tr \phi_{[x,y]}
    =
    (n+1) \lambda([x,y])  \,.
  \end{equation}
  On the other hand the linear subspace~$W_n$ is invariant under the action of both~$x$ (by the claim because~$x \in I$) and under the action of~$y$ (as seen above before the claim).
  These two elements act by endomorphisms~$\phi_x$ and~$\phi_y$ on~$W_n$.
  Because~$V$ is a representation of the Lie~algebra~$\glie$ it follows that~$\phi_{[x,y]} = [\phi_x, \phi_y]$ and thus~$\tr \phi_{[x,y]} = 0$.
  Together with \eqref{invariance lemma zero trace} it follows that~$\lambda([x,y]) = 0$.
\end{proof}


\begin{theorem}[Lie]
  \index{Lie’s theorem}
  Let~$\glie$ be a solvable Lie~subalgebra of~$\gllie(V)$ for some nonzero finite dimensional~{\vectorspace{$\kf$}}. Then there exists a common eigenvector for~$\glie$, i.e.\ some nonzero~$v \in V$ with~$x(v) \in \kf v$ for every~$x \in \glie$.
\end{theorem}


\begin{proof}
  We show the statement by induction over the dimension~$n$ of~$\glie$.
  If~$n = 0$ then~$\glie = 0$ and any nonzero~$v \in V$ does the job.
  If~$n = 1$ then~$\glie = \kf x$ for some nonzero~$x \in \gllie(V)$.
  Then any eigenvector of~$x$ does the job (and such an eigenvector exists because the field~$\kf$ is assumed to be algebraically closed and~$V \neq 0$).
 
  Suppose that~$n \geq 2$ and that the statement holds for every smaller dimension.
  Similarly to the proof of \cref{common eigenvector for nilpotent Lie algebras} we will split the proof into four consecutive parts:
  \begin{itemize}
    \item
      Finding an ideal~$I$ in~$\glie$ of codimension~$1$.
    \item
      Finding a common eigenvectors for~$I$ by induction hypothesis.
    \item
      Showing that~$\glie$ stabilizes as nonzero subspace~$U \subseteq V$ of such eigenvectors.
    \item
      Writing~$\glie = I \oplus \kf y$ (as vector spaces) and finding an eigenvector of~$y$ in~$U$.
  \end{itemize}

  For the first step we note that~$[\glie,\glie]$ in~$\glie$ because~$\glie$ is nonzero and solvable.
  Hence~$\glie/[\glie,\glie]$ is a nonzero abelian Lie~algebra.
  Any linear subspace~$J \subseteq \glie/[\glie,\glie]$ of codimension~$1$ is then an ideal in~$\glie/[\glie,\glie]$.
  Hence the preimage~$I = \pi^{-1}(J)$ under the canonical projection~$\glie \to \glie/[\glie,\glie]$ is an ideal in~$\glie$ of codimension~$1$.
 
  For the second step we note~$I$ is solvable because it is a Lie~subalgebra of~$\glie$.
  So by induction hypothesis there exists a common eigenvector for~$I$.
  Hence there exists some~$\lambda \in I^*$ with~$U \defined V_\lambda \neq 0$, where~$V_\lambda$ denotes the~{\weightspace{$I$}} with respect to~$\lambda$.
  
  For the third step we apply the invariance lemma to see that~$U$ is a~{\subrepresentation{$\glie$}} of~$V$.
 
  For the fourth step we may write~$\glie = I \oplus \kf y$ for some~$y \in \glie$ because~$I$ has codimension~$1$.
  Then~$y$ stabilizs~$U$ and hence admits an eigenvector~$v \in U$, as~$\kf$ is algebraically closed.
  This vector~$v$ is then a common eigenvector for~$\glie$.
\end{proof} 


\begin{remark}
 The idea of organizing the proof into four steps is taken from~\cite[\S 4.1]{Humphreys}.
 This organizational structure also serves to emphasize the similarities with the proof of Proposition~\ref{common eigenvector for nilpotent Lie algebras}.
\end{remark}


\begin{proposition}
  \label{common eigenvector for solvable Lie algebras}
  Let~$\glie$ be a Lie~subalgebra of~$\gllie(V)$ where~$V$ is a finite dimensional~{\vectorspace{$\kf$}}.
  Then the following conditions on~$\glie$ are equivalent:
  \begin{enumerate}
    \item
      $\glie$ is solvable.
    \item
      $\glie$ stabilizes some complete flag of~$V$, i.e.\ there exists a complete flag
      \[
        V
        =
        V_n
        \supsetneq
        V_{n-1}
        \supsetneq
        V_{n-2}
        \supsetneq
        \dotsb
        \supsetneq
        V_1
        \supsetneq
        V_0
        =
        0 \,,
      \]
      with~$x(V_i) \subseteq V_i$ for every~$x \in \glie$ and~$i = 0, \dotsc, n$.
    \item
      There exists a basis of~$V$ with respect to which every~$x \in \glie$ is represented by an upper triangular matrix.
      In particular,~$\glie$ is isomorphic to a Lie~subalgebra of~$\tlie_n(k)$ for~$n = \dim V$.
    \qed
  \end{enumerate}
\end{proposition}


\begin{corollary}
  A finite dimensional Lie~algebra~$\glie$ is solvable if and only if~$[\glie, \glie]$ is nilpotent.
\end{corollary}


\begin{proof}
  If~$[\glie, \glie] = \glie^{(1)}$ is nilpotent then it is solvable, and hence~$\glie$ is solvable.
 
  Suppose on the other hand that the Lie~algebra~$\glie$ is solvable.
  Then~$\ad(\glie) \cong \glie/\centerlie(\glie)$ is a solvable Lie~subalgebra of~$\gllie(\glie)$.
  By Lie’s~theorem there exists a basis of~$\glie$ with respect to which~$\ad(x)$ is represented by an upper triangular matrix for every~$x \in \glie$.
  Hence every~$x \in [\glie, \glie]$ is represented by a strictly upper triangular matrix with respect to this basis because~$\ad$ is a homomorphism of Lie~algebras and~$[\tlie_n(\kf), \tlie_n(\kf)] = \nlie_n(\kf)$.
  Hence every~$x \in [\glie, \glie]$ is {\adnilpotent} (and thus also~$\ad_{[\glie,\glie]}$\nobreakdash-nilpotent).
  The Lie algebra~$[\glie, \glie]$ is hence nilpotent by Engel’s~theorem.
\end{proof}


\begin{corollary}
  \label{irreducible representations of solvable lie algebras are onedimenisonal}
  Every irreducible representation of a solvable Lie~algebra is {\onedimensional}.
\end{corollary}


\begin{proof}
  Let~$\rho \colon \glie \to \glie(V)$ be an irreducible representation of~$\glie$.
  Then~$V \neq 0$ and~$\glie' \defined \im \rho$ is a solvable Lie~subalgebra of~$\gllie(V)$.
  There exists a common eigenvector~$v$ for~$\glie'$ by Lie’s theorem.
  Then~$\kf v$ is a nonzero subrepresentation of~$V$ and it follows that~$V = \kf v$ because~$V$ is irreducible.
\end{proof}


\begin{remark}
 \Cref{irreducible representations of solvable lie algebras are onedimenisonal} is actually equivalent to Lie’s~theorem:
 If~$\glie$ is a Lie~subalgebra of~$\gllie(V)$ then~$V$ becomes a representation of~$\glie$ via~$x.v = x(v)$ for all~$x \in \glie$ and~$v \in V$. 
 If the vector space~$V$ is nonzero and finite dimensional then~$V$ contains an irreducible subrepresentation~$U$ for the above action of~$\glie$, as one can take any nonzero subrepresentation of minimal dimension.
 If the Lie algebra~$\glie$ is solvable then by \cref{irreducible representations of solvable lie algebras are onedimenisonal} the irreducible subrepresentation~$U$ is {\onedimensional}, and thus of the form~$U = \kf v$ for some nonzero~$v \in V$.
 From the definition of the action of~$\glie$ on~$V$ it follows that this vector~$v$ is common eigenvector for~$\glie$.
\end{remark}


\begin{remark}
  \Cref{irreducible representations of solvable lie algebras are onedimenisonal} does not hold for general fields~$\kf$, even if algebraically closed.
  To see this let~$\kf$ be an algebraically closed field with~$\ringchar \kf = 2$ and let~$\glie \defined \sllie_2(\kf)$.
  Recall that the standard basis of~$\glie$ is given by the matrices
  \begin{gather*}
    e
    =
    \begin{pmatrix}
      0 & 1 \\
      0 & 0
    \end{pmatrix} \,,
    \qquad
    h
    =
    \begin{pmatrix*}[r]
      1 &  0  \\
      0 & -1
    \end{pmatrix*},
    \qquad
    f
    =
    \begin{pmatrix}
      0 & 0 \\
      1 & 0
    \end{pmatrix}
  \shortintertext{with}
    [h,e] = 0 \,,
    \qquad
    [h,f] = 0 \,,
    \qquad
    [e,f] = h \,.
  \end{gather*}
  We find that the Lie algebra~$\glie$ is nilpotent (with~$\glie^2 = 0$) and in particular also solvable.
  Let~$V \defined \kf^2$ be the natural representation of~$\glie$, i.e.~$\glie$ acts on~$V$ by~$x.v = x(v)$ for all~$x \in \glie$ and~$v \in V$.
  Then
  \[
    e.\vect{x \\ y}
    =
    \vect{y \\ 0}
    \qquad\text{and}\qquad
    f.\vect{x \\ y}
    =
    \vect{0 \\ x} \,.
  \]
  It follows that if~$U \subseteq V$ is a nonzero subrepresentation then~$U$ contains one of the standard basis vectors~$e_1$,~$e_2$ of~$V$, and then also the other one.
  Hence~$U = V$, which shows that~$V$ is an irreducible representation of~$\glie$.
  But~$V$ is not {\onedimensional}.
\end{remark}
