\chapter{The Universal Enveloping Algebra}


\begin{convention}
  The following holds for this section alone:
  We fix an arbitrary field~$\kf$.
  By a~{\algebra{$\kf$}} we always mean an associative and unitary one, and homomorphisms of~{\algebras{$\kf$}} have to respect the units.
  The resulting category of~{\algebras{$\kf$}} with homomorphisms of~{\algebra{$\kf$}} between them will be denoted by~$\cAlg{\kf}$.
  All Lie~algebras considered will have~$\kf$ as their ground field, unless otherwise stated.
  
  If~$A$ is a~{\algebra{$\kf$}} then by an~\defemph{\module{$A$}}\index{module} we mean a left unitial~{\module{$A$}}.
  The resulting category of~{\modules{$A$}} is denoted by~\gls*{module category}.
\end{convention}





\section{Tensor Algebra and Symmetric Algebra}


\begin{recall}[Tensor algebra]
  Let~$V$ be a vector space.
  \begin{description}
    \item[Construction:]
      For all~$v_1, \dotsc, v_d \in V$ we denote the resulting simple tensor~$v_1 \tensor \dotsb \tensor v_d$ in~$V^{\tensor d}$ by~$(v_1, \dotsc, v_d)$.
      Observe that for~$d = 0$ the tensor power~$V^{\tensor d} = V^{\tensor 0}$ has as a basis the emtpy simple tensor~$()$.
      We will therefore identify the tensor power~$V^{\tensor 0}$ with the ground field~$\kf$, so that empty simple tensor~$()$ corresponds to~$1 \in \kf$.
      
      For all~$p, q \geq 0$ we define a multiplication
      \[
        \mu_{p,q}
        \colon
        V^{\tensor p} \times V^{\tensor q}
        \to
        V^{\tensor (p+q)} \,,
        \quad
        (x,y)
        \mapsto
        x y
      \]
      that is on simple tensors~$(v_1, \dotsc, v_p) \in V^{\tensor p}$ and~$(v_{p+1}, \dotsc, v_{p+q}) \in V^{\tensor q}$ given by
      \[
        (v_1, \dotsc, v_p) \cdot (v_{p+1}, \dotsc, v_{p+q})
        =
        (v_1, \dotsc, v_{p+q})  \,.
      \]
      Note that for~$p = 0$ or~$q = 0$ this multiplication is just scalar multiplication.  
      These multiplications fit together associatively in the sense that for all~$p, q, r \geq 0$ and simple tensors~$x \in V^{\tensor p}$,~$y \in V^{\tensor q}$ and~$z \in V^{\tensor r}$ the equality
      \[
        x \cdot (y \cdot z)
        =
        (x \cdot y) \cdot z
      \]
      holds.
      
      Let~$\Tensor(V) \defined \bigoplus_{d \geq 0} V^{\tensor d}$.
      We can fit together the multiplications~$\mu_{p,q}$ with~$p, q \geq 0$ to a single multiplication
      \[
        \mu
        \colon
        \Tensor(V) \times \Tensor(V)
        \to
        \Tensor(V)  \,,
        \quad
        (x,y)
        \mapsto
        xy 
      \]
      that is given on elements~$x, y \in \Tensor(V)$ with~$x = (x_d)_{d \geq 0}$ and~$y = (y_d)_{d \geq 0}$ by
      \[
        x y
        =
        \left(
          \sum_{p+q = d} x_p y_q
        \right)_{d \geq 0} \,.
      \]
      This multiplication is built precisely so that it follows from the bilinearity of the multiplications~$\mu_{p,q}$ that the multipliation~$\mu$ is again bilinear.
      It follows from the associativities of the multiplications~$\mu_{p,q}$ that the multiplication~$\mu$ is associative.
      We may identify the ground field~$\kf = V^{\tensor 0}$ with the corresponding direct summand in~$\Tensor(V)$ to regard~$\kf$ as a linear subspace of~$\Tensor(V)$.
      We have seen above that~$1 \in \kf$ is then unital for the multiplication of~$\Tensor(V)$.
      We have thus altogether constructed a~{\algebra{$\kf$}}~$\Tensor(V)$.
      
      We may identify~$V = V^{\tensor 1}$ with the corresponding direct summand of~$\Tensor(V)$ to regard~$V$ as a linear subspace of~$\Tensor(V)$.
      We then have for all~$v_1, \dotsc, v_n \in V$ that
      \[
        v_1 \dotsm v_n
        =
        (v_1) \dotsm (v_n)
        =
        (v_1, \dotsc, v_n)
        =
        v_1 \tensor \dotsb \tensor v_n  \,.
      \]
      It follows in particular that~$\Tensor(V)$ is then generated by~$V$ as an algebra.
      The algebra~\gls*{tensor algebra} is the \defemph{tensor algebra of~$V$}
      
      We will more generally identify for all~$d \geq 0$ the tensor power~$V^{\tensor d}$ with the corresponding summand in~$\Tensor(V)$.
      The tensor algebra~$\Tensor(V)$ hence consists of linear combinations simple tensors~$v_1 \tensor \dotsb \tensor v_n$.
    
    \item[Universal Property:]
      The tensor algebra~$\Tensor(V)$ can be though of as the \enquote{free~{\algebra{$\kf$}} on~$V$}, in at least two ways:
      \begin{itemize}
        \item(Informal)
          The tensor algebra~$\Tensor(V)$ arises from~$V$ by starting with the elements of~$V$ and adding to~$V$ all kinds of expressions that can be constructed from the elements of~$V$ by algebra operations.
          But it follows from the axioms that many of these expressions have to be the same, so that we only end up with expressions of a certain form.
          
          Let us be a bit more explicit:
          Suppose that a~{\algebra{$\kf$}}~$A$ contains~$V$ as a linear subspace.
          Then it also contains products of the form~$v_1 \dotsm v_n$ with~$v_i \in V$ and hence sums of such products, i.e.\ elements of the form
          \[
            \sum_{i=k}^r v_{i_1} \dotsm v_{i_{n_k}}
          \]
          with~$r \geq 0$ and~$v_{ij} \in V$.
          If we continue to combine elements of this form with algebra operations then we do not gain any new elements, since by the axioms of an algebra they must already be of the above form.
          
          But in an arbitrary~{\algebra{$\kf$}} it may happen that some of these expressions are equal even though this does not follow pureley from the axioms of a~{\algebra{$\kf$}}.
          Consider for example the polynomial ring~$A = \kf[x, y]$ and the linear subspace~$V = \kspan(x, y)$.
          It follows from the axioms of a~{\algebra{$\kf$}} that the expressions~$x (x+y)$ and~$x^2 + xy$ are the same, but it does not follow just from the axioms that~$xy = yx$, even though this holds in~$A$.
          There are hence certain additional \emph{relations} between the elements~$x$ and~$y$ of~$V$ in the ambient {\algebra{$\kf$}}~$A$.
          
          In the tensor algebra~$\Tensor(V)$ this does not happen:
          Whenever two expressions~$x$ and~$y$ that are built from elements of~$V$ via algebra operations coincide, then this equality can be derived from the algebra axioms alone.
          Hence there exist no additional relations between the elements of~$V$ in~$\Tensor(V)$.
          The only required condition is that~$V$ is a linear subspace of~$\Tensor(V)$, i.e.\ that addition and scalar multiplication in~$V$ does coincide with the one coming from~$\Tensor(V)$.
          
          The tensor algebra~$\Tensor(V)$ is in this way the \enquote{freest} way of expanding~$V$ into a~{\algebra{$\kf$}}.
        \item(Formal)
          Let~$\iota \colon V \to \Tensor(V)$ be the inclusion, which is a~{\linear{$\kf$}} map.
          Then if~$A$ is any other~{\algebra{$\kf$}} and~$f \colon V \to A$ any~{\linear{$\kf$}} map (which one can think of as somewhat of an inclusion, albeit not injective), then~$f$ extends uniquely to an algebra homomorphism~$f^+ \colon \Tensor(V) \to A$, i.e.\ there exists a unique algebra homomorphism~$f^+ \colon \Tensor(V) \to A$ that makes the triangular diagram
          \[
            \begin{tikzcd}
              V
              \arrow{r}[above]{f}
              \arrow{d}[left]{i}
              &
              A
              \\
              \Tensor(V)
              \arrow[dashed]{ur}[below right]{f^+}
              &
              {}
            \end{tikzcd}
          \]
          commute.
          The algebra homomorphism~$f^+$ is given by
          \[
            f^+(v_1 \tensor \dotsb \tensor v_d)
            =
            f(v_1) \dotsm f(v_d)
          \]
          for all~$d \geq 0$ and simple tensors~$v_1 \tensor \dotsb \tensor v_d \in V^{\tensor d}$.
          This construction results in a {\onetoone} correspondence
          \begin{align*}
            \{ \text{\linear{$\kf$} maps~$V \to A$} \}
            &\longonetoone
            \{ \text{algebra homomorphisms~$\Tensor(V) \to A$} \} \,,
            \\
            f
            &\longmapsto
            f^+ \,,
            \\
            \restrict{F}{V}
            &\longmapsfrom
            F \,.
          \end{align*}
          Hence~$(\Tensor(V), i)$ is the \enquote{universal way} of mapping the vector space~$V$ into a~{\algebra{$\kf$}}.
          
          This formal explanation relates to the previous informal explanation in the following way:
          If~$A$ is any~{\algebra{$\kf$}} that contains~$V$ as a linear subspace then the inclusion~$V \to A$ extend uniquely to an algebra homomorphism~$\Tensor(V) \to A$.
          Every relation between expressions built from the elements of~$V$ that holds in~$\Tensor(V)$ must then also hold in~$A$.
          Therefore the only relations that hold in~$\Tensor(V)$ between such expressions are the one that hold in \emph{every}~{\algebra{$\kf$}} containing~$V$.
      \end{itemize}
      
    \item[Uniqueness]
      The above universal property determines the tensor algebra~$\Tensor(V)$ together with the inclusion~$i \colon V \to \Tensor(V)$ uniquely up to unique isomorphism, in the following sense:
      Let~$A$ be another~{\algebra{$\kf$}} and~$j \colon V \to T$ a~{\linear{$\kf$}} map such that for every~{\algebra{$\kf$}}~$A$ and every~{\linear{$\kf$}}~$f \colon V \to A$ there exists a unique algebra homomorphism~$F \colon T \to A$ that makes the triangular diagram
      \[
        \begin{tikzcd}
          V
          \arrow{r}[above]{f}
          \arrow{d}[left]{j}
          &
          A
          \\
          T
          \arrow{ur}[below right]{F}
          &
          {}
        \end{tikzcd}
      \]
      commute.
      
      Then there exist unique algebra homomorphisms~$f \colon A \to T$ and~$g \colon T \to A$ that make the triangular diagrams
      \[
        \begin{tikzcd}[column sep = small]
          {}
          &
          V
          \arrow{dl}[above left]{i}
          \arrow{dr}[above right]{j}
          &
          {}
          \\
          \Tensor(V)
          \arrow[dashed]{rr}[below]{f}
          &
          {}
          &
          T
        \end{tikzcd}
        \qquad\text{and}\qquad
        \begin{tikzcd}[column sep = small]
          {}
          &
          V
          \arrow{dl}[above left]{j}
          \arrow{dr}[above right]{i}
          &
          {}
          \\
          T
          \arrow[dashed]{rr}[below]{g}
          &
          {}
          &
          \Tensor(V)
        \end{tikzcd}
      \]
      commute.
      It then follows that the compositions~$g \circ f \colon \Tensor(V) \to \Tensor(V)$ and~$f \circ g \colon T \to T$ make the triangular diagrams
      \[
        \begin{tikzcd}[column sep = small]
          {}
          &
          V
          \arrow{dl}[above left]{i}
          \arrow{dr}[above right]{i}
          &
          {}
          \\
          \Tensor(V)
          \arrow[dashed]{rr}[below]{g \circ f}
          &
          {}
          &
          \Tensor(V)
        \end{tikzcd}
        \qquad\text{and}\qquad
        \begin{tikzcd}[column sep = small]
          {}
          &
          V
          \arrow{dl}[above left]{j}
          \arrow{dr}[above right]{j}
          &
          {}
          \\
          T
          \arrow[dashed]{rr}[below]{f \circ g}
          &
          {}
          &
          T
        \end{tikzcd}
      \]
      commute.
      The algebra homomorphisms~$g \circ f$ and~$f \circ g$ are unique with this propert by the universal properties of~$(\Tensor(V), i)$ and~$(T, j)$.
      But the identities~$\id_{\Tensor(V)}$ and~$\id_T$ also make these diagrams commute.
      We therefore find that~$g \circ f = \id_{\Tensor(V)}$ and~$f \circ g = \id_{\Tensor(V)}$, so that~$f$ and~$g$ are mutually inverse algebra isomorphisms.
    
    \item[Functoriality:]
      If~$f \colon V \to W$ is any~{\linear{$\kf$}} map then we can consider the following diagram:
      \[
        \begin{tikzcd}
          V
          \arrow{r}[above]{f}
          \arrow{d}
          &
          W
          \arrow{d}
          \\
          \Tensor(V)
          &
          \Tensor(W)
        \end{tikzcd}
      \]
      By applying the universal property of the tensor algebra~$\Tensor(V)$ to the composition~$V \to W \to \Tensor(W)$ it follows that there exists a unique algebra homomorphism~$f_* \colon \Tensor(V) \to \Tensor(W)$ that makes the square diagram
      \[
        \begin{tikzcd}
          V
          \arrow{r}[above]{f}
          \arrow{d}
          &
          W
          \arrow{d}
          \\
          \Tensor(V)
          \arrow[dashed]{r}[below]{f_*}
          &
          \Tensor(W)
        \end{tikzcd}
      \]
      commute.
      This induced algebra homorphism is functorial in the following sense:
      \begin{itemize}
        \item
          It holds that~$(\id_V)_* = \id_{\Tensor(V)}$.
          Indeed, it follows from the commutativity of the square 
          \[
            \begin{tikzcd}[column sep = large]
              V
              \arrow{r}[above]{f}
              \arrow{d}
              &
              V
              \arrow{d}
              \\
              \Tensor(V)
              \arrow[dashed]{r}[below]{(\id_V)_*}
              &
              \Tensor(V)
            \end{tikzcd}
          \]
          that the identity~$\id_{\Tensor(V)}$ satifies the defining property of the induced algebra homomorphism~$(\id_V)_*$.
        \item
          It holds for all linear maps~$f \colon U \to V$ and~$g \colon V \to W$ that~$(g \circ f)_* = g_* \circ f_*$.
          Indeed, it follows from the commutativity of the diagram
          \[
            \begin{tikzcd}
              U
              \arrow[dashed, bend left=45]{rr}[above]{g \circ f}
              \arrow{r}[above]{f}
              \arrow{d}
              &
              V
              \arrow{r}[above]{g}
              \arrow{d}
              &
              W
              \arrow{d}
              \\
              \Tensor(U)
              \arrow{r}[below]{f_*}
              \arrow[dashed, bend right=45]{rr}[below]{g_* \circ f_*}
              &
              \Tensor(V)
              \arrow{r}[below]{g_*}
              &
              \Tensor(W)
            \end{tikzcd}
          \]
          that the subdiagram
          \[
            \begin{tikzcd}[column sep = large]
              U
              \arrow{r}[above]{g \circ f}
              \arrow{d}
              &
              W
              \arrow{d}
              \\
              \Tensor(U)
              \arrow[dashed]{r}[below]{g_* \circ f_*}
              &
              \Tensor(W)
            \end{tikzcd}
          \]
          commutes.
          This shows that the composition~$g_* \circ f_*$ satisfies the defining property of the induced algebra homomorphism~$(g \circ f)_*$.
      \end{itemize}
      
      This shows that the assignment~$V \mapsto \Tensor(V)$ extends to a (covariant) functor~$\Tensor \colon \cVect{\kf} \to \cAlg{\kf}$.
      The universal property of the tensor algebra states that the functor~$\Tensor$ is left adjoint to the forgetful functor~$\cAlg{\kf} \to \cVect{\kf}$ that assigns to each~{\algebra{$\kf$}} its underlying~{\vectorspace{$\kf$}}.
    
    \item[Description via a basis:]
      If a basis~$(v_i)_{i \in I}$ of~$V$ is choosen then every tensor power~$V^{\tensor d}$ inherits a basis that is given by all simple tensors
      \[
        v_{i_1} \tensor \dotsb \tensor v_{i_d}
      \]
      with~$i_1, \dotsc, i_d \in I$.
      It follows that the tensor power has as a basis of all such simple tensors with~$d \geq 0$ and~$i_{i_1, \dotsc, i_d} \in I$.
      The product of two such basis vectors is again a basis vector.
      So we may think about the basis vectors as finite words~$i_1 \dotsm i_d$ in the alphabet~$I$, and as the multiplication of two basis vectors as the concatenation of the corresponding words.
      
      If we think about the basis vector~$v_i$ of~$V$ as a formal variable~$X_i$ then we see that the tensor algebra~$\Tensor(V)$ is isomorphic to the noncommutative polynomial ring~$\kf\gen{X_i \suchthat i \in I}$.
      This noncommutative polynomial ring is also the free~{\algebra{$\kf$}} on the generators~$X_i$ with~$i \in I$, while~$V$ is the free~{\vectorspace{$\kf$}} on the letters~$i \in I$.
      This gives another explanation for why~$\Tensor(V)$ is the free~{\algebra{$\kf$}} on the vector space~$V$.
      More exicitely, we have the following commutative diagram of forgetful functors:
      \[
        \begin{tikzcd}
          \cVect{\kf}
          \arrow{d}
          &
          \cAlg{\kf}
          \arrow{l}
          \arrow{dl}
          \\
          \cSet
          &
          {}
        \end{tikzcd}
      \]
      It then follows that the resulting diagram of left adjoint functors
      \[
        \begin{tikzcd}
          \cVect{\kf}
          \arrow{r}[above]{\Tensor}
          &
          \cAlg{\kf}
          \\
          \cSet
          \arrow{u}[left]{F}
          \arrow{ur}[below right]{\kf\gen{X_i \suchthat i \in (-)}}
          &
          {}
        \end{tikzcd}
      \]
      commutes up to natural isomorphism.
      Hence~$\Tensor(V) \cong \Tensor(F(I)) \cong \kf\gen{X_i \suchthat i \in I}$.
  \end{description}
\end{recall}


\begin{recall}[Symmetric algebra]
  Let~$V$ be a vector space.
  Just as the tensor algebra~$\Tensor(V)$ is the free~{\algebra{$\kf$}} on~$V$ and can be constructed by using the tensor powers~$V^{\tensor d}$ we can use the symmetric powers~$\Symm^d(V)$ to construct the \defemph{symmetric algebra}\index{symmetric algebra}~\gls*{symmetric algebra}.
  The argumentation is analogous to that for the tensor algebra, so we will skip some of the details this time.
  
  \begin{description}
    \item[Construction:]
      For all~$v_1, \dotsc, v_d \in V$ we denote the corresponding simple symmetric tensor in~$\Symm^d(V)$ by~$v_1 \dotsm v_d$.
      We can define on~$\Symm(V) \defined \bigoplus_{d \geq 0} \Symm^d(V)$ a multiplication such that
      \[
        (v_1 \dotsm v_p) \cdot (v_{p+1} \dotsm v_{p+q})
        =
        v_1 \dotsm v_p v_{p+1} \dotsm v_{p+q}
      \]
      for all~$p, q \geq 0$ and all simple symmetric tensors~$v_1 \dotsm v_p \in \Symm^p(V)$ and~$v_{p+1}, \dotsc, v_{p+q} \in \Symm^q(V)$.
      By identifying~$\Symm^0(V)$ with the ground field~$\kf$ this makes~$\Symm(V)$ into an associative~{\algebra{$\kf$}}.
      This is already a commutative~{\algebra{$\kf$}} because
      \[
        (v_1 \dotsm v_p) \cdot (v_{p+1} \dotsm v_{p+q})
        =
        v_1 \dotsm v_p v_{p+1} \dotsm v_{p+q}
        =
        v_{p+1} \dotsm v_{p+q} v_1 \dotsm v_p
        =
        (v_{p+1} \dotsm v_{p+q}) \cdot (v_1 \dotsm v_p)
      \]
      for all~$p, q \geq 0$ and all simple symmetric tensors~$v_1 \dotsm v_p \in \Symm^p(V)$ and~$v_{p+1}, \dotsc, v_{p+q} \in \Symm^q(V)$. 
      We can identify~$V = \Symm^1(V)$ with the corresponding direct summand of~$\Symm(V)$, and more generally every symmetric power~$\Symm^d(V)$ with the corresponding direct summand of~$\Symm(V)$.
      The algebra~$\Symm(V)$ thus consists of linear combinations of simple symmetric tensors.
      
    \item[Universal property:]
      The symmetric algebra~$\Symm(V)$ can be thought of as the \enquote{free commutative~{\algebra{$\kf$}}}:
      If~$i \colon V \to \Symm(V)$ is the inclusion then there exists for every~{\algebra{$\kf$}}~$A$ and every linear map~$f \colon V \to A$ a unique algebra homomorphism~$f^+ \colon \Symm(V) \to A$ that makes the triangular diagram
      \[
        \begin{tikzcd}
          V
          \arrow{r}[above]{f}
          \arrow{d}[left]{i}
          &
          A
          \\
          \Symm(V)
          \arrow{ur}[below right]{f^+}
          &
          A
        \end{tikzcd}
      \]
      commute.
      The algebra homomorphism~$f^+$ is given by
      \[
        f^+(v_1 \dotsm v_d)
        =
        f(v_1) \dotsm f(v_d)
      \]
      for all~$d \geq 0$ and simple symmetric tensors~$v_1, \dotsc, v_d \in V$.
      This construction results in a {\onetoone} correspondence
      \begin{align*}
        \{ \text{\linear{$\kf$} maps~$V \to A$} \}
        &\longonetoone
        \{ \text{algebra homomorphisms~$\Symm(V) \to A$} \} \,,
        \\
        f
        &\longmapsto
        f^+ \,,
        \\
        \restrict{F}{V}
        &\longmapsfrom
        F \,.
      \end{align*}
      
      It follows that a relations between elements of~$V$ holds in the symmetric algebra~$\Symm(V)$ if and only if it holds in every commutative algebra that contains~$V$.
      
    \item[Uniqueness]
      If~$S$ is a commutative~{\algebra{$\kf$}} and~$j \colon V \to S$ is a~{\linear{$\kf$}} such that~$(S, j)$ satisfies the same universal property as the symmetric algebra~$(\Symm(V), i)$  then there exists unique algebra homomorphisms~$f \colon \Symm(V) \to S$ and~$g \colon S \to \Symm(V)$ that make the triangular diagrams
      \[
        \begin{tikzcd}[column sep = small]
          {}
          &
          V
          \arrow{dl}[above left]{i}
          \arrow{dr}[above right]{j}
          &
          {}
          \\
          \Symm(V)
          \arrow[dashed]{rr}[below]{f}
          &
          {}
          &
          S
        \end{tikzcd}
        \qquad\text{and}\qquad
        \begin{tikzcd}[column sep = small]
          {}
          &
          V
          \arrow{dl}[above left]{j}
          \arrow{dr}[above right]{i}
          &
          {}
          \\
          S
          \arrow[dashed]{rr}[below]{g}
          &
          {}
          &
          \Symm(V)
        \end{tikzcd}
      \]
      commute.
      Then~$f$ and~$g$ are mutually inverse algebra isomorphisms.
      
    \item[Functoriality]
      For every linear map~$f \colon V \to W$ there exists a unique induced algebra homomorphism~$f_* \colon \Symm(V) \to \Symm(W)$ that makes the square diagram
      \[
        \begin{tikzcd}
          V
          \arrow{r}[above]{f}
          \arrow{d}
          &
          W
          \arrow{d}
          \\
          \Symm(V)
          \arrow[dashed]{r}[below]{f_*}
          &
          \Symm(W)
        \end{tikzcd}
      \]
      commmute.
      It holds that~$(\id_V)_* = \id_{\Symm(V)}$ and it holds for all composable~{\linear{$\kf$}} maps~$f \colon U \to V$ and~$g \colon V \to W$ that~$(g \circ f)_* = g_* \circ f_*$.
      This construction promotes the assignment~$V \mapsto \Symm(V)$ to a (covariant) functor~$\Symm \colon \cVect{\kf} \to \cCAlg{\kf}$, where~$\cCAlg{\kf}$ denotes the category of commutative~{\algebras{$\kf$}}.
      
    \item[Description via a basis]
      If~$(v_i)_{i \in I}$ is a basis of$~V$ where~$(I, \leq)$ is a linearly ordered set then the symmetric power~$\Symm^d(V)$ inherits a basis that is given by all simple symmetric tensors
      \[
        v_{i_1} \dotsm v_{i_d}
        \qquad
        \text{where~$i_1 \leq \dotsb \leq i_d$} \,.
      \]
      It follows that the symmetric algebra~$\Symm(V)$ has as a basis all simple symmetric tensors~$v_{i_1} \dotsm v_{i_d}$ with~$d \geq 0$ and~$i_1, \dotsc, i_d \in I$ with~$i_1 \leq \dotsb \leq i_d$.
      We see that the symmetric algebra~$\Symm(V)$ is isomorphic to the commutative polynomial ring~$k[X_i \suchthat i \in I]$, which is the free commutative~{\algebra{$\kf$}} on the generators~$i \in I$
      This can again be explained by considering the commutative diagram of forgetful functors
      \[
        \begin{tikzcd}
          \cVect{\kf}
          \arrow{d}
          &
          \cCAlg{\kf}
          \arrow{l}
          \arrow{dl}
          \\
          \cSet
          &
          {}
        \end{tikzcd}
      \]
      from which we see that the resulting diagram of left adjoints
      \[
        \begin{tikzcd}
          \cVect{\kf}
          \arrow{r}[above]{\Symm}
          &
          \cCAlg{\kf}
          \\
          \cSet
          \arrow{u}[left]{F}
          \arrow{ur}[below right]{\kf[X_i \suchthat i \in (-)]}
          &
          {}
        \end{tikzcd}
      \]
      commutes up to natural isomorphism.
      
    \item[Contruction via the tensor algbra]
      The symmetric algebra~$\Symm(V)$ can also be constructed as a quotient of the tensor algebra~$\Tensor(V)$.
      We give multiple ways how to see and think about this.
      Let in the following~$i \colon V \to \Tensor(V)$ and~$j \colon V \to \Symm(V)$ denote the 
      \begin{itemize}
        \item
          Let~$I$ be the commutator ideal of~$\Tensor(V)$, i.e.\ the two-sided ideal generated by all commutators
          \[
            x \tensor y - y \tensor x
          \]
          with~$x, y \in \Tensor(V)$.
          Let~$\pi \colon \Tensor(V) \to \Tensor(V)/I$ be the canonical projection.
          Then the quotient~$\Tensor(V)/I$ is commutative, and hence there exists by the universal property of the symmetric algebra a unique algebra homomorphism~$f \colon \Symm(V) \to \Tensor/I$ that makes the diagram
          \[
            \begin{tikzcd}
              {}
              &
              V
              \arrow[bend right]{ddl}[above left]{i}
              \arrow[bend left]{dr}[above right]{j}
              &
              {}
              \\
              {}
              &
              {}
              &
              \Tensor(V)
              \arrow{d}[right]{\pi}
              \\
              \Symm(V)
              \arrow[dashed]{rr}[above]{f}
              &
              {}
              &
              \Tensor(V)/I
            \end{tikzcd}
          \]
          commute.
          The homomorphism~$f$ is on the genareting set~$V$ of~$\Symm(V)$ given by~$f(v) = \class{v}$.
          On the other hand we get from the universal property of the tensor algebra~$\Tensor(V)$ a unique algebra homomorphism~$\tilde{g} \colon \Tensor(V) \to \Symm(V)$ that makes the diagram
          \[
            \begin{tikzcd}
              {}
              &
              V
              \arrow[bend right]{dl}[above left]{j}
              \arrow[bend left]{ddr}[above right]{i}
              &
              {}
              \\
              \Tensor(V)
              \arrow[bend left, dashed]{drr}[above right]{\tilde{g}}
              \arrow{d}[left]{\pi}
              &
              {}
              &
              {}
              \\
              \Tensor(V)/I
              &
              {}
              &
              \Symm(V)
            \end{tikzcd}
          \]
          commute.
          The commutator~$I$ is contained in the kernel of~$\tilde{g}$ because the algebra~$\Symm(V)$ is commutative.
          Hence there exists a unique algebra homomorphism~$g \colon \Tensor(V)/I \to \Symm(V)$ that makes the diagram
          \[
            \begin{tikzcd}
              {}
              &
              V
              \arrow[bend right]{dl}[above left]{j}
              \arrow[bend left]{ddr}[above right]{i}
              &
              {}
              \\
              \Tensor(V)
              \arrow[bend left]{drr}[above right]{\tilde{g}}
              \arrow{d}[left]{\pi}
              &
              {}
              &
              {}
              \\
              \Tensor(V)/I
              \arrow[dashed]{rr}[below]{g}
              &
              {}
              &
              \Symm(V)
            \end{tikzcd}
          \]
          commute.
          The algebra homomorphism~$g$ is given on the generators~$\class{v}$ with~$v \in V$ of~$\tensor(V)/I$ given by~$g(\class{v}) = v$.
          
          It follows from the explicit descriptions of~$f$ and~$g$ on generators that their are mutually inverse algebra isomorphisms.
          Thus~$\Symm(V) \cong \Tensor(V)/I$ via the isomorphism~$f$.
          
          Observe also that the commutator ideal~$I$ is already generated by the commutators~$v \tensor w - w \tensor v$ with~$v, w \in V$.
          Indeed, the ideal~$J$ generated by these elements is contained in~$I$.
          But on the other hand the quotient~$\Tensor(V)/J$ is already commutative because it is generated by the residue classes~$\class{v}$ with~$v \in V$, all of which commute with each other.
          The commutator ideal~$I$ is therefore contained in the kernel of the canonical projection~$\Tensor(V) \to \Tensor(V)/J$, i.e.\ it is containted in~$J$.
          
        \item
          The above argumentatio is not surprising if we remember that~$\Tensor(V)$ is the universal~{\algebra{$\kf$}} on~$V$ and that quotiening out the commutator ideal is the universal way of making an algebra commutative.
          The quotient~$\Tensor(V)/I$ therefore ought to be the universal commutative~{\algebra{$\kf$}}.
          
          This motivation can be formalized by observing that the diagram of forgetful functors
          \[
            \begin{tikzcd}
              \cAlg{\kf}
              \arrow{d}
              &
              \cCAlg{\kf}
              \arrow{l}
              \arrow{dl}
              \\
              \cVect{\kf}
              &
              {}
            \end{tikzcd}
          \]
          commutes.
          It follows that the resulting diagram of left adjoints
          \[
            \begin{tikzcd}
              \cAlg{\kf}
              \arrow{r}[above]{C}
              &
              \cCAlg{\kf}
              \\
              \cVect{\kf}
              \arrow{u}[left]{\Tensor}
              \arrow{ur}[below right]{\Symm}
              &
              {}
            \end{tikzcd}
          \]
          commutes up to natural isomorphism.
          The adjoint~$C$ of the forgetful functor~$\cCAlg{\kf} \to \cAlg{\kf}$ is given by quotiening out the commutator ideal, and hence~$\Symm(V) \cong \Tensor(V)/I$ as before.
        \item
          The above argumentation can also be formulized by observing that for every commutative~{\algebra{$\kf$}}~$A$ there exist natural bijections
          \begin{align*}
            {}&
            \{ \text{algebra homomorphisms~$\Symm(V) \to A$} \}
            \\
            \cong{}&
            \{ \text{{\linear{$\kf$}} maps~$V \to A$} \}
            \\
            \cong{}&
            \{ \text{algebra homomorphisms~$\Tensor(V) \to A$} \}
            \\
            \cong{}&
            \{ \text{algebra homomorphisms~$\Tensor(V)/I \to A$} \} \,,
          \end{align*}
          where the last bijection uses that the algebra~$A$ is commutative and therefore every algebra homomorphism~$\Tensor(V) \to A$ contains the commutator ideal~$I$ in its kernel.
          It now follows from Yoneda’s~lemma that~$\Symm(V) \cong \Tensor(V)/I$.
      \end{itemize}
  \end{description}
\end{recall}


\begin{remark}
  One can similarly construct the \emph{exterior algebra}~$\gls*{exterior algebra} = \bigoplus_{d \geq 0} \Exterior^d(V)$ of a vector space~$V$ by replacing the use of the tensor powers~$V^{\tensor d}$ or symmetric powers~$\Symm^d(V)$ by the exterior powers~$\Exterior^d(V)$.
  For any other~{\algebra{$\kf$}}~$A$ an algebra homomorphism~$F \colon \Exterior(V) \to A$ is then the same as a~{\linear{$\kf$}}~$f \colon V \to A$ with~$f(v)^2 = 0$ for every~$v \in V$.
% TODO: Do we have to worry about char(k) = 2?
  It thus follows from a similar argumentation as for the symmetric algebra that~$\Exterior(V) \cong \Tensor(V)/I$ for the two-sided ideal~$I$ in~$\Tensor(V)$ generated by all~$v \tensor v$ with~$v \in V$.
  
  
  If~$V$ is finite dimensional then the exterior algebra~$\Exterior(V)$ is again finite dimensional, namely with~$\dim \Exterior(V) = 2^{\dim V}$.
  This is different to both the tensor algebra~$\Tensor(V)$ and symmetric algebra~$\Symm(V)$ which are infinite dimensional whenever~$V \neq 0$.
\end{remark}






\section{Definition and Construction}


\begin{definition}
  A \defemph{universal enveloping algebra}\index{universal enveloping algebra} of a Lie~algebra~$\glie$ is a~\algebra{$\kf$}~\gls*{universal enveloping algebra} together with a homomorphism of Lie~algebras~$\iota \colon \glie \to \Univ(\glie)$ such that for every other~{\algebra{$\kf$}}~$A$ and homomorphism of Lie~algebras~$\phi \colon \glie \to A$ there exists a unique homomorphism of~\algebras{$\kf$}~$\Phi \colon \Univ(\glie) \to A$ that makes the triangular diagram
  \[
    \begin{tikzcd}
      \glie
      \arrow{r}[above]{\phi}
      \arrow{d}[left]{\iota}
      &
      A
      \\
      \Univ(\glie)
      \arrow[dashed]{ur}[below right]{\Phi}
      &
      {}
    \end{tikzcd}
  \]
  commute, i.e.\ such that~$\phi = \Phi \circ \iota$.
\end{definition}


\begin{remark}[Uniqueness of universal enveloping algebras]
  \label{uniqueness of universal enveloping algebras}
  Suppose that a Lie~algebra~$\glie$ admits two {\uas}~$(\Univ(\glie)_1, \iota_1)$ and~$(\Univ(\glie)_2, \iota_2)$.
  Then there exists unique algebra homomorphisms~$\phi \colon \Univ(\glie)_1 \to \Univ(\glie)_2$ and~$\psi \colon \Univ(\glie)_2 \to \Univ(\glie)_1$ that make the triangular diagrams
  \[
    \begin{tikzcd}[column sep = small]
      {}
      &
      \glie
      \arrow{dl}[above left]{\iota_1}
      \arrow{dr}[above right]{\iota_2}
      &
      {}
      \\
      \Univ(\glie)_1
      \arrow[dashed]{rr}[below]{\phi}
      &
      {}
      &
      \Univ(\glie)_2
    \end{tikzcd}
    \qquad\text{and}\qquad
    \begin{tikzcd}[column sep = small]
      {}
      &
      \glie
      \arrow{dl}[above left]{\iota_1}
      \arrow{dr}[above right]{\iota_2}
      &
      {}
      \\
      \Univ(\glie)_2
      \arrow[dashed]{rr}[below]{\phi}
      &
      {}
      &
      \Univ(\glie)_1
    \end{tikzcd}
  \]
  commute.
  It follows that the compositions~$\phi \circ \psi \colon \Univ(\glie)_1 \to \Univ(\glie)_1$ and~$\psi \circ \phi \colon \Univ(\glie)_2 \to \Univ(\glie)_2$ make the triangular diagrams
  \[
    \begin{tikzcd}[column sep = small]
      {}
      &
      \glie
      \arrow{dl}[above left]{\iota_1}
      \arrow{dr}[above right]{\iota_1}
      &
      {}
      \\
      \Univ(\glie)_1
      \arrow[dashed]{rr}[below]{\psi \circ \phi}
      &
      {}
      &
      \Univ(\glie)_1
    \end{tikzcd}
    \qquad\text{and}\qquad
    \begin{tikzcd}[column sep = small]
      {}
      &
      \glie
      \arrow{dl}[above left]{\iota_2}
      \arrow{dr}[above right]{\iota_2}
      &
      {}
      \\
      \Univ(\glie)_2
      \arrow[dashed]{rr}[below]{\phi \circ \psi}
      &
      {}
      &
      \Univ(\glie)_2
    \end{tikzcd}
  \]
  commutes.
  The algebra homomorphisms~$\phi \circ \psi$ and~$\psi \circ \phi$ are unique with this property by the universal property of the {\uas}~$(\Univ(\glie)_1, \iota_1)$ and~$(\Univ(\glie)_2, \iota_2)$.
  But the identities~$\id_{\Univ(\glie)_1}$ and~$\id_{\Univ(\glie)_2}$ also makes these diagrams commute.
  We thus find that~$\psi \circ \phi = \id_{\Univ(\glie)_1}$ and~$\phi \circ \psi = \id_{\Univ(\glie)_2}$, so that~$\phi$ and~$\psi$ are mutually inverse isomorphisms of~{\algebras{$\kf$}}.
  
  This shows that the {\ua} (if it exists) is \enquote{unique up to unique isomorphisms}.
  We will therefore talk about \emph{the} {\ua} of~$\glie$.
  We will often also surpress the algebra homorphism~$\iota \colon \glie \to \Univ(\glie)$ from our notation.
\end{remark}


% \begin{remark}
%   One can also formulate the above argument is a more categorical way:
%   Consider the category~$\catC$ where
%   \begin{itemize}
%     \item
%       objects of~$\catC$ is a pairs~$(A, i)$ consisting of a~{\algebra{$\kf$}}~$A$ and a Lie~algebra homomorphism~$i \colon \glie \to A$,
%     \item
%       a morphism~$\phi \colon (A, i) \to (B, j)$ is an algebra homomorphism~$\phi \colon A \to B$ that makes the triangular diagram
%       \[
%         \begin{tikzcd}[column sep = small]
%         {}
%         &
%         \glie
%         \arrow{dl}[above left]{i}
%         \arrow{dr}[above right]{j}
%         &
%         {}
%         \\
%         A
%         \arrow[dashed]{rr}[below]{\phi}
%         &
%         {}
%         &
%         B
%       \end{tikzcd}
%     \]
%       commute, and
%     \item
%       the composition of two morphisms is just their usual set-theoretic composition.
%   \end{itemize}
%   A {\ua} of~$\glie$ is nothing but an inital object in this category~$\catC$.
%   The argumentation from \cref{uniqueness of universal enveloping algebras} is then the usual argument for the uniqueness of inital objects up to unique isomorphism.
% \end{remark}


\begin{proposition}
  \label{representations are modules}
  Let~$V$ be a~{\vectorspace{$\kf$}}.
  Let~$\glie$ be a Lie~algebra and let~$\iota \colon \glie \to \Univ(\glie)$ be the canonical Lie~algebra homomorphism.
  Then the assignments
  \begin{align*}
    \left\{
    \begin{tabular}{@{}c@{}}
      representations of~$\glie$, \\
      $\rho \colon \glie \to \gllie(V)$
    \end{tabular}
    \right\}
    &\longonetoone
    \left\{
    \begin{tabular}{@{}c@{}}
      $\Univ(\glie)$-module structures \\
      $\theta \colon \Univ(\glie) \to \End_{\kf}(V)$
    \end{tabular}
    \right\}  \,,
    \\
    \rho
    &\longmapsto
    \hat{\rho} \,,
    \\
    \theta \circ \iota
    &\longmapsfrom
    \theta  \,,
  \end{align*}
  constitute a {\onetoone} correspondence,~where $\hat{\rho} \colon \Univ(\glie) \to \End_{\kf}(V)$ is the unique~\algebra{$\kf$} homomorphism induced by the homomorphism of Lie~algebras~$\rho \colon \glie \to \gllie(V)$ via the universal property of the~{\ua}~$\Univ(\glie)$.
  \qed
\end{proposition}


\begin{remark}
  \Cref{representations are modules} shows that representations of~$\glie$ are the same as~{\modules{$\Univ(\glie)$}}.
  The categories~$\cRep{\glie}$ and~$\cMod{\Univ(\glie)}$ are hence isomorphic.
\end{remark}


\begin{lemma}[Functoriality of the universal enveloping algebra]
  \label{functoriality of universal enveloping algebra}
  Let~$\glie$,~$\hlie$ and~$\klie$ be Lie~algebras.
  \begin{enumerate}
    \item
      Any homomorphism of Lie~algebras~$\phi \colon \glie \to \hlie$ induces a unique homomorphism of~\algebras{$\kf$}~$\phi^* \colon \Univ(\glie) \to \Univ(\hlie)$ that makes the following square diagram commute:
      \[
        \begin{tikzcd}[column sep = large]
          \glie
          \arrow{r}[above]{\phi}
          \arrow{d}
          &
          \hlie
          \arrow{d}
          \\
          \Univ(\glie)
          \arrow[dashed]{r}[below]{\phi_*}
          &
          \Univ(\hlie)
        \end{tikzcd}
      \]
    \item
      It holds that~$(\id_{\glie})_* = \id_{\Univ(\glie)}$.
    \item
      It holds for all composable homomorphisms of Lie~algebras~$\phi \colon \glie \to \hlie$ and~$\psi \colon \hlie \to \klie$ that
      \[
        (\psi \circ \phi)_*
        =
        \psi_* \circ \phi_* \,.
      \]
  \end{enumerate}
\end{lemma}


\begin{proof}
  \leavevmode
  \begin{enumerate}
    \item
      This follows from the universal property of the universal enveloping algebra~$\Univ(\glie)$ by applying it to the composition~$\glie \to \hlie \to \Univ(\hlie)$.
    \item
      The square diagram
      \[
        \begin{tikzcd}[column sep = huge]
          \glie
          \arrow{r}[above]{\id_{\glie}}
          \arrow{d}
          &
          \glie
          \arrow{d}
          \\
          \Univ(\glie)
          \arrow[dashed]{r}[below]{\id_{\Univ(\glie)}}
          &
          \Univ(\glie)
        \end{tikzcd}
      \]
      commutes, which shows that~$\id_{\Univ(\glie)}$ satisfies the defining property of the induced algebra homomorphism~$(\id_{\glie})_*$.
    \item
      We have the following commutative diagram:
      \[
        \begin{tikzcd}[column sep = large]
          \glie
          \arrow[dashed, bend left = 40]{rr}[above]{\psi \circ \phi}
          \arrow{r}[above]{\phi}
          \arrow{d}
          &
          \hlie
          \arrow{r}[above]{\psi}
          \arrow{d}
          &
          \klie
          \arrow{d}
          \\
          \Univ(\glie)
          \arrow{r}[below]{\phi_*}
          \arrow[dashed, bend right = 40]{rr}[below]{\psi_* \circ \phi_*}
          &
          \Univ(\hlie)
          \arrow{r}[below]{\psi_*}
          &
          \Univ(\klie)
        \end{tikzcd}
      \]
      The commutativity of the outer square diagram
      \[
        \begin{tikzcd}[column sep = huge]
          \glie
          \arrow{r}[above]{\psi \circ \phi}
          \arrow{d}
          &
          \klie
          \arrow{d}
          \\
          \Univ(\glie)
          \arrow[dashed]{r}[below]{\psi_* \circ \phi_*}
          &
          \Univ(\klie)
        \end{tikzcd}
      \]
      shows that~$\psi_* \circ \phi_*$ satisfies the defining property of the induced algebra homomorphism~$(\psi \circ \phi)_*$.
    \qedhere
  \end{enumerate}
\end{proof}


\begin{remark}
  \Cref{functoriality of universal enveloping algebra} shows that the assignment~$\glie \mapsto \Univ(\glie)$ of a Lie~algebra~$\glie$ to its universal eveloping algebra~$\Univ(\glie)$ can be extended to a (covariant) functor~$\Univ \colon \cLie{\kf} \to \cAlg{\kf}$.
  The universal property of the {\ua} states that the functor~$\Univ$ is left adjoint to the forgetful functor~$\cAlg{\kf} \to \cLie{\kf}$ that assigns to each~{\algebra{$\kf$}} its underlying Lie~algebra.
\end{remark}


\begin{remark}
  Let~$\glie$ be a Lie~algebra.
  It follows from the universal propery of the {\ua} that~$\Univ(\glie)$ is generated by the image of the canonical homomorphism~$\iota \colon \glie \to \Univ(\glie)$:
  
  Indeed, let~$U$ be the subalgebra of~$\Univ(\glie)$ that is generated by the image of~$\iota$, and let~$i \colon \glie \to U$ be the restriction of~$\iota$.
  Then for every~{\algebra{$\kf$}}~$A$ and every Lie~algebra homomorphism~$\phi \colon \glie \to A$ the induced algebra homomorphism~$\Phi' \colon \Univ(\glie) \to A$ restricts to an algebra homomorphism~$\Phi \colon U \to A$ that makes the triangular diagram
  \[
    \begin{tikzcd}
      \glie
      \arrow{r}[above]{\phi}
      \arrow{d}[left]{j}
      &
      A
      \\
      U
      \arrow[dashed]{ur}[below right]{\Phi}
      &
      {}
    \end{tikzcd}
  \]
  commute.
  The homomorphism~$\Phi$ is unique with this property because it is uniquely determined by the restriction~$\Phi \circ j = \phi$, since~$U$ is generated by the image of~$j$.
  This shows that~$(U, j)$ is again a {\ua} for~$\glie$.
  
  It follows from the uniqueness of the {\ua}, as discussed in \cref{uniqueness of universal enveloping algebras}, that the unique algebra homomorphism~$i \colon U \to \Univ(\glie)$ that makes the triangular diagram
  \[
    \begin{tikzcd}[column sep = small]
      {}
      &
      \glie
      \arrow{dl}[above left]{j}
      \arrow{dr}[above right]{\iota}
      &
      {}
      \\
      U
      \arrow[dashed]{rr}[below]{i}
      &
      {}
      &
      \Univ(\glie)
    \end{tikzcd}
  \]
  commute is already an isomorphism.
  This homomorphism is necessarily the inclusion of~$U$ into~$\Univ(\glie)$, and that it is an isomorphism means that already~$U = \Univ(\glie)$.
  
  The canonical homomorphism~$\iota \colon \glie \to \Univ(\glie)$ is in particular~{\linear{$\kf$}} and hence induces an algebra homomorphism~$\phi \colon \Tensor(\glie) \to \Univ(\glie)$ that makes the triangular diagram
  \[
    \begin{tikzcd}[column sep = small]
      {}
      &
      \glie
      \arrow{dl}
      \arrow{dr}[above right]{\iota}
      &
      {}
      \\
      \Tensor(\glie)
      \arrow[dashed]{rr}[below]{\phi}
      &
      {}
      &
      \Univ(\glie)
    \end{tikzcd}
  \]
  commute.
  That~$\Univ(\glie)$ is generated by the image of~$\iota$ means that~$\phi$ is surjective.
  Therefore~$\phi$ induces an algebra isomorphism
  \[
    \Phi
    \colon
    \Tensor(\glie)/I
    \to
    \Univ(\glie)
  \]
  for~$I = \ker \phi$, that makes the resulting diagram
   \[
    \begin{tikzcd}[column sep = small]
      {}
      &
      \glie
      \arrow[bend right]{dl}
      \arrow[bend left]{ddr}[above right]{\iota}
      &
      {}
      \\
      \Tensor(\glie)
      \arrow[bend left]{drr}[below left]{\phi}
      \arrow{d}
      &
      {}
      &
      {}
      \\
      \Tensor(\glie)/I
      \arrow[dashed]{rr}[below]{\Phi}
      &
      {}
      &
      \Univ(\glie)
    \end{tikzcd}
  \]
  commute.
  
  If~$A$ is any other~{\algebra{$\kf$}} then we know on the one hand that algebra homorphisms~$\Univ(\glie) \to A$ correspond to Lie~algebra homomorphisms~$\glie \to A$.
  We know find on the other hand that~{\algebra{$\kf$}} homomorphisms~$\Tensor(V)/I \to A$ correspond to algebra homomorphisms~$\Tensor(V) \to A$ that annihilate~$I$, with the algebra homorphisms~$\Tensor(V) \to A$ corresponding to~{\linear{$\kf$}} maps~$\glie \to A$.
  A linear map~$f \colon \glie \to A$ is a homorphism of Lie~algebras if and only if
  \[
      f(x)f(y)
    - f(y)f(x)
    - f([x,y])
    =
    0 \,,
  \]
  so it seems reasonable to assume that the corresponding ideal~$I$ of~$\Tensor(V)$ ought to be given by
  \[
    I
    =
    (x \tensor y - y \tensor x - [x,y] \suchthat x, y \in \glie)
  \]
  We will now show that this is indeed the case.
\end{remark}


\begin{proposition}[Existence of the universal enveloping algebra]
  Let~$\glie$ be a Lie~algebra.
  Let~$\Tensor(\glie)$ be the tensor algebra of the underlying vector space of~$\glie$ and let~$I$ the two-sided ideal in~$\Tensor(\glie)$ generated by the elements $x \tensor y - y \tensor x - [x,y]$ with~$x,y \in \glie$.
  The the quotient algebra~$U \defined T(\glie)/I$ together with the~{\linear{$\kf$}} map
  \[
    i
    \colon
    \glie
    \to
    \Univ(\glie) \,,
    \quad
    x
    \mapsto
    \class{x}
  \]
  is a {\ua} for~$\glie$.
\end{proposition}


\begin{proof}
  The map~$i$ is~{\linear{$\kf$}} and it compatible with the Lie brackets because
  \[
    [i(x), i(y)]
    =
    [\class{x}, \class{y}]
    =
    \class{x} \, \class{y} - \class{y} \, \class{x}
    =
    \class{x \tensor y - y \tensor x}
    =
    \class{[x,y]}
    =
    i([x,y]) \,.
  \]
  Given any~{\algebra{$\kf$}}~$A$ and Lie algebra homomorphism~$\phi \colon \glie \to A$ there exists a unique homorphism of~{\algebras{$\kf$}}~$\Phi' \colon \Tensor(\glie) \to A$ that makes the triangular diagram
  \[
    \begin{tikzcd}
      \glie
      \arrow{r}[above]{\phi}
      \arrow{d}[left]{\iota}
      &
      A
      \\
      \Tensor(V)
      \arrow[dashed]{ur}[below right]{\Phi'}
      &
      {}
    \end{tikzcd}
  \]
  commute, where~$\iota \colon \glie \to \Tensor(\glie)$ is the inclusion.
  The homomorphism~$\Phi'$ is given by~$\Phi'(x) = \phi(x)$ for every~$x \in \glie$.
  It follows that
  \begin{align*}
    \Phi'(x \tensor y - y \tensor x)
    &=
    \Phi'(x \tensor y) - \Phi'(y \tensor x)
    \\
    &=
    \Phi'(x) \Phi'(y) - \Phi'(y) \Phi'(x)
    \\
    &=
    \phi(x) \phi(y) - \phi(y) \phi(x)
    \\
    &=
    [\phi(x), \phi(y)]
    \\
    &=
    \phi([x,y])
    \\
    &=
    \Phi'([x,y])
  \end{align*}
  for all~$x, y \in \glie$, so that the ideal~$I$ is contained in the kernel of~$\Phi'$.
  It follows that there exists a unique algebra homomorphism~$\Phi \colon \Tensor(\glie)/I \to A$ that makes the triangular diagram
  \[
    \begin{tikzcd}
      \glie
      \arrow{r}[above]{\phi}
      \arrow{d}[left]{\iota}
      \arrow[bend right = 55]{dd}[left]{i}
      &
      A
      \\
      \Tensor(\glie)
      \arrow[bend right= 20]{ur}[above left]{\Phi'}
      \arrow{d}[left]{\pi}
      &
      {}
      \\
      \Tensor(\glie)/I
      \arrow[dashed, bend right = 30]{uur}[below right]{\Phi}
      &
      {}
    \end{tikzcd}
  \]
  commute, where~$\pi \colon \Tensor(V) \to \Tensor(V)/I$ denotes the canonical projection.
  Then the subdiagram
  \[
    \begin{tikzcd}
      \glie
      \arrow{r}[above]{\phi}
      \arrow{d}[left]{i}
      &
      A
      \\
      \Tensor(\glie)/I
      \arrow{ur}[below right]{\Phi}
      &
      {}
    \end{tikzcd}
  \]
  commutes.
  That~$\Phi$ is unique with this property follows from the uniqueness of~$\Phi'$.
\end{proof}


\begin{remark}
  The above proof may be reorganized by observing that we have bijections
  \begin{align*}
    {}&
    \{ \text{algebra homomorphisms~$\Phi \colon \Tensor(\glie)/I \to A$} \}
    \\
    \cong{}&
    \{ \text{algebra homomorphisms~$\Phi' \colon \Tensor(\glie) \to A$ with~$\Phi'(I) = 0$} \}
    \\
    \cong{}&
    \left\{
      \begin{tabular}{@{}c@{}}
        algebra homomorphisms~$\Phi' \colon \Tensor(\glie) \to A$ with  \\
        $\Phi'(x \tensor y - y \tensor x - [x,y]) = 0$ for all~$x, y \in \glie$
      \end{tabular}
    \right\}
    \\
    \cong{}&
    \left\{
      \begin{tabular}{@{}c@{}}
        algebra homomorphisms~$\Phi' \colon \Tensor(\glie) \to A$ with  \\
        $\Phi'(x) \Phi'(y) - \Phi'(y) \Phi'(x) - \Phi'([x,y]) = 0$ for all~$x, y \in \glie$
      \end{tabular}
    \right\}
    \\
    \cong{}&
    \left\{
      \begin{tabular}{@{}c@{}}
        algebra homomorphisms~$\Phi' \colon \Tensor(\glie) \to A$ with  \\
        $\Phi'(x) \Phi'(y) - \Phi'(y) \Phi'(x) = \Phi'([x,y])$ for all~$x, y \in \glie$
      \end{tabular}
    \right\}
    \\
    \cong{}&
    \left\{
      \begin{tabular}{@{}c@{}}
        {\linear{$\kf$}} maps~$\phi \colon \glie \to A$ with  \\
        $\phi(x) \phi(y) - \phi(y) \phi(x) = \phi([x,y])$ for all~$x, y \in \glie$
      \end{tabular}
    \right\}
    \\
    \cong{}&
    \left\{
      \begin{tabular}{@{}c@{}}
        {\linear{$\kf$}} maps~$\phi \colon \glie \to A$ with  \\
        $[\phi(x), \phi(y)] = \phi([x,y])$ for all~$x, y \in \glie$ 
      \end{tabular}
    \right\}
    \\
    ={}&
    \{ \text{Lie~algebra homomorphisms~$\phi \colon \glie \to A$} \}
  \end{align*}
  that are natural in~$A$.
  This shows that the~{\algebra{$\kf$}}~$\Tensor(\glie)/I$ represented the right kind of functor;
  and the identity~$\Tensor(\glie)/I \to \Tensor(\glie)/I$ corresponds under the above bijections to the map~$\iota \colon \glie \to \Tensor(\glie)/I$ as desired.
\end{remark}


\begin{examples}
  Let~$\glie$ be an abelian Lie~algebra.
  It follows from the explicit construction of the universal enveloping algebra~$\Univ(\glie)$ that
  \[
    \Univ(\glie)
    \cong
    \Tensor(\glie)/(x \tensor y - y \tensor x \suchthat x, y \in \glie)
    \cong
    \Symm(\glie)
  \]
  with the canonical Lie~algebra homomorphism~$\glie \to \Univ(\glie)$ corresponding to the inclusion~$\glie \to \Symm(\glie)$.
  This can also be seen more abstractly:
  
  We observe that if~$V$ is a vector space and~$A$ is an algebra then a linear map~$f \colon V \to A$ extends to an algebra homomorphism~$\Symm(V) \to A$ (necessarily uniquely) if and only if the image of~$f$ is contained in a commutative subalgebra of~$A$, if and only if the image of~$f$ is commutative.
  It hence follows that for any~{\algebra{$\kf$}} we have bijections
  \begin{align*}
    {}&
    \{ \text{Lie~algebra homomorphisms~$\glie \to A$} \}
    \\
    \cong{}&
    \{ \text{{\linear{$\kf$}} maps~$\glie \to A$ with commutative image} \}
    \\
    \cong{}&
    \{ \text{algebra homomorphisms~$\Symm(\glie) \to A$} \}
  \end{align*}
  that are natural in~$A$.
  This shows that the symmetric algebra~$\Symm(\glie)$ together with the inclusion~$\glie \to \Symm(\glie)$ satisfies the universal property of the universal enveloping algebra of~$\glie$.
\end{examples}


\begin{recall}
  \label{homomorphism out of a tensor product}
  Let~$A$ and~$B$ be two~{\algebras{$\kf$}}.
  Then the inclusions~$i \colon A \to A \tensor B$ and~$j \colon B \to A \tensor B$ are injective algebra homomorphisms.
  We may therefore identify~$A$ and~$B$ with the subalgebras~$A \tensor 1$ and~$1 \tensor B$ of~$A \tensor B$.
  Note that~$A$ and~$B$ commute in~$A \tensor B$ because
  \[
    i(a) j(b)
    =
    (a \tensor 1) (b \tensor 1)
    =
    a \tensor b
    =
    (b \tensor 1) (a \tensor 1)
    =
    j(b) i(a)
  \]
  for all~$a \in A$ and~$b \in B$.
  
  Let~$C$ be another~{\algebra{$\kf$}}.
  
  If~$f \colon A \tensor B \to C$ is an algebra homomorphism then the restrictions~$\phi = f \circ i$ and~$\psi = f \circ j$ are again algebra homomorphisms~$\phi \colon A \to C$ and~$\psi \colon B \to C$.
  The images of~$\phi$ and~$\psi$ in~$C$ commute with each other because~$A$ and~$B$ commute in~$A \tensor B$.
  More explicitely,
  \begin{align*}
    \phi(a) \psi(b)
    &=
    f(a \tensor 1) f(1 \tensor b)
    \\
    &=
    f((a \tensor 1) (1 \tensor b))
    \\
    &=
    f(a \tensor b)
    \\
    &=
    f((1 \tensor b) (a \tensor 1))
    \\
    &=
    f(1 \tensor b) f(a \tensor 1)
    \\
    &=
    \psi(b) \phi(a)
  \end{align*}
  for all~$a \in A$ and~$b \in B$.
  
  If on the other hand~$\phi \colon A \to C$ and~$\psi \colon B \to C$ are two algebra homomorphisms whose images commute with each other then the map
  \[
    f'
    \colon
    A \times B
    \to
    C \,,
    \quad
    (a,b)
    \mapsto
    \phi(a) \psi(b)
  \]
  is~{\bilinear{$\kf$}} and hence induces a~{\linear{$\kf$}} map
  \[
    f
    \colon
    A \tensor B
    \to
    C \,,
    \quad
    a \tensor b
    \mapsto
    \phi(a) \psi(b) \,.
  \]
  The map~$f$ is again an algebra homomorphism because
  \begin{align*}
    f(a_1 \tensor b_1) f(a_2 \tensor b_2)
    &=
    \phi(a_1) \psi(b_1) \phi(a_2) \psi(b_2)
    \\
    &=
    \phi(a_1) \phi(a_2) \psi(b_1) \psi(b_2)
    \\
    &=
    \phi(a_1 a_2) \psi(b_1 b_2)
    \\
    &=
    f( (a_1 a_2) \tensor (b_1 b_2) )
    \\
    &=
    f( (a_1 \tensor b_1) (a_2 \tensor b_2) )
  \end{align*}
  for all simple tensors~$a \tensor b \in A \tensor B$.
  
  These constructions are mutually inverse and hence result in a {\onetoone} correspondence
  \begin{align*}
    \left\{
      \begin{tabular}{@{}c@{}}
        algebra homomorphisms \\
        $f \colon A \tensor B \to C$
      \end{tabular}
    \right\}
    &\longonetoone
    \left\{
      (\phi, \psi)
    \suchthat*
      \begin{tabular}{@{}c@{}}
        algebra homomorphisms \\
        $\phi \colon A \to C$ and~$\psi \colon B \to C$ \\
        whose images commute with each other
      \end{tabular}
    \right\}  \,,
    \\
    f
    &\longmapsto
    (f \circ i, f \circ j)  \,,
    \\
    \biggl( a \tensor b \mapsto \phi(a)\psi(b) \biggr)
    &\longmapsfrom
    (\phi, \psi)  \,.
  \end{align*}
\end{recall}


\begin{example}
  If~$\glie$ and~$\hlie$ be two Lie~algebras then
  \[
    \Univ(\glie \times \hlie)
    \cong
    \Univ(\glie) \tensor \Univ(\hlie) \,.
  \]
  This can be seen in various ways:
  \begin{itemize}
    \item
      It follows from \cref{homomorphism out of a product} and \cref{homomorphism out of a tensor product} that we get for every~{\algebra{$\kf$}}~$A$ bijections
      \begin{align*}
        {}&
        \{ \text{algebra homomorphisms~$F \colon \Univ(\glie \times \hlie) \to A$} \}
        \\
        \cong{}&
        \{ \text{Lie~algebra homomorphisms~$f \colon \glie \times \hlie \to A$} \}
        \\
        \cong{}&
        \left\{
          \begin{tabular}{@{}c@{}}
            Lie~algebra homomorphisms \\
            $\phi \colon \glie \to A$ and~$\psi \colon \hlie \to A$ \\
            whose images commute with each other
          \end{tabular}
        \right\}
        \\
        \cong{}&
        \left\{
          \begin{tabular}{@{}c@{}}
            algebra homomorphisms \\
            $\Phi \colon \Univ(\glie) \to A$ and~$\Psi \colon \Univ(\hlie) \to A$ \\
            whose images commute with each other
          \end{tabular}
        \right\}
        \\
        \cong{}&
        \{ \text{algebra homomorphims~$F \colon \Univ(\glie) \tensor \Univ(\hlie) \to A$} \}  \,.
      \end{align*}
      The claimed isomorphism therefore follows from Yoneda’s lemma.
    \item
      More explicitely let~$i \colon \glie \to \glie \times \hlie$ and~$j \colon \hlie \to \glie \times \hlie$ be the canonical inclusions.
      Then for the induced algebra homomorphisms~$i_* \colon \Univ(\glie) \to \Univ(\glie \times \hlie)$ and~$j_* \colon \Univ(\hlie) \to \Univ(\glie \times \hlie)$ the images of~$i_*$ of~$j_*$ commute with each other.
      Indeed, the algebra~$\Univ(\glie \times \hlie)$ is generated by the image of~$\glie \times \hlie$ in~$\Univ(\glie \times \hlie)$, and~$i(\glie)$ and~$j(\hlie)$ commute in~$\glie \times \hlie$.
      Therefore
      \[
        i_*(\class{x}) j_*(\class{y})
        =
        \class{(x,0)} \class{(0,y)}
        =
        \class{(0,y)} \class{(x,0)}
        =
        j_*(\class{y}) i_*(\class{x}) \,,
      \]
      for all~$x \in \glie$ and~$y \in \hlie$, where we denote by~$\class{(-)}$ the corresponding elements of the {\uas}.
      We used for the middle equality that~$(x,0)$ and~$(0,y)$ commute in~$\glie \times \hlie$ and hence also in~$\Univ(\glie \times \hlie)$.
      
      It follows that~$i_*$ and~$j_*$ induce a common algebra homomorphism
      \[
        \phi
        \colon
        \Univ(\glie) \tensor \Univ(\hlie)
        \to
        \Univ(\glie \times \hlie) \,,
      \]
      that is given on elements by
      \[
        \phi(x \tensor y)
        =
        i_*(x) j_*(y)
      \]
      for all simple tensors~$x \tensor y \in \Univ(\glie) \tensor \Univ(\hlie)$.
      It holds in particular for all~$x \in \glie$ and~$y \in \glie$ that
      \[
        \phi(\class{x} \tensor \class{y})
        =
        i_*(\class{x}) j_*(\class{y})
        =
        \class{i(x)} \cdot \class{j(y)}
        =
        \class{(x,0)} \cdot \class{(0,y)}  \,,
      \]
      We observe that the map
      \[
        \psi'
        \colon
        \glie \times \hlie
        \to
        \Univ(\glie) \tensor \Univ(\hlie) \,,
        \quad
        (x,y)
        \mapsto
        \class{(x,0)} \tensor 1 + 1 \tensor \class{(0,y)} \,.
      \]
      To construct the inverse~$\psi$ of~$\phi$ we observe that the equalities
      \[
        \psi( \class{(x,0)} )
        =
        \psi( \class{(x,0)} \cdot 1 )
        =
        \psi( i_*(\class{x}) \cdot j_*(1) )
        =
        \psi( \phi(\class{x} \tensor 1) )
        =
        \class{x} \tensor 1
      \]
      and similarly~$\psi( \class{(0,y)} ) = 1 \tensor \class{y}$ have to hold for all~$x \in \glie$ and~$y \in \glie$.
      It then follows that more generally
      \[
        \psi( \class{(x,y)} )
        =
        \psi( \class{(x,0)} + \class{(0,y)} )
        =
        \class{x} \tensor 1 + 1 \tensor \class{y}
      \]
      for all~$(x,y) \in \glie \times \hlie$.
      
      Motivated by these observations we consider the map
      \[
        \psi'
        \colon
        \glie \times \hlie
        \to
        \Univ(\glie) \tensor \Univ(\hlie)
      \]
      that is given by
      \[
        \psi'((x,y))
        =
        \class{x} \tensor 1 + 1 \tensor \class{y} \,.
      \]
      This map is~{\linear{$\kf$}} and it is a homomorphism of Lie~algebras because
      \begin{align*}
        {}&
        [\psi'((x_1, y_1)), \psi'((x_2, y_2))]
        \\
        ={}&
        [
          \class{x_1} \tensor 1 + 1 \tensor \class{y_1},
          \class{x_2} \tensor 1 + 1 \tensor \class{y_2}
        ]
        \\
        ={}&
          [\class{x_1} \tensor 1, \class{x_2} \tensor 1]
        + \underbrace{ [\class{x_1} \tensor 1, 1 \tensor \class{y_2}] }_{=0}
        + \underbrace{ [1 \tensor \class{y_1}, \class{x_2} \tensor 1] }_{=0}
        + [1 \tensor \class{y_1}, 1 \tensor \class{y_2}]
        \\
        ={}&
          [\class{x_1}, \class{x_2}] \tensor 1
        + 1 \tensor [\class{y_1}, \class{y_2}]
        \\
        ={}&
          \class{[x_1, x_2]} \tensor 1
        + 1 \tensor \class{[y_1, y_2]}  \,.
        \\
        ={}&
        \psi'( ( [x_1, x_2], [y_1, y_2] ) )
        \\
        ={}&
        \psi'( [(x_1, y_1), (x_2, y_2)] ) \,.
      \end{align*}
      It hence follows from the universal property of the {\ua}~$\Univ(\glie \times \hlie)$ that there exists a unique algebra homomorphism~$\psi \colon \Univ(\glie \times \hlie) \to \Univ(\glie) \tensor \Univ(\hlie)$ that makes the triangular diagram
      \[
        \begin{tikzcd}[row sep = large]
          \glie \times \hlie
          \arrow{r}[above]{\psi'}
          \arrow{d}[left]{\class{(-)}}
          &
          \Univ(\glie) \tensor \Univ(\hlie)
          \\
          \Univ(\glie \times \hlie)
          \arrow[dashed]{ur}[below right]{\psi}
          &
          {}
        \end{tikzcd}
      \]
      commute.
      This homomorphism~$\psi$ is given for all~$(x, y) \in \glie \times \hlie$ by
      \[
        \psi( \class{(x,y)} )
        =
        \class{x} \tensor 1 + 1 \tensor \class{y} \,.
      \]
      
      The homomorphisms~$\phi$ and~$\psi$ are mutually inverse:
      It sufficies to check this on the generators~$\class{x} \tensor \class{y}$ of~$\Univ(\glie) \tensor \Univ(\hlie)$ with~$x \in \glie$ and~$y \in \hlie$, and on the generators~$\class{(x,y)}$ of~$\Univ(\glie \times \hlie)$ with~$(x,y) \in \glie \times \hlie$.
      And indeed, we find that
      \begin{align*}
        \phi( \psi( \class{(x,y)} ) )
        &=
        \phi( \class{x} \tensor 1 + 1 \tensor \class{y} )
        \\
        &=
        \phi( \class{x} \tensor 1 ) + \phi( 1 \tensor \class{y} )
        \\
        &=
        i_*(\class{x}) j_*(1) + i_*(1) j_*(\class{y})
        \\
        &=
        \class{i(x)} + \class{j(y)}
        \\
        &=
        \class{(x,0)} + \class{(0,y)}
        \\
        &=
        \class{(x,y)}
      \end{align*}
      and
      \begin{align*}
        \psi( \phi( \class{x} \tensor \class{y} ) )
        &=
        \psi( i_*(\class{x}) j_*(\class{y}) )
        \\
        &=
        \psi( \class{(x,0)} \class{(0,y)} )
        \\
        &=
        \psi( \class{(x,0)} ) \psi( \class{(0,y)} )
        \\
        &=
        ( \class{x} \tensor 1 + 1 \tensor 0 ) ( 0 \tensor 1 + 1 \tensor \class{y} )
        \\
        &=
        (\class{x} \tensor 1) (1 \tensor \class{y})
        \\
        &=
        \class{x} \tensor \class{y} \,.
      \end{align*}
  \end{itemize}
\end{example}


% TODO: UE of product

% 
% 
% \begin{corollary}
%  The homomorphism $\iota \colon \glie \to \Univ(\glie)$ is injective. As a \algebra{$\kf$} $\Univ(\glie)$ is generated by $\iota(\g)$.
% \end{corollary}
% 
% 
% \begin{remark}
%  We will always identify $\glie$ with its image under $\iota$.
% \end{remark}
% 
% 
% % TODO: Requires Hopf algebra structure
% % \begin{lemma}
% %  Let $\g_1$ and $\g_2$ be $k$-Lie algebras. Then $\Univ(\g_1 \times \g_2)$ and $\Univ(\g_1) \tensor \Univ(\g_2)$ are naturally isomorphic.
% % \end{lemma}
% 
% 
% 
% 
% 
% \section{Poincar\'{e}--Birkhoff--Witt}
% 
% 
% 
% \subsection{Graded \algebras{$\kf$}}
% 
% 
% \begin{definition}\label{defi: graded algebras}
%  A \emph{grading}, also called \emph{gradation}, of a \algebra{$\kf$} $A$ is a direct sum decomposition $A = \bigoplus_{i \in \N} A_i$ into linear subspaces such that
%  \[
%   A_i A_j \subseteq A_{i+j} \quad \text{for all $i,j \in \N$}.
%  \]
%  A \emph{graded \algebra{$\kf$}} is a \algebra{$\kf$} $A$ together with a grading $A = \bigoplus_{n \in \N} A_n$.
% \end{definition}
% 
% 
% \begin{remark}
%  While a graded \algebra{$\kf$} is formally a pair $(A, (A_n)_{n \in \N})$ consisting of a \algebra{$\kf$} $A$ and a grading $A = \bigoplus_{n \in \N} A_n$ we will often just call $A$ a graded \algebra{$\kf$} without explicitily mentioning the grading. We also set $A_n \defined 0$ for every $n < 0$.
% \end{remark}
% 
% 
% \begin{remark}
%  Given any semigroup $(S,\cdot)$ an \emph{$S$-grading} of a \algebra{$\kf$} $A$ is a decomposition $A = \bigoplus_{s \in S} A_s$ into linear subspaces such that $A_s A_t \subseteq A_{s \cdot t}$ for all $s,t \in S$. An $S$-graded \algebra{$\kf$} is a \algebra{$\kf$} $A$ together with an $S$-grading $A = \bigoplus_{s \in S} A_s$. An graded \algebra{$\kf$} in the sense of Definition~\ref{defi: graded algebras} is then the special case of an $\N$-graded \algebra{$\kf$}.
% \end{remark}
% 
% 
% \begin{lemma}
%  Let $A$ be a graded \algebra{$\kf$}. Then $1 \in A_0$ and $A_0$ is a $k$-subalgebra.
% \end{lemma}
% \begin{proof}
%  Let $1 = \sum_{i \in \N} e_i$ with respect to $A = \bigoplus_{n \in \N} A_n$. Then for any $j \in \N$ and $a \in A_j$
%  \[
%   A_j \ni a
%   = a \cdot 1
%   = a \left( \sum_{i \in \N} e_i \right)
%   = \sum_{i \in \N} \underbrace{a e_i}_{\in A_{i+j}},
%  \]
%  and it follows from the directness of the decomposition $A = \bigoplus_{n \in \N} A_n$ that $a = a e_0$. It follows that $a e_0 = a$ for every $a \in A$, hence $e_0$ is the unit of $A$.
%  
%  That $A_0$ is a linear subspace which is closed under the multiplication follows from the definition of a graded \algebra{$\kf$}. As it contains the unit of $A$ it is a $k$-subalgebra.
% \end{proof}
% 
% 
% \begin{examples}\label{expls: graded algebras}
%  \begin{enumerate}[leftmargin=*]
%   \item
%    Any \algebra{$\kf$} $A$ becomes a graded \algebra{$\kf$} by setting $A_0 \defined A$ and $A_i \defined 0$ for every $i > 1$.
%   \item
%    The polynomial ring $A = k[x_1, \dotsc, x_n]$ is a graded \algebra{$\kf$} by setting
%    \[
%     A_d \defined \vspan_k \{ x_1^{p_1} \dotsm x_n^{p_n} \mid p_1 + \dotsb + p_n = d\}
%     \quad \text{for every $d \in \N$},
%    \]
%    i.e.\ $A_d$ consists of the homogeneous polynomials of degree $d$.
%   \item
%    Let $V$ be a $k$-vector space. Then the tensor algebra $T(V) = \bigoplus_{n \in \N} V^{\tensor n}$, the symmetric algebra $S(V) = \bigoplus_{n \in \N} S^n(V)$ and the exterior algebra $\Lambda(V) = \bigoplus_{n \in \N} \Lambda^n(V)$ carry the structure of a graded \algebra{$\kf$} via $T(V)_n \defined V^{\tensor n}$, $S(V)_n \defined S^n(V)$ and $\Lambda(V)_n \defined \Lambda^n(V)$ for every $n \in \N$.
%  \end{enumerate}
% \end{examples}
% 
% 
% \begin{definition}
%  Let $A$ and $B$ be graded \algebras{$\kf$}. A homomorphism of \algebras{$\kf$} $\varphi \colon A \to B$ is called a \emph{homomorphism of graded \algebras{$\kf$}} if $\varphi(A_n) \subseteq B_n$ for every $n \in \N$, and the induced homomorphisms of vector spaces are denoted by $\varphi_n \colon A_n \to B_n$ for every $n \in \N$. An homomorphism of graded \algebras{$\kf$} is called an isomorphism if it is bijective.
% \end{definition}
% 
% 
% \begin{remark}
%  If $A$ is a graded \algebra{$\kf$} then $\id_A$ is a homomorphism of graded \algebras{$\kf$}, and if $B$ and $C$ are two other graded \algebras{$\kf$} and $\varphi \colon A \to B$ and $\psi \colon B \to C$ homomorphisms of graded \algebras{$\kf$} then so is $\psi \colon \varphi \colon A \to C$. Hence the graded \algebras{$\kf$} together with the homomorphisms of graded \algebras{$\kf$} between them form a category, which will be refered to by $\cGrad{k}$.
% \end{remark}
% 
% 
% \begin{example}
%  \begin{enumerate}[leftmargin=*]
%   \item
%    For any vector space $V$ the two maps
%    \begin{gather*}
%     T(V) \to S(V), \quad x_1 \tensor \dotsb \tensor x_n \mapsto x_1 \dotsm x_n
%    \shortintertext{and}
%     T(V) \to \Lambda(V), \quad x_1 \tensor \dotsb \tensor x_n \mapsto x_1 \wedge \dotsb \wedge x_n
%    \end{gather*}
%    are homomorphisms of graded \algebras{$\kf$}.
%   \item
%    If $V$ is a finite dimensional vector space with basis $x_1, \dotsc, x_n$ the the isomorphism of \algebras{$\kf$}
%    \[
%     k[T_1, \dotsc, T_n] \to S(V), \quad T_i \mapsto x_i \quad \text{for every $i = 1, \dotsc, n$}
%    \]
%    is already an isomorphism of graded \algebras{$\kf$}.
%  \end{enumerate}
% \end{example}
% 
% 
% \begin{definition}
%  Let $A$ be a graded \algebra{$\kf$}. A two-sided ideal $J \subseteq A$ is called \emph{homogeneous} if $J = \bigoplus_{n \in \N} (J \cap A_n)$. Equivalently, given any $x \in J$ with the decomposition $x = \sum_{n \in \N} x_n$ with respect to $A = \bigoplus_{n \in \N} A_n$ it follows that $x_n \in J$ for every $n \in \N$.
% \end{definition}
% 
% 
% \begin{lemma}
%  Let $A$ be a graded \algebra{$\kf$} and $J \subseteq A$ a two-sided homogeneous ideal. Then $A/J$ is a graded \algebra{$\kf$} via $(A/J)_n = A_n/(J \cap A_n)$ for every $n \in \N$ and the canonical projection $\pi \colon A \to A/J, a \mapsto a + J$ is a homomorphism of graded \algebras{$\kf$}.
% \end{lemma}
% \begin{proof}
%  This follows directly form the definition of a homogeneous ideal.
% \end{proof}
% 
% 
% \begin{lemma}
%  Let $A$ be a graded \algebra{$\kf$}, $J \subseteq A$ a two-sided ideal and call an element $x \in J$ \emph{homogeneous (in $J$)} if $x_n \in J$ for every $n \in \N$ where $x = \sum_{n \in \N} x_n$ with respect to $A = \bigoplus_{n \in \N} A_n$. Then $J$ is homogeneous if and only if $J$ is generated by elements which are homogeneous in $J$.
% \end{lemma}
% \begin{proof}
%  Let $I \defined \{x \in J \mid \text{$x$ is homogeneous in $J$}\}$ be the linear subspace of elements which are homogeneous in $J$. If $x \in I$ then $x_n \in J$ for every $n \in \N$ where $x = \sum_{n \in \N} x_n$ with respect to $A = \bigoplus_{n \in \N} A_n$. Given $a \in A$ with $a = \sum_{m \in \N} a_m$ with respect to $A = \bigoplus_{m \in \N} A_m$ it follows that $a_m x_n \in J$ for all $m,n \in \N$ because $J$ is left ideal, and therefore $ax = \sum_{m,n \in \N} a_m x_n \in J$. Hence $I$ is a left ideal. In the same way it follows that $I$ is also a right ideal and hence already a two-sided ideal in $A$.
%  
%  The ideal $J$ is homogeneous if and only if any of its elements is homogeneous in $J$, i.e.\ if $I = J$, from which the statement follows.
% \end{proof}
% 
% 
% \begin{examples}
%  \begin{enumerate}[leftmargin=*]
%   \item
%    If $A$ and $B$ are graded \algebras{$\kf$} and $\varphi \colon A \to B$ a homomorphism of graded \algebras{$\kf$} then $\ker \varphi$ is a homogeneous ideal.
%   \item
%    Let $V$ be any vector space. The two-sided ideal $I$ of $T(V)$ generated by the elements $x \tensor y - y \tensor x$ with $x,y \in V$ is a homogeneous ideal of $T(V)$. The same goes for the two-sided ideal $J$ generated by the elements $x \tensor x$ with $x \in V$. The resulting (graded) quotient algebras are $S(V)$ and $\Lambda(V)$.
%  \end{enumerate}
% \end{examples}
% 
% 
% 
% \subsection{Filtered \algebras{$\kf$}}
% 
% 
% \begin{definition}
%  A \emph{filtration} of a \algebra{$\kf$} $A$ is an increasing sequence
%  \[
%   A_{(0)}
%   \subseteq A_{(1)}
%   \subseteq A_{(2)}
%   \subseteq \dotsb
%   \subseteq A
%  \]
%  such that $A = \bigcup_{i \in \N} A_{(i)}$ and $A_{(i)} A_{(j)} \subseteq A_{(i+j)}$ for all $i,j \in \N$, as well as $1 \in A_{(0)}$. A \emph{filtered \algebra{$\kf$}} is a \algebra{$\kf$} $A$ together with a filtration $A = \bigcup_{n \in \N} A_{(n)}$.
% \end{definition}
% 
% 
% \begin{remark}
%  As for graded \algebras{$\kf$} we will refer to a filtered \algebra{$\kf$} $A$ without explicitely mentioning the filtration. We also set $A_{(n)} \defined 0$ for every $n < 0$.
% \end{remark}
% 
% 
% \begin{definition}
%  Let $A$ and $B$ be filtered \algebras{$\kf$}. A homomorphism of \algebras{$\kf$} $\varphi \colon A \to B$ is called a \emph{homomorphism of filtered \algebras{$\kf$}} if $\varphi(A_{(n)}) \subseteq B_{(n)}$ for every $n \in \N$.
% \end{definition}
% 
% 
% \begin{remark}
%  If $A$, $B$ and $C$ are filtered \algebras{$\kf$} then $\id_A$ is a homomorphism of filtered \algebras{$\kf$} and if $\varphi \colon A \to B$ and $\psi \colon B \to C$ are homomorphisms of filtered \algebras{$\kf$} then $\psi \circ \varphi$ is also a homomorphism of filtered \algebras{$\kf$}. It follows that filtered \algebras{$\kf$} together with homomorphisms of filtered \algebras{$\kf$} between them form a category, which will be refered to by $\cFilt{k}$.
% \end{remark}
% 
% 
% \begin{examples}
%  \begin{enumerate}[leftmargin=*]
%   \item
%    Any graded \algebra{$\kf$} $A$ also carries the structure of an filtered \algebra{$\kf$} by setting $A_{(n)} \defined \bigoplus_{i \leq n} A_i$ for every $n \in \N$. If $\varphi \colon A \to B$ is a homomorphism of graded \algebras{$\kf$} then it is also a homomorphism of filtered \algebras{$\kf$}. Hence this construction results into a functor $\flt \colon \cGrad{k} \to \cFilt{k}$.
%   \item 
%    If $A$ is a filtered algebra and $I \subseteq A$ any two-sided ideal then $A/I$ is a filtered \algebra{$\kf$} via $(A/I)_{(n)} \defined \pi(A_{(n)})$ for every $n \in \N$, where $\pi \colon A \to A/I, a \mapsto a + I$ denotes the canonical projection.
%   \item
%    If $\glie$ is a $k$-Lie algebra then $\Univ(\glie)$ carries the structure of a filtered \algebra{$\kf$} induced by the filtration of $T(\g)$, which in turn is induced by the gradation of $T(\g)$. Explicitely
%    \[
%    \Univ(\glie)_{(n)}
%    = \vspan_k \{x_1 \dotsm x_m \mid m \in \N, m \leq n, x_1, \dotsc, x_m \in \g\}
%    \quad\text{for every $n \in \N$}.
%    \]
%  \end{enumerate}
% \end{examples}
% 
% 
% \begin{lemma}
%  Let $A$ be a graded \algebra{$\kf$} and set $B_n \defined A_n/A_{n-1}$ for every $n \in \N$. Then for all $n,m \in \N$ the map
%  \[
%   B_n \times B_m \to B_{n+m}, ([a], [b]) \mapsto [ab]
%  \]
%  is well-defined and bilinear.
% \end{lemma}
% \begin{proof}
%  Let $a, a' \in A_n$ and $b, b' \in A_m$ with $[a] = [a']$ and $[b] = [b']$. Then $ab, a'b' \in A_{n+m}$ and because $[a] = [a']$ and $[b] = [b']$ it follows that $a-a' \in A_{n-1}$ and $b-b' \in A_{m-1}$. Therefore
%  \[
%   ab
%   = (a'+(a-a'))(b'+(b-b'))
%   = a'b' + \underbrace{(a-a')b'}_{\in A_{n+m-1}} + \underbrace{a'(b-b')}_{\in A_{n+m-1}} + \underbrace{(a-a')(b-b')}_{\in A_{n+m-2}}
%  \]
%  and thus $[ab] = [a'b']$.
% \end{proof}
% 
% 
% \begin{definition}
%  Let $A$ be a filtered \algebra{$\kf$}. Its \emph{associated graded \algebra{$\kf$}} is the graded \algebra{$\kf$} consisting of the underlying vector space $\gr(A) \defined \bigoplus_{n \in \N} \gr_n(A)$ with $\gr_i(A) \defined A_{(i)} / A_{(i-1)}$ for every $i \in \N$ and the multiplication $\gr(A) \times \gr(A) \to \gr(A)$ induced by the well-defined bilinear maps
%  \[
%   \gr_n(A) \times \gr_m(A) \to \gr_{n+m}(A), \quad
%   ([a],[b]) \to [ab]
%   \quad\text{for all $n,m \in \N$}.
%  \]
%  together with the grading given by $\gr(A)_n \defined \gr_n(A)$ for every $n \in \N$.
% \end{definition}
% 
% 
% \begin{remark}
%  If $A$ and $B$ are filtered \algebras{$\kf$} and $\varphi \colon A \to B$ is a homomorphism of filtered \algebras{$\kf$} then $\varphi$ induces a $k$-linear map $\varphi_n \colon \gr(A)_n \to \gr(B)_n, [a] \mapsto [\varphi(a)]$ for every $n \in \N$, which result in a homomorphism of graded \algebras{$\kf$}
%  \[
%   \gr(\varphi) \defined \bigoplus_{n \in \N} \varphi_n \colon \gr(A) \to \gr(B),
%   \quad
%   \sum_{n \in \N} a_n \mapsto \sum_{n \in \N} \varphi(a_n).
%  \]
%  Hence $\gr$ can be seen as a functor $\gr \colon \cFilt{k} \to \cGrad{k}$.
% \end{remark}
% 
% 
% \begin{examples}
%  \begin{enumerate}[leftmargin=*]
%   \item
%    If $A$ is a graded \algebra{$\kf$} then $\gr(\flt(A))$ is naturally isomorphic to $A$ in the following way: The filtration on $\flt(A)$ is given by $A_{(n)} = \bigoplus_{i \leq n} A_i$ for every $n \in \N$. Hence there is for every $n \in \N$ an isomorphism of vector spaces
%    \[
%     \varphi_n \colon A_n \to A_{(n)}/A_{(n-1)} = \gr(\flt(A))_n, \quad a \mapsto [a].
%    \]
%    Combining these isomorphisms results in an isomorphism of graded \algebras{$\kf$}
%    \[
%     \bigoplus_{n \in \N} \varphi_n \colon
%     \bigoplus_{n \in \N} A_n \to \bigoplus_{n \in \N} \gr(\flt(A))_n, \quad
%     \sum_{n \in \N} a_n \mapsto \sum_{n \in \N} [a_n].
%    \]
%    We will therefore identify $A$ with $\gr A$ in the above way.
%   \item
%    Let $\glie$ be a $k$-Lie algebra. The canonical projection
%    \[
%     \pi \colon T(\g) \to \Univ(\glie), \quad x_1 \tensor \dotsb \tensor x_n \to x_1 \dotsm x_n
%     \quad \text{for every $n \in \N$ and all $x_1, \dotsc, x_n \in \g$}
%    \]
%    is a homomorphism of filtered \algebras{$\kf$}, where the filtration of $T(\g)$ is induced by the gradation discussed in Examples \ref{expls: graded algebras}. Hence it induces a homomorphism of graded \algebras{$\kf$}
%    \[
%     \gr(\pi) \colon T(\g) \to \gr(\Univ(\glie)),
%    \]
%    where $T(\g)$ is identified with $\gr(T(\g))$ as above. This homomorphism maps an element $x_1 \tensor \dotsb \tensor x_n$ with $x_1, \dots, x_n \in \g$ to the residue class $[x_1 \dotsm x_n] \in \gr(\Univ(\glie))_n$.
%  \end{enumerate}
% \end{examples}
% 
% 
% \begin{proposition}\label{prop: associated graded algebra and zero divisors}
%  Let $A$ be a filtered \algebra{$\kf$}. If $\gr(A)$ is an integral domain then so is $A$.
% \end{proposition}
% \begin{proof}
%  Suppose $A$ is no integral domain. Then there exist $a,b \in A$ with $a \neq 0$ and $b \neq 0$ but $ab = 0$. Then there exists a minimal $n \in \N$ with $a \in A_{(n)}$ and a minimal $b \in \N$ with $b \in A_{(m)}$. By the minimality of $n$ and $m$ it follows that $[a] \in \gr(A)_n$ and $[b] \in \gr(A)_m$ are nonzero residue classes with $[a] \cdot [b] = [ab] = 0$. Hence $\gr(A)$ is no integral domain.
% \end{proof}
% 
% 
% \begin{remark}
%  The converse of Proposition~\ref{prop: associated graded algebra and zero divisors} is not true, i.e.\ if $A$ is a filtered \algebra{$\kf$} which is an integral domain, then $\gr(A)$ is not necessarily an integral domain.
%  
%  To see this let $A$ be any \algebra{$\kf$} with $k \subsetneq A$ and with filtration
%  \[
%   A_{(0)} \defined k
%   \quad\text{and}\quad
%   A_{(n)} \defined A
%   \quad\text{for every $n \geq 1$}.
%  \]
%  Then $\gr(A)_0 = k$, $\gr(A)_1 = A/k \neq 0$ and $\gr(A)_n = 0$ for every $n \geq 2$, hence $\gr(A)_1 \gr(A)_1 = 0$. So $\gr(A)$ is no integral domain, even if $A$ is.
% \end{remark}
% 
% 
% 
% 
% 
% \subsection{The Poincar\'{e}--Birkhoff--Witt theorem (concrete version)}
% For this subsection we fix some $k$-Lie algebra $\glie$ with basis $(x_i)_{i \in I}$ where $(I, \leq)$ is a totally ordered index set. Before stating and proving the \emph{Poincar\'{e}--Birkhoff--Witt theorem} (PBW) we fix some notation which we will only use in this subsection.
% 
% 
% \begin{definition}
%  Let $\mc{I}_n \defined \{(i_1, \dotsc, i_n) \mid i_1, \dotsc, i_n \in I, \; i_1 \leq \dotsb \leq i_n\}$ for every $n \in \N$ and set $\mc{I} \defined \bigcup_{n \in \N} \mc{I}_n$. For every $\alpha = (i_1, \dotsc, i_n) \in I^n$ with $n \in \N$ let $x_\alpha \defined x_{i_1} \dotsm x_{i_n} \in \Univ(\glie)$.
% \end{definition}
% 
% 
% \begin{remark}
%  Notice that $\mc{I}_0$ contains the empty tupel.
% \end{remark}
% 
% 
% \begin{theorem}[PBW (concrete version)] \label{thrm: pbw concrete}
%  The familiy $(x_\alpha \mid \alpha \in \mc{I})$ is a $k$-basis of $\Univ(\glie)$.
% \end{theorem}
% 
% 
% \begin{remark}
%  The basis $(x_\alpha \mid \alpha \in \mc{I})$ can also be written as
%  \[
%   \left(
%    x_{i_1}^{p_1} \dotsm x_{i_n}^{p_n}
%   \mid
%    n \in \N,\;
%    i_1, \dotsc, i_n \in I,\;
%    i_1 < \dotsb < i_n,\;
%    p_1, \dotsc, p_n \geq 1
%   \right).
%  \]
% \end{remark}
% 
% 
% \begin{example}
%  If $\glie$ is a finite dimensional $k$-Lie algebra with basis $x_1, \dotsc, x_n$ then $\Univ(\glie)$ has a basis given by $(x_1^{p_1} \dotsm x_n^{p_n} \mid p_1, \dotsc, p_n \in \N)$. In particular a basis of $\Univ(\sll_2(k))$ is given by $(e^\ell h^m f^n \mid \ell, m ,n \in \N)$.
% \end{example}
% 
% 
% \begin{lemma}\label{lem: pbw concrete generating part}
%  The collection $(x_\alpha \mid \alpha \in \mc{I})$ generates $\Univ(\glie)$ as a vector space.
% \end{lemma}
% \begin{proof}
%  To show that $(x_\alpha \mid \alpha \in \mc{I})$ generates $\Univ(\glie)$ as a vector space it sufficies to show that $\mc{B}_n \defined (x_\alpha \mid m \leq n, \alpha \in \mc{I}_m)$ generates $\Univ(\glie)_{(n)}$ as a vector space, which will be shown by induction over $n \in \N$: For $n = 0$ it holds because $\Univ(\glie) = k$ is one-dimensional and thus spanned by $x_{(\;)}$, the monomial corresponding to the empty tupel.
%  
%  Suppose that the statement holds for some $n \in \N$. Then $\Univ(\glie)_{(n+1)}$ is generated as a vector space as a by the monomials $(x_{(i_1, \dotsc, i_n)} \mid m \leq n+1, \; i_1, \dotsc, i_m \in I)$. Therefore it sufficies to express these monomials in terms of $\mc{B}_{n+1}$. By induction hypothesis is it enough to check this for the monomials $(x_{(i_1, \dotsc, i_{n+1})} \mid i_1, \dotsc, i_{n+1} \in I)$. For this let $\alpha = (i_1, \dotsc, i_{n+1})$ be some fixed multiindex with $i_1, \dotsc, i_{n+1} \in I$.
%  
%  For any two $x,y \in \g$ one has $xy = yx + [x,y]$ with $[x,y] \in \Univ(\glie)_{(1)}$. Hence there exists for any permutation $\sigma \in S_{n+1}$ a linear combination $R_\sigma \in \Univ(\glie)_{(1)}$ with
%  \[
%   x_\alpha
%   = x_{i_1} \dotsm x_{i_{n+1}}
%   = x_{i_{\sigma(1)}} \dotsm x_{i_{\sigma(n+1)}} + R_\sigma
%   = x_{(i_{\sigma(1)}, \dotsc, x_{\sigma(n+1)})} + R_\sigma.
%  \]
%  Let $\sigma \in S_{n+1}$ be a permutation with $i_{\sigma(1)} \leq \dotsb \leq i_{\sigma(n+1)}$. By induction hypothesis $R_\sigma \in \Univ(\glie)_{(n)}$ can be expressed as a linear combination of the monomials $\mc{B}_n$. Hence $x_\alpha = x_{(i_{\sigma(1)}, \dotsc, i_{\sigma(n+1)})} + R_\sigma$ can be expressed as a linear combination of the monomials $\mc{B}_{n+1}$ because $x_{(i_{\sigma(1)}, \dotsc, i_{\sigma(n+1)})}$ is one of them.
% \end{proof}
% 
% 
% \begin{proof}[Proof of PBW (concrete version)]
%  By Lemma~\ref{lem: pbw concrete generating part} the collection $(x_\alpha \mid \alpha \in \mc{I})$ generates $\Univ(\glie)$ as a vector space, so all that’s left to show is that it is linearly independent.
%  
%  Let $V \defined k[Z_i \mid i \in I]$ and for every $n \in \N$ let $V_{(n)}$ be the polynomials of degree $\leq n$. For every $n \in \N$ and $\alpha = (i_1, \dotsc, i_n) \in I^n$ write $Z_\alpha \defined Z_{i_1} \dotsm Z_{i_n}$. If $i \in I$ and $\alpha = (i_1, \dotsc, i_n) \in I^n$ then write $i \leq \alpha$ if $i \leq i_j$ for every $j = 1, \dotsc, n$. Also set $i \cdot \alpha = (i, i_1, \dotsc, i_n) \in I^{n+1}$.
%  
%  To show that $(x_\alpha \mid \alpha \in \mc{I})$ is linearly independent $V$ will be given the structure of a representation of $\glie$ such that
%  \[
%   x_i.Z_\alpha = Z_{i \cdot \alpha}
%   \quad \text{for every $i \in I$ and $\alpha \in \mc{I}$ with $i \leq \alpha$}.
%  \]
%  Then for the corresponding $\Univ(\glie)$-module structure on $V$ it follows that
%  \[
%   x_\alpha \cdot 1
%   = x_\alpha \cdot Z_{(\;)}
%   = Z_\alpha
%   \quad \text{for every $\alpha \in \mc{I}$}
%  \]
%  where $1 \in V = k[Z_i \mid i \in I]$ and $(\;)$ denotes the empty tupel. Because $(Z_\alpha \mid \alpha \in \mc{I})$ is linearly independent it then follows that $(x_\alpha \mid \alpha \in \mc{I})$ is linearly independent. The existence of such an action follows from the following:
%  
%  \begin{claim*}
%   There exists a unique sequence $(\varphi_n)_{n \in \N}$ of bilinear maps
%   \[
%    \varphi_n \colon \glie \times V_{(n)} \to V_{(n+1)}, \quad (x,p) \mapsto x.p
%   \]
%   satisfying the following conditions:
%   \begin{enumerate}
%    \item\label{enum: pbw restriction coincides}
%     The restriction of $\varphi_{n+1}$ to $\glie \times V_{(n)}$ coincides with $\varphi_n$ for every $n \in \N$
%    \item\label{enum: pbw essential condition}
%     $x_i.Z_\alpha = Z_{i \cdot \alpha}$ for every $i \in I$ and $\alpha \in \mc{I}$ with $i \leq \alpha$.
%    \item\label{enum: pbw representation of lie algebra}
%     $x_i.x_j.Z_\alpha - x_j.x_i.Z_\alpha = [x_i, x_j].Z_\alpha$ for all $i,j \in I$ and every $\alpha \in \mc{I}$.
%    \item\label{enum: pbw technical detail for construction}
%     $x_i.Z_\alpha - Z_{i \cdot \alpha} \in V_{(n)}$ for every $n \in \N$, $i \in I$ and $\alpha \in \mc{I}_n$.
%    \end{enumerate}
%    (Condition \ref{enum: pbw restriction coincides} actually follows from the other conditions by the uniqueness of the sequence $(\varphi_n)_{n \in \N}$. See \cite[\S 17.4]{Humphreys} for more details.)
%  \end{claim*}
%  \begin{proof}
%   Notice thet the notation $x.p$ with $x \in \g$ and $p \in V$ is unambiguous by condition \ref{enum: pbw restriction coincides}. The maps $\varphi_n$ will be defined by induction over $n$:
%   
%   As $V_{(0)}$ is one-dimensional and spanned by $1 = Z_{(\;)}$ it follows from condition \ref{enum: pbw essential condition} that $x_i.1 = x_i \cdot Z_{(\;)} = Z_{i \cdot (\;)} = Z_i$ for every $i \in I$. This defines $\varphi_0$ uniquely. Conditions \ref{enum: pbw essential condition} and \ref{enum: pbw technical detail for construction} hold by construction and the conditions \ref{enum: pbw restriction coincides} and \ref{enum: pbw representation of lie algebra} do not affect $\varphi_0$.
%   
%   Let $n \in \N$ and suppose $\varphi_m$ is constructed for every $m \leq n$. By condition \ref{enum: pbw restriction coincides} all that is left to define is $x_i.Z_\alpha$ for $\alpha \in \mc{I}_{n+1}$. If $i \leq \alpha$ then $x_i.Z_\alpha = Z_{i \cdot \alpha}$ by condition \ref{enum: pbw essential condition}.
%   
%   If $i > \alpha$ then there exists $\beta \in \mc{I}_n$ and $j \in I$ with $\alpha = j \cdot \beta$ such that $i > j$ and $j \leq \beta$. If condition \ref{enum: pbw representation of lie algebra} was to hold for $\varphi_{n+1}$ it follows that
%   \begin{equation}\label{eqn: action defined as lie action}
%    x_i.Z_\alpha
%    = x_i.x_j.Z_\beta
%    = x_j.x_i.Z_\beta + [x_i, x_j].Z_\beta.
%   \end{equation}
%   Because $\beta \in \mc{I}_n$ the term $[x_i, x_j].Z_\beta$ in the above sum is already defined. Because $x_i.Z_\beta \equiv Z_{i \cdot \beta} \pmod{V_{(n)}}$ there exists some $Y \in V_{(n)}$ with $x_i.Z_\beta = Z_{i \cdot \beta} + Y$. Let $\gamma \in \mc{I}_{n+1}$ be defined by taking $\beta$ and inserting $i$ at the right position. Then $Z_\gamma = Z_{i \cdot \beta}$. Because $j < i$ and $j \leq \beta$ it follows that $j \leq \gamma$ and therefore 
%   \[
%    x_j.Z_{i \cdot \beta}
%    = x_j.Z_\gamma
%    = Z_{i \cdot \gamma}
%    = Z_{i \cdot (j \cdot \beta)}
%    = Z_{i \cdot \alpha}
%   \]
%   Because every summand in
%   \begin{equation}\label{eqn: I really wish Humphreys had explained this}
%    \begin{aligned}
%     x_i.Z_\alpha
%     = x_j.x_i.Z_\beta + [x_i, x_j].Z_\beta
%     &= x_j.(Z_{i \cdot \beta} + Y) + [x_i, x_j].Z_\beta \\
%     &= Z_{i \cdot \alpha} + x_j.Y + [x_i, x_j].Z_\beta
%    \end{aligned}
%   \end{equation}
%   is defined it follows that $\varphi_{n+1}$ is uniquely defined.
%   
%   Conditions \ref{enum: pbw restriction coincides} and \ref{enum: pbw essential condition} hold for $\varphi_{n+1}$ by construction. Condition \ref{enum: pbw technical detail for construction} holds for $i \leq \alpha$ by condition \ref{enum: pbw essential condition} and for $i > \alpha$ by \eqref{eqn: I really wish Humphreys had explained this} because $x_j.Y \in V_{(n+1)}$ and $[x_i, x_j].Z_\beta \in V_{(n+1)}$.
%   
%   It remains to check Condition \ref{enum: pbw representation of lie algebra} for $\varphi_{n+1}$, i.e.\ when $i,j \in I$ and $\alpha \in \mc{I}_n$. For $i = j$ this follows from the Lie bracket being alternating. Suppose that $i \neq j$. By the Lie bracket is antisymmetric it can be w.l.o.g.\ assumed that $i < j$.
%   
%   If $i \leq \alpha$ then $x_j.x_i.Z_\alpha$ is defined above by using \eqref{eqn: action defined as lie action} (where $\beta$ has to be replaced by $\alpha$ and $i$ and $j$ have to be switched), hence condition \ref{enum: pbw essential condition} holds in this case by construction. Notice that if $j \leq \alpha$ then also $i \leq \alpha$ because $i < j$.
%   
%   Hence the only case left is $i \nleq \alpha$. By the above it then follows that also $j \nleq \alpha$. As this cannot happen for $n = 0$ it can be w.l.o.g.\ assumed that $n \geq 1$. Let $k \in I$ and $\beta \in \mc{I}_{n-1}$ with $\alpha = k \cdot \beta$. Because condition \ref{enum: pbw representation of lie algebra} holds for $\varphi_n$ it follows that
%   \begin{align*}
%    x_i.x_j.Z_\alpha
%    = x_i.x_j.x_k.Z_\beta
%    &= x_i.(x_k.x_j.Z_\beta + [x_j, x_k].Z_\beta) \\
%    &= x_i.x_k.x_j.Z_\beta + x_i.[x_j, x_k].Z_\beta
%   \end{align*}
%   Because $k < j$ and $k \leq \beta$ it follows from the previous discussed cases that
%   \[
%    x_i.x_k.(x_j.Z_\beta)
%    = x_k.x_i.(x_j.Z_\beta) + [x_i, x_k].(x_j.Z_\beta).
%   \]
%   Combining the above results in the equality
%   \[
%    x_i.x_j.Z_\alpha
%    = x_k.x_i.x_j.Z_\beta + [x_i, x_k].x_j.Z_\beta + x_i.[x_j, x_k].Z_\beta
%   \]
%   By switching $i$ and $j$ in the above calculations it also follows that 
%   \[
%    x_j.x_i.Z_\alpha
%    = x_k.x_j.x_i.Z_\beta + [x_j, x_k].x_i.Z_\beta + x_j.[x_i, x_k].Z_\beta
%   \]
%   By using that condition \ref{enum: pbw representation of lie algebra} holds for $\varphi_n$ it follows from these two equalities and the Jacobi identity that
%   \begin{align*}
%         x_i.x_j.Z_\alpha - x_j.x_i.Z_\alpha 
%    =&\, x_k.x_i.x_j.Z_\beta + [x_i, x_k].x_j.Z_\beta + x_i.[x_j, x_k].Z_\beta \\
%     &\, - x_k.x_j.x_i.Z_\beta - [x_j, x_k].x_i.Z_\beta - x_j.[x_i, x_k].Z_\beta \\
%    =&\, x_k.[x_i, x_j].Z_\beta + [[x_i, x_k], x_j].Z_\beta +  [x_i,[x_j,x_k]].Z_\beta \\
%    =&\, x_k.[x_i, x_j].Z_\beta - [x_k, [x_i, x_j]].Z_\beta
%    =    [x_i, x_j].x_k.Z_\beta
%    =    [x_i, x_j].Z_\alpha.
%   \qedhere
%   \end{align*}
%  \end{proof}
%  This finishes the proof. 
% \end{proof}
% 
% 
% \begin{corollary}
%  Let $\glie$ be a Lie algebra and $\h, \n \subseteq \g$ Lie subalgebras with $\glie = \h \oplus \n$ as vector spaces. Then the map
%  \[
%   \Univ(\h) \tensor \Univ(\h) \to \Univ(\glie), \quad x \tensor y \mapsto xy
%  \]
%  is a isomorphism of vector spaces.
% \end{corollary}
% \begin{proof}
%  Let $(x_i)_{i \in I}$ is a basis of $\h$ and $(x_j)_{j \in J}$ a basis of $\n$. Then $(x_k)_{k \in K}$ for the index set $K \defined I \dotcup J$ is a basis of $\glie$. Then the statement follows directly from the concrete PBW theorem.
% \end{proof}
% 
% 
% 
% \subsection{The Poincar\'{e}--Birkhoff--Witt theorem (abstract version)}
% 
% 
% \begin{theorem}[PBW (abstract version)] \label{thrm: pbw abstract}
%  Let $\glie$ be a Lie algebra over $k$ and denote by $\pi$ the canonical projection
%  \[
%   \pi \colon T(\g) \to \Univ(\glie), \quad
%   x_1 \tensor \dotsb \tensor x_n \mapsto x_1 \dotsm x_n
%   \quad \text{for all $x_1, \dotsc, x_n \in \g$}.
%  \]
%  Then the two homomorphisms of graded \algebras{$\kf$} $\gr(\pi) \colon T(\g) \to \gr(\Univ(\glie))$ and
%  \[
%   \pi' \colon T(\g) \to S(\g), \quad 
%   x_1 \tensor \dotsb \tensor x_n \mapsto x_1 \dotsm x_n
%   \quad \text{for all $x_1, \dotsc, x_n \in \g$}
%  \]
%  have the same kernel and thus induce an isomorphism of graded algebras
%  \begin{equation}\label{eqn: induced isomorphism of graded algebras}
%   \varphi \colon S(\g) \to \gr(\Univ(\glie)), \quad
%   x_1 \dotsm x_n \mapsto [x_1 \dotsm x_n]
%   \quad \text{for all $x_1, \dotsc, x_n \in \g$}.
%  \end{equation}
% \end{theorem}
% 
% 
% \begin{remark}
%  Notice that the two multiplications in \eqref{eqn: induced isomorphism of graded algebras} live in different \algebras{$\kf$}.
% \end{remark}
% 
% 
% \begin{proposition}
%  The concrete version and the abstract versions of the PBW-theorem are equivalent.
% \end{proposition}
% \begin{proof}
%  For $x, y \in \g$ it follows from the definition of $\gr(\pi)$ that
%  \[
%   \gr(\pi)(x \tensor y - y \tensor x) = [xy-yx] \in \gr(\Univ(\glie))_2,
%  \]
%  with representative $xy-yx \in \Univ(\glie)_{(2)}$. By the definition of $\Univ(\glie)$ it follows that already \mbox{$xy-xy = [x,y] \in \Univ(\glie)_{(1)}$}. Hence it follows for the residue class of $xy-yx$ in $\gr(\Univ(\glie))_2 = \gr(\Univ(\glie))_{(2)}/\gr(\Univ(\glie))_{(1)}$ that $[xy-yx] = [[x,y]] = [0] = 0$. Hence $\gr(\pi)(x \tensor y - y \tensor x) = 0$.
%  
%  As the kernel of $\pi'$ is generated by the element $x \tensor y - y \tensor x$ with $x,y \in \g$ it follows that $\pi'$ factorizes through a homomorphism of graded \algebras{$\kf$} $\varphi$ as in Theorem~\ref{thrm: pbw abstract}.
%  
%  (concrete $\Rightarrow$ abstract) The algebra $\Univ(\glie)$ has a basis $(x_\alpha \mid \alpha \in \mc{I})$ It follows that $\gr(\Univ(\glie))_n$ has a basis given by the residue classes $([x_\alpha] \mid \alpha \in \mc{I}_n)$. The linear subspace $S(\g)_n$ has a basis $(x_1 \dotsm x_n \mid (i_1, \dotsc, i_n) \in \mc{I}_n)$ which is mapped by $\varphi_n$ to the above basis of $\gr(\Univ(\glie))_n$. Hence $\varphi_n$ is an isomorphism for every $n \in \N$, which is why $\varphi$ is an isomorphism.
%  
%  (abstract $\Rightarrow$ concrete) As $(x_\alpha \mid \alpha \in \mc{I})$ generates $\Univ(\glie)$ as a vector spaces by Lemma~\ref{lem: pbw concrete generating part} it sufficies to show that this collection is linearly independent. Suppose otherwise. Then there exists some minimal $n \in \N$ such that $(x_\alpha \mid m \leq n, \alpha \in \mc{I}_m)$ is linearly dependent. Hence there exists a non-trivial linear combination
%  \[
%   0 = \sum_{m=0}^n \sum_{\alpha \in \mc{I}_m} \lambda_\alpha x_\alpha
%  \quad
%  \text{where $\lambda_\alpha = 0$ for all but finitely many $\alpha \in \bigcup_{m=0}^n \mc{I}_m$}.
%  \]
%  From this it follows that
%  \[
%   0
%   = \sum_{m=0}^n \sum_{\alpha \in \mc{I}_m} \lambda_\alpha x_\alpha
%   \equiv \sum_{\alpha \in \mc{I}_n} \lambda_\alpha x_\alpha
%   \mod \Univ(g)_{(n-1)}
%  \]
%  and hence that the equality $\sum_{\alpha \in \mc{I}_n} \lambda_\alpha [x_\alpha] = 0$ holds in $\gr(\Univ(\glie))_n = \Univ(\glie)_{(n)}/\Univ(\glie)_{(n-1)}$. By the minimality of $n$ it follows that this is a non-trivial linear combination in $\gr(\Univ(\glie))_n$, so $([x_\alpha] \mid \alpha \in \mc{I}_n)$ in $\gr(\Univ(\glie))_n$ is linearly dependent.
%  
%  By assumption $\varphi$ is a homomorphism of graded \algebras{$\kf$} and therefore $\varphi_n$ is an isomorphism of vector spaces. As the basis $(x_{i_1} \dotsm x_{i_n} \mid (i_1, \dotsc, i_n) \in \mc{I}_n)$ of $S(\g)_n$ is mapped by $\varphi_n$ bijectively to $([x_\alpha] \mid \alpha \in \mc{I}_n)$ it follows that this is a basis of $\gr(\Univ(\glie))_n$, contradicting the linearly dependency.
% \end{proof}
% 
% 
% \begin{corollary}
%  The {\ua} $\Univ(\glie)$ is an integral domain.
% \end{corollary}
% \begin{proof}
%  Because $\gr(\Univ(\glie)) \cong S(\g)$ is in integral domain the statement follows from Proposition~\ref{prop: associated graded algebra and zero divisors}.
% \end{proof}
% 
% 
% 
% 
% 
% \section{Free Lie algebras}
% 
% 
% \begin{definition}
%  Let $X$ be a set. A \emph{free $k$-Lie algebra on $X$} is a Lie algebra $F(X)$ together with a map $\iota \colon X \to F(X)$ such that for every Lie algebra $\glie$ and map $\phi \colon X \to \g$ there exists a unique homomorphism of Lie algebras $\Phi \colon F(X) \to \g$ with $\phi = \Phi \circ \iota$, i.e.\ making the following diagram commute:
%  \[
%    \begin{tikzcd}
%      X
%      \arrow{dr}[above right]{\phi}
%      \arrow{d}[left]{\iota}
%      &
%      {}
%      \\
%      F(X)
%      \arrow[dashed]{r}[below]{\Phi}
%      &
%      \g
%    \end{tikzcd}
%  \]
% \end{definition}
% 
% 
% \begin{remark}
%  As usual with free objects it follows that any two free Lie algebras over a set $X$ are unique up to unique isomorphism, i.e.\ if $F(X)_1$ with $\iota_1 \colon X \to F(X)_1$ and $F(X)_2$ with $\iota_2 \colon X \to F(X)_2$ are two free Lie algebras over $X$ then there exists a unique isomorphism of Lie algebras $\varphi \colon F(X)_1 \to F(X)_2$ making the following diagram commute:
%  \[
%    \begin{tikzcd}
%      {}
%      &
%      X
%      \arrow{dl}[above left]{\iota_1}
%      \arrow{dr}[above right]{\iota_2}
%      &
%      {}
%      \\
%      F(X)_1
%      \arrow[dashed]{rr}[below]{\varphi}
%      &
%      {}
%      &
%      F(X)_2
%    \end{tikzcd}
%  \]
%  We will therefore always talk about \emph{the} free $k$-Lie algebra over $X$.
% \end{remark}
% 
% 
% \begin{lemma}\label{lem: existince of free Lie algebras}
%  Let $X$ be a set. Then there exists a free Lie algebra over $X$.
% \end{lemma}
% \begin{proof}
%  Let $A(X)$ be the free (unitary and associative) \algebra{$\kf$} over $X$ (which can be constructed as $T(kX)$, i.e.\ the tensor algebra over the free vector space $kX$ with basis $X$). Let $F(X)$ be the Lie subalgebra of $A(X)$ generated by $X$, i.e.
%  \[
%   F(X) = \bigcap \{\g \mid \text{$\glie \subseteq A(X)$ is a Lie subalgebra with $X \subseteq \g$}\}.
%  \]
%  Let $\glie$ be a $k$-Lie algebra and $\phi \colon X \to \g$ a map. By the universal property of the free \algebra{$\kf$} the map $\phi$ induces a homomorphism of \algebras{$\kf$} $\theta \colon A(X) \to \Univ(\glie)$ making the following diagram commute, where the vertical maps are the canonical inclusions:
%  \[
%    \begin{tikzcd}
%      X
%      \arrow{r}[above]{\phi}
%      \arrow{d}
%      &
%      \g
%      \arrow{d}
%      \\
%      F(X)
%      \arrow[dashed]{r}[above]{\theta}
%      &
%      \Univ(\glie)
%    \end{tikzcd}
%  \]
%  As $\theta(X) = \phi(X) \subseteq \g$ it follows that $X \subseteq \theta^{-1}(\g)$. Because $\theta^{-1}(\g)$ is a Lie subalgebra of $F(X)$ containing $X$ it follows that $\theta^{-1}(\g) = F(X)$ and therefore $\theta(F(X)) \subseteq \g$. Hence $\theta$ restricts to a map $\Phi \colon F(X) \to \g$. Because $\theta$ is a homomorphism of \algebras{$\kf$} it is in particular a homomorphism of Lie algebras and therefore the same goes for $\Phi$. This shows the existence.
%  
%  For the uniqueness notice that $F(X)$ is by definition generated by $X$ and hence any homomorphism of Lie algebras $\Psi \colon F(X) \to \g$ is uniquely determinad by the restriction $\Psi|_X$.
% \end{proof}
% 
% 
% \begin{remark}
%  The {\ua} is used in the proof of Lemma \ref{lem: existince of free Lie algebras} to ensure that any Lie algebra can be embedded into a \algebra{$\kf$} as a Lie subalgebra.
% \end{remark}
% 
% 
% \begin{remark}
%  Using the concept of free Lie algebras one can define Lie algebras by giving a set of generators $X$ and a set of relations $R \subseteq F(X)$. As an example the Lie algebra $\sll(k)$ can be defined by the generators $R \defined \{e,h,f\}$ with $e$, $h$, $f$ being pairwise different and the relations $R = \{[h,e]-2e, [h,f]+2f, [e,f]-h\}$, which can also be written as $[h,e] = 2e$, $[h,f] = -2f$ and $[e,f] = h$ as usual.
%  
%  More generally $\sll_{n+1}(k)$ for $n \geq 1$ can be defined as the Lie algebra generated by the $3n$ elements $\{e_i, f_i, h_i \mid i = 1, \dotsc, n\}$ together with the relations
%  \begin{align*}
%   [h_i, h_j] &= 0, \\
%   [h_i, e_j] &= a_{ij} e_j, \\
%   [h_i, f_j] &= -a_{ij} f_j, \\
%   [e_i, f_j] &= \delta_{ij} h_i,
%  \end{align*}
%  for all $i,j = 1, \dotsc, n$ and
%  \[
%   \ad(e_i)^{1-a_{ij}}(e_j) = 0
%   \quad\text{and}\quad
%   \ad(f_i)^{1-a_{ij}}(f_j) = 0
%   \quad \text{for $1 \leq i \neq j \leq n$},
%  \]
%  where the numbers $a_{ij}$ are for all $i,j = 1, \dotsc, n$ defined as
%  \[
%   a_{ij} =
%   \begin{cases}
%    \phantom{-}2 & \text{if $i = j$}, \\
%              -1 & \text{if $|i-j| = 1$}, \\
%    \phantom{-}0 & \text{otherwise}.
%   \end{cases}
%  \]
% \end{remark}
% 
% 
% \begin{lemma}
%  Let $X$ be any set and $F(X)$ the free Lie algebra over $X$. Then $\Univ(F(X))$ is the free \algebra{$\kf$} over $X$, where the canonical inclusion $X \inc \Univ(F(X))$ is given by composition of the canonical inclusions $X \inc F(X)$ and $F(X) \inc \Univ(F(X))$.
% \end{lemma}
% \begin{proof}
%  It will be shown that $\Univ(F(X))$ together with the canonical inclusion $X \inc \Univ(F(X))$ satisfies universal property of the free \algebras{$\kf$}. Let $A$ be a \algebra{$\kf$} and $\phi \colon X \to A$ a map. Then $\phi$ induces a unique homomorphism of Lie algebras $\psi \colon F(X) \to A$ by the universal property of the free Lie algebra. Then $\psi$ induces a unique homomorphism of \algebras{$\kf$} $\Psi \colon \Univ(F(X)) \to A$ by the universal property of the {\ua}. Hence the following diagram commutes, where the vertical maps denote the canonical inclusions.
%  \[
%    \begin{tikzcd}
%      X
%      \arrow{dr}[above right]{\phi}
%      \arrow{d}
%      &
%      {}
%      \\
%      F(X)
%      \arrow{r}[above]{\phi}
%      \arrow{d}
%      &
%      A
%      \\
%      \Univ(F(X))
%      \arrow{ur}[below right]{\Psi}
%      &
%      {}
%    \end{tikzcd}
%  \]
%  Leaving out the middle part of the diagram shows the existence. The uniqueness of $\Psi$ follows from the uniqueness of $\psi$.
% \end{proof}
% 
% 
% 
% 
% 
% %TODO: Hopf algebra structure
% 
% 
% 
% 
% 
% \section{Casimir elements}
% For this subsection we additionaly assume that $\glie$ is finite-dimensional. We also fix some bilinear form $\beta \colon \glie \times \glie \to k$ which is associative and non-degenerate.
% 
% 
% \begin{definition}\label{defi: definition of Casimir element}
%  Let $\varphi_1 \colon \glie \tensor \g^* \to \End_k(\g)$ and $\varphi_2 \colon \glie \to \g^*$ be the isomorphisms of vector spaces defined by
%  \[
%   \varphi_1(x \tensor \phi)(y) = \phi(y) x
%   \quad\text{and}\quad
%   \varphi_2(x) = \beta(x, \cdot)
%   \quad\text{for all $x,y \in \g$ and $\phi \in \g^*$}.
%  \]
%  Then the image of $1$ under the map
%  \begin{equation}\label{eqn: Casimir without coordinates}
%   k
%   \xrightarrow{\lambda \mapsto \lambda \id_\g}
%   \End_k(\g)
%   \xrightarrow{\varphi_1^{-1}}
%   \glie \tensor \g^*
%   \xrightarrow{\id_\g \tensor \varphi_2^{-1}}
%   \glie \tensor \g
%   \xrightarrow{x \tensor y \mapsto x y}
%   \Univ(\glie)
%  \end{equation}
%  is called the \emph{Casimir element of $\beta$} and denoted by $C_\beta$.
% \end{definition}
% 
% 
% \begin{lemma}
%  The Casimir element $C_\beta$ in central in $\Univ(\glie)$, i.e.\
%  \[
%   x C_\beta = C_\beta x \quad \text{for every $x \in \Univ(g)$}.
%  \]
% \end{lemma}
% \begin{proof}
%  Let $\varphi_1$ and $\varphi_2$ as in Definition \ref{defi: definition of Casimir element}.
%  
%  Because $\Univ(\glie)$ is generated by $\glie$ as a \algebra{$\kf$} it sufficies to show $C_\beta$ commutes with every $x \in \g$. Hence it is to show that
%  \[
%   [x,C_\beta] = 0 \quad \text{for every $x \in \g$},
%  \]
%  where $[\cdot,\cdot]$ denotes the Lie bracket in $\Univ(\glie)$. To see this notice that in \eqref{eqn: Casimir without coordinates} every map is a homomorphism of representations of $\glie$, where $\glie$ acts trivially on $k$, i.e. $x.\lambda = 0$ for every $x \in \g$ and $\lambda \in \g$.
%  
%  That the first map $k \to \End_k(\g)$ is a homomorphism of representations follows from the fact that $\glie$ acts trivially on $k$ and also trivially on the one-dimensional subspace $k \id_\g \subseteq \End_k(\g)$.
%  
%  That $\varphi_1$ is an isomorphism of representations is known from Propositon \ref{prop: list of homomorphism of representations}.
%  
%  That the third map $\glie \tensor \g^* \to \glie \tensor \g$ is a homomorphism of representations follows from Proposition \ref{prop: list of homomorphism of representations}, because the identity $\id_\g$ is a homomorphism of representations and and the isomorphism $\varphi_2$ is one by the associativity of $\beta$, as seen in Lemma \ref{lem: associative bilinear form induces homomorphism of representations}.
%  
%  That the fourth map $\psi \colon \glie \tensor \glie \to \Univ(\glie), x \tensor y \mapsto xy$ is a homomorphism of representations follows from direct calculation, because for all $x,y,z \in \g$
%  \begin{align*}
%   \psi(x.(y \tensor z))
%   &= \psi((x.y) \tensor z + y \tensor (x.z))
%   = (x.y)z + y(x.z) \\
%   &= [x,y]z + y[x,z]
%   = xyz - yxz + yxz - yzx
%   = xyz - yzx \\
%   &= [x,yz]
%   = x.(yz)
%   = x.\psi(y \tensor z).
%  \end{align*}
%  
%  Because every map in \eqref{eqn: Casimir without coordinates} is a homomorphism of representations it follows that their composition $\phi \colon k \to \Univ(\glie)$ is also a homomorphism of representations. Definition \ref{defi: definition of Casimir element} is then equivalent to $\phi(1) = C_\beta$. Because $\glie$ acts trivially on $k$ and $\phi$ is a homomorphism of representations it follows that $\glie$ also acts trivially on the span of $C_\beta$. In particular
%  \[
%   0 = x.C_\beta = [x,C_\beta] \quad \text{for every $x \in \g$}.
%   \qedhere
%  \]
% \end{proof}
% 
% 
% \begin{corollary}\label{cor: Casimir homomorphism of a representation}
%  Let $V$ be a representation of $\glie$ and $\Univ(\glie) \times V \to V, (x,v) \mapsto x \cdot v$ the corresponding $\Univ(\glie)$-module structure on $V$. Then the map
%  \[
%   C_\beta^V \colon V \to V, \quad v \mapsto C_\beta \cdot v = \sum_{i=1}^n x_i.x^i.v
%  \]
%  is an endomorphism of representations of $\glie$ (equivalently an endomorphism of $\Univ(\glie)$-modules).
% \end{corollary}
% \begin{proof}
%  Because $C_\beta \in Z(\Univ(\glie))$ it follows that for every $x \in \Univ(\glie)$ and $v \in V$
%  \[
%   x \cdot C_\beta^V(v)
%   = x \cdot C_\beta \cdot v
%   = C_\beta \cdot x \cdot v
%   = C_\beta^V(x \cdot v).
%   \qedhere
%  \]
% \end{proof}
% 
% 
% \begin{lemma}[Casimir in coordinates] \label{lem: casimir in coordinates}
%  Let $x_1, \dotsc, x_n$ be a basis of $\glie$ and $x^1, \dotsc, x^n$ the dual basis of $\glie$ with respect to $\beta$, i.e.\ $\beta(x_i, x^j) = \delta_{ij}$ for all $i,j = 1, \dotsc, n$. Then
%  \[
%   C_\beta = \sum_{i=1}^n x_i x^i.
%  \]
% \end{lemma}
% \begin{proof}
%  Let $\varphi_1$ and $\varphi_2$ as in Definition~\ref{defi: definition of Casimir element}. In \eqref{eqn: Casimir without coordinates} $1$ is mapped to $\id_\g$, which is then mapped to $\sum_{i=1}^n x_i \tensor x_i^*$, where $x_1^*, \dotsc, x_n^*$ denotes the dual basis of $\g^*$. As $\varphi_2(x^i) = x_i^*$ it follows that $\sum_{i=1}^n x_i \tensor x_i^*$ is then mapped to $\sum_{i=1}^n x_i \tensor x^i$, which is then further mapped to the element $\sum_{i=1}^n x_i x^i$ in $\Univ(\glie)$.
% \end{proof}
% 
% 
% \begin{remark}
%  Using Lemma \ref{lem: casimir in coordinates} it can be shown that $C_\beta$ is central in $\Univ(\glie)$ using coordinates: Let $x \in \g$ and $a_{ij}, b_{ij} \in k$ such that $[x,x_i] = \sum_{j=1}^n a_{ij} x_j$ and $[x,x^i] = \sum_{j=1}^n b_{ij} x^j$ for all $i = 1, \dotsc, n$. Then for all $i,j = 1, \dotsc, n$
%  \begin{align*}
%   a_{ij}
%   &= \sum_{k=1}^n a_{ik} \beta(x_k, x^j)
%   = \beta\left( \sum_{k=1}^n a_{ik} x_k, x^j \right)
%   = \beta([x, x_i], x^j)
%   = -\beta([x_i, x], x^j) \\
%   &= -\beta(x_i, [x, x^j])
%   = -\beta\left( x_i, \sum_{k=1}^n b_{jk} x^k \right)
%   = -\sum_{k=1}^n b_{jk} \beta(x_i, x^k)
%   = -b_{ji}.
%  \end{align*}
%  It follows that
%  \begin{align*}
%   x C_\beta - C_\beta x
%   &= \sum_{i=1}^n x x_i x^i - \sum_{i=1}^n x_i x^i x
%   = \sum_{i=1}^n [x, x_i] x^i - \sum_{i=1}^n x_i [x^i, x] \\
%   &= \sum_{i,j=1}^n a_{ij} x_j x^i + \sum_{i,j=1}^n b_{ij} x_i x^j
%   = \sum_{i,j=1}^n a_{ij} x_j x^i - \sum_{i,j=1}^n a_{ij} x_j x^i
%   = 0.
%  \end{align*}
% \end{remark}




