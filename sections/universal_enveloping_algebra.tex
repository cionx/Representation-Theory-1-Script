\chapter{The Universal Enveloping Algebra}


\begin{convention}
  For this section we fix the following conventions:
  
  We fix an arbitrary field~$\kf$.
  By a~{\algebra{$\kf$}} we always mean an associative and unitary one, and homomorphisms of~{\algebras{$\kf$}} have to respect the units.
  The resulting category of~{\algebras{$\kf$}} with homomorphisms of~{\algebra{$\kf$}} between them will be denoted by~$\cAlg{\kf}$.
  All Lie~algebras considered will have~$\kf$ as their ground field, unless otherwise stated.
  
  If~$A$ is a~{\algebra{$\kf$}} then by an~\defemph{\module{$A$}}\index{module} we mean a left unitial~{\module{$A$}}.
  The resulting category of~{\modules{$A$}} is denoted by~\gls*{module category}.
\end{convention}

% 
% \begin{fluff}
%   If~$A$ and~$B$ are two~{\algebras{$\kf$}} and~$\varphi \colon A \to B$ is a homomorphism of~{\algebras{$\kf$}} then~$\varphi$ induces for every other~{\algebra{$\kf$}}~$C$ a map~$\varphi^*_C \colon \Hom_{\cAlg{\kf}}(B,C) \to \Hom_{\cAlg{\kf}}(A,C)$.
%   These induced maps are compatible in the sense that for all~{\algebras{$\kf$}}~$C$ and~$D$ and every homomorphism of~{\algebras{$\kf$}}~$f \colon C \to D$ the following square diagram commutes:
%   \[
%     \begin{tikzcd}
%       \Hom_{\cAlg{\kf}}(A,C)
%       \arrow{r}[above]{\varphi^*_C}
%       \arrow{d}[left]{f_*}
%       &
%       \Hom_{\cAlg{\kf}}(B,C)
%       \arrow{d}[right]{f_*}
%       \\
%       \Hom_{\cAlg{\kf}}(A,D)
%       \arrow{r}[below]{\varphi^*_D}
%       &
%       \Hom_{\cAlg{\kf}}(B,D)
%     \end{tikzcd}
%   \]
%   If~$\varphi$ is an isomorphism then the induced map~$\varphi^*_C$ is for every~{\algebra{$\kf$}}~$C$ a bijection.
%   
%   The following \lcnamecref{yoneda lemma very weak version}, which is a weak form of Yoneda’s lemma applied to the category of~{\algebras{$\kf$}}, gives a converse to this observation.
% \end{fluff}
% 

\begin{lemma}[Yoneda for~{\algebras{$\kf$}}]
  \label{yoneda lemma very weak version}
  Let~$A$ and~$B$ be two~{\algebras{$\kf$}}.
  \begin{enumerate}
    \item
      If~$\varphi \colon A \to B$ is a homomorphism of~{\algebras{$\kf$}} then~$\varphi$ induces for every other~{\algebra{$\kf$}}~$T$ a map~$\varphi^*_T \colon \Hom_{\cAlg{\kf}}(B,T) \to \Hom_{\cAlg{\kf}}(A,T)$ given by~$h \mapsto h \circ \varphi$.%
      \footnote{The letter~\enquote{$T$} is choosen because~$T$ is a \enquote{test object}.}
      These induced maps are compatible in the sense that for any two~{\algebras{$\kf$}}~$T$ and~$U$ and every homomorphism of~{\algebras{$\kf$}}~$f \colon T \to U$ the following square diagram commutes:
      \[
        \begin{tikzcd}
          \Hom_{\cAlg{\kf}}(B,T)
          \arrow{r}[above]{\varphi^*_T}
          \arrow{d}[left]{f_*}
          &
          \Hom_{\cAlg{\kf}}(A,T)
          \arrow{d}[right]{f_*}
          \\
          \Hom_{\cAlg{\kf}}(B,U)
          \arrow{r}[below]{\varphi^*_U}
          &
          \Hom_{\cAlg{\kf}}(A,U)
        \end{tikzcd}
      \]
      If~$\varphi$ is an isomorphism then the induced map~$\varphi^*_T$ is for every~{\algebra{$\kf$}}~$T$ a bijection.
    \item
      \label{natural homomorphisms}
      Let conversely~$(\eta_T)_T$ be a family of maps~$\eta_T \colon \Hom_{\cAlg{\kf}}(B,T) \to \Hom_{\cAlg{\kf}}(A,T)$ where~$T$ runs through all~{\algebra{$\kf$}} such that these maps are compatible in the sense that for all~{\algebras{$\kf$}}~$T$ and~$U$ and every homomorphism of~{\algebras{$\kf$}}~$f \colon T \to U$ the square diagram
      \begin{equation}
        \label{natural square}
        \begin{tikzcd}
          \Hom_{\cAlg{\kf}}(B,T)
          \arrow{r}[above]{\eta_T}
          \arrow{d}[left]{f_*}
          &
          \Hom_{\cAlg{\kf}}(A,T)
          \arrow{d}[right]{f_*}
          \\
          \Hom_{\cAlg{\kf}}(B,U)
          \arrow{r}[below]{\eta_U}
          &
          \Hom_{\cAlg{\kf}}(A,U)
        \end{tikzcd}
      \end{equation}
      commutes.
      Then there exists an algebra homomorphism~$\varphi \colon A \to B$ such that~$\eta_T = \varphi^*_T$ for every~{\algebra{$\kf$}}~$T$.
      The homomorphism~$\varphi$ is unique and given by~$\varphi = \eta_B(\id_B)$.
    \item
      \label{onetoone correspondence morphisms and natural trans}
      The above constructions give a {\onetoone} correspondence
      \begin{align*}
        \left\{
          \begin{tabular}{@{}c@{}}
            algebra homomorphism \\
            $\varphi \colon A \to B$
          \end{tabular}
        \right\}
        &\longonetoone
        \left\{
          \begin{tabular}{@{}c@{}}
            families~$(\eta_T)_T$ of maps \\
            $\eta_T \colon \Hom_{\cAlg{\kf}}(B,T) \to \Hom_{\cAlg{\kf}}(A,T)$ \\
            satisfying~\eqref{natural square} for every algebra homo.~$f \colon T \to U$
          \end{tabular}
        \right\}  \,,
      \\
        \varphi
        &\longmapsto
        (\varphi^*_T)_T \,,
      \\
        \eta_B(\id_B)
        &\longmapsfrom
        (\eta_T)_T  \,.
      \end{align*}
    \item
      If in part~\ref*{natural homomorphisms} every map~$\eta_T$ is bijective then the corresponding algebra homomorphism~$\varphi \colon A \to B$ is bijective.
      The {\onetoone} correspondence from part~\ref*{onetoone correspondence morphisms and natural trans} hence restricts to a {\onetoone} correspondence.
      \[
        \left\{
          \begin{tabular}{@{}c@{}}
            algebra isomorphism \\
            $\varphi \colon A \to B$
          \end{tabular}
        \right\}
        \longonetoone
        \left\{
          \begin{tabular}{@{}c@{}}
            families~$(\eta_T)_T$ of bijections \\
            $\eta_T \colon \Hom_{\cAlg{\kf}}(B,T) \to \Hom_{\cAlg{\kf}}(A,T)$ \\
            satisfying~\eqref{natural square} for every algebra homo.~$f \colon T \to U$
          \end{tabular}
        \right\}  \,.
      \]
  \end{enumerate}
\end{lemma}


\begin{proof}
  \leavevmode
  \begin{enumerate}
    \item
      The maps~$\eta_T$ are well-defined because the composition of algebra homomorphism is again an algebra homomorphism.
      The square diagram~\eqref{natural square} commutes because for every~$ \in \Hom_{\cAlg{\kf}}(B,T)$,
      \[
        f_*( \varphi^*_T( h ) )
        =
        f_* (h \circ \varphi)
        =
        f \circ h \circ \varphi
        =
        \varphi^*_U(\spacing f \circ h)
        =
        \varphi^*_U(\spacing f_*(h))  \,.
      \]
    \item
      To see the claimed uniqueness of~$\varphi$ we observe that~$\eta_B(\id_B) = \varphi^*_B(\id_B) = \id_B \circ \varphi = \varphi$.
      To show the existence of~$\varphi$ we set~$\varphi \defined \eta_B(\id_B)$.
      Then~$\varphi \in \Hom_{\cAlg{\kf}}(A,B)$.
      If~$T$ is any~{\algebra{$\kf$}} then it follows for every~$f \in \Hom_{\cAlg{\kf}}(B,T)$ from the commutativity of the square diagram
      \[
        \begin{tikzcd}
          \Hom_{\cAlg{\kf}}(B,B)
          \arrow{r}[above]{\eta_B}
          \arrow{d}[left]{f_*}
          &
          \Hom_{\cAlg{\kf}}(A,B)
          \arrow{d}[right]{\spacing f_*}
          \\
          \Hom_{\cAlg{\kf}}(B,T)
          \arrow{r}[below]{\eta_T}
          &
          \Hom_{\cAlg{\kf}}(A,T)
        \end{tikzcd}
      \]
      that
      \[
        \eta_T(\spacing f)
        =
        \eta_T(\spacing f \circ \id_B)
        =
        \eta_T(\spacing f_*(\id_B))
        =
        f_*(\eta_B(\id_B))
        =
        f_*(\varphi)
        =
        f \circ \varphi \,.
      \]
    \item
      If~$\varphi \colon A \to B$ is any algebra homomorphism then~$\varphi^*_B(\id_B) = \id_B \circ \varphi = \varphi$.
      Together with part~\ref*{onetoone correspondence morphisms and natural trans} this shows that both constructions are inverse to each other.
    \item
      If~$\varphi$ is an isomorphism then~$\varphi^*_T$ is bijective for every~{\algebra{$\kf$}} because the maps~$\varphi^*_T$ and~$(\varphi^{-1})^*_T$ are mutually inverse.
      Suppose on the other hand that~$\varphi^*_T$ is a bijection for every~{\algebra{$\kf$}}~$T$.
      Then the family~$( (\varphi^*_T)^{-1} )_T$ makes for every algebra homomorphism~$f \colon T \to U$ the square diagram
      \[
        \begin{tikzcd}[column sep = large]
          \Hom_{\cAlg{\kf}}(A,T)
          \arrow{r}[above]{ (\varphi^*_T)^{-1} }
          \arrow{d}[left]{f_*}
          &
          \Hom_{\cAlg{\kf}}(B,T)
          \arrow{d}[right]{\spacing f_*}
          \\
          \Hom_{\cAlg{\kf}}(A,U)
          \arrow{r}[below]{ (\varphi^*_U)^{-1} }
          &
          \Hom_{\cAlg{\kf}}(B,U)
        \end{tikzcd}
      \]
      commute.
      There hence exists by part~\ref*{natural homomorphisms} a (unique) algebra homomorphism~$\psi \colon B \to A$ for which~$(\varphi^*_T)^{-1} = \psi^*_T$ for every~{\algebra{$\kf$}}~$T$.
      We find for the composition~$\psi \varphi \colon A \to A$ that
      \[
        (\psi \circ \varphi)^*_T
        =
        \varphi^*_T \circ \psi^*_T
        =
        \varphi^*_T \circ (\varphi^*_T)^{-1}
        =
        \id_{\Hom_{\cAlg{\kf}}(A,T)}
        =
        (\id_A)^*_T
      \]
      for every~{\algebra{$\kf$}}~$T$.
%       It hence follows from the commutativity of the diagram
%       \[
%         \begin{tikzcd}
%           \Hom_{\cAlg{\kf}}(A,T)
%           \arrow[bend left]{rr}[above]{\varphi^*_T \circ \psi^*_T}
%           \arrow{r}[above]{\psi^*_T}
%           \arrow{d}[left]{f_*}
%           &
%           \Hom_{\cAlg{\kf}}(B,T)
%           \arrow{r}[above]{\varphi^*_T}
%           \arrow{d}[left]{f_*}
%           &
%           \Hom_{\cAlg{\kf}}(A,T)
%           \arrow{d}[left]{f_*}
%           \\
%           \Hom_{\cAlg{\kf}}(A,U)
%           \arrow{r}[above]{\psi^*_U}
%           \arrow[bend right]{rr}[below]{\varphi^*_U \circ \psi^*_U}
%           &
%           \Hom_{\cAlg{\kf}}(B,U)
%           \arrow{r}[above]{\varphi^*_U}
%           &
%           \Hom_{\cAlg{\kf}}(A,U)
%         \end{tikzcd}
%       \]
%       that the outer square diagram
%       \[
%         \begin{tikzcd}[column sep = large]
%           \Hom_{\cAlg{\kf}}(A,T)
%           \arrow{r}[above]{(\psi \circ \varphi)^*_T}
%           \arrow{d}[left]{f_*}
%           &
%           \Hom_{\cAlg{\kf}}(A,T)
%           \arrow{d}[right]{\spacing f_*}
%           \\
%           \Hom_{\cAlg{\kf}}(A,U)
%           \arrow{r}[below]{(\psi \circ \varphi)^*_U}
%           &
%           \Hom_{\cAlg{\kf}}(A,U)
%         \end{tikzcd}
%       \]
%       commutes.
%       If we replace~$\psi \circ \varphi$ in this diagram by the identity~$\id_A$ then the resulting square diagram
%       \[
%         \begin{tikzcd}[column sep = large]
%           \Hom_{\cAlg{\kf}}(A,T)
%           \arrow{r}[above]{(\id_A)^*_T}
%           \arrow{d}[left]{f_*}
%           &
%           \Hom_{\cAlg{\kf}}(A,T)
%           \arrow{d}[right]{\spacing f_*}
%           \\
%           \Hom_{\cAlg{\kf}}(A,U)
%           \arrow{r}[below]{(\id_A)^*_U}
%           &
%           \Hom_{\cAlg{\kf}}(A,U)
%         \end{tikzcd}
%       \]
%       also commutes.
      Hence~$\psi \circ \varphi = \id_A$ by the uniqueness in part~\ref*{natural homomorphisms}.
      We find in the same way that also~$\varphi \circ \psi = \id_B$.
      This shows that~$\varphi$ and~$\psi$ are mutually inverse isomorphisms.
    \qedhere
  \end{enumerate}
\end{proof}


\begin{definition}
  A family~$(\eta_T)_T$ of maps~$\eta_T \colon \Hom_{\cAlg{\kf}}(B,T) \to \Hom_{\cAlg{\kf}}(A,T)$, where~$A$ and~$B$ are two~{\algebras{$\kf$}} and~$T$ ranges over all~{\algebras{$\kf$}}, is \defemph{natural}\index{natural} if the square diagram~\eqref{natural square} commutes for every algebra homomorphisms~$f \colon A \to B$.
\end{definition}


\begin{remark}
  \Cref{yoneda lemma very weak version} holds with the same proof for every kind of mathematical structure that has a suitable notion of homomorphisms between them, i.e.\ in every category.
  It is then know as the \defemph{Yoneda lemma}\index{Yoneda lemma}, which is one of the most important statement in all of category theory (and mathematics).
\end{remark}



\section{Tensor Algebra and Symmetric Algebra}


% \begin{example}[Monoid algebra]
%   In the following all monoids will be written multiplicaitvely unless otherwise mentioned.
%   The neutral element of a monoid~$M$ will be denoted by~$1$ or~$1_M$.
%   Given two monoids~$M$ and~$N$ a map~$f \colon M \to N$ is a homomorphism of monids if~$f(m \cdot m') = f(m) \cdot f(m')$ for all~$m, m' \in M$ and~$f(1_M) = 1_N$.
%   If~$M$ is any monoid then the identity~$\id_M$ is a homomorphism and if~$f \colon M \to N$ and~$g \colon N \to P$ are composable homomorphisms of monoids then their composition~$g \circ f \colon M \to P$ is again a homomorphism of monoids.
%   The resulting category of monoids is denoted by~$\cMon$.
%   
%   \begin{description}
%     \item[Construction:]
%       If~$M$ is a monoid then the monoid algebra~\gls*{monoid algebra} is the (free) vector space with basis~$M$ together with the unique bilinear extension~$\kf[M] \times \kf[M] \to \kf[M]$ of the multiplication~$M \times M \to M$ as its multiplication.
%   
%       This means that the elements of~$\kf[M]$ are formal {\linear{$\kf$}} combinations~$\sum_{m \in M} a_m m$ with~$a_m = 0$ for all but finitely many~$m \in M$.
%       The multiplication of two such elements is given by
%       \[
%         \left(
%           \sum_{m \in M} a_m m
%         \right)
%         \left(
%           \sum_{n \in M} b_n n
%         \right)
%         =
%         \sum_{m, n \in M} (a_m b_n) m n  \,.
%       \]
%       We identify every element~$m \in M$ with the corresponding element~$1 \cdot m \in \kf[M]$.
%       The product~$m \cdot n$ of two elements~$m, n \in M$ in~$\kf[M]$ is then the same as their product in~$M$.
%       The associativity of the multiplication of~$\kf[M]$ follows from the associativity of the multiplication of~$M$, and the neutral element of~$M$ is given by the multiplicative neutral element for~$\kf[M]$.
%       
%     \item[Universal Property:]
%       If~$A$ is any~{\algebra{$\kf$}} then~$(A, \cdot)$ is a multiplicative monoid, which we will denote by~$A^-$.
%       If~$M$ is any monoid and~$f \colon M \to A^-$ is a monoid hommorphism then~$f$ extends uniquely to an algebra homomorphism~$F \colon \kf[M] \to A$.
%       The algebra homomorphism~$F$ is given on elements by
%       \[
%         F\left( \sum_{m \in M} a_m m \right)
%         =
%         \sum_{m \in M} a_m f(m) \,.
%       \]
%       On the other hand every algebra homomorphism~$\kf[M] \to A$ restricts to a monoid homomorphism~$M \to A^-$.
%       This construction results in a {\onetoone} correspondence
%       \[
%         \{
%           \text{monoid homomorphisms~$M \to A^-$}
%         \}
%         \longonetoone
%         \{
%           \text{algebra homomorphisms~$\kf[M] \to A$}
%         \}  \,.
%       \]
%     
%     \item[Uniqueness:]
%       The monoid algebra~$\kf[M]$ together with the inclusion~$i \colon M \to \kf[M]$ is uniquely determined by its universal property up to isomorphism:
%   \end{description}
% \end{example}


% \begin{recall}[Free algebra]
%   Let~$I$ be any set.
%   The \defemph{noncommutative polynomial algebra}~$\kf\gen{X_i \suchthat i \in I}$\index{noncommutative polynomial algebra} has as a basis the set of all monomials
%   \[
%     X_{i_1} \dotsm X_{i_n}
%     \qquad
%     \text{with~$i_1, \dotsc, i_n$}
%   \]
%   and the multiplication is on these basis elements given by
%   \[
%     X_{i_1} \dotsm X_{i_n}
%     \cdot
%     X_{j_1} \dotsm X_{j_m}
%     =
%     X_{i_1} \dotsm X_{i_n} X_{j_1} \dotsm X_{j_m} \,.
%   \]
%   In contrast to the usual (commutative) polynomial algebra~$\kf[X_i \suchthat i \in I]$ the variables~$X_i$ are not required to commute with each other.
%   
%   We can alternatively construct~$\kf\gen{X_i \suchthat i \in I}$ as the monomial algebra of the free monoid on~$I$:
%   Let~$M$ be the set of all words in~$I$, i.e.\ the set of all finite sequences
%   \[
%     (i_1, \dotsc, i_n)
%     \qquad
%     \text{with~$i_1, \dotsc, i_n \in I$}  \,.
%   \]
%   Then~$M$ is a monoid with respect to concatenation of words given by
%   \[
%     (i_1, \dotsc, i_n) (j_1, \dotsc, j_m)
%     =
%     (i_1, \dotsc, i_n, j_1, \dotsc, j_m)
%   \]
%   for all words~$(i_1, \dotsc, i_n), (j_1, \dotsc, j_m) \in M$.
%   The neutral element of~$M$ is given by the empty word~$()$.
% \end{recall}





\subsection{Reviewing the Tensor Algebra}


\begin{recall}[Tensor algebra]
  Let~$V$ be a vector space.
  \begin{description}
    \item[Construction:]
      For all~$v_1, \dotsc, v_d \in V$ we denote the resulting simple tensor~$v_1 \tensor \dotsb \tensor v_d$ in~$V^{\tensor d}$ by~$(v_1, \dotsc, v_d)$.
      Observe that for~$d = 0$ the tensor power~$V^{\tensor d} = V^{\tensor 0}$ has as a basis the emtpy simple tensor~$()$.
      We will therefore identify the tensor power~$V^{\tensor 0}$ with the ground field~$\kf$, so that empty simple tensor~$()$ corresponds to~$1 \in \kf$.
      
      For all~$p, q \geq 0$ we define a multiplication
      \[
        \mu_{p,q}
        \colon
        V^{\tensor p} \times V^{\tensor q}
        \to
        V^{\tensor (p+q)} \,,
        \quad
        (x,y)
        \mapsto
        x y
      \]
      that is on simple tensors~$(v_1, \dotsc, v_p) \in V^{\tensor p}$ and~$(v_{p+1}, \dotsc, v_{p+q}) \in V^{\tensor q}$ given by
      \[
        (v_1, \dotsc, v_p) \cdot (v_{p+1}, \dotsc, v_{p+q})
        =
        (v_1, \dotsc, v_{p+q})  \,.
      \]
      Note that for~$p = 0$ or~$q = 0$ this multiplication is just scalar multiplication.  
      These multiplications fit together associatively in the sense that for all~$p, q, r \geq 0$ and simple tensors~$x \in V^{\tensor p}$,~$y \in V^{\tensor q}$ and~$z \in V^{\tensor r}$ the equality
      \[
        x \cdot (y \cdot z)
        =
        (x \cdot y) \cdot z
      \]
      holds.
      
      Let~$\Tensor(V) \defined \bigoplus_{d \geq 0} V^{\tensor d}$.
      We can fit together the multiplications~$\mu_{p,q}$ with~$p, q \geq 0$ to a single multiplication
      \[
        \mu
        \colon
        \Tensor(V) \times \Tensor(V)
        \to
        \Tensor(V)  \,,
        \quad
        (x,y)
        \mapsto
        xy 
      \]
      that is given on elements~$x, y \in \Tensor(V)$ with~$x = (x_d)_{d \geq 0}$ and~$y = (y_d)_{d \geq 0}$ by
      \[
        x y
        =
        \left(
          \sum_{p+q = d} x_p y_q
        \right)_{d \geq 0} \,.
      \]
      This multiplication is built precisely so that it follows from the bilinearity of the multiplications~$\mu_{p,q}$ that the multipliation~$\mu$ is again bilinear.
      It follows from the associativities of the multiplications~$\mu_{p,q}$ that the multiplication~$\mu$ is associative.
      We may identify the ground field~$\kf = V^{\tensor 0}$ with the corresponding direct summand in~$\Tensor(V)$ to regard~$\kf$ as a linear subspace of~$\Tensor(V)$.
      We have seen above that~$1 \in \kf$ is then unital for the multiplication of~$\Tensor(V)$.
      We have thus altogether constructed a~{\algebra{$\kf$}}~$\Tensor(V)$.
      
      We may identify~$V = V^{\tensor 1}$ with the corresponding direct summand of~$\Tensor(V)$ to regard~$V$ as a linear subspace of~$\Tensor(V)$.
      We then have for all~$v_1, \dotsc, v_n \in V$ that
      \[
        v_1 \dotsm v_n
        =
        (v_1) \dotsm (v_n)
        =
        (v_1, \dotsc, v_n)
        =
        v_1 \tensor \dotsb \tensor v_n  \,.
      \]
      It follows in particular that~$\Tensor(V)$ is then generated by~$V$ as an algebra.
      The algebra~\gls*{tensor algebra} is the \defemph{tensor algebra of~$V$}
      
      We will more generally identify for all~$d \geq 0$ the tensor power~$V^{\tensor d}$ with the corresponding summand in~$\Tensor(V)$.
      The tensor algebra~$\Tensor(V)$ hence consists of linear combinations simple tensors~$v_1 \tensor \dotsb \tensor v_n$.
    
    \item[Universal Property:]
      The tensor algebra~$\Tensor(V)$ can be though of as the \enquote{free~{\algebra{$\kf$}} on~$V$}, in at least two ways:
      \begin{itemize}
        \item(Informal)
          The tensor algebra~$\Tensor(V)$ arises from~$V$ by starting with the elements of~$V$ and adding to~$V$ all kinds of expressions that can be constructed from the elements of~$V$ by algebra operations.
          But it follows from the axioms that many of these expressions have to be the same, so that we only end up with expressions of a certain form.
          
          Let us be a bit more explicit:
          Suppose that a~{\algebra{$\kf$}}~$A$ contains~$V$ as a linear subspace.
          Then it also contains products of the form~$v_1 \dotsm v_n$ with~$v_i \in V$ and hence sums of such products, i.e.\ elements of the form
          \[
            \sum_{i=k}^r v_{i_1} \dotsm v_{i_{n_k}}
          \]
          with~$r \geq 0$ and~$v_{ij} \in V$.
          If we continue to combine elements of this form with algebra operations then we do not gain any new elements, since by the axioms of an algebra they must already be of the above form.
          
          But in an arbitrary~{\algebra{$\kf$}} it may happen that some of these expressions are equal even though this does not follow pureley from the axioms of a~{\algebra{$\kf$}}.
          Consider for example the polynomial ring~$A = \kf[x, y]$ and the linear subspace~$V = \gen{x, y}_{\kf}$.
          It follows from the axioms of a~{\algebra{$\kf$}} that the expressions~$x (x+y)$ and~$x^2 + xy$ are the same, but it does not follow just from the axioms that~$xy = yx$, even though this holds in~$A$.
          There are hence certain additional \emph{relations} between the elements~$x$ and~$y$ of~$V$ in the ambient {\algebra{$\kf$}}~$A$.
          
          In the tensor algebra~$\Tensor(V)$ this does not happen:
          Whenever two expressions~$x$ and~$y$ that are built from elements of~$V$ via algebra operations coincide, then this equality can be derived from the algebra axioms alone.
          Hence there exist no additional relations between the elements of~$V$ in~$\Tensor(V)$.
          The only required condition is that~$V$ is a linear subspace of~$\Tensor(V)$, i.e.\ that addition and scalar multiplication in~$V$ does coincide with the one coming from~$\Tensor(V)$.
          
          The tensor algebra~$\Tensor(V)$ is in this way the \enquote{freest} way of expanding~$V$ into a~{\algebra{$\kf$}}.
        \item(Formal)
          Let~$\iota \colon V \to \Tensor(V)$ be the inclusion map, which is~{\linear{$\kf$}}.
          Then if~$A$ is any other~{\algebra{$\kf$}} and~$f \colon V \to A$ any~{\linear{$\kf$}} map (which one can think of as somewhat of an inclusion, albeit not injective), then~$f$ extends uniquely to an algebra homomorphism~$f^+ \colon \Tensor(V) \to A$, i.e.\ there exists a unique algebra homomorphism~$f^+ \colon \Tensor(V) \to A$ that makes the triangular diagram
          \[
            \begin{tikzcd}
              V
              \arrow{r}[above]{f}
              \arrow{d}[left]{i}
              &
              A
              \\
              \Tensor(V)
              \arrow[dashed]{ur}[below right]{f^+}
              &
              {}
            \end{tikzcd}
          \]
          commute.
          The algebra homomorphism~$f^+$ is given by
          \[
            f^+(v_1 \tensor \dotsb \tensor v_d)
            =
            f(v_1) \dotsm f(v_d)
          \]
          for all~$d \geq 0$ and simple tensors~$v_1 \tensor \dotsb \tensor v_d \in V^{\tensor d}$.
          This construction results in a {\onetoone} correspondence
          \begin{align*}
            \{ \text{\linear{$\kf$} maps~$V \to A$} \}
            &\longonetoone
            \{ \text{algebra homomorphisms~$\Tensor(V) \to A$} \} \,,
            \\
            f
            &\longmapsto
            f^+ \,,
            \\
            \restrict{F}{V}
            &\longmapsfrom
            F \,.
          \end{align*}
          Hence~$(\Tensor(V), i)$ is the \enquote{universal way} of mapping the vector space~$V$ into a~{\algebra{$\kf$}}.
          
          This formal explanation relates to the previous informal explanation in the following way:
          If~$A$ is any~{\algebra{$\kf$}} that contains~$V$ as a linear subspace then the inclusion~$V \to A$ extend uniquely to an algebra homomorphism~$\Tensor(V) \to A$.
          Every relation between expressions built from the elements of~$V$ that holds in~$\Tensor(V)$ must then also hold in~$A$.
          Therefore the only relations that hold in~$\Tensor(V)$ between such expressions are the one that hold in \emph{every}~{\algebra{$\kf$}} containing~$V$.
      \end{itemize}
      
    \item[Uniqueness]
      The above universal property determines the tensor algebra~$\Tensor(V)$ together with the inclusion~$i \colon V \to \Tensor(V)$ uniquely up to unique isomorphism, in the following sense:
      Let~$A$ be another~{\algebra{$\kf$}} and let~$j \colon V \to T$ be a~{\linear{$\kf$}} map such that for every~{\algebra{$\kf$}}~$A$ and every~{\linear{$\kf$}} map~$f \colon V \to A$ there exists a unique algebra homomorphism~$F \colon T \to A$ that makes the triangular diagram
      \[
        \begin{tikzcd}
          V
          \arrow{r}[above]{f}
          \arrow{d}[left]{j}
          &
          A
          \\
          T
          \arrow{ur}[below right]{F}
          &
          {}
        \end{tikzcd}
      \]
      commute.
      
      Then there exist unique algebra homomorphisms~$f \colon A \to T$ and~$g \colon T \to A$ that make the triangular diagrams
      \[
        \begin{tikzcd}[column sep = small]
          {}
          &
          V
          \arrow{dl}[above left]{i}
          \arrow{dr}[above right]{j}
          &
          {}
          \\
          \Tensor(V)
          \arrow[dashed]{rr}[below]{f}
          &
          {}
          &
          T
        \end{tikzcd}
        \qquad\text{and}\qquad
        \begin{tikzcd}[column sep = small]
          {}
          &
          V
          \arrow{dl}[above left]{j}
          \arrow{dr}[above right]{i}
          &
          {}
          \\
          T
          \arrow[dashed]{rr}[below]{g}
          &
          {}
          &
          \Tensor(V)
        \end{tikzcd}
      \]
      commute.
      It then follows that the compositions~$g \circ f \colon \Tensor(V) \to \Tensor(V)$ and~$f \circ g \colon T \to T$ make the triangular diagrams
      \[
        \begin{tikzcd}[column sep = small]
          {}
          &
          V
          \arrow{dl}[above left]{i}
          \arrow{dr}[above right]{i}
          &
          {}
          \\
          \Tensor(V)
          \arrow[dashed]{rr}[below]{g \circ f}
          &
          {}
          &
          \Tensor(V)
        \end{tikzcd}
        \qquad\text{and}\qquad
        \begin{tikzcd}[column sep = small]
          {}
          &
          V
          \arrow{dl}[above left]{j}
          \arrow{dr}[above right]{j}
          &
          {}
          \\
          T
          \arrow[dashed]{rr}[below]{f \circ g}
          &
          {}
          &
          T
        \end{tikzcd}
      \]
      commute.
      The algebra homomorphisms~$g \circ f$ and~$f \circ g$ are unique with this propert by the universal properties of~$(\Tensor(V), i)$ and~$(T, j)$.
      But the identities~$\id_{\Tensor(V)}$ and~$\id_T$ also make these diagrams commute.
      We therefore find that~$g \circ f = \id_{\Tensor(V)}$ and~$f \circ g = \id_{\Tensor(V)}$, so that~$f$ and~$g$ are mutually inverse algebra isomorphisms.
    
    \item[Functoriality:]
      If~$f \colon V \to W$ is any~{\linear{$\kf$}} map then we can consider the following diagram:
      \[
        \begin{tikzcd}
          V
          \arrow{r}[above]{f}
          \arrow{d}
          &
          W
          \arrow{d}
          \\
          \Tensor(V)
          &
          \Tensor(W)
        \end{tikzcd}
      \]
      By applying the universal property of the tensor algebra~$\Tensor(V)$ to the composition~$V \to W \to \Tensor(W)$ it follows that there exists a unique algebra homomorphism~$f_* \colon \Tensor(V) \to \Tensor(W)$ that makes the square diagram
      \[
        \begin{tikzcd}
          V
          \arrow{r}[above]{f}
          \arrow{d}
          &
          W
          \arrow{d}
          \\
          \Tensor(V)
          \arrow[dashed]{r}[below]{f_*}
          &
          \Tensor(W)
        \end{tikzcd}
      \]
      commute.
      This induced algebra homorphism is functorial in the following sense:
      \begin{itemize}
        \item
          It holds that~$(\id_V)_* = \id_{\Tensor(V)}$.
          Indeed, the commutativity of the square 
          \[
            \begin{tikzcd}[column sep = large]
              V
              \arrow{r}[above]{f}
              \arrow{d}
              &
              V
              \arrow{d}
              \\
              \Tensor(V)
              \arrow[dashed]{r}[below]{(\id_V)_*}
              &
              \Tensor(V)
            \end{tikzcd}
          \]
          shows that the identity~$\id_{\Tensor(V)}$ satifies the defining property of the induced algebra homomorphism~$(\id_V)_*$.
        \item
          It holds for all linear maps~$f \colon U \to V$ and~$g \colon V \to W$ that~$(g \circ f)_* = g_* \circ f_*$.
          Indeed, it follows from the commutativity of the diagram
          \[
            \begin{tikzcd}
              U
              \arrow[dashed, bend left=45]{rr}[above]{g \circ f}
              \arrow{r}[above]{f}
              \arrow{d}
              &
              V
              \arrow{r}[above]{g}
              \arrow{d}
              &
              W
              \arrow{d}
              \\
              \Tensor(U)
              \arrow{r}[below]{f_*}
              \arrow[dashed, bend right=45]{rr}[below]{g_* \circ f_*}
              &
              \Tensor(V)
              \arrow{r}[below]{g_*}
              &
              \Tensor(W)
            \end{tikzcd}
          \]
          that the subdiagram
          \[
            \begin{tikzcd}[column sep = large]
              U
              \arrow{r}[above]{g \circ f}
              \arrow{d}
              &
              W
              \arrow{d}
              \\
              \Tensor(U)
              \arrow[dashed]{r}[below]{g_* \circ f_*}
              &
              \Tensor(W)
            \end{tikzcd}
          \]
          commutes.
          This shows that the composition~$g_* \circ f_*$ satisfies the defining property of the induced algebra homomorphism~$(g \circ f)_*$.
      \end{itemize}
      
      This shows that the assignment~$V \mapsto \Tensor(V)$ extends to a (covariant) functor~$\Tensor \colon \cVect{\kf} \to \cAlg{\kf}$.
      The universal property of the tensor algebra states that the functor~$\Tensor$ is left adjoint to the forgetful functor~$\cAlg{\kf} \to \cVect{\kf}$ that assigns to each~{\algebra{$\kf$}} its underlying~{\vectorspace{$\kf$}}.
    
    \item[Description via a basis:]
      If a basis~$(v_i)_{i \in I}$ of~$V$ is choosen then every tensor power~$V^{\tensor d}$ inherits a basis that is given by all simple tensors
      \[
        v_{i_1} \tensor \dotsb \tensor v_{i_d}
      \]
      with~$i_1, \dotsc, i_d \in I$.
      It follows that the tensor power has as a basis of all such simple tensors with~$d \geq 0$ and~$i_{i_1, \dotsc, i_d} \in I$.
      The product of two such basis vectors is again a basis vector.
      So we may think about the basis vectors as finite words~$i_1 \dotsm i_d$ in the alphabet~$I$, and as the multiplication of two basis vectors as the concatenation of the corresponding words.
      
      If we think about the basis vector~$v_i$ of~$V$ as a formal variable~$X_i$ then we see that the tensor algebra~$\Tensor(V)$ is isomorphic to the noncommutative polynomial ring~$\kf\gen{X_i \suchthat i \in I}$.
      This noncommutative polynomial ring is also the free~{\algebra{$\kf$}} on the generators~$X_i$ with~$i \in I$, while~$V$ is the free~{\vectorspace{$\kf$}} on the letters~$i \in I$.
      This gives another explanation for why~$\Tensor(V)$ is the free~{\algebra{$\kf$}} on the vector space~$V$.
      More exicitely, we have the following commutative diagram of forgetful functors:
      \[
        \begin{tikzcd}
          \cVect{\kf}
          \arrow{d}
          &
          \cAlg{\kf}
          \arrow{l}
          \arrow{dl}
          \\
          \cSet
          &
          {}
        \end{tikzcd}
      \]
      It then follows that the resulting diagram of left adjoint functors
      \[
        \begin{tikzcd}
          \cVect{\kf}
          \arrow{r}[above]{\Tensor}
          &
          \cAlg{\kf}
          \\
          \cSet
          \arrow{u}[left]{F}
          \arrow{ur}[below right]{\kf\gen{X_i \suchthat i \in (-)}}
          &
          {}
        \end{tikzcd}
      \]
      commutes up to natural isomorphism.
      Hence~$\Tensor(V) \cong \Tensor(F(I)) \cong \kf\gen{X_i \suchthat i \in I}$.
  \end{description}
\end{recall}





\subsection{Reviewing the Symmetric Algebra}


\begin{recall}[Symmetric power]
  Let~$V$ be a vector space and let~$d \geq 0$.
  The~{\howmanyth{$d$}} \defemph{symmetric power}\index{symmetric!power}~$\Symm^d(V)$ is the quotient vector space of the tensor power~$\Symm^d(V)$ by the~{\linear{$\kf$}} subspace~$U_d$ that is generated by all all differences
  \[
      v_1 \tensor \dotsb \tensor v_d
    - v_{\sigma(1)} \tensor \dotsb \tensor v_{\sigma(d)}
  \]
  where~$v_1 \tensor \dotsb \tensor v_d \in V^{\tensor d}$ is a simple tensor and~$\sigma \in S_n$ is a permuation.
  Hence
  \begin{align*}
    \Symm^d(V)
    &=
    V^{\tensor d} / U_d
    \\
    &=
    V^{\tensor d}
    /
    \gen{
        v_1 \tensor \dotsb \tensor v_d
      - v_{\sigma(1)} \tensor \dotsb \tensor v_{\sigma(d)} 
    \suchthat
      v_1, \dotsc, v_n \in V,
      \sigma \in S_n
    }_{\kf} \,.
  \end{align*}
  Observe that~$\Symm^0(V) = V^{\tensor 0} = \kf$ because~$U_0 = 0$.
  For~$v_1, \dotsc, v_n \in V$ we denote the residue class of the simple tensor~$v_1 \tensor \dotsb \tensor v_d$ in~$\Symm^d(V)$ by~$v_1 \dotsm v_d$, and call this a \defemph{symmetric simple tensor}\index{symmetric!simple tensor}.
  
  We have by construction of~$\Symm^d(V)$ that
  \[
    v_1 \dotsm v_d
    =
    v_{\sigma(1)} \dotsm v_{\sigma(d)}
  \]
  for all simple symmetric tensors~$v_1 \dotsm v_n \in \Symm^d(V)$ and permutations~$\sigma \in S_d$, and for~$d \geq 1$ the symmetric power~$\Symm^d(V)$ is universal with this property in the following sense:
  The map
  \[
    V^{\times d}
    \to
    \Symm^d(V)  \,,
    \quad
    (v_1, \dotsc, v_d)
    \mapsto
    v_1 \dotsm v_d
  \]
  is symmetric and multilinear, and if~$f \colon V^{\times d} \to W$ is any other symmetric multilinear map into any vector space~$W$ then there exists a unique linear map~$g \colon \Symm^d(V) \to W$ that makes the triangular diagram
  \[
    \begin{tikzcd}
      V^{\times d}
      \arrow{d}
      \arrow{dr}[above right]{f}
      &
      {}
      \\
      \Symm^d(V)
      \arrow[dashed]{r}[below]{g}
      &
      W
    \end{tikzcd}
  \]
  commute.
  A linear map~$\Symm^d(V) \to W$ is in this sense the same as a symmetric bilinear map~$V^{\times d} \to W$.
  
  If~$(v_i)_{i \in I}$ is a basis of~$V$ such that~$(I, \leq)$ is a linearly ordered set then the ordered monomials
  \[
    v_{i_1} \dotsm v_{i_d}
    \qquad
    \text{with~$i_1 \leq \dotsb \leq i_d$}
  \]
  form a basis of the symmetric power~$\Symm^d(V)$.
  If~$V$ is of finite dimension~$n$ then it follows that
  \[
    \dim \Symm^d(V)
    =
    \binom{n+d-1}{d}  \,.
  \]
\end{recall}


\begin{recall}[Symmetric algebra]
  Let~$V$ be a vector space.
  Just as the tensor algebra~$\Tensor(V)$ is the free~{\algebra{$\kf$}} on~$V$ and can be constructed by using the tensor powers~$V^{\tensor d}$ we can use the symmetric powers~$\Symm^d(V)$ to construct the \defemph{symmetric algebra}\index{symmetric algebra}~\gls*{symmetric algebra}.
  The argumentation is analogous to that for the tensor algebra, so we will skip some of the details this time.
  
  \begin{description}
    \item[Construction:]
      For all~$v_1, \dotsc, v_d \in V$ we denote the corresponding simple symmetric tensor in~$\Symm^d(V)$ by~$v_1 \dotsm v_d$.
      We can define on~$\Symm(V) \defined \bigoplus_{d \geq 0} \Symm^d(V)$ a multiplication such that
      \[
        (v_1 \dotsm v_p) \cdot (v_{p+1} \dotsm v_{p+q})
        =
        v_1 \dotsm v_p v_{p+1} \dotsm v_{p+q}
      \]
      for all~$p, q \geq 0$ and all simple symmetric tensors~$v_1 \dotsm v_p \in \Symm^p(V)$ and~$v_{p+1}, \dotsc, v_{p+q} \in \Symm^q(V)$.
      By identifying~$\Symm^0(V)$ with the ground field~$\kf$ this makes~$\Symm(V)$ into an associative~{\algebra{$\kf$}}.
      This is already a commutative~{\algebra{$\kf$}} because
      \begin{align*}
        (v_1 \dotsm v_p) \cdot (v_{p+1} \dotsm v_{p+q})
        &=
        v_1 \dotsm v_p v_{p+1} \dotsm v_{p+q}
        \\
        &=
        v_{p+1} \dotsm v_{p+q} v_1 \dotsm v_p
        \\
        &=
        (v_{p+1} \dotsm v_{p+q}) \cdot (v_1 \dotsm v_p)
      \end{align*}
      for all~$p, q \geq 0$ and all simple symmetric tensors~$v_1 \dotsm v_p \in \Symm^p(V)$ and~$v_{p+1}, \dotsc, v_{p+q} \in \Symm^q(V)$. 
      We can identify~$V = \Symm^1(V)$ with the corresponding direct summand of~$\Symm(V)$, and more generally every symmetric power~$\Symm^d(V)$ with the corresponding direct summand of~$\Symm(V)$.
      The algebra~$\Symm(V)$ thus consists of linear combinations of simple symmetric tensors.
      
    \item[Universal property:]
      The symmetric algebra~$\Symm(V)$ is the \enquote{free commutative~{\algebra{$\kf$}}} on the vector space~$V$ in the following sense:
      If~$i \colon V \to \Symm(V)$ is the inclusion then there exists for every~{\algebra{$\kf$}}~$A$ and every linear map~$f \colon V \to A$ a unique algebra homomorphism~$f^+ \colon \Symm(V) \to A$ that makes the triangular diagram
      \[
        \begin{tikzcd}
          V
          \arrow{r}[above]{f}
          \arrow{d}[left]{i}
          &
          A
          \\
          \Symm(V)
          \arrow{ur}[below right]{f^+}
          &
          {}
        \end{tikzcd}
      \]
      commute.
      The algebra homomorphism~$f^+$ is given by
      \[
        f^+(v_1 \dotsm v_d)
        =
        f(v_1) \dotsm f(v_d)
      \]
      for all~$d \geq 0$ and simple symmetric tensors~$v_1, \dotsc, v_d \in V$.
      This construction results in a {\onetoone} correspondence
      \begin{align*}
        \{ \text{\linear{$\kf$} maps~$V \to A$} \}
        &\longonetoone
        \{ \text{algebra homomorphisms~$\Symm(V) \to A$} \} \,,
        \\
        f
        &\longmapsto
        f^+ \,,
        \\
        \restrict{F}{V}
        &\longmapsfrom
        F \,.
      \end{align*}
      
      It follows that a relations between elements of~$V$ holds in the symmetric algebra~$\Symm(V)$ if and only if it holds in every commutative algebra that contains~$V$.
      
    \item[Uniqueness]
      If~$S$ is a commutative~{\algebra{$\kf$}} and~$j \colon V \to S$ is a~{\linear{$\kf$}} such that~$(S, j)$ satisfies the same universal property as the symmetric algebra~$(\Symm(V), i)$  then there exists unique algebra homomorphisms~$f \colon \Symm(V) \to S$ and~$g \colon S \to \Symm(V)$ that make the triangular diagrams
      \[
        \begin{tikzcd}[column sep = small]
          {}
          &
          V
          \arrow{dl}[above left]{i}
          \arrow{dr}[above right]{j}
          &
          {}
          \\
          \Symm(V)
          \arrow[dashed]{rr}[below]{f}
          &
          {}
          &
          S
        \end{tikzcd}
        \qquad\text{and}\qquad
        \begin{tikzcd}[column sep = small]
          {}
          &
          V
          \arrow{dl}[above left]{j}
          \arrow{dr}[above right]{i}
          &
          {}
          \\
          S
          \arrow[dashed]{rr}[below]{g}
          &
          {}
          &
          \Symm(V)
        \end{tikzcd}
      \]
      commute.
      Then~$f$ and~$g$ are mutually inverse algebra isomorphisms.
      
    \item[Functoriality]
      For every linear map~$f \colon V \to W$ there exists a unique induced algebra homomorphism~$f_* \colon \Symm(V) \to \Symm(W)$ that makes the square diagram
      \[
        \begin{tikzcd}
          V
          \arrow{r}[above]{f}
          \arrow{d}
          &
          W
          \arrow{d}
          \\
          \Symm(V)
          \arrow[dashed]{r}[below]{f_*}
          &
          \Symm(W)
        \end{tikzcd}
      \]
      commmute.
      It holds that~$(\id_V)_* = \id_{\Symm(V)}$ and it holds for all composable~{\linear{$\kf$}} maps~$f \colon U \to V$ and~$g \colon V \to W$ that~$(g \circ f)_* = g_* \circ f_*$.
      This construction promotes the assignment~$V \mapsto \Symm(V)$ to a (covariant) functor~$\Symm \colon \cVect{\kf} \to \cCAlg{\kf}$, where~$\cCAlg{\kf}$ denotes the category of commutative~{\algebras{$\kf$}}.
      
    \item[Description via a basis]
      If~$(v_i)_{i \in I}$ is a basis of$~V$ where~$(I, \leq)$ is a linearly ordered set then the symmetric power~$\Symm^d(V)$ inherits a basis that is given by all simple symmetric tensors
      \[
        v_{i_1} \dotsm v_{i_d}
        \qquad
        \text{where~$i_1 \leq \dotsb \leq i_d$} \,.
      \]
      It follows that the symmetric algebra~$\Symm(V)$ has as a basis all simple symmetric tensors~$v_{i_1} \dotsm v_{i_d}$ with~$d \geq 0$ and~$i_1, \dotsc, i_d \in I$ with~$i_1 \leq \dotsb \leq i_d$.
      This basis may also be written as
      \[
        v_{i_1}^{\nu_1} \dotsm v_{i_r}^{\nu_r}
      \]
      with~$r \geq 0$,~$i_1, \dotsc, i_r \in I$ such that~$i_1 < \dotsb < i_r$ and~$\nu_1, \dotsc, \nu_r \geq 0$ (which is connected to the above description via~$d = \nu_1 + \dotsb + \nu_r$).
      
      We see from this description that the symmetric algebra~$\Symm(V)$ is isomorphic to the commutative polynomial ring~$\kf[X_i \suchthat i \in I]$, which is the free commutative~{\algebra{$\kf$}} on the generators~$i \in I$
      This can again be explained by considering the commutative diagram of forgetful functors
      \[
        \begin{tikzcd}
          \cVect{\kf}
          \arrow{d}
          &
          \cCAlg{\kf}
          \arrow{l}
          \arrow{dl}
          \\
          \cSet
          &
          {}
        \end{tikzcd}
      \]
      from which we see that the resulting diagram of left adjoints
      \[
        \begin{tikzcd}
          \cVect{\kf}
          \arrow{r}[above]{\Symm}
          &
          \cCAlg{\kf}
          \\
          \cSet
          \arrow{u}[left]{F}
          \arrow{ur}[below right]{\kf[X_i \suchthat i \in (-)]}
          &
          {}
        \end{tikzcd}
      \]
      commutes up to natural isomorphism.
      
    \item[Contruction via the tensor algbra]
      The symmetric algebra~$\Symm(V)$ can also be constructed as a quotient of the tensor algebra~$\Tensor(V)$.
      We give multiple ways how to see and think about this.
      Let in the following~$i \colon V \to \Tensor(V)$ and~$j \colon V \to \Symm(V)$ denote the 
      \begin{itemize}
        \item
          Let~$I$ be the commutator ideal of~$\Tensor(V)$, i.e.\ the two-sided ideal generated by all commutators
          \[
            x \tensor y - y \tensor x
          \]
          with~$x, y \in \Tensor(V)$.
          Let~$\pi \colon \Tensor(V) \to \Tensor(V)/I$ be the canonical projection.
          Then the quotient~$\Tensor(V)/I$ is commutative, and hence there exists by the universal property of the symmetric algebra a unique algebra homomorphism~$f \colon \Symm(V) \to \Tensor/I$ that makes the diagram
          \[
            \begin{tikzcd}
              {}
              &
              V
              \arrow[bend right]{ddl}[above left]{i}
              \arrow[bend left]{dr}[above right]{j}
              &
              {}
              \\
              {}
              &
              {}
              &
              \Tensor(V)
              \arrow{d}[right]{\pi}
              \\
              \Symm(V)
              \arrow[dashed]{rr}[above]{f}
              &
              {}
              &
              \Tensor(V)/I
            \end{tikzcd}
          \]
          commute.
          The homomorphism~$f$ is on the genareting set~$V$ of~$\Symm(V)$ given by~$f(v) = \class{v}$.
          On the other hand we get from the universal property of the tensor algebra~$\Tensor(V)$ a unique algebra homomorphism~$\tilde{g} \colon \Tensor(V) \to \Symm(V)$ that makes the diagram
          \[
            \begin{tikzcd}
              {}
              &
              V
              \arrow[bend right]{dl}[above left]{j}
              \arrow[bend left]{ddr}[above right]{i}
              &
              {}
              \\
              \Tensor(V)
              \arrow[bend left, dashed]{drr}[above right]{\tilde{g}}
              \arrow{d}[left]{\pi}
              &
              {}
              &
              {}
              \\
              \Tensor(V)/I
              &
              {}
              &
              \Symm(V)
            \end{tikzcd}
          \]
          commute.
          The commutator~$I$ is contained in the kernel of~$\tilde{g}$ because the algebra~$\Symm(V)$ is commutative.
          Hence there exists a unique algebra homomorphism~$g \colon \Tensor(V)/I \to \Symm(V)$ that makes the diagram
          \[
            \begin{tikzcd}
              {}
              &
              V
              \arrow[bend right]{dl}[above left]{j}
              \arrow[bend left]{ddr}[above right]{i}
              &
              {}
              \\
              \Tensor(V)
              \arrow[bend left]{drr}[above right]{\tilde{g}}
              \arrow{d}[left]{\pi}
              &
              {}
              &
              {}
              \\
              \Tensor(V)/I
              \arrow[dashed]{rr}[below]{g}
              &
              {}
              &
              \Symm(V)
            \end{tikzcd}
          \]
          commute.
          The algebra homomorphism~$g$ is given on the generators~$\class{v}$ with~$v \in V$ of~$\tensor(V)/I$ given by~$g(\class{v}) = v$.
          
          It follows from the explicit descriptions of~$f$ and~$g$ on generators that their are mutually inverse algebra isomorphisms.
          Thus~$\Symm(V) \cong \Tensor(V)/I$ via the isomorphism~$f$.
          
          Observe also that the commutator ideal~$I$ is already generated by the commutators~$v \tensor w - w \tensor v$ with~$v, w \in V$.
          Indeed, the ideal~$J$ generated by these elements is contained in~$I$.
          But on the other hand the quotient~$\Tensor(V)/J$ is already commutative because it is generated by the residue classes~$\class{v}$ with~$v \in V$, all of which commute with each other.
          The commutator ideal~$I$ is therefore contained in the kernel of the canonical projection~$\Tensor(V) \to \Tensor(V)/J$, i.e.\ it is containted in~$J$.
          
        \item
          The above argumentatio is not surprising if we remember that~$\Tensor(V)$ is the universal~{\algebra{$\kf$}} on~$V$ and that quotiening out the commutator ideal is the universal way of making an algebra commutative.
          The quotient~$\Tensor(V)/I$ therefore ought to be the universal commutative~{\algebra{$\kf$}}.
          
          This motivation can be formalized by observing that the diagram of forgetful functors
          \[
            \begin{tikzcd}
              \cAlg{\kf}
              \arrow{d}
              &
              \cCAlg{\kf}
              \arrow{l}
              \arrow{dl}
              \\
              \cVect{\kf}
              &
              {}
            \end{tikzcd}
          \]
          commutes.
          It follows that the resulting diagram of left adjoints
          \[
            \begin{tikzcd}
              \cAlg{\kf}
              \arrow{r}[above]{C}
              &
              \cCAlg{\kf}
              \\
              \cVect{\kf}
              \arrow{u}[left]{\Tensor}
              \arrow{ur}[below right]{\Symm}
              &
              {}
            \end{tikzcd}
          \]
          commutes up to natural isomorphism.
          The adjoint~$C$ of the forgetful functor~$\cCAlg{\kf} \to \cAlg{\kf}$ is given by quotiening out the commutator ideal, and hence~$\Symm(V) \cong \Tensor(V)/I$ as before.
        \item
          The above argumentation be also expressed by observing that for every commutative~{\algebra{$\kf$}}~$A$ there exist natural bijections
          \begin{align*}
            {}&
            \{ \text{algebra homomorphisms~$\Symm(V) \to A$} \}
            \\
            \cong{}&
            \{ \text{{\linear{$\kf$}} maps~$V \to A$} \}
            \\
            \cong{}&
            \{ \text{algebra homomorphisms~$\Tensor(V) \to A$} \}
            \\
            \cong{}&
            \{ \text{algebra homomorphisms~$\Tensor(V)/I \to A$} \} \,,
          \end{align*}
          where the last bijection uses that the algebra~$A$ is commutative and therefore every algebra homomorphism~$\Tensor(V) \to A$ contains the commutator ideal~$I$ in its kernel.
          It now follows from Yoneda’s~lemma that~$\Symm(V) \cong \Tensor(V)/I$.
      \end{itemize}
  \end{description}
\end{recall}


\begin{remark}  % TODO: Fix glossary and bigwedge
  One can similarly construct the \emph{exterior algebra}~$\gls*{exterior algebra} = \bigoplus_{d \geq 0} \Exterior^d(V)$ of a vector space~$V$ by replacing the use of the tensor powers~$V^{\tensor d}$ or symmetric powers~$\Symm^d(V)$ by the exterior powers~$\Exterior^d(V)$.
  For any other~{\algebra{$\kf$}}~$A$ an algebra homomorphism~$F \colon \Exterior(V) \to A$ is then the same as a~{\linear{$\kf$}}~$f \colon V \to A$ with~$f(v)^2 = 0$ for every~$v \in V$.
% TODO: Do we have to worry about char(k) = 2?
  It thus follows from a similar argumentation as for the symmetric algebra that~$\Exterior(V) \cong \Tensor(V)/I$ for the two-sided ideal~$I$ in~$\Tensor(V)$ generated by all~$v \tensor v$ with~$v \in V$.
  
  
  If~$V$ is finite dimensional then the exterior algebra~$\Exterior(V)$ is again finite dimensional, namely with~$\dim \Exterior(V) = 2^{\dim V}$.
  This is different to both the tensor algebra~$\Tensor(V)$ and symmetric algebra~$\Symm(V)$, which are infinite dimensional whenever~$V \neq 0$.
\end{remark}





\section{Universal Enveloping Algebra}





\subsection{Definition}


\begin{definition}
  Let~$\glie$ be a~\liealgebra{$\kf$}.
  A \defemph{universal enveloping algebra}\index{universal enveloping algebra} of~$\glie$ is a~\algebra{$\kf$}~$\Univ(\glie)$\glsadd{universal enveloping algebra} together with a homomorphism of Lie~algebras~$\iota$ from~$\glie$ to~$\Univ(\glie)$ such that the following universal property holds:
  for every~{\algebra{$\kf$}}~$A$ and every homomorphism of Lie~algebras~$\varphi$ from~$\glie$ to~$A$ there exists a unique homomorphism of~\algebras{$\kf$}~$\Phi$  from~$\Univ(\glie)$ to~$A$ that makes the triangular diagram
  \[
    \begin{tikzcd}
      \glie
      \arrow{r}[above]{\phi}
      \arrow{d}[left]{\iota}
      &
      A
      \\
      \Univ(\glie)
      \arrow[dashed]{ur}[below right]{\Phi}
      &
      {}
    \end{tikzcd}
  \]
  commute, i.e.\ such that~$\varphi = \Phi \circ \iota$.
\end{definition}


\begin{remark}[Uniqueness of universal enveloping algebras]
  \label{uniqueness of universal enveloping algebras}
  Let~$\glie$ be a Lie algebra and suppose that~$(\Univ(\glie)_1, \iota_1)$ and~$(\Univ(\glie)_2, \iota_2)$ are two~{\uas} of~$\glie$.
  Then there exist unique algebra homomorphisms~$\Phi$ from~$\Univ(\glie)_1$ to~$\Univ(\glie)_2$ and~$\Psi$ from~$\Univ(\glie)_2$ to~$\Univ(\glie)_1$ that make the triangular diagrams
  \[
    \begin{tikzcd}[column sep = small]
      {}
      &
      \glie
      \arrow{dl}[above left]{\iota_1}
      \arrow{dr}[above right]{\iota_2}
      &
      {}
      \\
      \Univ(\glie)_1
      \arrow[dashed]{rr}[below]{\Phi}
      &
      {}
      &
      \Univ(\glie)_2
    \end{tikzcd}
    \qquad\text{and}\qquad
    \begin{tikzcd}[column sep = small]
      {}
      &
      \glie
      \arrow{dl}[above left]{\iota_2}
      \arrow{dr}[above right]{\iota_1}
      &
      {}
      \\
      \Univ(\glie)_2
      \arrow[dashed]{rr}[below]{\Psi}
      &
      {}
      &
      \Univ(\glie)_1
    \end{tikzcd}
  \]
  commute.
  It follows that the composites~$\Psi \circ \Phi$ and~$\Phi \circ \Psi$ make the triangle diagrams
  \[
    \begin{tikzcd}[column sep = small]
      {}
      &
      \glie
      \arrow{dl}[above left]{\iota_1}
      \arrow{dr}[above right]{\iota_1}
      &
      {}
      \\
      \Univ(\glie)_1
      \arrow[dashed]{rr}[below]{\Psi \circ \Phi}
      &
      {}
      &
      \Univ(\glie)_1
    \end{tikzcd}
    \qquad\text{and}\qquad
    \begin{tikzcd}[column sep = small]
      {}
      &
      \glie
      \arrow{dl}[above left]{\iota_2}
      \arrow{dr}[above right]{\iota_2}
      &
      {}
      \\
      \Univ(\glie)_2
      \arrow[dashed]{rr}[below]{\Phi \circ \Psi}
      &
      {}
      &
      \Univ(\glie)_2
    \end{tikzcd}
  \]
  commute.

  The algebra homomorphisms~$\Phi \circ \Psi$ and~$\Psi \circ \Phi$ are unique with this property by the universal properties of the {\uas}~$(\Univ(\glie)_1, \iota_1)$ and~$(\Univ(\glie)_2, \iota_2)$.
  But the identities~$\id_{\Univ(\glie)_1}$ and~$\id_{\Univ(\glie)_2}$ also makes these diagrams commute.
  We thus find that the composite~$\Psi \circ \Phi$ equals~$\id_{\Univ(\glie)_1}$ and the composite~$\Phi \circ \Psi$ equals~$\id_{\Univ(\glie)_2}$.
  The homomorphisms~$\Phi$ and~$\Psi$ are therefore mutually inverse isomorphisms.
  
  This shows that a {\ua} of~$\glie$ is unique up to unique isomorphism.
  We will therefore talk about \emph{the} {\ua} of~$\glie$.
  We will often also surpress the algebra homorphism~$\iota$ from~$\glie$ to~$\Univ(\glie)$ from our notation.
\end{remark}


% \begin{remark}
%   One can also formulate the above argument is a more categorical way:
%   Consider the category~$\catC$ where
%   \begin{itemize}
%     \item
%       objects of~$\catC$ is a pairs~$(A, i)$ consisting of a~{\algebra{$\kf$}}~$A$ and a Lie~algebra homomorphism~$i \colon \glie \to A$,
%     \item
%       a morphism~$\phi \colon (A, i) \to (B, j)$ is an algebra homomorphism~$\phi \colon A \to B$ that makes the triangular diagram
%       \[
%         \begin{tikzcd}[column sep = small]
%         {}
%         &
%         \glie
%         \arrow{dl}[above left]{i}
%         \arrow{dr}[above right]{j}
%         &
%         {}
%         \\
%         A
%         \arrow[dashed]{rr}[below]{\phi}
%         &
%         {}
%         &
%         B
%       \end{tikzcd}
%     \]
%       commute, and
%     \item
%       the composition of two morphisms is just their usual set-theoretic composition.
%   \end{itemize}
%   A {\ua} of~$\glie$ is nothing but an inital object in this category~$\catC$.
%   The argumentation from \cref{uniqueness of universal enveloping algebras} is then the usual argument for the uniqueness of inital objects up to unique isomorphism.
% \end{remark}



\subsection{Construction}


\begin{fluff}
  Let~$\glie$ be a Lie~algebra.
  We will in the following show that the {\ua} of~$\glie$ exists.
  For this we will first conclude from the universal property of~$\Univ(\glie)$ that we should be able to construct~$\Univ(\glie)$ as a certain quotient algebra of the tensor algebra~$\Tensor(\glie)$.
  We then show that this quotient does indeed have the correct universal property.

  Suppose that~$\glie$ admits a universal enveloping algebra~$\Univ(\glie)$ and let~$\iota$ be the canonical homomorphism of Lie~algebras from~$\glie$ to~$\Univ(\glie)$.
  We first observe that the algebra~$\Univ(\glie)$ is generated by the image of~$\iota$.

  Indeed, let~$U$ be the subalgebra of~$\Univ(\glie)$ which is generated by the image of~$\iota$, and let~$\iota'$ be the restriction of~$\iota$ to a homomorphism of Lie~algebras from~$\glie$ to~$U$.
  For every~{\algebra{$\kf$}}~$A$ and every Lie~algebra homomorphism~$\varphi$ from~$\glie$ to~$A$ the induced homomorphism of algebras~$\Phi$ from~$\Univ(\glie)$ to~$A$ restricts to an homomorphism of algebras~$\Phi'$ from~$U$ to~$A$.
  This homomorphism~$\Phi'$ makes the triangular diagram
  \[
    \begin{tikzcd}
      \glie
      \arrow{r}[above]{\varphi}
      \arrow{d}[left]{\iota'}
      &
      A
      \\
      U
      \arrow[dashed]{ur}[below right]{\Phi'}
      &
      {}
    \end{tikzcd}
  \]
  commute.
  The homomorphism~$\Phi'$ is unique with this property because the algebra~$U$ is generated by the image of~$\iota'$, and the composite~$\Phi \circ \iota'$ equals the fixed homomorphism of Lie~algebras~$\varphi$.
  This shows that the algebra~$U$ together with the homomorphism of Lie~algebras~$\iota'$ is again a {\ua} for~$\glie$.
  
  It follows from the uniqueness of the {\ua} of~$\glie$, as discussed in \cref{uniqueness of universal enveloping algebras}, that there exists a unique homomorphism of algebras~$\Iota$ from~$U$ to~$\Univ(\glie)$ that makes the triangular diagram
  \[
    \begin{tikzcd}[column sep = small]
      {}
      &
      \glie
      \arrow{dl}[above left]{\iota'}
      \arrow{dr}[above right]{\iota}
      &
      {}
      \\
      U
      \arrow[dashed]{rr}[below]{\Iota}
      &
      {}
      &
      \Univ(\glie)
    \end{tikzcd}
  \]
  commute, and that this homomorphism is already an isomorphism.
  The homomorphism~$\Iota$ is the inclusion map from~$U$ to~$\Univ(\glie)$ because this is a homomorphism of algebras from~$U$ to~$\Univ(\glie)$ which makes the above triangular diagram commute.
  We have thus found that the inclusion map from~$U$ to~$\Univ(\glie)$ is an isomorphism of algebras, whence~$U$ equals~$\Univ(\glie)$.

  We now apply the universal property of the tensor algebra~$\Tensor(\glie)$ to the linear map~$\iota$.
  We find that there exists a unique homomorphism of algebras~$\Phi'$ from~$\Tensor(\glie)$ to~$\Univ(\glie)$ that makes the triangular diagram
  \[
    \begin{tikzcd}[column sep = small]
      {}
      &
      \glie
      \arrow{dl}
      \arrow{dr}[above right]{\iota}
      &
      {}
      \\
      \Tensor(\glie)
      \arrow[dashed]{rr}[below]{\Phi'}
      &
      {}
      &
      \Univ(\glie)
    \end{tikzcd}
  \]
  commute.
  The homomorphism~$\Phi'$ is surjective because~$\Univ(\glie)$ is generated by the image of~$\iota$ as an algebra.
  It follows that~$\Phi'$ induces an isomorphism of algebras
  \[
    \Psi
    \colon
    \Tensor(\glie) / I
    \to
    \Univ(\glie)
  \]
  where the ideal~$I$ is the kernel of~$\Phi'$.
  This isomorphism makes the resulting diagram
   \[
    \begin{tikzcd}[column sep = small]
      {}
      &
      \glie
      \arrow[bend right]{dl}
      \arrow[bend left]{ddr}[above right]{\iota}
      &
      {}
      \\
      \Tensor(\glie)
      \arrow[bend left]{drr}[below left]{\Phi}
      \arrow{d}
      &
      {}
      &
      {}
      \\
      \Tensor(\glie)/I
      \arrow[dashed]{rr}[below]{\Psi}
      &
      {}
      &
      \Univ(\glie)
    \end{tikzcd}
  \]
  commute.
 
  Let~$A$ be another~\algebra{$\kf$}.
  Every linear map~$g$ from~$\glie$ to~$A$ factors through a homomorphism of algebras~$\Psi'$ from~$\Tensor(\glie)$ to~$A$.
  It follows from the above isomorphism~$\Phi$ between~$\Tensor(\glie) / I$ and~$\Univ(\glie)$ that the homomorphism~$\Psi'$ factors trough a homorphism from~$\Tensor(\glie) / I$ to~$A$ if and only if the linear map~$g$ is a homomorphism of Lie~algebras.

  That~$g$ is a homomorphism of Lie~algebras means that
  \[
    g(x) g(y) - g(y) g(x) - g([x,y]) = 0
  \]
  for all~$x, y \in \glie$.
  This is equivalent to the condition
  \[
    \Psi(x) \Psi(y) - \Psi(y) \Psi(x) - \Psi([x,y]_{\glie})
    =
    0
  \]
  for all~$x, y \in \glie$, and further äquivalent to the condition
  \[
    \Psi( xy - yx - [x,y]_{\glie} )
    =
    0
  \]
  for all~$x, y \in \glie$.

  We have now seen that an algebra homomorphism~$\Psi$ from~$\Tensor(\glie)$ to some~\algebra{$\kf$}~$A$ factors trough the quotient~$\Tensor(\glie)/I$ if and only if~$\Psi$ annihilates all those elements of~$\Tensor(\glie)$ that are of the form~$xy - yx - [x,y]_{\glie}$ with~$x$,~$y$ in~$\glie$.
  This means that the ideal~$I$ needs to be generated by those elements.
  
  We have now altogether seen that the universal enveloping algebra~$\Univ(\glie)$ needs to be constructable as the quotient of the tensor algebra~$\Tensor(\glie)$ by the ideal~$I$ which is generated by all those elements of the form~$x y - y x - [x,y]_{\glie}$ with~$x$,~$y$ in~$\glie$.
  We will conversely show in the following \lcnamecref{existence of uea} that this construction will indeed give us the universal enveloping algebra.
\end{fluff}


\begin{proposition}[Existence of the universal enveloping algebra]
  \label{existence of uea}
  Let~$\glie$ be a Lie~algebra.
  Let~$\Tensor(\glie)$ be the tensor algebra of the underlying vector space of~$\glie$ and let~$I$ the two-sided ideal of~$\Tensor(\glie)$ generated by all the elements $x y - y x - [x,y]_{\glie}$ with~$x$,~$y$ in~$\glie$.
  The quotient algebra~$U \defined T(\glie)/I$ together with the~{\linear{$\kf$}} map
  \[
    \iota
    \colon
    \glie
    \to
    \Univ(\glie) \,,
    \quad
    x
    \mapsto
    \class{x}
  \]
  is a {\ua} for~$\glie$.
\end{proposition}


\begin{proof}
  The map~$\iota$ is~{\linear{$\kf$}} and it compatible with the Lie brackets because
  \[
    [\iota(x), \iota(y)]
    =
    [\class{x}, \class{y}]
    =
    \class{x} \, \class{y} - \class{y} \, \class{x}
    =
    \class{x y - y x}
    =
    \class{[x,y]_{\glie}}
    =
    \iota([x,y]_{\glie}) \,.
  \]
  for all~$x, y \in \glie$.
  Given any~\algebra{$\kf$}~$A$ and Lie algebra homomorphism~$\varphi$ from~$\glie$ to~$A$ there exists a unique homorphism of~\algebras{$\kf$}~$\Phi'$ from~$\Tensor(\glie)$ to~$A$ that makes the triangular diagram
  \[
    \begin{tikzcd}
      \glie
      \arrow{r}[above]{\varphi}
      \arrow{d}
      &
      A
      \\
      \Tensor(V)
      \arrow[dashed]{ur}[below right]{\Phi'}
      &
      {}
    \end{tikzcd}
  \]
  commute.
  The homomorphism~$\Phi'$ is given by~$\Phi'(x) = \varphi(x)$ for all~$x \in \glie$.
  It follows that
  \begin{align*}
    \Phi'(x y - y x)
    &=
    \Phi'(x) \Phi'(y) - \Phi'(y) \Phi'(x)
    \\
    &=
    \varphi(x) \varphi(y) - \varphi(y) \varphi(x)
    \\
    &=
    [ \varphi(x), \varphi(y) ]
    \\
    &=
    \varphi( [x,y]_{\glie} )
    \\
    &=
    \Phi'( [x,y]_{\glie} )
  \end{align*}
  for all~$x, y \in \glie$.
  The ideal~$I$ is therefore contained in the kernel of the homomorphism~$\Phi'$.
  It follows that there exists a unique homomorphism of algebras~$\Phi$ from~$U$ to~$A$ that makes the diagram
  \[
    \begin{tikzcd}
      \glie
      \arrow{r}[above]{\varphi}
      \arrow{d}
      &
      A
      \\
      \Tensor(\glie)
      \arrow[bend right= 20]{ur}[above left]{\Phi'}
      \arrow{d}[left]{\pi}
      &
      {}
      \\
      U
      \arrow[dashed, bend right = 30]{uur}[below right]{\Phi}
      &
      {}
    \end{tikzcd}
  \]
  commute, where~$\Pi$ denotes the canonical projection from~$\Tensor(V)$ to~$\Tensor(V)/I$.
  We may add the homomorphism of Lie~algebras~$\iota$ to this diagram.
  We then arrive at the following commutative diagram.
  \[
    \begin{tikzcd}
      \glie
      \arrow{r}[above]{\varphi}
      \arrow{d}
      \arrow[bend right = 55]{dd}[left]{\iota}
      &
      A
      \\
      \Tensor(\glie)
      \arrow[bend right= 20]{ur}[above left]{\Phi'}
      \arrow{d}[left]{\pi}
      &
      {}
      \\
      U
      \arrow[bend right = 30]{uur}[below right]{\Phi}
      &
      {}
    \end{tikzcd}
  \]
  We have in particular the following commutative subdiagram.
  \[
    \begin{tikzcd}
      \glie
      \arrow{r}[above]{\varphi}
      \arrow{d}[left]{i}
      &
      A
      \\
      U
      \arrow{ur}[below right]{\Phi}
      &
      {}
    \end{tikzcd}
  \]
  We have thus shown that every homomorphism of Lie~algebras~$\varphi$ from~$\glie$ to~$A$ extends to a homomorphism of algebra~$\Phi$ from~$U$ to~$A$ .
  The algebra~$U$ is generated by the image of~$\iota$ whence the homomorphism of algebras~$\Phi$ is unique with this property.
\end{proof}


\begin{remark}
  The above proof may be summarized by observing that we have bijections
  \begin{align*}
    {}&
    \{ \textstyle\text{algebra homomorphisms~$\Phi \colon \Tensor(\glie)/I \to A$} \}
    \\
    \cong{}&
    \{ \text{algebra homomorphisms~$\Phi' \colon \Tensor(\glie) \to A$ with~$\Phi'(I) = 0$} \}
    \\
    \cong{}&
    \left\{
      \begin{tabular}{@{}c@{}}
        algebra homomorphisms~$\Phi' \colon \Tensor(\glie) \to A$ with  \\
        $\Phi'(x y - y x - [x,y]_{\glie}) = 0$ for all~$x, y \in \glie$
      \end{tabular}
    \right\}
    \\
    \cong{}&
    \left\{
      \begin{tabular}{@{}c@{}}
        algebra homomorphisms~$\Phi' \colon \Tensor(\glie) \to A$ with  \\
        $\Phi'(x) \Phi'(y) - \Phi'(y) \Phi'(x) - \Phi'([x,y]_{\glie}) = 0$ for all~$x, y \in \glie$
      \end{tabular}
    \right\}
    \\
    \cong{}&
    \left\{
      \begin{tabular}{@{}c@{}}
        algebra homomorphisms~$\Phi' \colon \Tensor(\glie) \to A$ with  \\
        $\Phi'(x) \Phi'(y) - \Phi'(y) \Phi'(x) = \Phi'([x,y]_{\glie})$ for all~$x, y \in \glie$
      \end{tabular}
    \right\}
    \\
    \cong{}&
    \left\{
      \begin{tabular}{@{}c@{}}
        {\linear{$\kf$}} maps~$\varphi \colon \glie \to A$ with  \\
        $\varphi(x) \varphi(y) - \varphi(y) \varphi(x) = \varphi([x,y])$ for all~$x, y \in \glie$
      \end{tabular}
    \right\}
    \\
    \cong{}&
    \left\{
      \begin{tabular}{@{}c@{}}
        {\linear{$\kf$}} maps~$\varphi \colon \glie \to A$ with  \\
        $[\varphi(x), \varphi(y)] = \varphi([x,y])$ for all~$x, y \in \glie$ 
      \end{tabular}
    \right\}
    \\
    ={}&
    \{ \textstyle\text{Lie~algebra homomorphisms~$\varphi \colon \glie \to A$} \} \,,
  \end{align*}
  and that these bijections are natural in~$A$.
  This shows that the~{\algebra{$\kf$}}~$\Tensor(\glie)/I$ represents the right kind of functor.
  We can also see that the identity of~$\Tensor(\glie)/I$ corresponds under the above bijections (for~$A = \Tensor(\glie)/I$) to the map~$\iota$ from~$\glie$ to~$\Tensor(\glie)/I$.
\end{remark}



\subsection{Properties}

\subsubsection{Anti-Homomorphisms}

\begin{proposition}
  Let~$\glie$ be a Lie~algebra, let~$\iota$ be the canonical homomorphism of Lie~algebras from~$\glie$ to~$\Univ(\glie)$ and let~$A$ be a~\algebra{$\kf$}.
  We have a well-defined {\onetoonetext} correspondence given by
  \begin{align*}
    \SwapAboveDisplaySkip
    \left\{
      \begin{tabular}{@{}c@{}}
        anti-homomorphisms \\
        of Lie~algebras
        $\varphi \colon \glie \to A$
      \end{tabular}
    \right\}
    &\onetoone
    \left\{
      \begin{tabular}{@{}c@{}}
        anti-homomorphisms \\
        of algebras
        $\Phi \colon \Univ(\glie) \to A$
      \end{tabular}
    \right\} \,,
    \\
    \Phi \circ \iota
    &\mapsfrom
    \Phi \,.
  \end{align*}
\end{proposition}

\begin{proof}
  We have {\onetoonetext} correspondence given by
  \begin{align*}
    \left\{
      \begin{tabular}{@{}c@{}}
        homomorphisms \\
        of Lie~algebras
        $\varphi \colon \glie \to A^{\op}$
      \end{tabular}
    \right\}
    &\onetoone
    \left\{
      \begin{tabular}{@{}c@{}}
        homomorphisms \\
        of algebras
        $\Phi \colon \Univ(\glie) \to A^{\op}$
      \end{tabular}
    \right\} \,,
    \\
    \Phi \circ \iota
    &\mapsfrom
    \Phi \,.
  \end{align*}
  An anti-homomorphism of Lie~algebras from~$\glie$ to~$A$ is the same as a homomorphim of Lie~algebras from~$\glie$ to~$A^{\op}$, and an anti-homomorphism of algebras from~$\Univ(\glie)$ to~$A$ is the same as a homomorphism of algebras from~$\Univ(\glie)$ to~$A^{\op}$.
\end{proof}

\subsubsection{Representations and Modules}

\begin{proposition}
  \label{representations are modules}
  Let~$M$ be a~{\vectorspace{$\kf$}} and let~$\glie$ be a Lie~algebra.
  Let~$\Univ(\glie)$ be the universal enveloping algebra of~$\glie$ and let~$\iota$ be the canonical homorphism of Lie~algebras from~$\glie$ to~$\Univ(\glie)$.
  \begin{enumerate}
    \item
      For every homomorphism of Lie~algebras~$\rho$ from~$\glie$ to~$\gllie(M)$ let~$\widehat{\rho}$ denote the corresponding homomorphism of algebras from~$\Univ(\glie)$ to~$\End_{\kf}(M)$.
      Then the assignments
      \begin{align*}
        \left\{
        \begin{tabular}{@{}c@{}}
          representations \\
          $\rho \colon \glie \to \gllie(M)$
        \end{tabular}
        \right\}
        &\onetoone
        \left\{
        \begin{tabular}{@{}c@{}}
          $\Univ(\glie)$-module structures \\
          $\Rho \colon \Univ(\glie) \to \End_{\kf}(M)$
        \end{tabular}
        \right\}  \,,
        \\
        \rho
        &\mapsto
        \widehat{\rho} \,,
        \\
        \Rho \circ \iota
        &\mapsfrom
        \Rho  \,,
      \end{align*}
      constitute a {\onetoonetext} correspondence.
    \item
      For every anti-homomorphism of Lie~algebras~$\rho$ from~$\glie$ to~$\gllie(M)$ let~$\widehat{\rho}$ denote the corresponding anti-homomorphism of algebras from~$\Univ(\glie)$ to~$\End_{\kf}(M)$.
      Then the assignments
      \begin{align*}
        \left\{
        \begin{tabular}{@{}c@{}}
          right representations \\
          $\rho \colon \glie \to \gllie(M)$
        \end{tabular}
        \right\}
        &\onetoone
        \left\{
        \begin{tabular}{@{}c@{}}
          right $\Univ(\glie)$-module structures \\
          $\Rho \colon \Univ(\glie) \to \End_{\kf}(M)$
        \end{tabular}
        \right\}  \,,
        \\
        \rho
        &\mapsto
        \widehat{\rho} \,,
        \\
        \Rho \circ \iota
        &\mapsfrom
        \Rho  \,,
      \end{align*}
      constitute a {\onetoonetext} correspondence.
  \end{enumerate}
\end{proposition}

\begin{fluff}
  Let~$\glie$ be a Lie~algebra.
  If~$M$ is a left~\module{$\Univ(\glie)$} then the corresponding act of~$\glie$ on~$M$ is given by
  \[
    x \act m
    =
    \class{x} \cdot m
  \]
  for all~$x \in \glie$,~$m \in M$.
  Similarly, if~$M$ is a right~\module{$\Univ(\glie)$} then the corresponding right action of~$\glie$ on~$M$ is given by
  \[
    m \act x
    =
    m \cdot \class{x}
  \]
  for all~$x \in \glie$,~$m \in M$.
\end{fluff}

\subsubsection{Functoriality}

\begin{remark}
  Let~$\glie$ be a Lie~algebra with universal enveloping algebra~$\Univ(\glie)$.
  \Cref{representations are modules} shows that representations of~$\glie$ are the same as~{\modules{$\Univ(\glie)$}}.
  We get from this correspondence an isomorphism of categories between~$\cRep{\glie}$ and~$\cMod{\Univ(\glie)}$.
\end{remark}


\begin{lemma}[Functoriality of the universal enveloping algebra]
  \label{functoriality of universal enveloping algebra}
  Let~$\glie$,~$\hlie$ and~$\klie$ be Lie~algebras.
  \begin{enumerate}
    \item
      For every homomorphism of Lie~algebras~$\varphi$ from~$\glie$ to~$\hlie$ there exists a unique homomorphism of algebras~$\Univ(\varphi)$ from~$\Univ(\glie)$ to~$\Univ(\hlie)$ that makes the following square diagram commute.
      \[
        \begin{tikzcd}[column sep = large]
          \glie
          \arrow{r}[above]{\varphi}
          \arrow{d}[left]{\iota_{\glie}}
          &
          \hlie
          \arrow{d}[right]{\iota_{\hlie}}
          \\
          \Univ(\glie)
          \arrow[dashed]{r}[below]{\Univ(\varphi)}
          &
          \Univ(\hlie)
        \end{tikzcd}
      \]
    \item
      It holds that~$\Univ(\glie) = \id_{\Univ(\glie)}$.
    \item
      It holds for all composable homomorphisms of Lie~algebras~$\varphi$ from~$\glie$ to~$\hlie$ and~$\psi$ from~$\hlie$ to~$\klie$ that
      \[
        \Univ( \psi \circ \varphi )
        =
        \Univ( \psi ) \circ \Univ( \varphi ) \,.
      \]
  \end{enumerate}
\end{lemma}


\begin{proof}
  \leavevmode
  \begin{enumerate}
    \item
      The composite~$\iota_{\hlie} \circ \varphi$ is a homomorphism of Lie~algebras from~$\glie$ to~$\Univ(\hlie)$.
      By the universal property of the universal enveloping algebra~$\Univ(\glie)$ there exists a unique homomorphism of algebras~$\Univ(\varphi)$ from~$\Univ(\glie)$ to~$\Univ(\hlie)$ with~$\Univ(\varphi) \circ \iota_{\glie} = \iota_{\hlie} \circ \varphi$.
    \item
      The square diagram
      \[
        \begin{tikzcd}[column sep = huge]
          \glie
          \arrow{r}[above]{\id_{\glie}}
          \arrow{d}
          &
          \glie
          \arrow{d}
          \\
          \Univ(\glie)
          \arrow[dashed]{r}[below]{\id_{\Univ(\glie)}}
          &
          \Univ(\glie)
        \end{tikzcd}
      \]
      commutes, which shows that the identity homomorphism~$\id_{\Univ(\glie)}$ satisfies the defining property of the induced algebra homomorphism~$\Univ( \id_{\glie} )$.
    \item
      We have the following commutative diagram:
      \[
        \begin{tikzcd}[column sep = large]
          \glie
          \arrow[dashed, bend left = 40]{rr}[above]{\psi \circ \varphi}
          \arrow{r}[above]{\varphi}
          \arrow{d}
          &
          \hlie
          \arrow{r}[above]{\psi}
          \arrow{d}
          &
          \klie
          \arrow{d}
          \\
          \Univ(\glie)
          \arrow{r}[below]{\Univ(\varphi)}
          \arrow[dashed, bend right = 40]{rr}[below]{\Univ(\psi) \circ \Univ(\varphi)}
          &
          \Univ(\hlie)
          \arrow{r}[below]{\Univ(\psi)}
          &
          \Univ(\klie)
        \end{tikzcd}
      \]
      The commutativity of the outer square diagram
      \[
        \begin{tikzcd}[column sep = huge]
          \glie
          \arrow{r}[above]{\psi \circ \varphi}
          \arrow{d}
          &
          \klie
          \arrow{d}
          \\
          \Univ(\glie)
          \arrow[dashed]{r}[below]{\Univ(\psi) \circ \Univ(\varphi)}
          &
          \Univ(\klie)
        \end{tikzcd}
      \]
      shows that the composite~$\Univ(\psi) \circ \Univ(\varphi)$ satisfies the defining property of the induced algebra homomorphism~$\Univ(\psi \circ \varphi)$.
    \qedhere
  \end{enumerate}
\end{proof}


\begin{remark}
  \Cref{functoriality of universal enveloping algebra} shows that the assignment~$\glie \mapsto \Univ(\glie)$ of a Lie~algebra~$\glie$ to its universal eveloping algebra~$\Univ(\glie)$ can be extended to a (covariant) functor~$\Univ$ from~$\cLie{\kf}$ to~$\cAlg{\kf}$.
  The universal property of the {\ua} states that the functor~$\Univ$ is left adjoint to the forgetful functor from~$\cAlg{\kf}$ to~$\cLie{\kf}$, which assigns to each~{\algebra{$\kf$}} its underlying Lie~algebra.
\end{remark}

\subsubsection{Derivations}

\begin{proposition}
  \label{extending derivation to universal enveloping algebra}
  Let~$\glie$ be a Lie~algebra.
  Every derivation of~$\glie$ extends uniquely to an extension of~$\Univ(\glie)$.
  More explicitely, there exists for every Lie~algebra derivation~$\delta$ of~$\glie$ a unique algebra derivation~$\Delta$ of~$\Univ(\glie)$ such that the following square diagram commutes.
  \[
    \begin{tikzcd}
      \glie
      \arrow{r}[above]{\delta}
      \arrow{d}
      &
      \glie
      \arrow{d}
      \\
      \Univ(\glie)
      \arrow[dashed]{r}[above]{\Delta}
      &
      \Univ(\glie) \,.
    \end{tikzcd}
  \]
\end{proposition}


\begin{lemma}
  \label{translating between derivations and homomorphisms}
  \leavevmode
  \begin{enumerate}
    \item
      Let~$A$ be a~\algebra{$\kf$} and let~$B$ be the~\algebra{$\kf$}
      \[
        B
        \defined
        \begin{pmatrix}
          A & A \\
          0 & A
        \end{pmatrix} \,.
      \]
      A map~$\Delta$ from~$A$ to~$A$ is a derivation of~$A$ if and only if the map
      \[
        \Phi
        \colon
        A
        \to
        B \,,
        \quad
        a
        \mapsto
        \begin{pmatrix}
          a & \Delta(a) \\
          0 & a
        \end{pmatrix}
      \]
      is a homomorphism of algebras.
    \item
      Let~$\glie$ be a Lie~algebra and let
      \[
        B
        \defined
        \begin{pmatrix}
          \Univ(\glie)  & \Univ(\glie)  \\
          0             & \Univ(\glie)
        \end{pmatrix} \,.
      \]
      A map~$\delta$ from~$\glie$ to~$B$ is a derivation of~$\glie$ if and only if the map
      \[
        \varphi
        \colon
        \glie
        \to
        B \,,
        \quad
        x
        \mapsto
        \begin{pmatrix}
          x & x \\
          0 & x
        \end{pmatrix}
      \]
      is a homomorphism of Lie~algebras
  \end{enumerate}
\end{lemma}


\begin{proof}
  \leavevmode
  \begin{enumerate}
    \item
      The map~$\Phi$ is linear if and only if the map~$\Delta$ is linear.
      We have
      \[
        \Phi(a) \cdot \Phi(b)
        =
        \begin{pmatrix}
          a & \Delta(a) \\
          0 & a
        \end{pmatrix}
        \begin{pmatrix}
          b & \Delta(b) \\
          0 & b
        \end{pmatrix}
        =
        \begin{pmatrix}
          ab  & \Delta(a) b + a \Delta(b)
          0   & ab
        \end{pmatrix}
      \]
      for all~$a, b \in A$.
      The map~$\Phi$ is thus multiplicaitve if and only if~$\Delta(ab) = \Delta(a) b + a \Delta(b)$ for all~$a, b \in A$.
      If~$\Delta$ is a derivation then we also have~$\Delta(1) = 0$ and thus~$\Phi(1) = 1$.
      This shows altogether that~$\Delta$ is a derivation if and only if~$\Phi$ is a homomorphism of algebras.
    \item
      We have
      \begin{align*}
        [ \varphi(x), \varphi(y) ]
        &=
        \Biggl[
          \begin{pmatrix}
            x & \delta(x) \\
            0 & x
          \end{pmatrix},
          \begin{pmatrix}
            y & \delta(y) \\
            0 & y
          \end{pmatrix}
        \Biggr]
        \\
        &=
        \begin{pmatrix}
          x & \delta(x) \\
          0 & x
        \end{pmatrix}
        \begin{pmatrix}
          y & \delta(y) \\
          0 & y
        \end{pmatrix}
        -
        \begin{pmatrix}
          y & \delta(y) \\
          0 & y
        \end{pmatrix}
        \begin{pmatrix}
          x & \delta(x) \\
          0 & x
        \end{pmatrix}
        \\
        &=
        \begin{pmatrix}
          xy  & x \delta(y) + \delta(x) y \\
          0   & xy
        \end{pmatrix}
        -
        \begin{pmatrix}
          yx  & y \delta(x) + \delta(y) x \\
          0   & yx
        \end{pmatrix}
        \\
        &=
        \begin{pmatrix}
          xy - yx & \delta(x) y - y \delta(x) + x \delta(y) - \delta(y) x \\
          0       & xy - yx
        \end{pmatrix}
        \\
        &=
        \begin{pmatrix}
          [x,y] & [\delta(x), y] + [x, \delta(y)] \\
          0     & [x,y]
        \end{pmatrix}
      \end{align*}
      for all~$x, y \in \glie$.
      It follows that the map~$\varphi$ is a homomorphism of Lie~algebras if and only if~$\delta([x,y]) = [\delta(x), y] + [x, \delta(y)]$ for all~$x, y \in \glie$, i.e. if and only if the map~$\delta$ is a derivation of~$\glie$.
    \qedhere
  \end{enumerate}
\end{proof}


\begin{proof}[Proof of \cref{extending derivation to universal enveloping algebra}]
  The uniqueness of~$\Delta$ follows from \cref{dervation is uniquely determined by algebra generators} because~$\glie$ generates the algebra~$\Univ(\glie)$.

  Let~$B$ be the~\algebra{$\kf$} given by
  \[
    B
    \defined
    \begin{pmatrix}
      \Univ(\glie)  & \Univ(\glie) \\
      0             & \Univ(\glie)
    \end{pmatrix} \,.
  \]
  It follows from \cref{translating between derivations and homomorphisms} that the map
  \[
    \varphi
    \colon
    \glie
    \to
    A \,,
    \quad
    x
    \mapsto
    \begin{pmatrix}
      x & \delta(x) \\
      0 & x
    \end{pmatrix}
  \]
  is a homomorphism of Lie~algebras because~$\delta$ is a derivation of~$\glie$.
  It follows from the universal property of the universal enveloping algebra~$\Univ(\glie)$ that the homomorphism of Lie~algebras~$\varphi$ extends uniquely to a homomorphism of algebras~$\Phi$ from~$\Univ(\glie)$ to~$B$.
  This homomorphism~$\Phi$ is of the form
  \[
    \Phi(x)
    =
    \begin{pmatrix}
      \Phi_1(x) & \Delta(x) \\
      0         & \Phi_2(x)
    \end{pmatrix}
  \]
  for all~$x \in \Univ(\glie)$ for some unique linear mape
  \[
    \Phi_1, \Phi_2, \Delta
    \colon
    \Univ(\glie)
    \to
    \Univ(\glie) \,.
  \]
  The maps~$\Phi_1$ and~$\Phi_2$ are homomorphisms of algebras because~$\Phi$ is a homomorphism of algebras.
  They satisfy the equalities~$\Phi_1(x) = \varphi(x) = x$ and~$\Phi_2(x) = \varphi(x) = x$ for all~$x \in \glie$.
  It follows that~$\Phi_1(x) = x$ and~~$\Phi_2(x) = x$ for all~$x \in \Univ(\glie)$ because~$\glie$ generates~$\Univ(\glie)$ as an algebra.
  The homomorphism~$\Phi$ is thus of the form
  \[
    \Phi(x)
    =
    \begin{pmatrix}
      x & \Delta(x) \\
      0 & x
    \end{pmatrix}
  \]
  for all~$x \in \glie$.
  It follows from \cref{translating between derivations and homomorphisms} that the linear map~$\Delta$ is an algebra derivation of~$\Univ(\glie)$.
  This darivation satisfies the equality~$\Delta(x) = \delta(x)$ for all~$x \in \glie$ because~$\Phi$ is an extension of~$\varphi$.
  In other words,~$\Delta$ is an extension of~$\delta$.
\end{proof}



\chapter{The Poincaré--Birkhoff--Witt Theorem}


\subsection{Graded \texorpdfstring{$\kf$}{k}-Algebras}


\begin{definition}
  \label{graded algebras}
  A \defemph{grading}\index{grading!of an algebra} or \defemph{gradation} of a~\algebra{$\kf$}~$A$ is a direct sum decomposition
  \[
    A
    =
    \bigoplus_{i \geq 0} A_i
  \]
  into linear subspaces~$A_i$ such that
  \[
    A_i A_j
    \subseteq
    A_{i+j}
  \]
  for all~$i, j \geq 0$.
  A \defemph{graded~\algebra{$\kf$}}\index{graded!algebra} is a~\algebra{$\kf$}~$A$ together with a grading~$A = \bigoplus_{i \geq 0} A_i$.
\end{definition}


\begin{remark}
  We will often just say that~$A$ is a graded algebra without mentioning the grading.
\end{remark}


\begin{remark}
  Given any abelian semigroup~$(S, +)$ an~\defemph{\grading{$S$}}\index{grading} of a~\algebra{$\kf$}~$A$ is a decomposition
  \[
    A
    =
    \bigoplus_{s \in S} A_s
  \]
  into linear subspaces such that
  \[
    A_s A_t
    \subseteq
    A_{s + t}
  \]
  for all~$s,t \in S$.
  An~{\graded{$S$}}~\algebra{$\kf$} is a~\algebra{$\kf$}~$A$ together with an~\grading{$S$}~$A = \bigoplus_{s \in S} A_s$.
  A graded~\algebra{$\kf$} in the sense of \cref{graded algebras} is then the special case of an~{\graded{$\Natural$}}~\algebra{$\kf$}.
  We will restrict our attention to~{\gradings{$\Natural$}} and refer to \cite[II.{\S}11, III.{\S}3]{bourbaki_algebra} for more general gradings.
\end{remark}


\begin{definition}
  Let~$A$ be a graded~{\algebra{$\kf$}} with grading~$A = \bigoplus_{i \geq 0} A_i$.
  \begin{enumerate}
    \item
      An element~$x \in A$ is \defemph{homogeneous}\index{homogeneous!element} of \defemph{degree}~$i$\index{homogeneous!degree}\index{degree!homogeneous} if~$x \in A_i$.
    \item
      If the decomposition of~$x \in A$ with respect to the grading~$A = \bigoplus_{i \geq 0} A_i$ is given by~$x = \sum_{i \geq 0} x_i$ then the summands~$x_i$ are the~\defemph{homogeneous components}\index{homogeneous!component} of~$x$.
  \end{enumerate}
\end{definition}


\begin{lemma}
  If~$A$ is a graded~\algebra{$\kf$} then~$1_A$ is homogeneous of degree~$0$.
\end{lemma}


\begin{proof}
  Let~$1 = \sum_{i \geq 0} e_i$ with respect to the grading~$A = \bigoplus_{i \geq 0} A_i$.
  Then for any~$j \geq 0$ and~$a \in A_j$ we have that
  \[
    a
    =
    1 \cdot a
    =
    \left( \sum_{i \geq 0} e_i \right) \cdot a
    =
    \sum_{i \geq 0} \underbrace{ e_i a }_{\in A_{i+j}}  \,,
  \]
  and it follows from the directness of the decomposition~$A = \bigoplus_{i \geq 0} A_i$ that already~$a = e_0 a$.
  It follows that~$e_0 a = a$ for every~$a \in A$, so that already~$1 = e_0$.
\end{proof}


\begin{corollary}
  If~$A$ is a graded~\algebra{$\kf$} then~$A_0$ is a~{\subalgebra{$\kf$}} of~$A$.
  \qed
\end{corollary}



\begin{examples}
  \label{examples of graded algebras}
  \leavevmode
  \begin{enumerate}
    \item
      Any~\algebra{$\kf$}~$A$ becomes a graded~\algebra{$\kf$} by setting~$A_0 \defined A$ and~$A_i \defined 0$ for every~$i \geq 1$.
    \item
      The polynomial ring~$A = \kf[x_i \suchthat i \in I]$ is a graded~\algebra{$\kf$} by setting
      \[
        A_d
        \defined
        \gen{
          x_{i_1}^{p_1} \dotsm x_{i_n}^{p_n}
        \suchthat
          n \geq 0,
          i_1, \dotsc, i_n \in I,
          p_1 + \dotsb + p_n = d
        }_{\kf}
      \]
      for every~$d \geq 0$.
      Then~$A_d$ consists of the homogeneous polynomials of degree~$d$, and the monomials~$x_{i_1}^{p_1} \dotsm x_{i_n}^{p_n}$ are homogeneous of degree~$p_1 + \dotsb + p_n$.
      
      We can more generally put the variable~$x_i$ is any degree~$d_i$:
      Given any family~$(d_i)_{i \in I}$ of natural numbers~$d_i \geq 0$ we can define a grading on~$A$ via
      \[
        A_d
        \defined
        \gen{
          x_{i_1}^{p_1} \dotsm x_{i_n}^{p_n}
        \suchthat
          n \geq 0,
          i_1, \dotsc, i_n \in I,
          p_1 d_1 + \dotsb + p_n d_n = d
        }_\kf
      \]
      for every~$d \geq 0$.
      Then the monomials~$x_{i_1}^{p_1} \dotsm x_{i_n}^{p_n}$ are homogeneous of degree~$p_1 d_1 + \dotsb + p_n d_n$.
    \item
      Similarly the free~{\algebra{$\kf$}}~$A \defined \kf\gen{x_i \suchthat i \in I}$ can be graded via
      \[
        A_d
        \defined
        \gen{
          x_{i_1}^{p_1} \dotsm x_{i_n}^{p_n}
        \suchthat
          n \geq 0,
          i_1, \dotsc, i_n \in I,
          p_1 + \dotsb + p_n = d
        }_{\kf}
      \]
      for every~$d \geq 0$, and for any family~$(d_i)_{i \in I}$ of natural numbers $d_i \geq 0$ via
      \[
        A_d
        \defined
        \gen{
          x_{i_1}^{p_1} \dotsm x_{i_n}^{p_n}
        \suchthat
          n \geq 0,
          i_1, \dotsc, i_n \in I,
          p_1 d_1 + \dotsb + p_n d_n = d
        }_{\kf} \,.
      \]
      This grading makes the monomials~$x_{i_1}^{p_1} \dotsm x_{i_n}^{p_n}$ homogeneous of degree~$p_1 d_1 + \dotsb + d_n p_n$.
    \item
      The tensor algebra~$\Tensor(V)$ of a vector space~$V$ has a canonical grading with~$\Tensor(V)_d = V^{\tensor d}$ for all~$d \geq 0$.
      Similarly both the symmetric algebra~$\Symm(V)$ and the exterior algebra~$\Exterior(V)$ have canonical gradings given by~$\Symm(V)_d = \Symm^d(V)$ and~$\Exterior(V)_d = \Exterior^d(V)$ for all~$d \geq 0$.
    \item
      Let~$(M, \cdot)$ be a monoid and let~$M = \coprod_{i \geq 0} M_i$ be a grading of~$M$, i.e.\ a decomposition into subsets with~$M_i \cdot M_j \subseteq M_{i+j}$ for all~$i, j \geq 0$.
      Then the monoid algebra~$\kf[M]$ inhericts a grading via
      \[
        \kf[M]_i
        \defined
        \gen{ M_i }_{\kf}
      \]
      for every~$i \geq 0$.
      As special cases of this construction we get the following examples:
      \begin{itemize}
        \item
          If~$I$ is any index set then for~$M = \Natural^{\oplus I}$ a grading~$M = \coprod_{d \geq 0} M_d$ is given by the subsets
          \[
            M_d
            \defined
            \left\{
              (p_i)_{i \in I} \in M
            \suchthat*
              \sum_{i \in I} p_i = d
            \right\}
          \]
          with~$d \geq 0$.
          Then~$\kf[M]$ with the induced grading is the commutative polynomial algebra~$\kf[x_i \suchthat i \in I]$.
        \item
          If~$I$ is any index set then let~$\Sigma \defined \{x_i \suchthat i \in I\}$ be an alphabet with letters~$x_i$ and let~$M = \Sigma^*$ be the monoid of words in the alphabet~$\Sigma$ together with concatenation of words as rule of composition.
          Then~$\kf[M] = \kf\gen{X_i \suchthat i \in I}$ and the grading of~$\kf[M]$ (where every variable~$x_i$ is in degree~$1$) comes from the grading of~$M$ given by
          \[
            M_d
            \defined
            \{
              w \in M
            \suchthat
              \text{$w$ is a word in~$\Sigma$ of length~$d$}
            \}  \,.
          \]
      \end{itemize}
  \end{enumerate}
\end{examples}


\begin{remark}
  \label{external description of graded algebras}
  The grading of the tensor algebra~$\Tensor(V)$, symmetric algebra~$\Symm(V)$ and exterior algebra~$\Exterior(V)$ are basically built into the construction of these algebras.
  This way of constructing graded~{\algebras{$\kf$}} can be generalized as follows:
  
  Suppose that we are given a sequence of vector spaces~$A_i$ with~$i \geq 0$ and bilinear maps
  \[
    \mu_{i,j}
    \colon
    A_i \times A_j
    \to
    A_{i+j} \,,
    \quad
    (x,y)
    \mapsto
    xy
  \]
  for all~$i, j \geq 0$ such that
  \begin{itemize}
    \item
      the~$\mu_{i, j}$ are relatively associative in the sense that
      \[
        x(yz)
        =
        (xy)z
      \]
      for all~$i, j, k \geq 0$ and~$x \in A_i$,~$y \in A_j$ and~$z \in A_k$ and
    \item
      there exists an element~$1 \in A_0$ with
      \[
        1 x
        =
        x
        =
        x 1
      \]
      for all~$i \geq 0$ and every~$x \in A_i$.
  \end{itemize}
  For the direct sum~$A \defined \bigoplus_{i \geq 0} A_i$ we can then fit together the multiplications~$\mu_{i,j}$ to a single multiplication~$\mu \colon A \times A \to A$ that is given on elements~$x, y \in A$ with~$x = (x_i)_{i \geq 0}$ and~$y = (y_i)_{i \geq 0}$ by
  \[
    x y
    =
    \left( \sum_{j=0}^i x_j y_{i-j} \right)_{j \geq 0}  \,.
  \]
  The bilinearity of~$\mu$ follows from the bilinearities of the~$\mu_{i,j}$, it follows from the relative associativity of the multiplications~$\mu_{i,j}$ that the multiplication~$\mu$ is associative, and~$1 \in A_0$ is multiplicative neutral for~$\mu$.
  
  This construction gives an external description of graded~{\algebras{$\kf$}}, in constrast to the previous internal description.
\end{remark}


\begin{definition}
  Let~$A$ and~$B$ be two graded~\algebras{$\kf$}.
  \begin{enumerate}
    \item
      A \defemph{homomorphism of graded~{\algebras{$\kf$}}}~$f \colon A \to B$\index{homomorphism!of graded $\kf$-algebras} is a homomorphism of~{\algebras{$\kf$}} such that~$f(A_i) \subseteq B_i$ for all~$i \geq 0$.
    \item
      If~$f \colon A \to B$ is a homomorphism of graded~{\algebras{$\kf$}} then for every~$i \geq 0$ the restriction of~$f$ to a linear map~$A_i \to B_i$ is denoted by~$f_i$.
  \end{enumerate}
\end{definition}


\begin{remark}
  \leavevmode
  \begin{enumerate}
    \item
      Let~$A$,~$B$ and~$C$ be graded~{\algebra{$\kf$}}.
      \begin{enumerate}
        \item
          The identity~$\id_A \colon A \to A$ is a homomorphism of graded~{\algebras{$\kf$}}.
        \item
          If~$f \colon A \to B$ and~$g \colon B \to C$ are homomorphisms of graded~{\algebras{$\kf$}} then their composition~$g \circ f \colon A \to C$ is again a homomorphism of graded~{\algebras{$\kf$}}.
      \end{enumerate}
      Hence the graded~\algebras{$\kf$} together with the homomorphisms of graded~\algebras{$\kf$} between them form a category, which we will denote by~\gls*{graded algebras}.
    \item
      For a homomorphism of graded~{\algebras{$\kf$}}~$f \colon A \to B$ the following conditions are equivalent:
      \begin{equivalenceslist}
        \item
          $f$ is an isomorphism of graded~{\algebras{$\kf$}}, i.e.\ there exists an inverse homomorphism of graded~{\algebras{$\kf$}}~$g \colon B \to A$ with~$fg = \id_B$ and~$gf = \id_A$.
        \item
          $f$ is bijective.
        \item
          For every~$i \geq 0$ the restriction~$f_i \colon A_i \to B_i$ is bijective.
      \end{equivalenceslist}
  \end{enumerate}
\end{remark}


\begin{example}
  \leavevmode
  \begin{enumerate}
    \item
      For any vector space~$V$ the two maps
      \begin{alignat*}{2}
        \Tensor(V)
        &\to
        \Symm(V) \,,
        &
        \quad
        x_1 \tensor \dotsb \tensor x_n
        &\mapsto
        x_1 \dotsm x_n
      \shortintertext{and}
        \Tensor(V)
        &\to
        \Exterior(V) \,,
        &
        \quad
        x_1 \tensor \dotsb \tensor x_n
        &\mapsto
        x_1 \wedge \dotsb \wedge x_n
      \end{alignat*}
      are homomorphisms of graded~\algebras{$\kf$}.
    \item
      If~$V$ is a finite dimensional vector space with basis~$x_1, \dotsc, x_n$ then the resulting isomorphism of~\algebras{$\kf$}
      \[
        \kf[X_1, \dotsc, X_n]
        \to
        \Symm(V) \,,
        \quad
        X_i
        \mapsto
        x_i
      \]
      is already an isomorphism of graded~\algebras{$\kf$}.
  \end{enumerate}
\end{example}


\begin{lemma}
  \label{characterizations of homogeneous ideals}
  Let~$I$ be some kind of ideal in a graded~{\algebra{$\kf$}} (i.e.\ a left ideal, right ideal or two-sided ideal).
  \begin{enumerate}
    \item
      The linear subspace~$\bigoplus_{i \geq 0} (I \cap A_i)$ is again an ideal in~$A$ of the same kind.
    \item
      The following conditions on~$I$ are equivalent:
      \begin{enumerate}
        \item
          \label{direct sum of linear subspaces}
          There exists linear subspaces~$I_i$ of~$A_i$ with~$i \geq 0$ such that~$I = \bigoplus_{i \geq 0} I_i$.
        \item
          \label{direct sum of intersections}
          It holds that~$I = \bigoplus_{i \geq 0} (I \cap A_i)$.
        \item
          \label{contains all homogeneous components}
          The ideal~$I$ contains the homogeneous components of all its elements.
        \item
          \label{generated by homogeneous}
          The ideal~$I$ is generated by homogeneous elements.
      \end{enumerate}
  \end{enumerate}
\end{lemma}


\begin{proof}
  \leavevmode
  \begin{enumerate}
    \item
      We check that~$I' \defined \bigoplus_{i \geq 0} I \cap A_i$ is again a left ideal if~$I$ is one.
      Indeed, we find that
      \begin{align*}
        A \cdot I'
        &=
              \left( \sum_{j \geq 0} A_j \right)
        \cdot \left( \sum_{i \geq 0} (I \cap A_i) \right)
        \\
        &=
        \sum_{i,j \geq 0} A_j \cdot (I \cap A_i)
        \\
        &\subseteq
        \sum_{i, j \geq 0} (A_j \cdot I) \cap (A_j \cdot A_i)
        \\
        &\subseteq
        \sum_{i,j \geq 0} (I \cap A_{i+j})
        \\
        &=
        \sum_{k \geq 0} (I \cap A_k)
        \\
        &=
        I'  \,.
      \end{align*}
      We find in the same way that~$I'$ is again a right ideal if~$I$ is one.
      It follows from these two cases that~$I'$ is a two-sided ideal if~$I$ is one.
    \item
      \begin{implicationlist}
        \item[\ref*{direct sum of linear subspaces}~$\implies$~\ref*{direct sum of intersections}]
          It follows from the assumption that~$I_i = I \cap A_i$.
        \item[\ref*{direct sum of intersections}~$\implies$~\ref*{direct sum of linear subspaces}]
          We may take~$I_i = I \cap A_i$.
        \item[\ref*{direct sum of linear subspaces}~$\implies$~\ref*{contains all homogeneous components}]
          We may decompose~$x \in I$ with respect to the decomposition~$A = \bigoplus_{i \geq 0} A_i$ into homogeneuos components~$x = \sum_{i \geq 0} x_i$ and with respect to the decomposition~$I = \bigoplus_{i \geq 0} I_i$ as~$x = \sum_{i \geq 0} x'_i$.
          Then~$x'_i \in I_i \subseteq A_i$ and hence~$x_i = x'_i$ by uniqueness of the decomposition in~$A$.
          Thus~$x_i = x'_i \in I_i \subseteq I$ for all~$i \geq 0$.
        \item[\ref*{contains all homogeneous components}~$\implies$~\ref*{generated by homogeneous}]
          We may start with any generating set for~$I$ and then replace each generator by all of its homogeneous components.
        \item[\ref*{generated by homogeneous}~$\implies$~\ref*{direct sum of intersections}]
          Each homogeneous generator in contained in some summand~$I \cap A_i$.
          Hence~$I$ is contained in the ideal~$\bigoplus_{i \geq 0} I \cap A_i$ which is in turn contained in~$I$.
          Hence~$I = \bigoplus_{i \geq 0} I \cap A_i$.
        \qedhere
      \end{implicationlist}
  \end{enumerate}
\end{proof}


\begin{definition}
  An ideal~$I$ (of any kind) in a graded~{\algebra{$\kf$}}~$A$ is \defemph{homogeneous ideal}\index{homogeneous!ideal} if it satisfies the equivalent conditions from \cref{characterizations of homogeneous ideals}.
  We set~$I_i \defined I \cap A_i$ for all~$i \geq 0$.
\end{definition}


\begin{example}
  If~$f \colon A \to B$ is a homomorphism of graded~{\algebras{$\kf$}} then~$\ker f$ is a homogeneous two-sided ideal in~$A$ because~$\ker f = \bigoplus_{i \geq 0} \ker f_i$.
  The converse also holds:
\end{example}


\begin{proposition}
  Let~$A$ be a graded~{\algebra{$\kf$}} and let~$I$ be a two-sided homogeneous ideal in~$A$.
  Let~$\pi \colon A \to A/I$ be the canonical projection.
  \begin{enumerate}
    \item
      The algebra~$A/I$ carries a grading via~$(A/I)_i \defined \pi(A_i)$ for every~$i \geq 0$, making~$A/I$ into a graded~{\algebra{$\kf$}}.
    \item
      This is the unique grading on~$A/I$ that makes~$\pi$ a homomorphism of graded~{\algebras{$\kf$}}.
  \end{enumerate}
\end{proposition}


\begin{proof}
  \leavevmode
  \begin{enumerate}
    \item
      It follows from the usual isomorphism
      \[
        A/I
        =
        \left( \bigoplus_{i \geq 0} A_i \right)
        \bigg/
        \left(\bigoplus_{i \geq 0} I_i \right)
        \cong
        \bigoplus_{i \geq 0} (A_i / I_i)
      \]
      that~$A/I = \bigoplus_{i \in I} (A/I)_i$.
      It holds for all~$i, j \geq 0$ that
      \[
        (A/I)_i (A/I)_j
        =
        \pi(A_i) \pi(A_j)
        =
        \pi(A_i A_j)
        \subseteq
        \pi(A_{i+j})
        =
        (A/I)_{i+j} \,.
      \]
    \item
      Any such grading~$A/I = \bigoplus_{i \geq 0} (A/I)_i$ must satisfy~$\pi(A_i) \subseteq (A/I)_i$ for every~$i \geq 0$.
      It follows from~$A/I = \bigoplus_{i \geq 0} \pi(A_i)$ and~$A/I = \bigoplus_{i \geq 0} (A/I)_i$ that~$\pi(A_i) = (A/I)_i$ for every~$i \geq 0$.
    \qedhere
  \end{enumerate}
\end{proof}


\begin{examples}
  Let~$V$ be a vector space.
  The two-sided ideal~$I$ in~$\Tensor(V)$ generated by the elements~$x \tensor y - y \tensor x$ with~$x,y \in V$ is a homogeneous ideal of~$\Tensor(V)$ because it is generated by homogeneous components.
  The two sided ideal~$J$ in~$\Tensor(V)$ generated by the elements~$x \tensor x$ with~$x \in V$ is homogeneous because it generated by homogeneous elements.
  The resulting quotient graded~{\algebras{$\kf$}} are the symmetric algebra~$\Symm(V) \cong A/I$ and the exterior algebra~$\Exterior(V) \cong A/J$.
\end{examples}


% Grading is on the wrong level.
% 
% \begin{remark}
%   One can also consider graded Lie~algebras, and if~$\glie$ is a graded Lie~algebra then~$\Univ(\glie)$ inherits the structure of a graded~{\algebra{$\kf$}}:
%   \begin{enumerate}
%     \item
%       A \defemph{grading}\index{grading!of a vector space} of a vector space~$V$ is a direct sum decomposition~$V = \bigoplus_{i \geq 0} V_i$.
%       A \defemph{graded vector space}\index{graded!vector space} is a vector space~$V$ together with a grading of~$V$.
%     \item
%       A \defemph{grading}\index{grading!of a Lie algebra} of a Lie~algebra~$\glie$ is a direct sum decomposition~$\glie = \bigoplus_{i \geq 0} \glie_i$ such that~$[\glie_i, \glie_j] \subseteq \glie_{i+j}$ for all~$i, j \geq 0$.
%       A \defemph{graded Lie~algebra}\index{graded!Lie~algebra} is a Lie~algebra~$\glie$ together with a grading of~$\glie$.
%     \item
%       If~$V$ is a graded vector space then the tensor algebra~$\Tensor(V)$ inherits a grading from~$V$:
%       We get for every~$d \geq 0$ a decomposition
%       \[
%         V^{\tensor d}
%         =
%         \left(
%           \bigoplus_{i \geq 0} V_i
%         \right)^{\tensor d}
%         =
%         \bigoplus_{i_1, \dotsc, i_d \geq 0} V_{i_1} \tensor \dotsb \tensor V_{i_d}  \,.
%       \]
%       This overall results for the tensor algebra~$\Tensor(V)$ in a decomposition
%       \[
%         \Tensor(V)
%         =
%         \bigoplus_{r \geq 0}
%         V^{\tensor r}
%         =
%         \bigoplus_{\substack{r \geq 0 \\ i_1, \dotsc, i_r \geq 0}}
%         V_{i_1} \tensor \dotsb \tensor V_{i_r}  \,.
%       \]
%       We define for all~$d \geq 0$ the homogeneous component~$\Tensor(V)_d$ as
%       \[
%         \Tensor(V)_d
%         \defined
%         \bigoplus_{
%           \substack{r \geq 0 \\
%                     i_1, \dotsc, i_r \geq 0 \\
%                     i_1 + \dotsb + i_r = d}
%         }
%         V_{i_1} \tensor \dotsb \tensor V_{i_r}  \,.
%       \]
%       This defines a grading on~$\Tensor(V)$ which makes the inclusion~$V \inclusion \Tensor(V)$ into a homomorphism of graded vector spaces.
%     \item
%       If~$\glie$ is a graded Lie~algebra with grading~$\glie = \bigoplus_{i \neq 0} \glie_i$ then we regard the tensor algebra~$\Tensor(\glie)$ as a graded~{\algebra{$\kf$}} in the above way.
%       Let~$I$ be the two-sided ideal in~$\Tensor(\glie)$ generated by all elements~$c_{x,y} \defined x \tensor y - y \tensor x - [x,y]$ with~$x, y \in \glie$.
%       The ideal~$I$ is already generated by all~$c_{x,y}$ with~$x, y \in \glie$ homogeneous because~$c_{x,y}$ is bilinear in~$x$ and~$y$.
%       The ideal~$I$ is hence graded and so the quotient~$\Univ(\glie) = \Tensor(\glie)/I$ inherits a grading from~$\Tensor(\glie)$.
%       This is the unique grading that makes the canonical homomorphism~$\glie \to \Univ(\glie)$ a homomorphism of graded~{\algebras{$\kf$}}.
%   \end{enumerate}
% \end{remark}





\section{Filtered \texorpdfstring{$\kf$}{k}-Algebras}



\subsection{Definition}

\begin{definition}
  Let~$A$ be a~\algebra{$\kf$}.
  A \defemph{filtration}\index{filtration} of~$A$ is an increasing sequence
  \[
    A_{(0)}
    \subseteq
    A_{(1)}
    \subseteq
    A_{(2)}
    \subseteq
    \dotsb
  \]
  of linear subspaces of~$A$ such that~$A = \bigcup_{p \geq 0} A_{(p)}$ and~$A_{(p)} A_{(q)} \subseteq A_{(p+q)}$ for all~$p, q \geq 0$, as well as~$1 \in A_{(0)}$.
  A \defemph{filtered~\algebra{$\kf$}} is a~\algebra{$\kf$}~$A$ together with a filtration of~$A$.
\end{definition}


\begin{remark}
  \label{filtration conventions}
  \leavevmode
  \begin{enumerate}
    \item
      We often say that~\enquote{$A$ is a filtered algebra} without explicitely mentioning the filtration.
      The parts of the filtration will then be denoted by~$A_{(p)}$.
    \item
      If~$A$ is a filtered~{\algebra{$\kf$}} then we set~$A_{(-p)} \defined 0$ for all~$p < 0$ for convenience.
      The relation~$A_{(p)} A_{(q)} \subseteq A_{(p+q)}$ does then holds for all~$p, q \in \Integer$.
  \end{enumerate}
\end{remark}


\begin{definition}
  Let~$A$ and~$B$ be two filtered~\algebras{$\kf$}.
  \begin{enumerate}
    \item
      A~\defemph{homomorphism of filtered algebras} from~$A$ to~$B$ is a homomorphism of algebras~$\Phi$ from~$A$ to~$B$ such that~$\Phi( A_{(p)} ) \subseteq B_{(p)}$ for all~$p \geq 0$.
    \item
      Let~$\Phi$ be a homomorphism of filtered algebras from~$A$ to~$B$.
      Then the restiction of~$\Phi$ to a linear map from~$A_{(p)}$ to~$B_{(p)}$ is denoted by~$\Phi_{(p)}$ for all~$p \geq 0$.
  \end{enumerate}
\end{definition}


\begin{remark}
  Let~$A$,~$B$ and~$C$ be filtered~\algebras{$\kf$}.
  \begin{enumerate}
    \item
      The identity map~$\id_A$ is a homomorphism of filtered algebras from~$A$ to~$A$.
    \item
      Let~$\Phi$ be a homomorphism of filtered algebras from~$A$ to~$B$ and let~$\Psi$ be a homomorphisms of filtered~\algebras{$\kf$} from~$B$ to~$C$.
      Their composite~$\Psi \circ \Phi$ is a homomorphism of filtered algebras from~$A$ to~$C$.
  \end{enumerate}
  This shows that filtered~\algebras{$\kf$} together with homomorphisms of filtered algebras between them form a category.
  We will denote this category by~$\cfAlg{\kf}$\glsadd{filtered algebras}.
  \begin{enumerate}[resume]
    \item
      A homomorphism of filtered algebras~$\Phi$ from~$A$ to~$B$ is an isomorphism if and only if there exists a homomorphism of filtered algebras~$\Psi$ from~$B$ to~$A$ with~$\Psi \circ \Phi = \id_A$ and~$\Phi \circ \Psi = \id_B$.
  \end{enumerate}
\end{remark}


\begin{warning}
  Let~$A$ and~$B$ be two filtered~{\algebras{$\kf$}}.
  A bijective homomorphism of filtered algebra~$\Phi$ from $A$ to~$B$ is not necessarily an isomorphism of filtered algebras.
  
  Indeed, let~$A$ be some~\algebra{$\kf$} with~$A \neq \kf$.
  We can endow~$A$ with a filtration
  \[
    A_{(0)}
    =
    \kf
    \subseteq
    A
    =
    A
    =
    A
    =
    \dotsb
  \]
  which results in a filtered~{\algebra{$\kf$}}~$B_1$.
  But we can also endow~$A$ with the filtration
  \[
    A_{(0)}
    =
    A
    =
    A
    =
    A
    =
    A
    =
    \dotsb
  \]
  which results in a filtered~{\algebra{$\kf$}}~$B_2$.
  The identity~$\id_A$ is a bijective homomorphism of filtered~algebras from~$B_1$ to~$B_2$.
  But this is not an isomorphism of filtered algebras because its set-theoretic inverse, which is just~$\id_A$ again, does not map the filtration of~$B_2$ into the filtration of~$B_1$.
\end{warning}


\begin{remark}
  \leavevmode
  \begin{enumerate}
    \item
      Let~$A$ be a~\algebra{$\kf$}.
      Any grading~$A = \bigoplus_{p \geq 0} A_p$ of~$A$ results in a filtration~$A = \bigcup_{p \geq 0} A_{(p)}$ of~$A$ given by
      \[
        A_{(p)}
        =
        \bigoplus_{q \leq p} A_q
      \]
      for all~$p \geq 0$.
      We can therefore regard every graded~{\algebra{$\kf$}} as a filtered~{\algebra{$\kf$}}.
      Every homomorphism of graded algebras is then also a hommorphism of filtered algebras.
      This construction gives us therefore a forgetful functor from~$\cgAlg{\kf}$ to~$\cfAlg{\kf}$.
    \item 
      Let~$A$ be a filtered algebra, let~$I$ is a two-sided ideal in~$A$ andl let~$\Pi$ be the canonical quotient homomorphism from~$A$ to~$A/I$.
      The quotient algebra~$A/I$ inherits from~$A$ a filtration given by~$(A/I)_{(p)} \defined \Pi(A_{(p)})$ for all~$p \geq 0$.
  \end{enumerate}
\end{remark}


\begin{examples}
  \label{examples for filtered algebras}
  \leavevmode
  \begin{enumerate}
    \item
      Let~$V$ be a vector space.
      The tensor algebra~$\Tensor(V)$, the symmetric algebra~$\Symm(V)$ and the exterior algebra~$\Exterior(V)$ admit gradings as explained in \cref{examples for graded algebras}.
      They can hence be endowed with the resulting filtrations.
    \item
      Let~$\glie$ be a Lie~algebra.
      The universal enveloping algebra~$\Univ(\glie)$ inherits from the tensor algebra~$\Tensor(\glie)$ a filtration~$\Univ(\glie) = \bigcup_{p \geq 0} \Univ(\glie)_{(p)}$.
      This filtration is given by
      \[
        \Univ(\glie)_{(p)}
        =
        \gen{
          \class{x_1 \dotsm x_q}
        \suchthat
          q \leq p,
          x_1, \dotsc, x_q \in \glie
        }_{\kf}
      \]
      for all~$p \geq 0$.
    \item
      Let~$M$ be a multiplicative monoid and let
      \[
        M_{(0)}
        \subseteq
        M_{(1)}
        \subseteq
        M_{(2)}
        \subseteq
        M_{(3)}
        \subseteq
        \dotsb
      \]
      be a filtration of~$M$, i.e. it holds that~$M = \bigcup_{p \geq 0} M_{(p)}$ with~$M_{(p)} M_{(q)} \subseteq M_{(p+q)}$ for all~$p, q \geq 0$, and~$1 \in M_{(0)}$.
      The monoid algebra~$A \defined \kf[M]$ inherits a filtration given by
      \[
        A_{(p)}
        =
        \gen{ M_{(p)} }_{\kf}
      \]
      for all~$p \geq 0$.
    \item
      Let us give a special case of the previous example.

      Let~$G$ be a group and let~$S$ be a generating set of~$G$.
      The \defemph{length} of an element~$g$ of~$G$ with respect to the generating set~$S$ is given by
      \[
        \ell_S(g)
        \defined
        \min
        \left\{
          n \geq 0
        \suchthat*
          \begin{tabular}{@{}c@{}}
            there exist~$s_1, \dotsc, s_n \in S$ \\
            and~$\varepsilon_1, \dots, \varepsilon_n \in \{1 -1\}$ \\
            with~$g = s_1^{\varepsilon_1} \dotsm s_n^{\varepsilon_n}$
          \end{tabular}
        \right\}  \,.
      \]
      The length function is subadditive in the sense that
      \[
        \ell_S(gh)
        \leq
        \ell_S(g) + \ell_S(h)
      \]
      for all~$g, h \in G$.
      For all~$p \geq 0$ let
      \[
        G_{(p)}
        \defined
        \{
          g \in G
        \suchthat
          \ell_S(g) \leq p
        \}  \,.
      \]
      This is the ball of radius~$p$ with respect to the generating set~$S$.
      These subsets given a filtration of the group~$G$.
      The condition~$G_{(p)} G_{(q)} \subseteq G_{(p + q)}$ for all~$p, q \geq 0$ follows from the subadditivity of the length function~$\ell_S$.
      It follows that the group algebra~$\kf[G]$ inherits a filtration given by
      \[
        \kf[G]_{(p)}
        \defined
        \gen{ G_{(p)} }_{\kf}
        =
        \gen{
          g \in G
        \suchthat
          \ell_S(g) \leq p
        }_{\kf} 
      \]
      for all~$p \geq 0$.
  \end{enumerate}
\end{examples}


\begin{definition}
  Let~$A$ be a filtered algebra.
  The \defemph{degree}\index{degree!filtration} of an element~$x$ of~$A$ is the minimal natural number~$p$ for which the element~$x$ is contained~$A_{(p)}$.
  This degree is denoted by~$\deg(x)$.
\end{definition}


\begin{example}
  \leavevmode
  \begin{enumerate}
    \item
      Let~$A \defined \kf[t]$ be the polynomial ring in one variable.
      The standard grading~$\kf[t] = \bigoplus_{i \geq 0} \kf t^i$ gives the filtration
      \[
        A_{(0)}
        =
        \kf
        =
        \gen{ 1 }_{\kf}
        \subseteq
        \gen{ 1, t }_{\kf}
        \subseteq
        \gen{ 1, t, t^2 }_{\kf}
        \subseteq
        \gen{ 1, t, t^2, t^3 }_{\kf}
        \subseteq
        \dotsb
      \]
      The degree of a nonzero polynomial with respect to this filtration is the usual degree of a polynomial.
    \item
      Let more generall~$A$ be a graded algebra with associated filtration.
      Then the degree of a nonzero element~$x$ of~$A$ with homogeneous decomposition~$x = \sum_{p \geq 0} x_p$ is the maximal natural number~$p$ for which the homogeneous component~$x_p$ is nonzero.
  \end{enumerate}
\end{example}



\subsection{The Associated Graded Algebra}

\subsubsection{Construction}

\begin{definition}
  Let~$A$ be filtered algebra.
  Two elements~$x$ and~$y$ ~$A$ are \defemph{equal up to smaller degree}\index{equal up to smaller degree}\index{up to smaller degree}\index{degree!up to smaller} if either~$x = y$ or~$\deg(x-y) < \deg(x), \deg(y)$.
\end{definition}


\begin{proposition}
  Let~$A$ be a filtered algebra and let~$x$ and~$y$ be two elements of~$A$.
  The elements~$x$ and~$y$ are equal up to smaller degree if and only if they have the same degree~$d$ and their difference~$x - y$ is contained in~$A_{(d-1)}$.
\end{proposition}


\begin{proof}
  Suppose first that that the elements~$x$ and~$y$ are equal up to smaller degree.
  Let~$x$ be of degree~$d$ and let~$y$ be of degree~$d'$ with~$d \geq d'$.
  If~$x = y$ then~$d = d'$ and the difference~$x - y = 0$ is contained in~$A_{(d-1)}$.

  Suppose now in addition that~$x \neq y$.
  Then by assumption~$\deg(x - y) < \deg(y) = d'$.
  It follows that the difference~$x - y$ is contained in~$A_{(d')}$.
  But the element~$y$ is also contained in~$A_{(d')}$.
  It follows that~$x = (x-y) + y$ is contained in~$A_{(d')}$.
  Therefore~$d' \leq d$ by choice of~$d$, and thus~$d = d'$.
  It also follows from the condition~$\deg(x-y) < \deg(x) = d$ that the difference~$x-y$ is contained in~$A_{(d-1)}$.
 
  Suppose now on the other hand that the two elements~$x$ and~$y$ have the same degree~$d$ and that their difference~$x - y$ is contained in~$A_{(d-1)}$.
  If~$d = 0$ then~$A_{(d-1)} = A_{-1} = 0$ and it follws that~$x = y$.
  If~$d > 0$ then~$d-1$ is a natural number and it follows from the condition~$x - y \in A_{(d-1)}$ that~$\deg(x-y) < d = \deg(x), \deg(y)$.
\end{proof}


\begin{corollary}
  Let~$A$ be a filtered~\algebra{$\kf$}.
  The notion of \enquote{being equal up to smaller degree} is an equivalence relation on~$A$.
  The equivalence class of an element~$x$ of~$A$ of degree~$d$ is given by the coset~$x + A_{(d-1)}$.
  \qed
\end{corollary}


\begin{fluff}
  Let~$A$ be a filtered~{\algebra{$\kf$}} and let~$\sim$ be the equivalence relation \enquote{equal up to smaller degree} on~$A$.
  
  When calculating in~$A$ we sometimes want to replace an element~$x \in A$ by another element~$x' \in A$ that is equal to~$x$ up to smaller degree, while hoping that the result ouf our calculation also stays the same up to smaller degree.
  This can be useful if terms of smaller degree are not important in the given situation, or if they can be dealt with by induction.
  
  We would therefore like to have the properties
  \begin{equation}
    \label{wanted compatibility}
    x + y
    \sim
    x' + y' \,,
    \qquad
    x \cdot y
    \sim
    x' \cdot y' \,,
    \qquad
    \lambda x \sim \lambda x'
  \end{equation}
  for all elements~$x$,~$x'$~$y$,~$y'$ of~$A$ with~$x \sim x'$ and~$y \sim y'$ and all scalars~$\lambda$ in~$\kf$.
  This would then mean that the quotient set~$A/{\sim}$ inherits from~$A$ the structure of a~{\algebra{$\kf$}}, which would allow us to do calculations \enquote{up to smaller degree} by switching from~$A$ to~$A/{\sim}$.
  
  But alas the properties~\eqref{wanted compatibility} do not hold in general.%
  \begin{itemize}
    \item
      The equivalence relation~$\sim$ is not necessarily compatible with addition.

      As a counterexample we can take for~$A$ the polynomial algebra~$\kf[t]$ wih the filtration induced by the standard grading~$A = \bigoplus_{d \geq 0} \gen{ t^d }_{\kf}$.
      We can the consider the elements~$x = t$ and~$x' = t+1$ as well as the elements~$y, y' = -t$.
      Then~$x$ and~$x'$ are equal up to smaller degree and~$y$ and~$y'$ are equal, but~$x + y = 0$ and~$x' + y' = 1$ are not equal up to smaller degree.
    \item
      The equivalence relation~$\sim$ also hasn’t to be compatible with multiplication.
      
      We take on the~\algebra{$\kf$}~$\kf[t]$ the standard grading~$\kf[t] = \bigoplus_{d \geq 0} \gen{ t^d }_{\kf}$.
      The ideal~$\ideal{I}$ of~$\kf[t]$ is homogeneous, whence the quotient~$A \defined \kf[t] / \ideal{ t^2 }$ inherits a grading given by
      \[
        A
        =
        \kf
        \oplus
        \kf t
        \oplus
        0
        \oplus
        0
        \oplus
        \dotsb
      \]
      The resulting filtration of~$A$ is given by
      \[
        A_0
        =
        \kf
        \subsetneq
        A
        =
        A
        =
        A
        =
        \dotsb
      \]
      The two elements~$x = t$ and~$x' = t+1$ are equal up to smaller degree but for~$y, y' = t$ the two products~$xy = 0$ and~$x' y' = t$ are not equal up to smaller degree.
    \item
      The equivalence relation~$\sim$ is actually compatible with scalar multiplication.

      Suppose that~$x$ and~$x'$ are two element of~$A$ that are equal up to smaller degree.
      This means that~$x$ and~$x'$ are of the same degree~$d$ and that th difference~$x - x'$ is contained in~$A_{(d-1)}$.
      For~$\lambda = 0$ we have~$\lambda x = 0 = \lambda x'$ and hence~$\lambda x \sim \lambda x'$.
      For~$\lambda \neq 0$ we observe that~$x$ is contained in~$A_{(d')}$ for some~$d' \geq 0$ if and only if~$\lambda x$ is contained in~$A_{(d')}$, and similar for~$x'$.
      We thus find that~$\lambda x$ and~$\lambda x'$ have tha same degrees as~$x$ as~$x'$, and thus the same degree.
      The difference~$\lambda x - \lambda x' = \lambda (x - x')$ is contained in~$A_{(d-1)}$ because~$x - x'$ is contained in~$A_{(d-1)}$.
  \end{itemize}

  We can also argue in a more abstract way why the set~$A / {\sim}$ can in general not inherit the structure of a~\algebra{$\kf$} from~$A$.
  We would otherwise have~$A/{\sim} = A/I$ for a two-sided ideal~$I$ of~$A$, which would be given by~$I = \{x \in A \suchthat x \sim 0\}$.
  But the only element that is equal to~$0$ up to smaller degree is~$0$ itself.
  Hence~$I = 0$ and we would have that~$A/{\sim} = A / I = A / 0 = A$.
  This would mean that~$\sim$ is the trivial equivalence relation.
\end{fluff}


\begin{fluff}
  The following construction gives us a way to circumvent these above problems and to calculate with elements of~$A$ \enquote{up to smaller degree} in a rigorous way.
\end{fluff}


\begin{construction}[The associated graded algebra]
  \label{construction of associated graded}
  To every filtered~{\algebra{$\kf$}}~$A$ we can associate a graded~{\algebra{$\kf$}} as follows.

  For every~$p \geq 0$ let
  \[
    \gr[p](A)
    \defined
    A_{(p)} / A_{(p-1)}
  \]
  where we use the convention~$A_{(-1)} = 0$ from \cref{filtration conventions}.
  We denote the residue class of an element~$x$ of~$A_{(p)}$ in~$\gr[p](A)$ by~$\fclass{x}_p$.
  
  The multiplication of~$A$ restricts for any two natural numbers~$p$,~$q$ to a bilinear map
  \[
    A_{(p)} \times A_{(q)} \to A_{(p+q)} \,,
  \]
  which in turn induces a well-defined bilinear map
  \[
    \mu_{p,q}
    \colon
    \gr[p](A) \times \gr[q](A)
    \to
    \gr[p+q](A) \,,
    \quad
    (\fclass{x}_p, \fclass{y}_q)
    \mapsto
    \fclass{xy}_{p+q}  \,.
  \]
  We will write~$\fclass{x}_p \cdot \fclass{y}_q$ or just~$\fclass{x}_p \fclass{y}_q$ instead of~$\mu_{p,q}(\fclass{x}_p, \fclass{y}_q)$.
  To see that the multipliction map~$\mu_{p,q}$ is well-defined let~$x$,~$x'$ be elements of~$A_{(p)}$ and let~$y$,~$y'$ be elements of~$A_{(q)}$ such that~$x - x ' \in A_{(p-1)}$ and~$y - y' \in A_{(q-1)}$.
  Then
  \begin{align*}
    x y
    &=
    (x' + (x - x')) (y' + (y - y'))
    \\
    &=
    x' y' + (x - x') y' + x' (y - y') + (x - x')(y - y')
    \\
    &\in
    x' y' + A_{(p-1)} A_{(q)} + A_{(p)} A_{(q-1)} + A_{(p-1)} A_{(q-1)}
    \\
    &\subseteq
    x' y' + A_{(p+q-1)} + A_{(p+q-1)} + A_{(p+q-2)}
    \\
    &\subseteq
    x' y' + A_{(p+q-1)} \,,
  \end{align*}
  and hence~$\fclass{xy}_{p+q} = \fclass{x'y'}_{p+q}$.
  
  The element~$[1]_0$ of~$\gr[0](A)$ satisfies the property~$[1]_0 \cdot \fclass{x}_p = \fclass{x}_p$ for all~$p \geq 0$,~$\fclass{x}_p \in \gr[p](A)$, and the multiplications~$\mu_{p,q}$ are relatively associative in the sense that
  \[
    \fclass{x}_p \cdot (\fclass{y}_q \cdot [z]_r)
    =
    [xyz]_{p+q+r}
    =
    (\fclass{x}_p \cdot \fclass{y}_q) \cdot [z]_r
  \]
  for all~$p, q, r \geq 0$,~$\fclass{x}_p \in \gr[p](A)$,~$\fclass{y}_q \in \gr[q](A)$ and~$[z]_r \in \gr[r](A)$.
  It follows from \cref{external description of graded algebras} that on the direct sum
  \[
    \gr(A)
    \defined
    \bigoplus_{p \geq 0} \gr[p](A)
  \]
  the partial multiplications~$\mu_{p,q}$ assemble into a single multiplication
  \[
    \mu
    \colon
    \gr(A) \times \gr(A)
    \to
    \gr(A)
  \]
  which makes~$\gr(A)$ into a graded~{\algebra{$\kf$}}.
  The multiplicative neutral element of~$\gr(A)$ is given by~$[1]_0$.
\end{construction}


\begin{definition}
  For a filtered~{\algebra{$\kf$}}~$A$ the graded algebra~\gls*{associated graded} resulting from~$A$ by \cref{construction of associated graded} is the \defemph{associated graded algebra}\index{associated graded algebra} of~$A$.
\end{definition}


\begin{fluff}
  Let~$A$ be a filtered algebra.
  \begin{enumerate}
    \item
      To every element~$x$ of~$A$ we can associate an element in~$\gr(A)$ as follows.

      We observe that if~$x$ is of degree~$d$ then the residue class~$\fclass{x}_p$ is not defined for~$p < d$ because~$x$ is not contained in~$A_{(q)}$, whereas this residue class is defined for all~$p \geq d$.
      But for~$p > d$ we find that~$\fclass{x}_p = 0$ because the element~$x_p$ is contained in~$A_{(p-1)}$.
      The only interesting residue class associated to~$x$ is therefore~$\fclass{x}_p$.
      We also note that this residue class~$\fclass{x}_p$ is nonzero if~$x$ is nonzero, because then~$x$ is not contained in~$A_{(p-1)}$.
    \item
      We observe that two elements~$x$ and~$y$ of~$A$ give the same associated element in~$\gr(A)$ if and only if~$x$ and~$y$ are equal up to smaller degree.
      To show this we denote by~$\gamma$ the set-theoretic function from~$A$ to~$\gr(A)$ that maps ever every element~$x$ of~$A$ to the associated element of~$\gr(A)$.
      
      Suppose first that the elements~$x$ and~$y$ are equal up to smaller degree.
      Then~$x$ and~$y$ are of the same degree~$d$ and the difference~$x - y$ is containted in~$A_{(d-1)}$.
      Hence
      \[
        \gamma(x) = \fclass{x}_d = \fclass{y}_d = \gamma(y) \,.
      \]
      
      Suppose now that~$\gamma(x) = \gamma(y)$.
      If the elements~$x$ and~$y$ had different degrees then~$\gamma(x)$ and~$\gamma(y)$ would be contained in different homogeneous parts of~$\gr(A)$.
      This would contradict the assumption~$\gamma(x) = \gamma(y)$.
      The elements~$x$ and~$y$ are thus of the same degree~$d$.
      In now further follows from the equality
      \[
        \fclass{x}_d
        =
        \gamma(x)
        =
        \gamma(y)
        =
        \fclass{y}_d
      \]
      that~$\fclass{x - y}_d = 0$ and thus~$x - y \in A_{(d-1)}$.
      This shows that~$x$ and~$y$ are equal up to smaller degree.
    \item
      We observe that the image of the map~$\gamma$ consists precisely of the homogeneous elements of~$\gr(A)$.
      If an element~$x$ of~$A$ has degree~$d$ then the associated element~$\fclass{x}_d$ is homogeneous of degree~$d$.
  \end{enumerate}
\end{fluff}


\begin{warning}
  \label{generators of associated graded}
  Let~$A$ be a filtered~{\algebra{$\kf$}}
  \begin{enumerate}
    \item
      The map~$\gamma$ from~$A$ to~$\gr(A)$ which assigns to each element~$x$ of~$A$ the corresponding element of~$\gr(A)$ is very much not a homomorphism.
      It is in general neither multiplicative nor additive.
    \item
      \label{generators of associated graded part}
      Suppose that~$x_i$ with~$i$ in~$I$ is a generating set for the algebra~$A$.
      Then the associated elements~$\gamma(x_i)$ of~$\gr(A)$ do not have to form a generating set for~$\gr(A)$.
      We will give an explicit counterexample in~\cref{converse to warning about generating set for the associated graded}.
  \end{enumerate}
\end{warning}


\begin{remark}
  Let~$A$ be a filtered algebra with finite algebra generating set~$x_1, \dotsc, x_n$ such that each generator~$x_i$ is of degree~$d_i$.
  Then the associated elements in~$\gr(A)$ form a set of algebra generators of~$\gr(A)$ if and only if the generators~$x_1, \dotsc, x_n$ are compatible with the filtration of~$A$ in the sense that for every~$p \geq 0$ the linear subspace~$A_{(p)}$ of~$A$ is spanned by all those monomials~$x_{i_1} \dotsm x_{i_r}$ with~$r \geq 0$,~$d_{i_r} + \dotsb + d_{i_r} \leq p$.
  We refer to \cite{associated_generated} for more details on this.
\end{remark}

\subsubsection{Functoriality}

\begin{construction}
  Let~$A$ and~$B$ be two filtered~\algebra{$\kf$}.
  
  Let~$\Phi$ be a homomorphism of filtered algebras from~$A$ to~$B$.
  Then for every natural number~$p$ the restriction~$\Phi_{(p)}$ induces a linear map
  \[
    \gr[p](\Phi)
    \colon
    \gr[p](A)
    \to
    \gr[p](B) \,,
    \quad
    \fclass{x}_p
    \mapsto
    [\Phi(x)]_p  \,.
  \]
  These linear maps come together to form a linear map
  \[
    \gr(\Phi)
    \colon
    \gr(A)
    \to
    \gr(B)  \,.
  \]
  This map is already homomorphism of algebras because
  \begin{align*}
    \gr(\Phi)( \fclass{x}_p \cdot \fclass{y}_q )
    &=
    \gr(\Phi)( \fclass{xy}_{p+q} )
    \\
    &=
    [ \Phi(xy) ]_{p+q}
    \\
    &=
    [ \Phi(x) \cdot \Phi(y) ]_{p+q}
    \\
    &=
    [ \Phi(x) ]_p \cdot [ \Phi(y) ]_q
    \\
    &=
    \gr(\Phi)(\fclass{x}_p) \cdot \gr(\Phi)(\fclass{y}_q) \,.
  \end{align*}
  for all~$p, q \geq 0$,~$\fclass{x}_p \in \gr[p](A)$,~$\fclass{y}_q \in \gr[q](A)$, and
  \[
    \gr(\Phi)([1]_0)
    =
    [ \Phi(1) ]_0
    =
    [1]_0 \,.
  \]
  The map~$\gr(\Phi)$ is thus a homomorphism of graded algebras.

  It holds for every filtered~\algebra{$\kf$}~$A$ that~$\gr(\id_A) = \id_{\gr(A)}$.
  It also holds for every homomorphism of filtered algebras~$\Phi$ from~$A$ to~$B$ and every homomorphism of filtered algebras~$\Psi$ from~$B$ to~$C$ that~$\gr(\Psi \circ \Phi) = \gr(\Psi) \circ \gr(\Phi)$.
  
  We have hence constructed a functor~$\gr$ from~$\cfAlg{\kf}$ to~$\cgAlg{\kf}$.
\end{construction}


\begin{proposition}
  Let~$A$ be a graded~{\algebra{$\kf$}}.
  Then the linear map~$\Phi$ from~$A$ to~$\gr(A)$ given by~$\Phi(x) = \fclass{x}_p$ for all~$p \geq 0$,~$x \in A_p$ is an isomorphism of graded~algebras.
\end{proposition}


\begin{proof}
  We have for every natural number~$p$ that~$A_{(p)} = \bigoplus_{q=0}^p A_q$ and hence
  \[
    \gr[p](A)
    =
    A_{(p)} / A_{(p-1)}
    \cong
    A_p
  \]
  as vector spaces.
  It follows that~$\Phi$ is an isomorphism of vector spaces.
  We have for all natural numbers~$p$,~$q$ and homogeneous elements~$x$ in~$A_p$ and~$y$ in~$A_q$ that
  \[
    \Phi(x) \cdot \Phi(y)
    =
    \fclass{x}_p \cdot \fclass{y}_q
    =
    \fclass{xy}_{p+q}
    =
    \Phi(xy) \,,
  \]
  where the last equality holds because the element~$xy$ is homogeneous of degree~$p + q$.
  This shows that~$\Phi$ is multiplicative and hence already an isomorphism of graded algebras.
\end{proof}


\begin{warning}
  If~$A$ is a filtered~{\algebra{$\kf$}} for which the filtration does not come from a grading then~$A$ and~$\gr(A)$ are not isomorphic as filtered algebras (where the filtration of~$\gr(A)$ is induced by its grading).
\end{warning}

\subsubsection{Zero Divisors}

\begin{definition}
  Let~$A$ be a ring.
  \begin{enumerate}
    \item
      An element~$x$ of~$A$ is a \defemph{left zero divisor}\index{zero divisor!left} if there exists a nonzero element~$y$ of~$A$ with~$xy = 0$.
    \item
      An element~$x$ of~$A$ is a \defemph{right zero divisor}\index{zero divisor!right} if there exists a nonzero element~$y$ of~$A$ with~$yx = 0$.
    \item
      An element~$x$ of~$A$ is a \defemph{zero divisor}\index{zero divisor} if it is a left zero divisor or a right zero divisor.
  \end{enumerate}
\end{definition}


\begin{remark}
  Let~$A$ be a ring.
  \begin{enumerate}
    \item
      An element~$x$ of~$A$ is a left zero divisor in~$A$ if and only if the left multiplication map
      \[
        A \to A \,,
        \quad
        y \mapsto xy
      \]
      is not injective.
      The element~$x$ is a right zero divisor in~$A$ if and only if the right multiplication map
      \[
        A \to A \,,
        \quad
        y \mapsto yx
      \]
      is not injective.
    \item
      The zero element of~$A$ is both a left zero divisor and right zero divisor if~$A$ is nonzero.
      If~$A$ is the zero algebra then the zero element is not a zero divisor.
  \end{enumerate}
\end{remark}


\begin{proposition}
  \label{associated graded algebra and zero divisors}
  Let~$A$ be a filtered~{\algebra{$\kf$}}.
  \begin{enumerate}
    \item
      \label{associated graded has left zero divisors}
      If~$\gr(A)$ doesn’t contain any nonzero left zero divisor then neither does~$A$.
    \item
      \label{associated graded has right zero divisors}
      If~$\gr(A)$ doesn’t contain any nonzero right zero divisor then neither does~$A$.
    \item
      If~$\gr(A)$ doesn’t contain any nonzero zero divisor then neither does~$A$.
  \end{enumerate}
\end{proposition}


\begin{proof}
  \leavevmode
  \begin{enumerate}
    \item
      Suppose that the algebra~$A$ contains a nonzero left zero divisor~$x$.
      Then there exists some nonzero element~$y$ of~$A$ with~$xy = 0$.
      If the element~$x$ is of degree~$p$ and the element~$y$ is of degree~$q$ then both~$\fclass{x}_p$ and~$\fclass{y}_q$ are nonzero.
      But
      \[
        \fclass{x}_p \cdot \fclass{y}_q
        =
        \fclass{xy}_{p+q}
        =
        [0]_{p+q}
        =
        0 \,.
      \]
      This shows that the element~$\fclass{x}_p$ is again a nonzero left zero divisor.
    \item
      This can be shown in the same way as part~\ref*{associated graded has left zero divisors}.
    \item
      This is a combination of parts~\ref*{associated graded has left zero divisors} and~\ref*{associated graded has right zero divisors}.
    \qedhere
  \end{enumerate}
\end{proof}


\begin{remark}
  The converse of \cref{associated graded algebra and zero divisors} does not hold.
  More explicitely, it may happen that the algebra~$\gr(A)$ has nonzero zero divisors even though~$A$ doesn’t have any.
  To see this let~$A$ be any~\algebra{$\kf$} with~$\kf \subsetneq A$ and consider the filtration
  \[
    A_{(0)}
    =
    \kf
    \subsetneq
    A
    \subseteq
    A
    \subseteq
    A
    \subseteq
    \dotsb
  \]
  Then~$\gr[0](A) = \kf$, the homogeneous part~$\gr[1](A) = A / {\kf}$ is nonzero, but all homogeneous parts~$\gr[p](A)$ with~$p \geq 2$ vanish.
  Thus~$\gr[1](A) \cdot \gr[1](A) = 0$ even though~$\gr[1](A)$ is nonzero.
  This means that every elements of~$\gr[1](A)$ is both a left zero divisor and right zero divisor.
\end{remark}

\subsubsection{Ideals and Chain Conditions}

\begin{proposition}
  \label{associated graded ideals}
  Let~$A$ be a filtered algebra. 
  \begin{enumerate}
    \item
      Let~$I$ be some kind of ideal of~$A$.
      For every natural number~$p$ let
      \[
        \gr[p](I)
        \defined
        \bigl( A_{(p-1)} + I \cap A_{(p)} \bigr) / A_{(p-1)} \,,
      \]
      and let~$\gr[p](I) \defined \bigoplus_{p \geq 0} \gr[p](I)$.%
      \footnote{The term~$(A_{(p-1)} + I) \cap A_{(p)} = A_{(p-1)} + (I \cap A_{(p)})$ does not depend on the choice of parenthesization because the lattice of subspaces is modular.}
      This linear subspace~$\gr(I)$ of~$\gr(A)$ is a homogeneous ideal of~$A$, of the same kind as the original ideal~$I$.
    \item
      \label{pulling back generating set from graded ideal}
      Let~$S$ be a subset of~$I$.
      If the associated elements~$[s]_{\deg(s)}$ with~$s$ in~$S$ generate the homogeneous ideal~$\gr(I)$ then~$S$ generates the original ideal~$I$.
    \item
      The mapping
      \begin{align*}
        \{ \text{ideals in~$A$} \}
        &\to
        \{ \text{homogeneous ideals in~$\gr(A)$} \}  \,,
        \\
        I
        &\mapsto
        \gr(I)
      \end{align*}
      is strictly order preserving, i.e.\ if~$I$ and~$J$ are ideals of~$A$ such that~$I$ is strictly contained in~$J$ then~$\gr(I)$ is also strictly contained in~$\gr(J)$.
  \end{enumerate}
\end{proposition}


\begin{proof}
  \leavevmode
  \begin{enumerate}
    \item
      The graded components of~$\gr(A)$ can be described as
      \begin{align*}
        \gr[p](A)
        &=
        \bigl( A_{(p-1)} + I \cap A_{(p)} \bigr) / A_{(p-1)}
        \\
        &=
        \{
          y + A_{(p-1)}
        \suchthat
          y \in A_{(p-1)} + I \cap A_{(p)}
        \}
        \\
        &=
        \{
          y + A_{(p-1)}
        \suchthat
          y \in I \cap A_{(p)}
        \}
        \\
        &=
        \{
          \fclass{ y }_{p}
        \suchthat
          y \in I \cap A_{(p)}
        \}
      \end{align*}
      for all~$p \geq 0$.

      Suppose now that~$I$ is a left ideal of~$A$.
      It holds for every element~$x$ of~$A_{(p)}$ and every element~$y$ of~$I \cap A_{(q)}$ that
      \[
        xy
        \in
        A_{(p)} (I \cap A_{(q)})
        \subseteq
        A_{(p)} I \cap A_{(p)} A_{(q)}
        \subseteq
        I \cap A_{(p+q)} \,,
      \]
      whence the product~$\fclass{x}_p \cdot \fclass{y}_q = [xy]_{p+q}$ is contained in~$\gr[p+q](I)$.
      Together with the above description of~$\gr[p](A)$ this shows that~$\gr(I)$ is a left ideal of~$\gr(A)$.

      We can show in the same way that~$\gr(I)$ is a right ideal of~$\gr(A)$ if~$I$ is a right ideal of~$A$.
      By combining these two cases it follows that~$\gr(I)$ is a two-sided ideal of~$\gr(A)$ if~$I$ is a two-sided ideal of~$A$.
    \item
      Let~$x$ be an element of~$I$.
      If~$x = 0$ then the assertion holds.

      In general the element~$\fclass{x}_{\deg(x)}$ is contained in~$\gr(I)$ and is therefore of the form
      \[
        \fclass{x}_{\deg(x)}
        =
        \sum_{s \in S} b_s \cdot [s]_{\deg(s)}
      \]
      for some coefficients~$b_s$ in~$\gr(A)$.
      We can now replace each coefficient~$b_s$ by its homogeneous component of degree~$\deg(x) - \deg(s)$ to assume that~$b_s$ is homogeneous.
      This means that~$b_s$ is contained in~$\gr[\deg(x)-\deg(s)](A)$ and thus of the form~$b_s = [a_s]_{\deg(x)-\deg(s)}$ for some element~$a_s$ of~$A_{(\deg(x)-\deg(s))}$.
      We have in particular~$a_s = 0$ whenever~$\deg(s) > \deg(x)$.
      It now follows that
      \begin{align*}
        \fclass{x}_{\deg(x)}
        &=
        \sum_{s \in S} b_s \cdot [s]_{\deg(s)}
        \\
        &=
        \sum_{s \in S} [a_s]_{\deg(x) - \deg(s)} \cdot [s]_{\deg(s)}
        \\
        &=
        \sum_{s \in S} [a_s s]_{\deg(x)}
        \\
        &=
        \left[ \sum_{s \in S} a_s s \right]_{\deg(x)} \,.
      \end{align*}
      This means that the difference~$x - \sum_{s \in S} a_s s$ is contained in~$A_{(\deg(x)-1)}$.
      We can now procced by induction to express this difference as a linear combination of~$S$ with coefficients in~$A$.
    \item
      Let~$I$ and~$J$ be two ideal of~$A$ such that~$I$ is contained in~$J$.
      Then~$\gr[p](I)$ is contained in~$\gr[p](J)$ for all~$p \geq 0$, and hence~$\gr(I)$ is contained in~$\gr(J)$.
      Suppose now that~$\gr(I)$ equals~$\gr(J)$.
      We need to show that~$I$ already equals~$J$.

      The ideal~$\gr(I)$ is homogeneous and thus generated by homogeneous elements.
      There hence exists some subset~$S$ of~$I$ such that~$\gr(I)$ is generated by the elements~$[s]_{\deg(s)}$ with~$s$ in~$S$.
      (Here we use that the homogeneous elements of~$\gr(I)$ are precisely those elements of the form~$\fclass{x}_{\deg(x)}$ with~$x$ in~$I$.)
      It follows from the equality~$\gr(I) = \gr(J)$ and part~\ref*{pulling back generating set from graded ideal} that the set~$S$ is a generating set of both the ideals~$I$ and~$J$.
      Thus~$I = J$.
    \qedhere
  \end{enumerate}
\end{proof}


\begin{corollary}
  \label{universal enveloping reflects chain conditions}
  Let~$A$ be a fieltered~\algebra{$\kf$}.
  If the associated graded algebra~$\gr(A)$ is left noetherian, right noetherian, left artinian or right artinian then the same property holds for~$A$.
\end{corollary}


\begin{proof}
  Every strictly increasing sequence of left ideals in~$A$ results by \cref{associated graded ideals} in a strictly increasing sequence of (homogeneous) left ideals in~$\gr(A)$.
  So if~$A$ is not left noetherian then neither is~$\gr(A)$.
  The other assertions can be shown in the same way.
\end{proof}


\begin{remark}
  \leavevmode
  \begin{enumerate}
    \item
      The idea of \cref{associated graded ideals} is taken from \cite[6.7,~6.9]{noncommutative_noetherian}.
    \item
      Let~$A$ be a filtered~\algebra{$\kf$}.
      If~$a$ is an element of~$A$ then one may think about the associated element~$[a]_{\deg(a)}$ of~$\gr(A)$ as the \enquote{leading term of~$a$}.
      We note that if~$A$ is already a graded~{\algebra{$\kf$}} then under the identification of~$\gr(A)$ with~$A$ the element~$[a]_{\deg(a)}$ is precisely the leading term of~$a$ in the usual sense.
      
      In view of this interpretation of~$[a]_{\deg(a)}$ one might compare the proof of part~\ref{pulling back generating set from graded ideal} of \cref{associated graded ideals} to the proof of Hilbert’s basis theorem.
      One may also compare the associated graded ideal~$\gr(I)$ to the \enquote{ideal of leading coefficients} as considered in the theory of Gröbner bases.
    \item
      Let~$A$ be a filtered~\algebra{$\kf$} and let~$I$ be an ideal of~$A$.
      Then~$I$ inherits from~$A$ a filtration given by~$I_{(p)} = I \cap A_{(p)}$ for all~$p \geq 0$.
      Then
      \begin{align*}
        \gr[p](I)
        &=
        \bigl(A_{(p-1)} + I \cap A_{(p)} \bigr) / A_{(p-1)}
        \\
        &\cong
        \bigl( I \cap A_{(p)} \bigr) / \bigl( A_{(p-1)} \cap I \cap A_{(p)} \bigr)
        \\
        &=
        \bigl( I \cap A_{(p)} \bigr) / \bigl( I \cap A_{(p-1)} \bigr)
        \\
        &=
        I_{(p)} / I_{(p-1)}
      \end{align*}
      by the second isomorphism theorem.
      This justifies the notation~$\gr[p](I)$.
  \end{enumerate}
\end{remark}

\subsubsection{Bases}

\begin{proposition}
  \label{checking basis via associated graded}
  Let~$A$ be filtered algebra and let~$s_i$ with~$i$ in~$I$ be elements of~$A$.
  Suppose that~$s_i$ is contained in~$A_{(d_i)}$ with~$d_i \geq 0$ for all~$i \in I$.
  If the elements~$\fclass{ s_i }_{d_i}$ form a vector space basis of~$\gr(A)$ then the elements~$s_i$ form a basis of~$A$.
\end{proposition}


\begin{proof}
  We first show that the elements~$s_i$ span the algebra~$A$ as a vector space.
  Let~$x$ be an element of~$A$.
  This element is contained in~$A_{(d)}$ for some~$d \geq -1$.
  We show that~$x$ is a linear combination of the elements~$s_i$ by induction over~$d$.
  The assertion holds for~$d = -1$ because then~$x = 0$.

  For the induction step we write the residue class~$\fclass{ x }_d$ as a linear combination
  \[
    \fclass{ x }_d
    =
    \sum_{i \in I}
    \lambda_i \fclass{ s_i }_{d_i} \,.
  \]
  It folllows by comparing degrees that
  \[
    \fclass{ x }_d
    =
    \sum_{i \in I, d_i = d}
    \lambda_i \fclass{ s_i }_d
    =
    \fclass*{
      \sum_{i \in I, d_i = d}
      \lambda_i s_i
    }_d \,.
  \]
  This shows that the difference~$x - \sum_{i \in I, d_i = d} \lambda_i s_i$ is contained in~$A_{(d-1)}$.
  It follows from the induction hypothesis that this difference can be written as the linear combination of the elements~$s_i$.
  It follows that the element~$x$ can be written as such a linear combination.

  We now show that the elements~$s_i$ are linearly independent.
  We suppose that there exists a linear combination
  \[
    0
    =
    \lambda_1 s_{i_1} + \dotsb + \lambda_n s_{i_n}
  \]
  whose coefficients~$\lambda_1, \dotsc, \lambda_n$ are nonzero, with~$n \geq 1$.
  We may assume that~$d_{i_1} \leq \dotsb \leq d_{i_n}$.
  Let~$d \defined d_{i_n}$ be the largest occuring degree and let~$j$ be the minimal index with~$d_{i_j} = d$.
  Then~$\fclass{ s_{i_k} }_d = 0$ for all~$k = 0, \dotsc, j-1$ and~$d_{i_j}, \dotsc, d_{i_n} = d$.
  It follows that the elements~$\fclass{ s_{i_j} }_d, \dotsc, \fclass{ s_{i_n} }_d$ are linearly independent by assumption, and that
  \[
    0
    =
    \lambda_j \fclass{ s_{i_j} }_d + \dotsb + \lambda_n \fclass{ s_{i_n} }_d \,.
  \]
  Thus~$\lambda_j = \dotsc = \lambda_n$.
  But this contradicts the assumption that these coefficients are nonzero.
\end{proof}






\section{Concrete Version of the Poincar\'{e}--Birkhoff--Witt Theorem}


\begin{convention}
  For this section we fix a Lie algebra~$\glie$ with basis~$(x_i)_{i \in I}$.
  For better readability we will denote for every~$x \in \glie$ the corresponding element~$\class{x} \in \Univ(\glie)$ simply by~$x$ again.
\end{convention}


\begin{remark}
  As a consequence of the upcoming PBW~theorem we will see in \cref{embedding into uea} that for any Lie~algebra~$\glie$ the canonical homomorphism~$\glie \to \Univ(\glie)$ is actually injective, allowing us to identify~$\glie$ with a Lie~subalgebra~$\Univ(\glie)$.
  This will a posteriori justify our abuse of notation.
\end{remark}


\begin{fluff}
  Let~$A \defined \kf\gen{X,Y}/(YX - XY - 1)$ be the Weyl~algebra with the filtration induced from the standard grading of~$\kf\gen{X,Y}$ and let~$x, y \in A$ denote the residue classes of~$X$ and~$Y$.
  We have seen in \cref{weyl algebra} that~$A$ has a basis given by all monomials~$x^n y^m$ with~$n, m \geq 0$ and that the associated graded algebra~$\gr(A)$ does therefore have a basis given by all residue classes~$[x^n y^m]_i$ with~$n, m \geq 0$ and~$i = n+m$.
  Moreover, the multiplication of~$\gr(A)$ is on these basis elements given by~$[x^n y^m]_i \cdot [x^k y^l]_j = [x^{n+k} y^{m+l}]_{i+j}$ for all~$n, m, k, l \geq 0$ and~$i = n+m$,~$j = k+l$. 
  
  The key observation behind these results was the relation~$yx = xy + 1$ in~$A$, which tells us that the elements~$x$ and~$y$ commute up to terms of smaller degre, and hence actually commute in~$\gr(A)$ (in the sense that the associated elements~$[x]_1, [y]_1 \in \gr(A)$ commute).
  For the universal enveloping algebra~$\Univ(\glie)$ a similar situation occurs:
  For any two elements~$x, y \in \glie$ the relation~$yx = xy + [x,y]$ holds in~$\Univ(\glie)$.
  Hence the elements~$x$ and~$y$ ought to commute up to a term of smaller degree in~$\Univ(\glie)$.
  We will therefore try to generalize the above results from~$A$ to~$\Univ(\glie)$ to describe a vector space basis of~$\Univ(\glie)$.
  
  We already know that~$\Univ(\glie)$ is generated by the image of~$\glie$ under the canonical map~$\glie \to \Univ(\glie)$.
  Hence~$\Univ(\glie)$ is spanned by the elements~$x_i$ with~$i \in I$.
  This means that the momomials~$x_{i_1} \dotsm x_{i_n}$ with~$n \geq 0$ and~$i_1, \dotsc, i_n \in I$ span~$\Univ(\glie)$ as a vector space.
  By the above discussion we should hopefully be able to rearrange the terms~$x_{i_j}$ in these monomials without destroying the property of being a vector space generating set.
  We will see in \cref{pbw concrete generating part} that this is indeed the case.
  
  We will overall show in \cref{pbw concrete} that if we fix an ordering of the basis elements~$x_i$ then the collection of all ordered monomials will form a basis of the universal enveloping algebra~$\Univ(\glie)$.
  We will also see un \cref{pbw abstract} that this is equivalent to~$\gr(\Univ(\glie))$ being isomorphic (as graded algebras) to the symmetric algebra~$\Symm(\glie)$ and hence a polynomial algebra~$\kf[t_i \suchthat i \in I]$.
\end{fluff}


\begin{lemma}
  \label{rearranging lemma}
  There exists for all~$y_1, \dotsc, y_n \in \glie$ and every permutation~$\sigma \in S_n$ some error term~$r \in \Univ(\glie)_{(n-1)}$ with
  \[
    y_1 \dotsm y_n
    =
    y_{\sigma(1)} \dotsm y_{\sigma(n)}
    +
    r  \,.
  \]
\end{lemma}


\begin{proof}
  It sufficies to prove the statement for~$\sigma$ being a simple transposition because these generate the symmetric group~$S_n$.
  So let~$\sigma = (i, i+1)$.
  Then
  \[
    y_i y_{i+1}
    =
    y_{i+1} y_i + [y_i, y_{i+1}]
  \]
  and hence
  \begin{align*}
    y_1 \dotsm y_n
    =
    y_1 \dotsm y_i y_{i+1} \dotsm y_n
    =
    (y_1 \dotsm y_{i+1} y_i \dotsm y_n)
    +
    (y_1 \dotsm [y_i, y_{i+1}] \dotsm y_n)
  \end{align*}
  with~$y_1 \dotsm [y_i, y_{i+1}] \dotsm y_n \in \Univ(\glie)_{(n-1)}$.
\end{proof}


\begin{convention}
  We suppose that~$(I, \leq)$ is a totally ordered set.
  For every~$n \geq 0$ we set
  \[
    I^n
    \defined
    \{
      (i_1, \dotsc, i_n)
    \suchthat
      \text{$i_1, \dotsc, i_n \in I$ with~$i_1 \leq \dotsb \leq i_n$}
    \}
  \]
  as well as~$I^{(n)} \defined \bigcup_{m=0}^n I^m$ and overall~$I^* \defined \bigcup_{n \geq 0} I^n$.
  For every tupel~$\alpha \in I^*$ with~$\alpha = (i_1, \dotsc, i_n)$ we define the associated \defemph{ordered monomial}~$\gls*{ordered monomial} \in \Univ(\glie)$\index{ordered monomial} as
  \[
    x_\alpha
    \defined
    x_{i_1} \dotsm x_{i_n}  \,.
  \]
  
  For~$\alpha, \beta \in I^*$ we denote by~$\alpha \cdot \beta \in I^*$ the tupel that results from the concatenation of~$\alpha$ and~$\beta$ by reordering of the entries.
  More explicitely, if~$\alpha = (i_1, \dotsc, i_n) \in I^n$ and~$\beta = (j_1, \dotsc, j_m) \in I^m$ then by reordering the entries of the concatenation~$(i_1, \dotsc, i_n, j_1, \dotsc, j_m)$ into increasing order we arrive at~$\alpha \cdot \beta \in I^{n+m}$.
  For~$i \in I$ and~$\alpha \in I^*$ we define~$i \cdot \alpha \in I^*$ by identifying~$I$ with~$I^1$.
  More explicitely, if~$\alpha = (i_1, \dotsc, i_n)$ with~$i_1 \leq \dotsb \leq i_k \leq i \leq i_{k+1} \leq \dotsb \leq i_n$ then~$i \cdot \alpha = (i_1, \dotsc, i_k, i, i_{k+1}, \dotsc, i_n)$.
  
  If~$i \in I$ and~$\alpha \in I^*$ with~$\alpha = (i_1, \dotsc, i_n)$ then we write~$i \leq \alpha$ to mean that~$i \leq i_1$.
  For~$\alpha = ()$ we have~$i \leq \alpha$ for every~$i \in I$ by definition.
\end{convention}


\begin{theorem}[Poincar\'{e}--Birkhoff--Witt (concrete version)]
  \label{pbw concrete}
  The elements~$x_\alpha$ with~$\alpha \in I^*$ form a~{\basis{$\kf$}} of~$\Univ(\glie)$.%
  \footnote{This includes that they are pairwise different.}
\end{theorem}


\begin{remark}
  The proposed basis~$x_\alpha$ with~$\alpha \in I^*$ may be written as
  \[
    x_{i_1}^{p_1} \dotsm x_{i_n}^{p_n}
    \qquad
    \text{with~$n \geq 0$,~$i_1, \dotsc, i_n \in I$,~$i_1 < \dotsb < i_n$,~$p_1, \dotsc, p_n \geq 1$} \,,
  \]
  which explains why we call them \emph{ordered monomials}.
\end{remark}


\begin{example}
  \leavevmode
  \begin{enumerate}
    \item
      If~$\glie$ is a finite-dimensional Lie algebra with basis~$x_1, \dotsc, x_n$ then the algebra~$\Univ(\glie)$ has the ordered monomials~$x_1^{p_1} \dotsm x_n^{p_n}$ with~$p_1, \dotsc, p_n \geq 0$ as a basis.
    \item
      A basis of~$\Univ(\sllie_2(\kf))$ is given by the ordered monomials~$e^n h^m f^k$ with~$n, m, k \geq 0$.
  \end{enumerate}
\end{example}


\begin{example}
  Once the PBW~theorem is proven we can give a counterexample to part~\ref*{generators of associated graded part} of \cref{generators of associated graded}:
  Let~$\glie = \sllie_2(\kf)$ and let~$A \defined \Univ(\glie)$ with the standard filtration induced from the standard grading of the tensor algebra~$\Tensor(\glie)$.
  We already know that~$\Univ(\glie)$ is generated as a~{\algebra{$\kf$}} by the elements~$e$,~$h$~and~$f$.
  It follows from the relation~$[e,f] = h$ that~$A$ is already generated by the two elements~$e$~and~$f$.
  But the associated graded algebra~$\gr(A)$ is not generated by the associated elements~$[e]_1$ and~$[f]_1$ because they only result is the basis elements~$e^n f^k$ with~$n, k \geq 0$.
  We are hence missing all basis elements~$e^n h^m f^k$ with~$n, k \geq 0$ and~$m \geq 1$.
\end{example}


\begin{lemma}
  \label{pbw concrete generating part filtered part}
  The ordered monomials~$x_\alpha$ with~$\alpha \in I^{(n)}$ span the vector space~$\Univ(\glie)_{(n)}$.
\end{lemma}


\begin{proof}
  We show the \lcnamecref{pbw concrete generating part filtered part} by induction over~$n$.
  For~$n = 0$ we have
  \[
    \Univ(\glie)_{(0)}
    =
    \kf
    =
    \gen{ 1 }_{\kf}
    =
    \gen{ x_{()} }_{\kf}
    =
    \gen{ x_\alpha \suchthat \alpha \in I^{(0)} }_{\kf}
  \]
  where~$()$ denotes the empty tupel, the only element of~$I^{(0)} = I^{0}$.
  
  Suppose now that the statement holds for some~$n \geq 0$.
  The homogeneous part~$\gr(\Univ(\glie))_{(n+1)}$ is spanned by all monomials~$x_{i_1} \dotsm x_{i_{n+1}}$ with~$i_1, \dotsc, i_{n+1} \in I$.
  It hence sufficies to show that every such monomials can be expressed via the proposed generators~$x_\alpha$ with~$\alpha \in I^{(n+1)}$.
  We know from \cref{rearranging lemma} that
  \[
    x_{i_1} \dotsm x_{i_{n+1}}
    =
    x_\alpha + r
  \]
  for some~$\alpha \in I^{(n+1)}$ and~$r \in \gr(\Univ(\glie))_{(n)}$ by rearranging the factors~$x_{i_j}$.
  The additional term~$r$ can by induction hypothesis be expressed as a linear combination of those~$x_\beta$ with~$\beta \in I^{(n)}$.
  The claim now follows because~$I^{(n)} \subseteq I^{(n+1)}$.
\end{proof}


\begin{corollary}
  \label{pbw concrete generating part}
  The elements~$x_\alpha$ with~$\alpha \in I^*$ span the algebra~$\Univ(\glie)$ as a vector space.
  \qed
\end{corollary}


\begin{construction}
  To show the linear independence of the elements~$x_\alpha$ with~$\alpha \in I^*$ we use a general construction from representation theory that we’re about to introduce:
  
  Suppose that we are given a vector space generating set~$(b_j)_{j \in J}$ of~$A$, which we would like to show to be a basis of~$A$.
  For this we construct an~{\module{$A$}}~$M$ together with an element~$m_0 \in M$ such that the elements~$b_j m_0$ with~$j \in J$ are linearly dependent in~$M$.
  It then follows that the elements~$b_j$ with$~j \in J$ are linearly independent.
  
  The main idea behind the construction of~$M$ --- and why such a module should exist in the first place --- is that if the elements~$b_j$ with~$j \in J$ are indeed a basis for$~A$ then we could choose~$M = A$ and~$m_0 = 1$.
  To construct the module~$M$ we therefore start by taking as its underlying vector space the free vector space on a basis~$B_j$ with~$j \in J$.
  We then want~$A$ to act on~$M$ in the same way as it acts on itself.
  
  How exactly the construction of this~{\module{$A$}} structure on~$M$ can be achieved depends on the given situation.
  We will outline two strategies here:
  The first is a very general one, whereas the second one will be used for the proof of the PBW~theorem:
  \begin{enumerate}
    \item
      We assume that~$A$ is given by some algebra generators~$x_i$ with~$i \in I$ and some relations between these generators.
      We can encode the action of the generators~$x_i$ on~$A$ by considering for all~$i \in I$ and~$j \in J$ a linear combination~$x_i b_j = \sum_{k \in J} c_{ij}^k b_k$ (if the~$b_j$ are indeed a basis for~$A$ then the coefficients~$c_{ij}^k$ will a posteriori turn out to be unique).
      We want to define an action of~$A$ on~$M$ via~$x_i B_j = \sum_{k \in J} c_{ij}^k B_k$ for all~$i \in I$ and~$j \in J$.
      We do so in two steps:
      \begin{enumerate}
        \item
          We first define a~{\module{$\kf\gen{X_i \suchthat i \in I}$}} structure on~$M$ via~$X_i \cdot B_j = \sum_{k \in I} c_{ij}^k B_k$.
          We can do this because a~{\module{$\kf\gen{X_i \suchthat i \in I}$}} structure on~$M$ is the same an algebra homomorphism~$\kf\gen{X_i \suchthat i \in I} \to \End_{\kf}(M)$, which can be constructed by using the universal property of~$\kf\gen{X_i \suchthat i \in I}$.
        \item
          We then check that this action descends to an~{\module{$A$}} structure on~$M$ by checking that it is compatible with the given relations between the algebra generators~$x_i$.
      \end{enumerate}
    \item
      Suppose that~$A = \Univ(\glie)$ is the universal enveloping algebra of some Lie~algebra~$\glie$.
      Then an~{\module{$A$}} structure on~$M$ is the same as letting~$\glie$ act on~$M$ so that~$x.y.m - y.x.m = [x,y].m$ for all~$x, y \in \glie$ and~$m \in M$.
      We will therefore try to construct such an action instead.
      We will do so in the given situation by considering a filtration~$M = \bigcup_{n \geq 0} M_{(0)}$ by linear subspaces~$M_{(i)}$ of~$M$ and then inductively construct partial actions~$\varphi_n \colon \glie \times M_{(n)} \to M_{(n+1)}$ that extend each other.
  \end{enumerate}
  
  For the last step we assume that~$1_A = b_k$ for some~$k \in J$.
  We then show that for~$m_0 \defined B_k$ we have~$b_j m_0 = B_j$ for all~$j \in J$, giving us the desired linear independence of the elements~$b_j m_0$ with~$j \in J$.

  We observe that if~$(b_j)_{j \in J}$ is indeed a basis of~$A$ then this construction of the~{\module{$A$}}~$M$ has to go through, as it is just an artifical reconstruction of the welldefined~{\module{$A$}}~$M = A$.
  So if the construction succeeds then we have shown that~$(b_j)_{j \in J}$ is indeed a basis for~$A$, but if the cnostruction fails then we have shown that it’s not a basis.
\end{construction}


\begin{example}
  \label{linear independence for weyl algebra}
  Let us return to the Weyl~algebra~$A \defined \kf\gen{X, Y}/(YX - XY - 1)$\index{Weyl!algebra} from \cref{weyl algebra}.
  We again denote the residue classes of~$X$ and~$Y$ in~$A$ by~$x$ and~$y$.
  We still need to prove \cref{linear independence of monomials} from \cref{weyl algebra}, that the monomials~$x^n y^m$ with~$n, m \geq 0$ are linearly independent in~$A$.
  
  Let~$M$ be the free vector space on a basis~$T^n U^m$ with~$n, m \geq 0$.
  (The basis element~$T^n U^m$ of~$M$ corresponds to the proposed basis element~$x^n y^m$ of~$A$.)
  The action of the two algebra generators~$x$ and~$y$ of~$A$ on the vector space generators~$x^n y^m$ of~$A$ is given by
  \begin{align*}
    x \cdot x^n y^m
    &=
    x^{n+1} y^m \,,
    \\
    y \cdot x^n y^m
    &=
    x^n y^{m+1} + n x^{n-1} y^m
  \end{align*}
  for all~$n, m \geq 0$.
  We therefore define a~{\module{$\kf\gen{X,Y}$}} structure on~$M$ by
  \begin{align*}
    X \cdot T^n U^m
    &=
    T^{n+1} U^m \,,
    \\
    Y \cdot T^n U^m
    &=
    T^n U^{m+1} + n T^{n-1} U^m
  \end{align*}
  for all~$m, n \geq 0$.
  This action descends to an~{\module{$A$}} structure on~$M$ because for all~$m, n \geq 0$,
  \begin{gather*}
    YX \cdot T^n U^m
    =
    Y \cdot T^{n+1} U^m
    =
    T^{n+1} U^{m+1} + (n+1) T^n U^m
  \shortintertext{and}
    XY \cdot T^n U^m
    =
    X \cdot ( T^n U^{m+1} + n T^{n-1} U^m )
    =
    T^{n+1} U^{m+1} + n T^n U^m
  \end{gather*}
  and hence
  \[
    (YX - XY - 1) \cdot T^n U^m
    =
    0 \,.
  \]
  We have that for every~$m \geq 0$ that
  \[
    y \cdot T^0 U^m
    =
    Y \cdot T^0 U^m
    =
    T^0 U^{m+1}
  \]
  and hence inductively
  \[
    y^k \cdot T^0 U^m
    =
    T^0 U^{m+k}
  \]
  for every~$k \geq 0$.
  It follows that
  \[
    x^n y^m \cdot T^0 U^0
    =
    x^n \cdot T^0 U^m
    =
    T^n U^m
  \]
  for all~$n, m \geq 0$.
  It now follows from the linear independence of the basis elements~$T^n U^m$ of~$M$ that the elements~$x^n y^m$ are linearly independent in~$A$.
\end{example}


\begin{example}
  Similar approaches can be applied when it comes to other algebraic structures, as for example groups:
  \begin{enumerate}
    \item
      If~$X$ is a set and~$F$ is the free group on~$X$ then every element of~$F$ can be uniquely written as a reduced word in~$X^{\pm 1}$.
      The uniqueness of such a reduced expression can be shown by letting~$F$ act on the set of all reduced words.
    \item
      Let~$G_i$ with~$i \in I$ be a collection of groups and let~$H$ be a group.
      For every~$i \in I$ let~$f_i \colon H \to G_i$ be a group homomorphism.
      Then elements of the resulting amalgamated product~$A = \ast_{H,i \in I} G_i$ can be represented in certain a canonical form, the uniqueness of which can be shown by letting~$A$ act on the set of all possible canonical forms.
      We refer to \cite[\S 1.2]{trees} for more information on this example.
  \end{enumerate}
\end{example}


\begin{proof}[Proof of \cref{pbw concrete}]
  By \cref{pbw concrete generating part} the elements~$x_\alpha$ with~$\alpha \in I^*$ span~$\Univ(\glie)$ as a vector space, but we still need to show that they are linearly independent.
  
  Let~$V$ be a vector space with basis~$X_{i_1} \dotsm X_{i_n}$ where~$n \geq 0$ and~$(i_1, \dotsc, i_n) \in I^n$.
  This means that for every~$\alpha \in I^*$ with~$\alpha = (i_1, \dotsc, i_n)$ we have an associated basis element
  \[
    X_\alpha
    \defined
    X_{i_1} \dotsm X_{i_n}
  \]
  of~$V$.
  For every~$n \geq 0$ we denote by~$V_{(n)} \subseteq V$ the~{\linear{$\kf$}} subspace spanned by all~$X_\alpha$ with~$\alpha \in I^{(n)}$.

  \begin{claim*}
    There exists a unique sequence~$(\varphi_n)_{n \geq 0}$ of bilinear maps
    \[
      \varphi_n
      \colon
      \glie \times V_{(n)}
      \to
      V_{(n+1)},
      \quad
      (x,p)
      \mapsto
      x.p
    \]
    that satisfy the following conditions:
    \begin{enumerate}
    \item
      \label{pbw restriction coincides}
      The restriction of~$\varphi_{n+1}$ to~$\glie \times V_{(n)}$ coincides with~$\varphi_n$ for every~$n \geq 0$.%
      \footnote{This condition actually follows from the other conditions by the uniqueness of the sequence~$(\varphi_n)_{n \geq 0}$.
    See \cite[\S 17.4]{humphreys} for more details on this.}
    \item
      \label{pbw representation of lie algebra}
      $x_i.x_j.X_\alpha - x_j.x_i.X_\alpha = [x_i, x_j].X_\alpha$ for all~$i,j \in I$ and~$\alpha \in I^*$.
    \item
      \label{pbw essential condition}
      $x_i.X_\alpha = X_{i \cdot \alpha}$ for all~$i \in I$ and~$\alpha \in I^*$ with $i \leq \alpha$.
    \item
      \label{pbw technical detail for construction}
      $x_i.X_\alpha \equiv X_{i \cdot \alpha} \pmod{V_{(n)}}$ for all~$n \geq 0$,~$i \in I$ and~$\alpha \in I^{(n)}$.
    \end{enumerate}
 \end{claim*}

  \begin{proof}
    Let us first point out that the notation~$x.p$ with~$x \in \glie$ and~$p \in V$ is unambiguous by condition~\ref{pbw restriction coincides};
    we do not need to worry about which~$\varphi_n$ it refers to.
    The maps~$\varphi_n$ will be defined by induction over~$n \geq 0$:
    
    To define the map~$\varphi_0$ we do not have to consider condition~\ref*{pbw restriction coincides}.
    The linear subspace~$V_{(0)}$ is {\onedimensional} and spanned by the single element $X_{()}$.
    We see from condition~\ref*{pbw essential condition} that we need
    \[
      x_i.1
      =
      x_i.X_{()}
      =
      X_{i \cdot ()}
      =
      X_i
    \]
    for every~$i \in I$.
    We take this as the definition for~$\varphi_0$.
    Conditions~\ref*{pbw essential condition} hence hold by choice of~$\varphi_0$~and condition~\ref*{pbw technical detail for construction} follows from this (because always~$i \leq \alpha$ for~$n = 0$).
    Condition~\ref*{pbw representation of lie algebra} does not affect the case~$n = 0$ yet.
  
    Let now~$n \geq 0$ and suppose that~$\varphi_m$ is constructed for every~$m \leq n$.
    To construct~$\varphi_{n+1}$ we need by condition~\ref*{pbw restriction coincides} define only~$x_i.X_\alpha$ for~$\alpha \in I^{n+1}$.
    
    If~$i \leq \alpha$ then we set
    \begin{equation}
      \label{formula for smaller case}
      x_i.X_\alpha
      \defined
      X_{i \cdot \alpha}
    \end{equation}
    to ensure condition~\ref*{pbw essential condition} for~$\varphi_{n+1}$.
    Otherwise we can write~$\alpha = j \cdot \beta$ for some~$j \in I$ and~$\beta \in I^n$ with~$j \leq \beta$ and~$i > j$.
    It then follows from condition~\ref*{pbw representation of lie algebra} and~\ref*{pbw essential condition} that we need
    \[
      x_i.X_\alpha
      =
      x_i.X_{j \cdot \beta}
      =
      x_i.x_j.X_\beta
      =
      x_j.x_i.X_\beta + [x_i, x_j].X_\beta  \,.
    \]
    The second summand~$[x_i, x_j].X_\beta$ is already defined by induction hypothesis.
    For the first summand we observe that by condition~\ref*{pbw technical detail for construction} for~$\varphi_n$ we have
    \[
      x_i.X_\beta
      \equiv
      X_{i \cdot \beta}
      \mod
      V_{(n)}
    \]
    and hence~$x_i.X_\beta = X_{i \cdot \beta} + R$ for some unique error term~$R \in V_{(n)}$.
    It follows from~$j < i$ and~$j \leq \beta$ and that~$j \leq i \cdot \beta$.
    Hence the product~$x_j.X_{i \cdot \beta}$ needs to be given by
    \[
      x_j.X_{i \cdot \beta}
      =
      X_{j \cdot i \cdot \beta}
      =
      X_{i \cdot j \cdot \beta}
      =
      X_{i \cdot \alpha}  \,.
    \]
    Altogether we need that
    \begin{equation}
      \label{formula for bigger case}
      \begin{aligned}
        x_i.X_\alpha
        &=
        x_i.X_{j \cdot \beta}
        \\
        &=
        x_i.x_j.X_\beta
        \\
        &=
        x_j.x_i.X_\beta + [x_i, x_j].X_\beta
        \\
        &=
        x_j.(X_{i \cdot \beta} + R) + [x_i, x_j].X_\beta
        \\
        &=
        x_j.X_{i \cdot \beta} + x_j.R + [x_i, x_j].X_\beta
        \\
        &=
        X_{i \cdot \alpha} + x_j.R + [x_i, x_j].X_\beta
      \end{aligned}
    \end{equation}
    with~$j$ the first term of~$\alpha$,~$\beta$ the rest of~$\alpha$ and~$R = x_i.X_\beta - X_{i \cdot \beta}$.
    Every term of the last expression is already defined by~$\varphi_n$, which shows the uniqueness of~$\varphi_{n+1}$.
    We take~\eqref{formula for bigger case} as the definition of~$x_i.X_\alpha$ for~$i \nleq \alpha$.
    
    We overall define~$\varphi_{n+1}$ as the unique bilinear extension of~$\varphi_n$ such that~$x_i.X_\alpha$ is given on the additional elements with~$\alpha \in I^{n+1}$ given by~\eqref{formula for smaller case} and~\eqref{formula for bigger case} (depending on whether~$i \leq \alpha$ or not).
    
    Condition~\ref*{pbw restriction coincides} holds by construction of~$\varphi_{n+1}$ an an extension of~$\varphi_n$.
    Condition~\ref*{pbw essential condition} also holds by construction, namely by the corresponding property of~$\varphi_n$ together with the definition of~$x_i.X_\alpha$ for~$i \leq \alpha$ and~$\alpha \in I^{n+1}$ by formula~\eqref{formula for smaller case}.
    Condition~\ref*{pbw technical detail for construction} holds in the case~$i \leq \alpha$ by condition~\ref*{pbw essential condition} and otherwise by the corresponding property for~$\varphi_n$ or for~$\alpha \in I^{n+1}$ by the definition of~$\varphi_{n+1}$ via~\eqref{formula for bigger case}, because
    \[
      x_i.X_\alpha - X_{i.\alpha}
      =
      x_j.R + [x_i, x_j].X_\beta
      \in
      V_{(n)} \,.
    \]
  
    It remains to check condition~\ref*{pbw representation of lie algebra} for~$\varphi_{n+1}$.
    We only need to check this condition for~$i,j \in I$ and~$\alpha \in I^n$ as all other cases are covered by the corresponding property of~$\varphi_n$.
    For~$i = j$ the needed equality follows from the Lie~bracket being alternating.
    We therefore consider in the following only the case~$i \neq j$.
    We consider different cases:
    
    \begin{description}
      \item[Case~1 \textup($i \leq \alpha$\textup):]
        We consider two subcases:
        \begin{description}
          \item[Case~1.1 \textup($i \leq \alpha$,~$i < j$\textup):]
            In this case~$j \nleq i \cdot \alpha$ and we can apply (the first lines of)~\eqref{formula for bigger case} to find that
            \[
              x_j.x_i.X_\alpha
              =
              x_j.X_{i \cdot \alpha}
              =
              x_i.x_j.X_\alpha + [x_j,x_i].X_\alpha \,.
            \]
            By rearranging this equation we get the desired equality for the given case.
          \item[Case~1.2 \textup($i \leq \alpha$,~$i > j$\textup):]
            It follows in this case that~$j \leq \alpha$.
            We therefore find by the previous case that
            \[
              x_j.x_i.X_\alpha - x_i.x_j.X_\alpha
              =
              [x_j, x_i].X_\alpha \,.
            \]
            By using that the Lie~bracket~$[-,-]$ is skew symmetric we get the desired equality
            \[
              [x_i, x_j].X_\alpha
              =
              -[x_j, x_i].X_\alpha
              =
              -(x_j.x_i.X_\alpha - x_i.x_j.X_\alpha)
              =
              x_i.x_j.X_\alpha - x_j.x_i.X_\alpha \,.
            \]
        \end{description}
      \item[Case~2 \textup($j \leq \alpha$\textup):]
        This follows from the previous case because the Lie~bracket~$[-,-]$ is skew-symmetric.
      \item[Case~3 \textup($i \nleq \alpha$,~$j \nleq \alpha$\textup):]
        This case cannot happen for~$n = 0$ because then~$i \leq () = \alpha$.
        So we may assume that~$n \geq 1$.
        We can then split up the index~$\alpha$ as~$\alpha = k \cdot \beta$ where~$k \in I$ is the first term of~$\alpha$ and~$\beta \in I^{n-1}$ is the rest of~$\alpha$.
        We then have
        \[
          X_\alpha
          =
          X_{k \cdot \beta}
          =
          x_k.X_\beta \,.
        \]
        We know from the induction hypothesis that condition~\ref*{pbw representation of lie algebra} holds for~$\varphi_n$.
        We therefore have
        \[
          x_j.X_\alpha
          =
          x_j.x_k.X_\beta
          =
          x_k.x_j.X_\beta + [x_j, x_k].X_\beta  \,.
        \]
        By acting with~$x_i$ on this equation we find the equality
        \begin{equation}
          \label{long equation}
          x_i.x_j.X_\alpha
          =
          x_i.x_k.x_j.X_\beta + x_i.[x_j,x_k].X_\beta \,.
        \end{equation}
        To reshape the first summand we want to apply condition~\ref*{pbw representation of lie algebra} to find for the element~$Y = x_j.X_\beta$ that
        \begin{equation}
          \label{wanted equation for pbw}
          x_i.x_k.Y
          =
            x_k.x_i.Y
          + [x_i, x_k].Y  \,.
        \end{equation}
        But for this we need to explain why condition~\ref*{pbw representation of lie algebra} can be applied.
        We observe that
        \[
          x_j.X_\beta
          \equiv
          X_{j \cdot \beta}
          \mod
          V_{(n-1)}
        \]
        by condition~\ref*{pbw technical detail for construction}, so that
        \[
          Y
          =
          x_j.X_\beta
          =
          X_{j \cdot \beta} + R 
        \]
        for some rest term~$R \in V_{(n-1)}$.
        The desired equation~\eqref{wanted equation for pbw} holds when we replace~$Y$ by~$R$ because~$\varphi_n$ satisfies condition~\ref*{pbw representation of lie algebra} for every~$X_\gamma$ with~$\gamma \in I^{(n-1)}$ and therefore for every~$R' \in V_{(n-1)}$.
        It follows from the previous cases that equation~\eqref{wanted equation for pbw} also holds when we replace~$Y$ by~$X_{j \cdot \beta}$ because it follows from~$k < j$ and~$k \leq \beta$ that~$k \leq j \cdot \beta$.5
        It follows by additivity that equation~\eqref{wanted equation for pbw} also holds for~$Y = X_{j \cdot \beta} + R$.
        
        By inserting~\eqref{wanted equation for pbw} with~$Y = x_j.X_\beta$ into the identity~\eqref{long equation} we find
        \[
          x_i.x_j.X_\alpha
          =
            x_k.x_i.x_j.X_\beta
          + [x_i, x_k].x_j.X_\beta
          + x_i.[x_j,x_k].X_\beta \,.
        \]
        We may swap the roles of~$i$ and~$j$ to also find
        \[
          x_j.x_i.X_\alpha
          =
            x_k.x_j.x_i.X_\beta
          + [x_j, x_k].x_i.X_\beta
          + x_j.[x_i,x_k].X_\beta \,.
        \]
        By subtracting these two identities we find
        \begin{align*}
          {}&
          x_i.x_j.X_\alpha - x_j.x_i.X_\alpha
          \\
          ={}&
            x_k.x_i.x_j.X_\beta
          + [x_i, x_k].x_j.X_\beta
          + x_i.[x_j,x_k].X_\beta
          - x_k.x_j.x_i.X_\beta
          - [x_j, x_k].x_i.X_\beta
          - x_j.[x_i,x_k].X_\beta \,.
        \end{align*}
        By using that condition~\ref*{pbw representation of lie algebra} holds for~$\varphi_n$ we find that
        \begin{align*}
          {}&
            \underbrace{ x_k.x_i.x_j.X_\beta }_{1}
          + \underbrace{ [x_i, x_k].x_j.X_\beta }_{2}
          + \underbrace{ x_i.[x_j,x_k].X_\beta }_{3}
          - \underbrace{ x_k.x_j.x_i.X_\beta }_{1}
          - \underbrace{ [x_j, x_k].x_i.X_\beta }_{3}
          - \underbrace{ x_j.[x_i,x_k].X_\beta }_{2}
          \\
          ={}&
            x_k.[x_i, x_j].X_\beta
          + [[x_i, x_k], x_j].X_\beta
          + [x_i, [x_j, x_k]].X_\beta \,.
        \end{align*}
        It follows from the Jacobi~identity that
        \[
          [[x_i, x_k], x_j] + [x_i, [x_j, x_k]]
          =
          [x_j, [x_k, x_i]] + [x_i, [x_j, x_k]]
          =
          -[x_k, [x_i, x_j]]
        \]
        and therefore
        \begin{align*}
          {}&
            x_k.[x_i, x_j].X_\beta
          + [[x_i, x_k], x_j].X_\beta
          + [x_i, [x_j, x_k]].X_\beta
          \\
          ={}&
            x_k.[x_i, x_j].X_\beta
          - [x_k, [x_i, x_j]].X_\beta
          \\
          ={}&
            x_k.[x_i, x_j].X_\beta
          - x_k.[x_i, x_j].X_\beta
          + [x_i, x_j].x_k.X_\beta
          \\
          ={}&
          [x_i, x_j].x_k.X_\beta
          \\
          ={}&
          [x_i, x_j].X_{k \cdot \beta}
          \\
          ={}&
          [x_i, x_j].X_\alpha \,.
        \end{align*}
        We used for the last two steps that~$k \leq \beta$ with~$k \cdot \beta = \alpha$ by choice of~$\alpha$ and~$\beta$.
        This shows altogether that
        \[
          x_i.x_j.X_\alpha - x_j.x_i.X_\alpha
          =
          [x_i, x_j].X_\alpha
        \]
        as desired.
      \qedhere
    \end{description}
  \end{proof}
  
  Condition~\ref*{pbw restriction coincides} ensures that the maps~$(\varphi_n)_{n \geq 0}$ combine into a single bilinear map~$\varphi \colon \glie \times V \to V$.
  Condition~\ref*{pbw representation of lie algebra} means that~$\varphi$ makes~$V$ into a~{\representation{$\glie$}}.
  The action~$\varphi$ corresponds to an~{\module{$\Univ(\glie)$}} structure on~$V$.
  For every~$\alpha \in I^*$ with~$\alpha = (i_1, \dotsc, i_n)$ we have~$i_1 \leq \dotsb \leq i_n$ and hence
  \[
    x_\alpha . X_{()}
    =
    x_{i_1} \dotsm x_{i_n} \cdot X_{()}
    =
    x_{i_1} \dotsm x_{i_{n-1}} \cdot X_{(i_n)}
    =
    \dotsb
    =
    X_{(i_1, \dotsc, i_n)}
    =
    X_\alpha  \,.
  \]
  It follows from the linear independence of the elements~$X_\alpha$ with~$\alpha \in I^*$ in~$V$ that the elements~$x_\alpha$ with~$\alpha \in I^*$ are linearly independent in~$\glie$.
\end{proof}


\begin{corollary}
  \label{pbw concrete basis part filtered part}
  For every~$n \geq 0$ the filtered part~$\Univ(\glie)_{(n)}$ has the monomials~$x_\alpha$ with~$\alpha \in I^{(n)}$ as a basis.
\end{corollary}


\begin{proof}
  These monomials are a vector space generating set by \cref{pbw concrete generating part filtered part} and they are linearly independent by the PBW~theorem.
\end{proof}


\begin{corollary}
  \label{embedding into uea}
  For any Lie~algebra~$\glie$ the canonical homomorphism~$\glie \to \Univ(\glie)$ is injective.
\end{corollary}


\begin{proof}
  If~$(x_i)_{i \in I}$ is any basis of~$\glie$ thes the associated residue classes~$\class{x_i} \in \Univ(\glie)$ are part of a basis of~$\Univ(\glie)$ by the PBW~theorem and hence linearly independent.
\end{proof}


\begin{remark}
  \label{identification into uea}
  By \cref{embedding into uea} we may identify any Lie~algebra~$\glie$ with its image in its universal enveloping algebra~$\Univ(\glie)$.
  In the rest of these notes we will therefore regard every Lie~algebra~$\glie$ as a Lie~subalgebra of its universal enveloping algebra and the canonical homomorphism~$\glie \to \Univ(\glie)$ as the inclusion.
\end{remark}


\begin{remark}
  We may rewrite the isomorphism~$\Univ(\glie^{\op}) \cong \Univ(\glie)^{\op}$ from \cref{uea of opposite by first principles}, that is given by~$x \mapsto x$ for every~$x \in \glie$, as~$\Univ(\glie^{\op}) = \Univ(\glie)^{\op}$ by the aformentioned identification from \cref{identification into uea}.
\end{remark}


\begin{corollary}[Existence of free Lie~algebras]
  For every set~$I$ the free~{\liealgebra{$\kf$}}\index{free Lie algebra}\index{Lie algebra!free} on~$I$ exists.
\end{corollary}


\begin{proof}
  We have seen in \cref{uea of free lie algebra} that the universal enveloping algebra of the desired free~{\liealgebra{$\kf$}}~$F(I)$ is given by the free~{\algebra{$\kf$}}~$\kf\gen{x_i \suchthat i \in I}$.
  Since then we have learned that~$F(I)$ will therefore be a Lie~subalgebra of~$\kf\gen{x_i \suchthat i \in I}$.
  We thus define~$F(I)$ to be the Lie~subalgebra of~$\kf\gen{x_i \suchthat i \in I}$ generated by the variables~$x_i$ and let~$\iota \colon I \to \glie$ be the inclusion~$i \mapsto x_i$.
  We need to check that~$(F(I), \iota)$ defined in this way is indeed a free~{\liealgebra{$\kf$}} on~$I$.
  
  So let~$\hlie$ be any other~{\liealgebra{$\kf$}} and let~$\phi \colon I \to \hlie$ be any set-theoretic map.
  Then there exists by the universal property of the free~{\algebra{$\kf$}}~$\kf\gen{x_i \suchthat i \in I}$ a (unique) algebra homomorphism~$\Phi' \colon \kf\gen{x_i \suchthat i \in I} \to \Univ(\hlie)$ with~$\Phi'(x_i) = \phi(i)$ for every~$i \in I$.
  The image of~$F(I)$ under~$\Phi'$ lies in the Lie~subalgebra of~$\Univ(\hlie)$ that is generated by all~$\phi(i)$ with~$i \in I$, which in turn is contained in~$\hlie$.
  Hence the algebra homomorphism~$\Phi'$ restricts to a linear map~$\Phi \colon F(I) \to \hlie$, which is again a homomorphism of Lie~algebras.
  It follows from the commutativity of the square diagram
  \[
    \begin{tikzcd}[column sep = large]
      I
      \arrow{r}[above]{\phi}
      \arrow{d}[left]{i \mapsto x_i}
      &
      \hlie
      \arrow{d}
      \\
      \kf\gen{x_i \suchthat i \in I}
      \arrow[dashed]{r}[below]{\Phi'}
      &
      \Univ(F(I))
    \end{tikzcd}
  \]
  that the restricted diagram
  \[
    \begin{tikzcd}
      I
      \arrow{r}[above]{\phi}
      \arrow{d}[left]{\iota}
      &
      \hlie
      \arrow[equal]{d}
      \\
      F(I)
      \arrow[dashed]{r}[below]{\Phi}
      &
      \hlie
    \end{tikzcd}
  \]
  also commutes.
  We have thus shown the existence of an induced homomorphism~$\Phi \colon F(I) \to \hlie$ that makes the triangular diagram
  \[
    \begin{tikzcd}
      I
      \arrow{d}[left]{\iota}
      \arrow{r}[above]{\phi}
      &
      \hlie
      \\
      F(I)
      \arrow[dashed]{ur}[below right]{\Phi}
      &
      {}
    \end{tikzcd}
  \]
  commute.
  The uniqueness of this induced homomorphism follows from~$F(I)$ being generated by the image of~$\iota$.
\end{proof}


\begin{remark}
  Using the concept of free Lie~algebras one can define Lie~algebras by giving a set of generators~$X$ and a set of relations~$R \subseteq F(X)$.
\end{remark}


\begin{example}
  The Lie~algebra~$\sllie_2(\kf)$ can be defined by the generators~$X = \{e, h, f\}$ and the relations~$R = \{[h,e] - 2e, [h,f] + 2f, [e,f] - h\}$, which are usually written as~$[h,e] = 2e$,~$[h,f] = -2f$ and~$[e,f] = h$.
  
  One can more generally describe for every~$n \geq 1$ the Lie~algebra~$\sllie_{n+1}(\kf)$ by generators~$e_i$,~$f_i$, and~$h_i$ with~$i = 1, \dotsc, n$ subject to the relations
  \begin{align*}
    [h_i, h_j] = 0  \,,
    \quad
    [h_i, e_j] = a_{ij} e_j \,,
    \quad
    [h_i, f_j] = -a_{ij}f_j  \,,
    \quad
    [e_i, f_j] = \delta_{ij} h_i
  \end{align*}
  for all~$i,j = 1, \dotsc, n$ together with the relations
  \[
    \ad(e_i)^{1-a_{ij}}(e_j) = 0
    \quad\text{and}\quad
    \ad(f_i)^{1-a_{ij}}(f_j) = 0
  \]
  for all~$1 \leq i \neq j \leq n$, where the numbers~$a_{ij}$ are for all~$i,j = 1, \dotsc, n$ given by
  \[
    a_{ij} =
    \begin{cases}
    \phantom{-}2 & \text{if $i = j$}  \,, \\
              -1 & \text{if $|i-j| = 1$}  \,, \\
    \phantom{-}0 & \text{otherwise} \,.
    \end{cases}
  \]
\end{example}


\begin{corollary}
  \label{uea of direct sum of subspaces}
  Let~$\glie$ be a Lie algebra and let~$\hlie$ and~$\klie$ be two Lie~subalgebras of~$\glie$ with~$\glie = \hlie \oplus \klie$ as vector spaces.
  Then the multiplication map
  \[
    \Univ(\hlie) \tensor \Univ(\klie)
    \to
    \Univ(\glie) \,,
    \quad
    x \tensor y
    \mapsto
    xy
  \]
  is an isomorphism of vector spaces.
\end{corollary}


\begin{proof}
  Let~$(x_i)_{i \in I}$ be a basis of~$\hlie$ and let $(x_j)_{j \in J}$ be a basis of~$\klie$ where the index sets~$I$ and~$J$ are disjoint.
  Then $(x_k)_{k \in K}$ for the index set~$K \defined I \cup J$ is a basis of~$\glie$.
  Let~$(I, \leq)$ and~$(J, \leq)$ be linearly ordered.
  Then we can extend these two orders to a linear order on~$K$ such that~$i \leq j$ for all~$i \in I$ and~$j \in J$.
  We get from the PBW~theorem induced bases of~$\Univ(\hlie)$ and~$\Univ(\klie)$, and hence an induced basis of~$\Univ(\hlie) \tensor \Univ(\klie)$, as well as an induced basis of~$\Univ(\glie)$.
  The multiplication map~$\Univ(\hlie) \tensor \Univ(\klie) \to \Univ(\glie)$ is bijective on these bases and hence an isomorphism of vector spaces.
\end{proof}


\begin{remark}
  \Cref{uea of direct sum of subspaces} proves again that for all Lie~algebras~$\glie$ and~$\hlie$ the algebra homomorphism~$\Univ(\glie) \tensor \Univ(\hlie) \to \Univ(\glie \times \hlie)$ induced by the inclusions~$\glie \to \glie \times \hlie$ and~$\hlie \to \glie \times \hlie$ is an isomorphism of vector spaces, and hence an isomorphism of~{\algebras{$\kf$}}.
\end{remark}


\begin{remark}[The diamond lemma]
  The above proof of the PBW~theorem relies on a tactic from representation theory:
  Constructing an action on a suitable vector space for which the given generators act linearly independent (on some fixed element).
  We may ask if there is instead a proof which continues to study the commutator relation~$x_j x_i = x_i x_j + [x_i, x_j]$ with~$i, j \in I$ by exploiting that~$[x_i, x_j]$ is of smaller degree than~$x_j x_i$ and~$x_i x_j$.
  A similar question may also be asked for the Weyl~algebra.
  
  One way to do this is via the \defemph{diamond lemma}\index{diamond lemma}\index{lemma!diamond} as introduced by Bergman in~\cite{diamond_lemma}:
  Suppose that we are given an algebra~$A$ in the form of generators~$(x_i)_{i \in I}$ and relations~$\sigma \in \kf\gen{x_i \suchthat i \in I}$, $\sigma \in S$ (so that~$A = \kf\gen{x_i \suchthat i \in I}/(\sigma \suchthat \sigma \in S)$).
  Then the diamond lemma gives a criterion for showing that~$A$ admits a basis consisting of monomials in the~$x_i$, and also tells us how these monomials can be choosen.
  
  The main idea behind the diamond lemma is that one needs to be able to resolve ambiguities:
  We can represent every element of~$A$ as a linear combination of monomials in the~$x_i$, and hence as an element~$z \in \kf\gen{x_i \suchthat i \in I}$.
  Every relation~$\sigma$ between the generators can be seen as rewriting rules, that give us linear \enquote{rewriting maps}~$r_\sigma \colon \kf\gen{x_i \suchthat i \in I} \to \kf\gen{x_i \suchthat i \in I}$.
  To apply the diamond lemma we need two conditions:
  \begin{itemize}
    \item
      Whenever we apply rewritings~$r_{\sigma_1}, r_{\sigma_2}, \dotsc$ to an element~$z \in \kf\gen{x_i \suchthat i \in I}$ we want this process to stabilize after finitely many step.
    \item
      We need to be resolve ambiguities:
      Given an element~$z \in \kf\gen{x_i \suchthat i \in I}$ we can often apply different rewriting rules~$r_\sigma$ and~$r_\tau$ to it, resulting in different results~$r_\sigma(z)$ and~$r_\tau(z)$.
      We want to reconcile these different results by reducing both~$r_\sigma(z)$ and~$r_\tau(z)$ under finitely many rewritings to the same result~$z' \in \kf\gen{x_i \suchthat i \in I}$.
      We may visualize this process by the following diagram:
      \[
        \begin{tikzcd}[column sep = tiny]
          {}
          &
          z
          \arrow{dl}
          \arrow{dr}
          &
          {}
          \\
          r_\sigma(z)
          \arrow{dr}
          &
          {}
          &
          r_\tau(z)
          \arrow{dl}
          \\
          {}
          &
          z'
          &
          {}
        \end{tikzcd}
      \]
      The shape of this diagram is where the diamond lemma takes its name from.
  \end{itemize}
  
  Despite its usefulness the diamond is neither hard to state nor to prove.
  We strongly encourage the reader to check out the original paper \cite{diamond_lemma}.
  The main theorem and its proof in \cite[\S 1]{diamond_lemma} takes only three pages and is self-contained.
  In \cite[\S 3]{diamond_lemma} the PBW~theorem is then shown as an example.
\end{remark}






\subsection{Abstract Version of the Poincar\'{e}--Birkhoff--Witt Theorem}


\begin{theorem}[Poincar\'{e}--Birkhoff--Witt (abstract version)]
  \label{pbw abstract}
  Let~$\glie$ be a Lie~algebra and denote by~$\pi$ the canonical projection~$\Tensor(\glie) \to \Univ(\glie)$ given by
  \[
    \pi
    \colon
    \Tensor(\glie)
    \to
    \Univ(\glie) \,,
    \quad
    x_1 \tensor \dotsb \tensor x_n
    \mapsto
    x_1 \dotsm x_n
  \]
  for all $x_1, \dotsc, x_n \in \glie$.
  Then~$\pi$ induces a homomorphism of graded~{\algebras{$\kf$}}
  \[
    \gr(\pi)
    \colon
    \Tensor(\glie)
    =
    \gr(\Tensor(\glie))
    \to
    \gr(\Univ(\glie))
  \]
  and we also have a homomorphism of graded~{\algebras{$\kf$}}~$\pi' \colon \Tensor(\glie) \to \Symm(\glie)$ that  is given by
  \[
    \pi'
    \colon
    \Tensor(\glie)
    \to
    \Symm(\glie) \,,
    \quad
    x_1 \tensor \dotsb \tensor x_n
    \mapsto
    x_1 \dotsm x_n
  \]
  for all~$x_1, \dotsc, x_n \in \glie$.
  Then both~$\pi$ and~$\pi'$ have the same kernel and thus induce an isomorphism of graded algebras~$\Symm(\glie) \to \gr(\Univ(\glie))$ given by
  \begin{equation}
    \label{definition of abstract pbw isomorphism}
    \varphi
    \colon
    \Symm(\glie)
    \to
    \gr(\Univ(\glie)) \,,
    \quad
    x_1 \dotsm x_n
    \mapsto
    [x_1 \dotsm x_n]_n \,.
  \end{equation}
  for all~$x_1, \dotsc, x_n \in \glie$.
\end{theorem}


\begin{proposition}
  The abstract versions of the PBW~theorem is equivalent to the concrete version.
\end{proposition}


\begin{proof}
  It follows from the definition of~$\gr(\pi)$ that
  \[
    \gr(\pi)(x \tensor y - y \tensor x)
    =
    [xy - yx]_2
    \in
    \gr(\Univ(\glie))_2
  \]
  for all~$x, y \in \glie$.
  We have for the occuring representative~$xy - yx \in \Univ(\glie)_{(2)}$ that
  \[
    xy - yx
    =
    [x,y]
    \in
    \Univ(\glie)_{(1)}
  \]
  and hence
  \[
    [xy - yx]_2
    =
    0 \,.
  \]
  This shows that~$\gr(\pi)(x \tensor y - y \tensor x) = 0$ for all~$x, y \in \glie$.
  The kernel of~$\pi'$ is generated by the elements~$x \tensor y - y \tensor x$ with~$x,y \in \glie$ so it follows that~$\pi'$ factors through a homomorphism of graded~\algebras{$\kf$}~$\varphi \colon \Symm(\glie) \to \gr(\Univ(\glie))$ that is given by~\eqref{definition of abstract pbw isomorphism}.
  
  \begin{implicationlist}
    \item[concrete~$\implies$~abstract:]
      It follows for every degree~$d \geq 0$ from \cref{pbw concrete basis part filtered part} that the graded component~$\gr[d](\Univ(\glie))$ has the monomials~$x_\alpha$ with~$\alpha \in I^d$ as a basis.
      The homogeneous component~$\Symm^d(\glie)$ has as a basis the simple symmetric tensors~$x_{i_1} \dotsm x_{i_d}$ with~$i_1, \dotsc, i_d \in I$ and~$i_1 \leq \dotsb \leq$.
      The homomorphism~$\varphi$ restricts to a bijection between these bases and is therefore an isomorphism (of vector spaces and hence of graded~{\algebras{$\kf$}}).
    \item[abstract~$\implies$~concrete:]
      We need to show that the monomials~$x_\alpha$ with~$\alpha \in I^*$ are linearly independent.
      Suppose otherwise.
      Then there exists some minimal~$n \geq 0$ such that the monomials~$x_\alpha$ with~$\alpha \in I^{(n)}$ are linearly dependent.
      Let
      \[
        0
        =
        \sum_{\alpha \in I^{(n)}}
        \lambda_\alpha x_\alpha
      \]
      be a non-trivial linear relation.
      Then
      \[
        0
        =
        \sum_{\alpha \in I^{(n)}}
        \lambda_\alpha x_\alpha
        \equiv
        \sum_{\alpha \in I^n}
        \lambda_\alpha x_\alpha
        \mod
        \Univ(\glie)_{(n-1)}
      \]
      and therefore
      \begin{equation}
        \label{nontrivial linear combination in graded}
        0
        =
        \sum_{\alpha \in I^n}
        \lambda_\alpha [x_\alpha]_n
      \end{equation}
      in~$\gr[n](\Univ(\glie))$.
      The minimality of~$n$ means that~\eqref{nontrivial linear combination in graded} is a non-trivial linear relation, i.e.\ that~$\lambda_\alpha \neq 0$ for some~$\alpha \in I^n$.
      By using the isomorphism~$\varphi$ we get in the corresponding graded component~$\Symm^n(\glie)$ of~$\Symm(\glie)$ a non-trivial linear relation between the simple symmetric tensors.
      But these are linear independent --- a contradiction.
    \qedhere
  \end{implicationlist}
\end{proof}


\begin{corollary}
  \label{uea contains no zero divisors}
  Let~$\glie$ be a Lie~algebra.
  \begin{enumerate}
    \item
      The universal enveloping algebra~$\Univ(\glie)$ contains no nonzero zero divisors.
    \item
      If~$\glie$ is finite dimensional then~$\Univ(\glie)$ is both left noetherian and right noetherian.
  \end{enumerate}
\end{corollary}


\begin{proof}
  \leavevmode
  \begin{enumerate}
    \item
      The statement holds for the associated graded algebra~$\gr(\Univ(\glie)) \cong \Symm(\glie)$ because it is an integral domain.
      It follows from \cref{associated graded algebra and zero divisors} that this also holds for~$\Univ(\glie)$.
    \item
      If~$\glie$ is~{\dimensional{$n$}} then~$\gr(\Univ(\glie)) \cong \Symm(\glie) \cong k[t_1, \dotsc, t_n]$ is noetherian and therefore~$\Univ(\glie)$ is both left noetherian and right noetherian by \cref{universal enveloping reflects chain conditions}
    \qedhere
  \end{enumerate}
\end{proof}


\begin{lemma}
  If~$\glie$ is any nonzero Lie~algebra then~$\Univ(\glie)$ is neither left artinian nor right artinian.
\end{lemma}


\begin{proof}
  We show that~$\Univ(\glie)$ is not left artinian;
  that~$\Univ(\glie)$ is not right artinian can then be shown in the same way.
  
  Let~$(x_i)_{i \in I}$ be a basis of~$\glie$ such that~$(I, \leq)$ is linearly ordered with maximal element~$j$.
  Then for every~$m \geq 0$ the left ideal~$\Univ(\glie) x_j^m$ has as a basis all monomials~$x_{i_1}^{n_1} \dotsm x_{i_r}^{n_r} x_j^m$ with~$r \geq 0$,~$n_i \geq 0$ and~$i_1 < \dotsb < i_r < j$.
  Therefore
  \[
    \Univ(\glie)
    \supsetneq
    \Univ(\glie) x_j
    \supsetneq
    \Univ(\glie) x_j^2
    \supsetneq
    \Univ(\glie) x_j^3
    \supsetneq
    \dotsb
  \]
  is a strictly decreasing sequence of left ideals in~$\Univ(\glie)$.
\end{proof}


\begin{remark}
  There exist examples for infinite dimensional Lie~algebras whose universal enveloping algebras are not noetherian, see \cite{uea_of_witt_algebra_not_noetherian}.
  It is (according to \cite[p.\ xix]{introduction_to_noncommutative_noetherian}) an open problem if the universal enveloping algebra of an infinite dimensional Lie~algebra must be non-noetherian.
\end{remark}


\begin{remark}
  One can think about the universal enveloping algebra~$\Univ(\glie)$ as a deformation of the symmetric algebra~$\Symm(\glie)$, in the sense that we have a family of algebras
  \[
    A_t
    \defined
    \Tensor(\glie)/(x \tensor y - y \tensor x - t[x,y] \suchthat x, y \in \glie)
  \]
  that is parametrized over a parameter~$t \in \kf$ with~$A_0 \cong \Symm(\glie)$ and~$A_1 \cong \Univ(\glie)$.
  One can then prove the abstract version of the PBW~theorem by examining this deformation.
  See~\cite{pbw_deformation} for more details on this.
\end{remark}












% TODO: Hopf algebra structure




