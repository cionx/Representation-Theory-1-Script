\chapter{The Universal Enveloping Algebra}


\begin{convention}
  The following holds for this section alone:
  We fix an arbitrary field~$\kf$.
  By a~{\algebra{$\kf$}} we always mean an associative and unitary one, and homomorphisms of~{\algebras{$\kf$}} have to respect the units.
  The resulting category of~{\algebras{$\kf$}} with homomorphisms of~{\algebra{$\kf$}} between them will be denoted by~$\cAlg{\kf}$.
  All Lie~algebras considered will have~$\kf$ as their ground field, unless otherwise stated.
  
  If~$A$ is a~{\algebra{$\kf$}} then by an~\defemph{\module{$A$}}\index{module} we mean a left unitial~{\module{$A$}}.
  The resulting category of~{\modules{$A$}} is denoted by~\gls*{module category}.
\end{convention}





\section{Tensor Algebra and Symmetric Algebra}


% \begin{example}[Monoid algebra]
%   In the following all monoids will be written multiplicaitvely unless otherwise mentioned.
%   The neutral element of a monoid~$M$ will be denoted by~$1$ or~$1_M$.
%   Given two monoids~$M$ and~$N$ a map~$f \colon M \to N$ is a homomorphism of monids if~$f(m \cdot m') = f(m) \cdot f(m')$ for all~$m, m' \in M$ and~$f(1_M) = 1_N$.
%   If~$M$ is any monoid then the identity~$\id_M$ is a homomorphism and if~$f \colon M \to N$ and~$g \colon N \to P$ are composable homomorphisms of monoids then their composition~$g \circ f \colon M \to P$ is again a homomorphism of monoids.
%   The resulting category of monoids is denoted by~$\cMon$.
%   
%   \begin{description}
%     \item[Construction:]
%       If~$M$ is a monoid then the monoid algebra~\gls*{monoid algebra} is the (free) vector space with basis~$M$ together with the unique bilinear extension~$\kf[M] \times \kf[M] \to \kf[M]$ of the multiplication~$M \times M \to M$ as its multiplication.
%   
%       This means that the elements of~$\kf[M]$ are formal {\linear{$\kf$}} combinations~$\sum_{m \in M} a_m m$ with~$a_m = 0$ for all but finitely many~$m \in M$.
%       The multiplication of two such elements is given by
%       \[
%         \left(
%           \sum_{m \in M} a_m m
%         \right)
%         \left(
%           \sum_{n \in M} b_n n
%         \right)
%         =
%         \sum_{m, n \in M} (a_m b_n) m n  \,.
%       \]
%       We identify every element~$m \in M$ with the corresponding element~$1 \cdot m \in \kf[M]$.
%       The product~$m \cdot n$ of two elements~$m, n \in M$ in~$\kf[M]$ is then the same as their product in~$M$.
%       The associativity of the multiplication of~$\kf[M]$ follows from the associativity of the multiplication of~$M$, and the neutral element of~$M$ is given by the multiplicative neutral element for~$\kf[M]$.
%       
%     \item[Universal Property:]
%       If~$A$ is any~{\algebra{$\kf$}} then~$(A, \cdot)$ is a multiplicative monoid, which we will denote by~$A^-$.
%       If~$M$ is any monoid and~$f \colon M \to A^-$ is a monoid hommorphism then~$f$ extends uniquely to an algebra homomorphism~$F \colon \kf[M] \to A$.
%       The algebra homomorphism~$F$ is given on elements by
%       \[
%         F\left( \sum_{m \in M} a_m m \right)
%         =
%         \sum_{m \in M} a_m f(m) \,.
%       \]
%       On the other hand every algebra homomorphism~$\kf[M] \to A$ restricts to a monoid homomorphism~$M \to A^-$.
%       This construction results in a {\onetoone} correspondence
%       \[
%         \{
%           \text{monoid homomorphisms~$M \to A^-$}
%         \}
%         \longonetoone
%         \{
%           \text{algebra homomorphisms~$\kf[M] \to A$}
%         \}  \,.
%       \]
%     
%     \item[Uniqueness:]
%       The monoid algebra~$\kf[M]$ together with the inclusion~$i \colon M \to \kf[M]$ is uniquely determined by its universal property up to isomorphism:
%   \end{description}
% \end{example}


% \begin{recall}[Free algebra]
%   Let~$I$ be any set.
%   The \defemph{noncommutative polynomial algebra}~$\kf\gen{X_i \suchthat i \in I}$\index{noncommutative polynomial algebra} has as a basis the set of all monomials
%   \[
%     X_{i_1} \dotsm X_{i_n}
%     \qquad
%     \text{with~$i_1, \dotsc, i_n$}
%   \]
%   and the multiplication is on these basis elements given by
%   \[
%     X_{i_1} \dotsm X_{i_n}
%     \cdot
%     X_{j_1} \dotsm X_{j_m}
%     =
%     X_{i_1} \dotsm X_{i_n} X_{j_1} \dotsm X_{j_m} \,.
%   \]
%   In contrast to the usual (commutative) polynomial algebra~$\kf[X_i \suchthat i \in I]$ the variables~$X_i$ are not required to commute with each other.
%   
%   We can alternatively construct~$\kf\gen{X_i \suchthat i \in I}$ as the monomial algebra of the free monoid on~$I$:
%   Let~$M$ be the set of all words in~$I$, i.e.\ the set of all finite sequences
%   \[
%     (i_1, \dotsc, i_n)
%     \qquad
%     \text{with~$i_1, \dotsc, i_n \in I$}  \,.
%   \]
%   Then~$M$ is a monoid with respect to concatenation of words given by
%   \[
%     (i_1, \dotsc, i_n) (j_1, \dotsc, j_m)
%     =
%     (i_1, \dotsc, i_n, j_1, \dotsc, j_m)
%   \]
%   for all words~$(i_1, \dotsc, i_n), (j_1, \dotsc, j_m) \in M$.
%   The neutral element of~$M$ is given by the empty word~$()$.
% \end{recall}


\begin{recall}[Tensor algebra]
  Let~$V$ be a vector space.
  \begin{description}
    \item[Construction:]
      For all~$v_1, \dotsc, v_d \in V$ we denote the resulting simple tensor~$v_1 \tensor \dotsb \tensor v_d$ in~$V^{\tensor d}$ by~$(v_1, \dotsc, v_d)$.
      Observe that for~$d = 0$ the tensor power~$V^{\tensor d} = V^{\tensor 0}$ has as a basis the emtpy simple tensor~$()$.
      We will therefore identify the tensor power~$V^{\tensor 0}$ with the ground field~$\kf$, so that empty simple tensor~$()$ corresponds to~$1 \in \kf$.
      
      For all~$p, q \geq 0$ we define a multiplication
      \[
        \mu_{p,q}
        \colon
        V^{\tensor p} \times V^{\tensor q}
        \to
        V^{\tensor (p+q)} \,,
        \quad
        (x,y)
        \mapsto
        x y
      \]
      that is on simple tensors~$(v_1, \dotsc, v_p) \in V^{\tensor p}$ and~$(v_{p+1}, \dotsc, v_{p+q}) \in V^{\tensor q}$ given by
      \[
        (v_1, \dotsc, v_p) \cdot (v_{p+1}, \dotsc, v_{p+q})
        =
        (v_1, \dotsc, v_{p+q})  \,.
      \]
      Note that for~$p = 0$ or~$q = 0$ this multiplication is just scalar multiplication.  
      These multiplications fit together associatively in the sense that for all~$p, q, r \geq 0$ and simple tensors~$x \in V^{\tensor p}$,~$y \in V^{\tensor q}$ and~$z \in V^{\tensor r}$ the equality
      \[
        x \cdot (y \cdot z)
        =
        (x \cdot y) \cdot z
      \]
      holds.
      
      Let~$\Tensor(V) \defined \bigoplus_{d \geq 0} V^{\tensor d}$.
      We can fit together the multiplications~$\mu_{p,q}$ with~$p, q \geq 0$ to a single multiplication
      \[
        \mu
        \colon
        \Tensor(V) \times \Tensor(V)
        \to
        \Tensor(V)  \,,
        \quad
        (x,y)
        \mapsto
        xy 
      \]
      that is given on elements~$x, y \in \Tensor(V)$ with~$x = (x_d)_{d \geq 0}$ and~$y = (y_d)_{d \geq 0}$ by
      \[
        x y
        =
        \left(
          \sum_{p+q = d} x_p y_q
        \right)_{d \geq 0} \,.
      \]
      This multiplication is built precisely so that it follows from the bilinearity of the multiplications~$\mu_{p,q}$ that the multipliation~$\mu$ is again bilinear.
      It follows from the associativities of the multiplications~$\mu_{p,q}$ that the multiplication~$\mu$ is associative.
      We may identify the ground field~$\kf = V^{\tensor 0}$ with the corresponding direct summand in~$\Tensor(V)$ to regard~$\kf$ as a linear subspace of~$\Tensor(V)$.
      We have seen above that~$1 \in \kf$ is then unital for the multiplication of~$\Tensor(V)$.
      We have thus altogether constructed a~{\algebra{$\kf$}}~$\Tensor(V)$.
      
      We may identify~$V = V^{\tensor 1}$ with the corresponding direct summand of~$\Tensor(V)$ to regard~$V$ as a linear subspace of~$\Tensor(V)$.
      We then have for all~$v_1, \dotsc, v_n \in V$ that
      \[
        v_1 \dotsm v_n
        =
        (v_1) \dotsm (v_n)
        =
        (v_1, \dotsc, v_n)
        =
        v_1 \tensor \dotsb \tensor v_n  \,.
      \]
      It follows in particular that~$\Tensor(V)$ is then generated by~$V$ as an algebra.
      The algebra~\gls*{tensor algebra} is the \defemph{tensor algebra of~$V$}
      
      We will more generally identify for all~$d \geq 0$ the tensor power~$V^{\tensor d}$ with the corresponding summand in~$\Tensor(V)$.
      The tensor algebra~$\Tensor(V)$ hence consists of linear combinations simple tensors~$v_1 \tensor \dotsb \tensor v_n$.
    
    \item[Universal Property:]
      The tensor algebra~$\Tensor(V)$ can be though of as the \enquote{free~{\algebra{$\kf$}} on~$V$}, in at least two ways:
      \begin{itemize}
        \item(Informal)
          The tensor algebra~$\Tensor(V)$ arises from~$V$ by starting with the elements of~$V$ and adding to~$V$ all kinds of expressions that can be constructed from the elements of~$V$ by algebra operations.
          But it follows from the axioms that many of these expressions have to be the same, so that we only end up with expressions of a certain form.
          
          Let us be a bit more explicit:
          Suppose that a~{\algebra{$\kf$}}~$A$ contains~$V$ as a linear subspace.
          Then it also contains products of the form~$v_1 \dotsm v_n$ with~$v_i \in V$ and hence sums of such products, i.e.\ elements of the form
          \[
            \sum_{i=k}^r v_{i_1} \dotsm v_{i_{n_k}}
          \]
          with~$r \geq 0$ and~$v_{ij} \in V$.
          If we continue to combine elements of this form with algebra operations then we do not gain any new elements, since by the axioms of an algebra they must already be of the above form.
          
          But in an arbitrary~{\algebra{$\kf$}} it may happen that some of these expressions are equal even though this does not follow pureley from the axioms of a~{\algebra{$\kf$}}.
          Consider for example the polynomial ring~$A = \kf[x, y]$ and the linear subspace~$V = \gen{x, y}_{\kf}$.
          It follows from the axioms of a~{\algebra{$\kf$}} that the expressions~$x (x+y)$ and~$x^2 + xy$ are the same, but it does not follow just from the axioms that~$xy = yx$, even though this holds in~$A$.
          There are hence certain additional \emph{relations} between the elements~$x$ and~$y$ of~$V$ in the ambient {\algebra{$\kf$}}~$A$.
          
          In the tensor algebra~$\Tensor(V)$ this does not happen:
          Whenever two expressions~$x$ and~$y$ that are built from elements of~$V$ via algebra operations coincide, then this equality can be derived from the algebra axioms alone.
          Hence there exist no additional relations between the elements of~$V$ in~$\Tensor(V)$.
          The only required condition is that~$V$ is a linear subspace of~$\Tensor(V)$, i.e.\ that addition and scalar multiplication in~$V$ does coincide with the one coming from~$\Tensor(V)$.
          
          The tensor algebra~$\Tensor(V)$ is in this way the \enquote{freest} way of expanding~$V$ into a~{\algebra{$\kf$}}.
        \item(Formal)
          Let~$\iota \colon V \to \Tensor(V)$ be the inclusion map, which is~{\linear{$\kf$}}.
          Then if~$A$ is any other~{\algebra{$\kf$}} and~$f \colon V \to A$ any~{\linear{$\kf$}} map (which one can think of as somewhat of an inclusion, albeit not injective), then~$f$ extends uniquely to an algebra homomorphism~$f^+ \colon \Tensor(V) \to A$, i.e.\ there exists a unique algebra homomorphism~$f^+ \colon \Tensor(V) \to A$ that makes the triangular diagram
          \[
            \begin{tikzcd}
              V
              \arrow{r}[above]{f}
              \arrow{d}[left]{i}
              &
              A
              \\
              \Tensor(V)
              \arrow[dashed]{ur}[below right]{f^+}
              &
              {}
            \end{tikzcd}
          \]
          commute.
          The algebra homomorphism~$f^+$ is given by
          \[
            f^+(v_1 \tensor \dotsb \tensor v_d)
            =
            f(v_1) \dotsm f(v_d)
          \]
          for all~$d \geq 0$ and simple tensors~$v_1 \tensor \dotsb \tensor v_d \in V^{\tensor d}$.
          This construction results in a {\onetoone} correspondence
          \begin{align*}
            \{ \text{\linear{$\kf$} maps~$V \to A$} \}
            &\longonetoone
            \{ \text{algebra homomorphisms~$\Tensor(V) \to A$} \} \,,
            \\
            f
            &\longmapsto
            f^+ \,,
            \\
            \restrict{F}{V}
            &\longmapsfrom
            F \,.
          \end{align*}
          Hence~$(\Tensor(V), i)$ is the \enquote{universal way} of mapping the vector space~$V$ into a~{\algebra{$\kf$}}.
          
          This formal explanation relates to the previous informal explanation in the following way:
          If~$A$ is any~{\algebra{$\kf$}} that contains~$V$ as a linear subspace then the inclusion~$V \to A$ extend uniquely to an algebra homomorphism~$\Tensor(V) \to A$.
          Every relation between expressions built from the elements of~$V$ that holds in~$\Tensor(V)$ must then also hold in~$A$.
          Therefore the only relations that hold in~$\Tensor(V)$ between such expressions are the one that hold in \emph{every}~{\algebra{$\kf$}} containing~$V$.
      \end{itemize}
      
    \item[Uniqueness]
      The above universal property determines the tensor algebra~$\Tensor(V)$ together with the inclusion~$i \colon V \to \Tensor(V)$ uniquely up to unique isomorphism, in the following sense:
      Let~$A$ be another~{\algebra{$\kf$}} and let~$j \colon V \to T$ be a~{\linear{$\kf$}} map such that for every~{\algebra{$\kf$}}~$A$ and every~{\linear{$\kf$}} map~$f \colon V \to A$ there exists a unique algebra homomorphism~$F \colon T \to A$ that makes the triangular diagram
      \[
        \begin{tikzcd}
          V
          \arrow{r}[above]{f}
          \arrow{d}[left]{j}
          &
          A
          \\
          T
          \arrow{ur}[below right]{F}
          &
          {}
        \end{tikzcd}
      \]
      commute.
      
      Then there exist unique algebra homomorphisms~$f \colon A \to T$ and~$g \colon T \to A$ that make the triangular diagrams
      \[
        \begin{tikzcd}[column sep = small]
          {}
          &
          V
          \arrow{dl}[above left]{i}
          \arrow{dr}[above right]{j}
          &
          {}
          \\
          \Tensor(V)
          \arrow[dashed]{rr}[below]{f}
          &
          {}
          &
          T
        \end{tikzcd}
        \qquad\text{and}\qquad
        \begin{tikzcd}[column sep = small]
          {}
          &
          V
          \arrow{dl}[above left]{j}
          \arrow{dr}[above right]{i}
          &
          {}
          \\
          T
          \arrow[dashed]{rr}[below]{g}
          &
          {}
          &
          \Tensor(V)
        \end{tikzcd}
      \]
      commute.
      It then follows that the compositions~$g \circ f \colon \Tensor(V) \to \Tensor(V)$ and~$f \circ g \colon T \to T$ make the triangular diagrams
      \[
        \begin{tikzcd}[column sep = small]
          {}
          &
          V
          \arrow{dl}[above left]{i}
          \arrow{dr}[above right]{i}
          &
          {}
          \\
          \Tensor(V)
          \arrow[dashed]{rr}[below]{g \circ f}
          &
          {}
          &
          \Tensor(V)
        \end{tikzcd}
        \qquad\text{and}\qquad
        \begin{tikzcd}[column sep = small]
          {}
          &
          V
          \arrow{dl}[above left]{j}
          \arrow{dr}[above right]{j}
          &
          {}
          \\
          T
          \arrow[dashed]{rr}[below]{f \circ g}
          &
          {}
          &
          T
        \end{tikzcd}
      \]
      commute.
      The algebra homomorphisms~$g \circ f$ and~$f \circ g$ are unique with this propert by the universal properties of~$(\Tensor(V), i)$ and~$(T, j)$.
      But the identities~$\id_{\Tensor(V)}$ and~$\id_T$ also make these diagrams commute.
      We therefore find that~$g \circ f = \id_{\Tensor(V)}$ and~$f \circ g = \id_{\Tensor(V)}$, so that~$f$ and~$g$ are mutually inverse algebra isomorphisms.
    
    \item[Functoriality:]
      If~$f \colon V \to W$ is any~{\linear{$\kf$}} map then we can consider the following diagram:
      \[
        \begin{tikzcd}
          V
          \arrow{r}[above]{f}
          \arrow{d}
          &
          W
          \arrow{d}
          \\
          \Tensor(V)
          &
          \Tensor(W)
        \end{tikzcd}
      \]
      By applying the universal property of the tensor algebra~$\Tensor(V)$ to the composition~$V \to W \to \Tensor(W)$ it follows that there exists a unique algebra homomorphism~$f_* \colon \Tensor(V) \to \Tensor(W)$ that makes the square diagram
      \[
        \begin{tikzcd}
          V
          \arrow{r}[above]{f}
          \arrow{d}
          &
          W
          \arrow{d}
          \\
          \Tensor(V)
          \arrow[dashed]{r}[below]{f_*}
          &
          \Tensor(W)
        \end{tikzcd}
      \]
      commute.
      This induced algebra homorphism is functorial in the following sense:
      \begin{itemize}
        \item
          It holds that~$(\id_V)_* = \id_{\Tensor(V)}$.
          Indeed, the commutativity of the square 
          \[
            \begin{tikzcd}[column sep = large]
              V
              \arrow{r}[above]{f}
              \arrow{d}
              &
              V
              \arrow{d}
              \\
              \Tensor(V)
              \arrow[dashed]{r}[below]{(\id_V)_*}
              &
              \Tensor(V)
            \end{tikzcd}
          \]
          shows that the identity~$\id_{\Tensor(V)}$ satifies the defining property of the induced algebra homomorphism~$(\id_V)_*$.
        \item
          It holds for all linear maps~$f \colon U \to V$ and~$g \colon V \to W$ that~$(g \circ f)_* = g_* \circ f_*$.
          Indeed, it follows from the commutativity of the diagram
          \[
            \begin{tikzcd}
              U
              \arrow[dashed, bend left=45]{rr}[above]{g \circ f}
              \arrow{r}[above]{f}
              \arrow{d}
              &
              V
              \arrow{r}[above]{g}
              \arrow{d}
              &
              W
              \arrow{d}
              \\
              \Tensor(U)
              \arrow{r}[below]{f_*}
              \arrow[dashed, bend right=45]{rr}[below]{g_* \circ f_*}
              &
              \Tensor(V)
              \arrow{r}[below]{g_*}
              &
              \Tensor(W)
            \end{tikzcd}
          \]
          that the subdiagram
          \[
            \begin{tikzcd}[column sep = large]
              U
              \arrow{r}[above]{g \circ f}
              \arrow{d}
              &
              W
              \arrow{d}
              \\
              \Tensor(U)
              \arrow[dashed]{r}[below]{g_* \circ f_*}
              &
              \Tensor(W)
            \end{tikzcd}
          \]
          commutes.
          This shows that the composition~$g_* \circ f_*$ satisfies the defining property of the induced algebra homomorphism~$(g \circ f)_*$.
      \end{itemize}
      
      This shows that the assignment~$V \mapsto \Tensor(V)$ extends to a (covariant) functor~$\Tensor \colon \cVect{\kf} \to \cAlg{\kf}$.
      The universal property of the tensor algebra states that the functor~$\Tensor$ is left adjoint to the forgetful functor~$\cAlg{\kf} \to \cVect{\kf}$ that assigns to each~{\algebra{$\kf$}} its underlying~{\vectorspace{$\kf$}}.
    
    \item[Description via a basis:]
      If a basis~$(v_i)_{i \in I}$ of~$V$ is choosen then every tensor power~$V^{\tensor d}$ inherits a basis that is given by all simple tensors
      \[
        v_{i_1} \tensor \dotsb \tensor v_{i_d}
      \]
      with~$i_1, \dotsc, i_d \in I$.
      It follows that the tensor power has as a basis of all such simple tensors with~$d \geq 0$ and~$i_{i_1, \dotsc, i_d} \in I$.
      The product of two such basis vectors is again a basis vector.
      So we may think about the basis vectors as finite words~$i_1 \dotsm i_d$ in the alphabet~$I$, and as the multiplication of two basis vectors as the concatenation of the corresponding words.
      
      If we think about the basis vector~$v_i$ of~$V$ as a formal variable~$X_i$ then we see that the tensor algebra~$\Tensor(V)$ is isomorphic to the noncommutative polynomial ring~$\kf\gen{X_i \suchthat i \in I}$.
      This noncommutative polynomial ring is also the free~{\algebra{$\kf$}} on the generators~$X_i$ with~$i \in I$, while~$V$ is the free~{\vectorspace{$\kf$}} on the letters~$i \in I$.
      This gives another explanation for why~$\Tensor(V)$ is the free~{\algebra{$\kf$}} on the vector space~$V$.
      More exicitely, we have the following commutative diagram of forgetful functors:
      \[
        \begin{tikzcd}
          \cVect{\kf}
          \arrow{d}
          &
          \cAlg{\kf}
          \arrow{l}
          \arrow{dl}
          \\
          \cSet
          &
          {}
        \end{tikzcd}
      \]
      It then follows that the resulting diagram of left adjoint functors
      \[
        \begin{tikzcd}
          \cVect{\kf}
          \arrow{r}[above]{\Tensor}
          &
          \cAlg{\kf}
          \\
          \cSet
          \arrow{u}[left]{F}
          \arrow{ur}[below right]{\kf\gen{X_i \suchthat i \in (-)}}
          &
          {}
        \end{tikzcd}
      \]
      commutes up to natural isomorphism.
      Hence~$\Tensor(V) \cong \Tensor(F(I)) \cong \kf\gen{X_i \suchthat i \in I}$.
  \end{description}
\end{recall}


\begin{recall}[Symmetric power]
  Let~$V$ be a vector space and let~$d \geq 0$.
  The~{\howmanyth{$d$}} \defemph{symmetric power}\index{symmetric!power}~$\Symm^d(V)$ is the quotient vector space of the tensor power~$\Symm^d(V)$ by the~{\linear{$\kf$}} subspace~$U_d$ that is generated by all all differences
  \[
      v_1 \tensor \dotsb \tensor v_d
    - v_{\sigma(1)} \tensor \dotsb \tensor v_{\sigma(d)}
  \]
  where~$v_1 \tensor \dotsb \tensor v_d \in V^{\tensor d}$ is a simple tensor and~$\sigma \in S_n$ is a permuation.
  Hence
  \begin{align*}
    \Symm^d(V)
    &=
    V^{\tensor d} / U_d
    \\
    &=
    V^{\tensor d}
    /
    \gen{
        v_1 \tensor \dotsb \tensor v_d
      - v_{\sigma(1)} \tensor \dotsb \tensor v_{\sigma(d)} 
    \suchthat
      v_1, \dotsc, v_n \in V,
      \sigma \in S_n
    }_{\kf} \,.
  \end{align*}
  Observe that~$\Symm^0(V) = V^{\tensor 0} = \kf$ because~$U_0 = 0$.
  For~$v_1, \dotsc, v_n \in V$ we denote the residue class of the simple tensor~$v_1 \tensor \dotsb \tensor v_d$ in~$\Symm^d(V)$ by~$v_1 \dotsm v_d$, and call this a \defemph{symmetric simple tensor}\index{symmetric!simple tensor}.
  
  We have by construction of~$\Symm^d(V)$ that
  \[
    v_1 \dotsm v_d
    =
    v_{\sigma(1)} \dotsm v_{\sigma(d)}
  \]
  for all simple symmetric tensors~$v_1 \dotsm v_n \in \Symm^d(V)$ and permutations~$\sigma \in S_d$, and for~$d \geq 1$ the symmetric power~$\Symm^d(V)$ is universal with this property in the following sense:
  The map
  \[
    V^{\times d}
    \to
    \Symm^d(V)  \,,
    \quad
    (v_1, \dotsc, v_d)
    \mapsto
    v_1 \dotsm v_d
  \]
  is symmetric and multilinear, and if~$f \colon V^{\times d} \to W$ is any other symmetric multilinear map into any vector space~$W$ then there exists a unique linear map~$g \colon \Symm^d(V) \to W$ that makes the triangular diagram
  \[
    \begin{tikzcd}
      V^{\times d}
      \arrow{d}
      \arrow{dr}[above right]{f}
      &
      {}
      \\
      \Symm^d(V)
      \arrow[dashed]{r}[below]{g}
      &
      W
    \end{tikzcd}
  \]
  commute.
  A linear map~$\Symm^d(V) \to W$ is in this sense the same as a symmetric bilinear map~$V^{\times d} \to W$.
  
  If~$(v_i)_{i \in I}$ is a basis of~$V$ such that~$(I, \leq)$ is a linearly ordered set then the ordered monomials
  \[
    v_{i_1} \dotsm v_{i_d}
    \qquad
    \text{with~$i_1 \leq \dotsb \leq i_d$}
  \]
  form a basis of the symmetric power~$\Symm^d(V)$.
  If~$V$ is of finite dimension~$n$ then it follows that
  \[
    \dim \Symm^d(V)
    =
    \binom{n+d-1}{d}  \,.
  \]
\end{recall}


\begin{recall}[Symmetric algebra]
  Let~$V$ be a vector space.
  Just as the tensor algebra~$\Tensor(V)$ is the free~{\algebra{$\kf$}} on~$V$ and can be constructed by using the tensor powers~$V^{\tensor d}$ we can use the symmetric powers~$\Symm^d(V)$ to construct the \defemph{symmetric algebra}\index{symmetric algebra}~\gls*{symmetric algebra}.
  The argumentation is analogous to that for the tensor algebra, so we will skip some of the details this time.
  
  \begin{description}
    \item[Construction:]
      For all~$v_1, \dotsc, v_d \in V$ we denote the corresponding simple symmetric tensor in~$\Symm^d(V)$ by~$v_1 \dotsm v_d$.
      We can define on~$\Symm(V) \defined \bigoplus_{d \geq 0} \Symm^d(V)$ a multiplication such that
      \[
        (v_1 \dotsm v_p) \cdot (v_{p+1} \dotsm v_{p+q})
        =
        v_1 \dotsm v_p v_{p+1} \dotsm v_{p+q}
      \]
      for all~$p, q \geq 0$ and all simple symmetric tensors~$v_1 \dotsm v_p \in \Symm^p(V)$ and~$v_{p+1}, \dotsc, v_{p+q} \in \Symm^q(V)$.
      By identifying~$\Symm^0(V)$ with the ground field~$\kf$ this makes~$\Symm(V)$ into an associative~{\algebra{$\kf$}}.
      This is already a commutative~{\algebra{$\kf$}} because
      \begin{align*}
        (v_1 \dotsm v_p) \cdot (v_{p+1} \dotsm v_{p+q})
        &=
        v_1 \dotsm v_p v_{p+1} \dotsm v_{p+q}
        \\
        &=
        v_{p+1} \dotsm v_{p+q} v_1 \dotsm v_p
        \\
        &=
        (v_{p+1} \dotsm v_{p+q}) \cdot (v_1 \dotsm v_p)
      \end{align*}
      for all~$p, q \geq 0$ and all simple symmetric tensors~$v_1 \dotsm v_p \in \Symm^p(V)$ and~$v_{p+1}, \dotsc, v_{p+q} \in \Symm^q(V)$. 
      We can identify~$V = \Symm^1(V)$ with the corresponding direct summand of~$\Symm(V)$, and more generally every symmetric power~$\Symm^d(V)$ with the corresponding direct summand of~$\Symm(V)$.
      The algebra~$\Symm(V)$ thus consists of linear combinations of simple symmetric tensors.
      
    \item[Universal property:]
      The symmetric algebra~$\Symm(V)$ is the \enquote{free commutative~{\algebra{$\kf$}}} on the vector space~$V$ in the following sense:
      If~$i \colon V \to \Symm(V)$ is the inclusion then there exists for every~{\algebra{$\kf$}}~$A$ and every linear map~$f \colon V \to A$ a unique algebra homomorphism~$f^+ \colon \Symm(V) \to A$ that makes the triangular diagram
      \[
        \begin{tikzcd}
          V
          \arrow{r}[above]{f}
          \arrow{d}[left]{i}
          &
          A
          \\
          \Symm(V)
          \arrow{ur}[below right]{f^+}
          &
          {}
        \end{tikzcd}
      \]
      commute.
      The algebra homomorphism~$f^+$ is given by
      \[
        f^+(v_1 \dotsm v_d)
        =
        f(v_1) \dotsm f(v_d)
      \]
      for all~$d \geq 0$ and simple symmetric tensors~$v_1, \dotsc, v_d \in V$.
      This construction results in a {\onetoone} correspondence
      \begin{align*}
        \{ \text{\linear{$\kf$} maps~$V \to A$} \}
        &\longonetoone
        \{ \text{algebra homomorphisms~$\Symm(V) \to A$} \} \,,
        \\
        f
        &\longmapsto
        f^+ \,,
        \\
        \restrict{F}{V}
        &\longmapsfrom
        F \,.
      \end{align*}
      
      It follows that a relations between elements of~$V$ holds in the symmetric algebra~$\Symm(V)$ if and only if it holds in every commutative algebra that contains~$V$.
      
    \item[Uniqueness]
      If~$S$ is a commutative~{\algebra{$\kf$}} and~$j \colon V \to S$ is a~{\linear{$\kf$}} such that~$(S, j)$ satisfies the same universal property as the symmetric algebra~$(\Symm(V), i)$  then there exists unique algebra homomorphisms~$f \colon \Symm(V) \to S$ and~$g \colon S \to \Symm(V)$ that make the triangular diagrams
      \[
        \begin{tikzcd}[column sep = small]
          {}
          &
          V
          \arrow{dl}[above left]{i}
          \arrow{dr}[above right]{j}
          &
          {}
          \\
          \Symm(V)
          \arrow[dashed]{rr}[below]{f}
          &
          {}
          &
          S
        \end{tikzcd}
        \qquad\text{and}\qquad
        \begin{tikzcd}[column sep = small]
          {}
          &
          V
          \arrow{dl}[above left]{j}
          \arrow{dr}[above right]{i}
          &
          {}
          \\
          S
          \arrow[dashed]{rr}[below]{g}
          &
          {}
          &
          \Symm(V)
        \end{tikzcd}
      \]
      commute.
      Then~$f$ and~$g$ are mutually inverse algebra isomorphisms.
      
    \item[Functoriality]
      For every linear map~$f \colon V \to W$ there exists a unique induced algebra homomorphism~$f_* \colon \Symm(V) \to \Symm(W)$ that makes the square diagram
      \[
        \begin{tikzcd}
          V
          \arrow{r}[above]{f}
          \arrow{d}
          &
          W
          \arrow{d}
          \\
          \Symm(V)
          \arrow[dashed]{r}[below]{f_*}
          &
          \Symm(W)
        \end{tikzcd}
      \]
      commmute.
      It holds that~$(\id_V)_* = \id_{\Symm(V)}$ and it holds for all composable~{\linear{$\kf$}} maps~$f \colon U \to V$ and~$g \colon V \to W$ that~$(g \circ f)_* = g_* \circ f_*$.
      This construction promotes the assignment~$V \mapsto \Symm(V)$ to a (covariant) functor~$\Symm \colon \cVect{\kf} \to \cCAlg{\kf}$, where~$\cCAlg{\kf}$ denotes the category of commutative~{\algebras{$\kf$}}.
      
    \item[Description via a basis]
      If~$(v_i)_{i \in I}$ is a basis of$~V$ where~$(I, \leq)$ is a linearly ordered set then the symmetric power~$\Symm^d(V)$ inherits a basis that is given by all simple symmetric tensors
      \[
        v_{i_1} \dotsm v_{i_d}
        \qquad
        \text{where~$i_1 \leq \dotsb \leq i_d$} \,.
      \]
      It follows that the symmetric algebra~$\Symm(V)$ has as a basis all simple symmetric tensors~$v_{i_1} \dotsm v_{i_d}$ with~$d \geq 0$ and~$i_1, \dotsc, i_d \in I$ with~$i_1 \leq \dotsb \leq i_d$.
      This basis may also be written as
      \[
        v_{i_1}^{\nu_1} \dotsm v_{i_r}^{\nu_r}
      \]
      with~$r \geq 0$,~$i_1, \dotsc, i_r \in I$ such that~$i_1 < \dotsb < i_r$ and~$\nu_1, \dotsc, \nu_r \geq 0$ (which is connected to the above description via~$d = \nu_1 + \dotsb + \nu_r$).
      
      We see from this description that the symmetric algebra~$\Symm(V)$ is isomorphic to the commutative polynomial ring~$\kf[X_i \suchthat i \in I]$, which is the free commutative~{\algebra{$\kf$}} on the generators~$i \in I$
      This can again be explained by considering the commutative diagram of forgetful functors
      \[
        \begin{tikzcd}
          \cVect{\kf}
          \arrow{d}
          &
          \cCAlg{\kf}
          \arrow{l}
          \arrow{dl}
          \\
          \cSet
          &
          {}
        \end{tikzcd}
      \]
      from which we see that the resulting diagram of left adjoints
      \[
        \begin{tikzcd}
          \cVect{\kf}
          \arrow{r}[above]{\Symm}
          &
          \cCAlg{\kf}
          \\
          \cSet
          \arrow{u}[left]{F}
          \arrow{ur}[below right]{\kf[X_i \suchthat i \in (-)]}
          &
          {}
        \end{tikzcd}
      \]
      commutes up to natural isomorphism.
      
    \item[Contruction via the tensor algbra]
      The symmetric algebra~$\Symm(V)$ can also be constructed as a quotient of the tensor algebra~$\Tensor(V)$.
      We give multiple ways how to see and think about this.
      Let in the following~$i \colon V \to \Tensor(V)$ and~$j \colon V \to \Symm(V)$ denote the 
      \begin{itemize}
        \item
          Let~$I$ be the commutator ideal of~$\Tensor(V)$, i.e.\ the two-sided ideal generated by all commutators
          \[
            x \tensor y - y \tensor x
          \]
          with~$x, y \in \Tensor(V)$.
          Let~$\pi \colon \Tensor(V) \to \Tensor(V)/I$ be the canonical projection.
          Then the quotient~$\Tensor(V)/I$ is commutative, and hence there exists by the universal property of the symmetric algebra a unique algebra homomorphism~$f \colon \Symm(V) \to \Tensor/I$ that makes the diagram
          \[
            \begin{tikzcd}
              {}
              &
              V
              \arrow[bend right]{ddl}[above left]{i}
              \arrow[bend left]{dr}[above right]{j}
              &
              {}
              \\
              {}
              &
              {}
              &
              \Tensor(V)
              \arrow{d}[right]{\pi}
              \\
              \Symm(V)
              \arrow[dashed]{rr}[above]{f}
              &
              {}
              &
              \Tensor(V)/I
            \end{tikzcd}
          \]
          commute.
          The homomorphism~$f$ is on the genareting set~$V$ of~$\Symm(V)$ given by~$f(v) = \class{v}$.
          On the other hand we get from the universal property of the tensor algebra~$\Tensor(V)$ a unique algebra homomorphism~$\tilde{g} \colon \Tensor(V) \to \Symm(V)$ that makes the diagram
          \[
            \begin{tikzcd}
              {}
              &
              V
              \arrow[bend right]{dl}[above left]{j}
              \arrow[bend left]{ddr}[above right]{i}
              &
              {}
              \\
              \Tensor(V)
              \arrow[bend left, dashed]{drr}[above right]{\tilde{g}}
              \arrow{d}[left]{\pi}
              &
              {}
              &
              {}
              \\
              \Tensor(V)/I
              &
              {}
              &
              \Symm(V)
            \end{tikzcd}
          \]
          commute.
          The commutator~$I$ is contained in the kernel of~$\tilde{g}$ because the algebra~$\Symm(V)$ is commutative.
          Hence there exists a unique algebra homomorphism~$g \colon \Tensor(V)/I \to \Symm(V)$ that makes the diagram
          \[
            \begin{tikzcd}
              {}
              &
              V
              \arrow[bend right]{dl}[above left]{j}
              \arrow[bend left]{ddr}[above right]{i}
              &
              {}
              \\
              \Tensor(V)
              \arrow[bend left]{drr}[above right]{\tilde{g}}
              \arrow{d}[left]{\pi}
              &
              {}
              &
              {}
              \\
              \Tensor(V)/I
              \arrow[dashed]{rr}[below]{g}
              &
              {}
              &
              \Symm(V)
            \end{tikzcd}
          \]
          commute.
          The algebra homomorphism~$g$ is given on the generators~$\class{v}$ with~$v \in V$ of~$\tensor(V)/I$ given by~$g(\class{v}) = v$.
          
          It follows from the explicit descriptions of~$f$ and~$g$ on generators that their are mutually inverse algebra isomorphisms.
          Thus~$\Symm(V) \cong \Tensor(V)/I$ via the isomorphism~$f$.
          
          Observe also that the commutator ideal~$I$ is already generated by the commutators~$v \tensor w - w \tensor v$ with~$v, w \in V$.
          Indeed, the ideal~$J$ generated by these elements is contained in~$I$.
          But on the other hand the quotient~$\Tensor(V)/J$ is already commutative because it is generated by the residue classes~$\class{v}$ with~$v \in V$, all of which commute with each other.
          The commutator ideal~$I$ is therefore contained in the kernel of the canonical projection~$\Tensor(V) \to \Tensor(V)/J$, i.e.\ it is containted in~$J$.
          
        \item
          The above argumentatio is not surprising if we remember that~$\Tensor(V)$ is the universal~{\algebra{$\kf$}} on~$V$ and that quotiening out the commutator ideal is the universal way of making an algebra commutative.
          The quotient~$\Tensor(V)/I$ therefore ought to be the universal commutative~{\algebra{$\kf$}}.
          
          This motivation can be formalized by observing that the diagram of forgetful functors
          \[
            \begin{tikzcd}
              \cAlg{\kf}
              \arrow{d}
              &
              \cCAlg{\kf}
              \arrow{l}
              \arrow{dl}
              \\
              \cVect{\kf}
              &
              {}
            \end{tikzcd}
          \]
          commutes.
          It follows that the resulting diagram of left adjoints
          \[
            \begin{tikzcd}
              \cAlg{\kf}
              \arrow{r}[above]{C}
              &
              \cCAlg{\kf}
              \\
              \cVect{\kf}
              \arrow{u}[left]{\Tensor}
              \arrow{ur}[below right]{\Symm}
              &
              {}
            \end{tikzcd}
          \]
          commutes up to natural isomorphism.
          The adjoint~$C$ of the forgetful functor~$\cCAlg{\kf} \to \cAlg{\kf}$ is given by quotiening out the commutator ideal, and hence~$\Symm(V) \cong \Tensor(V)/I$ as before.
        \item
          The above argumentation be also expressed by observing that for every commutative~{\algebra{$\kf$}}~$A$ there exist natural bijections
          \begin{align*}
            {}&
            \{ \text{algebra homomorphisms~$\Symm(V) \to A$} \}
            \\
            \cong{}&
            \{ \text{{\linear{$\kf$}} maps~$V \to A$} \}
            \\
            \cong{}&
            \{ \text{algebra homomorphisms~$\Tensor(V) \to A$} \}
            \\
            \cong{}&
            \{ \text{algebra homomorphisms~$\Tensor(V)/I \to A$} \} \,,
          \end{align*}
          where the last bijection uses that the algebra~$A$ is commutative and therefore every algebra homomorphism~$\Tensor(V) \to A$ contains the commutator ideal~$I$ in its kernel.
          It now follows from Yoneda’s~lemma that~$\Symm(V) \cong \Tensor(V)/I$.
      \end{itemize}
  \end{description}
\end{recall}


\begin{remark}  % TODO: Fix glossary and bigwedge
  One can similarly construct the \emph{exterior algebra}~$\gls*{exterior algebra} = \bigoplus_{d \geq 0} \Exterior^d(V)$ of a vector space~$V$ by replacing the use of the tensor powers~$V^{\tensor d}$ or symmetric powers~$\Symm^d(V)$ by the exterior powers~$\Exterior^d(V)$.
  For any other~{\algebra{$\kf$}}~$A$ an algebra homomorphism~$F \colon \Exterior(V) \to A$ is then the same as a~{\linear{$\kf$}}~$f \colon V \to A$ with~$f(v)^2 = 0$ for every~$v \in V$.
% TODO: Do we have to worry about char(k) = 2?
  It thus follows from a similar argumentation as for the symmetric algebra that~$\Exterior(V) \cong \Tensor(V)/I$ for the two-sided ideal~$I$ in~$\Tensor(V)$ generated by all~$v \tensor v$ with~$v \in V$.
  
  
  If~$V$ is finite dimensional then the exterior algebra~$\Exterior(V)$ is again finite dimensional, namely with~$\dim \Exterior(V) = 2^{\dim V}$.
  This is different to both the tensor algebra~$\Tensor(V)$ and symmetric algebra~$\Symm(V)$, which are infinite dimensional whenever~$V \neq 0$.
\end{remark}






\section{Definition and Construction of the UEA}


\begin{definition}
  A \defemph{universal enveloping algebra}\index{universal enveloping algebra} of a Lie~algebra~$\glie$ is a~\algebra{$\kf$}~\gls*{universal enveloping algebra} together with a homomorphism of Lie~algebras~$\iota \colon \glie \to \Univ(\glie)$ such that for every other~{\algebra{$\kf$}}~$A$ and homomorphism of Lie~algebras~$\phi \colon \glie \to A$ there exists a unique homomorphism of~\algebras{$\kf$}~$\Phi \colon \Univ(\glie) \to A$ that makes the triangular diagram
  \[
    \begin{tikzcd}
      \glie
      \arrow{r}[above]{\phi}
      \arrow{d}[left]{\iota}
      &
      A
      \\
      \Univ(\glie)
      \arrow[dashed]{ur}[below right]{\Phi}
      &
      {}
    \end{tikzcd}
  \]
  commute, i.e.\ such that~$\phi = \Phi \circ \iota$.
\end{definition}


\begin{remark}[Uniqueness of universal enveloping algebras]
  \label{uniqueness of universal enveloping algebras}
  Suppose that a Lie~algebra~$\glie$ admits two {\uas}~$(\Univ(\glie)_1, \iota_1)$ and~$(\Univ(\glie)_2, \iota_2)$.
  Then there exists unique algebra homomorphisms~$\phi \colon \Univ(\glie)_1 \to \Univ(\glie)_2$ and~$\psi \colon \Univ(\glie)_2 \to \Univ(\glie)_1$ that make the triangular diagrams
  \[
    \begin{tikzcd}[column sep = small]
      {}
      &
      \glie
      \arrow{dl}[above left]{\iota_1}
      \arrow{dr}[above right]{\iota_2}
      &
      {}
      \\
      \Univ(\glie)_1
      \arrow[dashed]{rr}[below]{\phi}
      &
      {}
      &
      \Univ(\glie)_2
    \end{tikzcd}
    \qquad\text{and}\qquad
    \begin{tikzcd}[column sep = small]
      {}
      &
      \glie
      \arrow{dl}[above left]{\iota_1}
      \arrow{dr}[above right]{\iota_2}
      &
      {}
      \\
      \Univ(\glie)_2
      \arrow[dashed]{rr}[below]{\phi}
      &
      {}
      &
      \Univ(\glie)_1
    \end{tikzcd}
  \]
  commute.
  It follows that the compositions~$\phi \circ \psi \colon \Univ(\glie)_1 \to \Univ(\glie)_1$ and~$\psi \circ \phi \colon \Univ(\glie)_2 \to \Univ(\glie)_2$ make the triangular diagrams
  \[
    \begin{tikzcd}[column sep = small]
      {}
      &
      \glie
      \arrow{dl}[above left]{\iota_1}
      \arrow{dr}[above right]{\iota_1}
      &
      {}
      \\
      \Univ(\glie)_1
      \arrow[dashed]{rr}[below]{\psi \circ \phi}
      &
      {}
      &
      \Univ(\glie)_1
    \end{tikzcd}
    \qquad\text{and}\qquad
    \begin{tikzcd}[column sep = small]
      {}
      &
      \glie
      \arrow{dl}[above left]{\iota_2}
      \arrow{dr}[above right]{\iota_2}
      &
      {}
      \\
      \Univ(\glie)_2
      \arrow[dashed]{rr}[below]{\phi \circ \psi}
      &
      {}
      &
      \Univ(\glie)_2
    \end{tikzcd}
  \]
  commutes.
  The algebra homomorphisms~$\phi \circ \psi$ and~$\psi \circ \phi$ are unique with this property by the universal property of the {\uas}~$(\Univ(\glie)_1, \iota_1)$ and~$(\Univ(\glie)_2, \iota_2)$.
  But the identities~$\id_{\Univ(\glie)_1}$ and~$\id_{\Univ(\glie)_2}$ also makes these diagrams commute.
  We thus find that~$\psi \circ \phi = \id_{\Univ(\glie)_1}$ and~$\phi \circ \psi = \id_{\Univ(\glie)_2}$, so that~$\phi$ and~$\psi$ are mutually inverse isomorphisms of~{\algebras{$\kf$}}.
  
  This shows that the {\ua} (if it exists) is \enquote{unique up to unique isomorphisms}.
  We will therefore talk about \emph{the} {\ua} of~$\glie$.
  We will often also surpress the algebra homorphism~$\iota \colon \glie \to \Univ(\glie)$ from our notation.
\end{remark}


% \begin{remark}
%   One can also formulate the above argument is a more categorical way:
%   Consider the category~$\catC$ where
%   \begin{itemize}
%     \item
%       objects of~$\catC$ is a pairs~$(A, i)$ consisting of a~{\algebra{$\kf$}}~$A$ and a Lie~algebra homomorphism~$i \colon \glie \to A$,
%     \item
%       a morphism~$\phi \colon (A, i) \to (B, j)$ is an algebra homomorphism~$\phi \colon A \to B$ that makes the triangular diagram
%       \[
%         \begin{tikzcd}[column sep = small]
%         {}
%         &
%         \glie
%         \arrow{dl}[above left]{i}
%         \arrow{dr}[above right]{j}
%         &
%         {}
%         \\
%         A
%         \arrow[dashed]{rr}[below]{\phi}
%         &
%         {}
%         &
%         B
%       \end{tikzcd}
%     \]
%       commute, and
%     \item
%       the composition of two morphisms is just their usual set-theoretic composition.
%   \end{itemize}
%   A {\ua} of~$\glie$ is nothing but an inital object in this category~$\catC$.
%   The argumentation from \cref{uniqueness of universal enveloping algebras} is then the usual argument for the uniqueness of inital objects up to unique isomorphism.
% \end{remark}


\begin{proposition}
  \label{representations are modules}
  Let~$V$ be a~{\vectorspace{$\kf$}}.
  Let~$\glie$ be a Lie~algebra and let~$\iota \colon \glie \to \Univ(\glie)$ be the canonical Lie~algebra homomorphism.
  Then the assignments
  \begin{align*}
    \left\{
    \begin{tabular}{@{}c@{}}
      representations of~$\glie$, \\
      $\rho \colon \glie \to \gllie(V)$
    \end{tabular}
    \right\}
    &\longonetoone
    \left\{
    \begin{tabular}{@{}c@{}}
      $\Univ(\glie)$-module structures \\
      $\theta \colon \Univ(\glie) \to \End_{\kf}(V)$
    \end{tabular}
    \right\}  \,,
    \\
    \rho
    &\longmapsto
    \hat{\rho} \,,
    \\
    \theta \circ \iota
    &\longmapsfrom
    \theta  \,,
  \end{align*}
  constitute a {\onetoone} correspondence,~where $\hat{\rho} \colon \Univ(\glie) \to \End_{\kf}(V)$ is the unique~\algebra{$\kf$} homomorphism induced by the homomorphism of Lie~algebras~$\rho \colon \glie \to \gllie(V)$ via the universal property of the~{\ua}~$\Univ(\glie)$.
  \qed
\end{proposition}


\begin{remark}
  \Cref{representations are modules} shows that representations of~$\glie$ are the same as~{\modules{$\Univ(\glie)$}}.
  The categories~$\cRep{\glie}$ and~$\cMod{\Univ(\glie)}$ are hence isomorphic.
\end{remark}


\begin{lemma}[Functoriality of the universal enveloping algebra]
  \label{functoriality of universal enveloping algebra}
  Let~$\glie$,~$\hlie$ and~$\klie$ be Lie~algebras.
  \begin{enumerate}
    \item
      For every homomorphism of Lie~algebras~$\phi \colon \glie \to \hlie$ there exists a unique induced homomorphism of~\algebras{$\kf$}~$\phi^* \colon \Univ(\glie) \to \Univ(\hlie)$ that makes the following square diagram commute:
      \[
        \begin{tikzcd}[column sep = large]
          \glie
          \arrow{r}[above]{\phi}
          \arrow{d}
          &
          \hlie
          \arrow{d}
          \\
          \Univ(\glie)
          \arrow[dashed]{r}[below]{\phi_*}
          &
          \Univ(\hlie)
        \end{tikzcd}
      \]
    \item
      It holds that~$(\id_{\glie})_* = \id_{\Univ(\glie)}$.
    \item
      It holds for all composable homomorphisms of Lie~algebras~$\phi \colon \glie \to \hlie$ and~$\psi \colon \hlie \to \klie$ that
      \[
        (\psi \circ \phi)_*
        =
        \psi_* \circ \phi_* \,.
      \]
  \end{enumerate}
\end{lemma}


\begin{proof}
  \leavevmode
  \begin{enumerate}
    \item
      This follows from the universal property of the universal enveloping algebra~$\Univ(\glie)$ by applying it to the composition~$\glie \to \hlie \to \Univ(\hlie)$.
    \item
      The square diagram
      \[
        \begin{tikzcd}[column sep = huge]
          \glie
          \arrow{r}[above]{\id_{\glie}}
          \arrow{d}
          &
          \glie
          \arrow{d}
          \\
          \Univ(\glie)
          \arrow[dashed]{r}[below]{\id_{\Univ(\glie)}}
          &
          \Univ(\glie)
        \end{tikzcd}
      \]
      commutes, which shows that~$\id_{\Univ(\glie)}$ satisfies the defining property of the induced algebra homomorphism~$(\id_{\glie})_*$.
    \item
      We have the following commutative diagram:
      \[
        \begin{tikzcd}[column sep = large]
          \glie
          \arrow[dashed, bend left = 40]{rr}[above]{\psi \circ \phi}
          \arrow{r}[above]{\phi}
          \arrow{d}
          &
          \hlie
          \arrow{r}[above]{\psi}
          \arrow{d}
          &
          \klie
          \arrow{d}
          \\
          \Univ(\glie)
          \arrow{r}[below]{\phi_*}
          \arrow[dashed, bend right = 40]{rr}[below]{\psi_* \circ \phi_*}
          &
          \Univ(\hlie)
          \arrow{r}[below]{\psi_*}
          &
          \Univ(\klie)
        \end{tikzcd}
      \]
      The commutativity of the outer square diagram
      \[
        \begin{tikzcd}[column sep = huge]
          \glie
          \arrow{r}[above]{\psi \circ \phi}
          \arrow{d}
          &
          \klie
          \arrow{d}
          \\
          \Univ(\glie)
          \arrow[dashed]{r}[below]{\psi_* \circ \phi_*}
          &
          \Univ(\klie)
        \end{tikzcd}
      \]
      shows that~$\psi_* \circ \phi_*$ satisfies the defining property of the induced algebra homomorphism~$(\psi \circ \phi)_*$.
    \qedhere
  \end{enumerate}
\end{proof}


\begin{remark}
  \Cref{functoriality of universal enveloping algebra} shows that the assignment~$\glie \mapsto \Univ(\glie)$ of a Lie~algebra~$\glie$ to its universal eveloping algebra~$\Univ(\glie)$ can be extended to a (covariant) functor~$\Univ \colon \cLie{\kf} \to \cAlg{\kf}$.
  The universal property of the {\ua} states that the functor~$\Univ$ is left adjoint to the forgetful functor~$\cAlg{\kf} \to \cLie{\kf}$ that assigns to each~{\algebra{$\kf$}} its underlying Lie~algebra.
\end{remark}


\begin{remark}
  Let~$\glie$ be a Lie~algebra.
  It follows from the universal propery of the {\ua} that~$\Univ(\glie)$ is generated by the image of the canonical homomorphism~$\iota \colon \glie \to \Univ(\glie)$:
  
  Indeed, let~$U$ be the subalgebra of~$\Univ(\glie)$ that is generated by the image of~$\iota$, and let~$i \colon \glie \to U$ be the restriction of~$\iota$.
  Then for every~{\algebra{$\kf$}}~$A$ and every Lie~algebra homomorphism~$\phi \colon \glie \to A$ the induced algebra homomorphism~$\Phi' \colon \Univ(\glie) \to A$ restricts to an algebra homomorphism~$\Phi \colon U \to A$ that makes the triangular diagram
  \[
    \begin{tikzcd}
      \glie
      \arrow{r}[above]{\phi}
      \arrow{d}[left]{j}
      &
      A
      \\
      U
      \arrow[dashed]{ur}[below right]{\Phi}
      &
      {}
    \end{tikzcd}
  \]
  commute.
  The homomorphism~$\Phi$ is unique with this property because it is uniquely determined by the restriction~$\Phi \circ j = \phi$, since~$U$ is generated by the image of~$j$.
  This shows that~$(U, j)$ is again a {\ua} for~$\glie$.
  
  It follows from the uniqueness of the {\ua}, as discussed in \cref{uniqueness of universal enveloping algebras}, that the unique algebra homomorphism~$i \colon U \to \Univ(\glie)$ that makes the triangular diagram
  \[
    \begin{tikzcd}[column sep = small]
      {}
      &
      \glie
      \arrow{dl}[above left]{j}
      \arrow{dr}[above right]{\iota}
      &
      {}
      \\
      U
      \arrow[dashed]{rr}[below]{i}
      &
      {}
      &
      \Univ(\glie)
    \end{tikzcd}
  \]
  commute is already an isomorphism.
  This homomorphism is necessarily the inclusion of~$U$ into~$\Univ(\glie)$, and that it is an isomorphism means that already~$U = \Univ(\glie)$.
  
  The canonical homomorphism~$\iota \colon \glie \to \Univ(\glie)$ is in particular~{\linear{$\kf$}} and hence induces an algebra homomorphism~$\phi \colon \Tensor(\glie) \to \Univ(\glie)$ that makes the triangular diagram
  \[
    \begin{tikzcd}[column sep = small]
      {}
      &
      \glie
      \arrow{dl}
      \arrow{dr}[above right]{\iota}
      &
      {}
      \\
      \Tensor(\glie)
      \arrow[dashed]{rr}[below]{\phi}
      &
      {}
      &
      \Univ(\glie)
    \end{tikzcd}
  \]
  commute.
  That~$\Univ(\glie)$ is generated by the image of~$\iota$ means that~$\phi$ is surjective.
  Therefore~$\phi$ induces an algebra isomorphism
  \[
    \Phi
    \colon
    \Tensor(\glie)/I
    \to
    \Univ(\glie)
  \]
  for~$I = \ker \phi$, that makes the resulting diagram
   \[
    \begin{tikzcd}[column sep = small]
      {}
      &
      \glie
      \arrow[bend right]{dl}
      \arrow[bend left]{ddr}[above right]{\iota}
      &
      {}
      \\
      \Tensor(\glie)
      \arrow[bend left]{drr}[below left]{\phi}
      \arrow{d}
      &
      {}
      &
      {}
      \\
      \Tensor(\glie)/I
      \arrow[dashed]{rr}[below]{\Phi}
      &
      {}
      &
      \Univ(\glie)
    \end{tikzcd}
  \]
  commute.
  
  If~$A$ is any other~{\algebra{$\kf$}} then we know on the one hand that algebra homorphisms~$\Univ(\glie) \to A$ correspond to Lie~algebra homomorphisms~$\glie \to A$.
  We know find on the other hand that~{\algebra{$\kf$}} homomorphisms~$\Tensor(V)/I \to A$ correspond to algebra homomorphisms~$\Tensor(V) \to A$ that annihilate~$I$, with the algebra homorphisms~$\Tensor(V) \to A$ corresponding to~{\linear{$\kf$}} maps~$\glie \to A$.
  A linear map~$f \colon \glie \to A$ is a homorphism of Lie~algebras if and only if
  \[
      f(x)f(y)
    - f(y)f(x)
    - f([x,y])
    =
    0 \,,
  \]
  so it seems reasonable to assume that the corresponding ideal~$I$ of~$\Tensor(V)$ ought to be given by
  \[
    I
    =
    (x \tensor y - y \tensor x - [x,y] \suchthat x, y \in \glie)
  \]
  We will now show that this is indeed the case.
\end{remark}


\begin{proposition}[Existence of the universal enveloping algebra]
  Let~$\glie$ be a Lie~algebra.
  Let~$\Tensor(\glie)$ be the tensor algebra of the underlying vector space of~$\glie$ and let~$I$ the two-sided ideal in~$\Tensor(\glie)$ generated by the elements $x \tensor y - y \tensor x - [x,y]$ with~$x,y \in \glie$.
  The the quotient algebra~$U \defined T(\glie)/I$ together with the~{\linear{$\kf$}} map
  \[
    i
    \colon
    \glie
    \to
    \Univ(\glie) \,,
    \quad
    x
    \mapsto
    \class{x}
  \]
  is a {\ua} for~$\glie$.
\end{proposition}


\begin{proof}
  The map~$i$ is~{\linear{$\kf$}} and it compatible with the Lie brackets because
  \[
    [i(x), i(y)]
    =
    [\class{x}, \class{y}]
    =
    \class{x} \, \class{y} - \class{y} \, \class{x}
    =
    \class{x \tensor y - y \tensor x}
    =
    \class{[x,y]}
    =
    i([x,y]) \,.
  \]
  Given any~{\algebra{$\kf$}}~$A$ and Lie algebra homomorphism~$\phi \colon \glie \to A$ there exists a unique homorphism of~{\algebras{$\kf$}}~$\Phi' \colon \Tensor(\glie) \to A$ that makes the triangular diagram
  \[
    \begin{tikzcd}
      \glie
      \arrow{r}[above]{\phi}
      \arrow{d}[left]{\iota}
      &
      A
      \\
      \Tensor(V)
      \arrow[dashed]{ur}[below right]{\Phi'}
      &
      {}
    \end{tikzcd}
  \]
  commute, where~$\iota \colon \glie \to \Tensor(\glie)$ is the inclusion.
  The homomorphism~$\Phi'$ is given by~$\Phi'(x) = \phi(x)$ for every~$x \in \glie$.
  It follows that
  \begin{align*}
    \Phi'(x \tensor y - y \tensor x)
    &=
    \Phi'(x \tensor y) - \Phi'(y \tensor x)
    \\
    &=
    \Phi'(x) \Phi'(y) - \Phi'(y) \Phi'(x)
    \\
    &=
    \phi(x) \phi(y) - \phi(y) \phi(x)
    \\
    &=
    [\phi(x), \phi(y)]
    \\
    &=
    \phi([x,y])
    \\
    &=
    \Phi'([x,y])
  \end{align*}
  for all~$x, y \in \glie$, so that the ideal~$I$ is contained in the kernel of~$\Phi'$.
  It follows that there exists a unique algebra homomorphism~$\Phi \colon \Tensor(\glie)/I \to A$ that makes the triangular diagram
  \[
    \begin{tikzcd}
      \glie
      \arrow{r}[above]{\phi}
      \arrow{d}[left]{\iota}
      \arrow[bend right = 55]{dd}[left]{i}
      &
      A
      \\
      \Tensor(\glie)
      \arrow[bend right= 20]{ur}[above left]{\Phi'}
      \arrow{d}[left]{\pi}
      &
      {}
      \\
      \Tensor(\glie)/I
      \arrow[dashed, bend right = 30]{uur}[below right]{\Phi}
      &
      {}
    \end{tikzcd}
  \]
  commute, where~$\pi \colon \Tensor(V) \to \Tensor(V)/I$ denotes the canonical projection.
  Then the subdiagram
  \[
    \begin{tikzcd}
      \glie
      \arrow{r}[above]{\phi}
      \arrow{d}[left]{i}
      &
      A
      \\
      \Tensor(\glie)/I
      \arrow{ur}[below right]{\Phi}
      &
      {}
    \end{tikzcd}
  \]
  commutes.
  That~$\Phi$ is unique with this property follows from the uniqueness of~$\Phi'$.
\end{proof}


\begin{remark}
  The above proof may be reorganized by observing that we have bijections
  \begin{align*}
    {}&
    \{ \text{algebra homomorphisms~$\Phi \colon \Tensor(\glie)/I \to A$} \}
    \\
    \cong{}&
    \{ \text{algebra homomorphisms~$\Phi' \colon \Tensor(\glie) \to A$ with~$\Phi'(I) = 0$} \}
    \\
    \cong{}&
    \left\{
      \begin{tabular}{@{}c@{}}
        algebra homomorphisms~$\Phi' \colon \Tensor(\glie) \to A$ with  \\
        $\Phi'(x \tensor y - y \tensor x - [x,y]) = 0$ for all~$x, y \in \glie$
      \end{tabular}
    \right\}
    \\
    \cong{}&
    \left\{
      \begin{tabular}{@{}c@{}}
        algebra homomorphisms~$\Phi' \colon \Tensor(\glie) \to A$ with  \\
        $\Phi'(x) \Phi'(y) - \Phi'(y) \Phi'(x) - \Phi'([x,y]) = 0$ for all~$x, y \in \glie$
      \end{tabular}
    \right\}
    \\
    \cong{}&
    \left\{
      \begin{tabular}{@{}c@{}}
        algebra homomorphisms~$\Phi' \colon \Tensor(\glie) \to A$ with  \\
        $\Phi'(x) \Phi'(y) - \Phi'(y) \Phi'(x) = \Phi'([x,y])$ for all~$x, y \in \glie$
      \end{tabular}
    \right\}
    \\
    \cong{}&
    \left\{
      \begin{tabular}{@{}c@{}}
        {\linear{$\kf$}} maps~$\phi \colon \glie \to A$ with  \\
        $\phi(x) \phi(y) - \phi(y) \phi(x) = \phi([x,y])$ for all~$x, y \in \glie$
      \end{tabular}
    \right\}
    \\
    \cong{}&
    \left\{
      \begin{tabular}{@{}c@{}}
        {\linear{$\kf$}} maps~$\phi \colon \glie \to A$ with  \\
        $[\phi(x), \phi(y)] = \phi([x,y])$ for all~$x, y \in \glie$ 
      \end{tabular}
    \right\}
    \\
    ={}&
    \{ \text{Lie~algebra homomorphisms~$\phi \colon \glie \to A$} \}
  \end{align*}
  that are natural in~$A$.
  This shows that the~{\algebra{$\kf$}}~$\Tensor(\glie)/I$ represented the right kind of functor;
  and the identity~$\Tensor(\glie)/I \to \Tensor(\glie)/I$ corresponds under the above bijections to the map~$\iota \colon \glie \to \Tensor(\glie)/I$ as desired.
\end{remark}


\begin{examples}
  Let~$\glie$ be an abelian Lie~algebra.
  It follows from the explicit construction of the universal enveloping algebra~$\Univ(\glie)$ that
  \[
    \Univ(\glie)
    \cong
    \Tensor(\glie)/(x \tensor y - y \tensor x \suchthat x, y \in \glie)
    \cong
    \Symm(\glie)
  \]
  with the canonical Lie~algebra homomorphism~$\glie \to \Univ(\glie)$ corresponding to the inclusion~$\glie \to \Symm(\glie)$.
  This can also be seen more abstractly:
  
  We observe that if~$V$ is a vector space and~$A$ is an algebra then a linear map~$f \colon V \to A$ extends to an algebra homomorphism~$\Symm(V) \to A$ (necessarily uniquely) if and only if the image of~$f$ is contained in a commutative subalgebra of~$A$, if and only if the image of~$f$ is commutative.
  It hence follows that for any~{\algebra{$\kf$}} we have bijections
  \begin{align*}
    {}&
    \{ \text{Lie~algebra homomorphisms~$\glie \to A$} \}
    \\
    \cong{}&
    \{ \text{{\linear{$\kf$}} maps~$\glie \to A$ with commutative image} \}
    \\
    \cong{}&
    \{ \text{algebra homomorphisms~$\Symm(\glie) \to A$} \}
  \end{align*}
  that are natural in~$A$.
  This shows that the symmetric algebra~$\Symm(\glie)$ together with the inclusion~$\glie \to \Symm(\glie)$ satisfies the universal property of the universal enveloping algebra of~$\glie$.
\end{examples}


\begin{recall}
  \label{homomorphism out of a tensor product}
  Let~$A$ and~$B$ be two~{\algebras{$\kf$}}.
  Then the inclusions~$i \colon A \to A \tensor B$ and~$j \colon B \to A \tensor B$ are injective algebra homomorphisms.
  We may therefore identify~$A$ and~$B$ with the subalgebras~$A \tensor 1$ and~$1 \tensor B$ of~$A \tensor B$.
  Note that~$A$ and~$B$ commute in~$A \tensor B$ because
  \[
    i(a) j(b)
    =
    (a \tensor 1) (b \tensor 1)
    =
    a \tensor b
    =
    (b \tensor 1) (a \tensor 1)
    =
    j(b) i(a)
  \]
  for all~$a \in A$ and~$b \in B$.
  
  Let~$C$ be another~{\algebra{$\kf$}}.
  
  If~$f \colon A \tensor B \to C$ is an algebra homomorphism then the restrictions~$\phi = f \circ i$ and~$\psi = f \circ j$ are again algebra homomorphisms~$\phi \colon A \to C$ and~$\psi \colon B \to C$.
  The images of~$\phi$ and~$\psi$ in~$C$ commute with each other because~$A$ and~$B$ commute in~$A \tensor B$.
  More explicitely,
  \begin{align*}
    \phi(a) \psi(b)
    &=
    f(a \tensor 1) f(1 \tensor b)
    \\
    &=
    f((a \tensor 1) (1 \tensor b))
    \\
    &=
    f(a \tensor b)
    \\
    &=
    f((1 \tensor b) (a \tensor 1))
    \\
    &=
    f(1 \tensor b) f(a \tensor 1)
    \\
    &=
    \psi(b) \phi(a)
  \end{align*}
  for all~$a \in A$ and~$b \in B$.
  
  If on the other hand~$\phi \colon A \to C$ and~$\psi \colon B \to C$ are two algebra homomorphisms whose images commute with each other then the map
  \[
    f'
    \colon
    A \times B
    \to
    C \,,
    \quad
    (a,b)
    \mapsto
    \phi(a) \psi(b)
  \]
  is~{\bilinear{$\kf$}} and hence induces a~{\linear{$\kf$}} map
  \[
    f
    \colon
    A \tensor B
    \to
    C \,,
    \quad
    a \tensor b
    \mapsto
    \phi(a) \psi(b) \,.
  \]
  The map~$f$ is again an algebra homomorphism because
  \begin{align*}
    f(a_1 \tensor b_1) f(a_2 \tensor b_2)
    &=
    \phi(a_1) \psi(b_1) \phi(a_2) \psi(b_2)
    \\
    &=
    \phi(a_1) \phi(a_2) \psi(b_1) \psi(b_2)
    \\
    &=
    \phi(a_1 a_2) \psi(b_1 b_2)
    \\
    &=
    f( (a_1 a_2) \tensor (b_1 b_2) )
    \\
    &=
    f( (a_1 \tensor b_1) (a_2 \tensor b_2) )
  \end{align*}
  for all simple tensors~$a \tensor b \in A \tensor B$.
  
  These constructions are mutually inverse and hence result in a {\onetoone} correspondence
  \begin{align*}
    \left\{
      \begin{tabular}{@{}c@{}}
        algebra homomorphisms \\
        $f \colon A \tensor B \to C$
      \end{tabular}
    \right\}
    &\longonetoone
    \left\{
      (\phi, \psi)
    \suchthat*
      \begin{tabular}{@{}c@{}}
        algebra homomorphisms \\
        $\phi \colon A \to C$ and~$\psi \colon B \to C$ \\
        whose images commute with each other
      \end{tabular}
    \right\}  \,,
    \\
    f
    &\longmapsto
    (f \circ i, f \circ j)  \,,
    \\
    \biggl( a \tensor b \mapsto \phi(a)\psi(b) \biggr)
    &\longmapsfrom
    (\phi, \psi)  \,.
  \end{align*}
\end{recall}


\begin{example}
  If~$\glie$ and~$\hlie$ be two Lie~algebras then
  \[
    \Univ(\glie \times \hlie)
    \cong
    \Univ(\glie) \tensor \Univ(\hlie) \,.
  \]
  This can be seen in various ways:
  \begin{itemize}
    \item
      It follows from \cref{homomorphism out of a product} and \cref{homomorphism out of a tensor product} that we get for every~{\algebra{$\kf$}}~$A$ bijections
      \begin{align*}
        {}&
        \{ \text{algebra homomorphisms~$F \colon \Univ(\glie \times \hlie) \to A$} \}
        \\
        \cong{}&
        \{ \text{Lie~algebra homomorphisms~$f \colon \glie \times \hlie \to A$} \}
        \\
        \cong{}&
        \left\{
          \begin{tabular}{@{}c@{}}
            Lie~algebra homomorphisms \\
            $\phi \colon \glie \to A$ and~$\psi \colon \hlie \to A$ \\
            whose images commute with each other
          \end{tabular}
        \right\}
        \\
        \cong{}&
        \left\{
          \begin{tabular}{@{}c@{}}
            algebra homomorphisms \\
            $\Phi \colon \Univ(\glie) \to A$ and~$\Psi \colon \Univ(\hlie) \to A$ \\
            whose images commute with each other
          \end{tabular}
        \right\}
        \\
        \cong{}&
        \{ \text{algebra homomorphims~$F \colon \Univ(\glie) \tensor \Univ(\hlie) \to A$} \}  \,.
      \end{align*}
      The claimed isomorphism therefore follows from Yoneda’s lemma.
    \item
      More explicitely let~$i \colon \glie \to \glie \times \hlie$ and~$j \colon \hlie \to \glie \times \hlie$ be the canonical inclusions.
      Then for the induced algebra homomorphisms~$i_* \colon \Univ(\glie) \to \Univ(\glie \times \hlie)$ and~$j_* \colon \Univ(\hlie) \to \Univ(\glie \times \hlie)$ the images of~$i_*$ of~$j_*$ commute with each other.
      Indeed, the algebra~$\Univ(\glie \times \hlie)$ is generated by the image of~$\glie \times \hlie$ in~$\Univ(\glie \times \hlie)$, and~$i(\glie)$ and~$j(\hlie)$ commute in~$\glie \times \hlie$.
      Therefore
      \[
        i_*(\class{x}) j_*(\class{y})
        =
        \class{(x,0)} \class{(0,y)}
        =
        \class{(0,y)} \class{(x,0)}
        =
        j_*(\class{y}) i_*(\class{x}) \,,
      \]
      for all~$x \in \glie$ and~$y \in \hlie$, where we denote by~$\class{(-)}$ the corresponding elements of the {\uas}.
      We used for the middle equality that~$(x,0)$ and~$(0,y)$ commute in~$\glie \times \hlie$ and hence also in~$\Univ(\glie \times \hlie)$.
      
      It follows that~$i_*$ and~$j_*$ induce a common algebra homomorphism
      \[
        \phi
        \colon
        \Univ(\glie) \tensor \Univ(\hlie)
        \to
        \Univ(\glie \times \hlie) \,,
      \]
      that is given on elements by
      \[
        \phi(x \tensor y)
        =
        i_*(x) j_*(y)
      \]
      for all simple tensors~$x \tensor y \in \Univ(\glie) \tensor \Univ(\hlie)$.
      It holds in particular for all~$x \in \glie$ and~$y \in \glie$ that
      \[
        \phi(\class{x} \tensor \class{y})
        =
        i_*(\class{x}) j_*(\class{y})
        =
        \class{i(x)} \cdot \class{j(y)}
        =
        \class{(x,0)} \cdot \class{(0,y)}  \,,
      \]
      We observe that the map
      \[
        \psi'
        \colon
        \glie \times \hlie
        \to
        \Univ(\glie) \tensor \Univ(\hlie) \,,
        \quad
        (x,y)
        \mapsto
        \class{(x,0)} \tensor 1 + 1 \tensor \class{(0,y)} \,.
      \]
      To construct the inverse~$\psi$ of~$\phi$ we observe that the equalities
      \[
        \psi( \class{(x,0)} )
        =
        \psi( \class{(x,0)} \cdot 1 )
        =
        \psi( i_*(\class{x}) \cdot j_*(1) )
        =
        \psi( \phi(\class{x} \tensor 1) )
        =
        \class{x} \tensor 1
      \]
      and similarly~$\psi( \class{(0,y)} ) = 1 \tensor \class{y}$ have to hold for all~$x \in \glie$ and~$y \in \glie$.
      It then follows that more generally
      \[
        \psi( \class{(x,y)} )
        =
        \psi( \class{(x,0)} + \class{(0,y)} )
        =
        \class{x} \tensor 1 + 1 \tensor \class{y}
      \]
      for all~$(x,y) \in \glie \times \hlie$.
      
      Motivated by these observations we consider the map
      \[
        \psi'
        \colon
        \glie \times \hlie
        \to
        \Univ(\glie) \tensor \Univ(\hlie)
      \]
      that is given by
      \[
        \psi'((x,y))
        =
        \class{x} \tensor 1 + 1 \tensor \class{y} \,.
      \]
      This map is~{\linear{$\kf$}} and it is a homomorphism of Lie~algebras because
      \begin{align*}
        {}&
        [\psi'((x_1, y_1)), \psi'((x_2, y_2))]
        \\
        ={}&
        [
          \class{x_1} \tensor 1 + 1 \tensor \class{y_1},
          \class{x_2} \tensor 1 + 1 \tensor \class{y_2}
        ]
        \\
        ={}&
          [\class{x_1} \tensor 1, \class{x_2} \tensor 1]
        + \underbrace{ [\class{x_1} \tensor 1, 1 \tensor \class{y_2}] }_{=0}
        + \underbrace{ [1 \tensor \class{y_1}, \class{x_2} \tensor 1] }_{=0}
        + [1 \tensor \class{y_1}, 1 \tensor \class{y_2}]
        \\
        ={}&
          [\class{x_1}, \class{x_2}] \tensor 1
        + 1 \tensor [\class{y_1}, \class{y_2}]
        \\
        ={}&
          \class{[x_1, x_2]} \tensor 1
        + 1 \tensor \class{[y_1, y_2]}  \,.
        \\
        ={}&
        \psi'( ( [x_1, x_2], [y_1, y_2] ) )
        \\
        ={}&
        \psi'( [(x_1, y_1), (x_2, y_2)] ) \,.
      \end{align*}
      It hence follows from the universal property of the {\ua}~$\Univ(\glie \times \hlie)$ that there exists a unique algebra homomorphism~$\psi \colon \Univ(\glie \times \hlie) \to \Univ(\glie) \tensor \Univ(\hlie)$ that makes the triangular diagram
      \[
        \begin{tikzcd}[row sep = large]
          \glie \times \hlie
          \arrow{r}[above]{\psi'}
          \arrow{d}[left]{\class{(-)}}
          &
          \Univ(\glie) \tensor \Univ(\hlie)
          \\
          \Univ(\glie \times \hlie)
          \arrow[dashed]{ur}[below right]{\psi}
          &
          {}
        \end{tikzcd}
      \]
      commute.
      This homomorphism~$\psi$ is given for all~$(x, y) \in \glie \times \hlie$ by
      \[
        \psi( \class{(x,y)} )
        =
        \class{x} \tensor 1 + 1 \tensor \class{y} \,.
      \]
      
      The homomorphisms~$\phi$ and~$\psi$ are mutually inverse:
      It sufficies to check this on the generators~$\class{x} \tensor \class{y}$ of~$\Univ(\glie) \tensor \Univ(\hlie)$ with~$x \in \glie$ and~$y \in \hlie$, and on the generators~$\class{(x,y)}$ of~$\Univ(\glie \times \hlie)$ with~$(x,y) \in \glie \times \hlie$.
      And indeed, we find that
      \begin{align*}
        \phi( \psi( \class{(x,y)} ) )
        &=
        \phi( \class{x} \tensor 1 + 1 \tensor \class{y} )
        \\
        &=
        \phi( \class{x} \tensor 1 ) + \phi( 1 \tensor \class{y} )
        \\
        &=
        i_*(\class{x}) j_*(1) + i_*(1) j_*(\class{y})
        \\
        &=
        \class{i(x)} + \class{j(y)}
        \\
        &=
        \class{(x,0)} + \class{(0,y)}
        \\
        &=
        \class{(x,y)}
      \end{align*}
      and
      \begin{align*}
        \psi( \phi( \class{x} \tensor \class{y} ) )
        &=
        \psi( i_*(\class{x}) j_*(\class{y}) )
        \\
        &=
        \psi( \class{(x,0)} \class{(0,y)} )
        \\
        &=
        \psi( \class{(x,0)} ) \psi( \class{(0,y)} )
        \\
        &=
        ( \class{x} \tensor 1 + 1 \tensor 0 ) ( 0 \tensor 1 + 1 \tensor \class{y} )
        \\
        &=
        (\class{x} \tensor 1) (1 \tensor \class{y})
        \\
        &=
        \class{x} \tensor \class{y} \,.
      \end{align*}
  \end{itemize}
\end{example}


% WRONG PLACE
% 
% \begin{corollary}
%  The homomorphism $\iota \colon \glie \to \Univ(\glie)$ is injective. As a \algebra{$\kf$} $\Univ(\glie)$ is generated by $\iota(\g)$.
% \end{corollary}
% 
% 
% \begin{remark}
%  We will always identify $\glie$ with its image under $\iota$.
% \end{remark}
% 
% 
% 
% 





\section{Poincar\'{e}--Birkhoff--Witt}





\subsection{Graded \texorpdfstring{$\kf$}{k}-Algebras}


\begin{definition}
  \label{graded algebras}
  A \defemph{grading}\index{grading!of an algebra} or \defemph{gradation} of a~\algebra{$\kf$}~$A$ is a direct sum decomposition
  \[
    A
    =
    \bigoplus_{i \geq 0} A_i
  \]
  into linear subspaces~$A_i$ such that
  \[
    A_i A_j
    \subseteq
    A_{i+j}
  \]
  for all~$i, j \geq 0$.
  A \defemph{graded~\algebra{$\kf$}}\index{graded!algebra} is a~\algebra{$\kf$}~$A$ together with a grading~$A = \bigoplus_{i \geq 0} A_i$.
\end{definition}


\begin{remark}
  We will often just say that~$A$ is a graded algebra without mentioning the grading.
\end{remark}


\begin{remark}
  Given any abelian semigroup~$(S, +)$ an~\defemph{\grading{$S$}}\index{grading} of a~\algebra{$\kf$}~$A$ is a decomposition
  \[
    A
    =
    \bigoplus_{s \in S} A_s
  \]
  into linear subspaces such that
  \[
    A_s A_t
    \subseteq
    A_{s + t}
  \]
  for all~$s,t \in S$.
  An~{\graded{$S$}}~\algebra{$\kf$} is a~\algebra{$\kf$}~$A$ together with an~\grading{$S$}~$A = \bigoplus_{s \in S} A_s$.
  A graded~\algebra{$\kf$} in the sense of \cref{graded algebras} is then the special case of an~{\graded{$\Natural$}}~\algebra{$\kf$}.
  We will restrict our attention to~{\gradings{$\Natural$}} and refer to \cite[II.{\S}11, III.{\S}3]{bourbaki_algebra} for more general gradings.
\end{remark}


\begin{definition}
  Let~$A$ be a graded~{\algebra{$\kf$}} with grading~$A = \bigoplus_{i \geq 0} A_i$.
  \begin{enumerate}
    \item
      An element~$x \in A$ is \defemph{homogeneous}\index{homogeneous!element} of \defemph{degree}~$i$\index{homogeneous!degree}\index{degree!homogeneous} if~$x \in A_i$.
    \item
      If the decomposition of~$x \in A$ with respect to the grading~$A = \bigoplus_{i \geq 0} A_i$ is given by~$x = \sum_{i \geq 0} x_i$ then the summands~$x_i$ are the~\defemph{homogeneous components}\index{homogeneous!component} of~$x$.
  \end{enumerate}
\end{definition}


\begin{lemma}
  If~$A$ is a graded~\algebra{$\kf$} then~$1_A$ is homogeneous of degree~$0$.
\end{lemma}


\begin{proof}
  Let~$1 = \sum_{i \geq 0} e_i$ with respect to the grading~$A = \bigoplus_{i \geq 0} A_i$.
  Then for any~$j \geq 0$ and~$a \in A_j$ we have that
  \[
    a
    =
    1 \cdot a
    =
    \left( \sum_{i \geq 0} e_i \right) \cdot a
    =
    \sum_{i \geq 0} \underbrace{ e_i a }_{\in A_{i+j}}  \,,
  \]
  and it follows from the directness of the decomposition~$A = \bigoplus_{i \geq 0} A_i$ that already~$a = e_0 a$.
  It follows that~$e_0 a = a$ for every~$a \in A$, so that already~$1 = e_0$.
\end{proof}


\begin{corollary}
  If~$A$ is a graded~\algebra{$\kf$} then~$A_0$ is a~{\subalgebra{$\kf$}} of~$A$.
  \qed
\end{corollary}



\begin{examples}
  \label{examples of graded algebras}
  \leavevmode
  \begin{enumerate}
    \item
      Any~\algebra{$\kf$}~$A$ becomes a graded~\algebra{$\kf$} by setting~$A_0 \defined A$ and~$A_i \defined 0$ for every~$i \geq 1$.
    \item
      The polynomial ring~$A = \kf[x_i \suchthat i \in I]$ is a graded~\algebra{$\kf$} by setting
      \[
        A_d
        \defined
        \gen{
          x_{i_1}^{p_1} \dotsm x_{i_n}^{p_n}
        \suchthat
          n \geq 0,
          i_1, \dotsc, i_n \in I,
          p_1 + \dotsb + p_n = d
        }_{\kf}
      \]
      for every~$d \geq 0$.
      Then~$A_d$ consists of the homogeneous polynomials of degree~$d$, and the monomials~$x_{i_1}^{p_1} \dotsm x_{i_n}^{p_n}$ are homogeneous of degree~$p_1 + \dotsb + p_n$.
      
      We can more generally put the variable~$x_i$ is any degree~$d_i$:
      Given any family~$(d_i)_{i \in I}$ of natural numbers~$d_i \geq 0$ we can define a grading on~$A$ via
      \[
        A_d
        \defined
        \gen{
          x_{i_1}^{p_1} \dotsm x_{i_n}^{p_n}
        \suchthat
          n \geq 0,
          i_1, \dotsc, i_n \in I,
          p_1 d_1 + \dotsb + p_n d_n = d
        }_\kf
      \]
      for every~$d \geq 0$.
      Then the monomials~$x_{i_1}^{p_1} \dotsm x_{i_n}^{p_n}$ are homogeneous of degree~$p_1 d_1 + \dotsb + p_n d_n$.
    \item
      Similarly the free~{\algebra{$\kf$}}~$A \defined \kf\gen{x_i \suchthat i \in I}$ can be graded via
      \[
        A_d
        \defined
        \gen{
          x_{i_1}^{p_1} \dotsm x_{i_n}^{p_n}
        \suchthat
          n \geq 0,
          i_1, \dotsc, i_n \in I,
          p_1 + \dotsb + p_n = d
        }_{\kf}
      \]
      for every~$d \geq 0$, and for any family~$(d_i)_{i \in I}$ of natural numbers $d_i \geq 0$ via
      \[
        A_d
        \defined
        \gen{
          x_{i_1}^{p_1} \dotsm x_{i_n}^{p_n}
        \suchthat
          n \geq 0,
          i_1, \dotsc, i_n \in I,
          p_1 d_1 + \dotsb + p_n d_n = d
        }_{\kf} \,.
      \]
      This grading makes the monomials~$x_{i_1}^{p_1} \dotsm x_{i_n}^{p_n}$ homogeneous of degree~$p_1 d_1 + \dotsb + d_n p_n$.
    \item
      The tensor algebra~$\Tensor(V)$ of a vector space~$V$ has a canonical grading with~$\Tensor(V)_d = V^{\tensor d}$ for all~$d \geq 0$.
      Similarly both the symmetric algebra~$\Symm(V)$ and the exterior algebra~$\Exterior(V)$ have canonical gradings given by~$\Symm(V)_d = \Symm^d(V)$ and~$\Exterior(V)_d = \Exterior^d(V)$ for all~$d \geq 0$.
    \item
      Let~$(M, \cdot)$ be a monoid and let~$M = \coprod_{i \geq 0} M_i$ be a grading of~$M$, i.e.\ a decomposition into subsets with~$M_i \cdot M_j \subseteq M_{i+j}$ for all~$i, j \geq 0$.
      Then the monoid algebra~$\kf[M]$ inhericts a grading via
      \[
        \kf[M]_i
        \defined
        \gen{ M_i }_{\kf}
      \]
      for every~$i \geq 0$.
      As special cases of this construction we get the following examples:
      \begin{itemize}
        \item
          If~$I$ is any index set then for~$M = \Natural^{\oplus I}$ a grading~$M = \coprod_{d \geq 0} M_d$ is given by the subsets
          \[
            M_d
            \defined
            \left\{
              (p_i)_{i \in I} \in M
            \suchthat*
              \sum_{i \in I} p_i = d
            \right\}
          \]
          with~$d \geq 0$.
          Then~$\kf[M]$ with the induced grading is the commutative polynomial algebra~$\kf[x_i \suchthat i \in I]$.
        \item
          If~$I$ is any index set then let~$\Sigma \defined \{x_i \suchthat i \in I\}$ be an alphabet with letters~$x_i$ and let~$M = \Sigma^*$ be the monoid of words in the alphabet~$\Sigma$ together with concatenation of words as rule of composition.
          Then~$\kf[M] = \kf\gen{X_i \suchthat i \in I}$ and the grading of~$\kf[M]$ (where every variable~$x_i$ is in degree~$1$) comes from the grading of~$M$ given by
          \[
            M_d
            \defined
            \{
              w \in M
            \suchthat
              \text{$w$ is a word in~$\Sigma$ of length~$d$}
            \}  \,.
          \]
      \end{itemize}
  \end{enumerate}
\end{examples}


\begin{remark}
  \label{external description of graded algebras}
  The grading of the tensor algebra~$\Tensor(V)$, symmetric algebra~$\Symm(V)$ and exterior algebra~$\Exterior(V)$ are basically built into the construction of these algebras.
  This way of constructing graded~{\algebras{$\kf$}} can be generalized as follows:
  
  Suppose that we are given a sequence of vector spaces~$A_i$ with~$i \geq 0$ and bilinear maps
  \[
    \mu_{i,j}
    \colon
    A_i \times A_j
    \to
    A_{i+j} \,,
    \quad
    (x,y)
    \mapsto
    xy
  \]
  for all~$i, j \geq 0$ such that
  \begin{itemize}
    \item
      the~$\mu_{i, j}$ are relatively associative in the sense that
      \[
        x(yz)
        =
        (xy)z
      \]
      for all~$i, j, k \geq 0$ and~$x \in A_i$,~$y \in A_j$ and~$z \in A_k$ and
    \item
      there exists an element~$1 \in A_0$ with
      \[
        1 x
        =
        x
        =
        x 1
      \]
      for all~$i \geq 0$ and every~$x \in A_i$.
  \end{itemize}
  For the direct sum~$A \defined \bigoplus_{i \geq 0} A_i$ we can then fit together the multiplications~$\mu_{i,j}$ to a single multiplication~$\mu \colon A \times A \to A$ that is given on elements~$x, y \in A$ with~$x = (x_i)_{i \geq 0}$ and~$y = (y_i)_{i \geq 0}$ by
  \[
    x y
    =
    \left( \sum_{j=0}^i x_j y_{i-j} \right)_{j \geq 0}  \,.
  \]
  The bilinearity of~$\mu$ follows from the bilinearities of the~$\mu_{i,j}$, it follows from the relative associativity of the multiplications~$\mu_{i,j}$ that the multiplication~$\mu$ is associative, and~$1 \in A_0$ is multiplicative neutral for~$\mu$.
  
  This construction gives an external description of graded~{\algebras{$\kf$}}, in constrast to the previous internal description.
\end{remark}


\begin{definition}
  Let~$A$ and~$B$ be two graded~\algebras{$\kf$}.
  \begin{enumerate}
    \item
      A \defemph{homomorphism of graded~{\algebras{$\kf$}}}~$f \colon A \to B$\index{homomorphism!of graded $\kf$-algebras} is a homomorphism of~{\algebras{$\kf$}} such that~$f(A_i) \subseteq B_i$ for all~$i \geq 0$.
    \item
      If~$f \colon A \to B$ is a homomorphism of graded~{\algebras{$\kf$}} then for every~$i \geq 0$ the restriction of~$f$ to a linear map~$A_i \to B_i$ is denoted by~$f_i$.
  \end{enumerate}
\end{definition}


\begin{remark}
  \leavevmode
  \begin{enumerate}
    \item
      Let~$A$,~$B$ and~$C$ be graded~{\algebra{$\kf$}}.
      \begin{enumerate}
        \item
          The identity~$\id_A \colon A \to A$ is a homomorphism of graded~{\algebras{$\kf$}}.
        \item
          If~$f \colon A \to B$ and~$g \colon B \to C$ are homomorphisms of graded~{\algebras{$\kf$}} then their composition~$g \circ f \colon A \to C$ is again a homomorphism of graded~{\algebras{$\kf$}}.
      \end{enumerate}
      Hence the graded~\algebras{$\kf$} together with the homomorphisms of graded~\algebras{$\kf$} between them form a category, which we will denote by~\gls*{graded algebras}.
    \item
      For a homomorphism of graded~{\algebras{$\kf$}}~$f \colon A \to B$ the following conditions are equivalent:
      \begin{equivalenceslist}
        \item
          $f$ is an isomorphism of graded~{\algebras{$\kf$}}, i.e.\ there exists an inverse homomorphism of graded~{\algebras{$\kf$}}~$g \colon B \to A$ with~$fg = \id_B$ and~$gf = \id_A$.
        \item
          $f$ is bijective.
        \item
          For every~$i \geq 0$ the restriction~$f_i \colon A_i \to B_i$ is bijective.
      \end{equivalenceslist}
  \end{enumerate}
\end{remark}


\begin{example}
  \leavevmode
  \begin{enumerate}
    \item
      For any vector space~$V$ the two maps
      \begin{alignat*}{2}
        \Tensor(V)
        &\to
        \Symm(V) \,,
        &
        \quad
        x_1 \tensor \dotsb \tensor x_n
        &\mapsto
        x_1 \dotsm x_n
      \shortintertext{and}
        \Tensor(V)
        &\to
        \Exterior(V) \,,
        &
        \quad
        x_1 \tensor \dotsb \tensor x_n
        &\mapsto
        x_1 \wedge \dotsb \wedge x_n
      \end{alignat*}
      are homomorphisms of graded~\algebras{$\kf$}.
    \item
      If~$V$ is a finite dimensional vector space with basis~$x_1, \dotsc, x_n$ then the resulting isomorphism of~\algebras{$\kf$}
      \[
        \kf[X_1, \dotsc, X_n]
        \to
        \Symm(V) \,,
        \quad
        X_i
        \mapsto
        x_i
      \]
      is already an isomorphism of graded~\algebras{$\kf$}.
  \end{enumerate}
\end{example}


\begin{lemma}
  \label{characterizations of homogeneous ideals}
  Let~$I$ be some kind of ideal in a graded~{\algebra{$\kf$}} (i.e.\ a left ideal, right ideal or two-sided ideal).
  \begin{enumerate}
    \item
      The linear subspace~$\bigoplus_{i \geq 0} (I \cap A_i)$ is again an ideal in~$A$ of the same kind.
    \item
      The following conditions on~$I$ are equivalent:
      \begin{enumerate}
        \item
          \label{direct sum of linear subspaces}
          There exists linear subspaces~$I_i$ of~$A_i$ with~$i \geq 0$ such that~$I = \bigoplus_{i \geq 0} I_i$.
        \item
          \label{direct sum of intersections}
          It holds that~$I = \bigoplus_{i \geq 0} (I \cap A_i)$.
        \item
          \label{contains all homogeneous components}
          The ideal~$I$ contains the homogeneous components of all its elements.
        \item
          \label{generated by homogeneous}
          The ideal~$I$ is generated by homogeneous elements.
      \end{enumerate}
  \end{enumerate}
\end{lemma}


\begin{proof}
  \leavevmode
  \begin{enumerate}
    \item
      We check that~$I' \defined \bigoplus_{i \geq 0} I \cap A_i$ is again a left ideal if~$I$ is one.
      Indeed, we find that
      \begin{align*}
        A \cdot I'
        &=
              \left( \sum_{j \geq 0} A_j \right)
        \cdot \left( \sum_{i \geq 0} (I \cap A_i) \right)
        \\
        &=
        \sum_{i,j \geq 0} A_j \cdot (I \cap A_i)
        \\
        &\subseteq
        \sum_{i, j \geq 0} (A_j \cdot I) \cap (A_j \cdot A_i)
        \\
        &\subseteq
        \sum_{i,j \geq 0} (I \cap A_{i+j})
        \\
        &=
        \sum_{k \geq 0} (I \cap A_k)
        \\
        &=
        I'  \,.
      \end{align*}
      We find in the same way that~$I'$ is again a right ideal if~$I$ is one.
      It follows from these two cases that~$I'$ is a two-sided ideal if~$I$ is one.
    \item
      \begin{implicationlist}
        \item[\ref*{direct sum of linear subspaces}~$\implies$~\ref*{direct sum of intersections}]
          It follows from the assumption that~$I_i = I \cap A_i$.
        \item[\ref*{direct sum of intersections}~$\implies$~\ref*{direct sum of linear subspaces}]
          We may take~$I_i = I \cap A_i$.
        \item[\ref*{direct sum of linear subspaces}~$\implies$~\ref*{contains all homogeneous components}]
          We may decompose~$x \in I$ with respect to the decomposition~$A = \bigoplus_{i \geq 0} A_i$ into homogeneuos components~$x = \sum_{i \geq 0} x_i$ and with respect to the decomposition~$I = \bigoplus_{i \geq 0} I_i$ as~$x = \sum_{i \geq 0} x'_i$.
          Then~$x'_i \in I_i \subseteq A_i$ and hence~$x_i = x'_i$ by uniqueness of the decomposition in~$A$.
          Thus~$x_i = x'_i \in I_i \subseteq I$ for all~$i \geq 0$.
        \item[\ref*{contains all homogeneous components}~$\implies$~\ref*{generated by homogeneous}]
          We may start with any generating set for~$I$ and then replace each generator by all of its homogeneous components.
        \item[\ref*{generated by homogeneous}~$\implies$~\ref*{direct sum of intersections}]
          Each homogeneous generator in contained in some summand~$I \cap A_i$.
          Hence~$I$ is contained in the ideal~$\bigoplus_{i \geq 0} I \cap A_i$ which is in turn contained in~$I$.
          Hence~$I = \bigoplus_{i \geq 0} I \cap A_i$.
        \qedhere
      \end{implicationlist}
  \end{enumerate}
\end{proof}


\begin{definition}
  An ideal~$I$ (of any kind) in a graded~{\algebra{$\kf$}}~$A$ is \defemph{homogeneous ideal}\index{homogeneous!ideal} if it satisfies the equivalent conditions from \cref{characterizations of homogeneous ideals}.
  We set~$I_i \defined I \cap A_i$ for all~$i \geq 0$.
\end{definition}


\begin{example}
  If~$f \colon A \to B$ is a homomorphism of graded~{\algebras{$\kf$}} then~$\ker f$ is a homogeneous two-sided ideal in~$A$ because~$\ker f = \bigoplus_{i \geq 0} \ker f_i$.
  The converse also holds:
\end{example}


\begin{proposition}
  Let~$A$ be a graded~{\algebra{$\kf$}} and let~$I$ be a two-sided homogeneous ideal in~$A$.
  Let~$\pi \colon A \to A/I$ be the canonical projection.
  \begin{enumerate}
    \item
      The algebra~$A/I$ carries a grading via~$(A/I)_i \defined \pi(A_i)$ for every~$i \geq 0$, making~$A/I$ into a graded~{\algebra{$\kf$}}.
    \item
      This is the unique grading on~$A/I$ that makes~$\pi$ a homomorphism of graded~{\algebras{$\kf$}}.
  \end{enumerate}
\end{proposition}


\begin{proof}
  \leavevmode
  \begin{enumerate}
    \item
      It follows from the usual isomorphism
      \[
        A/I
        =
        \left( \bigoplus_{i \geq 0} A_i \right)
        \bigg/
        \left(\bigoplus_{i \geq 0} I_i \right)
        \cong
        \bigoplus_{i \geq 0} (A_i / I_i)
      \]
      that~$A/I = \bigoplus_{i \in I} (A/I)_i$.
      It holds for all~$i, j \geq 0$ that
      \[
        (A/I)_i (A/I)_j
        =
        \pi(A_i) \pi(A_j)
        =
        \pi(A_i A_j)
        \subseteq
        \pi(A_{i+j})
        =
        (A/I)_{i+j} \,.
      \]
    \item
      Any such grading~$A/I = \bigoplus_{i \geq 0} (A/I)_i$ must satisfy~$\pi(A_i) \subseteq (A/I)_i$ for every~$i \geq 0$.
      It follows from~$A/I = \bigoplus_{i \geq 0} \pi(A_i)$ and~$A/I = \bigoplus_{i \geq 0} (A/I)_i$ that~$\pi(A_i) = (A/I)_i$ for every~$i \geq 0$.
    \qedhere
  \end{enumerate}
\end{proof}


\begin{examples}
  Let~$V$ be a vector space.
  The two-sided ideal~$I$ in~$\Tensor(V)$ generated by the elements~$x \tensor y - y \tensor x$ with~$x,y \in V$ is a homogeneous ideal of~$\Tensor(V)$ because it is generated by homogeneous components.
  The two sided ideal~$J$ in~$\Tensor(V)$ generated by the elements~$x \tensor x$ with~$x \in V$ is homogeneous because it generated by homogeneous elements.
  The resulting quotient graded~{\algebras{$\kf$}} are the symmetric algebra~$\Symm(V) \cong A/I$ and the exterior algebra~$\Exterior(V) \cong A/J$.
\end{examples}


% Grading is on the wrong level.
% 
% \begin{remark}
%   One can also consider graded Lie~algebras, and if~$\glie$ is a graded Lie~algebra then~$\Univ(\glie)$ inherits the structure of a graded~{\algebra{$\kf$}}:
%   \begin{enumerate}
%     \item
%       A \defemph{grading}\index{grading!of a vector space} of a vector space~$V$ is a direct sum decomposition~$V = \bigoplus_{i \geq 0} V_i$.
%       A \defemph{graded vector space}\index{graded!vector space} is a vector space~$V$ together with a grading of~$V$.
%     \item
%       A \defemph{grading}\index{grading!of a Lie algebra} of a Lie~algebra~$\glie$ is a direct sum decomposition~$\glie = \bigoplus_{i \geq 0} \glie_i$ such that~$[\glie_i, \glie_j] \subseteq \glie_{i+j}$ for all~$i, j \geq 0$.
%       A \defemph{graded Lie~algebra}\index{graded!Lie~algebra} is a Lie~algebra~$\glie$ together with a grading of~$\glie$.
%     \item
%       If~$V$ is a graded vector space then the tensor algebra~$\Tensor(V)$ inherits a grading from~$V$:
%       We get for every~$d \geq 0$ a decomposition
%       \[
%         V^{\tensor d}
%         =
%         \left(
%           \bigoplus_{i \geq 0} V_i
%         \right)^{\tensor d}
%         =
%         \bigoplus_{i_1, \dotsc, i_d \geq 0} V_{i_1} \tensor \dotsb \tensor V_{i_d}  \,.
%       \]
%       This overall results for the tensor algebra~$\Tensor(V)$ in a decomposition
%       \[
%         \Tensor(V)
%         =
%         \bigoplus_{r \geq 0}
%         V^{\tensor r}
%         =
%         \bigoplus_{\substack{r \geq 0 \\ i_1, \dotsc, i_r \geq 0}}
%         V_{i_1} \tensor \dotsb \tensor V_{i_r}  \,.
%       \]
%       We define for all~$d \geq 0$ the homogeneous component~$\Tensor(V)_d$ as
%       \[
%         \Tensor(V)_d
%         \defined
%         \bigoplus_{
%           \substack{r \geq 0 \\
%                     i_1, \dotsc, i_r \geq 0 \\
%                     i_1 + \dotsb + i_r = d}
%         }
%         V_{i_1} \tensor \dotsb \tensor V_{i_r}  \,.
%       \]
%       This defines a grading on~$\Tensor(V)$ which makes the inclusion~$V \inclusion \Tensor(V)$ into a homomorphism of graded vector spaces.
%     \item
%       If~$\glie$ is a graded Lie~algebra with grading~$\glie = \bigoplus_{i \neq 0} \glie_i$ then we regard the tensor algebra~$\Tensor(\glie)$ as a graded~{\algebra{$\kf$}} in the above way.
%       Let~$I$ be the two-sided ideal in~$\Tensor(\glie)$ generated by all elements~$c_{x,y} \defined x \tensor y - y \tensor x - [x,y]$ with~$x, y \in \glie$.
%       The ideal~$I$ is already generated by all~$c_{x,y}$ with~$x, y \in \glie$ homogeneous because~$c_{x,y}$ is bilinear in~$x$ and~$y$.
%       The ideal~$I$ is hence graded and so the quotient~$\Univ(\glie) = \Tensor(\glie)/I$ inherits a grading from~$\Tensor(\glie)$.
%       This is the unique grading that makes the canonical homomorphism~$\glie \to \Univ(\glie)$ a homomorphism of graded~{\algebras{$\kf$}}.
%   \end{enumerate}
% \end{remark}




\subsection{Filtered \texorpdfstring{$\kf$}{k}-Algebras}


\begin{definition}
  A \defemph{filtration}\index{filtration} of a~\algebra{$\kf$}~$A$ is an increasing sequence
  \[
    A_{(0)}
    \subseteq
    A_{(1)}
    \subseteq
    A_{(2)}
    \subseteq
    \dotsb
  \]
  of linear subspaces of~$A$ such that~$A = \bigcup_{i \geq 0} A_{(i)}$ and~$A_{(i)} A_{(j)} \subseteq A_{(i+j)}$ for all~$i,j \geq 0$, as well as~$1 \in A_{(0)}$.
  A \defemph{filtered~\algebra{$\kf$}} is a~\algebra{$\kf$}~$A$ together with a filtration of~$A$.
\end{definition}


\begin{remark}
  \label{filtration conventions}
  \leavevmode
  \begin{enumerate}
    \item
      We will just say that~\enquote{$A$ is a filtered algebra} without mentioning the filtration explicitely.
      The parts of the filtration will then be denoted by~$A_{(i)}$ with~$i \geq 0$.
    \item
      If~$A$ is a filtered~{\algebra{$\kf$}} then we set~$A_{(-i)} \defined 0$ for all~$i < 0$ for convenience.
      The relation~$A_{(i)} A_{(j)} \subseteq A_{(i+j)}$ with~$i,j \geq 0$ then extends to all~$i, j \in \Integer$.
  \end{enumerate}
\end{remark}


\begin{definition}
  Let~$A$ and~$B$ be two filtered~\algebras{$\kf$}.
  \begin{enumerate}
    \item
      A~\defemph{homomorphism of filtered~\algebras{$\kf$}}~$A \to B$ is a homomorphism of the underlying~{\algebras{$\kf$}} such that~$f(A_{(i)}) \subseteq B_{(i)}$ for every~$i \geq 0$.
    \item
      If~$f \colon A \to B$ is a homomorphism of filtered~{\algebras{$\kf$}} then for every~$i \geq 0$ the restriction of~$f$ to a linear map~$A_{(i)} \to B_{(i)}$ is denoted by~$f_{(i)}$.
  \end{enumerate}
\end{definition}


\begin{remark}
  Let~$A$,~$B$ and~$C$ be filtered~\algebras{$\kf$}.
  \begin{enumerate}
    \item
      The identity~$\id_A \colon A \to A$ is a homomorphism of filtered~\algebras{$\kf$}.
    \item
      If~$f \colon A \to B$ and~$g \colon B \to C$ are homomorphisms of filtered~\algebras{$\kf$} then their composition~$g \circ f \colon A \to C$ is again a homomorphism of filtered~\algebras{$\kf$}.
  \end{enumerate}
  This shows that filtered~\algebras{$\kf$} together with homomorphisms of filtered~\algebras{$\kf$} between them form a category, which we will denote by~\gls*{filtered algebras}.
  \begin{enumerate}[start=3]
    \item
      A homomorphism of filtered~{\algebras{$\kf$}}~$f \colon A \to B$ is hence an isomorphism if and only if there exists a homomorphism of filtered~{\algebras{$\kf$}}~$g \colon B \to A$ with~$fg = \id_B$ and~$gf = \id_A$.
  \end{enumerate}
\end{remark}


\begin{warning}
  Let~$A$ and~$B$ be two filtered~{\algebras{$\kf$}}.
  A bijective homomorphism of filtered~{\algebras{$\kf$}}~$f \colon A \to B$ is not necessarily an isomorphism of filtered~{\algebras{$\kf$}}.
  Indeed, take any~{\algebra{$\kf$}}~$A$ with~$\kf \neq A$.
  We can endow~$A$ with a filtration
  \[
    A_{(0)}
    =
    \kf
    \subseteq
    A
    =
    A
    =
    A
    =
    \dotsb
  \]
  which results in a filtered~{\algebra{$\kf$}}~$B_1$.
  But we can also endow~$A$ with the filtration
  \[
    A_{(0)}
    =
    A
    =
    A
    =
    A
    =
    \dotsb
  \]
  which results in a filtered~{\algebra{$\kf$}}~$B_2$.
  Then the identity~$\id_A \colon A \to A$  results in a bijective homomorphism of filtered~{\algebras{$\kf$}}~$B_1 \to B_2$.
  But this is not an isomorphism of filtered~{\algebras{$\kf$}} because the inverse~$B_2 \to B_1$ does not respect the filtrations.
\end{warning}


\begin{remark}
  \leavevmode
  \begin{enumerate}
    \item
      Any grading~$A = \bigoplus_{i \geq 0} A_i$ of a~{\algebra{$\kf$}}~$A$ results in a filtration~$A = \bigcup_{i \geq 0} A_{(i)}$ given by
      \[
        A_{(i)}
        =
        \bigoplus_{j \geq i} A_j
      \]
      for every~$i \geq 0$.
      We can therefore regard every graded~{\algebra{$\kf$}} as a filtered~{\algebra{$\kf$}}.
      Every homomorphism of graded~{\algebras{$\kf$}}~$f \colon A \to B$ is then also a homomorphism of filtered \algebras{$\kf$}.
      This construction gives us a forgetful functor~$\cgAlg{\kf} \to \cfAlg{\kf}$.
    \item 
      If~$A$ is a filtered algebra and~$I$ is any two-sided ideal in~$A$ then the quotient algebra~$A/I$ inherits from~$A$ a filtration given by~$(A/I)_{(i)} \defined \pi(A_{(i)})$ for every~$i \geq 0$.
      Here~$\pi \colon A \to A/I$ denotes the canonical projection.
  \end{enumerate}
\end{remark}


\begin{examples}
  \leavevmode
  \begin{enumerate}
    \item
      If~$V$ is any vector space then the tensor algebra~$\Tensor(V)$, the symmetric algebra~$\Symm(V)$ and the exterior algebra~$\Exterior(V)$ admit gradings as explained in \cref{examples of graded algebras} and hence also induced filtrations.
    \item
      If~$\glie$ is a Lie algebra then its universal enveloping algebra~$\Univ(\glie)$ inherits from the tensor algebra~$\Tensor(\glie)$ a filtration~$\Univ(\glie) = \bigcup_{i \geq 0} \Univ(\glie)_{(i)}$ with terms
      \[
        \Univ(\glie)_{(i)}
        =
        \gen{
          \class{x_1 \dotsm x_j}
        \suchthat
          j \leq i,
          x_1, \dotsc, x_j \in \glie
        }_{\kf} \,.
      \]
    \item
      Let~$M$ be a multiplicative monoid and let
      \[
        M_{(0)}
        \subseteq
        M_{(1)}
        \subseteq
        M_{(2)}
        \subseteq
        M_{(3)}
        \subseteq
        \dotsb
      \]
      be a filtration of~$M$, i.e.\ it holds that~$M = \bigcup_{i \geq 0} M_{(i)}$ with~$M_{(i)} M_{(j)} \subseteq M_{(i+j)}$ for all~$i, j \geq 0$ and~$1 \in M_{(0)}$.
      Then the monoid algebra~$A \defined \kf[M]$ inherits a filtration given by
      \[
        A_{(i)}
        =
        \gen{ M_{(i)} }_{\kf}
      \]
      for all~$i \geq 0$.
    \item
      Let us give a special case of the previous example:
      Let~$G$ be a group with generating set~$S$.
      The length of an element~$g \in G$ with respect to the generating set~$S$ is given by
      \[
        l_S(g)
        =
        \min
        \{
          n \in \Natural
        \suchthat
          \text{there exist~$s_1, \dotsc, s_n \in S$ with~$g = s_1^{\pm 1} \dotsm s_n^{\pm 1}$}
        \}  \,.
      \]
      The length function is subadditive in the sense that
      \[
        l_S(gh)
        \leq
        l_S(g) + l_S(h)
      \]
      for all~$g, h \in G$.
      For every~$i \geq 0$ let
      \[
        G_{(i)}
        \defined
        \{
          g \in G
        \suchthat
          l_S(g) \leq i
        \}  \,,
      \]
      which is the ball of length~$i$.
      These subsets given a filtration of~$G$.
      That~$G_{(i)} G_{(j)} \subseteq G_{(i+j)}$ for all~$i, j \geq 0$ follows from the subadditivity of the length function~$l_S$.
      We hence get for the group algebra~$A \defined \kf[G]$ a filtration given by
      \[
        A_{(i)}
        \defined
        \gen{ G_{(i)} }_{\kf}
        =
        \gen{
          g \in G
        \suchthat
          l_S(g) \leq i
        }_{\kf} \,.
      \]
  \end{enumerate}
\end{examples}


\begin{definition}
  The \defemph{degree}\index{degree!filtration} of an element~$x \in A$ is the minimal~$i \geq 0$ with~$x \in A_{(i)}$.
\end{definition}


\begin{example}
  \leavevmode
  \begin{enumerate}
    \item
      Let~$A \defined \kf[t]$ be the polynomial ring in one variable.
      The standard grading~$\kf[t] = \bigoplus_{i \geq 0} \kf t^i$ gives a filtration
      \[
        A_0
        =
        \kf
        =
        \gen{ 1 }_{\kf}
        \subseteq
        \gen{ 1, t }_{\kf}
        \subseteq
        \gen{ 1, t, t^2 }_{\kf}
        \subseteq
        \gen{ 1, t, t^2, t^3 }_{\kf}
        \subseteq
        \dotsb
      \]
      of~$A$.
      The degree of a nonzero polynomial~$p \in A$ with respect to this filtration is the usual degree of a polynomial.
    \item
      If more generally~$A$ is any graded algebra with associated filtration then the degree of any nonzero element~$x \in A$ with homogeneous decomposition~$x = \sum_{i \geq 0} x_i$ is the largest index~$i$ with~$x_i \neq 0$.
  \end{enumerate}
\end{example}



\subsubsection{The Associated Graded Algebra}


\begin{definition}
  Two elements~$x$ and~$y$ of a filtered~{\algebra{$\kf$}}~$A$ are \defemph{equal up to smaller degree}\index{equal up to smaller degree}\index{up to smaller degree}\index{degree!up to smaller} if~$x = y$ or their difference~$x - y$ is of degree strictly smaller than both~$x$ and~$y$.
\end{definition}


\begin{lemma}
  Let~$A$ be a filtered algebra.
  Two elements~$x, y \in A$ are equal up to smaller degree if and only if they have the same degree~$i \geq 0$ and~$x - y \in A_{(i-1)}$.
\end{lemma}


\begin{proof}
  Suppose first that that~$x$ and~$y$ are equal up to smaller degree.
  Let~$x$ be of degree~$i$ and let~$y$ be of degree~$j$ with~$i \geq j$.
  Then~$x - y \in A_{(j)}$ and thus~$x = x' + (x- x') \in A_{(j)}$.
  Hence~$i \leq j$ and thus~$i = j$.  
  If now~$x = y$ then~$x - y = 0 \in A_{(i-1)}$.
  (Recall for the case~$i = 0$ that~$A_{(-1)} = 0$.)
  Otherwise~$x - y \in A_{(k)}$ for some~$0 \leq k < i$ and hence~$x - y \in A_{(k)}$.
  
  If on the other hand~$x$ and~$y$ have the same degree~$i \geq 0$ with~$x - y \in A_{(i-1)}$ then~$x = y$ or the degree of~$x - y$ is at most~$i-1$ and hence strictly smaller than~$i$.
  This means that~$x$ and~$y$ are equal up to smaller degree.
\end{proof}


\begin{corollary}
  If~$A$ is a filtered~{\algebra{$\kf$}} then \enquote{being equal up to smaller degree} is an equivalence relation on~$A$.
  The equivalence class of an element~$x \in A$ of degree~$i$ is given by the coset~$x + A_{(i-1)}$.
  \qed
\end{corollary}


\begin{remark}
  Let~$A$ be a filtered~{\algebra{$\kf$}} and let~$\sim$ be the equivalence relation \enquote{equal up to smaller degree} on~$A$.
  
  When calculating in~$A$ we sometimes want to replace an element~$x \in A$ by another element~$x' \in A$ that is equal to~$x$ up to smaller degree, while hoping that the result ouf our calculation also stays the same up to smaller degree.
  This can be useful if terms of smaller degree are not important in the given situation, or if they can be dealt with by induction.
  
  We would therefore like to have the properties
  \begin{equation}
    \label{wanted compatibility}
    x + y
    \sim
    x' + y' \,,
    \qquad
    x \cdot y
    \sim
    x' \cdot y' \,,
    \qquad
    \lambda x \sim \lambda x'
  \end{equation}
  for all elements~$x, x', y, y' \in A$ with~$x \sim x'$ and~$y \sim y'$ and scalars~$\lambda \in \kf$.
  This would then mean that the quotient set~$A/{\sim}$ inherits from~$A$ the structure of a~{\algebra{$\kf$}}, allowing us do calculations \enquote{up to smaller degree} by switching from~$A$ to~$A/{\sim}$.
  
  But alas the properties~\eqref{wanted compatibility} do not hold in general:%
  \footnote{The equivalence relation~$\sim$ is actually compatible with scalar multiplication:
  Suppose that~$x, x' \in A$ are two element that are equal up to smaller degree, so that~$x$ and~$x'$ are of the same degree~$i$ and~$x - x' \in A_{(i-1)}$.
  If~$\lambda = 0$ then~$\lambda x = 0 = \lambda x'$ and hence~$\lambda x \sim \lambda x'$.
  If~$\lambda \neq 0$ then it follows from~$x \in A_{(i)}$ and~$x \notin A_{(i-1)}$ that also~$\lambda x \in A_{(i)}$ and~$\lambda x' \notin A_{(i-1)}$.
  This means that the element~$\lambda x$ is again of degree~$i$.
  We find in the same way that~$\lambda x'$ is again of degree~$i$.
  We moverover have~$\lambda x - \lambda x' = \lambda (x-x') \in A_{(i-1)}$ because~$x - x' \in A_{(i-1)}$.
  Altogether this shows that~$\lambda x \sim \lambda x'$.}
  \begin{itemize}
    \item
      Suppose that~$x$ and~$x'$ are of degree~$i$ whereas~$y$ and~$y'$ are of degree~$j$.
      We would like to argue that~$x + y$ and~$x' + y'$ have the same degree~$\max(i,j)$ and that their difference
      \[
        (x + y) - (x' + y')
        =
        (x - x') + (y + y')
      \]
      has degree at most~$\max(i-1,j-1)$ and hence strictly smaller than~$i+j$.
      But this argumentation doesn’t work because the sums~$x+y$ and~$x'+y'$ may have smaller degree than~$i+j$, which would mean that the difference~$(x + y) - (x' + y')$ may have too large of a degree.
      
      As a counterexample we can take the polynomial ring~$A = \kf[t]$ wih the filtration induced by the standard grading~$A = \bigoplus_{i \geq 0} \kf t^i$.
      Then~$x = t$ and~$x' = t+1$ are the equal to smaller degree but for~$y = -t$ the sums~$x + y = 0$ and~$x' + y' = 1$ are not equal up to smaller degree.
      
      This shows that the equivalence relation~$\sim$ is not necessarily compatible with addition.
    \item
      The equivalence relation~$\sim$ also hasn’t to be compatible with multiplication.
      
      We take on~$\kf[t]$ the standard grading~$\kf[t] = \bigoplus_{i \geq 0} \kf t^i$.
      The ideal~$(t^2)$ is homogeneous, and the quotient~$A \defined \kf[t]/(t^2)$ therefore inherits a grading given by
      \[
        A
        =
        \kf
        \oplus
        \kf t
        \oplus
        0
        \oplus
        0
        \oplus
        \dotsb
      \]
      The resulting filtration of~$A$ is given by
      \[
        A_0
        =
        \kf
        \subsetneq
        (t)/t^2
        \subsetneq
        A
        =
        A
        =
        A
        =
        \dotsb
      \]
      The elements~$x = t$ and~$x' = t+1$ are equal up to smaller degree but for~$y = y' = t$ the two products~$xy = 0$ and~$x' y' = t$ are not equal up to smaller degree.
  \end{itemize}
  We can also argue more abstractly that this approach to \enquote{working with elements up to smaller degree} cannot work:
  We would have~$A/{\sim} = A/I$ for the then two-sided ideal~$I = \{x \in A \suchthat x \sim 0\}$.
  But the only element that is equal to~$0$ up to smaller degree is~$0$ itself.
  Hence~$I = 0$ and we would have that~$A = A/{\sim}$, meaning that no two elements~$x, y \in A$ with~$x \neq y$ could be equal up to smaller degree.
  
  The following construction gives us a way to circumvent these problems and to calculate \enquote{up to smaller degree} in a rigorous way:
\end{remark}


\begin{construction}[The associated graded algebra]
  \label{construction of associated graded}
  To every filtered~{\algebra{$\kf$}}~$A$ we can associate a graded~{\algebra{$\kf$}} as follows:
  For every~$i \geq 0$ let
  \[
    \gr[i](A)
    \defined
    A_{(i)} / A_{(i-1)}
  \]
  where we use the convention~$A_{(-1)} = 0$ from \cref{filtration conventions}.
  We denote the residue class of~$x \in A_{(i)}$ in~$\gr[i](A)$ by~$[x]_i$.
  
  For all~$i, j \geq 0$ the multiplication~$A \times A \to A$ restricts to a bilinear map~$A_{(i)} \times A_{(j)} \to A_{(i+j)}$ that in turn induces a well-defined bilinear map
  \[
    \mu_{i,j}
    \colon
    \gr[i](A) \times \gr[j](A)
    \to
    \gr[i+j](A) \,,
    \quad
    ([x]_i, [y]_j)
    \mapsto
    [xy]_{i+j}  \,.
  \]
  We will write~$[x]_i \cdot [y]_j$ or just~$[x]_i [y]_j$ instead of~$\mu_{i,j}([x]_i, [y]_j)$.
  To see that the multipliction~$\mu_{i,j}$ is well-defined let~$x, x' \in A_{(i)}$ and~$y, y' \in A_{(j)}$ with~$x - x ' \in A_{(i-1)}$ and~$y - y' \in A_{(j-1)}$.
  Then
  \begin{align*}
    x y
    &=
    (x' + (x - x')) (y' + (y - y'))
    \\
    &=
    x' y' + (x - x') y' + x' (y - y') + (x - x')(y - y')
    \\
    &\in
    x' y' + A_{(i-1)} A_{(j)} + A_{(i)} A_{(j-1)} + A_{(i-1)} A_{(j-1)}
    \\
    &\subseteq
    x' y' + A_{(i+j-1)} + A_{(i+j-1)} + A_{(i+j-2)}
    \\
    &\subseteq
    x' y' + A_{(i+j-1)}
  \end{align*}
  and hence~$[xy]_{i+j} = [x'y']_{i+j}$.
  
  The element~$[1]_0 \in \gr[0](A)$ satisfies~$[1]_0 \cdot [x]_i = [x]_i$ for all~$i \geq 0$ and every~$[x]_i \in \gr[i](A)$, and the multiplications~$\mu_{i,j}$ are relatively associative in the sense that
  \[
    [x]_i \cdot ([y]_j \cdot [z]_k)
    =
    [xyz]_{i+j+k}
    =
    ([x]_i \cdot [y]_j) \cdot [z]_k
  \]
  for all~$i, j, k \geq 0$ and~$[x]_i \in \gr[i](A)$,~$[y]_j \in \gr[j](A)$ and~$[z]_k \in \gr[k](A)$.
  It follows from \cref{external description of graded algebras} that on the direct sum
  \[
    \gr(A)
    \defined
    \bigoplus_{i \geq 0} \gr[i](A)
  \]
  the multiplications~$\mu_{i,j}$ assemble into a single multiplication
  \[
    \mu
    \colon
    \gr(A) \times \gr(A)
    \to
    \gr(A)
  \]
  that makes~$\gr(A)$ into a graded~{\algebra{$\kf$}}.
  The multiplicative neutral element of~$\gr(A)$ is given by~$[1]_0$.
\end{construction}


\begin{definition}
  For a filtered~{\algebra{$\kf$}}~$A$ the graded algebra~$\gr(A)$ resulting from~$A$ by \cref{construction of associated graded} is the \defemph{associated graded algebra}\index{associated graded algebra} of~$A$.
\end{definition}


\begin{remark}
  Let~$A$ be a filtered algebra.
  \begin{enumerate}
    \item
      To every element~$x \in A$ we can associate an element in~$\gr(A)$:
      We observe that if~$x$ is of degree~$i \geq 0$ then the residue class~$[x]_j$ is not defined for~$j < i$ because~$x \notin A_{(j)}$, whereas it is defined for all~$j \geq i$.
      But for~$j > i$ we find that~$[x]_j = 0$ because~$x_i \in A_{(i)} \subseteq A_{(j-1)}$.
      Hence the only interesting residue class associated to~$x$ is~$[x]_i$;
      we note that~$[x]_i \neq 0$ because~$x \notin A_{(i-1)}$.
    \item
      We observe that two elements~$x, y \in A$ give the same associated element in~$\gr(A)$ if and only if~$x$ and~$y$ are equal up to smaller degree.
      To show this we denote by~$\gamma \colon A \to \gr(A)$ the function that maps every~$x \in A$ to the associated element of~$\gr(A)$.
      
      Suppose first that~$x$ and~$y$ are equal up to smaller degree.
      Then~$x$ and~$y$ are of the same degree~$i$ and~$x - y \in A_{(i-1)}$.
      Hence~$\gamma(x) = [x]_i = [y]_i = \gamma(y)$.
      
      Suppose now that~$x$ and~$y$ are not equal up to smaller degree, where~$x$ is of degree~$i$ and that~$y$ is of degree~$j$.
      If~$i \neq j$ then~$\gamma(x) = [x]_i \neq [y]_j = \gamma(y)$ because the terms~$[x]_i$ or~$[y]_j$ live in different homogeneous degrees and at least one of them is nonzero.
      If~$i = j$ then~$x - y \notin A_{(i-1)}$ and hence~$\gamma(x) = [x]_i \neq [y]_i = \gamma(y)$.
    \item
      Observe that the image of~$\gamma$ consists precisely of the homogeneous elements of~$\gr(A)$.
      If~$x \in A$ has degree~$i$ then~$[x]_i$ is homogeneous of degree~$i$ (or at least for~$x$ nonzero).
  \end{enumerate}
\end{remark}


\begin{warning}
  Let~$A$ be a filtered~{\algebra{$\kf$}}
  \begin{enumerate}
    \item
      The map~$\gamma \colon A \to \gr(A)$ that assigns to each element~$x \in A$ the corresponding element of~$\gr(A)$ is in no way a homomorphism.
      It is in general neither multiplicative nor additive.
    \item
      If~$x_i$,~$i \in I$ is a generating set for the algebra~$A$ then the associated elements of~$\gr(A)$ do not have to form a generating set for~$\gr(A)$.
%     TODO: Add a reference to the universal eveloping algebra of heisenberg/sl2 later on.
  \end{enumerate}
\end{warning}


\begin{remark}
  Let~$A$ be a filtered algebra with finite algebra generating set~$x_1, \dotsc, x_n$ such that the generator~$x_i$ is of degree~$d_i$.
  Then the associated elements in~$\gr(A)$ form a set of algebra generators of~$\gr(A)$ if and only if the generators~$x_1, \dotsc, x_n$ are compatible with the filtration of~$A$ in the sense that for every~$i \geq 0$ the linear subspace~$A_{(i)}$ is spanned by all those monomials~$x_{i_1} \dotsm x_{i_p}$ with~$d_{i_1} + \dotsb + d_{i_p} \leq i$.
  See \cite{associated_generated} for more details on this.
\end{remark}


\begin{example}[Weyl algebra]
  \label{weyl algebra}
  We consider the \defemph{Weyl~algebra}\index{algebra!Weyl}\index{Weyl!algebra}
  \[
    A
    \defined
    \kf\gen{X, Y}/(YX - XY - 1) \,.%
    \footnote{The algebra~$A$ is isomorphic to the subalgebra of~$\End_{\kf}(\kf[t])$ generated by the endomorphisms~$x$ and~$y$ that are given by~$x(p(t)) = t p(t)$ and~$y(p(t)) = p'(t)$.
  These endomorphisms satsify the relations~$yx - xy - \id = 0$ by the product rule.}
  \]
  We denote the residue classes of~$X$ and~$Y$ in~$A$ by~$x$ and~$y$.
  Then~$yx = xy + 1$, which one may think about as a rewriting rule for words in~$x$ and~$y$.
  It follows from induction that
  \[
    y x^n
    =
    x^n y + n x^{n-1}
  \]
  for all~$n \geq 0$ and thus
  \[
    y p(x)
    =
    p(x) y + p'(x)
  \]
  for every polynomial~$p$.
  
  The~{\algebra{$\kf$}}~$T \defined \kf\gen{X, Y}$ carries a unique grading with~$X$ and~$Y$ in degree~$1$.
  The resulting filtration of~$T$ is given by
  \[
    T_{(i)}
    =
    \gen{
      Z_1 \dotsm Z_j
    \suchthat
      1 \leq j \leq i,
      Z_k \in \{X, Y\}
    }_{\kf}
    =
    \gen{
      \text{words in~$X$,~$Y$ of length~$\leq i$}
    }
  \]
  The Weyl~algebra~$A$ inherits a filtration given by
  \begin{align*}
    A_{(i)}
    =
    \gen{
      z_1 \dotsm z_j
    \suchthat
      1 \leq j \leq i,
      z_k \in \{x, y\}
    }_\kf
    =
    \gen{
      \text{words in~$x$,~$y$ of length~$\leq i$}
    }
  \end{align*}
  
  We observe that the two elements~$x, y \in T$ commute up to smaller degree, i.e.\ the two products~$xy$ and~$yx$ are equal up to smaller degree.
  Hence the associated elements~$[x]_1$ and~$[y]_1$ in~$\gr(A)$ do actually commute because
  \[
    [x]_1 [y]_1
    =
    [xy]_2
    =
    [yx - 1]_2
    =
    [yx]_2
    =
    [y]_1 [x]_1 \,.
  \]
  We will in the following show that~$\gr(A) \cong \kf[t,u]$ if~$\ringchar(\kf) = 0$ with~$[x]_1$ and~$[y]_1$ corresponding to~$t$ and~$u$.
  
  \begin{claim}
    \label{subspace spanned by monomials}
    For every~$i \geq 0$ the linear subspace~$A_{(i)}$ is spanned by the monomial~$x^n y^m$ with~$n + m \leq i$.
  \end{claim}
  
  \begin{proof}
    We show the statement by induction over~$i$.
    It holds for~$i = 0$ that
    \[
      A_{(0)}
      =
      \kf
      =
      \gen{1}_{\kf}
      =
      \gen{x^0 y^0}_{\kf}
    \]
    Suppose now that the statement holds for some~$i \geq 0$.
    We have
    \[
      T_{(i+1)}
      =
      x T_{(i)} \oplus y T_{(i)}
    \]
    and in the quotient~$A$ therefore
    \[
      A_{(i+1)}
      =
      x A_{(i)} + y A_{(i)} \,.
    \]
    We find by the induction hypothesis that the summand~$x A_{(i)}$ is spanned by the monomials~$x^{n+1} y^m$ with~$n+m \leq i$ and that the summand~$y A_{(i)}$ is spanned by the monomials~$y x^n y^m$ with~$n + m \leq i$.
    We have for these generators of~$y A_{(i)}$ that
    \[
      y x^n y^m
      =
      (x^n y + n x^{n-1}) y^m
      =
      x^n y^{m+1} + n x^{n-1} y^m \,.
    \]
    This shows that~$A_{(i+1)}$ is spanned by the monomials~$x^n y^m$ with~$n+m \leq i+1$.
  \end{proof}
  
  It follows that the homogeneous component~$\gr[i](A)$ is spanned by the residue classes~$[x^n y^m]_i$ with~$n + m = i$.
  The collection of all these residue classes therefore spans~$\gr(A)$ as a vector space.
  These generators commute:
  
  \begin{claim}
    It holds for all~$i, j \geq 0$ and~$n, m, k, l \geq 0$ with~$i = n+m$ and~$j = k+l$ that
    \[
      [x^n y^m]_i \cdot [x^k y^l]_j
      =
      [x^{n+k} y^{m+l}]_{i+j} \,.
    \]
  \end{claim}
  
  \begin{proof}
    For the case~$i = 0$ we find that~$n = m = 0$ and hence
    \[
      [x^0 y^0]_0 \cdot [x^k y^l]_j
      =
      [1]_0 \cdot [x^k y^l]_j
      =
      [x^k y^l]_j
    \]
    For~$i = 1$ we have the two cases~$n = 1$,~$m = 0$ and~$n = 0$,~$m = 1$.
    We find in these cases that
    \[
      [x^1 y^0]_1 \cdot [x^k y^l]_j
      =
      [x]_1 \cdot [x^k y^l]_j
      =
      [x^{k+1} y^l]_{j+1}
    \]
    and
    \begin{align*}
      [x^0 y^1]_1 \cdot [x^k y^l]_j
      &=
      [y]_1 \cdot [x^k y^l]_j
      \\
      &=
      [y x^k y^l]_{j+1}
      \\
      &=
      [(x^k y + k x^{k-1}) y^l]_{j+1}
      \\
      &=
      [x^k y^{l+1} + k x^{k-1} y^l]_{j+1}
      \\
      &=
      [x^k y^{l+1}]_{j+1} + k [x^{k-1} y^l]_{j+1}
      \\
      &=
      [x^k y^{l+1}]_{j+1}
    \end{align*}
    where we used that~$x y^k = y^k x + k y^{k-1}$ and that~$[z]_{j+1} = 0$ for all~$z \in A_{(j)}$.
    It follows for the general case that
    \begin{align*}
      {}&
      [x^n y^m]_i \cdot [x^k y^l]_j
      \\
      ={}&
      \underbrace{ [x]_1 \dotsm [x]_1 }_{n}
      \underbrace{ [y]_1 \dotsm [y]_1 }_{m}
      \cdot
      [x^k y^l]_j
      \\
      ={}&
      \underbrace{ [x]_1 \dotsm [x]_1 }_{n}
      \underbrace{ [y]_1 \dotsm [y]_1 }_{m-1}
      \cdot
      [x^{1+k}  y^l]_j
      \\
      ={}&
      \dotsb
      \\
      ={}&
      [x^{n+k} y^{m+l}]_{i+j}
    \end{align*}
    as claimed.
  \end{proof}

  We have now seen that~$\gr(A)$ is spanned by the monomials~$[x^n y^m]_{n+m}$ with~$n, m \geq 0$ and these monomials commute with each other.
  It only remains to show that these monomials~$[x^n y^m]_{n+m}$ are linearly indepependent.
  For this it sufficies to show the following:
  
  \begin{claim}
    \label{linear independence of monomials}
    The monomials~$x^n y^m$ with~$n, m \geq 0$ in~$A$ are linearly independent if~$\ringchar(\kf) = 0$.
  \end{claim}
  
  We will defer the proof of \cref{linear independence of monomials} to a later occasion.
  
  It follows from \cref{subspace spanned by monomials} and \cref{linear independence of monomials} that for every~$i \geq 0$ the quotient~$\gr[i](A) = A_{(i)} / A_{(i-1)}$ has the residue classes~$[x^n y^m]_{n+m}$ with~$n+m = i$ as a basis, and hence overall that the residue classes~$[x^n y^m]_{n+m}$ with~$n, m \geq 0$ form a basis of~$\gr(A)$.
  
  We find altogether that~$\gr(A) \cong \kf[t,u]$ with~$[x]_1$ and~$[y]_1$ corresponding to the variables~$x$ and~$y$.
\end{example}


\begin{remark}
  The Weyl algebra is an example of a \defemph{skew polynomial ring}:
  If~$R$ is a~{\algebra{$\kf$}} and~$\delta$ is a derivation of~$R$ then one can endow the polynomial vector space~$A = R[t]$ uniquely with the structure of a~{\algebra{$\kf$}}, different from the usual one (unless~$\delta = 0$), such that
  \begin{itemize}
    \item
      the inclusion~$R \to A$,~$r \mapsto r t^0$ is a homomorphism of~{\algebras{$\kf$}},
    \item
      $t^i$ is the~{\howmanyth{$i$}} power of~$t$ in~$A$ and
    \item
      the variable~$t$ interacts with the elements~$r \in R$ via~$t r = r t + \delta(r)$.
  \end{itemize}
  The resulting~{\algebra{$\kf$}} is a \defemph{skew polynomial algebra}\index{skew polyomial algebra} over~$R$ and denoted by~\gls*{skew polynomial algebra}.
  Observe that the multiplication in~$R[t;\delta]$ is built precisely so that the derivation~$[t,-]$ of~$R[t;\delta]$ acts on~$R \cong R t^0$ in the same way as~$\delta$.
  The following are two special cases of skew polynomial algebras:
  \begin{enumerate}
    \item
      If~$R$ is any~{\algebra{$\kf$}} and~$\delta = 0$ then~$R[t; \delta] = R[t]$ is the usual polynomial ring.
    \item
      If~$R = \kf[t]$ is the polynomial ring in one variable and~$\delta = \partial_t$ is the derivative with respect to the variable~$t$ than~$R[t;\delta] = R[t;\partial_t]$ is the Weyl~algebra.
  \end{enumerate}
  
%   Currently not needed and distracting.
%
%   Skew polynomial algebras can be further generalized:
%   If~$\sigma \colon R \to R$ is an algebra homomorphism then a~{\linear{$\kf$}} map~$\delta \colon R \to R$ is a~{\derivation{$\sigma$}}\index{$\sigma$-derivation}\index{derivation!$\sigma$-} if
%   \[
%     \delta(rs)
%     =
%     \delta(r) s + \sigma(r) \delta(s)
%   \]
%   for all~$r, s \in R$.
%   Then the polynomial vector space~$A = R[t]$ can be endowed uniquely with the structure of a~{\algebra{$\kf$}} such that
%   \begin{itemize}
%     \item
%       the inclusion~$R \to A$,~$r \mapsto r t^0$ is a homomorphism of~{\algebras{$\kf$}},
%     \item
%       $t^i$ is the~{\howmanyth{$i$}} power of~$t$ in~$A$ and
%     \item
%       the variable~$t$ interacts with the elements~$r \in R$ via~$t r = \sigma(r) t + \delta(r)$.
%   \end{itemize}
%   The resulting algebra~$A$ is an \defemph{Ore extension}\index{Ore extension} of~$R$ and is denoted by~\gls*{ore extension}.
\end{remark}


% Bad example: Already graded
%
% \begin{example}[Quantum plane]
%   Let~$q \in \kf$ wit~$q \neq 0$.
%   The \emph{quantum plane} is the~{\algebra{$\kf$}}~$A \defined \kf\gen{X,Y}/(YX - qXY)$.%
%   \footnote{The usual polynomial ring~$\kf[x,y] \cong \kf\gen{X,Y}/(YX - XY)$ is the algebraic object corresponding to the plane~$\Aff^2 = k^2$ in the sense that~$\kf[x,y]$ is the algebra of (regular) functions on~$\Aff^2$.
%   The given algebra~$A$ should therefore correspond to some (non-existing) plane, the functions on which don’t commute.}
%   Let~$x, y \in A$ be the residue classes of~$X$ and~$Y$.
%   The~{\algebra{$\kf$}}~$A$ inherits from the standard grading of the free algebra~$\kf\gen{X,Y}$ a filtration given by
%   \[
%     A_{(i)}
%     =
%     \gen{
%       z_1 \dotsm z_j
%     \suchthat
%       0 \leq j \leq i,
%       z_1, \dotsc, z_j \in \{x, y\}
%     }_{\kf} \,.
%   \]
%   It can be shown as for the Weyl~algebra (actually easer than for the Weyl~algebra) for all~$i \geq 0$ the vector space~$A_{(i)}$ is spanned by the monomials~$x^n y^m$ with~$n + m \leq i$.
%   The quotient vector space~$\gr[i](A)$ is therefore spanned for every~$i \geq 0$ by the residue classes~$[x^n y^m]_i$ with~$n + m = i$.
%   The algebra~$\gr(A)$ is hence spanned by the collection of all such residue classes~$[x^n y^m]$ with~$n, m \geq 0$.
% \end{example}


\begin{warning}
  For a filtered~{\algebra{$\kf$}} there does not exist a \enquote{canonical} algebra homomorphism~$A \to \gr(A)$.
  There does not even have to exist any algebra homomorphism~$A \to \gr(A)$ as \cref{weyl algebra} shows:
  If~$A \defined \kf\gen{X,Y}/(YX - XY - 1)$ denotes the Weyl algebra with the filtration from \cref{weyl algebra} then we have seen that~$\gr(A) = \kf[t,u]$ is the commutative polynomial ring in two variables.
  Algebra homomorphisms~$A \to \gr(A)$ correspond bijectively to elements~$x, y \in \kf[t,u]$ which satisfy the commutator relation~$yx - xy -1 = 0$, i.e.~$yx = xy + 1$.
  But no such element exist in~$\kf[t,u]$.
  Hence there exist no algebra homorphism~$A \to \gr(A)$.
\end{warning}


\begin{construction}
  If~$f \colon A \to B$ is a homomorphism of filtered~{\algebras{$\kf$}} then for every~$i \geq 0$ the restriction~$f_{(i)}$ induces a linear map
  \[
    \gr[i](f)
    \colon
    \gr[i](A)
    \to
    \gr[i](B) \,,
    \quad
    [x]_i
    \mapsto
    [f(x)]_i  \,.
  \]
  These linear maps come together to form a linear map
  \[
    \gr(f)
    \colon
    \gr(A)
    \to
    \gr(B)  \,.
  \]
  This map is already an algebra homomorphism because~$\gr(f)([1]_0) = [f(1)]_0 = [1]_0$ and
  \begin{align*}
    \gr(f)( [x]_i [y]_j )
    =
    \gr(f)( [xy]_{i+j} )
    =
    [f(xy)]_{i+j}
    =
    [f(x) f(y)]_{i+j}
    &=
    [f(x)]_i [f(y)]_j
    \\
    &=
    \gr(f)([x]_i) \gr(f)([y]_j) \,.
  \end{align*}
  for all~$i, j \geq 0$ and all~$[x]_i \in \gr[i](A)$ and~$[y]_j \in \gr[j](A)$.  
  It holds for every filtered~{\algebra{$\kf$}}~$A$ that~$\gr(\id_A) = \id_{\gr(A)}$ and for all homomorphisms of filtered~{\algebras{$\kf$}}~$f \colon A \to B$ and~$g \colon B \to C$ that~$\gr(g \circ f) = \gr(g) \circ \gr(f)$.
  
  We have hence constructed a functor~$\gr \colon \cfAlg{\kf} \to \cgAlg{\kf}$.
\end{construction}



% TODO: Move this at the right position.
% 
% \begin{examples}
%  \begin{enumerate}[leftmargin=*]
%   \item
%    Let $\glie$ be a $k$-Lie algebra. The canonical projection
%    \[
%     \pi \colon T(\g) \to \Univ(\glie), \quad x_1 \tensor \dotsb \tensor x_n \to x_1 \dotsm x_n
%     \quad \text{for every $n \in \N$ and all $x_1, \dotsc, x_n \in \g$}
%    \]
%    is a homomorphism of filtered \algebras{$\kf$}, where the filtration of $T(\g)$ is induced by the gradation discussed in Examples \ref{expls: graded algebras}. Hence it induces a homomorphism of graded \algebras{$\kf$}
%    \[
%     \gr(\pi) \colon T(\g) \to \gr(\Univ(\glie)),
%    \]
%    where $T(\g)$ is identified with $\gr(T(\g))$ as above. This homomorphism maps an element $x_1 \tensor \dotsb \tensor x_n$ with $x_1, \dots, x_n \in \g$ to the residue class $[x_1 \dotsm x_n] \in \gr(\Univ(\glie))_n$.
%  \end{enumerate}
% \end{examples}
% 
% 

\begin{lemma}
  Let~$A$ be a graded~{\algebra{$\kf$}}.
  Then the linear map
  \[
    \varphi 
    \colon
    A
    \to
    \gr(A)
  \]
  given by~$\varphi(x) = [x]_i$ for all~$i \geq 0$ and~$x \in A_i$ is an isomorphism of graded~{\algebras{$\kf$}}.
\end{lemma}


\begin{proof}
  We have for all~$i \geq 0$ that~$A_{(i)} = \bigoplus_{j=0}^i A_j$ and hence
  \[
    \gr[i](A)
    =
    A_{(i)} / A_{(i-1)}
    \cong
    A_i
  \]
  as vector spaces.
  It follows that~$\varphi$ is an isomorphism of vector spaces.
  Moreover, we have for all homogeneous elements~$x \in A_i$ and~$y \in A_j$ with~$i, j \geq 0$ that
  \[
    \varphi(x) \varphi(y)
    =
    [x]_i [y]_j
    =
    [xy]_{i+j}
    =
    \varphi(xy)
  \]
  where the last equality holds because~$xy \in A_{i+j}$.
  This shows that~$\varphi$ is multiplicative and hence already an isomorphism of~{\algebras{$\kf$}}.
\end{proof}


\begin{example}
  Let~$A \defined \kf\gen{X, Y}/(YX - XY - 1)$ be the Weyl~algebra with the filtration from \cref{weyl algebra}.
  Then~$\gr(A) \cong \kf[t,u]$ is commutative but~$A$ is not, hence~$\gr(A)$ and~$A$ are not isomorphic as algebras.
  It follows that the filtration on~$A$ cannot come from a grading, as otherwise~$\gr(A) \cong A$ (already as graded~{\algebras{$\kf$}}).
\end{example}


\begin{warning}
  If~$A$ is a filtered~{\algebra{$\kf$}} for which the filtration does not come from a grading then~$A$ and~$\gr(A)$ are not isomorphic as filtered algebras (where the filtration of~$\gr(A)$ is induced by the grading).
\end{warning}


\begin{definition}
  Let~$A$ be a ring.
  \begin{enumerate}
    \item
      An element~$x \in A$ is a \defemph{left zero divisor}\index{zero divisor!left} if there exists some nonzero~$y \in A$ with~$xy = 0$.
    \item
      An element~$x \in A$ is a \defemph{right zero divisor}\index{zero divisor!right} if there exists some nonzero~$y \in A$ with~$yx = 0$.
    \item
      An element~$x \in A$ is a \defemph{zero divisor}\index{zero divisor} if it is a left zero divisor or a right zero divisor.
  \end{enumerate}
\end{definition}


\begin{remark}
  Let~$A$ be a ring.
  \begin{enumerate}
    \item
      An element~$x \in A$ is a left zero divisor if and only if the left multiplication map~$A \to A$,~$y \mapsto xy$ is not injective.
      It is a right zero divisor if and only if the right multiplication map~$A \to A$,~$y \mapsto yx$ is not injective.
    \item
      The zero element~$0 \in A$ is both a left zero divisor and right zero divisor if~$A \neq 0$.
      It is not a zero divisor for~$A = 0$.
  \end{enumerate}
\end{remark}


\begin{proposition}
  \label{associated graded algebra and zero divisors}
  Let~$A$ be a filtered~{\algebra{$\kf$}}.
  \begin{enumerate}
    \item
      \label{associated graded has left zero divisors}
      If~$\gr(A)$ doesn’t contain nonzero left zero divisors then neither does~$A$.
    \item
      \label{associated graded has right zero divisors}
      If~$\gr(A)$ doesn’t contain nonzero right zero divisors then neither does~$A$.
    \item
      If~$\gr(A)$ doesn’t contain nonzero zero divisors then neither does~$A$.
  \end{enumerate}
\end{proposition}


\begin{proof}
  \leavevmode
  \begin{enumerate}
    \item
      Suppose that~$A$ contains a nonzero left zero divisor~$x$.
      Then there exists some nonzero~$y \in A$ with~$xy = 0$.
      If~$x$ is of degree~$i$ and~$y$ is of degree~$j$ then~$[x]_i \neq 0$ and~$[y]_j \neq 0$ because both~$x$ and~$y$ are nonzero.
      But~$[x]_i [y]_j = [xy]_{i+j} = [0]_{i+j} = 0$.
      Hence~$[x]_i$ is again a nonzero left zero divisor.
    \item
      This can be shown in the same way as part~\ref*{associated graded has left zero divisors}.
    \item
      This is a combination of parts~\ref*{associated graded has left zero divisors} and~\ref*{associated graded has right zero divisors}.
    \qedhere
  \end{enumerate}
\end{proof}


\begin{remark}
  The converse of \cref{associated graded algebra and zero divisors} does not hold.
  To see this let~$A$ be any~\algebra{$\kf$} with~$\kf \subsetneq A$ and consider the filtration
  \[
    A_{(0)}
    =
    \kf
    \subseteq
    A
    \subseteq
    A
    \subseteq
    A
    \subseteq
    \dotsb
  \]
  Then~$\gr[0](A) = \kf$,~$\gr[1](A) = A/{\kf} \neq 0$ and~$\gr[i](A) = 0$ for every~$i \geq 2$.
  Hence~$\gr[1](A) \gr[1](A) = 0$ even though~$\gr[1](A) \neq 0$.
  This means that the elements of~$\gr[1](A)$ are both left zero divisors and right zero divisors.
\end{remark}
% 
% 
% 
% 
% 
% \subsection{Poincar\'{e}--Birkhoff--Witt Theorem (Concrete Version)}
% For this subsection we fix some $k$-Lie algebra $\glie$ with basis $(x_i)_{i \in I}$ where $(I, \leq)$ is a totally ordered index set. Before stating and proving the \emph{Poincar\'{e}--Birkhoff--Witt theorem} (PBW) we fix some notation which we will only use in this subsection.
% 
% 
% \begin{definition}
%  Let $\mc{I}_n \defined \{(i_1, \dotsc, i_n) \mid i_1, \dotsc, i_n \in I, \; i_1 \leq \dotsb \leq i_n\}$ for every $n \in \N$ and set $\mc{I} \defined \bigcup_{n \in \N} \mc{I}_n$. For every $\alpha = (i_1, \dotsc, i_n) \in I^n$ with $n \in \N$ let $x_\alpha \defined x_{i_1} \dotsm x_{i_n} \in \Univ(\glie)$.
% \end{definition}
% 
% 
% \begin{remark}
%  Notice that $\mc{I}_0$ contains the empty tupel.
% \end{remark}
% 
% 
% \begin{theorem}[PBW (concrete version)] \label{thrm: pbw concrete}
%  The familiy $(x_\alpha \mid \alpha \in \mc{I})$ is a $k$-basis of $\Univ(\glie)$.
% \end{theorem}
% 
% 
% \begin{remark}
%  The basis $(x_\alpha \mid \alpha \in \mc{I})$ can also be written as
%  \[
%   \left(
%    x_{i_1}^{p_1} \dotsm x_{i_n}^{p_n}
%   \mid
%    n \in \N,\;
%    i_1, \dotsc, i_n \in I,\;
%    i_1 < \dotsb < i_n,\;
%    p_1, \dotsc, p_n \geq 1
%   \right).
%  \]
% \end{remark}
% 
% 
% \begin{example}
%  If $\glie$ is a finite dimensional $k$-Lie algebra with basis $x_1, \dotsc, x_n$ then $\Univ(\glie)$ has a basis given by $(x_1^{p_1} \dotsm x_n^{p_n} \mid p_1, \dotsc, p_n \in \N)$. In particular a basis of $\Univ(\sll_2(k))$ is given by $(e^\ell h^m f^n \mid \ell, m ,n \in \N)$.
% \end{example}
% 
% 
% \begin{lemma}\label{lem: pbw concrete generating part}
%  The collection $(x_\alpha \mid \alpha \in \mc{I})$ generates $\Univ(\glie)$ as a vector space.
% \end{lemma}
% \begin{proof}
%  To show that $(x_\alpha \mid \alpha \in \mc{I})$ generates $\Univ(\glie)$ as a vector space it sufficies to show that $\mc{B}_n \defined (x_\alpha \mid m \leq n, \alpha \in \mc{I}_m)$ generates $\Univ(\glie)_{(n)}$ as a vector space, which will be shown by induction over $n \in \N$: For $n = 0$ it holds because $\Univ(\glie) = k$ is one-dimensional and thus spanned by $x_{(\;)}$, the monomial corresponding to the empty tupel.
%  
%  Suppose that the statement holds for some $n \in \N$. Then $\Univ(\glie)_{(n+1)}$ is generated as a vector space as a by the monomials $(x_{(i_1, \dotsc, i_n)} \mid m \leq n+1, \; i_1, \dotsc, i_m \in I)$. Therefore it sufficies to express these monomials in terms of $\mc{B}_{n+1}$. By induction hypothesis is it enough to check this for the monomials $(x_{(i_1, \dotsc, i_{n+1})} \mid i_1, \dotsc, i_{n+1} \in I)$. For this let $\alpha = (i_1, \dotsc, i_{n+1})$ be some fixed multiindex with $i_1, \dotsc, i_{n+1} \in I$.
%  
%  For any two $x,y \in \g$ one has $xy = yx + [x,y]$ with $[x,y] \in \Univ(\glie)_{(1)}$. Hence there exists for any permutation $\sigma \in S_{n+1}$ a linear combination $R_\sigma \in \Univ(\glie)_{(1)}$ with
%  \[
%   x_\alpha
%   = x_{i_1} \dotsm x_{i_{n+1}}
%   = x_{i_{\sigma(1)}} \dotsm x_{i_{\sigma(n+1)}} + R_\sigma
%   = x_{(i_{\sigma(1)}, \dotsc, x_{\sigma(n+1)})} + R_\sigma.
%  \]
%  Let $\sigma \in S_{n+1}$ be a permutation with $i_{\sigma(1)} \leq \dotsb \leq i_{\sigma(n+1)}$. By induction hypothesis $R_\sigma \in \Univ(\glie)_{(n)}$ can be expressed as a linear combination of the monomials $\mc{B}_n$. Hence $x_\alpha = x_{(i_{\sigma(1)}, \dotsc, i_{\sigma(n+1)})} + R_\sigma$ can be expressed as a linear combination of the monomials $\mc{B}_{n+1}$ because $x_{(i_{\sigma(1)}, \dotsc, i_{\sigma(n+1)})}$ is one of them.
% \end{proof}
% 
% 
% \begin{proof}[Proof of PBW (concrete version)]
%  By Lemma~\ref{lem: pbw concrete generating part} the collection $(x_\alpha \mid \alpha \in \mc{I})$ generates $\Univ(\glie)$ as a vector space, so all that’s left to show is that it is linearly independent.
%  
%  Let $V \defined k[Z_i \mid i \in I]$ and for every $n \in \N$ let $V_{(n)}$ be the polynomials of degree $\leq n$. For every $n \in \N$ and $\alpha = (i_1, \dotsc, i_n) \in I^n$ write $Z_\alpha \defined Z_{i_1} \dotsm Z_{i_n}$. If $i \in I$ and $\alpha = (i_1, \dotsc, i_n) \in I^n$ then write $i \leq \alpha$ if $i \leq i_j$ for every $j = 1, \dotsc, n$. Also set $i \cdot \alpha = (i, i_1, \dotsc, i_n) \in I^{n+1}$.
%  
%  To show that $(x_\alpha \mid \alpha \in \mc{I})$ is linearly independent $V$ will be given the structure of a representation of $\glie$ such that
%  \[
%   x_i.Z_\alpha = Z_{i \cdot \alpha}
%   \quad \text{for every $i \in I$ and $\alpha \in \mc{I}$ with $i \leq \alpha$}.
%  \]
%  Then for the corresponding $\Univ(\glie)$-module structure on $V$ it follows that
%  \[
%   x_\alpha \cdot 1
%   = x_\alpha \cdot Z_{(\;)}
%   = Z_\alpha
%   \quad \text{for every $\alpha \in \mc{I}$}
%  \]
%  where $1 \in V = k[Z_i \mid i \in I]$ and $(\;)$ denotes the empty tupel. Because $(Z_\alpha \mid \alpha \in \mc{I})$ is linearly independent it then follows that $(x_\alpha \mid \alpha \in \mc{I})$ is linearly independent. The existence of such an action follows from the following:
%  
%  \begin{claim*}
%   There exists a unique sequence $(\varphi_n)_{n \in \N}$ of bilinear maps
%   \[
%    \varphi_n \colon \glie \times V_{(n)} \to V_{(n+1)}, \quad (x,p) \mapsto x.p
%   \]
%   satisfying the following conditions:
%   \begin{enumerate}
%    \item\label{enum: pbw restriction coincides}
%     The restriction of $\varphi_{n+1}$ to $\glie \times V_{(n)}$ coincides with $\varphi_n$ for every $n \in \N$
%    \item\label{enum: pbw essential condition}
%     $x_i.Z_\alpha = Z_{i \cdot \alpha}$ for every $i \in I$ and $\alpha \in \mc{I}$ with $i \leq \alpha$.
%    \item\label{enum: pbw representation of lie algebra}
%     $x_i.x_j.Z_\alpha - x_j.x_i.Z_\alpha = [x_i, x_j].Z_\alpha$ for all $i,j \in I$ and every $\alpha \in \mc{I}$.
%    \item\label{enum: pbw technical detail for construction}
%     $x_i.Z_\alpha - Z_{i \cdot \alpha} \in V_{(n)}$ for every $n \in \N$, $i \in I$ and $\alpha \in \mc{I}_n$.
%    \end{enumerate}
%    (Condition \ref{enum: pbw restriction coincides} actually follows from the other conditions by the uniqueness of the sequence $(\varphi_n)_{n \in \N}$. See \cite[\S 17.4]{Humphreys} for more details.)
%  \end{claim*}
%  \begin{proof}
%   Notice thet the notation $x.p$ with $x \in \g$ and $p \in V$ is unambiguous by condition \ref{enum: pbw restriction coincides}. The maps $\varphi_n$ will be defined by induction over $n$:
%   
%   As $V_{(0)}$ is one-dimensional and spanned by $1 = Z_{(\;)}$ it follows from condition \ref{enum: pbw essential condition} that $x_i.1 = x_i \cdot Z_{(\;)} = Z_{i \cdot (\;)} = Z_i$ for every $i \in I$. This defines $\varphi_0$ uniquely. Conditions \ref{enum: pbw essential condition} and \ref{enum: pbw technical detail for construction} hold by construction and the conditions \ref{enum: pbw restriction coincides} and \ref{enum: pbw representation of lie algebra} do not affect $\varphi_0$.
%   
%   Let $n \in \N$ and suppose $\varphi_m$ is constructed for every $m \leq n$. By condition \ref{enum: pbw restriction coincides} all that is left to define is $x_i.Z_\alpha$ for $\alpha \in \mc{I}_{n+1}$. If $i \leq \alpha$ then $x_i.Z_\alpha = Z_{i \cdot \alpha}$ by condition \ref{enum: pbw essential condition}.
%   
%   If $i > \alpha$ then there exists $\beta \in \mc{I}_n$ and $j \in I$ with $\alpha = j \cdot \beta$ such that $i > j$ and $j \leq \beta$. If condition \ref{enum: pbw representation of lie algebra} was to hold for $\varphi_{n+1}$ it follows that
%   \begin{equation}\label{eqn: action defined as lie action}
%    x_i.Z_\alpha
%    = x_i.x_j.Z_\beta
%    = x_j.x_i.Z_\beta + [x_i, x_j].Z_\beta.
%   \end{equation}
%   Because $\beta \in \mc{I}_n$ the term $[x_i, x_j].Z_\beta$ in the above sum is already defined. Because $x_i.Z_\beta \equiv Z_{i \cdot \beta} \pmod{V_{(n)}}$ there exists some $Y \in V_{(n)}$ with $x_i.Z_\beta = Z_{i \cdot \beta} + Y$. Let $\gamma \in \mc{I}_{n+1}$ be defined by taking $\beta$ and inserting $i$ at the right position. Then $Z_\gamma = Z_{i \cdot \beta}$. Because $j < i$ and $j \leq \beta$ it follows that $j \leq \gamma$ and therefore 
%   \[
%    x_j.Z_{i \cdot \beta}
%    = x_j.Z_\gamma
%    = Z_{i \cdot \gamma}
%    = Z_{i \cdot (j \cdot \beta)}
%    = Z_{i \cdot \alpha}
%   \]
%   Because every summand in
%   \begin{equation}\label{eqn: I really wish Humphreys had explained this}
%    \begin{aligned}
%     x_i.Z_\alpha
%     = x_j.x_i.Z_\beta + [x_i, x_j].Z_\beta
%     &= x_j.(Z_{i \cdot \beta} + Y) + [x_i, x_j].Z_\beta \\
%     &= Z_{i \cdot \alpha} + x_j.Y + [x_i, x_j].Z_\beta
%    \end{aligned}
%   \end{equation}
%   is defined it follows that $\varphi_{n+1}$ is uniquely defined.
%   
%   Conditions \ref{enum: pbw restriction coincides} and \ref{enum: pbw essential condition} hold for $\varphi_{n+1}$ by construction. Condition \ref{enum: pbw technical detail for construction} holds for $i \leq \alpha$ by condition \ref{enum: pbw essential condition} and for $i > \alpha$ by \eqref{eqn: I really wish Humphreys had explained this} because $x_j.Y \in V_{(n+1)}$ and $[x_i, x_j].Z_\beta \in V_{(n+1)}$.
%   
%   It remains to check Condition \ref{enum: pbw representation of lie algebra} for $\varphi_{n+1}$, i.e.\ when $i,j \in I$ and $\alpha \in \mc{I}_n$. For $i = j$ this follows from the Lie bracket being alternating. Suppose that $i \neq j$. By the Lie bracket is antisymmetric it can be w.l.o.g.\ assumed that $i < j$.
%   
%   If $i \leq \alpha$ then $x_j.x_i.Z_\alpha$ is defined above by using \eqref{eqn: action defined as lie action} (where $\beta$ has to be replaced by $\alpha$ and $i$ and $j$ have to be switched), hence condition \ref{enum: pbw essential condition} holds in this case by construction. Notice that if $j \leq \alpha$ then also $i \leq \alpha$ because $i < j$.
%   
%   Hence the only case left is $i \nleq \alpha$. By the above it then follows that also $j \nleq \alpha$. As this cannot happen for $n = 0$ it can be w.l.o.g.\ assumed that $n \geq 1$. Let $k \in I$ and $\beta \in \mc{I}_{n-1}$ with $\alpha = k \cdot \beta$. Because condition \ref{enum: pbw representation of lie algebra} holds for $\varphi_n$ it follows that
%   \begin{align*}
%    x_i.x_j.Z_\alpha
%    = x_i.x_j.x_k.Z_\beta
%    &= x_i.(x_k.x_j.Z_\beta + [x_j, x_k].Z_\beta) \\
%    &= x_i.x_k.x_j.Z_\beta + x_i.[x_j, x_k].Z_\beta
%   \end{align*}
%   Because $k < j$ and $k \leq \beta$ it follows from the previous discussed cases that
%   \[
%    x_i.x_k.(x_j.Z_\beta)
%    = x_k.x_i.(x_j.Z_\beta) + [x_i, x_k].(x_j.Z_\beta).
%   \]
%   Combining the above results in the equality
%   \[
%    x_i.x_j.Z_\alpha
%    = x_k.x_i.x_j.Z_\beta + [x_i, x_k].x_j.Z_\beta + x_i.[x_j, x_k].Z_\beta
%   \]
%   By switching $i$ and $j$ in the above calculations it also follows that 
%   \[
%    x_j.x_i.Z_\alpha
%    = x_k.x_j.x_i.Z_\beta + [x_j, x_k].x_i.Z_\beta + x_j.[x_i, x_k].Z_\beta
%   \]
%   By using that condition \ref{enum: pbw representation of lie algebra} holds for $\varphi_n$ it follows from these two equalities and the Jacobi identity that
%   \begin{align*}
%         x_i.x_j.Z_\alpha - x_j.x_i.Z_\alpha 
%    =&\, x_k.x_i.x_j.Z_\beta + [x_i, x_k].x_j.Z_\beta + x_i.[x_j, x_k].Z_\beta \\
%     &\, - x_k.x_j.x_i.Z_\beta - [x_j, x_k].x_i.Z_\beta - x_j.[x_i, x_k].Z_\beta \\
%    =&\, x_k.[x_i, x_j].Z_\beta + [[x_i, x_k], x_j].Z_\beta +  [x_i,[x_j,x_k]].Z_\beta \\
%    =&\, x_k.[x_i, x_j].Z_\beta - [x_k, [x_i, x_j]].Z_\beta
%    =    [x_i, x_j].x_k.Z_\beta
%    =    [x_i, x_j].Z_\alpha.
%   \qedhere
%   \end{align*}
%  \end{proof}
%  This finishes the proof. 
% \end{proof}
% 
% 
% \begin{corollary}
%  Let $\glie$ be a Lie algebra and $\h, \n \subseteq \g$ Lie subalgebras with $\glie = \h \oplus \n$ as vector spaces. Then the map
%  \[
%   \Univ(\h) \tensor \Univ(\h) \to \Univ(\glie), \quad x \tensor y \mapsto xy
%  \]
%  is a isomorphism of vector spaces.
% \end{corollary}
% \begin{proof}
%  Let $(x_i)_{i \in I}$ is a basis of $\h$ and $(x_j)_{j \in J}$ a basis of $\n$. Then $(x_k)_{k \in K}$ for the index set $K \defined I \dotcup J$ is a basis of $\glie$. Then the statement follows directly from the concrete PBW theorem.
% \end{proof}
% 
% 
% 
% \subsection{The Poincar\'{e}--Birkhoff--Witt theorem (abstract version)}
% 
% 
% \begin{theorem}[PBW (abstract version)] \label{thrm: pbw abstract}
%  Let $\glie$ be a Lie algebra over $k$ and denote by $\pi$ the canonical projection
%  \[
%   \pi \colon T(\g) \to \Univ(\glie), \quad
%   x_1 \tensor \dotsb \tensor x_n \mapsto x_1 \dotsm x_n
%   \quad \text{for all $x_1, \dotsc, x_n \in \g$}.
%  \]
%  Then the two homomorphisms of graded \algebras{$\kf$} $\gr(\pi) \colon T(\g) \to \gr(\Univ(\glie))$ and
%  \[
%   \pi' \colon T(\g) \to S(\g), \quad 
%   x_1 \tensor \dotsb \tensor x_n \mapsto x_1 \dotsm x_n
%   \quad \text{for all $x_1, \dotsc, x_n \in \g$}
%  \]
%  have the same kernel and thus induce an isomorphism of graded algebras
%  \begin{equation}\label{eqn: induced isomorphism of graded algebras}
%   \varphi \colon S(\g) \to \gr(\Univ(\glie)), \quad
%   x_1 \dotsm x_n \mapsto [x_1 \dotsm x_n]
%   \quad \text{for all $x_1, \dotsc, x_n \in \g$}.
%  \end{equation}
% \end{theorem}
% 
% 
% \begin{remark}
%  Notice that the two multiplications in \eqref{eqn: induced isomorphism of graded algebras} live in different \algebras{$\kf$}.
% \end{remark}
% 
% 
% \begin{proposition}
%  The concrete version and the abstract versions of the PBW-theorem are equivalent.
% \end{proposition}
% \begin{proof}
%  For $x, y \in \g$ it follows from the definition of $\gr(\pi)$ that
%  \[
%   \gr(\pi)(x \tensor y - y \tensor x) = [xy-yx] \in \gr(\Univ(\glie))_2,
%  \]
%  with representative $xy-yx \in \Univ(\glie)_{(2)}$. By the definition of $\Univ(\glie)$ it follows that already \mbox{$xy-xy = [x,y] \in \Univ(\glie)_{(1)}$}. Hence it follows for the residue class of $xy-yx$ in $\gr(\Univ(\glie))_2 = \gr(\Univ(\glie))_{(2)}/\gr(\Univ(\glie))_{(1)}$ that $[xy-yx] = [[x,y]] = [0] = 0$. Hence $\gr(\pi)(x \tensor y - y \tensor x) = 0$.
%  
%  As the kernel of $\pi'$ is generated by the element $x \tensor y - y \tensor x$ with $x,y \in \g$ it follows that $\pi'$ factorizes through a homomorphism of graded \algebras{$\kf$} $\varphi$ as in Theorem~\ref{thrm: pbw abstract}.
%  
%  (concrete $\Rightarrow$ abstract) The algebra $\Univ(\glie)$ has a basis $(x_\alpha \mid \alpha \in \mc{I})$ It follows that $\gr(\Univ(\glie))_n$ has a basis given by the residue classes $([x_\alpha] \mid \alpha \in \mc{I}_n)$. The linear subspace $S(\g)_n$ has a basis $(x_1 \dotsm x_n \mid (i_1, \dotsc, i_n) \in \mc{I}_n)$ which is mapped by $\varphi_n$ to the above basis of $\gr(\Univ(\glie))_n$. Hence $\varphi_n$ is an isomorphism for every $n \in \N$, which is why $\varphi$ is an isomorphism.
%  
%  (abstract $\Rightarrow$ concrete) As $(x_\alpha \mid \alpha \in \mc{I})$ generates $\Univ(\glie)$ as a vector spaces by Lemma~\ref{lem: pbw concrete generating part} it sufficies to show that this collection is linearly independent. Suppose otherwise. Then there exists some minimal $n \in \N$ such that $(x_\alpha \mid m \leq n, \alpha \in \mc{I}_m)$ is linearly dependent. Hence there exists a non-trivial linear combination
%  \[
%   0 = \sum_{m=0}^n \sum_{\alpha \in \mc{I}_m} \lambda_\alpha x_\alpha
%  \quad
%  \text{where $\lambda_\alpha = 0$ for all but finitely many $\alpha \in \bigcup_{m=0}^n \mc{I}_m$}.
%  \]
%  From this it follows that
%  \[
%   0
%   = \sum_{m=0}^n \sum_{\alpha \in \mc{I}_m} \lambda_\alpha x_\alpha
%   \equiv \sum_{\alpha \in \mc{I}_n} \lambda_\alpha x_\alpha
%   \mod \Univ(g)_{(n-1)}
%  \]
%  and hence that the equality $\sum_{\alpha \in \mc{I}_n} \lambda_\alpha [x_\alpha] = 0$ holds in $\gr(\Univ(\glie))_n = \Univ(\glie)_{(n)}/\Univ(\glie)_{(n-1)}$. By the minimality of $n$ it follows that this is a non-trivial linear combination in $\gr(\Univ(\glie))_n$, so $([x_\alpha] \mid \alpha \in \mc{I}_n)$ in $\gr(\Univ(\glie))_n$ is linearly dependent.
%  
%  By assumption $\varphi$ is a homomorphism of graded \algebras{$\kf$} and therefore $\varphi_n$ is an isomorphism of vector spaces. As the basis $(x_{i_1} \dotsm x_{i_n} \mid (i_1, \dotsc, i_n) \in \mc{I}_n)$ of $S(\g)_n$ is mapped by $\varphi_n$ bijectively to $([x_\alpha] \mid \alpha \in \mc{I}_n)$ it follows that this is a basis of $\gr(\Univ(\glie))_n$, contradicting the linearly dependency.
% \end{proof}
% 
% 
% \begin{corollary}
%  The {\ua} $\Univ(\glie)$ is an integral domain.
% \end{corollary}
% \begin{proof}
%  Because $\gr(\Univ(\glie)) \cong S(\g)$ is in integral domain the statement follows from Proposition~\ref{prop: associated graded algebra and zero divisors}.
% \end{proof}
% 
% 
% 
% 
% 
% \section{Free Lie algebras}
% 
% 
% \begin{definition}
%  Let $X$ be a set. A \emph{free $k$-Lie algebra on $X$} is a Lie algebra $F(X)$ together with a map $\iota \colon X \to F(X)$ such that for every Lie algebra $\glie$ and map $\phi \colon X \to \g$ there exists a unique homomorphism of Lie algebras $\Phi \colon F(X) \to \g$ with $\phi = \Phi \circ \iota$, i.e.\ making the following diagram commute:
%  \[
%    \begin{tikzcd}
%      X
%      \arrow{dr}[above right]{\phi}
%      \arrow{d}[left]{\iota}
%      &
%      {}
%      \\
%      F(X)
%      \arrow[dashed]{r}[below]{\Phi}
%      &
%      \g
%    \end{tikzcd}
%  \]
% \end{definition}
% 
% 
% \begin{remark}
%  As usual with free objects it follows that any two free Lie algebras over a set $X$ are unique up to unique isomorphism, i.e.\ if $F(X)_1$ with $\iota_1 \colon X \to F(X)_1$ and $F(X)_2$ with $\iota_2 \colon X \to F(X)_2$ are two free Lie algebras over $X$ then there exists a unique isomorphism of Lie algebras $\varphi \colon F(X)_1 \to F(X)_2$ making the following diagram commute:
%  \[
%    \begin{tikzcd}
%      {}
%      &
%      X
%      \arrow{dl}[above left]{\iota_1}
%      \arrow{dr}[above right]{\iota_2}
%      &
%      {}
%      \\
%      F(X)_1
%      \arrow[dashed]{rr}[below]{\varphi}
%      &
%      {}
%      &
%      F(X)_2
%    \end{tikzcd}
%  \]
%  We will therefore always talk about \emph{the} free $k$-Lie algebra over $X$.
% \end{remark}
% 
% 
% \begin{lemma}\label{lem: existince of free Lie algebras}
%  Let $X$ be a set. Then there exists a free Lie algebra over $X$.
% \end{lemma}
% \begin{proof}
%  Let $A(X)$ be the free (unitary and associative) \algebra{$\kf$} over $X$ (which can be constructed as $T(kX)$, i.e.\ the tensor algebra over the free vector space $kX$ with basis $X$). Let $F(X)$ be the Lie subalgebra of $A(X)$ generated by $X$, i.e.
%  \[
%   F(X) = \bigcap \{\g \mid \text{$\glie \subseteq A(X)$ is a Lie subalgebra with $X \subseteq \g$}\}.
%  \]
%  Let $\glie$ be a $k$-Lie algebra and $\phi \colon X \to \g$ a map. By the universal property of the free \algebra{$\kf$} the map $\phi$ induces a homomorphism of \algebras{$\kf$} $\theta \colon A(X) \to \Univ(\glie)$ making the following diagram commute, where the vertical maps are the canonical inclusions:
%  \[
%    \begin{tikzcd}
%      X
%      \arrow{r}[above]{\phi}
%      \arrow{d}
%      &
%      \g
%      \arrow{d}
%      \\
%      F(X)
%      \arrow[dashed]{r}[above]{\theta}
%      &
%      \Univ(\glie)
%    \end{tikzcd}
%  \]
%  As $\theta(X) = \phi(X) \subseteq \g$ it follows that $X \subseteq \theta^{-1}(\g)$. Because $\theta^{-1}(\g)$ is a Lie subalgebra of $F(X)$ containing $X$ it follows that $\theta^{-1}(\g) = F(X)$ and therefore $\theta(F(X)) \subseteq \g$. Hence $\theta$ restricts to a map $\Phi \colon F(X) \to \g$. Because $\theta$ is a homomorphism of \algebras{$\kf$} it is in particular a homomorphism of Lie algebras and therefore the same goes for $\Phi$. This shows the existence.
%  
%  For the uniqueness notice that $F(X)$ is by definition generated by $X$ and hence any homomorphism of Lie algebras $\Psi \colon F(X) \to \g$ is uniquely determinad by the restriction $\Psi|_X$.
% \end{proof}
% 
% 
% \begin{remark}
%  The {\ua} is used in the proof of Lemma \ref{lem: existince of free Lie algebras} to ensure that any Lie algebra can be embedded into a \algebra{$\kf$} as a Lie subalgebra.
% \end{remark}
% 
% 
% \begin{remark}
%  Using the concept of free Lie algebras one can define Lie algebras by giving a set of generators $X$ and a set of relations $R \subseteq F(X)$. As an example the Lie algebra $\sll(k)$ can be defined by the generators $R \defined \{e,h,f\}$ with $e$, $h$, $f$ being pairwise different and the relations $R = \{[h,e]-2e, [h,f]+2f, [e,f]-h\}$, which can also be written as $[h,e] = 2e$, $[h,f] = -2f$ and $[e,f] = h$ as usual.
%  
%  More generally $\sll_{n+1}(k)$ for $n \geq 1$ can be defined as the Lie algebra generated by the $3n$ elements $\{e_i, f_i, h_i \mid i = 1, \dotsc, n\}$ together with the relations
%  \begin{align*}
%   [h_i, h_j] &= 0, \\
%   [h_i, e_j] &= a_{ij} e_j, \\
%   [h_i, f_j] &= -a_{ij} f_j, \\
%   [e_i, f_j] &= \delta_{ij} h_i,
%  \end{align*}
%  for all $i,j = 1, \dotsc, n$ and
%  \[
%   \ad(e_i)^{1-a_{ij}}(e_j) = 0
%   \quad\text{and}\quad
%   \ad(f_i)^{1-a_{ij}}(f_j) = 0
%   \quad \text{for $1 \leq i \neq j \leq n$},
%  \]
%  where the numbers $a_{ij}$ are for all $i,j = 1, \dotsc, n$ defined as
%  \[
%   a_{ij} =
%   \begin{cases}
%    \phantom{-}2 & \text{if $i = j$}, \\
%              -1 & \text{if $|i-j| = 1$}, \\
%    \phantom{-}0 & \text{otherwise}.
%   \end{cases}
%  \]
% \end{remark}
% 
% 
% \begin{lemma}
%  Let $X$ be any set and $F(X)$ the free Lie algebra over $X$. Then $\Univ(F(X))$ is the free \algebra{$\kf$} over $X$, where the canonical inclusion $X \inc \Univ(F(X))$ is given by composition of the canonical inclusions $X \inc F(X)$ and $F(X) \inc \Univ(F(X))$.
% \end{lemma}
% \begin{proof}
%  It will be shown that $\Univ(F(X))$ together with the canonical inclusion $X \inc \Univ(F(X))$ satisfies universal property of the free \algebras{$\kf$}. Let $A$ be a \algebra{$\kf$} and $\phi \colon X \to A$ a map. Then $\phi$ induces a unique homomorphism of Lie algebras $\psi \colon F(X) \to A$ by the universal property of the free Lie algebra. Then $\psi$ induces a unique homomorphism of \algebras{$\kf$} $\Psi \colon \Univ(F(X)) \to A$ by the universal property of the {\ua}. Hence the following diagram commutes, where the vertical maps denote the canonical inclusions.
%  \[
%    \begin{tikzcd}
%      X
%      \arrow{dr}[above right]{\phi}
%      \arrow{d}
%      &
%      {}
%      \\
%      F(X)
%      \arrow{r}[above]{\phi}
%      \arrow{d}
%      &
%      A
%      \\
%      \Univ(F(X))
%      \arrow{ur}[below right]{\Psi}
%      &
%      {}
%    \end{tikzcd}
%  \]
%  Leaving out the middle part of the diagram shows the existence. The uniqueness of $\Psi$ follows from the uniqueness of $\psi$.
% \end{proof}
% 
% 
% 
% 
% 
% %TODO: Hopf algebra structure
% 
% 
% 
% 
% 
% \section{Casimir elements}
% For this subsection we additionaly assume that $\glie$ is finite-dimensional. We also fix some bilinear form $\beta \colon \glie \times \glie \to k$ which is associative and non-degenerate.
% 
% 
% \begin{definition}\label{defi: definition of Casimir element}
%  Let $\varphi_1 \colon \glie \tensor \g^* \to \End_k(\g)$ and $\varphi_2 \colon \glie \to \g^*$ be the isomorphisms of vector spaces defined by
%  \[
%   \varphi_1(x \tensor \phi)(y) = \phi(y) x
%   \quad\text{and}\quad
%   \varphi_2(x) = \beta(x, \cdot)
%   \quad\text{for all $x,y \in \g$ and $\phi \in \g^*$}.
%  \]
%  Then the image of $1$ under the map
%  \begin{equation}\label{eqn: Casimir without coordinates}
%   k
%   \xrightarrow{\lambda \mapsto \lambda \id_\g}
%   \End_k(\g)
%   \xrightarrow{\varphi_1^{-1}}
%   \glie \tensor \g^*
%   \xrightarrow{\id_\g \tensor \varphi_2^{-1}}
%   \glie \tensor \g
%   \xrightarrow{x \tensor y \mapsto x y}
%   \Univ(\glie)
%  \end{equation}
%  is called the \emph{Casimir element of $\beta$} and denoted by $C_\beta$.
% \end{definition}
% 
% 
% \begin{lemma}
%  The Casimir element $C_\beta$ in central in $\Univ(\glie)$, i.e.\
%  \[
%   x C_\beta = C_\beta x \quad \text{for every $x \in \Univ(g)$}.
%  \]
% \end{lemma}
% \begin{proof}
%  Let $\varphi_1$ and $\varphi_2$ as in Definition \ref{defi: definition of Casimir element}.
%  
%  Because $\Univ(\glie)$ is generated by $\glie$ as a \algebra{$\kf$} it sufficies to show $C_\beta$ commutes with every $x \in \g$. Hence it is to show that
%  \[
%   [x,C_\beta] = 0 \quad \text{for every $x \in \g$},
%  \]
%  where $[\cdot,\cdot]$ denotes the Lie bracket in $\Univ(\glie)$. To see this notice that in \eqref{eqn: Casimir without coordinates} every map is a homomorphism of representations of $\glie$, where $\glie$ acts trivially on $k$, i.e. $x.\lambda = 0$ for every $x \in \g$ and $\lambda \in \g$.
%  
%  That the first map $k \to \End_k(\g)$ is a homomorphism of representations follows from the fact that $\glie$ acts trivially on $k$ and also trivially on the one-dimensional subspace $k \id_\g \subseteq \End_k(\g)$.
%  
%  That $\varphi_1$ is an isomorphism of representations is known from Propositon \ref{prop: list of homomorphism of representations}.
%  
%  That the third map $\glie \tensor \g^* \to \glie \tensor \g$ is a homomorphism of representations follows from Proposition \ref{prop: list of homomorphism of representations}, because the identity $\id_\g$ is a homomorphism of representations and and the isomorphism $\varphi_2$ is one by the associativity of $\beta$, as seen in Lemma \ref{lem: associative bilinear form induces homomorphism of representations}.
%  
%  That the fourth map $\psi \colon \glie \tensor \glie \to \Univ(\glie), x \tensor y \mapsto xy$ is a homomorphism of representations follows from direct calculation, because for all $x,y,z \in \g$
%  \begin{align*}
%   \psi(x.(y \tensor z))
%   &= \psi((x.y) \tensor z + y \tensor (x.z))
%   = (x.y)z + y(x.z) \\
%   &= [x,y]z + y[x,z]
%   = xyz - yxz + yxz - yzx
%   = xyz - yzx \\
%   &= [x,yz]
%   = x.(yz)
%   = x.\psi(y \tensor z).
%  \end{align*}
%  
%  Because every map in \eqref{eqn: Casimir without coordinates} is a homomorphism of representations it follows that their composition $\phi \colon k \to \Univ(\glie)$ is also a homomorphism of representations. Definition \ref{defi: definition of Casimir element} is then equivalent to $\phi(1) = C_\beta$. Because $\glie$ acts trivially on $k$ and $\phi$ is a homomorphism of representations it follows that $\glie$ also acts trivially on the span of $C_\beta$. In particular
%  \[
%   0 = x.C_\beta = [x,C_\beta] \quad \text{for every $x \in \g$}.
%   \qedhere
%  \]
% \end{proof}
% 
% 
% \begin{corollary}\label{cor: Casimir homomorphism of a representation}
%  Let $V$ be a representation of $\glie$ and $\Univ(\glie) \times V \to V, (x,v) \mapsto x \cdot v$ the corresponding $\Univ(\glie)$-module structure on $V$. Then the map
%  \[
%   C_\beta^V \colon V \to V, \quad v \mapsto C_\beta \cdot v = \sum_{i=1}^n x_i.x^i.v
%  \]
%  is an endomorphism of representations of $\glie$ (equivalently an endomorphism of $\Univ(\glie)$-modules).
% \end{corollary}
% \begin{proof}
%  Because $C_\beta \in Z(\Univ(\glie))$ it follows that for every $x \in \Univ(\glie)$ and $v \in V$
%  \[
%   x \cdot C_\beta^V(v)
%   = x \cdot C_\beta \cdot v
%   = C_\beta \cdot x \cdot v
%   = C_\beta^V(x \cdot v).
%   \qedhere
%  \]
% \end{proof}
% 
% 
% \begin{lemma}[Casimir in coordinates] \label{lem: casimir in coordinates}
%  Let $x_1, \dotsc, x_n$ be a basis of $\glie$ and $x^1, \dotsc, x^n$ the dual basis of $\glie$ with respect to $\beta$, i.e.\ $\beta(x_i, x^j) = \delta_{ij}$ for all $i,j = 1, \dotsc, n$. Then
%  \[
%   C_\beta = \sum_{i=1}^n x_i x^i.
%  \]
% \end{lemma}
% \begin{proof}
%  Let $\varphi_1$ and $\varphi_2$ as in Definition~\ref{defi: definition of Casimir element}. In \eqref{eqn: Casimir without coordinates} $1$ is mapped to $\id_\g$, which is then mapped to $\sum_{i=1}^n x_i \tensor x_i^*$, where $x_1^*, \dotsc, x_n^*$ denotes the dual basis of $\g^*$. As $\varphi_2(x^i) = x_i^*$ it follows that $\sum_{i=1}^n x_i \tensor x_i^*$ is then mapped to $\sum_{i=1}^n x_i \tensor x^i$, which is then further mapped to the element $\sum_{i=1}^n x_i x^i$ in $\Univ(\glie)$.
% \end{proof}
% 
% 
% \begin{remark}
%  Using Lemma \ref{lem: casimir in coordinates} it can be shown that $C_\beta$ is central in $\Univ(\glie)$ using coordinates: Let $x \in \g$ and $a_{ij}, b_{ij} \in k$ such that $[x,x_i] = \sum_{j=1}^n a_{ij} x_j$ and $[x,x^i] = \sum_{j=1}^n b_{ij} x^j$ for all $i = 1, \dotsc, n$. Then for all $i,j = 1, \dotsc, n$
%  \begin{align*}
%   a_{ij}
%   &= \sum_{k=1}^n a_{ik} \beta(x_k, x^j)
%   = \beta\left( \sum_{k=1}^n a_{ik} x_k, x^j \right)
%   = \beta([x, x_i], x^j)
%   = -\beta([x_i, x], x^j) \\
%   &= -\beta(x_i, [x, x^j])
%   = -\beta\left( x_i, \sum_{k=1}^n b_{jk} x^k \right)
%   = -\sum_{k=1}^n b_{jk} \beta(x_i, x^k)
%   = -b_{ji}.
%  \end{align*}
%  It follows that
%  \begin{align*}
%   x C_\beta - C_\beta x
%   &= \sum_{i=1}^n x x_i x^i - \sum_{i=1}^n x_i x^i x
%   = \sum_{i=1}^n [x, x_i] x^i - \sum_{i=1}^n x_i [x^i, x] \\
%   &= \sum_{i,j=1}^n a_{ij} x_j x^i + \sum_{i,j=1}^n b_{ij} x_i x^j
%   = \sum_{i,j=1}^n a_{ij} x_j x^i - \sum_{i,j=1}^n a_{ij} x_j x^i
%   = 0.
%  \end{align*}
% \end{remark}




