\section{Examples}


% TODO: UEA of semidirect product.


\subsection{Abelian Lie~Algebras}


\begin{examples}
	Let~$\glie$ be an abelian Lie~algebra.
	It follows from the explicit construction of the universal enveloping algebra~$\Univ(\glie)$ that
	\[
		\Univ(\glie)
		\cong
		\Tensor(\glie)/(x \tensor y - y \tensor x \suchthat x, y \in \glie)
		\cong
		\Symm(\glie) \,,
	\]
	with the canonical homomorphism of Lie~algebras from~$\glie$ to~$\Univ(\glie)$ corresponding to the inclusion map from~$\glie$ to~$\Symm(\glie)$\index{universal enveloping algebra!of an abelian Lie algebra}.

	This can also be seen more abstractly, as follows.
	
	We observe that if~$V$ is any~\vectorspace{$\kf$} and~$A$ is any~\algebra{$\kf$}, then a linear map~$f$ from~$V$ to~$A$ extends to a homomorphism of algebras from~$\Symm(V)$ to~$A$ if and only if the image of~$f$ is contained in a commutative subalgebra of~$A$, if and only if the image of~$f$ is commutative in~$A$.
	(By this we mean that any two elements of the image of~$f$ commute with each other.)
	It follows that we have for any~\algebra{$\kf$}~$A$ bijections
	\begin{align*}
		{}&
		\{ \textstyle \text{Lie~algebra homomorphisms~$\glie \to A$} \}
		\\
		\cong{}&
		\{ \textstyle \text{{\linear{$\kf$}} maps~$\glie \to A$ with commutative image} \}
		\\
		\cong{}&
		\{ \textstyle \text{algebra homomorphisms~$\Symm(\glie) \to A$} \} \,.
	\end{align*}
	These bijections are natural in~$A$.
	This shows that the symmetric algebra~$\Symm(\glie)$ together with the inclusion map from~$\glie$ to~$\Symm(\glie)$ satisfies the universal property of the universal enveloping algebra of~$\glie$.
\end{examples}


\begin{example}
	We find for~$\glie = 0$ that~$\Univ(\glie) = \kf$.
\end{example}


\begin{definition}
	\leavevmode
	\begin{enumerate}
		\item
			Let~$A$ be an algebra.
			An \defemph{augumentation}\index{augumentation} of~$A$ is a homomorphism of algebras~$\varepsilon$\glsadd{augumentation} from~$A$ to~$\kf$.
		\item
			An \defemph{augumented~\algebra{$\kf$}}\index{augumented algebra} is a~\algebra{$\kf$}~$A$ together with an augumentation of~$A$.
		\item
			Let~$(A, \varepsilon)$\glsadd{augumented algebra} be an augumented algebra.
			The kerrnel of~$\varepsilon$ is the \defemph{augumentation ideal}\index{augumentation ideal} of~$A$.
	\end{enumerate}
\end{definition}


\begin{proposition}
	\label{decomposition for augumented algebra}
	Let~$(A, \varepsilon) $ be an augumented~\algebra{$\kf$}.
	Then~$A = \kf \oplus \ker(\varepsilon)$ as vector spaces.
\end{proposition}


\begin{proof}[First proof]
	We have~$\varepsilon(1) = 1$ because~$\varepsilon$ is a homomorphism of algebras.
	This shows that~$1$ is not contained in the augumentation ideal, and that the augumentation~$\varepsilon$ is nonzero.
	The image of~$\varepsilon$ is thus one-dimensional, whence the kernel of~$\varepsilon$ has codimension~$1$ in~$A$.
	It follows that~$A = \gen{1}_{\kf} \oplus \ker(\varepsilon) = \kf \oplus \ker(\varepsilon)$.
\end{proof}


\begin{proof}[Second proof]
	Let~$\eta$ be the inclusion from~$\kf$ to~$A$.
	The composito~$\varepsilon \circ \eta$ is the identity on~$\kf$ whence the composite~$\eta \circ \varepsilon$ is idempotent.
	It follows that
	\[
		A
		=
		\im(\eta \circ \varepsilon)
		\oplus \ker(\eta \circ \varepsilon) \,.
	\]
	It follows from the surjectivity of~$\varepsilon$ that~$\im(\eta \circ \varepsilon) = \im(\eta) = \kf$ and from the injectivity of~$\eta$ that~$\ker(\eta \circ \varepsilon) = \ker(\varepsilon)$.
\end{proof}


\begin{corollary}
	Every augumented~\algebra{$\kf$} is nonzero.
	\qedhere
\end{corollary}


\begin{construction}
	\label{construction of counit}
	Let~$\glie$ be a Lie~algebra.
	We have a unique homomorphism of Lie~algebras from~$\glie$ to the zero Lie~algebra.
	This homomorphism of Lie~algebras induces a homomorphism of algebras
	\[
		\varepsilon
		\colon
		\Univ(\glie)
		\to
		\kf \,.
	\]
	This homomorphism of algebras is uniquely determined by the condition
	\[
		\varepsilon( \class{x} ) = 0
		\qquad
		\text{for every~$x \in \glie$}
	\]
	because~$\Univ(\glie)$ is generated as an algebra by the image of~$\glie$ in~$\Univ(\glie)$.
\end{construction}


\begin{definition}
	Let~$\glie$ be a Lie~algebra.
	The homomorphism of algebras~$\varepsilon$\glsadd{counit} from~$\Univ(\glie)$ to~$\kf$ from \cref{construction of counit} is the \defemph{counit}\index{counit} of~$\glie$.
	Its makes~$\Univ(\glie)$ into an augumented algebra.
\end{definition}


\begin{remark}
	Let~$\glie$ be a Lie~algebra.
	We can regard the vector space~$\kf$ as the trivial representation of~$\glie$, and thus as a~\module{$\Univ(\glie)$}.
	This~\module{$\Univ(\glie)$} structure can also be explained with help of the counit~$\varepsilon$, because~$\varphi$ is a homomorphism of algebras from~$\Univ(\glie)$ to~$\kf = \End_{\kf}(\kf)$.
\end{remark}


\begin{proposition}
	\label{augumentation ideal is spanned by monomials}
	The augumentation ideal of~$\Univ(\glie)$ is spanned as a vector space by all the monomials~$\class{x_1} \dotsm \class{x_n}$ with~$n \geq 1$ and~$x_1, \dotsc, x_n \in \glie$.
\end{proposition}


\begin{proof}
	Let~$\varepsilon$ be the counit of~$\Univ(\glie)$, let~$J$ be the linear subspace of~$\Univ(\glie)$ spanned by all monomials~$\class{x_1} \dotsm \class{x_n}$ with~$n \geq 0$,~$x_1, \dotsc, x_n \in \glie$.
	We have
	\[
		\varepsilon( \class{x_1} \dotsm \class{x_n} )
		=
		\varepsilon( \class{x_1} ) \dotsm \varepsilon( \class{x_n} )
		=
		0 \dotsm 0
		=
		0
	\]
	for each such monomial, whence the linear space~$J$ is contained in~$\ker(\varepsilon)$.
	We know on the other hand that~$\Univ(\glie)$ is generated by the image of~$\glie$ in~$\Univ(\glie)$ as an algebra.
	This means that the monomials
	\[
		\class{x_1}  \dotsm \class{x_n}
		\qquad
		\text{with~$n \geq 0$ and~$x_1, \dotsc, x_n \in \glie$}
	\]
	span the algebra~$\Univ(\glie)$ as a vector space.
	We thus have~$\Univ(\glie) = \kf + J$.
	Together with the decomposition~$\Univ(\glie) = \kf \oplus \ker(\varepsilon)$ and the inclusion~$U \subseteq \ker(\varepsilon)$ we find that~$J = \ker(\varepsilon)$.
\end{proof}



\subsection{Opposite Lie~Algebra}

\begin{example}
	\label{uea of opposite by first principles}
	\leavevmode
	\begin{enumerate}
		\item
			Let~$\glie$ be a Lie~algebra.
			We have for every~{\algebra{$\kf$}}~$A$ bijections
			\begin{align*}
				{}&
				\{ \text{algebra homomorphisms~$\Univ(\glie^\op) \to A$   } \}
				\\
				\cong{}&
				\{ \text{Lie~algebra homomorphisms~$\glie^\op \to A$} \}
				\\
				={}&
				\{ \text{Lie~algebra homomorphisms~$\glie \to A^\op$} \}
				\\
				\cong{}&
				\{ \text{algebra homomorphisms~$\Univ(\glie) \to A^\op$} \}
				\\
				={}&
				\{ \text{algebra homomorphisms~$\Univ(\glie)^\op \to A$} \}
			\end{align*}
			that are natural in~$A$.%
			\footnote{
				We use here implicitely that taking the underlying Lie~algebra of a~{\algebra{$\kf$}} commutes with taking opposites.}
				It follows from Yoneda’s~lemma that~$\Univ(\glie^\op) \cong \Univ(\glie)^\op$\index{universal enveloping algebra!of the opposite Lie~algebra}.
			The canonical homomorphism of Lie~algebras from~$\glie^{\op}$ to~$\Univ(\glie^{\op})$ corresponds to the homomorphism from~$\glie^{\op}$ to~$\Univ(\glie)^{\op}$ given by~$\class{x^{\op}} \mapsto \class{x}{}^{\,\op}$ for every~$x \in \glie$.

			We can also derive the above isomorphism in a more explicit way.
			Indeed, the map
			\[
				\sigma
				\colon
				\glie^{\op}
				\to
				\glie \,,
				\quad
				x^{\op}
				\mapsto
				x
			\]
			is an anti-isomorphism of Lie~algebras, and thus induces the anti-isomorphism of algebras~$\Univ(\sigma)$ from~$\Univ(\glie^{\op})$ to~$\Univ(\glie)$.
			We can regard this anti-isomorphism~$\Univ(\sigma)$ as an isomorphism of algebras~$\Phi$ from~$\Univ(\glie^{\op})$ to~$\Univ(\glie)^{\op}$.
			This isomorphism is uniquely determined by
			\[
				\Phi\Bigl( \class{x^{\op}} \Bigr)
				=
				- \class{x}
				\qquad
				\text{for every~$x \in \glie$.}
			\]
		\item
			The map
			\[
				\tau
				\colon
				\glie^{\op}
				\to
				\glie \,,
				\quad
				x^{\op}
				\mapsto
				-x
			\]
			is an isomorphism of Lie~algebras and thus induces an isomorphism of algebras~$\Psi \defined \Univ(\tau)$ from~$\Univ(\glie^{\op})$ to~$\Univ(\glie)$.
			This isomorphism is uniquely determined by
			\[
				\Univ(\tau)\Bigl( \class{x^{\op}} \Bigr)
				=
				- \class{x} \,,
				\qquad
				\text{for every~$x \in \glie$.}
			\]
	\end{enumerate}
\end{example}




%\begin{remark}
%	Let~$\glie$ be a Lie~algebra.
%	For every representation~$M$ of~$\glie$ its dual~$M^*$ becomes again a representation of~$\glie$ via the action
%	\[
%		(x \act \varphi)(m)
%		=
%		- \varphi(m)
%	\]
%	for all~$x \in \glie$,~$\varphi \in M^*$,~$m \in M$.
%	In other words, for every~\module{$\Univ(\glie)$}~$M$ its dual~$M^*$ becomes again a~\module{$\Univ(\glie)$}.
%
%	This can also be explained via the antipode.
%	Let~$M$ be a~\module{$\Univ(\glie)$}.
%	The the dual~$M^*$ becomes a right~\module{$\Univ(\glie)$} via the multiplication
%	\[
%		(\varphi \cdot y)(m)
%		=
%		\varphi(ym)
%	\]
%	for all~$y \in \Univ(\glie)$,~$\varphi \in M^*$,~$m \in M$.
%	This right~\module{$\Univ(\glie)$} structure on~$M^*$ corresponds to a left~\module{$\Univ(\glie)^{\op}$} structure on~$M^*$ given by
%	\[
%		y^{\op} \cdot \varphi
%		=
%		\varphi \cdot y
%	\]
%	for all~$y \in \Univ(\glie)$,~$\varphi \in M^*$.
%	By using the isomorphism of algebras~$S$ from~$\Univ(\glie)$ to~$\Univ(\glie)^{\op}$ we can pull back this~\module{$\Univ(\glie)^{\op}$} structure to a~\module{$\Univ(\glie)$} structure given by
%	\[
%		y \cdot \varphi
%		=
%		S(y) \cdot \varphi
%	\]
%	for all~$y \in \Univ(\glie)$,~$\varphi \in M^*$.
%
%	For every element~$x$ of~$\glie$ we have
%	\[
%		(\class{x} \cdot \varphi)(m)
%		=
%		( S(\class{x}) \cdot \varphi )(m)
%		=
%		( - \class{x}^{\,\op} \cdot \varphi )(m)
%		=
%		( \varphi \cdot (- \class{x}) )(m)
%		=
%		\varphi( - \class{x} \cdot m )
%		=
%		- \varphi( \class{x} \cdot m )
%	\]
%	for all~$\varphi \in M^*$,~$m \in M$.
%	Both constructed~\module{$\Univ(\glie)$} structures on~$M^*$ hence coincide.
%\end{remark}



\subsection{Direct Sum of Lie~algebras}

\begin{recall}
	\label{homomorphism out of a tensor product}
	Let~$A$ and~$B$ be two~{\algebras{$\kf$}}.
	Then the inclusion maps
	\begin{alignat*}{2}
		\Iota_A
		&\colon
		A
		\to
		A \tensor B \,,
		&
		\quad
		a
		&\mapsto
		a \tensor 1 \,,
		\\
		\Iota_B
		&\colon
		B
		\to
		A \tensor B \,,
		&
		\quad
		b
		&\mapsto
		1 \tensor b
	\end{alignat*}
	are injective homomorphisms of algebras (unless~$A$ or~$B$ is the zero algebra, in which case~$A \tensor B$ is again the zero algebra)
	We may therefore identify the algebras~$A$ and~$B$ with the associated subalgebras~$A \tensor 1$ and~$1 \tensor B$ of~$A \tensor B$.
	We note that~$A$ and~$B$ commute in~$A \tensor B$ because
	\[
		\Iota_A(a) \Iota_B(b)
		=
		(a \tensor 1) (b \tensor 1)
		=
		a \tensor b
		=
		(b \tensor 1) (a \tensor 1)
		=
		\Iota_B(b) \Iota_A(a)
	\]
	for all~$a \in A$ and~$b \in B$.
	
	Let now~$C$ be another~\algebra{$\kf$}.
	
	If~$\Phi$ is a homomorphism of algebras from~$A \tensor B$ to~$C$, then the restrictions~$\Phi_A$ and~$\Phi_B$ given by~$\Phi_A \defined \Phi \circ \Iota_A$ and~$\Phi_B \defined \Phi \circ \Iota_B$ are again homomorphisms of algebras.
	The images of~$\Phi_A$ and~$\Phi_B$ commute with each other in~$C$ because~$A$ and~$B$ commute in~$A \tensor B$.
	Indeed, we have
	\[
		[ \Phi_A(a), \Phi_B(b) ]
		=
		[ \Phi(\Iota_A(a)), \Phi(\Iota_B(b)) ]
		=
		\Phi( [ \Iota_A(a), \Iota_B(b) ] )
		=
		\Phi( 0 )
		=
		0
	\]
	for all~$a \in A$ and~$b \in B$.
	
	Suppose on the other hand that~$\Psi_A$ is a homomorphism of algebras from~$A$ to~$C$ and that~$\Psi_B$ is a homomorphism of algebras from~$B$ to~$C$.
	There exists a unique linear map~$\Psi$ from~$A \tensor B$ to~$C$ given by
	\[
		\Psi(a \tensor b)
		=
		\Psi_A(a) \Psi_B(b)
	\]
	for all~$a \in A$ and~$b \in B$.
	Suppose that the images of~$\Psi_A$ and~$\Psi_B$ commute with each other in~$C$.
	The linear map~$\Psi$ is then again an homomorphism of algebras because
	\begin{align*}
		\Psi(a_1 \tensor b_1) \Psi(a_2 \tensor b_2)
		&=
		\Psi_A(a_1) \Psi_B(b_1) \Psi_A(a_2) \Psi_B(b_2)
		\\
		&=
		\Psi_A(a_1) \Psi_A(a_2) \Psi_B(b_1) \Psi_B(b_2)
		\\
		&=
		\Psi_A(a_1 a_2) \Psi_B(b_1 b_2)
		\\
		&=
		\Psi( (a_1 a_2) \tensor (b_1 b_2) )
		\\
		&=
		\Psi( (a_1 \tensor b_1) (a_2 \tensor b_2) )
	\end{align*}
	for all~$a_1, a_2 \in A$ and~$b_1, b_2 \in B$, as well as
	\[
		\Psi( 1_{A \tensor B} )
		=
		\Psi( 1_A \tensor 1_B )
		=
		\Psi_A( 1_A ) \Psi_B( 1_B )
		=
		1_C \cdot 1_C
		=
		1_C \,.
	\]
	
	These above two constructions are mutually inverse and hence result in a {\onetoonetext} correspondence
	\begin{align*}
		\SwapAboveDisplaySkip
		\left\{
			\begin{tabular}{c}
				algebra homomorphisms \\
				$\Phi \colon A \tensor B \to C$
			\end{tabular}
		\right\}
		&\onetoone
		\left\{
			\begin{tabular}{c}
				$(\Phi_A, \Phi_B)$
			\end{tabular}
		\suchthat*
			\begin{tabular}{c}
				algebra homomorphisms   \\
				$\Phi_A \colon A \to C$ \\
				$\Phi_B \colon B \to C$ \\
				whose images commute
			\end{tabular}
		\right\}  \,,
		\\
		\Phi
		&\mapsto
		(\Phi \circ \Iota_A, \Phi \circ \Iota_B)  \,,
		\\
		\biggl( a \tensor b \mapsto \Phi_A(a) \Phi_B(b) \biggr)
		&\mapsfrom
		(\Phi_A, \Phi_B)  \,.
	\end{align*}
\end{recall}


\begin{example}
	\label{explicit isomorphism for uea of direct sum}
	Let~$\glie$ and~$\hlie$ be two Lie~algebras.
	We show in the following that
	\[
		\Univ(\glie \oplus \hlie)
		\cong
		\Univ(\glie) \tensor \Univ(\hlie) \,.
		\index{universal eneveloping algebra!of the direct sum of Lie algebras}
	\]
	The isomorphism from~$\Univ(\glie \oplus \hlie)$ to~$\Univ(\glie) \tensor \Univ(\hlie)$ is given on the algebra generators~$\class{(x,y)}$ with~$(x,y)$ in~$\glie \oplus \hlie$ by
	\[
		\class{(x,y)}
		\mapsto
		\class{x} \tensor 1 + 1 \tensor \class{y} \,.
	\]
	The inverse isomorphism from~$\Univ(\glie) \tensor \Univ(\hlie)$ to~$\Univ(\glie \oplus \hlie)$ is given on the simple tensors~$\class{x} \tensor \class{y}$ with~$x$ in~$\Univ(\glie)$ and~$y$ in~$\Univ(\hlie)$ by
	\[
		\class{x} \tensor \class{y}
		\mapsto
		\Univ( \iota_1 )( \class{x} )
		\cdot
		\Univ( \iota_2 )( \class{y} ) \,.
	\]
	Here we denote by~$\iota_1$ is the canonical homomorphism of Lie~algebras from~$\glie$ to~$\glie \oplus \hlie$ (i.e. the inclusion into the first sammand) and similarly by~$\iota_2$ is the canonical homomorphism of Lie~algebras from~$\hlie$ to~$\glie \oplus \hlie$ (i.e. the inclusion into the second summand).

	We present two ways in which the above isomorphism(s) can be derived.
	\begin{itemize}
		\item
			It follows from \cref{homomorphism out of direct sum} and \cref{homomorphism out of a tensor product} that we get for every~\algebra{$\kf$}~$A$ bijections
			\begin{align*}
				{}&
				\left\{
					\begin{tabular}{c}
						algebra homomorphisms \\
						$\Phi \colon \Univ(\glie \oplus \hlie) \to A$
					\end{tabular}
				\right\}
				\\
				\cong{}&
				\left\{
					\begin{tabular}{c}
						Lie~algebra homomorphisms \\
						$\varphi \colon \glie \oplus \hlie \to A$
					\end{tabular}
				\right\}
				\\
				\cong{}&
				\left\{
					\begin{tabular}{c}
						$( \varphi_1, \varphi_2 )$
					\end{tabular}
				\suchthat*
					\begin{tabular}{c}
						Lie~algebra homomorphisms \\
						$\varphi_1 \colon \glie \to A$ and~$\varphi_2 \colon \hlie \to A$ \\
						whose images commute
					\end{tabular}
				\right\}
				\\
				\cong{}&
				\left\{
					\begin{tabular}{c}
						$(\Phi_1, \Phi_2)$
					\end{tabular}
				\suchthat*
					\begin{tabular}{c}
						algebra homomorphisms               \\
						$\Phi_1 \colon \Univ(\glie) \to A$  \\
						$\Phi_2 \colon \Univ(\hlie) \to A$  \\
						whose images commute
					\end{tabular}
				\right\}
				\\
				\cong{}&
				\left\{
					\begin{tabular}{c}
						 algebra homomorphims \\
						 $\Phi \colon \Univ(\glie) \tensor \Univ(\hlie) \to A$
					\end{tabular}
				\right\} \,.
			\end{align*}
			The claimed isomorphism therefore follows from Yoneda’s lemma.
		\item
			We can construct the isomorphism(s) more explicitely, as follows.

			We note that for the induced homomorphisms of algebras
			\begin{align*}
				\Univ(\iota_1) &\colon \Univ(\glie) \to \Univ(\glie \oplus \hlie) \,, \\
				\Univ(\iota_2) &\colon \Univ(\hlie) \to \Univ(\glie \oplus \hlie)
			\end{align*}
			the images of~$\Univ(\iota_1)$ and~$\Univ(\iota_2)$ commute.
			Indeed, the image of~$\Univ(\iota_1)$ in~$\Univ(\glie \oplus \hlie)$ is generated by the image of~$\glie \oplus 0$ in~$\Univ(\glie \oplus \hlie)$ as an algebra, and the image of~$\Univ(\iota_2)$ in~$\Univ(\glie \oplus \hlie)$ is generated by the image of~$0 \oplus \hlie$ in~$\Univ(\glie \oplus \hlie)$.
			But~$\glie \oplus 0$ and~$0 \oplus \hlie$ commute in~$\glie \oplus \hlie$, whence their images in~$\Univ(\glie \oplus \hlie)$ commute with each other.
			Therefore, the images of~$\Univ(\iota_1)$ and~$\Univ(\iota_2)$ commute with each other.

			It follows from this observation that the two homomorphisms of algebras~$\Univ( \iota_1 )$ and~$\Univ( \iota_2 )$ induce a homomorphism of algebras
			\[
				\Phi
				\colon
				\Univ(\glie) \tensor \Univ(\hlie)
				\to
				\Univ(\glie \oplus \hlie) \,.
			\]
			This homomorphism is given on simple tensors by
			\[
				\Phi(t \tensor u)
				=
				\Univ(\iota_1)(t) \cdot \Univ(\iota_2)(u)
			\]
			for all~$t \in \Univ(\glie)$ and~$u \in \Univ(\hlie)$.
			It holds in particular for all~$x$ in~$\glie$ and~$y$ in~$\hlie$ that
			\[
				\Phi(\class{x} \tensor \class{y})
				=
				\Univ(\iota_1)( \class{x} )
				\cdot
				\Univ(\iota_2)( \class{y} )
				=
				\class{ \iota_1(x) }
				\cdot
				\class{ \iota_2(y) }
				=
				\class{(x,0)} \cdot \class{(0,y)}  \,,
			\]
%      We observe that the map
%      \[
%        \psi'
%        \colon
%        \glie \times \hlie
%        \to
%        \Univ(\glie) \tensor \Univ(\hlie) \,,
%        \quad
%        (x,y)
%        \mapsto
%        \class{(x,0)} \tensor 1 + 1 \tensor \class{(0,y)} \,.
%      \]

			To construct the inverse~$\Psi$ of~$\Phi$ we observe that the equality
			\[
				\Psi\Bigl( \class{(x,0)} \Bigr)
				=
				\Psi\Bigl( \class{(x,0)} \cdot 1 \Bigr)
				=
				\Psi\Bigl( \Univ(\iota_1)( \class{x} ) \cdot \Univ(\iota_2)(1) \Bigr)
				=
				\Psi( \Phi( \class{x} \tensor 1 ) )
				=
				\class{x} \tensor 1
			\]
			has to hold for every~$x \in \glie$, and similarly
			
			\[
				\Psi\Bigl( \class{(0,y)} \Bigr)
				=
				1 \tensor \class{y}
			\]
			for every~$y \in \hlie$.
			It then follows that more generally
			\[
				\Psi\Bigl( \class{(x,y)} \Bigr)
				=
				\Psi\Bigl( \class{(x,0)} + \class{(0,y)} \Bigr)
				=
				\class{x} \tensor 1 + 1 \tensor \class{y}
			\]
			for every~$(x,y) \in \glie \oplus \hlie$.
			
			Motivated by these calculations we consider the map
			\[
				\psi
				\colon
				\glie \oplus \hlie
				\to
				\Univ(\glie) \tensor \Univ(\hlie) \,,
				\quad
				(x,y)
				\mapsto
				\class{x} \tensor 1 + 1 \tensor \class{y} \,.
			\]
			This map is a homomorphism of Lie~algebras because it is linear and we have
			\begin{align*}
				{}&
				[\psi((x_1, y_1)), \psi((x_2, y_2))]
				\\
				={}&
				[
					\class{x_1} \tensor 1 + 1 \tensor \class{y_1} \,,
					\class{x_2} \tensor 1 + 1 \tensor \class{y_2}
				]
				\\
				={}&
				\underbrace{ [\class{x_1} \tensor 1, \class{x_2} \tensor 1] }_{= [\class{x_1}, \class{x_2}] \tensor 1}
				+ \underbrace{ [\class{x_1} \tensor 1, 1 \tensor \class{y_2}] }_{=0}
				+ \underbrace{ [1 \tensor \class{y_1} \,, \class{x_2} \tensor 1] }_{=0}
				+ \underbrace{ [1 \tensor \class{y_1}, 1 \tensor \class{y_2}] }_{= 1 \tensor [\class{y_1}, \class{y_2}]}
				\\
				={}&
					[\class{x_1}, \class{x_2}] \tensor 1
				+ 1 \tensor [\class{y_1}, \class{y_2}]
				\\
				={}&
					\class{[x_1, x_2]} \tensor 1
				+ 1 \tensor \class{[y_1, y_2]}  \,.
				\\
				={}&
				\psi\bigl( ( [x_1, x_2], [y_1, y_2] ) \bigr)
				\\
				={}&
				\psi\bigl( [(x_1, y_1), (x_2, y_2)] \bigr)
			\end{align*}
			for all~$(x_1, y_1), (x_2, y_2) \in \glie \oplus \hlie$.
			It hence follows from the universal property of the {\ua}~$\Univ(\glie \oplus \hlie)$ that there exists a unique homomorphism of algebras~$\Psi$ from~$\Univ(\glie \oplus \hlie)$ to~$\Univ(\glie) \tensor \Univ(\hlie)$ that makes the triangular diagram
			\[
				\begin{tikzcd}[row sep = large]
					\glie \oplus \hlie
					\arrow{r}[above]{\psi}
					\arrow{d}[left]{\class{(\ph)}}
					&
					\Univ(\glie) \tensor \Univ(\hlie)
					\\
					\Univ(\glie \oplus \hlie)
					\arrow[dashed]{ur}[below right]{\Psi}
					&
					{}
				\end{tikzcd}
			\]
			commute.
			This homomorphism~$\psi$ is given by
			\[
				\Psi\Bigl( \class{(x,y)} \Bigr)
				=
				\class{x} \tensor 1 + 1 \tensor \class{y} \,.
			\]
			for every~$(x,y) \in \glie \oplus \hlie$.
			
			We now check that the two homomorphisms~$\Phi$ and~$\Psi$ are mutually inverse.
			We have on the one hand
			\begin{align*}
				\Phi\Bigl( \Psi\Bigl( \class{(x,y)} \Bigr) \Bigr)
				&=
				\Phi( \class{x} \tensor 1 + 1 \tensor \class{y} )
				\\
				&=
				\Phi( \class{x} \tensor 1 ) + \Phi( 1 \tensor \class{y} )
				\\
				&=
				\Univ(\iota_1)(\class{x}) \cdot \Univ(\iota_2)(1)
				+ \Univ(\iota_1)(1) \cdot \Univ(\iota_2)(\class{y})
				\\
				&=
				\class{\iota_1(x)} \cdot 1
				+ 1 \cdot \class{\iota_2(y)}
				\\
				&=
				\class{(x,0)} + \class{(0,y)}
				\\
				&=
				\class{(x,y)}
			\end{align*}
			for every~$(x,y) \in \glie \oplus \hlie$, and thus~$\Phi \circ \Psi = \id_{\Univ(\glie \oplus \hlie)}$.
			We have on the other hand
			\begin{align*}
				\Psi( \Phi( \class{x} \tensor \class{y} ) )
				&=
				\Psi( \Univ(\iota_1)(\class{x}) \cdot \Univ(\iota_2)(\class{y}) )
				\\
				&=
				\Psi\Bigl( \class{(x,0)} \cdot \class{(0,y)} \Bigr)
				\\
				&=
				\Psi\Bigl( \class{(x,0)} \Bigr)
				\Psi\Bigl( \class{(0,y)} \Bigr)
				\\
				&=
				( \class{x} \tensor 1 + 1 \tensor 0 )
				\cdot ( 0 \tensor 1 + 1 \tensor \class{y} )
				\\
				&=
				(\class{x} \tensor 1)
				\cdot (1 \tensor \class{y})
				\\
				&=
				\class{x} \tensor \class{y}
			\end{align*}
			for all~$x \in \glie$ and~$y \in \hlie$, and thus~$\Psi \circ \Phi = \id_{\Univ(\glie) \tensor \Univ(\hlie)}$.
	\end{itemize}
\end{example}


\begin{construction}
	\label{construction of comultiplication}
	Let~$\glie$ be a Lie~algebra.
	We have a homomorphism of Lie~algebras
	\[
		\delta
		\colon
		\glie
		\to
		\glie \oplus \glie \,,
		\quad
		x
		\mapsto
		(x,x) \,.
	\]
	This homomorphism of Lie~algebras extends a homomorphism of algebras
	\[
		\Delta'
		\defined
		\Univ(\delta)
		\colon
		\Univ(\glie)
		\to
		\Univ(\glie \oplus \glie) \,.
	\]
	As seen in \cref{explicit isomorphism for uea of direct sum} we have an isomorphism of algebras from~$\Univ(\glie \oplus \glie)$ to~$\Univ(\glie) \tensor \Univ(\glie)$ given by~$\class{(x,y)} \mapsto \class{x} \tensor 1 + 1 \tensor \class{y}$ for all~$x, y \in \glie$.
	Under this isomorphism, we can regard the homomorphism~$\Delta'$ as a homomorphism of algebras
	\[
		\Delta
		\colon
		\Univ(\glie)
		\to
		\Univ(\glie) \tensor \Univ(\glie) \,.
	\]
	This homomorphism is uniquely determined by
	\[
		\Delta( \class{x} )
		=
		\class{x} \tensor 1 + 1 \tensor \class{x}
		\qquad
		\text{for every~$x \in \glie$.}
	\]
\end{construction}


\begin{definition}
	Let~$\glie$ be a Lie~algebra
	The homomorphism of algebras~$\Delta$\glsadd{comultiplication} from \cref{construction of comultiplication}\index{comultiplication} is the \defemph{comultiplication} of~$\glie$.
\end{definition}


\begin{remark}
	Let~$\glie$ be a Lie~algebra and let~$M$ and~$N$ be two representations of~$\glie$.
	We have corresponding~\module{$\Univ(\glie)$} structures on~$M$ and~$N$ that are uniquely determined by
	\[
		\class{x} \cdot m
		=
		x \act m
		\qquad\text{and}\qquad
		\class{x} \cdot n
		=
		x \act n
	\]
	for all~$x \in \glie$,~$m \in M$,~$n \in N$.
	\begin{enumerate}
		\item
			These module structure correspond to homomorphisms of algebras
			\[
				\Phi_M
				\colon
				\Univ(\glie)
				\to
				\End_{\kf}(M) \,,
				\quad
				\Phi_N
				\colon
				\Univ(\glie)
				\to
				\End_{\kf}(N) \,.
			\]
			These homomorphism fit together into a homomorphism of algebras
			\[
				\Phi_M \tensor \Phi_N
				\colon
				\Univ(\glie) \tensor \Univ(\glie)
				\to
				\End_{\kf}(M) \tensor \End_{\kf}(N) \,.
			\]
			We have also a homomorphism of algebras
			\[
				\Psi
				\colon
				\End_{\kf}(M) \tensor \End_{\kf}(N)
				\to
				\End_{\kf}(M \tensor_{\kf} N) \,,
				\quad
				f \tensor g
				\mapsto
				f \tensor g \,.
			\]
			Let~$\Delta$ be the comultiplication of~$\Univ(\glie)$.
			The composite
			\[
				\Phi_{M \tensor N}
				\defined
				\Psi \circ (\Phi_M \tensor \Phi_N) \circ \Delta
				\colon
				\Univ(\glie)
				\to
				\End_{\kf}(M \tensor N) \,.
			\]
			is again a homomorphism of algebras.
			This homomorphism corresponds to a~\module{$\Univ(\glie)$} structure on~$M \tensor_{\kf} N$.
			For every element~$x$ of~$\glie$ we have
			\begin{align*}
				\class{x} \cdot (m \tensor n)
				&=
				\Phi_{M \tensor N}( \class{x} )( m \tensor n )
				\\
				&=
				\Psi( (\Phi_M \tensor \Phi_N)( \Delta(\class{x}) ) )( m \tensor n)
				\\
				&=
				\Psi( (\Phi_M \tensor \Phi_N)( \class{x} \tensor 1 + 1 \tensor \class{x} ) )( m \tensor n)
				\\
				&=
				\Psi( \Phi_M(\class{x}) \tensor \Phi_N(1) + \Phi_M(1) \tensor \Phi_N(\class{x}) )( m \tensor n)
				\\
				&=
				\Psi( \Phi_M(\class{x}) \tensor \id_N + \id_M \tensor \Phi_N(\class{x}) )( m \tensor n)
				\\
				&=
				\Phi_M(\class{x})(m) \tensor \id_N(n) + \id_M(m) \tensor \Phi_N(\class{x})(n)
				\\
				&=
				(\class{x} \cdot m) \tensor n + m \tensor (\class{x} \cdot n)
				\\
				&=
				(x \act m) \tensor n + m \tensor (x \act n)
			\end{align*}
			for all~$m \in M$,~$n \in N$.
			This~\module{$\Univ(\glie)$} structure on~$M \tensor_{\kf} N$ corresponds to an action of~$\glie$ on~$M \otimes_{\kf} N$.
			The above calculation shows that this is precisely the action from \cref{new representations from old ones}\index{tensor product of representations}.
		\item
			The~\module{$\Univ(\glie)$} structure on~$N$ induces a left~\module{$\Univ(\glie)$} structure on~$\Hom_{\kf}(M, N)$ given by
			\[
				(y \cdot f)(m)
				\defined
				y \cdot f(m)
			\]
			for all~$y \in \Univ(\glie)$,~$f \in \Hom_{\kf}(M, N)$,~$m \in M$.
			The~\module{$\Univ(\glie)$} structure on~$M$ induces a right~\module{$\Univ(\glie)$} structure on~$\Hom_{\kf}(M, N)$ given by
			\[
				(f \cdot y)(m)
				\defined
				f(y \cdot m)
			\]
			for all~$y \in \Univ(\glie)$,~$f \in \Hom_{\kf}(M, N)$,~$m \in M$.
			These two module structures commute with each other because
			\[
				((y_1 \cdot f) \cdot y_2)(m)
				=
				(y_1 \cdot f)(y_2 \cdot m)
				=
				y_1 \cdot f(y_2 \cdot m)
				=
				y_1 \cdot (f \cdot y_2)(m)
				=
				(y_1 \cdot (f \cdot y_2))(m)
			\]
			for all~$y_1, y_2 \in \Univ(\glie)$,~$f \in \Hom_{\kf}(M,N)$,~$m \in M$.
			We have thus a~\bimodule{$\Univ(\glie)$}{$\Univ(\glie)$} structure on~$\Hom_{\kf}(M,N)$ given by
			\[
				y_1 \cdot f \cdot y_2
				\defined
				y_1 \cdot (f \cdot y_2)
				=
				(y_1 \cdot f) \cdot y_2
			\]
			for all~$y_1, y_2 \in \Univ(\glie)$ and~$f \in \Hom_{\kf}(M, N)$.
			This~\bimodule{$\Univ(\glie)$}{$\Univ(\glie)$} structure corresponds to a left~\module{$\Univ(\glie) \tensor \Univ(\glie)^{\op}$} structure given by
			\[
				\Bigl( y_1 \tensor y_2^{\op} \Bigr) \cdot f
				\defined
				y_1 \cdot f \cdot y_2
			\]
			for all~$y_1, y_2 \in \Univ(\glie)$ and~$f \in \Hom_{\kf}(M, N)$.
			By using the isomorphism between~$\Univ(\glie^{\op})$ and~$\Univ(\glie)^{\op}$ from \cref{uea of opposite by first principles} we get a~\module{$\Univ(\glie) \tensor \Univ(\glie)$} structure on~$\Hom_{\kf}(M,N)$ given by
			\[
				( \class{x_1} \tensor \class{x_2} ) \cdot f
				\defined
				\class{x_1} \cdot f \cdot (-\class{x_2})
				=
				- \class{x_1} \cdot f \cdot \class{x_2}
			\]
			for all~$x_1, x_2 \in \glie$ and~$f \in \Hom_{\kf}(M,N)$.
			We can pull back this~\module{$\Univ(\glie) \tensor \Univ(\glie)$} structure along the comultiplication~$\Delta$ to get a~\module{$\Univ(\glie)$} structure on~$\Hom_{\kf}(M,N)$ given by
			\[
				\class{x} \cdot f
				\defined
				(\class{x} \tensor 1 + 1 \tensor \class{x}) \cdot f
				=
				\class{x} \cdot f - f \cdot \class{x}
			\]
			for all~$x_1, x_2 \in \glie$ and~$f \in \Hom_{\kf}(M,N)$.
			We have more explicitely
			\[
				(\class{x} \cdot f)(m)
				=
				\class{x} \cdot f(m) - f(\class{x} \cdot m)
			\]
			for all~$x \in \glie$,~$f \in \Hom_{\kf}(M,N)$ and~$m \in M$.
			This~\module{$\Univ(\glie)$} structure on~$\Hom_{\kf}(M,N)$ corresponds to an action of~$\glie$ on~$\Hom_{\kf}(M,N)$ given by
			\begin{equation}
				\label{induced action on homomorphisms made explicit}
				(x \act f)(m)
				=
				x \act f(m) - f(x \act m)
			\end{equation}
			for all~$x \in \glie$,~$f \in \Hom_{\kf}(M,N)$,~$m \in M$.
			We see from the explicit formula~\eqref{induced action on homomorphisms made explicit} that this is precisely the induced action of~$\glie$ on~$\Hom_{\kf}(M,N)$ from \cref{new representations from old ones}\index{Hom-representation@$\Hom$-representation}.
	\end{enumerate}
\end{remark}


\begin{remark}
	Let~$\glie$ be a Lie~algebra.
	The comultiplication~$\Delta$, counit~$\varepsilon$, and antipode~$S$ satisfy certain compatibility conditions.
	More precisely, the comultiplication~$\Delta$ satisfies the following \defemph{coassociativity}\index{coassociative} diagram.
	\[
		\begin{tikzcd}[sep = large]
			\Univ(\glie)
			\arrow{r}[above]{\Delta}
			\arrow{d}[left]{\Delta}
			&
			\Univ(\glie) \tensor \Univ(\glie)
			\arrow{d}[right]{\Delta \tensor \id}
			\\
			\Univ(\glie) \tensor \Univ(\glie)
			\arrow{r}[above]{\id \tensor \Delta}
			&
			\Univ(\glie) \tensor \Univ(\glie) \tensor \Univ(\glie)
		\end{tikzcd}
	\]
	The comultiplication~$\Delta$ and counit~$\varepsilon$ satisfy the folloing \defemph{counital}\index{counital} diagram.
	\[
		\begin{tikzcd}[row sep = large]
			\Univ(\glie) \tensor \Univ(\glie)
			\arrow{d}[left]{\id \tensor \varepsilon}
			&
			\Univ(\glie)
			\arrow{l}[above]{\Delta}
			\arrow{r}[above]{\Delta}
			\arrow[equal]{d}
			&
			\Univ(\glie) \tensor \Univ(\glie)
			\arrow{d}[right]{\varepsilon \tensor \id}
			\\
			\Univ(\glie) \tensor \kf
			\arrow{r}[above]{\cong}
			&
			\Univ(\glie)
			&
			\kf \tensor \Univ(\glie)
			\arrow{l}[above]{\cong}
		\end{tikzcd}
	\]
	The comultiplication~$\Delta$, counit~$\varepsilon$, and antipode~$S$ satisfy the following \defemph{antipode}\index{antipodal} diagram.
	\[
		\begin{tikzcd}[column sep = small, row sep = large]
			{}
			&
			\Univ(\glie) \tensor \Univ(\glie)
			\arrow{rr}{S \tensor \id}
			&
			{}
			&
			\Univ(\glie) \tensor \Univ(\glie)
			\arrow{dr}{\mathrm{mult}}
			&
			{}
			\\
			\Univ(\glie)
			\arrow{ur}[above left]{\Delta}
			\arrow{rr}[above]{\varepsilon}
			\arrow{dr}[below left]{\Delta}
			&
			{}
			&
			\kf
			\arrow{rr}[above]{\mathrm{incl}}
			&
			{}
			&
			\Univ(\glie)
			\\
			{}
			&
			\Univ(\glie) \tensor \Univ(\glie)
			\arrow{rr}[above]{\id \tensor S}
			&
			{}
			&
			\Univ(\glie) \tensor \Univ(\glie)
			\arrow{ur}[below right]{\mathrm{mult}}
			&
			{}
		\end{tikzcd}
	\]
	This means altogether that the algebra~$\Univ(\glie)$ together with the comultiplication~$\Delta$, counit~$\varepsilon$ and antipode~$S$ is a \defemph{Hopf algebra}.
\end{remark}


% TODO: A more detailed discussion of Hopf algebras.




\subsection{Quotient Lie~Algebra}


\begin{example}
	\label{universal enveloping algebra of quotient}
	Let~$\glie$ be a Lie~algebra, let~$I$ be an ideal of~$\glie$ and let~$\ideal{I}$ be the two-sided ideal of~$\Univ(\glie)$ generated by all elements of the form~$\class{x}$ with~$x$ in~$I$.
	Let~$\pi$ be the canonical quotient homomorphism from~$\glie$ to~$\glie/I$.
	Then the induced homomorphism of algebras~$\Univ(\pi)$ from~$\Univ(\glie)$ to~$\Univ(\glie/I)$ factors through~$\Univ(\glie) / \ideal{I}$, and induces an isomorphism
	\[
		\Univ(\glie) / \ideal{I}
		\to
		\Univ(\glie/I) \,,
		\index{universal enveloping algebra!of a quotient Lie algebra}
	\]
	which is is given by
	\[
		\class{ \class{x} }
		\mapsto
		\class{ \class{x} }
	\]
	for all~$x \in \glie$.

	Indeed, we have for every~{\algebra{$\kf$}}~$A$ bijections
	\begin{align*}
		{}&
		\{ \textstyle\text{algebra homomorphisms~$\Psi \colon \Univ(\glie/I) \to A$} \}
		\\
		\cong{}&
		\{ \textstyle\text{Lie~algebra homomorphisms~$\psi \colon \glie/I \to A$} \}
		\\
		\cong{}&
		\{ \textstyle\text{Lie~algebra homomorphisms~$\varphi \colon \glie \to A$ with~$\varphi(x) = 0$ for every~$x \in I$} \}
		\\
		\cong{}&
		\{ \textstyle\text{algebra homomorphisms~$\Phi \colon \Univ(\glie) \to A$ with~$\Phi(\class{x}) = 0$ for every~$x \in I$} \}
		\\
		={}&
		\{ \textstyle\text{algebra homomorphisms~$\Phi \colon \Univ(\glie) \to A$ with~$\Phi(y) = 0$ for every~$y \in \ideal{I}$} \}
		\\
		\cong{}&
		\{ \textstyle\text{algebra homomorphisms~$\Psi \colon \Univ(\glie) / \ideal{I} \to A$} \} \,.
	\end{align*}
	These bijections are natural in~$A$, whence~$\Univ(\glie/I) \cong \Univ(\glie) / \ideal{I}$ by Yoneda’s lemma.
	
	More explicitely, the composite
	\[
		\glie
		\to
		\Univ(\glie)
		\to
		\Univ(\glie) / \ideal{I} \,,
		\quad
		x \mapsto
		\class{ \class{x} }
	\]
	is a homomorphisms of Lie~algebras.
	This homomorphism annihilates the ideal~$I$ of~$\glie$ and thus induces a homomorphism of Lie~algebras
	\[
		\glie/I
		\to
		\Univ(\glie) / \ideal{I} \,,
		\quad
		\class{x}
		\mapsto
		\class{ \class{x} } \,,
	\]
	which in turn induces a homomorphism of algebras given by
	\begin{alignat*}{3}
		\Phi
		&\colon
		\Univ(\glie/I)
		\to
		\Univ(\glie) / \ideal{I}  \,,
		&
		\quad
		\class{ \class{x} }
		&\mapsto
		\class{ \class{x} }
		&
		\qquad
		&\text{for every~$x \in \glie$.}
	\intertext{
	On the other hand, the canonical quotient homomorphism~$\pi$ from~$\glie$ to~$\glie/I$ induces the homomorphism of algebras~$\Univ(\pi)$ from~$\Univ(\glie)$ to~$\Univ(\glie/I)$, given by
	}
		\Univ(\pi)
		&\colon
		\Univ(\glie)
		\to
		\Univ( \glie / I ) \,,
		&
		\quad
		\class{x}
		&\mapsto
		\class{ \class{x} }
		&
		\qquad
		&\text{for every~$x \in \glie$.}
	\intertext{
	This homomorphism annihilates all residue classes~$\class{x}$ with~$x \in I$, and hence induces a homomorphism of algebras given by
	}
		\Psi
		&\colon
		\Univ(\glie) / \ideal{I}
		\to
		\Univ(\glie/I)  \,,
		&
		\quad
		\class{ \class{x} }
		&\mapsto
		\class{ \class{x} }
		&
		\qquad
		&\text{for every~$x \in \glie$.}
	\end{alignat*}
	We have thus constructed two mutually inverse isomorphisms of algebras~$\Phi$ and~$\Psi$.
\end{example}

\subsubsection{Abelianzation of Lie Algebras}

\begin{example}
	Let~$\glie$ be a Lie~algebra.
	It follows from \cref{universal enveloping algebra of quotient} that
	\begin{align*}
		\Univ( \glie^{\ab} )
		&=
		\Univ( \glie/[\glie, \glie] )
		\\
		&\cong
		\Univ(\glie) / \ideal{[\glie, \glie]}
		\\
		&=
		\Univ(\glie)
		/
		\ideal[\Big]{ \class{[x,y]} \suchthat x, y \in \glie }
		\\
		&=
		\Univ(\glie)
		/
		\ideal{
			\class{x} \, \class{y} - \class{y} \, \class{x}
		\suchthat
			x, y \in \glie
		} \,.
	\end{align*}
	The ideal~$\ideal{ \class{x} \, \class{y} - \class{y} \, \class{x} \suchthat x, y \in \glie }$ is the commutator ideal of~$\Univ(\glie)$ because the elements~$\class{x}$ with~$x \in \glie$ form an algebra generating set of~$\Univ(\glie)$.
	The universal enveloping algebra of the abelianization of~$\glie$ is hence the abelianization (commutativization?) of the universal enveloping algebra of~$\glie$.\index{universal enveloping algebra!of the abelianization}
	
	This can also be seen from Yoneda’s lemma since we have for every commutative~{\algebra{$\kf$}}~$A$ bijections
	\begin{align*}
		\SwapAboveDisplaySkip
		{}&
		\{ \textstyle\text{algebra homomorphisms~$\Univ(\glie/[\glie,\glie]) \to A$} \}
		\\
		\cong{}&
		\{ \textstyle\text{Lie~algebra homomorphism~$\glie/[\glie, \glie] \to A$} \}
		\\
		\cong{}&
		\{ \textstyle\text{Lie~algebra homomorphisms~$\glie \to A$} \}
		\\
		\cong{}&
		\{ \textstyle\text{algebra homomorphisms~$\Univ(\glie) \to A$} \}
		\\
		\cong{}&
		\{ \textstyle\text{algebra homomorphisms~$\Univ(\glie)/I \to A$} \}
	\end{align*}
	that are natural in~$A$, where~$I$ denotes the commutator ideal of~$\Univ(\glie)$.
\end{example}





\subsection{Free Lie~Algebra}


\begin{definition}
	\leavevmode
	\begin{enumerate}
		\item
			Let~$X$ be a set.
			A \defemph{free~\liealgebra{$\kf$} on the set~$X$}\index{free Lie algebra!on a set} is a~\liealgebra{$\kf$}~$\freelieset(X)$\glsadd{free lie algebra set} together with a set-theoretic map~$i$ from~$X$ to~$\freelieset(X)$ such that the following universal property holds:
			For every~\liealgebra{$\kf$}~$\glie$ and every map~$f$ from~$X$ to~$\glie$ there exists a unique homomorphism of Lie~algebras~$\varphi$ from~$\freelieset(X)$ to~$\glie$ that makes the following triangular diagram commute.
			\[
				\begin{tikzcd}[sep = large]
					X
					\arrow{dr}[above right]{f}
					\arrow{d}[left]{i}
					&
					{}
					\\
					\freelieset(X)
					\arrow[dashed]{r}[below]{\varphi}
					&
					\glie
				\end{tikzcd}
			\]
		\item
			Let~$V$ be a~\vectorspace{$\kf$}.
			A \defemph{free~\liealgebra{$\kf$} on the vector space~$V$}\index{free Lie algebra!on a vector space} is a~\liealgebra{$\kf$}~$\freelievect(V)$\glsadd{free lie algebra vector space} together with a linear map~$i$ from~$V$ to~$\freelievect$ such that the following universal property holds:
			For every~\liealgebra{$\kf$}~$\glie$ and every linear map~$f$ from~$V$ to~$\glie$ there exists a unique homomorphism of Lie algebras~$\varphi$ from~$\freelievect(V)$ to~$\glie$ that makes the following diagram commute.
			\[
				\begin{tikzcd}[sep = large]
					V
					\arrow{dr}[above right]{f}
					\arrow{d}[left]{i}
					&
					{}
					\\
					\freelievect(V)
					\arrow[dashed]{r}[below]{\varphi}
					&
					\glie
				\end{tikzcd}
			\]
	\end{enumerate}
\end{definition}


\begin{remark}
	\leavevmode
	\begin{enumerate}
		\item
			Let~$X$ be a set.
			A free Lie~algebra on the set~$X$ consists of a Lie~algebra~$\freelieset(X)$ together with an element~$\class{x}$ of~$\freelieset(X)$ for every element~$x$ of~$X$, such that the following universal property holds:
			For every Lie~algebra~$\glie$ and every famile~$(y_x)_{x \in X}$ of elements of~$\glie$ there exists a unique homomorphism of Lie~algebras~$\varphi$ from~$\freelieset(X)$ to~$\glie$ such that~$\varphi( \class{x} ) = y_x$ for every~$x \in X$.
		\item
			The free Lie~algebra on a set~$X$ is unique up to unique isomorphism, in the following sense.

			If~$(L_1, i_1)$ and~$(L_2, i_2)$ are two free Lie~algebras on a set~$X$ then there exist unique homomorphisms of Lie~algebras~$\varphi$ from~$L_1$ to~$L_2$ and~$\psi$ from~$L_2$ to~$L_1$ that make the triangular diagrams
			\[
				\begin{tikzcd}[column sep = small]
					{}
					&
					X
					\arrow{dl}[above left]{\iota_1}
					\arrow{dr}[above right]{\iota_2}
					&
					{}
					\\
					F_1
					\arrow[dashed]{rr}[below]{\varphi}
					&
					{}
					&
					F_2
				\end{tikzcd}
				\qquad\text{and}\qquad
				\begin{tikzcd}[column sep = small]
					{}
					&
					X
					\arrow{dl}[above left]{\iota_2}
					\arrow{dr}[above right]{\iota_1}
					&
					{}
					\\
					F_2
					\arrow[dashed]{rr}[below]{\psi}
					&
					{}
					&
					F_1
				\end{tikzcd}
			\]
			commute.
			These homomorphisms~$\varphi$ and~$\psi$ are mutually inverse isomorphisms of Lie~algebras.

			Because of this uniqueness we will talk about \emph{the} free~\liealgebra{$\kf$} on~$X$.
		\item
			We find in the same way that the free~\liealgebra{$\kf$} on a~\vectorspace{$\kf$} is unique up to unique isomorphism.
		\item
			Let~$X$ and~$Y$ be two sets and let~$f$ be a map from~$X$ to~$Y$.
			There exists a unique homomorphism of Lie~algebras~$\freelieset(f)$\glsadd{free lie algebra set on morphisms}\index{induced homomorphism!on free Lie algebras} from~$\freelieset(X)$ to~$\freelieset(Y)$ that makes the square diagram
			\[
				\begin{tikzcd}[column sep = large]
					X
					\arrow{r}[above]{f}
					\arrow{d}
					&
					Y
					\arrow{d}
					\\
					\freelieset(X)
					\arrow[dashed]{r}[above]{\freelieset(f)}
					&
					\freelieset(Y)
				\end{tikzcd}
			\]
			commute.
			It holds for every set~$X$ that~$\freelieset(\id_X) = \id_{\freelieset(X)}$, and it holds for any two composable maps of sets~$f$ from~$X$ to~$Y$ and~$g$ from~$Y$ to~$Z$ that~$\freelieset(g \circ f) = \freelieset(g) \circ \freelieset(f)$.
			The assignment~$\freelieset$ is thus a functor from the category~$\cSet$ to the category~$\cLie{\kf}$.
			
			The universal property of the free Lie~algebra on a set states that the functor~$\freelieset$ is left adjoint\index{adjunction} to the forgetful functor from~$\cLie{\kf}$ to~$\cSet$, which assigns to each Lie~algebra its underlying set.
		\item
			We find similarly that for any two vector spaces~$V$ and~$W$ and every linear maps~$f$ from~$V$ to~$W$ there exist a unique homomorphism of Lie~algebras~$\freelievect(f)$\glsadd{free lie algebra vector space on morphisms} from~$\freelievect(V)$\index{induced homomorphism!on free Lie algebras} to~$\freelievect(W)$ that makes the square diagram
			\[
				\begin{tikzcd}[column sep = huge]
					V
					\arrow{r}[above]{f}
					\arrow{d}
					&
					W
					\arrow{d}
					\\
					\freelievect(V)
					\arrow[dashed]{r}[above]{\freelievect(f)}
					&
					\freelievect(W)
				\end{tikzcd}
			\]
			commute.
			The assignment~$\freelievect$ is then a functor from~$\cVect{\kf}$ to~$\cLie{\kf}$.
			The universal property of the free Lie~algebra on a vector space states that the functor~$\freelievect$ is left adjoint\index{adjunction} to the forgetful functor from~$\cLie{\kf}$ to~$\cVect{\kf}$, which assigns to each \liealgebra{$\kf$} its underlying~\vectorspace{$\kf$}.
		\item
			For every set~$X$ let~$F(X)$ denote the free~\vectorspace{$\kf$} on the set~$X$.
			The resulting Lie~algebra~$\freelievect(F(X))$ together with the composite
			\[
				X
				\to
				F(X)
				\to
				\freelievect(F(X))
			\]
			is a free~\liealgebra{$\kf$} on the set~$X$.

			Indeed, for every Lie~algebra~$\glie$ and every map~$f$ from~$X$ to~$\glie$ we can first extend~$f$ uniquely to a linear map~$g$ from~$F(X) \to \glie$ such that the following diagram commutes.
			\[
				\begin{tikzcd}[column sep = large]
					X
					\arrow{d}
					\arrow[bend left = 45]{ddr}[above right]{f}
					&
					{}
					\\[0.5em]
					F(X)
					\arrow{d}
					\arrow[dashed, bend left = 20]{dr}[above right]{g}
					&
					{}
					\\[0.5em]
					\freelievect(F(X))
					&
					\glie
				\end{tikzcd}
			\]
			We can then extend~$g$ uniquely to a homomorphism of Lie~algebras~$\psi$ from~$\freelievect(F(X))$ to~$\glie$.
			\[
				\begin{tikzcd}[column sep = large]
					X
					\arrow{d}
					\arrow[bend left = 45]{ddr}[above right]{f}
					&
					{}
					\\[0.5em]
					F(X)
					\arrow{d}
					\arrow[bend left = 20]{dr}[above right]{g}
					&
					{}
					\\[0.5em]
					\freelievect(F(X))
					\arrow[dashed]{r}[above]{\psi}
					&
					\glie
				\end{tikzcd}
			\]

			We have more abstractly for every~\liealgebra{$\kf$}~$\glie$ bijections
			\begin{align*}
				{}&
				\{
					\text{maps~$f \colon X \to \glie$}
				\}
				\\
				\cong{}&
				\{
					\text{linear maps~$g \colon F(X) \to \glie$}
				\}
				\\
				\cong{}&
				\{
					\text{homomorphism of Lie~algebras~$\psi \colon \freelievect(F(X)) \to \glie$}
				\} \,,
			\end{align*}
			which are natural in~$\glie$.
			It therefore follows from Yoneda’s lemma that~$\freelievect(F(X)) \cong \freelieset(X)$.

			We can also understand this isomorphism via adjunctions:
			We have the following commutative diagram of forgetful functors.
			\[
				\begin{tikzcd}
					\cLie{\kf}
					\arrow{d}
					\arrow[bend left = 60]{dd}
					\\
					\cVect{\kf}
					\arrow{d}
					\\
					\cSet
				\end{tikzcd}
			\]
			It follows that the following diagram of left adjoint functors commutes up to isomorphism:
			\[
				\begin{tikzcd}[row sep = large]
					\cLie{\kf}
					\\
					\cVect{\kf}
					\arrow{u}[right]{\freelievect}
					\\
					\cSet
					\arrow{u}[right]{F}
					\arrow[bend left = 60]{uu}[left]{\freelieset}
				\end{tikzcd}
			\]
	\end{enumerate}
\end{remark}


\begin{example}[Free Lie~algebras]
	\label{uea of free lie algebra}
	\leavevmode
	\begin{enumerate}
		\item
			Let~$V$ be a~\vectorspace{$\kf$}.
			For every~\algebra{$\kf$}~$A$ we have bijections
			\begin{align*}
				{}&
				\{
					\text{algebra homomorphisms~$\Phi \colon \Univ(\freelievect(V)) \to A$}
				\}
				\\
				\cong{}&
				\{
					\text{Lie~algebra homomorphisms~$\varphi \colon \freelievect(V) \to A$}
				\}
				\\
				\cong{}&
				\{
					\text{linear maps~$f \colon V \to A$}
				\}
				\\
				\cong{}&
				\{
					\text{algebra homomorphisms~$\Phi \colon \Tensor(V) \to A$}
				\} \,,
			\end{align*}
			which are natural in~$A$.
			It follows from Yoneda’s lemma that we have an isomorphism of algebras
			\[
				\Univ(\freelievect(V)) \cong \Tensor(V) \,.
				\index{universal enveloping algebra!of the free Lie algebra!on a vector space}
			\]

			We can also make this argumentation more explicit:
			For this, let~$i$ be the canonical linear map from~$V$ to its free Lie~algebra~$\freelievect(V)$, and let~$j$ be the canonical homomorphism of Lie~algebras from~$\freelievect(V)$ to its universal enveloping algebra~$\Univ(\freelievect(V))$.
			The composite~$j \circ i$ is a linear map from~$V$ to~$\Univ(\freelievect(V))$ and thus extends to a homomorphism of algebras
			\[
				\Phi
				\colon
				\Tensor(V)
				\to
				\Univ(\freelievect(V))
			\]
			that is given by
			\[
				\Phi(v)
				=
				j(i(v))
				=
				\class{i(v)}
			\]
			for every~$v \in V$.
			We find on the other hand that the inclusion map from~$V$ to~$\Tensor(V)$ is linear and therefore extends to a homomorphism of Lie~algebras
			\[
				\psi
				\colon
				\freelievect(V)
				\to
				\Tensor(V)
			\]
			that is given by
			\[
				\psi(i(v)) = v
			\]
			for every~$v \in V$.
			This homomorphism of Lie~algebras extends by the universal property of the universal enveloping algebra to a homomorphism of algebras
			\[
				\Psi
				\colon
				\Univ(\freelievect(V))
				\to
				\Tensor(V)
			\]
			that is given by
			\[
				\Psi(\class{i(v)})
				=
				\psi(i(v))
				=
				v
			\]
			for every~$v \in V$.
			The composite~$\Psi \circ \Phi$ is a homorphism of algebras from~$\Tensor(V)$ to~$\Tensor(V)$ with
			\[
				(\Psi \circ \Phi)(v)
				=
				\Psi(\Phi(v))
				=
				\Psi\Bigl( \class{i(v)} \Bigr)
				=
				v
			\]
			for every~$v \in V$.
			The tensor algebra~$\Tensor(V)$ is generated by the elements of~$V$, whence find that~$\Psi \circ \Phi = \id_{\Tensor(V)}$.
			The composite~$\Phi \circ \Psi$ on the other hand satisfies the condition
			\[
				(\Phi \circ \Psi)\Bigl(\class{i(v)} \Bigr)
				=
				\Phi\Bigl (\Psi\Bigl( \class{i(v)} \Bigr) \Bigr)
				=
				\Phi(v)
				=
				\class{i(v)}
			\]
			for every~$v \in V$.
			This shows that
			\[
				\Phi \circ \Psi \circ j \circ i
				=
				j \circ i \,.
			\]
			It follows from the universal property of the free Lie~algebra~$\freelievect(V)$ that
			\[
				\Phi \circ \Psi \circ j
				=
				j \,,
			\]
			and thus by the universal property of the universal enveloping algebra finally
			\[
				\Phi \circ \Psi
				=
				\id_{\Univ(\freelievect(V))} \,.
			\]
			We have shown altogether that~$\Phi$ and~$\Psi$ are mutually inverse isomorphisms of algebras.
		\item
			Let~$X$ be a set and let~$F(X)$ be the free vector space on~$X$.
			It follows from our previous discussions that
			\[
				\Univ(\freelieset(X))
				\cong
				\Univ(\freelievect(F(X))
				\cong
				\Tensor(F(X))
				\cong
				\kf\gen{ t_x \suchthat x \in X } \,.
				\index{universal enveloping algebra!of the free Lie algebra!on a set}
			\]
			The composite of canonical maps
			\[
				X
				\to
				\freelieset(X)
				\to
				\Univ(\freelieset(X))
				\cong
				\kf\gen{ t_x \suchthat x \in X }
			\]
			is given by~$x \mapsto t_x$ for every~$x \in X$.
	\end{enumerate}
\end{example}


\begin{remark}
	The Shirshov--Witt~theorem\index{Shirshow--Witt theorem}\index{theorem of!Shirshov--Witt} asserts that Lie~subalgebras of free Lie~algebras are again free.
	We refer to \cite{shirshov_subalgebras_of_free_lie_algebras} for a proof of this theorem.
\end{remark}





