\section{Homomorphisms of Representations}


\begin{definition}
	Let~$M$ und~$N$ be two representations of a~{\liealgebra{$\kf$}}~$\glie$.
	\begin{enumerate}
		\item
			A~{\linear{$\kf$}} map~$f$ from~$M$ to~$N$ is a \defemph{homomorphism of representations}\index{homomorphism!of representations} if it satisfies the condition
			\[
				f(x \act m) = x \act f(m)
				\qquad
				\text{for all~$x \in \glie$ and~$m \in M$.}
			\]
		\item
			The space of homomorphism from~$M$ to~$N$ is denoted by~$\Hom_{\glie}(M,N)$\glsadd{hom of representations}, and for~$N = M$ by~$\End_{\glie}(M)$\glsadd{end of representations}.
	\end{enumerate}
\end{definition}


\begin{remark}
	Let~$M$,~$N$, and~$P$ be representations of a Lie~algebra~$\glie$ over~$\kf$.
	\begin{enumerate}
		\item
			Let
			\[
				\rho_M
				\colon
				\glie
				\to
				\gllie(M) \,,
				\quad
				\rho_N
				\colon
				\glie
				\to
				\gllie(N)
			\]
			be the homomorphism of Lie algebras that correspond to the representations~$M$ and~$N$.
			A~{\linear{$\kf$}} map~$f$ from~$M$ to~$N$ is a homomorphism of representations if and only if it satisfies the equality
			\[
				f \circ \rho_M(x)
				=
				\rho_N(x) \circ f
				\qquad
				\text{for all~$x \in \glie$,}
			\]
			i.e.\ if and only if the square diagram
			\[
				\begin{tikzcd}[column sep = large]
					M
					\arrow{r}[above]{\rho_M(x)}
					\arrow{d}[left]{f}
					&
					M
					\arrow{d}[right]{f}
					\\
					N
					\arrow{r}[above]{\rho_N(x)}
					&
					N
				\end{tikzcd}
			\]
			commutes for every element~$x$ of~$\glie$.
		\item
			The identity map~$\id_M$ of~$M$ is a homomorphism of representations from~$M$ to~$M$.
		\item
			Let~$f$ be a homomorphism of representations from~$M$ to~$N$ and let~$g$ be a homomorphism of representations from~$N$ to~$P$.
			The composite~$g \circ f$ is a homomorphism of representations from~$M$ to~$P$.
		\item
			Let~$f$ and~$g$ be two homomorphisms of representations from~$M$ to~$N$.
			Their sum~$f + g$ is again a homomorphism of representations, and for every scalar~$\lambda$ in~$\kf$ the map~$\lambda f$ is again a homomorphism of representations.
			This means that~$\Hom_{\glie}(M,N)$ is a~{\linear{$\kf$}} subspace of~$\Hom_{\kf}(M,N)$.
		\item
			The above observations show that the representations of~$\glie$ together with the homomorphisms of representations between them form a~(\linear{$\kf$}) category\index{category!of representations}.
			We will denote this category by~$\cRep{\glie}$\glsadd{category lie algebra representation}.

			The class of objects of~$\cRep{\glie}$ is given by the class of representations of~$\glie$.
			The~\spaces{$\Hom$} of~$\cRep{\glie}$ are given by~$\Hom_{\cRep{\glie}}(M, N) = \Hom_{\glie}(M,N)$ for any two representations~$M$ and~$N$ of~$\glie$.
			The composition of morphisms in~$\cRep{\glie}$ is the usual composition of functions.
			The identity morphism of any represenation~$M$ of~$\glie$ is the usual identity function.
		\item
			The notions of an \defemph{isomorphism}\index{isomorphism!of representations}, \defemph{monomorphism}\index{monomorphism of representations}, \defemph{epimorphism}\index{epimorphism of representations}, \defemph{endomorphism}\index{endomorphisms of representations} and \defemph{automorphism}\index{automorphism!of representations} are defined in the usual category-theoretic way.
		\item
			If~$f$ is a bijective homomorphism of representations from~$M$ to~$N$ then its set-theoretic inverse~$f^{-1}$ is again a homomorphism of representations.
			Indeed, we have
			\[
				f^{-1}(x \act m)
				=
				f^{-1}(x \act f(f^{-1}(m)))
				=
				f^{-1}(f(x \act f^{-1}(m)))
				=
				x \act f^{-1}(m)
			\]
			for all~$x \in \glie$ and~$m \in M$.
			This shows that a homomorphism of representations is an isomorphism if and only if it bijective.
		\item
			One can show in the usual way by using kernels and cokernels -- which are again representations, as we will see below -- that the epimorphisms of representations are precisely the surjective homomorphisms, and the monomorphisms of representations are precisely the injective homomorphisms.
	\end{enumerate}
\end{remark}


\begin{lemma}
	Let~$M$ and~$N$ be two representations of a Lie~algebra~$\glie$ and let~$f$ be a homomorphism of representations from~$M$ to~$N$.
	The kernel\index{kernel} of~$f$ is a subrepresentation of~$M$ while the image\index{image} of~$f$ is a subrepresentation~$N$.
	\qed
\end{lemma}


\begin{proposition}
	\label{homomorphisms of representations as invariants}
	Let~$M$ and~$N$ be two representations of a Lie~algebra~$\glie$.
	A linear map~$f$ from~$M$ to~$N$ is a homomorphism of representations if and only if it is invariant under the induced action of~$\glie$ on~$\Hom_{\kf}(M,N)$.
	In other words,~$\Hom_{\glie}(M, N)$ equals~$\Hom_{\kf}(M,N)^{\glie}$.
\end{proposition}


\begin{proof}
	We have
	\begin{align*}
		\SwapAboveDisplaySkip
				{}& \text{$f$ is a homomorphism}  \\
		\iff{}& \text{$f(x \act m) = x \act f(m)$ for all~$x \in \glie$,~$m \in M$}  \\
		\iff{}& \text{$f(x \act m) - x \act f(m) = 0$ for all~$x \in \glie$,~$m \in M$}  \\
		\iff{}& \text{$(x \act f)(m) = 0$ for all~$x \in \glie$,~$m \in M$} \\
		\iff{}& \text{$x \act f = 0$ for all~$x \in \glie$}  \\
		\iff{}& \text{$f$ is invariant,}
	\end{align*}
	as claimed.
\end{proof}


\begin{proposition}
	Let~$M$ and~$N$ be two representations of a Lie~algebra~$\glie$ and let~$f$ be a homomorphism of representations from~$M$ to~$N$.
	Then~$f$ restricts to a homomorphism of representations from~$M^{\glie}$ to~$N^{\glie}$.
	\qed
\end{proposition}


\begin{remark}
	\label{invariants are right adjoint}
	Let~$\glie$ be a Lie~algebra.
	Let~$\cReptriv{\glie}$\glsadd{category lie algebra trivial representations} be the full sucategory of~$\cRep{\glie}$ whose objects are the trivial~\representations{$\glie$}\index{category!of trivial representations}.
	(This category is isomorphic to~$\cVect{\kf}$.)
	Let~$T$ be the inclusion functor from~$\cReptriv{\glie}$ to~$\cRep{\glie}$.

	We have a functor~$(\ph)^{\glie}$ from~$\cRep{\glie}$ to~$\cReptriv{\glie}$.
	This functor assigns to each~\representation{$\glie$}~$M$ its space of invariants~$M^{\glie}$.
	To each homomorphism of representations~$f$ from~$M$ to~$N$ it assigns the restriction of~$f$ to a homomorphism of representations from~$M^{\glie}$ to~$N^{\glie}$.

	Let~$M$ be trivial~\representation{$\glie$} and let~$N$ be any~\representation{$\glie$}.
	The inclusion map~$i$ from~$N^{\glie}$ to~$N$ is a homomorphism of representations, and it induces an isomorphism of vector spaces
	\[
		i_*
		\colon
		\Hom_{\glie}(M, N^{\glie})
		\to
		\Hom_{\glie}(M, N) \,,
		\quad
		f
		\mapsto
		i \circ f \,.
	\]
	This isomorphism is natural in both~$M$ and~$N$.
	The functor~$(\ph)^{\glie}$ is therefore right adjoint to the inclusion functor~$T$.
\end{remark}


\begin{proposition}
	\label{list of homomorphism of representations}
	Let~$\glie$ be a Lie algebra.
 \begin{enumerate}
		\item
			Let~$M$ be a representation of~$\glie$.
			The natural isomorphisms of vector spaces
			\begin{align*}
				\kf \tensor M
				\to
				M \,,
				\quad
				\lambda \tensor m
				\mapsto
				\lambda m
			\intertext{and}
				M \tensor \kf
				\to
				M \,,
				\quad
				m \tensor \lambda
				\mapsto
				\lambda m
			\end{align*}
			are natural isomorphism of representations.
		\item
			Let~$M$ be a representation of~$\glie$.
			The natural isomorphism of vector spaces
			\[
				\Hom_{\kf}( \kf, M )
				\to
				M \,,
				\quad
				f
				\mapsto
				f(1)
			\]
			is a natural isomorphism of representations.
		\item
			Let~$M$ and~$N$ be two representations of~$\glie$.
			For every homomorphism of representations~$f$ from~$M$ to~$N$ its dual linear map
			\[
				f^*
				\colon
				N^*
				\to
				M^* \,,
				\quad
				m^*
				\mapsto
				m^* \circ f
			\]
			is again a homomorphism of representations.
		\item
			Let more generally~$M$,~$N$,~$P$ be three representations of~$\glie$.
			For every homomorphism of representations~$f$ from~$M$ to~$N$ the linear map
			\begin{alignat*}{2}
				f^*
				\colon
				\Hom_{\kf}(N,P)
				&\to
				\Hom_{\kf}(M,P) \,,
				&
				\quad
				g
				&\mapsto
				g \circ f
			\intertext{is again a homomorphism of representations, and the linear map}
				f_*
				\colon
				\Hom_{\kf}(P,M)
				&\to
				\Hom_{\kf}(P,N) \,,
				&
				\quad
				g
				&\mapsto
				f \circ g
			\end{alignat*}
			is again a homomorphisms of representations.
		\item
			Let~$M$ and~$N$ be two representations of~$\glie$.
			The natural linear map
			\[
				M^* \tensor N
				\to
				\Hom_{\kf}(M,N) \,,
				\quad
				m^* \tensor n
				\mapsto
				(m \mapsto m^*(m) n)
			\]
			is a natural homomorphism of representations.
			If one (or both) of the representations~$M$ and~$N$ is finite-dimensional, then this natural homomorphism is already a natural isomorphism.
		\item
			Let~$M^1_1, \dotsc, M^1_{r_1}$ up to~$M^s_1, \dotsc, M^s_{r_s}$ be finitely many collections of finitely many representations of~$\glie$.
			The natural isomorphism of vector spaces
			\begin{align*}
				\bigl(
					M^1_1 \tensor \dotsb \tensor M^1_{r_1}
				\bigr)
				\tensor
				\dotsb
				\tensor
				\bigl(
					M^s_1 \tensor \dotsb \tensor M^s_{r_s}
				\bigr)
				&\to
				M^1_1 \tensor \dotsb \tensor M^s_{r_s}
			\shortintertext{given by}
				(m^1_1 \tensor \dotsb \tensor m^1_{r_1})
				\tensor
				\dotsb
				\tensor
				(m^s_1 \tensor \dotsb \tensor m^s_{r_s})
				&\mapsto
				m^1_1 \tensor \dotsb \tensor m^s_{r_s}
			\end{align*}
			for all~$m^i_j \in M^i_j$ is a natural isomorphism of representations.
		\item
			Let~$M_\lambda$ with~$\lambda$ in~$\Lambda$ be a collection of representations of~$\glie$ and let~$N$ be another representation of~$\glie$.
			The natural isomorphism of vector spaces
			\begin{alignat*}{2}
				\Biggl(
					\bigoplus_{\lambda \in \Lambda}
					M_\lambda
				\Biggr)
				\tensor
				N
				&\to
				\bigoplus_{\lambda \in \Lambda}
				{}
				( M_\lambda \tensor N ) \,,
				&\qquad
				(m_\lambda)_\lambda \tensor n
				&\mapsto
				(m_\lambda \tensor n)_\lambda
			\intertext{
			is a natural isomorphism of representations.
			Let similarly~$M$ be a representation of~$\glie$ and let~$N_\lambda$ with~$\lambda$ in~$\Lambda$ be a collection of representations of~$\glie$.
			The natural isomorphism of vector spaces
			}
				M
				\tensor
				\Biggl(
					\bigoplus_{\lambda \in \Lambda}
					N_\lambda
				\Biggr)
				&\to
				\bigoplus_{\lambda \in \Lambda}
				{}
				( M \tensor N_\lambda ) \,,
				&\qquad
				m \tensor (n_\lambda)_\lambda
				&\mapsto
				(m \tensor n_\lambda)_\lambda
			\end{alignat*}
			is a natural isomorphism of representations.
		\item
			Let~$M_1, \dotsc, M_r$ be representations of~$\glie$ and let~$\sigma$ be a permutation in~$\symm_r$.
			The natural isomorphism of vector spaces
			\begin{align*}
				M_1 \tensor \dotsb \tensor M_r
				&\to
				M_{\sigma(1)} \tensor \dotsb \tensor M_{\sigma(r)} \,,
				\\
				m_1 \tensor \dotsb \tensor m_r
				&\mapsto
				m_{\sigma(1)} \tensor \dotsb \tensor m_{\sigma(r)}
			\end{align*}
			is a natural isomorphism of representations.
		\item
			Let~$M_1, \dotsc, M_r$ and~$N_1, \dotsc, N_r$ be representations of~$\glie$.
			For all~$i = 1, \dotsc, n$ let
			\[
				f_i \colon M_i \to N_i
			\]
			be a homomorphisms of representations.
			The linear map
			\begin{align*}
				f_1 \tensor \dotsb \tensor f_r
				\colon
				M_1 \tensor \dotsb \tensor M_r
				&\to
				N_1 \tensor \dotsb \tensor N_r \,,
				\\
				m_1 \tensor \dotsb \tensor m_r
				&\mapsto
				f(m_1) \tensor \dotsb \tensor f(m_r)
			\end{align*}
			is again a homomorphism of representations.
		\qed
	\end{enumerate}
\end{proposition}


\begin{corollary}
	\label{invariants via internal hom}
	Let~$\glie$ be a Lie~algebra and let~$M$ be a representation of~$\glie$.
	The map
	\[
		\Hom_{\glie}(\kf, M)
		\to
		M^{\glie} \,,
		\quad
		f
		\mapsto
		f(1)
	\]
	is a well-defined isomorphism of vector spaces.
\end{corollary}


\begin{proof}
	The isomorphism of vector spaces
	\[
		\Hom_{\kf}(\kf, M)
		\to
		M \,,
		\quad
		f
		\mapsto
		f(1)
	\]
	is already an isomorphism of representations and thus restricts to an isomorphism
	\[
		\Hom_{\glie}(\kf, M)
		=
		\Hom_{\kf}(\kf, M)^{\glie}
		\to
		M^{\glie} \,,
	\]
	as desired.
\end{proof}


\begin{proposition}
	\label{checking enriched tensor hom adjunction}
	Let~$\glie$ be a Lie~algebra and let~$M$,~$N$ and~$P$ be representations of~$\glie$.
	\begin{enumerate}
		\item
			The natural isomorphism of vector spaces
			\[
				f
				\colon
				\Hom_{\kf}( M \tensor_{\kf} N, P )
				\to
				\Hom_{\kf}( M, \Hom_{\kf}(N, P) )
			\]
			given by
			\[
				f(\beta)(m)(n)
				=
				\beta(m \tensor n)
			\]
			for all~$\beta \in \Hom_{\kf}( M \tensor_{\kf} N, P )$,~$m \in M$,~$n \in N$ is a natural isomorphism of representations.
		\item
			The above natural isomorphism of representations restricts to a natural isomorphism of vector spaces
			\[
				\Hom_{\glie}( M \tensor_{\kf} N, P )
				\cong
				\Hom_{\glie}( M, \Hom_{\kf}(N, P) ) \,.
			\]
	\end{enumerate}
\end{proposition}


\begin{proof}
	\leavevmode
	\begin{enumerate}
		\item
			We have
			\begin{align*}
				\SwapAboveDisplaySkip
				f( (x \act \beta) )(m)(n)
				&=
				(x \act \beta)(m \tensor n)
				\\
				&=
				x \act \beta(m \tensor n)
				- \beta( x \act (m \tensor n) )
				\\
				&=
				x \act \beta(m \tensor n)
				- \beta( (x \act m) \tensor n )
				- \beta( m \tensor (x \act n) )
			\end{align*}
			as well as
			\begin{align*}
				( x \act f(\beta) )(m)(n)
				&=
				( x \act f(\beta)(m) - f(\beta)(x \act m) )(n)
				\\
				&=
				( x \act f(\beta)(m) )(n)
				- f(\beta)(x \act m)(n)
				\\
				&=
				x \act f(\beta)(m)(n)
				- f(\beta)(m)(x \act n)
				- f(\beta)(x \act m)(n)
				\\
				&=
				x \act \beta(m \tensor n)
				- \beta( m \tensor (x \act n) )
				- \beta( (x \act m) \tensor n )
			\end{align*}
			for all~$x \in \glie$,~$\beta \in \Hom_{\kf}(M \tensor N, P)$,~$m \in M$,~$n \in N$.
		\item
			By applying the functor~$(\ph)^{\glie}$ to the isomorphism
			\[
				\Hom_{\kf}( M \tensor_{\kf} N, P )
				\cong
				\Hom_{\kf}( M, \Hom_{\kf}(N, P) )
			\]
			we get a restricted isomorphism
			\[
				\Hom_{\glie}( M \tensor_{\kf} N, P )
				\cong
				\Hom_{\glie}( M, \Hom_{\kf}(N, P) ) \,
			\]
			as desired.
		\qedhere
	\end{enumerate}
\end{proof}


\begin{remark}
	\label{enriched tensor hom adjunction}
	It follows from \cref{checking enriched tensor hom adjunction} that for any representation~$M$ of~$\glie$ the functor
	\[
		(\ph) \tensor_{\kf} M
		\colon
		\cRep{\glie}
		\to
		\cRep{\glie}
	\]
	is left adjoint\index{adjunction} to the functor
	\[
		\Hom_{\kf}(M, \ph)
		\colon
		\cRep{\glie}
		\to
		\cRep{\glie} \,.
	\]
\end{remark}


\begin{proposition}[Homomorphism theorem]
	\index{homomorphism theorem!for representations}
	Let~$M$ be a representation of a Lie~algebra~$\glie$ and let~$N$ be a subrepresentation of~$M$.
	\begin{enumerate}
		\item
			The canonical projection map
			\[
				p
				\colon
				M
				\to
				M/N \,,
				\quad
				m
				\mapsto
				\class{m}
			\]
			is a homomorphism of representations.
	\end{enumerate}
	Let~$P$ be another representation of~$\glie$.
	\begin{enumerate}[resume*]
		\item
			Let~$g$ be a homomorphism of representations from~$M/N$ to~$P$.
			The composite~$g \circ p$ is a homomorphism of representations from~$M$ to~$P$ such that~$N$ is contained in the kernel of~$g \circ p$.
		\item
			Let~$f$ be a homomorphism of representations from~$M$ to~$P$.
			The homomorphism~$f$ factors through a homomorphism of representations~$g$ from~$M/N$ to~$P$ that makes the diagram
			\[
				\begin{tikzcd}[row sep = large]
					M
					\arrow{r}[above]{f}
					\arrow{d}[left]{p}
					&
					N
					\\
					M/N
					\arrow[dashed]{ur}[below right]{g}
					&
					{}
				\end{tikzcd}
			\]
			commute if and only if~$N$ is contained in the kernel of~$f$.
			The homomorphism~$g$ is unique and it is given by
			\[
				g(\class{m}) = f(m)
				\qquad
				\text{for every~$m \in M$.}
			\]
			Its image and kernel are given by~$\im(g) = \im(f)$ and~$\ker(g) = \ker(f)/N$.
		\qed
	\end{enumerate}
\end{proposition}


\begin{corollary}[Isomorphism theorems]
	\index{isomorphism theorems!for representations}
	Let~$M$ be a representation of a Lie~algebra~$\glie$.
	\begin{enumerate}
		\item
			Let~$N$ be another representation of~$\glie$ and let~$f$ be a homomorphism of representations from~$M$ to~$N$.
			The homomorphism~$f$ induces a unique well-defined isomorphism of representations
			\[
				M/{\ker(f)}
				\to
				\im(f) \,,
				\quad
				\class{m}
				\mapsto
				f(m)  \,.
			\]
		\item
			Let~$N$ and~$P$ be two subrepresentations of~$M$ such that~$N$ is contained in~$P$.
			The quotient representation~$N/P$ is a subrepresentation of~$M/P$ and the natural isomorphism of vector spaces
			\[
				(M/P) / (N/P)
				\to
				M/N \,,
				\quad
				\class{\class{m}}
				\mapsto
				\class{m}
			\]
			is a natural isomorphism of representations.
		\item
			Let~$N$ and~$P$ be subrepresentations of~$M$.
			Then~$P$ is a subrepresentation of the sum~$N+P$, the intersection~$N \cap P$ is a subrepresentation of~$P$, and the natural isomorphism of vector spaces
			\[
				N/(N \cap P)
				\to
				(N + P)/P  \,,
				\quad
				\class{n}
				\mapsto
				\class{n}
			\]
			is a natural isomorphism of representations.
	\end{enumerate}
\end{corollary}


\begin{lemma}
	Let~$M$ and~$N$ be two representations of a Lie~algebra~$\glie$ and let~$f$ be a homomorphism of representations from~$M$ to~$N$.
	\begin{enumerate}
		\item
			Let~$P$ be a subrepresentation of~$M$.
			The image~$f(P)$ is a subrepresentation of~$N$.
		\item
			Let~$P$ be a subrepresentation of~$N$.
			The preimage~$f^{-1}(P)$ is a subrepresentation of~$M$.
		\qed
	\end{enumerate}
\end{lemma}


\begin{proposition}[Correspondence theorem]
	\index{correspondence theorem!for representations}
	\label{correspondence theorem for representations}
	Let~$M$ be a representation of a Lie~algebra~$\glie$ and let~$N$ be a subrepresentation of~$M$.
	Let~$p$ the canonical projection map from~$M$ to~$M/N$.
	\begin{enumerate}
		\item
			Let~$P$ be a subrepresentation of~$M$ that contains~$N$.
			The quotient representation~$P/N$ is a subrepresentation of~$M/N$, and this construction results in a {\onetoonetext} correspondence
			\begin{align*}
				\left\{
					\begin{tabular}{@{}c@{}}
						subrepresentations~$P$ of~$M$ \\
						that contain~$N$
					\end{tabular}
				\right\}
				&\longleftrightarrow
				\{ \text{subrepresentations of~$M/N$} \}  \,,
				\\
				P
				&\mapsto
				P/N \,,
				\\
				p^{-1}(Q)
				&\mapsfrom
				Q  \,.
			\end{align*}
		\item
			Let~$P$ be a subrepresentation of~$M$ containing~$N$.
			Then it holds for the corresponding subrepre~$P/N$ of~$M/N$ that~$(M/N) / (P/N) \cong M/P$ by the third isomorphism theorem.
		\qed
	\end{enumerate}
\end{proposition}





