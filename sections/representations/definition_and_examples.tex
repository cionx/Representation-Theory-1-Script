\section{Definition and Examples}


\begin{definition}
	Let~$\glie$ be a~{\liealgebra{$\kf$}}.
	\begin{enumerate}
		\item
			A \defemph{representation}\index{representation} of~$\glie$ is a~{\vectorspace{$\kf$}}~$M$ together with a homomorphism of Lie~algebras~$\rho$ from~$\glie$ to~$\gllie(M)$.
		\item
			The \defemph{dimension}\index{dimension} of a representation~$(M, \rho)$ is the dimension of its underlying vector space~$M$.
		\item
			A representation~$(M, \rho)$ is~\defemph{faithful}\index{faithful} if the homomorphism~$\rho$ is injective.
		\item
			A \defemph{(left) action}\index{action!left}\index{left action} of~$\glie$ on a vector space~$M$ is a bilinear map
			\[
				\glie \times M \to M \,,
				\quad
				(x,m) \mapsto x \act m
				\glsadd{action on an element}
			\]
			such that
			\[
				[x,y] \act m
				=
				x \act (y \act m) - y \act (x \act m)
				\qquad
				\text{for all~$x, y \in \glie$ and~$m \in M$.}
			\]
	\end{enumerate}
\end{definition}


\begin{proposition}
	\label{correspodence between representations and actions}
	Let~$\glie$ be a~\liealgebra{$\kf$} and let~$M$ be a~\vectorspace{$\kf$}.
	\begin{enumerate}
		\item
			Given a homomorphism of Lie~algebras~$\rho$ from~$\glie$ to~$\gllie(M)$, the formula
			\[
				x \act m \defined \rho(x)(m)
				\qquad
				\text{for all~$x \in \glie$ and~$m \in M$}
			\]
			defines an action of~$\glie$ on~$M$.
		\item
			Suppose on the other hand that~$\glie$ acts on~$M$.
			Then for every element~$x$ of~$\glie$ the map
			\[
				\rho(x) \colon M \to M \,,
				\quad
				m \mapsto x \act m
			\]
			is~\linear{$\kf$}, and the resulting map
			\[
				\rho \colon \glie \to \gllie(M)
			\]
			is a homomorphism of Lie algebras.
		\item
			The above two constructions are mutually inverse and result in a {\onetoonetext} correspondence given by
			\[
				\left\{
					\begin{tabular}{@{}c@{}}
						homomorphism of
						\\
						Lie~algebras~$\glie \to \gllie(M)$
					\end{tabular}
				\right\}
				\onetoone
				\{
					\text{actions of~$\glie$ on~$M$}
				\} \,.
			\]
	\end{enumerate}
\end{proposition}


\begin{proof}
	We consider a map
	\[
		\alpha
		\colon
		\glie \times M
		\to
		M \,,
		\quad
		(x,m)
		\mapsto
		x \act m
	\]
	and the corresponding map
	\[
		\rho
		\colon
		\glie
		\to
		\Maps(M,M)
	\]
	given by
	\begin{equation}
		\label{action and homomorphism formula}
		\rho(x)(m) = x \act m
		\qquad
		\text{for all~$x \in \glie$ and~$m \in M$.}
	\end{equation}
	The linearity of~$\alpha$ in the second argument is equivalent to~$\rho$ taking values in~$\gllie(M)$.
	The additional linearity of~$\alpha$ in the first argument is equivalent to~$\rho$ being linear.
	That~$\rho$ is a homomorphism of Lie~algebras (when corestricted to a map into~$\gllie(M)$) is equivalent to the condition
	\[
		[x,y] \act m
		=
		x \act (y \act m) - y \act (x \act m)
		\qquad
		\text{for all~$x, y \in \glie$ and~$m \in M$.}
	\]
	This shows overall that the map~$\alpha$ is an action of~$\glie$ on~$M$ if and only if the map~$\rho$ gives a homomorphism of Lie~algebras from~$\glie$ to~$\gllie(M)$ (by corestriction).
	That both constructions are mutually inverse follows from the formula~\eqref{action and homomorphism formula}.
\end{proof}


\begin{fluff}
	Let~$\glie$ be a~\liealgebra{$\kf$}.
	According to \cref{correspodence between representations and actions} we can equivalently characterize a representation of~$\glie$ as a~\vectorspace{$\kf$}~$M$ together with an action of~$\glie$ on~$M$.
	In the following, we will most often work with this alternative characterization of representations via actions.
	But we will use the original definition when it is useful to do so.
\end{fluff}


%\begin{remark}
%  If~$(x_i)_{i \in I}$ is a basis of a Lie~algebra~$\glie$ then for a~{\linear{$\kf$}}~$\rho \colon \glie \to \gllie(M)$ to be a homomorphism of Lie~algebras it sufficies that~$\rho([x_i,x_j])= [\rho(x_i), \rho(x_j)]$ for all~$i, j \in I$.
%  It therefore sufficies to check condition~\eqref{representation via action} for basis elements, i.e.\ it sufficies to check that
%  \[
%    x_i.(x_j.v) - x_j.(x_i.v)
%    =
%    [x_i, x_j].v
%  \]
%  for all~$i, j \in I$ and~$m \in M$.
%\end{remark}


\begin{fluff}
	Ado’s theorem can be reformulated in the language of representations.
\end{fluff}


\begin{theorem}[Ado, second version]
	\index{Ado’s theorem}
	Every finite-dimensional Lie~algebra admits a finite-\hspace{0pt}dimensional, faithful representation.
\end{theorem}


\begin{examples}
	\label{examples for representations}
	\leavevmode
	\begin{enumerate}
		\item
			Let~$\glie$ be a Lie~subalgebra of~$\gllie(M)$ for some vector space~$M$.
			Then the inclusion map from~$\glie$ to~$\gllie(M)$ is a homomorphism of Lie~algebras, which makes~$M$ into a faithful representation of~$\glie$.
			The corresponding action of~$\glie$ on~$M$ is given by
			\[
				x \act m
				=
				x(m)
				\qquad
				\text{for all~$x \in \glie$,~$m \in M$.}
			\]
			This representation is the \defemph{natural representation}\index{natural representation} of~$\glie$.
		\item
			Let~$\glie$ be a Lie~subalgebra of~$\gllie(n, \kf)$ for natural number~$n$.
			Then the inclusion map
			\[
				\glie
				\to
				\gllie(n, \kf)
				\cong
				\gllie(\kf^n)
			\]
			is a homomorphism of Lie~algebras.
			This homomorphism makes~$\kf^n$ into a faithful representation of~$\gllie$.
			The corresponding action of~$\glie$ on~$\kf^n$ is given by
			\[
				x \act m
				=
				x \cdot m
				\qquad
				\text{for all~$x \in \glie$ and~$m \in \kf^n$.}
			\]
			This representation is the the \defemph{natural representation}\index{natural representation} of~$\glie$.
		\item
			Let~$M$ be the polynomial algebra~$\kf[x_1, \dotsc, x_n]$ and let~$p'_1, \dotsc, p'_n, q'_1, \dotsc, q'_n, c$ be the endomorphisms of~$M$ given as follows:
			$p'_i \defined \dd{x_i}$ is the~{\howmanyth{$i$}} partial derivate map,~$q'_i$ is the multiplication with the variable~$x_i$, and~$c' \defined \id_M$ is the identity function.
			These endomorphisms are linearly independent and satisfy the relations
			\begin{itemize*}
				\item
					$[p'_i, p'_j] = 0$ and $[q'_i, q'_j] = 0$ for all~$i, j = 1, \dotsc, n$,
				\item
					$[p'_i, c'] = 0$ and~$[q'_i, 0] = 0$ for every~$i = 1, \dotsc, n$,
				\item
					$[p'_i, q'_j] = \delta_{ij} c$ for all~$i, j = 1, \dotsc, n$.
			\end{itemize*}
			These relations are those of the Heisenberg Lie~algebra~$\heisenberglie$\index{Heisenberg Lie algebra} from \cref{examples for lie algebras}.
			The unique linear map
			\[
				\rho
				\colon
				\heisenberglie
				\to
				\gllie(M)
			\]
			that is given on the basis~$p_1, \dotsc, p_n, q_1, \dotsc, q_n, c$ of~$\heisenberglie$ by
			\[
				\rho(p_i) \defined p'_i \,,
				\quad
				\rho(p_i) \defined q'_i \,,
				\quad
				\rho(c) \defined c'
			\]
			is therefore a homomorphism of Lie~algebras.
			We have thus constructed a representation of~$\heisenberglie$ on~$M = \kf[x_1, \dotsc, x_n]$.
			Moreover, the elements~$p'_1, \dotsc, p'_n, q'_1, \dotsc, q'_n, c'$ of~$\gllie(M)$ are linearly independent, which means that~$\rho$ is injective.
			This representation of~$\heisenberglie$ is therefore faithful.

			This also shows that~$\heisenberglie$ is indeed a Lie~algebra, as it can be realized as a Lie~subalgebra of~$\gllie(M)$.
		\item
			Let~$\glie \defined \sllie(2, \kf)$.
			Then the polynomial ring~$\kf[x,y]$ becomes a representation of~$\glie$ via the homomorphism of Lie~algebras~$\rho$ from~$\glie$ to~$\gllie(\kf[x,y])$ given by
			\[
				\rho(e) = y \dd{x} \,,
				\qquad
				\rho(h) = y \dd{y} - x \dd{x} \,,
				\qquad
				\rho(f) = x \dd{y}  \,.
			\]
			The corresponding action of~$\sllie_2(\kf)$ on~$\kf[x,y]$ is given by
			\begin{align*}
				\SwapAboveDisplaySkip
				e \act (x^n y^m)
				&=
				n x^{n-1} y^{m+1} \,,
				\\
				h \act (x^n y^m)
				&=
				(m-n) x^n y^m \,,
				\\
				f \act (x^n y^m)
				&=
				m x^{n+1} y^{m-1}
			\end{align*}
			for all~$n, m \geq 0$.
			This is indeed an action of~$\sllie(2, \kf)$ on~$\kf[x]$ because
			\begin{gather*}
				\begin{aligned}
				e \act f \act (x^n y^m) - f \act e \act (x^n y^m)
				&=
				(n+1)m x^n y^m - n(m+1) x^n y^m
				\\
				&=
				(m-n) x^n y^m
				\\
				&= h \act m
				= [e,f] \act (x^n y^m) \,,
				\end{aligned}
			\shortintertext{as well as}
				\begin{aligned}
				h \act e \act (x^n y^m) - e \act h \act (x^n y^m)
				&=
				n(m-n+2) x^{n-1} y^{m+1} - n(m-n) x^{n-1} y^{m+1}
				\\
				&=
				2 x^{n-1} y^{m+1}
				\\
				&=
				2e \act (x^n y^m)
				\\
				&=
				[h,e] \act (x^n y^m) \,,
				\end{aligned}
			\shortintertext{and}
				\begin{aligned}
					h \act f \act (x^n y^m) - f \act h \act (x^n y^m)
					&=
					m(m-n-2) x^{n+1} y^{m-1} - m(m-n) x^{n+1} y^{m-1}
					\\
					&=
					-2 x^{n+1} y^{m-1}
					\\
					&=
					-2 f \act (x^n y^{m-1})
					\\
					&=
					[h,f] \act (x^n y^{m-1})
				\end{aligned}
			\end{gather*}
			for all~$n, m \geq 0$.
		\item
			Then polynomial ring in one variable,~$\kf[x]$, is a representation of~$\sllie(2, \kf)$ via the homomorphism of Lie~algebras~$\rho$ from~$\sllie(2, \kf)$ to~$\gllie(\kf[x])$ that is given by
			\[
				\rho(e)
				\defined
				\dd{x} \,,
				\qquad
				\rho(h)
				\defined
				-2x\dd{x} \,,
				\qquad
				\rho(f)
				\defined
				-\dd{x} \,,
			\]
			on the standard basis~$e$,~$h$,~$f$ of~$\sllie(2, \kf)$.
			The corresponding action of~$\sllie(2, \kf)$ on~$\kf[x]$ is given by
			\[
				e \act x^n = n x^{n-1} \,,
				\qquad
				h \act x^n = -2n x^n \,,
				\qquad
				f \act x^n = n x^{n+1}
			\]
			for every~$n \geq 0$.
			This is indeed an action of~$\sllie(2, \kf)$ on~$\kf[x]$ because
			\begin{gather*}
				e \act f \act x^n - f \act e \act x^n
				= -n(n+1) x^n + n(n-1) x^n
				= -2n x^n
				= h \act x^n
				= [e,f] \act x^n \,,
			\intertext{as well as}
				h \act e \act x^n - e \act h \act x^n
				= -2n(n-1) x^{n-1} + 2n^2 x^{n-1}
				= 2n x^{n-1}
				= 2e \act x^n
				= [h,e] \act x^n \,,
			\intertext{and}
				h \act f \act x^n - f \act h \act x^n
				= 2n(n+1) x^{n+1} - 2 n^2 x^{n+1}
				= 2n x^{n+1}
				= -2 f \act x^n
				= [h,f] \act x^n
			\end{gather*}
			for every~$n \geq 0$.
		\item
			Let~$M$ be a representation of~$\glie$ with corresponding homomorphism of Lie~algebras~$\rho$ from~$\glie$ to~$\gllie(M)$.
			Let~$\varphi$ be a homomorphism of Lie~algebras from~$\hlie$ to~$\glie$.
			Then the composite~$\rho \circ \varphi$ is a homomorphism of Lie~algebras from~$\hlie$ to~$\gllie(M)$, which makes the vector space~$M$ into a representation of~$\hlie$.
			The corresponding action of~$\hlie$ on~$M$ is given by
			\[
				x \act m = \varphi(x) \act m
				\qquad
				\text{for all~$x \in \hlie$ and~$m \in M$,}
			\]
			where the right hand side denotes the action of~$\glie$ on~$M$ corresponding to the representation~$\rho$.
		\item
			Let~$A$ be a~\algebra{$\kf$} and let~$M$ be a left~\module{$A$}.
			Then~$M$ is also a representation of~$A$ as a Lie~algebra because
			\[
				[a,b] \act m
				=
				(ab - ba) \act m
				=
				(ab - ba)m
				=
				abm - bam
				=
				a \act (b \act m) - b \act (a \act m)
			\]
			for all~$a, b \in A$ and~$m \in M$.
		\item
			The adjoint map
			\[
				\ad \colon \glie \to \gllie(\glie)
			\]
			is a homomorphism of Lie~algebras, which makes~$\glie$ into a representation of~$\glie$.
	\end{enumerate}
\end{examples}


\begin{definition}
	Let~$\glie$ be a Lie~algebra.
	The representation of~$\glie$ given by the homomorphism of Lie~algebras
	\[
		\ad
		\colon
		\glie
		\to
		\gllie(\glie) \,,
		\quad
		x
		\mapsto
		\ad(x)
		=
		[x,-]
	\]
	is the \defemph{adjoint representation}\index{adjoint representation} of~$\glie$.
\end{definition}


\begin{remark}
	It follows together with \cref{lie algebras act adjoint by derivations} that every Lie~algebras~$\glie$ acts on itself via derivations of itself through the adjoint representation.
	The author suspects that this is where much of the structure of Lie~algebras comes from.
\end{remark}


\begin{remark}
	\label{right representations}
	Let~$\glie$ be a~{\liealgebra{$\kf$}} and let~$M$ be a~\vectorspace{$\kf$}.
	A \defemph{right action}\index{action!right}\index{right action} of~$\glie$ on~$M$ is a~{\bilinear{$\kf$}} map
	\[
		M \times \glie
		\to
		M \,,
		\quad
		(m,x)
		\mapsto
		m \act x
	\]
	such that
	\begin{equation}
		\label{right action}
		m \act [x,y]
		=
		(m \act x) \act y - (m \act y) \act x
		\qquad
		\text{for all~$x, y \in \glie$,~$m \in M$.}
	\end{equation}
	It turns out that left actions and right actions are equivalent concepts, as we will now explain.

	We first note that \cref{correspodence between representations and actions} can be generalized to right actions:
	a map
	\[
		M \times \glie
		\to
		M \,,
		\quad
		(m,x)
		\mapsto
		m \act x
	\]
	is a right action if and only if the map
	\[
		\rho
		\colon
		\glie
		\to
		\gllie(M)
	\]
	given by
	\[
		\rho(x)(m)
		\defined
		m \act x
		\qquad
		\text{for all~$x \in \glie$ and~$m \in M$}
	\]
	is an anti-homomorphism of Lie~algebras.
	This means equivalently that~$\rho$ is a homomorphism of Lie~algebras from the opposite Lie~algebra~$\glie^{\op}$\index{opposite Lie algebra} to~$\gllie(M)$.
	Such a homomorphism does in turn correspond to a left action of~$\glie^{\op}$ on~$M$, which is given by
	\[
		x^{\op} \act m
		\defined
		m \act x
		\qquad
		\text{for all~$x \in \glie$ and~$m \in M$.}
	\]
	We have thus found {\onetoonetext} correspondences between the following concepts.
	\begin{equivalenceslist*}
		\item
			Right actions of~$\glie$ on~$M$.
		\item
			Left actions of~$\glie^{\op}$ on~$M$.
		\item
			Anti-homomorphisms of Lie~algebras from~$\glie$ to~$\gllie(M)$.
		\item
			Homomorphisms of Lie~algebras from~$\glie^{\op}$ to~$\gllie(M)$.
	\end{equivalenceslist*}

	We now recall \cref{lie algebra isomorphic to its opposite}, according to which the map
	\[
		\glie
		\to
		\glie^{\op} \,,
		\quad
		x
		\mapsto
		-x^{\op}
	\]
	is an isomorphism of Lie~algebras.
	By using this isomorphism of Lie~algebras we can now translate between right actions of~$\glie$ on~$M$, left actions of~$\glie^{\op}$ on~$M$, and then left actions of~$\glie$ on~$M$.
	More explicitely, if~$\glie$ acts on the vector space~$M$ from the right via
	\[
		M \times \glie
		\to
		M \,,
		\quad
		(m,x)
		\mapsto
		m \act x
		\qquad
		\text{for all~$x \in \glie$ and~$m \in M$,}
	\]
	then~$\glie$ acts on~$M$ from the left via
	\[
		x \act m
		\defined
		-m \act x
		\qquad
		\text{for all~$x \in \glie$ and~$m \in M$.}
	\]

	In the following we will sometimes encounter constructions which turn left actions into right actions.
	We are now able to translate these right actions back into left actions.
	In this way we only have to deal with left actions.
\end{remark}





