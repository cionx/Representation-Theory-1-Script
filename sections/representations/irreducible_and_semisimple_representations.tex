\section{Irreducible and Semisimple Representations}


\begin{definition}
	Let~$M$ be a representation of a Lie~algebra~$\glie$.
	\begin{enumerate}
		\item
			The representation~$M$ is \defemph{irreducible}\index{irreducible representation} or \defemph{simple}\index{simple representation} if it is nonzero and its only subrepresentations are~$0$ and itself.
		\item
			The representation~$M$ is \defemph{indecomposable}\index{indecomposable representation} if it is nonzero and it does not admit any decomposition~$M = N_1 \oplus N_2$ into subrepresentations~$N_1$ and~$N_2$ apart from the two decompositions~$M = M \oplus 0$ and~$M = 0 \oplus M$.
			If~$M$ is not indecomposable, then it is \defemph{decomposable}
		\item
			The representation~$M$ is~\defemph{completely reducible}\index{completely reducible representations} or~\defemph{semisimple}\index{semisimple representations} if it admits a decomposition
			\[
				M = \bigoplus_{i \in I} N_i
			\]
			into irreducible subrepresentations~$N_i$.
	\end{enumerate}
\end{definition}


\begin{remark}
	\leavevmode
	\begin{enumerate}
		\item
			A representation is irreducible if and only if it admits precisely two subrepresentations.
		\item
			Every irreducible representation indecomposable, but the converse does not hold in general not hold true.
		\item
			A representation is irreducible if and only if it is both indecomposable and semisimple.
	\end{enumerate}
\end{remark}


\begin{example}
	\leavevmode
	\begin{enumerate}
		\item
			Every {\onedimensional} representation is irreducible.
		\item
			The adjoint representation of a Lie~algebra~$\glie$ is irreducible if and only if~$\glie$ is nonzero and contains no ideals beside~$0$ and~$\glie$ itself.
			This is the case if and only if~$\glie$ is either {\onedimensional} (and thus abelian), or simple.
	\end{enumerate}
\end{example}


\begin{proposition}[Characterization of semisimple representations]
	Let~$M$ be a representation of a Lie~algebra~$\glie$.
	The following conditions on~$M$ are equivalent.
	\begin{equivalenceslist}
		\item
			The representation~$M$ is semisimple, i.e.~$M$ admits a decomposition~$M = \bigoplus_{i \in I} N_i$ into some irreducible subrepresentations~$N_i$ of~$M$.
		\item
			The representation~$M$ can be written as a (not necessarily direct) sum~$M = \sum_{i \in I} N_i$ for some irreducible subrepresentations~$N_i$ of~$M$.
		\item
			Every subrepresentation~$N$ of~$M$ admits a direct complement, i.e.\ there exists a subrepresentation~$P$ of~$M$ with~$M = N \oplus P$.
	\end{equivalenceslist}
\end{proposition}


\begin{proof}
	The proof is the same as for semisimple modules over a ring, which we won’t repeat here.
\end{proof}


\begin{corollary}
	Let~$\glie$ be a Lie~algebra.
	\begin{enumerate}
		\item
			Let~$M$ be a semisimple representation of~$\glie$.
			Every subrepresentation of~$M$ is again semisimple.
		\item
			Let~$M$ be a semisimple representation of~$\glie$ and let~$N$ be a subrepresentation of~$M$.
			The quotient representation~$M/N$ is again semisimple.
		\item
			Let~$M_\lambda$ with~$\lambda$ in~$\Lambda$ be a family of semisimple representations of~$\glie$.
			The direct sum~$\bigoplus_{\lambda \in \Lambda} M_\lambda$ is again semisimple.
		\item
			Let~$M$ be a representation of~$\glie$ and let~$N_\lambda$ with~$\lambda$ in~$\Lambda$ be a family of simisimple subrepresentations of~$\glie$.
			The intersection~$\bigcap_{\lambda \in \Lambda} N_\lambda$ and sum~$\sum_{\lambda \in \Lambda} N_\lambda$ are again semisimple.
		\qed
	\end{enumerate}
\end{corollary}


\begin{remark}
	Let~$A$ be a~{\algebra{$\kf$}}.
	Every simple~{\module{$A$}}~$M$ is cyclic and therefore satisfies the inequality~$\dim(M) \leq \dim(A)$.
	It follows that if~$A$ is finite-dimensional, then all simple~{\modules{$A$}} are also finite-dimensional, with their dimensions uniformly bounded by the dimension of~$A$.
	
	The same does not hold for finite-dimensional Lie~algebras.
	If~$\glie$ is a finite-dimensional Lie~algebra, then the irreducible representations of~$\glie$ do not have to be finite-dimensional.
	Moreover, those irreducible representations that are finite-dimensional can have arbitrarily large dimension.
	
	Indeed, we will see in \cref{highest weight irreps for sl2} that the Lie~algebra~$\sllie(2,\kf)$ admitsfor every nonzero, finite dimension~$d$ precisely one irreducible representation of dimension~$d$ up to isomorphism (if the field~$\kf$ is algebraically closed), and that~$\sllie(2,\kf)$ also admits infinite-dimensional irreducible representations.
	
	But we note that the following statements still hold for every Lie~algebra~$\glie$.
	\begin{enumerate}
		\item
			Every nonzero, finite-dimensional representation of~$\glie$ contains an irreducible subrepresentation.
			Indeed, we can take any subrepresentation of minimal nonzero dimension.
		\item
			There exists a finite-dimensional irreducible representation~$M$ for~$\glie$.
			Here we can take the one-dimensional trivial representation.
	\end{enumerate}
\end{remark}


\begin{proposition}(Schur’s lemma\footnote{Named after Issai Schur (1875--1941).})
	\index{Schur’s lemma}
	\label{schurs lemma}
	Let~$M$ and~$N$ be two representations of a Lie~algebra~$\glie$ and let~$f$ be a homomorphism of representations from~$M$ to~$N$.
	\begin{enumerate}
		\item
			If~$M$ is irreducible, then either~$f$ is injective or~$f = 0$, but not both.
		\item
			If~$N$ is irreducible, then either~$f$ is surjective or~$f = 0$, but not both.
		\item
			If both~$M$ and~$N$ are irreducible, then either~$f$ is bijective or~$f = 0$, but not both.
		\item
			If~$M$ is irreducible, then the endomorphism algebra~$\End_{\glie}(M)$ is a skew field extension of~$\kf$.
		\item
			\label{endomorphism algebra consists of scalars}
			Let~$\kf$ be algebraically closed.
			If~$M$ is both finite-dimensional and irreducible, then every endomorphism~$f$ of~$M$ is of the form~$f = \lambda \id_M$ for some scalar~$\lambda$ in~$\kf$.
			Thus~$\End_{\glie}(M) = \kf$.
	\end{enumerate}
\end{proposition}


\begin{proof}
	\leavevmode
	\begin{enumerate}
		\item
			The kernel of~$f$ is a subrepresentation of~$M$ and so either~$\ker(f) = 0$ or~$\ker(f) = M$, but not both because~$M$ is nonzero.
		\item
			The image of~$f$ is a subrepresentation of~$N$ and so either~$\im(f) = N$ or~$\im(f) = 0$, but not both because~$N$ is nonzero.
		\item
			This is a combination of the previous two statements.
		\item
			This is a reformulation of the previous statement.
			We only need to additionally observe that the algebra~$\End_{\glie}(M)$ is nonzero because~$M$ is nonzero (and thus~$\id_M$ is a nonzero element of~$\End_{\glie}(M)$).
		\item
			It follows from the finite-dimensionality of~$M$ that the endomorphism algebra~$\End_{\glie}(M)$ is finite-dimensional.
			It also follows from the irreducibility of~$M$ that~$\End_{\glie}(M)$ is a skew field extension of~$\kf$.
			We thus find that~$\End_{\glie}(M)$ is a finite-dimensional skew field extension of~$\kf$.
			It now follows that~$\End_{\glie}(M)$ equals~$\kf$ because~$\kf$ is algebraically closed.
		\qedhere
	\end{enumerate}
\end{proof}


\begin{remark}
	One can extend part~\ref{endomorphism algebra consists of scalars} of Schur’s lemma to the case that~$\kf$ is algebraically closed and that the dimension of~$M$ is strictly smaller than the cardinality of~$\kf$.
	This is a generalization of the above formulation because an algebraically closed field is always infinite.

	To prove this generalization we follow \cite{quillen_endomorphism_simple_module_enveloping_algebra} (where the proof is attributed to \cite{diximier_representations_lie_nilpotent}).
	Let~$M$ be an irreducible representation of a Lie~algebra~$\glie$ and let~$f$ be an endomorphism of of~$M$.
	Suppose that the dimension of~$M$ is strictly smaller than the cardinality of~$\kf$, and that~$\kf$ is algebraically closed.
	To show that~$f$ is a scalar multiple of the identity we proceed in two steps.

	We show first that~$f$ is algebraic over~$\kf$, i.e. that there exists some nonzero polynomial~$P(x)$ in~$\kf[t]$ with~$p(f) = 0$.
	Suppose otherwise.
	The endomorphism~$P(f)$ is then nonzero for every nonzero polynomial~$P(t)$ in~$\kf[t]$, and thus an isomorphism of~$M$ by Schur’s lemma.
	It follows that the homomorphism of~\algebras{$\kf$}
	\[
		\kf[t]
		\to
		\End_{\glie}(M) \,,
		\quad
		P(t)
		\mapsto
		P(f)
	\]
	extends to a homomorphism of~\algebras{$\kf$}
	\[
		\kf(t)
		\to
		\End_{\glie}(M) \,,
		\quad
		\frac{P(t)}{Q(t)}
		\mapsto
		P(f) Q(f)^{-1} \,.
	\]
	Through this homomorphisms of~\algebras{$\kf$} we can extend the~\vectorspace{$\kf$} structure of~$M$ to an~\vectorspace{$\kf(x)$} structure.
	It follows for the~\dimension{$\kf$} of~$M$ that
	\[
		\dim_{\kf}(M)
		=
		\dim_{\kf(x)}(M) \cdot \dim_{\kf}(\kf(x)) \,.
	\]
	The dimension of~$M$ as a~\vectorspace{$\kf(x)$} is nonzero because~$M$ is nonzero (since it is irreducible as a~\representation{$\glie$}).
	It follows that
	\[
		\dim_{\kf}(M)
		\geq
		\dim_{\kf}(\kf(x)) \,.
	\]
	But the~\dimension{$\kf$} of~$\kf(x)$ is larger or equal to the cardinality of~$\kf$ because the elements
	\[
		\frac{1}{t - \lambda}
		\quad
		\text{with~$\lambda \in \kf$}
	\]
	are linearly independent in~$\kf(x)$.
	It thus follows that the~\dimension{$\kf$} of~$M$ is at least the cardinality of~$\kf$, contradicting our assumption.

	We have shown that~$f$ is algebraic over~$\kf$
	There hence exist some nonzero polynomial~$P(t)$ in~$\kf[t]$ with~$P(f) = 0$.
	The polynomial~$P(x)$ decomposition into linear factors because the field~$\kf$ is algebraically closed.
	We therefore have the equality
	\[
		P(t) = (t - \lambda_1) \dotsm (t - \lambda_r)
	\]
	for some~$r \geq 1$ and scalars~$\lambda_1, \dotsc, \lambda_r$ in~$\kf$.
	It follows that the equality
	\[
		0
		=
		P(f)
		=
		(f - \lambda_1 \id_M) \circ \dotsb \circ (f - \lambda_r \id_M)
	\]
	holds in~$\End_{\glie}(M)$.
	But~$\End_{\glie}(M)$ is a skew field by Schur’s lemma, so it follows that one of the factors~$f - \lambda_i$ vanishes.
	This mains that~$f$ equals~$\lambda_i \id_M$.
\end{remark}





