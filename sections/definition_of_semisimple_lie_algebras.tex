\section{Definition and Basic Properties}


% TODO: Relative Killing form





\subsection{More on Bilinear Forms}


\begin{lemma}
  \label{bilinear forms and hom identification}
  Let~$V$ and~$W$ be representations of a Lie~algebra~$\glie$.
  Then the natural isomorphims of vector spaces
  \begin{alignat*}{2}
    \Phi_1
    &\colon
    \BF(V,W)
    \to
    \Hom_\kf(V, W^*) \,,
    &
    \quad
    \beta
    &\mapsto
    (
      v
      \mapsto
      \beta(v,-)
    )
  \shortintertext{and}
    \Phi_2
    &\colon
    \BF(V,W)
    \to
    \Hom_\kf(W, V^*) \,,
    &
    \quad
    \beta
    &\mapsto
    (
      w
      \mapsto
      \beta(-,w)
    )
  \end{alignat*}
  are already isomorphisms of representations.
  (Here~$\glie$ acts on~$\BF(V,W) \cong (V \tensor W)^*$ as discussed in \cref{associative is invariant}.)
\end{lemma}


\begin{proof}
  We have for all~$x \in \glie$,~$\beta \in \BF(V,W)$ and~$v \in V$,~$w \in W$ that
  \begin{align*}
    \Phi_1(x.\beta)(v)(w)
    &=
    ((x.\beta)(v,-))(w)
    \\
    &=
    (x.\beta)(v,w)
    \\
    &=
    - \beta(x.v, w) - \beta(v, x.w)
  \shortintertext{as well as}
    (x.\Phi_1(\beta))(v)(w)
    &=
    \bigl( x.\Phi_1(\beta)(v) - \Phi_1(\beta)(x.v) \bigr)(w)
    \\
    &=
    \bigl( x.\Phi_1(\beta)(v) - \Phi_1(\beta)(x.v) \bigr)(w)
    \\
    &=
    \bigl( x.\Phi_1(\beta)(v) - \Phi_1(\beta)(x.v) \bigr)(w)
    \\
    &=
    (x.\beta(v,-))(w) - \beta(x.v, w)
    \\
    &=
    - \beta(v,-)(x.w) - \beta(x.v, w)
    \\
    &=
    - \beta(v, x.w) - \beta(x.v, w) \,.
  \end{align*}
  This shows that~$\Phi_1$ is a homomorprhism, and hence isomorphism, of representations.
  For~$\Phi_2$ a similar caculation can be done.%
  \footnote{The author hasn’t actually done this but hopes that no sign problems occur.}
\end{proof}


\begin{corollary}
  \label{invariant bilinear form induces homomorphism of representations}
  Let~$V$ and~$W$ be representations of a Lie~algebra~$\glie$.
  Let~$\beta \colon V \times W \to \kf$ be a bilinear form with associated linear maps
  \begin{alignat*}{2}
    \phi_1
    &\colon
    V
    \to
    W^* \,,
    &
    \quad
    v
    &\mapsto
    \beta(v, -) \,,
  \shortintertext{and}
    \phi_2
    &\colon
    W
    \to
    V^* \,,
    &
    \quad
    w
    &\mapsto
    \beta(-, w) \,.
  \end{alignat*}
  Then the following conditions on~$\beta$ are equivalent:
  \begin{equivalenceslist}
    \item
      \label{beta invariant}
      $\beta$ is invariant.
    \item
      \label{phi1 a homomorphism}
      $\phi_1$ is a homomorphism of representations.
    \item
      \label{phi2 a homomorphism}
      $\phi_2$ is a homomorphism of representations.
  \end{equivalenceslist}
\end{corollary}


\begin{proof}
  The isomorphisms of representations~$\Phi_1$ from \cref{bilinear forms and hom identification} restrict to an isomorphism between the spaces of invariants~$\BF(V, W)^{\glie}$ and~$\Hom_\kf(V,W)^{\glie}$.
  We know from \cref{homomorphisms of representations as invariants} that~$\Hom_\kf(V,W)^{\glie} = \Hom_{\glie}(V,W)$, whence the equivalence of~\ref*{beta invariant} and~\ref*{phi1 a homomorphism}.
  The equivalence of~\ref*{beta invariant} and~\ref*{phi2 a homomorphism} can be seen in the same way.
\end{proof}


\begin{corollary}
  \label{associative bilinear form induces homomorphism of representations}
  Let~$\glie$ be a Lie~algebra.
  Let~$\beta \colon \glie \times \glie \to \kf$ be a bilinear form with associated linear maps
  \begin{alignat*}{3}
    \phi_1
    \colon
    \glie
    \to
    \glie^* \,,
    \quad
    x
    \mapsto
    \beta(x, -) \,,
  \shortintertext{and}
    \phi_2
    \colon
    \glie
    \to
    \glie^* \,,
    \quad
    y
    \mapsto
    \beta(-, y) \,.
  \end{alignat*}
  Then the following conditions on~$\beta$ are equivalent:
  \begin{equivalenceslist}
    \item
      $\beta$ is associative.
    \item
      $\phi_1$ is a homomorphism of representations.
    \item
      $\phi_2$ is a homomorphism of representations.
  \end{equivalenceslist}
\end{corollary}


\begin{proof}
  Let~$V$ and~$W$ be the adjoint representation of~$\glie$.
  We know from~\cref{associative is invariant} that~$\beta$ is associative if and only if~$\beta$ is invariant.
  We can therefore apply \cref{invariant bilinear form induces homomorphism of representations}.
\end{proof}


\begin{recall}[Non-degeneracy]
  Given two vector spaces~$V$ and~$W$ a bilinear form~$\beta \colon V \times W \to \kf$ is \defemph{non-degenerate}\index{non-degenerate bilinear form} if
  \begin{equivalenceslist}
    \item
      \label{nondegenerate in the left}
      for every nonzero~$v \in V$ there exists some~$w \in W$ with~$\beta(v,w) \neq 0$ and
    \item
      \label{nondegenerate in the right}
      for every nonzero~$w \in W$ there exists some~$v \in V$ with~$\beta(v,w) \neq 0$.
  \end{equivalenceslist}
  If more generally condition~\ref*{nondegenerate in the left} is satisfied then~$\beta$ is \defemph{non-degenerate in the left argument} and if condition~\ref*{nondegenerate in the right} is satisfied then~$\beta$ is \defemph{non-degenerate in the right argument}.
  The bilinear form~$\beta$ is therefore non-degenerate if and only if it is both non-degenerate in the left argument and non-degenerate in the right argument.
  
  If~$\phi_1 \colon V \to W^*$ and~$\phi_2 \colon W \to V^*$ are the linear maps associated to~$\beta$ then the conditions~\ref*{nondegenerate in the left} and~\ref*{nondegenerate in the right} may be rewritten, stating that
  \begin{equivalenceslist}[label = \roman*')]
    \item
      the linear map~$\phi_1 \colon V \to W^*$ is injective,
    \item
      the linear map~$\phi_2 \colon W \to V^*$ is injective.
  \end{equivalenceslist}
  
  Suppose now that both~$V$ and~$W$ are finite dimensional.
  For~$\beta$ to be non-degenerate in the left argument we need that~$\dim V \leq \dim W^* = \dim W$ because we need~$\phi_1$ to be injective.
  For~$\beta$ to be non-degenerate in the right argument we similarly need that~$\dim W \leq \dim V$.
  Hence for~$\beta$ to even have a chance at being non-degenerate we need that~$\dim V = \dim W$.%
  \footnote{
  This conclusion does not hold for infinite dimensional vector spaces:
  If~$V$ is any infinite dimensional then the evaluation map~$\beta \colon V \times V^* \to \kf$ given by~$\beta(v,\alpha) = \alpha(v)$ is a non-degenerate bilinear form.
  The linear map~$\phi_1 \colon V \to V^{**}$ is the canonical inclusion and the linear map~$\phi_2 \colon V^* \to V^*$ is just the identity, both of which are injective.
  But the dual space~$V^*$ has a strictly larger dimension than~$V$ itself.}
  
  So suppose now that both~$V$ and~$W$ have the same finite dimension~$n$.
  Then for basis~$v_1, \dotsc, v_n$ of~$V$ and a basis~$w_1, \dotsc, w_n$ of~$W$ we can represent the bilinear form~$\beta$ by a square matrix~$A \in \Mat_n(\kf)$ that is given by
  \[
    A_{ij}
    \defined
    \beta(v_i, w_j)
  \]
  for all~$i,j = 1, \dotsc, n$.
  Then with respect to the basis~$v_1, \dotsc, v_n$ of~$V$ and the (dual) basis~$w_1^*, \dotsc, w_n^*$ of~$W$ the linear map~$\phi_1 \colon V \to W^*$ is represented by the matrix~$A^T$;
  with respect to the basis~$w_1, \dotsc, w_n$ of~$W$ and the (dual) basis~$v_1^*, \dotsc, v_n^*$ of~$V^*$ the linear map~$\phi_2 \colon W \to V^*$ is given by the matrix~$A$.
  We therefore find that
  \begin{align*}
        {}& \text{$\beta$ is non-degenerate in the left argument} \\
    \iff{}& \text{$\phi_1$ is injective}  \\
    \iff{}& \text{$A^T$ is injective} \\
    \iff{}& \text{$A^T$ is invertible}
  \end{align*}
  and similarly that
  \begin{align*}
        {}& \text{$\beta$ is non-degenerate in the right argument} \\
    \iff{}& \text{$\phi_2$ is injective}  \\
    \iff{}& \text{$A$ is injective} \\
    \iff{}& \text{$A$ is invertible}  \,.
  \end{align*}
  But~$A$ is invertible if and only its transpose~$A^T$ is invertible.
  
  By using these observations, and that~$\dim V = \dim W = \dim V^* = \dim W^*$ we see that all of the following conditions are equivalent:
  \begin{equivalenceslist}
    \item
      The bilinear form~$\beta$ is non-degenerate.
    \item
      The bilinear form~$\beta$ is non-degenerate in the left argument.
    \item
      The bilinear form~$\beta$ is non-degenerate in the right argument.
    \item
      The linear map~$\phi_1$ is injective.
    \item
      The linear map~$\phi_1$ is bijective.
    \item
      The linear map~$\phi_2$ is injective.
    \item
      The linear map~$\phi_2$ is bijective.
    \item
      The matrix~$A$ is invertible.
  \end{equivalenceslist}
\end{recall}


\begin{recall}[Dual bases]
  Let~$\beta \colon V \times W \to \kf$ be a bilinear form where~$V$ and~$W$ are two~{\vectorspaces{$\kf$}}.
  
  One sufficient criterion for~$\beta$ to be non-degenerate in any one argument is the existence of a dual family:
  Suppose that~$(v_i)_{i \in I}$ is some basis of~$V$, for which there exists elements~$(w^i)_{i \in I}$ with~$\beta(v_i, w^i) = \delta_{ij}$ for all~$i, j \in I$.
  Then~$\beta$ is non-degenerate in the left argument:
  If~$v \in V$ with~$v \neq 0$ then in the linear combination~$v = \sum_{i \in I} \lambda_i v_i$ we have~$\lambda_j \neq 0$ for some~$j$.
  Then
  \[
    \beta(v, w^{\spacing j})
    =
    \beta\left( \sum_{i \in I} v_i, w^{\spacing j} \right)
    =
    \sum_{i \in I} \lambda_i \underbrace{\beta(v_i, w^{\spacing j})}_{= \delta_{ij}}
    =
    \lambda_j
    \neq
    0 \,.
  \]
  We note that by a similar calculation the family~$(w^i)_{i \in I}$ is again linearly independent.
  
  Similarly,~$\beta$ is non-degenerate in the right argument if there exists for some basis~$(w_j)_{j \in J}$ of~$W$ a family~$(v^{\spacing j})_{j \in J}$ in~$V$ with~$\beta(v^i, w_j) = \delta_{ij}$ for all~$i, j \in J$.
  
  But these sufficient criterion are not necessary:
  Consider for an infinite dimensional vector space~$V$ the standard bilinear form~$\varepsilon \colon V \times V^* \to \kf$ given by~$\epsilon(v, \varphi) = \varphi(v)$, which is non-degenerate (i.e.\ non-degenerate in both arguments).
  Then for every basis~$(v_i)_{i \in I}$ there does indeed exist a dual family~$(v_i^*)_{i \in I}$ in~$V^*$ with~$\epsilon(v_i, v_j^*) = \delta_{ij}$ for all~$i, j \in I$.
  But for no basis~$(\varphi_j)_{i \in I}$ of~$V^*$ can such a dual family in~$V$ exist:
  We have seen above that this famliy would be linearly independent, which is not possible because~$V$ is of strictly smaller dimension than~$V^*$.
  
  This problem does not occur in the finite dimensional case:
  If both~$V$ and~$W$ are of the same finite dimension~$n$ then the following conditions on~$\beta$ are equivalent:
  \begin{equivalenceslist}
    \item
      $\beta$ is non-degenerate.
    \item
      For some basis~$v_1, \dotsc, v_n$ of~$V$ there exists a dual family~$w^1, \dotsc, w^n$ of~$W$ with~$\beta(v_i, w^{\spacing j}) = \delta_{ij}$ for all~$i, j = 1, \dotsc, n$.
    \item
      For every basis~$v_1, \dotsc, v_n$ of~$V$ there exists a dual family~$w^1, \dotsc, w^n$ of~$W$ with~$\beta(v_i, w^{\spacing j}) = \delta_{ij}$ for all~$i, j = 1, \dotsc, n$.
    \item
      For some basis~$w_1, \dotsc, w_n$ of~$W$ there exists a dual family~$v^1, \dotsc, v^n$ of~$V$ with~$\beta(v^i, w_j) = \delta_{ij}$ for all~$i, j = 1, \dotsc, n$.
    \item
      For every basis~$w_1, \dotsc, w_n$ of~$W$ there exists a dual family~$v^1, \dotsc, v^n$ of~$V$ with~$\beta(v^i, w_j) = \delta_{ij}$ for all~$i, j = 1, \dotsc, n$.
  \end{equivalenceslist}
  The dual family is in each case again linearly independent and hence a basis;
  it is furthermore unique with the given property.
  
  To construct for a basis~$v_1, \dotsc, v_n$ of~$V$ the corresponding dual basis~$w^1, \dotsc, w^n$ of~$W$ one considers the induced isomorphism of vector spaces~$\psi \colon W \to V^*$ given by~$\psi(w) = \beta(-,w)$.
  Then necessarily~$w_i = \psi^{-1}(v_i^*)$ for every~$i = 1, \dotsc, n$ where~$v_1^*, \dotsc, v_n^*$ is the usual dual basis of the dual space~$V^*$.
  This then shows both the existence and uniqueness of such a dual basis.
  (The usual dual basis~$v_1^*, \dotsc, v_n^*$ of~$V^*$ is itself a dual basis in the more general sense of non-degenerate bilinear forms, namely via the standard bilinear form~$\varepsilon \colon V \times V^* \to \kf$.)
\end{recall}


\begin{recall}[Orthogonals and non-degeneracy]
  \label{reviewing orthogonals}
  If~$V$ is a finite dimensional (real or complex) inner product space then for every linear subspace~$U$ of~$V$ the restriction of the inner product to~$U$ is again an inner product,~$V = U \oplus U^\perp$ and~$(U^\perp)^\perp = U$.
  That the restriction is again an inner product holds because positive definiteness is a pointwise property.
  That~$U \cap U^\perp = 0$ also follows from the positive definiteness, and that~$V = U \oplus U^\perp$ can then be concluded from the dimension formula~$\dim V = \dim U + \dim U^\perp$.
  
  If~$\beta$ is non-degenerate symmetric bilinear form on~$V$ then for an arbitrary linear subspace~$U$ of~$V$ the restriction of~$\beta$ to~$U$ does not have to be non-degenerate again.
  The problem is that even though there exists for every~$u \in U$ some~$v \in V$ with~$\beta(u,v) \neq 0$, no such element~$v$ needs to exist in the subspace~$U$.
  
  But if~$V$ is finite dimensional then the dimension formula
  \[
    \dim V
    =
    \dim U + \dim U^\perp
  \]
  still holds.
  This can be seen by considering the composition~$\rho \circ \phi \colon V \to U^*$ of the isomorphism~$\phi \colon V \to V^*$,~$v \mapsto \beta(-,v)$ and the restriction map~$\rho \colon V^* \to U^*$,~$\varphi \mapsto \restrict{\varphi}{U}$.
  Then both~$\rho$ and~$\phi$ are surjective whence~$\rho \circ \phi$ is surjective.
  The kernel of~$\rho \circ \phi$ is given by the orthogonal~$U^\perp$.
  Hence
  \[
    \dim V
    =
    \dim \im (\rho \circ \phi)
    +
    \dim \ker (\rho \circ \phi)
    =
    \dim U^* + \dim U^\perp
    =
    \dim U + \dim U^\perp \,.
  \]
  It follows from that~$(U^\perp)^\perp = U$ because~$U \subseteq (U^\perp)^\perp$ and~$\dim (U^\perp)^\perp = \dim V - \dim U^\perp = \dim U$.
  
  We note that if~$\beta$ is degenerate then still~$\dim V \leq \dim U + \dim U^\perp$.
  
  If~$\beta$ is a non-degenerate symmetric bilinear form on a finite dimensional vector space~$V$ then we have moreover the following equivalent conditions for any linear subspace~$U$ of~$V$:
  \begin{equivalenceslist}
    \item
      \label{restriction to U is nondegenerate}
      The restriction of~$\beta$ to~$U$ is again non-degenerate.
    \item
      \label{restriction to Uorth is nondegenerate}
      The restriction of~$\beta$ to~$U^\perp$ is again non-degenerate.
    \item
      \label{U does not intersect Uorth}
      $U \cap U^\perp = 0$.
    \item
      \label{sum of U and Uorth}
      $V = U + U^\perp$.
    \item
      \label{direct sum of U and Uorth}
      $V = U \oplus U^\perp$.
  \end{equivalenceslist}
  The equivalence of~\ref*{restriction to U is nondegenerate} and~\ref*{U does not intersect Uorth} holds because~$U \cap U^\perp$ is the radical of the restriction of~$\beta$ to~$U$.
  The equivalences of~\ref*{U does not intersect Uorth},~\ref*{sum of U and Uorth} and~\ref*{direct sum of U and Uorth} follows from the dimension formula~$\dim V = \dim U + \dim U^\perp$.
  The last three conditions are symmetric in~$U$ and~$U^\perp$ because~$(U^\perp)^\perp = U$.
  The equivalence of~\ref*{restriction to U is nondegenerate} and~\ref*{restriction to Uorth is nondegenerate} therefore follows from the equivalences of~\ref*{restriction to U is nondegenerate} and~\ref*{U does not intersect Uorth}.
\end{recall}


\begin{example}
  The trace form~$(-,-)_{\tr}$ on~$\Mat_n(\kf)$ is non-degenerate because the standard basis~$(E_{ij})_{i,j = 1, \dotsc, n}$ of~$\Mat_n(\kf)$ has as its dual the transposed standard basis~$(E_{ji})_{i,j = 1, \dotsc, n}$.
\end{example}


\begin{corollary}
  \label{associative non-degenerate bilinear forms induce isomorphism to the dual}
  If~$\glie$ is a finite dimensional Lie~algebra and~$\beta \colon \glie \times \glie \to \kf$ is a bilinear form then for the associated bilinear maps
% TODO: Fix alignment.
  \begin{gather*}
    \phi_1
    \colon
    \glie
    \to
    \glie^*,
    \quad
    x
    \mapsto
    \beta(x,-)
  \shortintertext{and}
    \phi_2
    \colon
    \glie
    \to
    \glie^*,
    \quad
    y
    \mapsto
    \beta(-,y)
  \end{gather*}
  The following conditions are equivalent:
  \begin{equivalenceslist}
    \item
      $\beta$ is associative and non-degenerate.
    \item
      $\phi_1$ is an isomorphism of representations.
    \item
      $\phi_2$ is an isomorphism of representations.
    \qed
  \end{equivalenceslist}
\end{corollary}


\begin{definition}
 Let~$V$ be a vector space and let~$\beta \colon V \times V \to \kf$ be a symmetric bilinear form.
 Then
 \[
  \gls*{radical bilinear}
  \defined
  \{
    x \in V
  \suchthat
    \text{$\beta(x,y) = 0$ for every~$y \in V$}
  \}
 \]
 is the~\defemph{radical}\index{radical!of a bilinear form} of~$\beta$.
\end{definition}


\begin{lemma}
  \label{orthogonal complement of an ideal is again an ideal}
  Let~$\glie$ be a Lie~algebra over an arbitrary field~$\kf$ and~$\beta \colon \glie \times \glie \to \kf$ an associative symmetric bilinear form.
  Then for any ideal~$I$ in~$\glie$ the orthogonal
  \[
    I^\perp
    \defined
    \{
      x \in \glie
    \suchthat
      \text{$\beta(x,y) = 0$ for every $y \in I$}
    \}
  \]
  is again an ideal in~$\glie$.
\end{lemma}


\begin{proof}
  The associated linear map~$\phi \colon \glie \to \glie^*$ given by~$\phi(x) = \beta(x,-)$ is a homomorphism of representations because~$\beta$ is associative.
  The inclusion~$I \to \glie$ is a homomorphism of representions, and so its dual map
  \[
    \glie^*
    \to
    I^* \,,
    \quad
    \varphi
    \mapsto
    \restrict{\varphi}{I}
  \]
  is again a homomorphism of representations.
  It follows that the composition
  \[
    \psi
    \colon
    \glie
    \to
    I^* \,,
    \quad
    x
    \mapsto
    \restrict{\beta(x,-)}{I}
  \]
  is homomorphism of representations.
  Hence~$I^\perp = \ker \psi$ is a subrepresentation of~$\glie$, i.e.\ an ideal in~$\glie$.
\end{proof}


\begin{corollary}
  \label{radical of bilinear form is an ideal}
  If~$\beta$ is an associative symmetric bilinear form on a Lie~algebra~$\glie$ then its radical~$\rad \beta$ is an ideal in~$\glie$.
\end{corollary}


\begin{proof}
  Apply \cref{orthogonal complement of an ideal is again an ideal} to~$I = \glie$.
\end{proof}


\begin{remark}
 The proof of \ref{orthogonal complement of an ideal is again an ideal} did not use that~$\beta$ is symmetric.
 This artficial restraint is only there to simplify the situation and notation (we do not need to distinguish between orthogonal complements from the left and from the right).
 The main example of an associative bilinear form will be the Killing~form, which is symmetric, so this assumption will pose no problems to us.
\end{remark}





\subsection{Definition and Characterizations of Semisimple Lie Algebras}



\begin{definition}
  A Lie~algebra~$\glie$ is \defemph{semisimple}\index{semisimple!Lie algebra}\index{Lie algebra!semisimple} if it is the direct sum of finitely many simple ideals, i.e.\ if there exist ideals~$I_1, \dotsc, I_n$ in~$\glie$ with~$\glie = I_1 \oplus \dotsb \oplus I_n$ such that every~$I_j$ is simple (as a Lie~algebra).
\end{definition}


\begin{recall}
  If a Lie~algebra~$\glie$ can be decomposed into a direct sum~$\glie = I_1 \oplus \dotsb \oplus I_n$ where~$I_1, \dotsc, I_n$ are ideals of~$\glie$ then it follows (see \cref{direct sum of ideals}) that
  \[
    [x_1 + \dotsb + x_n, y_1 + \dotsb + y_n]
    =
    [x_1, y_1] + \dotsb + [x_n, y_n]
  \]
  for all~$x_j, y_j \in I_j$ where~$j = 1, \dotsc, n$.
  In particular
  \[
    [\glie, \glie]
    =
    [I_1, I_1] \oplus \dotsb \oplus [I_n, I_n]
  \]
  and
  \[
    \centerlie(\glie)
    =
    \centerlie(I_1) \oplus \dotsb \oplus \centerlie(I_n)  \,.
  \]
\end{recall}


\begin{lemma}
  Let~$\glie$ be a semisimple Lie~algebra.
  Then~$[\glie, \glie] = \glie$ and~$\centerlie(\glie) = 0$.
\end{lemma}


\begin{proof}
 Let~$I_1, \dotsc, I_n$ be simple ideals in~$\glie$ with~$\glie = I_1 \oplus \dotsb \oplus I_n$.
 Then
 \begin{gather*}
  [\glie, \glie]
  =
  [I_1, I_1] \oplus \dotsb \oplus [I_n, I_n]
  =
  I_1 \oplus \dotsb \oplus I_n
  =
  \glie
 \shortintertext{as well as}
  \centerlie(\glie)
  =
  Z(I_1) \oplus \dotsb \oplus Z(I_n)
  =
  0 \oplus \dotsb \oplus 0
  =
  0
 \end{gather*}
 because~$I_1, \dotsc, I_n$ are simple.
\end{proof}


\begin{corollary}
  \label{representation of semisimple lie algebra are traceless}
    Let~$\glie$ be a semisimple Lie~algebra and let~$\rho \colon \glie \to \gllie(V)$ be a finite dimensional representation of~$\glie$.
    Then~$\rho(\glie)$ is contained in~$\sllie(V)$, i.e.~$\tr(\rho(x)) = 0$ for every~$x \in \glie$.
\end{corollary}


\begin{proof}
  We observe that~$\rho(\glie) = \rho([\glie,\glie]) = [\rho(\glie), \rho(\glie)] \subseteq [\gllie(V), \gllie(V)] = \sllie(V)$.
\end{proof}


\begin{corollary}
  Every {\onedimensional} representation of a semisimple Lie~algebra~$\glie$ is trivial.
\end{corollary}


\begin{proof}
  The image of the corresponding Lie~algebra homomorphism~$\rho \colon \glie \to \gllie(V)$ is contained in~$\sllie(V) = 0$.
\end{proof}


\begin{lemma}
  \label{properties of simple decompositions}
  Let~$\glie$ be a Lie~algebra.
  \begin{enumerate}
    \item
      \label{intersection of simples}
      If~$I$ and~$J$ are two simple ideals in a Lie~algebra~$\glie$ then~$I \cap \spacing J = 0$ or~$I = J$.
    \item
      If~$\glie = I_1 \oplus \dotsb \oplus I_n$ is a decomposition into ideals and~$J$ is any other simple ideal in~$\glie$ then~$J = I_i$ for exactly one~$i$.
  \end{enumerate}
\end{lemma}


\begin{proof}
  \leavevmode
  \begin{enumerate}
    \item
      If then~$I \cap \spacing J \neq 0$ then this intersection is a nonzero ideal in~$\glie$, and hence a nonzero ideal in both~$I$ and~$J$.
      Then~$I = I \cap \spacing J = J$ because~$I$ and~$J$ are simple.
    \item
      We have~$J = [\glie, J]$ because on the one hand~$[\glie, J] \subseteq J$ because~$J$ is an ideal and on the other hand~$J = [\spacing J,J] \subseteq [\glie, J]$ because~$J$ is simple.
      It follows that
      \[
        J
        =
        [\spacing J,J]
        =
        \left[ \bigoplus_{i=1}^n I_i, J \right]
        =
        \sum_{i=1}^n [I_i, J] \,.
      \]
      The sum~$\sum_{i=1}^n [I_i, J]$ is direct because the summand~$[I_i, J]$ is contained in~$I_i$, the sum of which is direct.
      Hence
      \begin{equation}
        \label{decomposing into direct sum}
        J
        =
        \bigoplus_{i=1}^n [I_i, J]  \,.
      \end{equation}
      It follows from~$J \neq 0$ that~$[I_i, J] \neq 0$ for some~$i$.
      It follows from part~\ref*{intersection of simples} that already~$I_i = J$.
      It follows from the directness of the sum~\eqref{decomposing into direct sum} that~$[\spacing J, I_j] = 0$ for every~$j \neq i$, and hence~$J \neq I_j$ for every~$j \neq i$.
    \qedhere
  \end{enumerate}
\end{proof}


\begin{corollary}
  \label{uniqueness of semisimple decomposition}
  Let~$\glie$ be a Lie~algebra and let~$I_1, \dotsc, I_n$ be simple ideals in~$\glie$ with~$\glie = I_1 \oplus \dotsb \oplus I_n$.
  Then the ideals~$I_1, \dotsc, I_n$ are unique up to reordering.
\end{corollary}


\begin{proof}
  Let~$J_1, \dotsc, J_m$ be simple ideals in~$\glie$ with~$\glie = J_1 \oplus \dotsb \oplus J_m$.
  Then by \cref{properties of simple decompositions} there exists for every~$i \in \{1, \dotsc, n\}$ a unique index~$\sigma(i) \in \{1, \dotsc, m\}$ with~$I_i = J_{\sigma(i)}$.
  In the same way we find that there exists for every index~$j \in \{1, \dotsc, m\}$ some index~$\tau(j) \in \{1, \dotsc, n\}$ with~$J_j = I_{\tau(j)}$.
  Then~$I_{\tau(\sigma(i))} = I_i$ and hence~$\tau(\sigma(i)) = i$ for every~$i \in \{1, \dotsc, n\}$.
  Similarly~$\sigma(\tau(j)) = j$ for every~$j \in \{1, \dotsc, m\}$.
  This shows together that the maps~$\sigma$ and~$\tau$ are mutually inverse bijections, which give the claimed reordering.
\end{proof}
 
 
\begin{remark}
  If~$\glie$ is a semisimple Lie~algebra then we will often just talk about \emph{the} decomposition of~$\glie$ into  direct sum of simple ideals, as this decomposition is unique up to reordering of the summands.
\end{remark}


\begin{fluff}
  We will now show that finite dimensional semisimple Lie~algebras can equivalently be described in terms of their Killing form.
\end{fluff}


\begin{lemma}
  \label{rad kappa is a solvable ideal}
  Let~$\glie$ be a finite dimensional Lie~algebra with Killing~form~$\kappa$.
  Then~$\rad \kappa$ is a solvable ideal in~$\glie$
\end{lemma}


\begin{proof}
  \Cref{radical of bilinear form is an ideal} shows that~$\rad \kappa$ is an ideal in~$\glie$.
  It follows from \cref{restriction of the killing form to an ideal} that
  \[
    \kappa_{\rad \glie}(x,y)
    =
    \kappa(x,y)
    =
    0
  \]
  for all~$x, y \in \rad \kappa$.
  Hence~$\rad \kappa$ is solvable by Cartan’s criterion.
\end{proof}


\begin{corollary}
  If~$\glie$ is a finite dimensional Lie~algebra with Killing form~$\kappa$ then~$\rad \kappa \subseteq \rad \glie$.
  \qed
\end{corollary}


\begin{lemma}
  \label{orthogonal ideals with respect to the killing form}
  Let~$\glie$ be a finite dimensional Lie~algebra and suppose that~$\glie = I \oplus J$ for two ideals~$I$ and~$J$.
  \begin{enumerate}
    \item
      The ideals~$I$ and~$J$ are mutually orthogonal with respect to the Killing form of~$\glie$.
    \item
      If~$x,y \in \glie$ are given by~$x = x_1 + x_2$ and~$y = y_1 + y_2$ with respect to the decomposition~$\glie = I \oplus J$ then
      \[
        \kappa_{\glie}(x,y)
        =
        \kappa_I(x_1, y_1) + \kappa_J(x_2, y_2) \,.
      \]
  \end{enumerate}
\end{lemma}


\begin{proof}
  \leavevmode
  \begin{enumerate}
    \item
      We have for all~$x \in I$ and~$y \in J$ that~$\ad(x) \ad(y) = 0$ because
      \[
        \ad(x)\ad(y)(\glie)
        =
        [x,[y,\glie]]
        \subseteq
        [I,[\spacing J,\glie]]
        \subseteq
        [I,J]
        \subseteq
        I \cap \spacing J
        =
        0 \,.
      \]
      Is follows that
      \[
        \kappa(x,y)
        =
        \tr(\ad(x)\ad(y))
        =
        \tr(0)
        =
        0 \,.
      \]
    \item
      We know from \cref{restriction of the killing form to an ideal} that~$\kappa(x_1, y_1) = \kappa_I(x_1, y_1)$ and~$\kappa(x_2, y_2) = \kappa_J(x_2, y_2)$.
      Therefore
      \begin{align*}
        \kappa(x,y)
        &=
        \kappa(x_1 + x_2, y_1 + y_2)
        \\
        &=
          \kappa(x_1, y_1)
        + \underbrace{\kappa(x_1, y_2)}_{=0}
        + \underbrace{\kappa(x_2, y_1)}_{=0}
        + \kappa(x_2, y_2)
        \\
        &=
        \kappa_I(x_1, y_1) + \kappa_J(x_2, y_2) \,,
      \end{align*}
      as desired.
    \qedhere
  \end{enumerate}
\end{proof}


\begin{proposition}
  \label{characterisation of zero radical}
  For any finite dimensional Lie algebra~$\glie$ the following conditions are equivalent:
  \begin{equivalenceslist}
    \item
      \label{killing form is nondegenerate}
      The Killing~form~$\kappa$ of~$\glie$ is non-degenerate (i.e.~$\rad \kappa = 0$).
    \item
      \label{contains no solvable ideal}
      $\glie$ contains no nonzero solvable ideals (i.e.~$\rad \glie = 0$).
    \item
      \label{contains no abelian ideal}
      $\glie$ contains no nonzero abelian ideals.
  \end{equivalenceslist}
\end{proposition}


\begin{proof}
  \leavevmode
  \begin{implicationlist}
    \item[\ref*{killing form is nondegenerate}~$\implies$~\ref*{contains no abelian ideal}:]
      Let~$I$ be an abelian ideal in~$\glie$.
      We show that~$I$ is contained in the radical of~$\kappa$, from which it then follows that~$I = 0$ because~$\rad \kappa = 0$.
      We need to show that~$\kappa(x,y) = 0$ for all~$x \in I$ and~$y \in \glie$.
      We have~$(\ad(x) \ad(y))^2 = 0$ because
      \begin{align*}
        (\ad(x) \ad(y))^2(\glie)
        &=
        \ad(x)\ad(y)\ad(x)\ad(y)(\glie)
        \\
        &\subseteq
        \ad(x)\ad(y)\ad(x)(\glie)
        \\
        &\subseteq
        \ad(x)\ad(y)(I)
        \\
        &\subseteq
        \ad(x)(I)
        \\
        &\subseteq
        [I,I] \,,
      \end{align*}
      where we use that~$\ad(x)(\glie) \subseteq I$ because~$I$ is an ideal.
      Now~$[I,I] = 0$ because~$I$ is abelian, and thus~$(\ad(x)\ad(y))^2 = 0$.
      This shows that~$\ad(x)\ad(y)$ is nilpotent, which gives
      \[
        \kappa(x,y)
        =
        \tr(\ad(x) \ad(y))
        =
        0 \,.
      \]
    \item[\ref*{contains no abelian ideal}~$\implies$~\ref*{contains no solvable ideal}:]
      Suppose that~$I$ is a nonzero solvable ideal in~$I$.
      Then~$I^{(n)} \neq 0$ but~$I^{(n+1)} = 0$ for some~$n \geq 1$.
      This means that~$I^{(n)}$ is a nonzero abelian ideal in~$\glie$.
    \item[\ref*{contains no solvable ideal}~$\implies$~\ref*{killing form is nondegenerate}:]
      The radical~$\rad \kappa$ is a solvable ideal by \cref{rad kappa is a solvable ideal}.
    \qedhere
  \end{implicationlist}
\end{proof}


\begin{corollary}
  \label{decomposition into orthogonals for semisimple}
  Let~$\glie$ be a finite dimensional Lie~algebra and let~$I$ be an ideal in~$\glie$.
  Let~$I^\perp$ be the orthogonal of~$I$ with respect to the Killing~form of~$\glie$.
  If at least one of the Lie~algebras~$\glie$,~$I$ or~$I^\perp$ has non-degenerate Killing~form then~$\glie = I \oplus I^\perp$.
\end{corollary}


\begin{proof}
  The orthogonal~$I^\perp$ is again an ideal in~$\glie$ by \cref{orthogonal complement of an ideal is again an ideal}.
  The intersection~$I \cap I^\perp$ is thus an ideal in~$\glie$.
  Then~$\restrict{\kappa}{I \cap I^\perp} = 0$ and this restriction is the Killing~form of~$I \cap I^\perp$ by \cref{restriction of the killing form to an ideal}.
  It follows from Cartan’s~criterion that the ideal~$I \cap I^\perp$ is solvable.
  If one of the Lie~algebras~$\glie$,~$I$ and~$I^\perp$ has non-degenerate Killing~form then~$I \cap I^\perp = 0$ by \cref{characterisation of zero radical} because~$I \cap I^\perp$ is an ideal in any of these Lie~algebras.
  It also holds that~$\dim \glie \leq \dim I + \dim I^\perp$.
  Thus~$\glie = I \oplus I^\perp$.
\end{proof}


\begin{theorem}[Characterizations of finite dimensional semisimple Lie~algebras]
  \label{characterizations of fd semisimple lie algebras}
  For any finite dimensional Lie algebra~$\glie$ the following conditions are equivalent:
  \begin{equivalenceslist}
    \item
    \label{product of simple lie algebras general}
      $\glie \cong \glie_1 \times \dotsb \times \glie_r$ for some~$n \geq 0$ and simple Lie~algebras~$\glie_1, \dotsc, \glie_n$.
    \item
      \label{sum of simple ideals general}
      $\glie = I_1 \oplus \dotsb \oplus I_m$ for some~$m \geq 0$ and simple ideals~$I_1, \dotsc, I_m$ of~$\glie$ (i.e.~$\glie$ is semisimple).
    \item
      \label{killing form is nondegenerate general}
      The Killing~form~$\kappa$ of~$\glie$ is non-degenerate (i.e.~$\rad \kappa = 0$).
    \item
      \label{contains no solvable ideal general}
      $\glie$ contains no nonzero solvable ideals (i.e.~$\rad \glie = 0$).
    \item
      \label{contains no abelian ideal general}
      $\glie$ contains no nonzero abelian ideals.
  \end{equivalenceslist}
\end{theorem}


\begin{proof}
  The equivalence of the conditions~\ref*{killing form is nondegenerate general},~\ref*{contains no solvable ideal general} and~\ref*{contains no abelian ideal general} is \cref{characterisation of zero radical}.
  \begin{implicationlist}
    \item[\ref*{killing form is nondegenerate general}~$\implies$~\ref*{sum of simple ideals general}:]
      We show the implication by induction over the dimension~$n$ of~$\glie$.
      If~$n = 0$ then~$\glie = 0$ is the empty sum over all its simple ideals.
      Suppose that~$n \geq 1$ and that the implication holds for all strictly smaller dimensions.
      If~$\glie$ is simple then we are already finished.
      Otherwise~$\glie$ contains some nonzero proper ideal~$I$.
      Then~$\glie = I \oplus I^\perp$ by \cref{decomposition into orthogonals for semisimple}.
      The Killing~form~$\kappa$ of~$\glie$ is by \cref{orthogonal ideals with respect to the killing form} given by the orthogonal sum of the Killing~forms of~$I$ and~$I^\perp$.
      That~$\kappa$ is non-degenerate therefore means that both~$\kappa_{I}$ and~$\kappa_{I^\perp}$ are non-degenerate. 
      By induction hypothesis both~$I$ and~$I^\perp$ can be decomposed into direct sums of simple ideals.
      By combining these two decompositions we get a decomposition of~$\glie$ into simple ideals.
      (Here we use that if~$\glie = I \oplus J$ is a decomposition into ideals then any ideal of~$I$ (resp.~$J$) is again an ideal in~$\glie$.)
    \item[\ref*{sum of simple ideals general}~$\implies$~\ref*{product of simple lie algebras general}:]
      We can take~$\glie_i = I_i$ for every~$i$.
    \item[\ref*{product of simple lie algebras general}~$\implies$~\ref*{contains no solvable ideal general}:]
      For every~$i = 1, \dotsc, n$ let~$\pi_i \colon \glie \to \glie_i$ be the canonical projection.
      If~$I$ is a nonzero ideal in~$\glie$ then~$\pi_i(I) \neq 0$ for some~$i$, and~$\pi_i(I)$ is an ideal in~$\glie$ because~$\pi_i$ is surjective.
      Hence~$\pi_i(I) = \glie_i$ because~$\glie_i$ is simple.
      If~$I$ is solvable then so is~$\pi(I) = \glie_i$, but this contradicts~$\glie_i$ being simple.
    \qedhere
  \end{implicationlist}
\end{proof}


\begin{corollary}
  \label{ideals are again sum of simples}
  Let~$\glie$ be a finite dimensional semisimple Lie~algebra with decomposition into simple ideals~$\glie = I_1 \oplus \dotsb \oplus I_n$.
  Then any ideal of~$\glie$ is of the form~$I_{j_1} \oplus \dotsb \oplus I_{j_m}$ for some indices~$j_1, \dotsc, j_m \in \{1, \dotsc, n\}$.
\end{corollary}


\begin{proof}
  Let~$I$ be any ideal of~$\glie$.
  We proceed as in the proof of \cref{characterizations of fd semisimple lie algebras}:
  
  We have~$\glie = I \oplus I^\perp$ by \cref{decomposition into orthogonals for semisimple} and the Killing form of~$\glie$ is the orthogonal sum of the Killing forms of~$I$ and~$I^\perp$ by \cref{orthogonal ideals with respect to the killing form}.
  Both~$\restrict{\kappa}{I}$ and~$\restrict{\kappa}{I^\perp}$ are therefore non-degenerate.
  We can by \cref{characterizations of fd semisimple lie algebras} decompose both~$I$ and~$I^\perp$ into simple ideals
  \[
    I
    =
    J_1 \oplus \dotsb \oplus J_m \,,
    \qquad
    I^\perp
    =
    J_{m+1} \oplus \dotsb \oplus J_k  \,.
  \]
  Then
  \[
    \glie
    =
    I \oplus I^\perp
    =
    J_1 \oplus \dotsb \oplus J_m \oplus J_{m+1} \oplus \dotsb \oplus J_k
  \]
  is a decomposition of~$\glie$ into simple ideals.
  We know from \cref{uniqueness of semisimple decomposition} that this decomposition coincides with the given decomposition~$\glie = I_1 \oplus \dotsb \oplus I_n$ up to permutation of the direct summands~$I_j$.
  The direct summands~$J_1, \dotsc, J_m$ do therefore occur in the list of simple ideals~$I_1, \dotsc, I_n$.
\end{proof}


\begin{corollary}
  \label{ideals and quotients of semisimple again semisimple}
  If~$\glie$ is a semisimple Lie algebra and~$I$ is any ideal in~$\glie$ then both~$I$ and~$\glie/I$ are again semisimple.
  \qed
\end{corollary}


\begin{proof}
  Let~$\glie = I_1 \oplus \dotsb \oplus I_n$ be a decomposition into simple ideals.
  Then by \cref{ideals are again sum of simples} we may assume that~$I = I_1 \oplus \dotsb \oplus I_m$ for some~$m$.
  We see that~$I$ is semisimple, and it also follows that~$\glie/I \cong I_{m+1} \oplus \dotsb \oplus I_n$ is semisimple.
\end{proof}


\begin{warning}
  Lie~subalgebras of semisimple Lie~algebras need not be semisimple again.
  Indeed, the Lie~algebra~$\sllie_2(\kf)$ is simple and hence semisimple.
  But the Lie~subalgebra~$\blie$ of~$\sllie_2(\kf)$ that consist of upper triangular matrices of trace zero is solvable and can therefore not be semisimple.
\end{warning}



\begin{lemma}
  \label{semisimple lie algebra identification with dual space}
  Let~$\glie$ be a finite dimensional semisimple Lie~algebra.
  Then the map
  \[
    \glie
    \to
    \glie^* \,,
    \quad
    x
    \mapsto
    \kappa(x,-)
  \]
  is an isomorphism of~{\representations{$\glie$}}, where~$\kappa$ denotes the Killing~form of~$\glie$.
\end{lemma}


\begin{proof}
  This follows from \cref{associative non-degenerate bilinear forms induce isomorphism to the dual} because~$\kappa$ is non-degenerate.
\end{proof}


\begin{corollary}
  \label{uniqueness of invariant bilinear forms for simple}
  Let~$\glie$ be a finite dimensional simple Lie~algebra.
  Then any associative bilinear form~$\beta \colon \glie \times \glie \to \kf$ is a scalar multiple of the killing form~$\kappa$ of~$\glie$.
\end{corollary}


\begin{proof}
 The map
 \[
  \varphi
  \colon
  \glie
  \to
  \glie^* \,,
  \quad
  x
  \mapsto
  \beta(x, -)
 \]
 is a homomorphism of~{\representations{$\glie$}} by \cref{associative bilinear form induces homomorphism of representations}.
 The Killing~form~$\kappa$ is non-degenerate because~$\glie$ is simple (and thus semisimple).
 The map
 \[
  \psi
  \colon
  \glie
  \to
  \glie^* \,,
  \quad
  x
  \mapsto
  \kappa(x, -)
 \]
 is therefore an isomorphism of representations by \cref{associative non-degenerate bilinear forms induce isomorphism to the dual}.
 The composition
 \[
  \psi^{-1}
  \circ
  \varphi
  \colon
  \glie
  \to
  \glie 
 \]
 is again a homomorphism of representations, where~$\glie$ is irreducible as a representation because~$\glie$ is simple as a Lie~algebra.
 It follows from Schur’s~lemma that~$\psi^{-1} \circ \varphi = \lambda \id_{\glie}$ for some scalar~$\lambda \in \kf$.
 So~$\varphi = \lambda \psi$ and equivalently~$\beta = \lambda \kappa$.
\end{proof}




