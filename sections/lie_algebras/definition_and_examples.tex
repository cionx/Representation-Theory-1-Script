\section{Definition and Examples}


\subsection{Definition and Basic Properties}


\begin{definition}
	Let~$\glie$ be a vector space over some field~$\kf$.
	A~{\bilinear{$\kf$}} map
	\[
		[\ph, \ph]
		\colon
		\glie \times \glie
		\to
		\glie
		\glsadd{lie bracket}
	\]
	is a \defemph{Lie~bracket}\index{Lie bracket} if it satisfies the following two conditions.
	\begin{enumerate}
		\item
		$[\ph, \ph]$ is alternating, i.e.~$[x,x] = 0$ for every~$x \in \glie$.
		\item
		$[\ph, \ph]$ satisfies the \defemph{Jacobi~identity}\index{Jacobi identity!for elements}%
		\footnote{
			Named after Carl Gustav Jacob Jacobi\index{Jacobi, Carl Gustav Jacob}\index{Carl Gustav Jacob Jacobi} (1804--1851).
		}
		\[
			[x,[y,z]] + [y,[z,x]] + [z,[x,y]]
			=
			0
			\qquad
			\text{for all~$x, y, z \in \glie$.}
		\]
	\end{enumerate}
	A~{\vectorspace{$\kf$}}~$\glie$\glsadd{lie algebra} together with a Lie~bracket~$[\ph, \ph]$ on~$\glie$ is a~\defemph{\liealgebra{$\kf$}}\index{Lie algebra}.%
	\footnote{
		Named after Sophus Lie\index{Lie, Sophus}\index{Sophus Lie} (1842--1899).
	}
\end{definition}


\begin{remark}
	A Lie~bracket~$[\ph, \ph]$ on a vector space~$\glie$ is always antisymmetric\index{antisymmetric}, i.e.
	\[
		[y,x] = -[x,y]
		\qquad
		\text{for all~$x, y \in \glie$,}
	\]
	because
	\[
		0
		=
		[x+y, x+y]
		=
			\underbrace{ [x,x] }_{= 0}
		+ [x,y]
		+ [y,x]
		+ \underbrace{ [y,y] }_{= 0}
		=
		[x,y] + [y,x] \,.
	\]
\end{remark}


\begin{remark}
	The Jacobi~identity can be rewritten as
	\begin{alignat*}{2}
		[x,[y,z]]
		&=
		[[x,y],z] + [y,[x,z]]
		&\qquad
		&\text{for all~$x, y, z \in \glie$}
	\intertext{and equivalently as}
		[[x,y],z]
		&=
		[[x,z],y] + [x,[y,z]]
		&\qquad
		&\text{for all~$x, y, z \in \glie$}
	\end{alignat*}
	by using the antisymmetry of the Lie~bracket.
\end{remark}


\begin{remark}
	One can more generally define the notion of an~\liealgebra{$A$}, where~$A$ is any commutative ring.
	Such a Lie~algebra consists of an~\module{$A$}~$\glie$ together with an~\bilinear{$A$} map~$[\ph, \ph] \colon \glie \times \glie \to \glie$ that is alternating and satisfies the Jacobi~identity.
\end{remark}


\begin{definition}
	\label{definitions of an algebra}
	\leavevmode
	\begin{enumerate}
		\item
			A~\defemph{\enquote{\algebra{$\kf$}}}\index{algebra@\enquote{algebra}} is a~\vectorspace{$\kf$}~$\Aalg$ together with a bilinear map~$\Aalg \times \Aalg \to \Aalg$.
			This map is the \defemph{multiplication} of~$\Aalg$.
		\item
			A~\defemph{\algebra{$\kf$}}\index{algebra} is an~\enquote{\algebra{$\kf$}} that is both associative and unital.
	\end{enumerate}
\end{definition}


\subsection{Some First Examples of Lie~Algebras}


\begin{examples}
	\label{examples for lie algebras}
	\leavevmode
	\begin{enumerate}
		\item
			Any associative \enquote{\algebra{$\kf$}}~$A$ becomes a~{\liealgebra{$\kf$}} via
			\[
				[a,b]
				\defined
				ab - ba
				\qquad
				\text{for all~$a, b \in A$.}
			\]
			Indeed, this map~$[\ph, \ph] \colon A \times A \to A$ is bilinear and alternating, and it follows from the associativity of the multiplication of~$A$ that
			\begin{align*}
				 {}&  [a,[b,c]] + [b,[c,a]] + [c,[a,b]] \\
				={}&  [a, (bc-cb)] + [b, (ca-ac)] + [c, (ab-ba)] \\
				={}&  a(bc-cb)-(bc-cb)a + b(ca-ac) - (ca-ac)b + c(ab-ba) - (ab-ba)c \\
				={}&  \underbracket{abc}_{1}
							- \underbracket{acb}_{2}
							- \underbracket{bca}_{3}
							+ \underbracket{cba}_{4}
							+ \underbracket{bca}_{3}
							- \underbracket{bac}_{5}
							- \underbracket{cab}_{6}
							+ \underbracket{acb}_{2}
							+ \underbracket{cab}_{6}
							- \underbracket{cba}_{4}
							- \underbracket{abc}_{1}
							+ \underbracket{bac}_{5} \\
				={}&  0
			\end{align*}
			for all~$a, b, c \in A$.
			The element~$[a,b]$ of~$A$ is the \defemph{commutator}\index{commutator} of the two elements~$a$ and~$b$.
			This name stems from the fact that~$[a,b] = 0$ if and only if the elements~$a$ and~$b$ commute.

			Let us emphazise some special cases of this general construction.
			\begin{enumerate}
				\item
					The~{\algebra{$\kf$}} of~$(n \times n)$-matrices,~$\Mat(n, \kf)$, becomes a Lie~algebra via
					\[
						[A,B]
						\defined
						AB - BA
						\qquad
						\text{for all~$A, B \in \Mat(n, \kf)$.}
					\]
					This Lie~algebra is the \defemph{general linear Lie~algebra}\index{general linear Lie algebra}, and it is denoted by~$\gllie(n, \kf)$.\glsadd{general linear lie algebra matrix}
				\item
					For any~{\vectorspace{$\kf$}}~$V$ the~{\algebra{$\kf$}}~$\End_{\kf}(V)$ becomes a Lie~algebra via
					\[
						[f, g]
						\defined
						f \circ g - g \circ f
						\qquad
						\text{for all~$f, g \in \End_{\kf}(V)$.}
					\]
					This Lie~algebra is the \defemph{general linear Lie~algebra}\index{general linear Lie algebra} of~$V$, and it is denoted by~$\gllie(V)$.\glsadd{general linear lie algebra endomorphism}
			\end{enumerate}
		\item
			The \defemph{Heisenberg Lie~algebra}%
			\footnote{
				Named after Werner Heisenberg\index{Heisenberg, Werner}\index{Werner Heisenberg} (1901--1976).
			}%
			~$\heisenberglie(n, \kf)$\glsadd{heisenberg lie algebra}\index{Heisenberg Lie algebra} consists of the~\dimensional{$(2n+1)$}~{\vectorspace{$\kf$}} with basis
			\[
				p_1 \,, \dotsc \,, p_n \,,
				\quad
				q_1 \,, \dotsc \,, q_n \,,
				\quad
				c
				\label{heisenberg basis}
			\]
			together with the Lie~bracket~$[\ph, \ph]$ that is given on these basis elements by
			\begin{itemize*}
				\item
					$[p_i, p_j] = 0$ and~$[c_i, c_j] = 0$ for all~$i, j = 1, \dotsc, n$,
				\item
					$[p_i, c], [p_i, c] = 0$ for every~$i = 1, \dotsc, n$,
				\item
					$[p_i, q_j] = \delta_{ij} c$ for all~$i,j = 1, \dotsc, n$.
			\end{itemize*}
			In \cref{examples for representations} we will realize~$\heisenberglie$ as a Lie~subalgebra of~$\gllie(V)$ for a suitable~\vectorspace{$\kf$}~$V$.
			This will in particular show that~$\heisenberglie$ is indeed a Lie~algebra.
	\end{enumerate}
\end{examples}


\begin{remark}[Pre-Lie~algebras]
	Let~$\Aalg$ be an~\enquote{\algebra{$\kf$}}.
	Even if we don’t require the multiplication of~$\Aalg$ to be associative, we can still define a bilinear bracket~$[\ph, \ph]$ on~$\Aalg$ via
	\[
		[a,b]
		\defined
		ab-ba
		\qquad
		\text{for all~$a, b \in \Aalg$.}
	\]
	This bracket~$[\ph, \ph]$ is alternating.
	We have seen in \cref{examples for lie algebras} that this bracket satisfies the Jacobi~identity if the multiplication of~$\Aalg$ is associative, and that this bracket is then a Lie~bracket.
	But the converse does not hold true:
	it may happen that~$[\ph, \ph]$ is a Lie~bracket even though~$\Aalg$ is not associative.
	
	One important example of this are \defemph{pre-Lie~algebras}\index{pre-Lie algebra}:
	We say that~$\Aalg$ is a pre-Lie~algebra if its multiplication satisfies the condition
	\begin{equation}
		\label{abstract equation for pre-lie algebra}
		[l_a, l_b]
		=
		l_{[a,b]}
		\qquad
		\text{for all~$a, b \in \Aalg$,}
	\end{equation}
	where the left hand side is the endomorphism commutator~$[l_a, l_b] = l_a l_b - l_b l_a$, and where we denote for every element~$a$ of~$\Aalg$ by~$l_a$ the left multiplication with~$a$, i.e. the map
	\[
		l_a
		\colon
		A
		\to
		A \,,
		\quad
		x
		\mapsto
		ax \,.
	\]
	Writing out the condition~\eqref{abstract equation for pre-lie algebra} yields the equivalent condition
	\begin{alignat}{2}
		\label{pre lie algebra condition written out}
		a(bc) - b(ac)
		&=
		(ab-ba)c
		&
		\qquad
		&\text{for all~$a, b, c \in \Aalg$,}
	\intertext{or equivalently the condition}
		a(bc) - (ab)c
		&=
		b(ac) - (ba)c
		&
		\qquad
		&\text{for all~$a, b, c \in \Aalg$.}
		\notag
	\end{alignat}
	If~$\Aalg$ is a pre-Lie~algebra, then the resulting commutator bracket~$[\ph, \ph]$ on~$\Aalg$ is still a Lie~bracket, even if~$\Aalg$ itself is not associative.
	Indeed, the bracket~$[\ph, \ph]$ is alternating and it follows from the characterization~\eqref{pre lie algebra condition written out} of a pre-Lie~algebra that it also satisfies the Jacobi~identity, because
	\begin{align*}
		{}&
		[a, [b,c] ] + [b, [c, a]] + [c, [a, b]]
		\\
		={}&
		a (bc - cb) - (bc - cb) a
		+ b (ca - ac) - (ca - ac) b
		+ c (ab - ba) - (ab - ba) c
		\\
		={}&
		a (bc) - a (cb) - (bc - cb) a
		+ b (ca) - b (ac) - (ca - ac) b
		+ c (ab) - c (ba) - (ab - ba) c
		\\
		={}&
		\underbracket{a (bc)}_{1}
		- \underbracket{a (cb)}_{2}
		- \underbracket{b (ca)}_{3}
		- \underbracket{c (ba)}_{4}
		+ \underbracket{b (ca)}_{3}
		- \underbracket{b (ac)}_{5}
		- \underbracket{c (ab)}_{6}
		- \underbracket{a (cb)}_{2}
		+ \underbracket{c (ab)}_{6}
		- \underbracket{c (ba)}_{4}
		- \underbracket{a (bc)}_{1}
		- \underbracket{b (ac)}_{5}
		\\
		={}&
		0
	\end{align*}
	for all~$a, b, c \in \Aalg$.

	Every associative \enquote{algebra} is in particular a pre-Lie~algebra, and thus we see that the Lie~algebra structure on an associative \enquote{algebra} actually comes from its pre-Lie~algebra structure.
	The situation is hence as follows:
	\[
		\text{associative \enquote{algebra}}
		\to
		\text{pre-Lie~algebra}
		\to
		\text{Lie~algebra}
	\]
 Pre-Lie~algebras were first introduced by Gerstenhaber%
 \footnote{
	 Murray Gerstenhaber\index{Gerstenhaber, Murray}\index{Murray Gerstenhaber} (b. 1927).
 }
 and we refer to~\cite{gerstenhaber_cohomology_of_associative_ring} to see this concept in action.
\end{remark}


\subsection{Lie~subalgebras, Lie~ideals and Commutator Spaces}


\begin{definition}
	Let~$\glie$ be a~{\liealgebra{$\kf$}}.
	\begin{enumerate}
		\item
			A \defemph{Lie~subalgebra}\index{Lie subalgebra}\index{subalgebra} of~$\glie$ is a~{\linear{$\kf$}} subspace~$\hlie$ of~$\glie$ such that
			\begin{alignat*}{2}
				[x,y] &\in \hlie
				&\qquad
				&\text{for all~$x, y \in \hlie$.}
		\intertext{
		\item
			A \defemph{Lie~ideal}\index{Lie ideal} of~$\glie$, or simply~\defemph{ideal}\index{ideal} of~$\glie$ is a~{\linear{$\kf$}} subspace~$I$ of~$\glie$ such that
			}
				[x,y] &\in I
				&\qquad
				&\text{for all~$x \in \glie$,~$y \in I$.}
			\end{alignat*}
	\end{enumerate}
\end{definition}


\begin{remark}
	\label{on the notion of ideals}
	Let~$\glie$ be a Lie~algebra.
	\begin{enumerate}
		\item
			It is not necessary to distinguish between left ideals, right ideals and two-sided ideals of~$\glie$ (as one might be used to from ring theory) because the Lie~bracket of~$\glie$ is antisymmetric.
		\item
			Some people use the notation~$I \subideal \glie$ to express that~$I$ is an ideal of~$\glie$.
			We will refrain from doing so, but still want the reader to be aware of this notation.
		\item
			Every ideal of~$\glie$ is in particular a Lie~subalgebra of~$\glie$.
			This is different to the setting of rings%
			\footnote{
				We consider only unital rings, unless otherwise specified.
			}, where a proper ideal is never a subring.
			Lie~algebras do instead behave more like groups, where a normal subgroup is in particular a subgroup.
	\end{enumerate}
\end{remark}


\begin{remark}
	Let~$\glie$ be a Lie~algebra.
	Every Lie~subalgebra~$\hlie$ of~$\glie$ becomes a Lie~algebra in its own right by restricting the Lie~bracket of~$\glie$ to~$\hlie$.
	It follows in particular that every ideal of~$\glie$ is again a Lie~algebra in its own right.
	(This is different to the situation in ring theory, where the only ideal that is also a subring is the ring itself.)
\end{remark}


\begin{definition}
	A~\liealgebra{$\kf$}~$\glie$ is~\defemph{linear}\index{linear Lie algebra} if it is a Lie~subalgebra of~$\gllie(V)$ for some finite-dimensional~\vectorspace{$\kf$}~$V$, or a Lie~subalgebra of~$\gllie(n, \kf)$ for some natural number~$n$.
\end{definition}


\begin{recall}
	For calculations in~$\gllie(n, \kf)$ it is often useful to remember the identity
	\begin{equation}
		\label{product of basis matrices}
		E_{ij} E_{kl}
		=
		\begin{cases*}
			E_{il}  & if~$j = k$, \\
			0       & otherwise,
		\end{cases*}
	\end{equation}
	where the matrices~$E_{ij}$\glsadd{standard basis matrix} with~$i,j = 1, \dotsc, n$ are the standard basis matrices\index{standard basis matrices} of~$\gllie(n, \kf)$.
	One may think about the matrix~$E_{ij}$ as \enquote{going from~$j$ to~$i$}.
	The composition~$E_{ij} E_{kl}$ does then \enquote{go from~$l$ to~$i$} if the positions~$j$ and~$k$ match, and if they don’t fit, then this composition vanishes.
	
	This intuition can be formalized by observing that
	\[
		E_{ij} e_k
		=
		\begin{cases*}
			e_i & if~$j = k$, \\
			0   & otherwise,
		\end{cases*}
	\]
	where~$e_1, \dotsc, e_n$\glsadd{standard basis vector} denotes the standard basis vectors\index{standard basis vectors} of~$\kf^n$.
	The above formula tells us that the matrix~$E_{ij}$ maps one of the standard basis vectors (namely~$e_j$) to another standard basis vector (namely~$e_i$), but filters out all other standard basis vectors.
	Composing two standard basis matrices therefore gives the desired result.

	It follows from the identity~\eqref{product of basis matrices} that in particular
	\[
		[ E_{ij}, E_{kl} ]
		=
		E_{ij} E_{kl} - E_{kl} E_{ij}
		=
		\delta_{jk} E_{il} - \delta_{il} E_{kj}
	\]
	for all~$i, j, k, l = 1, \dotsc, n$.
\end{recall}


\begin{examples}[Linear Lie~algebras]
	\label{examples for linear lie algebras}
	\leavevmode
	\begin{enumerate}
		\item
			The two primordial examples for a linear~\liealgebra{$\kf$} are the general linear Lie~algebras~$\gllie(V)$ for some finite-dimensional~\vectorspace{$\kf$}~$V$ and~$\gllie(n, \kf)$ for some natural number~$n$.
		\item
			The linear subspace
			\[
				\sllie(n, \kf)
				\defined
				\{
					A \in \gllie(n, \kf)
				\suchthat
					\tr(A) = 0
				\}
				\glsadd{special linear lie algebra matrices}
			\]
			of~$\gllie(n, \kf)$ is actually an ideal of~$\gllie(n, \kf)$, as we will see in \cref{examples of commutator ideals}.
			The Lie~algebra~$\sllie(n, \kf)$ is the \defemph{special linear Lie~algebra}\index{special linear Lie algebra}.
		\item
			Let~$V$ be a finite-dimensional vector space.
			The linear subspace
			\[
				\sllie(V)
				\defined
				\{
					f \in \gllie(V)
				\suchthat
					\tr(f) = 0
				\}
				\glsadd{special linear lie algebra endomorphism}
			\]
			is an ideal of~$\gllie(V)$.
			This Lie~algebra is the \defemph{special linear Lie~algebra}\index{special linear Lie algebra} of~$V$.
		\item
			The set of upper triangular matrices\index{upper triangular matrices}, given by
			\[
				\trianglie(n, \kf)
				\defined
				\left\{
					\begin{pmatrix}
							a_{11}
						& \cdots
						& \cdots
						& a_{1n}
						\\
							{}
						& \ddots
						& {}
						& \vdots
						\\
							{}
						& {}
						& \ddots
						& \vdots
						\\
							{}
						& {}
						& {}
						& a_{nn}
					\end{pmatrix}
				\suchthat*
					\text{$a_{ij} \in \kf$ for all~$1 \leq i \leq j \leq n$}
				\right\} \,,
				\glsadd{upper triangular lie algebra}
			\]
			is a Lie~subalgebra of~$\gllie(n, \kf)$.
			This holds because the set~$\trianglie(n, \kf)$ is a~{\subalgebra{$\kf$}} of~$\Mat(n, \kf)$, and hence closed under the commutator bracket~$[A,B] = AB - BA$.
			A basis of~$\trianglie(n, \kf)$ is given by the matrices~$E_{ij}$ with~$i \leq j$.

		\item
			The set of strictly upper triangular matrices\index{strictly upper triangular matrices}
			\[
				\upperlie(n, \kf)
				\defined
				\left\{
					\begin{pmatrix}
							0
						& a_{12}
						& \cdots
						& a_{1n}
						\\
							{}
						& \ddots
						& \ddots
						& \vdots
						\\
							{}
						& {}
						& \ddots
						& a_{n-1,n}
						\\
							{}
						& {}
						& {}
						& 0
					\end{pmatrix}
				\suchthat*
					\text{$a_{ij} \in \kf$ for all~$1 \leq i < j \leq n$}
				\right\}
				\glsadd{strictly upper triangular lie algebra}
			\]
			is an ideal of~$\trianglie(n, \kf)$, as we will see in \cref{examples of commutator ideals}.
			A basis of~$\upperlie(n, \kf)$ is given by the matrices~$E_{ij}$ with~$i < j$.
		\item
			The set of diagonal matrices\index{diagonal matrices}
			\[
				\diaglie(n, \kf)
				=
				\left\{
					\begin{pmatrix}
							d_1
						& {}
						& {}
						\\
							{}
						& \ddots
						& {}
						\\
							{}
						& {}
						& d_n
					\end{pmatrix}
				\suchthat*
					d_1, \dotsc, d_n \in \kf
				\right\}
				\glsadd{diagonal lie algebra}
			\]
			is an~{\dimensional{$n$}} Lie~subalgebra of~$\gllie(n, \kf)$.
			A basis of~$\diaglie(n, \kf)$ is given by the matrices~$E_{ii}$ with~$i = 1, \dotsc, n$.
		\item
			The linear subspace
			\[
				\afflie(n, \kf)
				\defined
				\begin{pmatrix}
					\gllie(n, \kf)  & \kf^n \\
					0               & 0
				\end{pmatrix}
				=
				\left\{
					\begin{pmatrix}
						A & a \\
						0 & 0
					\end{pmatrix}
				\suchthat*
				\begin{array}{@{}c@{}}
						A \in \gllie(n, \kf), \\
						a \in \kf^n
					\end{array}
				\right\}
				\glsadd{affine transformation lie algebra}
			\]
			of~$\gllie(n+1, \kf)$ is a Lie~subalgebra of~$\gllie(n+1, \kf)$.
			Indeed, we have
			\[
				\begin{pmatrix}
					A & a \\
					0 & 0
				\end{pmatrix}
				\begin{pmatrix}
					B & b \\
					0 & 0
				\end{pmatrix}
				=
				\begin{pmatrix}
					A B & A b \\
					0   & 0
				\end{pmatrix}
			\]
			for all~$A, B \in \gllie(n, \kf)$ and~$a, b \in \kf$, and thus
			\[
				\Biggl[
					\begin{pmatrix}
						A & a \\
						0 & 0
					\end{pmatrix}
					,
					\begin{pmatrix}
						B & b \\
						0 & 0
					\end{pmatrix}
				\Biggr]
				=
				\begin{pmatrix}
					[A, B]  & A b - B a \\
					0       & 0
				\end{pmatrix} \,.
			\]
			This commutator is again contained in~$\afflie(n, \kf)$.
			This Lie~algebra~$\afflie(n, \kf)$ is the \defemph{Lie~algebra of affine transformations of~$\kf^n$}\index{affine transformations}.
	\end{enumerate}
\end{examples}


% TODO: Remark about the associated groups.


\begin{definition}
	Let~$\glie$ be a~\liealgebra{$\kf$} and let~$X$ and~$Y$ be two subsets of~$\glie$.
	The linear subspace of~$\glie$ given by
	\[
		[X,Y]
		\defined
		\gen{
			[x,y]
		\suchthat
			x \in X,
			y \in Y
		}_{\kf}
		\glsadd{commutator space}
	\]
	is the \defemph{commutator space}\index{commutator space} of~$X$ and~$Y$.
\end{definition}


\begin{remark}
	Let~$\glie$ be a Lie~algebra.
	\begin{enumerate}
		\item
			For any two subsets~$X$ and~$Y$ of~$\glie$ we find that
			\[
				[X,Y] = -[Y,X] = [Y,X]
			\]
			because the Lie~bracket of~$\glie$ is anti-symmetric.
		\item
			For any two subsets~$X$ and~$Y$ of~$\glie$ we have
			\[
				[X, Y]
				=
				[ \gen{X}_{\kf}, \gen{Y}_{\kf} ] \,.
			\]
			To understand the commutator space~$[X, Y]$ we can therefore often assume that~$X$ and~$Y$ are linear subspace of~$\glie$.
		\item
			For every collection~$X_\lambda$ with~$\lambda \in \Lambda$ of subsets of~$\glie$ and every subset~$Y$ of~$\glie$ we have
			\[
				\left[ \sum_{\lambda \in \Lambda} X_\lambda, Y \right]
				=
				\sum_{\lambda \in \Lambda} [ X_\lambda, Y ]
				\qquad\text{and}\qquad
				\left[ \bigcap_{\lambda \in \Lambda} X_\lambda, Y \right]
				\subseteq
				\bigcap_{\lambda \in \Lambda} [ X_\lambda, Y ] \,.
			\]
		\item
			For any three subsets~$X$,~$Y$, and~$Z$ of~$\glie$ it follows from the Jacobi~identity\index{Jacobi identity!for commutator spaces} that
			\begin{align*}
				[X, [Y, Z]]
				&\subseteq
				[[X, Y], Z] + [Y, [X, Z]]
			\shortintertext{as well as}
				[[X, Y], Z]
				&\subseteq
				[[X, Z], Y] + [X, [Y, Z]] \,.
			\end{align*}
	\end{enumerate}
	We will sometimes use these identities to simplify some calculations.
\end{remark}


\begin{proposition}[Lie~subalgebras and ideals via commutator spaces]
	Let~$\glie$ be a Lie~algebra.
	\begin{enumerate}
		\item
			A linear subspace~$\hlie$ of~$\glie$ is a Lie~subalgebra of~$\glie$ if and only if the commutator subspace~$[\hlie, \hlie]$ is again contained in~$\hlie$.
		\item
			A linear subspace~$I$ of~$\glie$ is an ideal of~$\glie$ if and only if the commutator subspace~$[\glie, I]$ is again contained in~$I$.
		\qed
	\end{enumerate}
\end{proposition}


\begin{proposition}[New Lie~subalgebras from old ones]
	Let~$\glie$ be a Lie~algebra.
	\begin{enumerate}
		\item
			Let~$\hlie_\lambda$ with~$\lambda \in \Lambda$ be a collection of Lie~subalgebras of~$\glie$.
			The intersection~$\bigcap_{\lambda \in \Lambda} \hlie_\lambda$\index{intersection!of Lie subalgebras} is again a Lie~subalgebra of~$\glie$.
		\item
			Let~$\hlie$ be a Lie~subalgebra of~$\glie$ and let~$I$ be an ideal of~$\glie$.
			The sum~$\hlie + I$\index{sum!of Lie subalgebra and ideal} is again an ideal of~$\glie$.
	\end{enumerate}
\end{proposition}


\begin{proof}
	\leavevmode
	\begin{enumerate}
		\item
			We have
			\[
				\left[
					\bigcap_{\lambda \in \Lambda} \glie_\lambda,
					\bigcap_{\lambda \in \Lambda} \glie_\lambda
				\right]
				\subseteq
				\bigcap_{\lambda, \mu \in \Lambda}
				[\glie_\lambda, \glie_\mu]
				\subseteq
				\bigcap_{\substack{\lambda, \mu \in \Lambda \\ \lambda = \mu}}
				[\glie_\lambda, \glie_\mu]
				=
				\bigcap_{\lambda \in \Lambda}
				[\glie_\lambda, \glie_\lambda]
				\subseteq
				\bigcap_{\lambda \in \Lambda}
				\glie_\lambda \,,
			\]
			which proves the assertion.
		\item
			We have
			\[
				[\hlie + I, \hlie + I]
				\subseteq
				[\hlie, \hlie] + [\hlie, I] + [I, \hlie] + [I, I]
				\subseteq
				\hlie + I + I + \hlie
				=
				\hlie + I \,,
			\]
			which proves the assertion.
		\qedhere
	\end{enumerate}
\end{proof}


\begin{warning}
	Let~$\glie$ be a Lie~algebra.
	\begin{enumerate}
		\item
			Given a Lie~subalgebra~$\hlie$ of~$\glie$ and an ideal~$I$ of~$\glie$, their commutator space~$[\hlie, I]$\index{commutator space!of Lie subalgebra and ideal} is not necessarily again a Lie~subalgebra of~$\glie$.

			Let, for example,~$\glie$ be the general linear Lie~algebra~$\gllie(n, \kf)$ for some~$n \geq 2$, let~$\hlie$ be the Lie~subalgebra of~$\glie$ of diagonal matrices~$\diaglie(n, \kf)$, and let~$I$ be~$\glie$ itself.
			Then
			\[
				[\hlie, I]
				=
				\left\{
				  \begin{pmatrix}
						    &         & a_1 \\
						    & \iddots &     \\
						a_n &         &
				  \end{pmatrix}
				\suchthat*
				  a_1, \dotsc, a_n \in \kf
				\right\} \,.
			\]
			(This can be seen for cheap from the upcoming \cref{background on diagonal matrices}.)
			We find in particular that the matrices~$E_{1n}$ and~$E_{n1}$ are contained in~$[\hlie, I]$, but that their commutator~$[E_{1n}, E_{n1}] = E_{11} - E_{nn}$ is not.
		\item
			Given two Lie~subalgebras~$\hlie$ and~$\klie$ of~$\glie$, their sum~$\hlie + \klie$\index{sum!of Lie subalgebras} is not necessarily a Lie~subalgebra of~$\glie$ again.

			Indeed, let us consider thie Lie~algebra~$\glie = \gllie(n, \kf)$ for some~$n \geq 2$.
			Let~$\hlie$ be the Lie~subalgebra of strictly upper triangular matrices -- i.e.~$\upperlie(n, \kf)$, and let~$\klie$ be the Lie~subalgebra of strictly lower triangular matrices -- i.e.~$\upperlie(n, \kf)^\trans$.
			The sum~$\hlie + \klie$ consistst of all those matrices in~$\glie$ whose diagonal entries vanish.
			It is in particular the cases that both~$E_{1n}$ and~$E_{n1}$ are contained in~$\hlie + \klie$ because~$n \geq 2$.
			But their commutator is given by
			\[
				[E_{1n}, E_{n1}]
				=
				E_{1n} E_{n1} - E_{n1} E_{1n}
				=
				E_{11} - E_{nn} \,,
			\]
			which is a nonzero diagonal matrix because~$n \geq 2$.
			We find that~$[E_{1n}, E_{n1}]$ is not again contained in~$\hlie + \klie$, whence~$\hlie + \klie$ is not again a Lie~subalgebra of~$\glie$.
	\end{enumerate}
\end{warning}


\begin{proposition}[New ideals from old ones]
	\label{construction of new ideals}
	Let~$\glie$ be a Lie~algebra.
	\begin{enumerate}
		\item
		Let~$I_\lambda$ with~$\lambda \in \Lambda$ be a collection of ideals of~$\glie$.
		Both the intersection~$\bigcap_{\lambda \in \Lambda} I_\lambda$\index{intersection!of ideals} and the sum~$\sum_{\lambda \in \Lambda} I_\lambda$\index{sum!of ideals} are again ideals of~$\glie$.
		\item
		Let~$I$ and~$J$ be two ideals of~$\glie$.
		Their commutator space~$[I,J]$\index{commutator space!of ideals} is again an ideal of~$\glie$.
	\end{enumerate}
\end{proposition}


\begin{proof}
	\leavevmode
	\begin{enumerate}
		\item
			We have~%
			$
				[ \glie, \sum_{\lambda \in \Lambda} I_\lambda ]
				=
				\sum_{\lambda \in \Lambda} [ \glie, I_\lambda ]
				\subseteq
				\sum_{\lambda \in \Lambda} I_\lambda
			$
			and~%
			$
				[ \glie, \bigcap_{\lambda \in \Lambda} I_\lambda ]
				\subseteq
				\bigcap_{\lambda \in \Lambda} [ \glie, I_\lambda ]
				\subseteq
				\bigcap_{\lambda \in \Lambda} I_\lambda
			$.
		\item
			We have
			$
				[ \glie, [I, J] ]
				\subseteq
				[ [\glie, I], J] + [ I, [\glie, J]]
				\subseteq
				[I, J] + [I, J]
				=
				[I, J]
			$
			by the Jacobi~identity.
	 \qedhere
 \end{enumerate}
\end{proof}


\begin{example}
	Any Lie~algebra~$\glie$ is an ideal of itself.
	It thus follows from \cref{construction of new ideals} that the commutator space
	\[
		[\glie, \glie]
		=
		\gen{ [x,y] \suchthat x, y \in \glie }_{\kf}
	\]
	is again an ideal of~$\glie$.
\end{example}


\begin{definition}
	Let~$\glie$ be a Lie~algebra.
	The ideal~$[ \glie, \glie ]$ of~$\glie$ is the \defemph{commutator ideal}\index{commutator ideal} of~$\glie$, or the \defemph{derived Lie~algebra}\index{derived Lie algebra} of~$\glie$.
\end{definition}


\begin{example}
	\label{examples of commutator ideals}
	\leavevmode
	\begin{enumerate}
		\item
			Let~$\heisenberglie(n, \kf)$ be the Heisenberg Lie~algebra from \cref{examples for lie algebras}.
			The commutator ideal of~$\heisenberglie(n, \kf)$ is one-dimensional and spanned by the basis element~$c$.
		\item
			The commutator ideal of the general linear Lie~algebra~$\gllie(n, \kf)$ is the special linear Lie~algebra~$\sllie(n, \kf)$\index{special linear Lie algebra}.
			Indeed, we have for any two matrices~$A$ and~$B$ in~$\gllie(n, \kf)$ that
			\[
				\tr(AB) = \tr(BA)
			\]
			and therefore
			\[
					\tr( [A,B] )
				= \tr(AB-BA)
				= \tr(AB) - \tr(BA)
				= \tr(AB) - \tr(AB)
				= 0  \,.
			\]
			This shows that the commutator ideal of~$\gllie(n, \kf)$ is contained~$\sllie(n, \kf)$.
			We observe on the other hand that~$\sllie(n, \kf)$ has a basis given by the matrices~$E_{ij}$ for~$i,j = 1, \dotsc, n$ and~$i \neq j$, together with the matrices~$E_{ii} - E_{i+1,i+1}$ for~$i = 2, \dotsc, n$.
			Each of these matrices is a commutator, namely
			\[
				[E_{ij}, E_{jj}]
				=
				E_{ij} E_{jj} - \underbrace{ E_{jj} E_{ij} }_{=0}
				=
				E_{ij}
			\]
			for all~$i, j = 1, \dotsc, n$ with~$i \neq j$, and
			\[
				[E_{i,i+1}, E_{i+1,i}]
				=
				E_{i,i+1} E_{i+1,i} - E_{i+1,i} E_{i,i+1}
				=
				E_{ii} - E_{i+1,i+1}
			\]
			for all~$i = 2, \dotsc, n$.
			This shows that~$\sllie(n, \kf)$ is contained in the commutator ideal of~$\gllie(n, \kf)$.

			We have now shown that~$\sllie(n, \kf)$ is indeed the commutator ideal of~$\gllie(n, \kf)$.
			This entails that~$\sllie(n, \kf)$ is an ideal of~$\gllie(n, \kf)$, as claimed in \cref{examples for linear lie algebras}.
		\item
			Let~$V$ be a finite-dimensional vector space.
			The commutator ideal of~$\gllie(V)$ is~$\sllie(V)$.
			% TODO: What happens if V is infinite-dimensional?
			% TODO: Reference to [sln, sln] = sln under suitable conditions.
		\item
			The commutator ideal of~$\trianglie(n, \kf)$\index{upper triangular matrices} is~$\upperlie(n, \kf)$\index{strictly upper triangular matrices}.
		
			Indeed, for any two upper triangular matrices~$A$ and~$B$ their two products~$AB$ and~$BA$ are again upper triangular, and both products have the same diagonal entries.
			The commutator~$[A,B] = AB - BA$ is therefore a strictly upper triangular matrix.
			This shows that
			\[
				[\trianglie(n, \kf), \trianglie(n, \kf)]
				\subseteq
				\upperlie(n, \kf) \,.
			\]
			We know on the other hand that~$\upperlie(n, \kf)$ has as a basis the matrices~$E_{ij}$ with~$1 \leq i < j \leq n$.
			Each of those matrices can be written as a commutator, namely as
			\[
				[E_{ii}, E_{ij}]
				=
				E_{ii} E_{ij} - \underbrace{ E_{ij} E_{ii} }_{= 0}
				=
				E_{ij} \,,
			\]
			with~$E_{ii}$ contained in~$\trianglie(n, \kf)$ and~$E_{ij}$ contained in~$\upperlie(n, \kf)$.
			This shows that
			\[
				\upperlie(n, \kf)
				\subseteq
				[\trianglie(n, \kf), \upperlie(n, \kf)] \,.
			\]
			It follows that
			\[
				[ \trianglie(n, \kf), \trianglie(n, \kf) ]
				\subseteq
				\upperlie(n, \kf)
				\subseteq
				[ \trianglie(n, \kf), \upperlie(n, \kf) ]
				\subseteq
				[ \trianglie(n, \kf), \trianglie(n, \kf) ] \,,
			\]
			and thus~$[ \trianglie(n, \kf), \trianglie(n, \kf) ] = \upperlie(n, \kf)$ as claimed.

			That~$\upperlie(n, \kf)$ is the commutator ideal of~$\trianglie(n, \kf)$ entails that~$\upperlie(n, \kf)$ is indeed an ideal in~$\trianglie(n, \kf)$, as claimed in \cref{examples for linear lie algebras}.
		\item
			The commutator ideal of~$\afflie(n, \kf)$\index{affine transformations} is given by
			\[
				\begin{pmatrix}
					\sllie(n, \kf)  & \kf^n \\
					0               & 0
				\end{pmatrix}
				=
				\Biggl\{
					\begin{pmatrix}
						A & a \\
						0 & 0
					\end{pmatrix}
				\suchthat[\Bigg]
					\begin{array}{@{}c@{}}
						A \in \sllie(n, \kf), \\
						a \in \kf^n
					\end{array}
				\Biggr\}
			\]
			To see this let us denote this linear subspace of~$\afflie(n, \kf)$ by~$C$.
			The Lie~bracket on~$\afflie(n, \kf)$ is given by
			\[
				\Biggl[
					\begin{pmatrix}
						A & a \\
						0 & 0
					\end{pmatrix}
					,
					\begin{pmatrix}
						B & b \\
						0 & 0
					\end{pmatrix}
				\Biggr]
				=
				\begin{pmatrix}
					[A, B]  & A b - B a \\
					0       & 0
				\end{pmatrix} \,.
			\]
			for all~$A, B \in \gllie(n, \kf)$ and~$a, b \in \kf^n$.
			We see from this that the term~$[A, B]$ is contained in~$\sllie(n, \kf)$.
			The commutator ideal of~$\afflie(n, \kf)$ is therefore contained in the subspace~$C$.

			We observe on the other hand that
			\begin{align*}
				[ \afflie(n, \kf), \afflie(n, \kf) ]
				&=
				\Biggl[
					\begin{pmatrix}
						\gllie(n, \kf)  & \kf^n \\
						0               & 0
					\end{pmatrix}
					,
					\begin{pmatrix}
						\gllie(n, \kf)  & \kf^n \\
						0               & 0
					\end{pmatrix}
				\Biggr]
				\\
				&\supseteq
				\Biggl[
					\begin{pmatrix}
						\gllie(n, \kf)  & 0 \\
						0               & 0
					\end{pmatrix}
					,
					\begin{pmatrix}
						\gllie(n, \kf)  & 0 \\
						0               & 0
					\end{pmatrix}
				\Biggr]
				\\
				&=
				\begin{pmatrix}
					[ \gllie(n, \kf), \gllie(n, \kf) ]  & 0  \\
					0                                   & 0
				\end{pmatrix}
				\\
				&=
				\begin{pmatrix}
					\sllie(n, \kf)  & 0 \\
					0               & 0
				\end{pmatrix} \,,
			\end{align*}
			as well as
			\[
				[ \afflie(n, \kf), \afflie(n, \kf) ]
				\supseteq
				\Biggl[
					\begin{pmatrix}
						\Id & 0 \\
						0   & 0
					\end{pmatrix}
					,
					\begin{pmatrix}
						0 & \kf^n \\
						0 & 0
					\end{pmatrix}
				\Biggr]
				=
				\begin{pmatrix}
					0 & \Id \cdot \kf^n \\
					0 & 0
				\end{pmatrix}
				=
				\begin{pmatrix}
					0 & \kf^n \\
					0 & 0
				\end{pmatrix} \,,
			\]
			and thus together
			\[
				[ \afflie(n, \kf), \afflie(n, \kf) ]
				\supseteq
				\begin{pmatrix}
					\sllie(n, \kf)  & 0 \\
					0               & 0
				\end{pmatrix}
				+
				\begin{pmatrix}
					0 & \kf^n \\
					0 & 0
				\end{pmatrix}
				=
				\begin{pmatrix}
					\sllie(n, \kf)  & \kf^n \\
					0               & 0
				\end{pmatrix}
				=
				C \,.
			\]
			This shows altogether the desired equality~$[ \afflie(n, \kf), \afflie(n, \kf) ] = C$.
	\end{enumerate}
\end{example}


\begin{definition}
	A Lie~algebra~$\glie$ is \defemph{perfect}\index{perfect Lie algebra} if it equals its commutator ideal, i.e. if~$\glie = [\glie, \glie]$.
\end{definition}


\begin{example}
	Let~$V$ be an infinite-dimensional vector space.
	Then general linear Lie~algebra~$\gllie(V)$ is perfect.
	In fact, every element of~$\gllie(V)$ is already a commutator itself.

	To prove this we note that the vector space~$V$ is isomorphic to the direct sum~$V^{\oplus \Natural_{\geq 1}}$.
	We may therefore assume that~$V = W^{\oplus \Natural_{\geq 1}}$ for some vector space~$W$.
	We can thus represent every endomorphism~$f$ of~$V$ as a matrix
	\[
		f
		\equiv
		\begingroup
		\renewcommand{\arraystretch}{1.3}
		\begin{pmatrix}
			f_{11}  & f_{12}  & f_{13}  & f_{14}  & \cdots  \\
			f_{21}  & f_{22}  & f_{23}  & f_{24}  & \cdots  \\
			f_{31}  & f_{32}  & f_{33}  & f_{34}  & \cdots  \\
			f_{41}  & f_{42}  & f_{43}  & f_{44}  & \cdots  \\
			\vdots  & \vdots  & \vdots  & \vdots  & \ddots
		\end{pmatrix} \,.
		\endgroup
	\]
	Each matrix entry~$f_{ij}$ is an endomorphism of~$W$, namely the composite~$\pi_i \circ f \circ \iota_j$, where~$\iota_j$ the inclusion from~$W$ into the~\howmanyth{$j$} direct summand of~$V$, and~$\pi_i$ the projection from~$V$ onto its~\howmanyth{$i$} direct summand.
	This matrix is pointwise column-finite, in the sense that for every element~$w$ of~$W$ and every column index~$j$ the images~$f_{ij}(w)$ vanish for all but finitely many row indices~$i$.
	Suppose conversely that we have given a family of endomorphismss~$g_{ij}$ of~$W$ with~$i, j \geq 1$ such that the resulting matrix is pointwise column-finite.
	Then this matrix describes an endomorphism~$g$ of~$V$.

	Let~$s$ be the shift endomorphism of~$V$ given by
	\[
		s( (w_1, w_2, w_3, \dotsc) )
		=
		( w_2, w_3, w_4, \dotsc )
	\]
	for all~$(w_1, w_2, w_3, \dotsc) \in V$.
	As a matrix, this endomorphism is given by
	\[
		s
		\equiv
		\begin{pmatrix}
			0 & 1 &   &   &         \\
				& 0 & 1 &   &         \\
				&   & 0 & 1 &         \\
				&   &   & 0 & \ddots  \\
				&   &   &   & \ddots
		\end{pmatrix}
	\]
	We therefore have
	\begin{align*}
		[s, f]
		&=
		s f - f s
		\\
		&\equiv
		\begin{pmatrix}
			f_{21}  & f_{22}  & f_{23}  & f_{24}  & \cdots  \\
			f_{31}  & f_{32}  & f_{33}  & f_{34}  & \cdots  \\
			f_{41}  & f_{42}  & f_{43}  & f_{44}  & \cdots  \\
			f_{51}  & f_{52}  & f_{53}  & f_{54}  & \cdots  \\
			\vdots  & \vdots  & \vdots  & \vdots  & \ddots
		\end{pmatrix}
		-
		\begin{pmatrix}
			0       & f_{11}  & f_{12}  & f_{13}  & \cdots  \\
			0       & f_{21}  & f_{22}  & f_{23}  & \cdots  \\
			0	      & f_{31}  & f_{32}  & f_{33}  & \cdots  \\
			0       & f_{41}  & f_{42}  & f_{43}  & \cdots  \\
			\vdots  & \vdots  & \vdots  & \vdots  & \ddots
		\end{pmatrix}
		\\
		&=
		\begingroup
		\renewcommand{\arraystretch}{1.5}
		\arraycolsep=8pt
		\begin{pmatrix}
			f_{21}  & f_{22} - f_{11} & f_{23} - f_{12} & f_{24} - f_{13} & \cdots  \\
			f_{31}  & f_{32} - f_{21} & f_{33} - f_{22} & f_{34} - f_{23} & \cdots  \\
			f_{41}  & f_{42} - f_{31} & f_{43} - f_{32} & f_{44} - f_{33} & \cdots  \\
			f_{51}  & f_{52} - f_{41} & f_{53} - f_{42} & f_{54} - f_{43} & \cdots  \\
			\vdots  & \vdots          & \vdots          & \vdots          & \ddots
		\end{pmatrix} \,.
		\endgroup
	\end{align*}
	Let now~$g$ be an endomorphism of~$V$.
	We can use the above explicit calculation of the commutator~$[s, f]$ to construct an endomorphism~$f$ with~$[s, f] = g$.

	Indeed:
	Starting with the first column we choose the entry~$f_{11}$ arbitrary and~$f_{i1}$ as~$g_{i-1,1}$ for every~$i \geq 2$.
	Once the entries~$f_{ij}$ are constucted for all~$i \geq 1$ and~$j = 1, \dotsc, k$, we choose the entries~$f_{1,k+1}$ arbitrary and the entry~$f_{i,k+1}$ as~$g_{i-1, j+1} + f_{i-1, j}$ for every~$i \geq 2$.
	The matrix described by these entries~$f_{ij}$ is pointwise column-finite because the matrix associated to~$g$ is pointwise column-finite.
	It follows that the entries~$f_{ij}$ describe an endomorphism~$f$ of~$V$.
	By construction, this endomorphism satisfies~$[s, f] = g$.

	We have thus shown that every endomorphism of~$V$ is a commutator (namely a commutator of the form~$[s, f]$ for some endomorphism~$f$ of~$V$).
\end{example}


\begin{question}
	In the above discussion, can we construct the inverse of~$[s, \ph]$ is a nicer way?
\end{question}




\subsection{Centralizers and the Center of a Lie~algebra}


\begin{definition}
	Let~$\glie$ be a Lie~algebra.
	\begin{enumerate}
		\item
			For every subset~$X$ of~$\glie$ its \defemph{centralizer}\index{centralizer} is the set
			\[
				\centerlie_{\glie}(X)
				\defined
				\{
					z \in \glie
				\suchthat
					\text{$[z,x] = 0$ for every~$x \in X$}
				\} \,.
				\glsadd{centralizer of subset}
			\]
		\item
			For any element~$x$ of~$\glie$ its \defemph{centralizer}\index{centralizer} is the set
			\[
				\centerlie_{\glie}(x)
				\defined
				\centerlie_{\glie}(\{x\})
				=
				\{
					z \in \glie
				\suchthat{} % don’t want \suchthat to interpret the following bracket as an optional argument
					[z,x] = 0
				\} \,.
				\glsadd{centralizer of element}
			\]
		\item
			The \defemph{center}\index{center} of~$\glie$ is the set
			\[
				\centerlie(\glie)
				\defined
				\centerlie_{\glie}(\glie)
				=
				\{
					z \in \glie
				\suchthat
					\text{$[z,x] = 0$ for every~$x \in \glie$}
				\} \,.
				\glsadd{center}
			\]
	\end{enumerate}
\end{definition}


\begin{proposition}
	\label{structure of centralizers}
	Let~$\glie$ be a Lie~algebra.
	\begin{enumerate}
		\item
			For every subset~$X$ of~$\glie$ its centralizer~$\centerlie_{\glie}(X)$ is a Lie~subalgebra of~$\glie$.
		\item
			\label{centralizer of an ideal is again an ideal}
			For every ideal~$I$ of~$\glie$ its centralizer~$\centerlie_{\glie}(I)$ is again an ideal of~$\glie$.
	\end{enumerate}
\end{proposition}


\begin{proof}
	\leavevmode
	\begin{enumerate}
		\item
			We have
			\begin{align*}
				\SwapAboveDisplaySkip
				[ [ \centerlie_{\glie}(X), \centerlie_{\glie}(X) ], X ]
				&\subseteq
				[ [ \centerlie_{\glie}(X), X ], \centerlie_{\glie}(X) ]
				+ [ \centerlie_{\glie}(X), [ \centerlie_{\glie}(X), X ] ]
				\\
				&=
				[ 0, \centerlie_{\glie}(X) ]
				+ [ \centerlie_{\glie}(X), 0 ]
				\\
				&=
				0 \,,
			\end{align*}
			which shows that the commutator subspace~$[ \centerlie_{\glie}(X), \centerlie_{\glie}(X) ]$ is again contained in~$\centerlie_{\glie}(X)$.
		\item
			We have
			\begin{align*}
				\SwapAboveDisplaySkip
				[ [\glie, \centerlie_{\glie}(I)], I ]
				&\subseteq
				[ [\glie, I], \centerlie_{\glie}(I) ]
				+ [ \glie, [ \centerlie_{\glie}(I), I] ]
				\\
				&\subseteq
				[ I, \centerlie_{\glie}(I) ]
				+ [ \glie, 0 ]
				\\
				&=
				0 + 0
				\\
				&=
				0 \,,
			\end{align*}
			which shows that the commutator space~$[\glie, \centerlie_{\glie}(I)]$ is again contained in~$\centerlie_{\glie}(I)$.
		\qedhere
	\end{enumerate}
\end{proof}
% TODO: Give alternate proofs one representations have been introduced.


\begin{corollary}
	Let~$\glie$ be a Lie~algebra.
	Its center~$\centerlie(\glie)$ is an ideal in~$\glie$.
\end{corollary}


\begin{proof}
	We apply part~\ref{centralizer of an ideal is again an ideal} of~\cref{structure of centralizers} to the special case~$I = \glie$.
\end{proof}


\begin{fluff}
	We will in the following compute some centralizers and centers.
	For linear Lie~algebras this can sometimes be done nicely with the help of diagonal matrices.
\end{fluff}


\begin{recall}
	\label{background on diagonal matrices}
	Let~$D$ be a diagonal matrix in~$\gllie(n, \kf)$ with diagonal entries~$\lambda_1, \dotsc, \lambda_n$.
	For any other matrix~$A$ in~$\gllie(n, \kf)$ the product~$DA$ results from~$A$ by multiplying for every~$i = 1, \dotsc, n$ the~\howmanyth{$i$} row of~$A$ by the scalar~$\lambda_i$.
	Similarly, the product~$AD$ results from~$A$ by multiplying for every~$j = 1, \dotsc, n$ the~\howmanyth{$j$} column of~$A$ by the scalar~$\lambda_j$.
	This means in formulae that
	\[
		(DA)_{ij}
		=
		\lambda_i A_{ij}
		\quad\text{and}\quad
		(AD)_{ij}
		=
		\lambda_j A_{ij}
	\]
	for all~$i, j = 1, \dotsc, n$.
	It follows that
	\[
		[D, A]_{ij}
		=
		(DA - AD)_{ij}
		=
		(DA)_{ij} - (AD)_{ij}
		=
		\lambda_i A_{ij} - \lambda_j A_{ij}
		=
		(\lambda_i - \lambda_j) A_{ij}
	\]
	for all~$i, j = 1, \dotsc, n$.
	We find in particular the following:
	\begin{enumerate}
		\item
			The matrices~$E_{ij}$ are eigenvector of~$\ad_{\gllie(n, \kf)}(D)$ with corresponding eigenvalues~$\lambda_i - \lambda_j$ for all~$i, j = 1, \dotsc, n$.
		\item
			The matrix~$A$ commutes with the diagonal matrix~$D$ if and only if its entry~$A_{ij}$ vanishes for all those indices~$i, j = 1, \dotsc, n$ with~$\lambda_i \neq \lambda_j$.

			Suppose that the diagonal entries of~$D$ are arranged in such a way that
			\[
				D
				=
				\diag
				(
					\underbrace{ \mu_1, \dotsc, \mu_1 }_{n_1},
					\dotsc,
					\underbrace{\mu_k, \dotsc, \mu_k}_{n_k}
				)
			\]
			where the diagonal matrices~$\mu_1, \dotsc, \mu_k$ are pairwise distinct.
			Then we can express the above condition on the commutativity with~$D$ as
			\begin{align*}
				\centerlie_{\gllie(n, \kf)}(D)
				&=
				\left\{
					\begin{pmatrix}
						A_1 &         &     \\
								& \ddots  &     \\
								&         & A_k
					\end{pmatrix}
				\suchthat*
					\text{$A_i \in \gllie(n_i, \kf)$ for all~$i = 1, \dotsc, k$}
				\right\}
				\\
				&=
				\begin{pmatrix}
					\gllie(n_1, \kf) &        &                   \\
					                 & \ddots &                   \\
					                 &        & \gllie(n_k, \kf)
				\end{pmatrix} \,.
			\end{align*}
			We want to emphasize a special case of this general calculation:
			If the diagonal entries of~$D$ are pairwise different, then it follows that every matrix that commutes with~$D$ is already diagonal itself.
	\end{enumerate}
\end{recall}


\begin{lemma}
	\label{center consists of diagonal matrices}
	Let~$\glie$ be a Lie~subalgebra of~$\gllie(n, \kf)$.
	\begin{enumerate}
		\item
			Suppose that the Lie~algebra~$\glie$ contains for any two indices~$i$,~$j$ with~$1 \leq i < j \leq n$ a diagonal matrix~$D$ with~$D_{ii} \neq D_{jj}$.
			Then the center of~$\glie$ consists of diagonal matrices.
		\item
			Suppose furthermore that the Lie~algebra~$\glie$ contains for any two indices~$i$,~$j$ with~$1 \leq i < j \leq n$ at least one of the two matrices~$E_{ij}$ or~$E_{ji}$.
			Then the center of~$\glie$ consists of scalar multiples of the identity matrix.
	\end{enumerate}
\end{lemma}


\begin{proof}
	\leavevmode
	\begin{enumerate}
		\item
			Let~$A$ be a matrix contained in the center of~$\glie$.
			There exists by assumption for any two indices~$i$,~$j$ with~$1 \leq i < j \leq n$ a diagonal matrix~$D$ in~$\glie$ with~$D_{ii} \neq D_{jj}$.
			The matrices~$A$ and~$D$ commute, whence it follows from \cref{background on diagonal matrices} that~$A_{ij} = 0$.
			This shows that~$A$ is a diagonal matrix.
		\item
			Suppose that the matrix~$E_{ij}$ is contained in~$\glie$ for some two indices~$i$,~$j$ with~$1 \leq i < j \leq n$.
			It then further follows from \cref{background on diagonal matrices} that
			\[
				0
				=
				[A, E_{ij}]
				=
				(A_{ii} - A_{jj}) E_{ij} \,,
			\]
			and thus~$A_{ii} = A_{jj}$.
			If instead~$E_{ji}$ is contained in~$\glie$, then we also find that~$A_{ii} = A_{jj}$.
			It follows overall that all diagonal entries of~$A$ are equal.
			The diagonal matrix~$A$ is hence a scalar multiple of the identity matrix.
		\qedhere
	\end{enumerate}
\end{proof}


\begin{example}
	\label{examples of centers}
	\leavevmode
	\begin{enumerate}
		\item
			It follows from \cref{center consists of diagonal matrices} that the center of~$\gllie(n, \kf)$ is spanned by the identity matrix for all~$n \geq 2$.
			This also holds for~$n = 1$ (and~$n = 0$).
		\item
			It also follows from \cref{center consists of diagonal matrices} that the center of~$\trianglie(n, \kf)$ is spanned by the identity matrix for all~$n \geq 1$.
			This assertion also holds for~$n = 1$ (and~$n = 0$).
		\item
			We will now determine the center of~$\upperlie(n, \kf)$ for all~$n \geq 0$.
			For~$n = 0$ and~$n = 1$ we have~$\upperlie(n, \kf) = 0$ and thus~$\centerlie( \upperlie(n, \kf) ) = 0$.

			We will now show that for~$n \geq 2$ the center of~$\upperlie(n, \kf)$ is spanned by the single matrix~$E_{1n}$.
			Indeed, the Lie~algebra~$\upperlie(n, \kf)$ has the matrices~$E_{ij}$ with~$1 \leq i < j \leq n$ as a basis.
			A matrix~$A$ in~$\upperlie(n, \kf)$ is therefore contained in the center of~$\upperlie(n, \kf)$ if and only if it satisfies
			\[
				[A, E_{ij}] = 0
				\qquad
				\text{for all~$1 \leq i < j \leq n$.}
			\]

			For the matrix~$E_{1n}$ we have~$E_{ij} E_{1n} = 0$ because~$j > 1$ and thus$~j \neq 1$, as well as~$E_{1n} E_{ij} = 0$ because~$i < n$ and thus~$i \neq n$.
			We therefore have
			\[
				[E_{1n}, E_{ij}]
				=
				E_{1n} E_{ij} - E_{ij} E_{1n}
				=
				0 - 0
				=
				0
			\]
			for all~$1 \leq i < j \leq n$.
			This shows that~$E_{1n}$ is indeed central in~$\upperlie(n, \kf)$.

			Suppose now on the other hand that the matrix~$A$ is central in~$\upperlie(n, \kf)$.
			We may write the matrix~$A$ as a linear combination
			\[
				A
				=
				\sum_{1 \leq i < j \leq n}
				a_{ij} E_{ij} \,.
			\]
			It follows that
			\begin{align*}
				\SwapAboveDisplaySkip
				0
				&=
				[A, E_{1k}]
				\\
				&=
				\sum_{1 \leq i < j \leq n} 
				a_{ij}
				[ E_{ij}, E_{1k} ]
				\\
				&=
				\sum_{1 \leq i < j \leq n} 
				a_{ij}
				( E_{ij} E_{1k} - E_{1k} E_{ij} )
				\\
				&=
				\sum_{1 \leq i < j \leq n} 
				a_{ij}
				( \underbrace{\delta_{j1}}_{=0} E_{ik} - \delta_{ki} E_{1j} )
				\\
				&=
				\sum_{1 \leq i < j \leq n}
				(-a_{ij}) \delta_{ki} E_{1j}
				\\
				&=
				\sum_{j = k+1}^n
				(-a_{kj}) E_{1j}
			\end{align*}
			for all~$k = 2, \dotsc, n$.
			This tells us that all coefficients~$a_{ij}$ with~$j = 2, \dotsc, n$ vanish.
			We find similarly that
			\begin{align*}
				0
				&=
				[A, E_{kn}]
				\\
				&=
				\sum_{1 \leq i < j \leq n}
				a_{ij}
				[E_{ij}, E_{kn}]
				\\
				&=
				\sum_{1 \leq i < j \leq n}
				a_{ij}
				( E_{ij} E_{kn} - E_{kn} E_{ij} )
				\\
				&=
				\sum_{1 \leq i < j \leq n}
				a_{ij}
				( \delta_{jk} E_{in} - \underbrace{\delta_{ni}}_{=0} E_{kj} )
				\\
				&=
				\sum_{1 \leq i < j \leq n}
				a_{ij} \delta_{jk} E_{in}
				\\
				&=
				\sum_{i=1}^{k-1}
				a_{ik} E_{in}
			\end{align*}
			for all~$k = 1, \dotsc, n-1$.
			This tells us that all coefficients~$a_{ij}$ with~$j = 1, \dotsc, n-1$ vanish.
			It shows altogether that~$A = a_{1n} E_{1n}$.
		\item
			We can also compute the center of~$\afflie(n, \kf)$.
			An element
			\[
				\begin{pmatrix}
					A & a \\
					0 & 0
				\end{pmatrix}
			\]
			with~$A \in \gllie(n, \kf)$ and~$a \in \kf^n$ is central in~$\afflie(n, \kf)$ if and only if
			\[
				0
				=
				\Biggl[
					\begin{pmatrix}
						A & a \\
						0 & 0
					\end{pmatrix}
					,
					\begin{pmatrix}
						B & b \\
						0 & 0
					\end{pmatrix}
				\Biggr]
				=
				\begin{pmatrix}
					[A, B]  & A b - B a \\
					0       & 0
				\end{pmatrix}
			\]
			for all~$B \in \gllie(n, \kf)$ and~$b \in \kf^n$.
			It sufficies to consider the special cases~$B = 0$ and~$b = 0$ by the bilinearity of the Lie~bracket of~$\afflie(n, \kf)$. 
			In the special case~$B = 0$ we arrive at the condition
			\[
				A b = 0
				\qquad
				\text{for all~$b \in \kf^n$,}
			\]
			and in the special case~$b = 0$ we arrive at the two additional conditions
			\[
				[A, B] = 0 \,,
				\quad
				B a = 0
				\qquad
				\text{for all~$B \in \gllie(n, \kf)$}.
			\]
			The first condition tells us that~$A = 0$, and the third condition tells us that~$a = 0$.
			We have thus shown that the center of~$\afflie(n, \kf)$ is trivial.
	\end{enumerate}
\end{example}


\begin{question}
	Is there a nice trick to compute the center of~$\upperlie(n, \kf)$?
\end{question}


\begin{definition}
	A Lie~algebra~$\glie$ is \defemph{abelian}\index{abelian} if any two elements of~$\glie$ commute, i.e. if~$[x,y] = 0$ for all~$x, y \in \glie$.
\end{definition}

\begin{proposition}

	Let~$\glie$ be a Lie~algebra.
	The following three conditions on~$\glie$ are equivalent.
	\begin{equivalenceslist}
		\item
			$\glie$ is abelian.
		\item
			$\centerlie(\glie) = \glie$.
		\item
			$[\glie, \glie] = 0$.
		\qed
	\end{equivalenceslist}
\end{proposition}


\begin{examples}
	\leavevmode
	\begin{enumerate}
		\item
			A~{\algebra{$\kf$}}~$A$ is commutative if and only if it is abelian as a Lie~algebra.
		\item
			In any Lie~algebra~$\glie$ every element~$x$ of~$\glie$ spans a {\onedimensional} abelian Lie~subalgebra of~$\glie$, given by~$\gen{x}_{\kf} = \{ \lambda x \suchthat \lambda \in \kf \}$.
		\item
			Every vector space~$\glie$ can be made into an abelian Lie~algebra is precisely one way, namely by setting~$[x, y] \defined 0$ for all~$x, y \in \glie$.
			In this sense, an abelian Lie~algebras is \enquote{the same} as a vector space.
	\end{enumerate}
\end{examples}


\begin{lemma}
	\label{center has codimension at least two for non-abelian}
	Let~$\glie$ be a non-abelian Lie~algebra.
	The codimension of~$\centerlie(\glie)$ in~$\glie$ it at least~$2$.
\end{lemma}


\begin{proof}
	Suppose otherwise that the codimension of~$\centerlie(\glie)$ in~$\glie$ is at most~$1$.
	We know that~$\centerlie(\glie)$ is a proper ideal of~$\glie$ becaus~$\glie$ is non-abelian.
	We thus find that the codimension of~$\centerlie(\glie)$ in~$\glie$ is precisely~$1$.

	Let~$x$ be some non-central element of~$\glie$.
	It follows from the above observation that
	\[
		\glie
		=
		\centerlie(\glie) \oplus \gen{x}_{\kf}
	\]
	as vector spaces.
	It follows that
	\begin{align*}
		[\glie, \glie]
		&=
		[ \centerlie(\glie) + \gen{x}_{\kf}, \centerlie(\glie) + \gen{x}_{\kf} ]
		\\
		&=
		[ \centerlie(\glie), \centerlie(\glie) ]
		+ [ \centerlie(\glie), \gen{x}_{\kf} ]
		+ [ \gen{x}_{\kf}, \centerlie(\glie) ]
		+ [ \gen{x}_{\kf}, \gen{x}_{\kf} ]
		\\
		&=
		0 + 0 + 0 + 0
		\\
		&=
		0 \,.
	\end{align*}
	But this shown that~$\glie$ is abelian -- a contradiction!
\end{proof}


\begin{remark}
	One might compare \cref{center has codimension at least two for non-abelian} and its proof to a proposition from group theory and its proof:
	If a group~$G$ is nonabelian, then the quotient~$G / {\centergroup(G)}$ is not cyclic.
\end{remark}


\begin{warning}
	Let~$\glie$ be a Lie~algebra.
	The reader who is already familiar with quotient Lie~algebras -- which we will introduce in \cref{definition of quotient representation} -- may wonder about a possible generalization of \cref{center has codimension at least two for non-abelian}:
	if the quotient~$\glie / {\centerlie(\glie)}$ is abelian, then does it follow that~$\glie$ is abelian?

	The answer to this question is no.
	Indeed, one might consider for~$\glie$ the Lie~algebra~$\upperlie(3, \kf)$.
	We have seen in \cref{examples of centers} that the center of~$\glie$ is one-dimensional and spanned by the matrix~$E_{13}$.
	The quotient~$\glie / {\centerlie(\glie)}$ is thus spanned by the two residue classes~$\class{ E_{12} }$ and~$\class{ E_{23} }$.
	Their commutator is
	\[
		\Bigl[ \class{ E_{12} }, \class{ E_{23} } \Bigr]
		=
		\class{ [ E_{12}, E_{23} ] }
		=
		\class{ E_{13} }
		=
		0 \,,
	\]
	which shows that~$\glie / {\centerlie(\glie)}$ is abelian.
	But~$\glie$ is not abelian because the matrices~$E_{12}$ and~$E_{23}$ do not commute in~$\glie$.
\end{warning}


\begin{fluff}
	The Lie~algebra~$\sllie(2,\kf)$ will play a crucial role in the second part of these notes.
	We have already observed in the above discussion that~$\sllie(2,\kf)$ has a basis given by the three matrices
	\[
		e
		\defined
		\begin{pmatrix}
			0 & 1 \\
			0 & 0
		\end{pmatrix} \,,
		\qquad
		h
		\defined
		\begin{pmatrix*}[r]
			1 &  0  \\
			0 & -1
		\end{pmatrix*}  \,,
		\qquad
		f
		\defined
		\begin{pmatrix}
			0 & 0 \\
			1 & 0
		\end{pmatrix} \,.
	\]
	The Lie~bracket~$[\ph, \ph]$ of~$\sllie(2, \kf)$ is on these basis elements given by
	\[
		[h, e]
		=
		2e  \,,
		\qquad
		[h, f]
		=
		-2 f \,,
		\qquad
		[e,f]
		=
		h \,.
	\]
	We will see these relations quite a lot later on.

	The above basis of~$\sllie(2, \kf)$ is so important that we give it a special name.
\end{fluff}


\begin{definition}
 Let~$\kf$ be any field.
 The basis
 \[
		e
		=
		\begin{pmatrix}
			0 & 1 \\
			0 & 0
		\end{pmatrix} \,,
		\qquad
		h
		=
		\begin{pmatrix*}[r]
			1 &  0  \\
			0 & -1
		\end{pmatrix*}  \,,
		\qquad
		f
		=
		\begin{pmatrix}
			0 & 0 \\
			1 & 0
		\end{pmatrix} \,.
		\glsadd{standard basis of sl2}
	\]
	of the Lie~algebra~$\sllie(2, \kf)$ is its \defemph{standard basis}\index{standard basis}.
\end{definition}


\begin{remark}
	Some authors prefer the letters~$x$,~$h$,~$y$ over~$e$,~$h$,~$f$.
\end{remark}


\begin{example}
	We now determine the center of~$\sllie(n, \kf)$ for any field~$\kf$ and every natural number~$n$.

	In the cases~$n = 0$ and~$n = 1$ we have~$\sllie(n, \kf) = 0$ and therefore also~$\centerlie( \sllie(n, \kf) ) = 0$.
	We consider from now on the case~$n \geq 2$.

	Suppose first that the characteristic of~$\kf$ is distinct from~$2$.
	The Lie~algebra~$\sllie(n, \kf)$ does contain for any two indices~$i$,~$j$ with~$1 \leq i < j \leq n$ the diagonal matrix~$D$ with diagonal entries
	\[
		D_{kk}
		\defined
		\begin{cases*}
			\phantom{-}1  & if~$k = i$, \\
								-1  & if~$k = j$, \\
			\phantom{-}0  & otherwise.
		\end{cases*}
	\]
	The diagonal entries~$1$ and~$-1$ are distinct by assumption, whence it follows from \cref{center consists of diagonal matrices} that the center of~$\sllie(n, \kf)$ consists of scalar multiples of the identity matrix.

	Suppose now that the characteristic of~$\kf$ is~$2$ and that~$n \geq 3$.
	For any two indices~$i$,~$j$ with~$1 \leq i < j \leq n$ there exists then a third index~$k$ with~$k \neq i, j$.
	It follows that~$\sllie(2, \kf)$ contains the diagonal matrix~$D$ given by
	\[
		D_{ll}
		=
		\begin{cases*}
			1 & if~$l = i$ or~$l = k$, \\
			0 & if~$l = j$,  \\
			0 & otherwise.
		\end{cases*}
	\]
	The matrix~$D$ is built so that its two diagonal entries~$D_{ii}$ and~$D_{jj}$ are distinct.
	It follows from \cref{center consists of diagonal matrices} that the center of~$\sllie(n, \kf)$ consists of scalar multiples of the identity matrix.

	Supose now that the characteristic of~$\kf$ is~$2$ and that~$n = 2$.
	For the standard basis~$e$,~$h$,~$f$ of~$\sllie(2, \kf)$ we have
	\[
		[h, e] = 0 \,,
		\quad
		[h, f] = 0 \,,
		\quad
		[e, f] = h \,.
	\]
	We find that~$\sllie(2, \kf)$ is non-abelian and that~$h$ is central in~$\sllie(2, \kf)$.
	It follows from \cref{center has codimension at least two for non-abelian} that the center of~$\sllie(2, \kf)$ is spanned by~$h$, which is the identity matrix (because~$1 = -1$).

	We have now seen that for~$n \geq 2$ the center of~$\sllie(n, \kf)$ consists of scalar multiples of the identity matrix.
	But the identity matrix is contained in~$\sllie(n, \kf)$ if and only if the characteristic of~$\kf$ divides~$n$.
	We have thus shown that
	\[
		\centerlie( \sllie(n, \kf) )
		=
		\begin{cases*}
			\gen{ \Id }_{\kf} & if~$\ringchar(\kf)$ divides~$n$, \\
			0                 & otherwise,
		\end{cases*}
	\]
	for every~$n \geq 2$.
	This description also holds for the special~$n = 0$ and~$n = 1$.
	However, the span~$\gen{ \Id }_{\kf}$ is only one-dimensional for~$n \geq 1$.
\end{example}


\subsection{Normalizers}


\begin{definition}
	Let~$\glie$ be a Lie~algebra and let~$U$ be a linear subspace of~$\glie$.
	The set
	\[
		\normallie_{\glie}(U)
		\defined
		\{
			x \in \glie
		\suchthat
			\text{$[x,y] \in U$ for every~$y \in U$}
		\}
		\glsadd{normalizer}
	\]
	is the \defemph{normalizer}\index{normalizer} of~$U$ in~$\glie$.
\end{definition}


\begin{proposition}
	Let~$\glie$ be a Lie~algebra and let~$U$ be a linear subspace of~$\glie$.
	\begin{enumerate}
		\item
			The normalizer~$\normallie_{\glie}(U)$ is a Lie~subalgebras of~$\glie$.
		\item
			The linear subspace~$U$ is a Lie~subalgebra of~$\glie$ if and only if~$U$ is contained in~$\normallie_{\glie}(U)$.
		\item
			The linear subspace~$U$ is an ideal of~$\glie$ if and only if its normalizer~$\normallie_{\glie}(U)$ is all of~$\glie$.
		\item
			The centralizer~$\centerlie_{\glie}(U)$ is an ideal of the normalizer~$\normallie_{\glie}(U)$.
	\end{enumerate}
\end{proposition}


\begin{proof}
	\leavevmode
	\begin{enumerate}
		\item
			We have
			\begin{align*}
				\SwapAboveDisplaySkip
				[ [ \normallie_{\glie}(U), \normallie_{\glie}(U) ], U ]
				&\subseteq
				[ [ \normallie_{\glie}(U), U ], \normallie_{\glie}(U) ]
				+ [ \normallie_{\glie}(U), [ \normallie_{\glie}(U), U ] ]
				\\
				&\subseteq
				[ U, \normallie_{\glie}(U) ]
				+ [ \normallie_{\glie}(U), U ]
				\\
				&\subseteq
				U + U
				\\
				&=
				U \,,
			\end{align*}
			which shows that~$[ \normallie_{\glie}(U), \normallie_{\glie}(U) ]$ is again contained in~$\normallie_{\glie}(U)$.
		\item
			The linear subspace~$U$ is a Lie~subalgebra of~$\glie$ if and only if the commutator space~$[ U, U ]$ is again contained in~$U$, which is the case if and only if~$U$ is contained in~$\normallie_{\glie}(U)$.
		\item
			The linear subspace~$U$ is an ideal of~$\glie$ if and only if the commutator space~$[\glie, U]$ is again contained in~$U$, which means precisely that~$\glie$ is contained in the normalizer~$\normallie_{\glie}(U)$.
			But~$\normallie_{\glie}(U)$ is aways contained in~$\glie$, whence this inclusion is then already an equality.
		\item
			It follows from the inclusion
			\[
				[\centerlie_{\glie}(U), U]
				=
				0
				\subseteq
				U
			\]
			that the centralizer~$\centerlie_{\glie}(U)$ is contained in the normalizer~$\normallie_{\glie}(U)$.
			We have
			\begin{align*}
				[ [ \normallie_{\glie}(U), \centerlie_{\glie}(U) ], U ]
				&\subseteq
				[ [\normallie_{\glie}(U), U], \centerlie_{\glie}(U) ]
				+ [ \normallie_{\glie}(U), [\centerlie_{\glie}(U),  U] ]
				\\
				&\subseteq
				[ U, \centerlie_{\glie}(U) ]
				+ [ \normallie_{\glie}(U), 0 ]
				\\
				&=
				0 + 0
				\\
				&=
				0
			\end{align*}
			by the Jacobi~identity.
			This shows that the commutator space~$[ \normallie_{\glie}(U), \centerlie_{\glie}(U) ]$ is again contained in the centralizer~$\centerlie_{\glie}(U)$, which shows that~$\centerlie_{\glie}(U)$ is indeed an ideal of~$\normallie_{\glie}(U)$.
		\qedhere
	\end{enumerate}
\end{proof}





