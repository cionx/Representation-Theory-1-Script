\section{New Lie~Algebras From Old Ones}


\subsection{Products of Lie~Algebras}


\begin{proposition}
	\label{construction of product lie algebra}
	Let~$\glie_{\lambda}$ with~$\lambda$ in~$\Lambda$ be a family of Lie~algebras.
	Then the bracket
	\[
		\bigl[ (x_\lambda)_\lambda, (y_\lambda)_\lambda \bigr]
		\defined
		( [x_\lambda, y_\lambda] )_\lambda
		\qquad
		\text{for all~$(x_\lambda)_\lambda, (y_\lambda)_\lambda \in \prod_{\lambda \in \Lambda} \glie_\lambda$}
	\]
	is a Lie~bracket on the vector space~$\prod_{\lambda \in \Lambda} \glie_\lambda$.
	\qed
\end{proposition}


\begin{definition}
	In the situation of \cref{construction of product lie algebra} the resulting Lie~algebra~$\prod_{\lambda \in \Lambda} \glie_\lambda$\glsadd{product of lie algebras} is the \defemph{product}\index{product of Lie algebras} of the Lie~algebras~$\glie_\lambda$.
\end{definition}


\subsection{External Direct Sums of Lie~Algebras}


\begin{proposition}
	\label{construction of direct sum of lie algebras}
	Let~$\glie_\lambda$ with~$\lambda$ in~$\Lambda$ be a family of Lie~algebras.
	Then the direct sum~$\bigoplus_{\lambda \in \Lambda} \glie_\lambda$ is a Lie~subalgebra of the product~$\prod_{\lambda \in \Lambda} \glie_\lambda$.
\end{proposition}


\begin{proof}
	Let~$(x_\lambda)_\lambda$ and~$(y_\lambda)_\lambda$ be two elements of~$\bigoplus_{\lambda \in \Lambda} \glie_\lambda$.
	Then there exist only finitely many indices~$\lambda \in \Lambda$ for which one of the vectors~$x_\lambda$ or~$y_\lambda$ does not vanish.
	There hence exist only finitely many indices~$\lambda \in \Lambda$ for which~$[x_\lambda, y_\lambda]$ does not vanish.
	This shows that~$[ (x_\lambda)_\lambda, (y_\lambda)_\lambda ]$ is again contained in~$\bigoplus_{\lambda \in \Lambda} \glie_\lambda$.
\end{proof}


\begin{definition}
	In the situation of \cref{construction of direct sum of lie algebras}, the resulting Lie~algebra~$\bigoplus_{\lambda \in \Lambda} \glie_\lambda$\glsadd{external direct sum of lie algebras}\index{external direct sum of Lie algebras}\index{direct sum of Lie algebras!external} is the \defemph{external direct sum} of the Lie~algebras~$\glie_\lambda$.
\end{definition}


\subsection{Quotients of Lie~Algebras}


\begin{proposition}
	\label{construction of quotient lie algebra}
	Let~$\glie$ be a Lie~algebra and let~$I$ an ideal of~$\glie$.
	The quotient vector space~$\glie/I$ inherits from~$\glie$ the structure of a Lie~algebra via the bracket
	\[
		[\class{x}, \class{y}]
		\defined
		\class{ [x,y] }
		\qquad
		\text{for all~$x, y \in \glie$.}
	\]
\end{proposition}


\begin{proof}
	The bilinear map~$[\ph, \ph]$ on~$\glie/I$ is well-defined:
	Suppose that~$x$,~$y$ and~$x'$,~$y'$ are elements of~$\glie$ such that both~$x$ and~$x'$, as well as~$y$ and~$y'$ represent the same element of~$\glie / I$.
	Then both differences~$x - x'$ and~$y - y'$ are contained in~$I$.
	It follows that the difference
	\begin{align*}
		[x,y] - [x', y']
		&=
		[x' + (x-x'), y' + (y-y')] - [x', y']
		\\
		&=
			\underbrace{[x', y-y']}_{\in I}
		+ \underbrace{[x-x', y']}_{\in I}
		+ \underbrace{[x-x', y-y']}_{\in I}
	\end{align*}
	is again contained in~$I$.

	This means that~$[x, y]$ and~$[x', y']$ represent the same element of~$\glie / I$.
	We have thus shown that the bracket on~$\glie/I$ is well-defined.
	
	It remains to show that the bracket~$[\ph, \ph]$ on~$\glie / I$ is bilinear and alternating, and that it satisfies the Jacobi~identity.
	But these properties are inherited from the bracket of~$\glie$, as they can be checked on representatives for the residue classes in~$\glie / I$.
\end{proof}


\begin{definition}
	\label{definition of quotient representation}
	The Lie~algebra~$\glie / I$\glsadd{quotient of lie algebra} from \cref{construction of quotient lie algebra} is the \defemph{quotient}\index{quotient!of Lie algebras} of~$\glie$ by~$I$.
\end{definition}


\subsection{Abelianization of Lie~Algebras}


\begin{proposition}
	\label{quotient is abelian iff moded out the commutator ideal}
	Let~$\glie$ be a Lie~algebra and let~$I$ be an ideal of~$\glie$.
	The quotient Lie~algebra~$\glie / I$ is abelian if and only if the ideal~$I$ contains the commutator ideal of~$\glie$.
\end{proposition}


\begin{proof}
	We have the chain of equivalences
	\begin{align*}
		{}&
		\text{$\glie / I$ is abelian}
		\\
		\iff{}&
		\text{$[ \class{x}, \class{y}] = 0$ for all~$x, y \in \glie$}
		\\
		\iff{}&
		\text{$\class{[x,y]} = 0$ for all~$x, y \in \glie$}
		\\
		\iff{}&
		\text{$[x,y] \in I$ for all~$x, y \in \glie$}
		\\
		\iff{}&
		[\glie, \glie] \subseteq I \,,
	\end{align*}
	which proves the assertion.
\end{proof}


\begin{fluff}
	It follows from \cref{quotient is abelian iff moded out the commutator ideal} that~$\glie / [\glie, \glie]$ is the largest abelian quotient Lie~algebra of~$\glie$.
	This Lie~algebra has a special name:
\end{fluff}


\begin{definition}
	Let~$\glie$ be a Lie~algebra.
	The quotient Lie~algebra~$\glie / [\glie, \glie]$ is the \defemph{abelianization}\index{abelianization} of~$\glie$.
	It is denoted by~$\glie^{\ab}$\glsadd{abelianization of lie algebra}.
\end{definition}


\subsection{Extension of Scalars for Lie~Algebras}


\begin{proposition}
	\label{quasi extension of scalars for lie algebras}
	
	Let~$\glie$ be a Lie~algebra over a field~$\kf$ and let~$A$ be a commutative~{\algebra{$\kf$}}.
	\begin{enumerate}
		\item
			The tensor product~$A \tensor_{\kf} \glie$ becomes an~\liealgebra{$A$} via the bracket
			\begin{alignat*}{2}
				[a \tensor x, b \tensor y]
				&\defined
				(ab) \tensor [x,y]
				&\qquad
				&\text{for all~$a, b \in A$ and~$x, y \in \glie$}.
		\intertext{
		\item
			The tensor product~$\glie \tensor_{\kf} A$ becomes an~\liealgebra{$A$} via the bracket
		}
				[x \tensor a, y \tensor b]
				&\defined
				[x,y] \tensor (ab)
				&\qquad
				&\text{for all~$a, b \in A$ and~$x, y \in \glie$}.
			\end{alignat*}
	\end{enumerate}
\end{proposition}


\begin{proof}
	We prove only the first assertion, as the second one can be shown in the same way.

	The Lie~bracket of~$\glie$ is~\bilinear{$\kf$} and can therefore be regarded as a~\linear{$\kf$} map
	\[
		l \colon \glie \tensor_{\kf} \glie \to \glie \,.
	\]
	By the functoriality of the extension of scalars we now get an induced~\linear{$A$} map
	\[
		\id_A \tensor l
		\colon
		A \tensor_{\kf} \glie \tensor_{\kf} \glie
		\to
		A \tensor_{\kf} \glie \,,
	\]
	which maps any simple tensor~$a \tensor x \tensor y$ onto the simple tensor~$a \tensor [x,y]$.
	We now recall that there exists a unique isomorphism of~\modules{$A$}
	\[
		A \tensor_{\kf} \glie \tensor_{\kf} \glie
		\cong
		( A \tensor_{\kf} \glie ) \tensor_A ( A \tensor_{\kf} \glie )
	\]
	under which a simple tensor~$a \tensor x \tensor b \tensor y$ of the right hand side corresponds to the simple tensor~$(ab) \tensor x \tensor y$ of the left hand side.
	By considering the composite
	\[
		(A \tensor_{\kf} \glie) \tensor_A (A \tensor_{\kf} \glie)
		\to
		A \tensor_{\kf} \glie \tensor_{\kf} \glie
		\xto{\id_A \tensor l}
		A \tensor_{\kf} \glie
	\]
	we get altogether an~{\linear{$A$}} map
	\[
		(A \tensor_{\kf} \glie) \tensor_A (A \tensor_{\kf} \glie)
		\to
		A \tensor_{\kf} \glie
	\]
	which maps any simple tensor~$a \tensor x \tensor b \tensor y$ onto the the simple tensor~$(ab) \tensor [x,y]$.
	We may reinterpret this~\linear{$A$} map as an~\bilinear{$A$} map
	\[
		[\ph, \ph]
		\colon
		(A \tensor_{\kf} \glie) \times (A \tensor_{\kf} \glie)
		\to
		A \tensor_{\kf} \glie
	\]
	that is given on (pairs of) simple tensors by
	\[
		[a \tensor x, b \tensor y]
		=
		(ab) \tensor [x,y] \,.
	\]
	This~\bilinear{$A$} map is precisely the desired bracket.

	It remains to show this bracket~$[\ph, \ph]$ on~$A \tensor_{\kf} \glie$ is already a Lie~bracket.
	For this we need to show that it is alternating and that it satisfies the Jacobi~identity.

	To see that it is alternating let~$t$ be an elements of~$A \tensor_{\kf} \glie$.
	We may write this element as sums of simple tensors, say~$t = \sum_{i=1}^n a_i \tensor x_i$.
	We can then compute the commutator~$[t, t]$ as
	\[
		[t, t]
		=
		\Biggl[
			\sum_{i=1}^n a_i \tensor x_i,
			\sum_{j=1}^n a_j \tensor x_j
		\Biggr]
		=
		\sum_{i, j = 1}^n
		[ a_i \tensor x_i, a_j \tensor x_j ]
		=
		\sum_{i, j = 1}^n
		( a_i a_j ) \tensor [ x_i, x_j ] \,.
	\]
	The summands for~$i = j$ vanish since in this case~$[x_i, x_j] = [x_i, x_i] = 0$.
	For~$i \neq j$ the two occuring summands~$(a_i a_j) \tensor [x_i, x_j]$ and~$(a_j a_i) \tensor [x_j, x_i]$ cancel each other because~$a_i a_j = a_j a_i$ while~$[x_i, x_j] = -[x_j, x_i]$.
	Thus overall~$[t, t] = 0$.
	This shows that the bracket~$[\ph, \ph]$ on~$A \tensor_{\kf} \glie$ is alternating.

	It remains to show that the bracket on~$A \tensor_{\kf} \glie$ satisfies the Jacobi~identity.
	We hence want to show that the map
	\begin{gather*}
		J
		\colon
		(A \tensor_{\kf} \glie) \times (A \tensor_{\kf} \glie) \times (A \tensor_{\kf} \glie)
		\to
		(A \tensor_{\kf} \glie)
	\shortintertext{given by}
		(t, u, v)
		\mapsto
		[t, [u, v]] + [u, [v, t]] + [v, [t, u]]
	\end{gather*}
	is the zero map.
	The map~$J$ is~\trilinear{$A$} by the~\bilinearity{$A$} of the bracket~$[\ph, \ph]$ on~$A \tensor_{\kf} \glie$.
	It hence sufficies to show that the map~$J$ is zero on an~\linear{$A$} generating set of~$A \tensor_{\kf} \glie$.
	Such a generating set is given by the simple tensors~$1 \tensor x$ with~$x \in \glie$.
	We find for these generators that
	\begin{align*}
		\SwapAboveDisplaySkip
		{}&
		J(1 \tensor x, 1 \tensor y, 1 \tensor z)
		\\
		={}&
			[1 \tensor x, [1 \tensor y, 1 \tensor z]]
		+ [1 \tensor y, [1 \tensor z, 1 \tensor x]]
		+ [1 \tensor z, [1 \tensor x, 1 \tensor y]]
		\\
		={}&
		1 \tensor \bigl( [x, [y, z]] + [y, [z, x]] + [z, [x, y]] \bigr)
		\\
		={}&
		1 \tensor 0
		\\
		={}&
		0
	\end{align*}
	for all~$x, y, z \in \glie$ by the Jacobi~identity for~$\glie$.
	It follows that~$J = 0$, as desired.
\end{proof}


\begin{definition}
	The Lie~algebras~$A \tensor_{\kf} \glie$\glsadd{extension of scalars of lie algebra} and~$\glie \tensor_{\kf} A$\glsadd{extension of scalars of lie algebra} from \cref{quasi extension of scalars for lie algebras} are the \defemph{extension of scalars}\index{extension of scalars} of~$\glie$ from~$\kf$ to~$A$.
\end{definition}


\begin{example}
	Let~$\Kf/\kf$ be a field extension and let~$\glie$ be a Lie~algebra over~$\kf$.
	Then the tensor product~$\Kf \tensor_{\kf} \glie$ is a~\liealgebra{$\Kf$} via the Lie~bracket given by
	\[
		[\lambda \tensor x, \mu \tensor y]
		= 
		(\lambda \mu) \tensor [x,y]
		\qquad
		\text{for all simple tensor~$\lambda, \mu \in \Kf$ and~$x, y \in \glie$.}
	\]
	We can moreover regard~$\glie$ as an~\liesubalgebra{$\kf$} of~$\Kf \tensor_{\kf} \glie$ via the inclusion map
	\[
		\glie
		\to
		\Kf \tensor_{\kf} \glie \,,
		\quad
		x
		\mapsto
		1 \tensor x \,.
	\]
\end{example}


\begin{example}
	Let~$\glie$ be a Lie~algebra and let~$\kf[t, t^{-1}]$ be the~\algebra{$\kf$} of Laurent~polynomials over~$\kf$.
	Then
	\[
		\looplie(\glie)
		\defined
		\glie \tensor_{\kf} \kf[t, t^{-1}]
		\glsadd{loop lie algebra}
	\]
	with the Lie~bracket as in~\cref{quasi extension of scalars for lie algebras} is the \defemph{loop Lie~algebra} of~$\glie$.
\end{example}


% \begin{example}
%   Another example for constructing new Lie~algebras out of old ones are \defemph{central extensions}:
%   Let~$\glie$ be any~$\kf$-Lie~algebra.
%   Then let
%   \[
%     \widetilde{\glie}
%     \defined
%     \glie \oplus \kf
%     =
%     \{
%       x + \lambda c
%     \suchthat
%       x \in \glie,
%       \lambda \in \kf
%     \},
%   \]
%   where we understand~$c$ as a formal variable.
%   Suppose that~$\kappa \colon \glie \times \glie \to \kf$ is a~{\bilinear{$\kf$}} map satisfying the following properties:
%   \begin{enumerate}
%   \item
%     $\kappa$ is antisymmetric, i.e.~$\kappa(x,y) = -\kappa(y,x)$ for all~$x,y \in \glie$.
%   \item
%     $\kappa$ satisfies the \defemph{$2$-cocycle condition}
%     \[
%       \kappa([x,y],z) + \kappa([y,z],x) + \kappa([z,x],y) = 0
%     \]
%     for all~$x, y, z \in \glie$.
%   \end{enumerate}
%   Then~$\widetilde{\glie}$ becomes a Lie~algebra via
%   \[
%     [x + \lambda c, y + \mu c]
%     \defined
%     [x,y] + \kappa(x,y) c
%   \]
%   for all~$x, y \in \glie$ and~$\lambda, \mu \in \kf$.
%   Note that the element~$c$ is central in~$\widetilde{\glie}$ in the sense that~$[x,c] = 0$ for all~$x \in \glie$.
%   
%   Take for example~$\glie \defined \gllie(n, \kf)$.
%   We can then define a symmetric bilinear form~$(-,-)_{\tr}$ on~$\glie$ via
%   \[
%     (A,B)_{\tr}
%     \defined
%     \tr(AB)
%   \]
%   for all~$A, B \in \glie$.
%   We can use~$(-,-)_{\tr}$ to define on the Loop~algebra~$\looplie(\glie)$ a~{\bilinear{$\kf[t,t^{-1}]$}} form~$(-,-)$ via
%   \[
%     \looplie(\glie) \times \looplie(\glie)
%     \to
%     \kf[t,t^{-1}] \,,
%     \quad
%     (x \tensor p, y \tensor q)
%     \mapsto
%     (x,y)_{\tr} \, pq \,.
%   \]
%   We get from this a bilinear form a~{\twococycle}~$\kappa \colon \looplie(\glie) \times \looplie(\glie) \to \kf$ via
%   \[
%     \kappa(a,b)
%     \defined
%     \Res\left( \frac{\partial a}{\partial t}, b \right) \,.
%   \]
%   The bilinear form~$\kappa$ is also antisymmetric:
%   Let~$a = x \tensor t^i$ and~$b = y \tensor t^{j}$ with~$x,y \in \glie$ and~$i,j \in \Integer$.
%   Then
%   \begin{align*}
%     \kappa(x \tensor t^i, y \tensor t^{j})
%     &=
%     \Res(i x \tensor t^{i-1}, y \tensor t^{j})
%     \\
%     &= 
%     \Res(i t^{i+j-1} (x,y)_{\tr})
%     \\
%     &=
%     \begin{cases}
%       i (x,y)_{\tr} & \text{if~$i+j = 0$} \,,\\
%                   0 & \text{otherwise}  \,.
%     \end{cases}
%   \end{align*}
%   In the same way we find that
%   \[
%     \kappa(y \tensor t^{j}, x \tensor t^i)
%     =
%     \begin{cases}
%        j (x,y)_{\tr} & \text{if~$i+j = 0$} \,, \\
%                    0 & \text{otherwise}  \,.
%     \end{cases}
%   \]
%   Since~$(\cdot,\cdot)_{\tr}$ is symmetric we find that
%   \begin{align*}
%     \kappa(x \tensor t^i, y \tensor t^{j})
%     &=
%     \begin{cases}
%     i (x,y)_{\tr} & \text{if~$i+j = 0$} \,, \\
%                 0 & \text{otherwise}  \,,
%     \end{cases} \\
%     &=
%     \begin{cases}
%     -j (x,y)_{\tr} & \text{if~$i+j = 0$}  \,, \\
%                   0 & \text{otherwise}  \,,
%     \end{cases} \\
%     &=
%     -\kappa(y \tensor t^{j}, x \tensor t^i) \,.
%   \end{align*}
% \end{example}
% 
% 
% \begin{remark}
%   During the rest of these notes we will never see the Loop algebra again.
% \end{remark}


\subsection{The Opposite Lie~Algebra}


\begin{recall}
	If~$\Aalg$ is a~\enquote{\algebra{$\kf$}}, then we can form its \defemph{opposite \enquote{algebra}}~$\Aalg^{\op}$\glsadd{opposite algebra}\index{opposite algebra@opposite \enquote{algebra}}.

	The \enquote{algebras}~$\Aalg$ and~$\Aalg^{\op}$ have the same underlying~\vectorspace{$\kf$}.
	When we want to regard an element~$a$ of~$\Aalg$ as an element of~$\Aalg^{\op}$, then we write~$a^{\op}$\glsadd{opposite element} instead.
	(So~$a = a^{\op}$ but~$a$ lives in~$\Aalg$ while~$a^{\op}$ lives in~$\Aalg^{\op}$.)
	The multplication on~$\Aalg^{\op}$ is given by
	\[
		a^{\op} \cdot b^{\op}
		\defined
		(b \cdot a)^{\op}
		\qquad
		\text{for all~$a, b \in \Aalg$.}
	\]
	It holds that~$(A^{\op})^{\op} = \Aalg$, and~$(a^{\op})^{\op} = a$ for every element~$a$ of~$\Aalg$.

	The \enquote{algebra}~$\Aalg$ is associative if and only if its opposite \enquote{algebra}~$\Aalg^{\op}$ is associative, and an element~$1$ of~$\Aalg$ is a multiplicative neutral element of~$\Aalg$ if and only if~$1^{\op}$ is a multiplicative neutral element of~$\Aalg^{\op}$.
	It follows that~$\Aalg$ is a~\algebra{$\kf$} if and only if~$\Aalg^{\op}$ is a~\algebra{$\kf$}.
\end{recall}


\begin{proposition}
	Let be a~\liealgebra{$\kf$} with Lie~bracket~$[\ph, \ph]$.
	The opposite \enquote{algebra}~$\glie^{\op}$, whose bracket is given by
	\[
		[x^\op, y^\op]
		=
		[y,x]^{\op}
		=
		-[x,y]^{\op}
		\qquad
		\text{for all~$x, y \in \glie$,}
	\]
	is again a Lie~algebra.
	\qed
\end{proposition}


\begin{definition}
	Let~$\glie$ be a Lie~algebra.
	The Lie~algebra~$\glie^{\op}$\glsadd{opposite lie algebra} is the \defemph{opposite Lie~algebra}\index{opposite Lie algebra} of~$\glie$.
\end{definition}





