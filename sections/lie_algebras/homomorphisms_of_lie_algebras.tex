\section{Homomorphisms of Lie~Algebras}



\subsection{Definition and Basic Examples}


\begin{definition}
	Let~$\glie$ and~$\hlie$ be two~\liealgebras{$\kf$}.
	A~\linear{$\kf$} map~$\varphi$ from~$\glie$ to~$\hlie$ is a \defemph{homomorphism of Lie~algebras}\index{homomorphism!of Lie algebras} if it satisfies the condition
 \[
	 \varphi([x,y])
	 =
	 [\varphi(x), \varphi(y)]
	 \qquad
	 \text{for all~$x, y \in \glie$.}
 \]
\end{definition}


\begin{proposition}
	Let~$\glie$ and~$\hlie$ be two Lie~algebras and let~$\varphi$ be a homomorphism of Lie~algebras from~$\glie$ to~$\hlie$.
	Then
	\[
		[ \varphi(X), \varphi(Y) ]
		=
		\varphi( [X, Y] )
	\]
	for any two subsets~$X$ and~$Y$ of~$\glie$.
	\qed
\end{proposition}


\begin{proposition}
	Let~$\glie$ and~$\hlie$ be two Lie~algebras and let~$\varphi$ be a homomorphism of Lie~algebras from~$\glie$ to~$\hlie$.
	\begin{enumerate}
		\item
			The image\index{image} of~$\varphi$ is a Lie~subalgebra of~$\hlie$.
		\item
			The kernel\index{kernel} of~$\varphi$ is an ideal of~$\glie$.
	\end{enumerate}
\end{proposition}


\begin{proof}
	\leavevmode
	\begin{enumerate}
		\item
			The image of~$\varphi$ is a linear subspace of~$\hlie$ because~$\varphi$ is linear.
			We have
			\[
				[ \im(\varphi), \im(\varphi) ]
				=
				[ \varphi(\glie), \varphi(\glie) ]
				=
				\varphi( [\glie, \glie] )
				\subseteq
				\varphi(\glie)
				=
				\im(\varphi) \,,
			\]
			which shows that~$\im(\varphi)$ is indeed a Lie~subalgebra of~$\hlie$.
		\item
			The kernel of~$\varphi$ is a linear subspace of~$\glie$ because~$\varphi$ is linear.
			We have
			\[
				\varphi( [\glie, \ker(\varphi)] )
				=
				[ \varphi(\glie), \varphi(\ker(\varphi)) ]
				=
				[ \varphi(\glie), 0 ]
				=
				0 \,,
			\]
			which shows that the commutator space~$[\glie, \ker(\varphi)]$ is again contained in~$\ker(\varphi)$.
			This means that~$\ker(\varphi)$ is indeed an ideal in~$\glie$.
		\qedhere
	\end{enumerate}
\end{proof}


\begin{examples}
	\label{homomorphisms of lie algebras}
	Let~$\glie$,~$\hlie$, and~$\klie$ be Lie~algebras.
	\begin{enumerate}
		\item
			\label{identity is a homomorphism of lie algebras}
			The identity map of~$\glie$ is a homomorphism of Lie~algebras from~$\glie$ to~$\glie$.
		\item
			\label{composite of homomorphisms of lie algebras}
			Let~$\varphi$ be a homomorphism of Lie~algebras from~$\glie$ to~$\hlie$ and let~$\psi$ be a homomorphism of Lie~algebras from~$\hlie$ to~$\klie$.
			The composite~$\psi \circ \varphi$ is a homomorphism of Lie~algebras from~$\glie$ to~$\klie$.
		\item
			Let~$\hlie$ be a Lie~subalgebra of~$\glie$.
			The inclusion map from~$\hlie$ to~$\glie$ is a homomorphism of Lie~algebras.
		\item
			Suppose that the Lie~algebras~$\glie$ and~$\hlie$ are both abelian.
			Then any~\linear{$\kf$} map from~$\glie$ to~$\hlie$ is already a homomorphism of Lie~algebras.
		\item
			For every element~$x$ of~$\glie$ let~$\ad(x)$ be the linear map
			\[
				\ad(x)
				\colon
				\glie
				\to
				\glie \,,
				\qquad
				y
				\mapsto
				[x,y] \,.
				\index{adjoint representation}
			\]
			It follows from the bilinearity of the Lie~bracket of~$\glie$ that this describes a~\linear{$\kf$} map
			\[
				\ad
				\glsadd{adjoint map}
				\colon
				\glie
				\to
				\gllie(\glie) \,.
			\]
			It follows from the Jacobi~identity that the map~$\ad$ is already a homomorphism of Lie~algebras.
			Indeed, we have for all element~$x$,~$y$,~$z$ of~$\glie$ that
			\begin{align*}
					\ad([x,y])(z)
					&=
					[[x,y], z]
					\\
					&=
					[ [x,z], y ] + [ x, [y,z] ]
					\\
					&=
					[x,[y,z]] - [y,[x,z]]
					\\
					&=
					\ad(x)(\ad(y)(z)) - \ad(y)(\ad(x)(z))
					\\
					&=
					( \ad(x) \ad(y) - \ad(y) \ad(x) )(z)
					\\
					&=
					[\ad(x), \ad(y)](z) \,,
			\end{align*}
			and therefore~$\ad([x,y]) = [ \ad(x), \ad(y) ]$.
		\item
			\label{algebra homomorphisms are lie algebra homomorphisms}
			Let~$A$ and~$B$ be two associative~{\algebras{$\kf$}}.
			Every homomorphism of~{\algebras{$\kf$}}~$\Phi$ from~$A$ to~$B$ is also a homomorphism of Lie~algebras.
			Indeed, we have for all elements~$a$,~$b$ of~$A$ that
			\[
				\Phi([a,b])
				=
				\Phi(ab - ba)
				=
				\Phi(a) \Phi(b) - \Phi(b) \Phi(a)
				=
				[\Phi(a), \Phi(b)] \,.
			\]
		\item
			Let~$\glie$ be a Lie~algebra over an arbitary field~$\kf$.
			If~$\varphi$ is a homomorphism of Lie~algebras from~$\sllie(2, \kf)$ to~$\glie$, then the images
			\[
				E \defined \varphi(e)  \,,
				\qquad
				H \defined \varphi(h)  \,,
				\qquad
				F \defined \varphi(f)
			\]
			satisfy the commutator relations
			\begin{equation}
				\label{relations for sl2 tripel}
				[H, E] = 2E  \,,
				\qquad
				[H, F] = -2F  \,,
				\qquad
				[E, F] = H \,.
			\end{equation}
			Conversely, every such triple~$(E', H', F')$ of elements of~$\glie$ satisfying the above commutator relations (with~$X$ replaced by~$X'$ for~$X = E, H, F$) gives rise to a unique homomorphism of Lie~algebras~$\varphi'$ from~$\sllie(2, \kf)$ to~$\glie$ that is given by
			\[
				\varphi'(e) = E' \,,
				\qquad
				\varphi'(h) = H' \,,
				\qquad
				\varphi'(f) = F' \,.
			\] 
			These two constructions are mutually inverse.
			We have thus constructed a bijection
			\begin{align*}
				\left\{
					\begin{tabular}{c}
						homomorphisms of \\
						Lie~algebras~$\varphi \colon \sllie(2, \kf) \to \glie$
					\end{tabular}
				\right\}
				&\to
				\left\{
					\begin{tabular}{c}
						triples $(E, H, F)$ of elements of~$\glie$ \\
						that satisfy the relations~\eqref{relations for sl2 tripel}
					\end{tabular}
				\right\} \,,
				\\
				\varphi
				&\mapsto
				\bigl( \varphi(e), \varphi(h), \varphi(f) \bigr) \,.
			\end{align*}
	%     Such triples will play an important role later on.
	\end{enumerate}
\end{examples}


\begin{remark}
	It follows from part~\ref{identity is a homomorphism of lie algebras} and part~\ref{composite of homomorphisms of lie algebras} of \cref{homomorphisms of lie algebras} that the~\liealgebras{$\kf$} form a category\index{category!of Lie algebras}.
	We will denote this category by
	\[
		\cLie{\kf} \,.
		\glsadd{category lie algebras}
	\]
	The class of objects of~$\cLie{\kf}$ is the class of~\liealgebras{$\kf$}.
	For any two Lie~algebras~$\glie$ and~$\hlie$ over~$\kf$ we have
	\[
		\Hom_{\cLie{\kf}}(\glie, \hlie)
		=
		\{
			\text{homomorphisms of Lie~algebras~$\textstyle \glie \to \hlie$} 
		\} \,.
	\]
	The composition of morphisms in~$\cLie{\kf}$ is the usual composition of functions.
	The identity morphism of an object of~$\cLie{\kf}$ is the usual identity function.

	We have a forgetful functor
	\[
		\cLie{\kf}
		\to
		\cVect{\kf}
	\]
	that assigns to every~\liealgebra{$\kf$} its underlying~\vectorspace{$\kf$}.
	It follows from part~\ref{algebra homomorphisms are lie algebra homomorphisms} of \cref{homomorphisms of lie algebras} that we also have a forgetful functor
	\[
		\cAlg{\kf}
		\to
		\cLie{\kf} \,,
	\]
	where~$\cAlg{\kf}$\glsadd{category algebras} denotes the category of~\algebras{$\kf$}\index{category!of algebras}.
	The forgetful functor from~$\cAlg{\kf}$ to~$\cVect{\kf}$ is the composite of the forgetful functor from~$\cAlg{\kf}$ to~$\cLie{\kf}$ and the forgetful functor from~$\cLie{\kf}$ to~$\cVect{\kf}$.
\end{remark}


\begin{proposition}
	\label{inverse of homomorphism of lie algebras is again a homomorphism of lie algebras}
	Let~$\glie$ and~$\hlie$ be two Lie~algebras and let~$\varphi$ be a bijective homomorphism of Lie~algebras from~$\glie$ to~$\hlie$.
	The set-theoretic inverse map of~$\varphi$ is a homomorphism of Lie~algebras from~$\hlie$ to~$\glie$.
\end{proposition}


\begin{proof}
	The inverse~$\varphi^{-1}$ is~\linear{$\kf$} because the original map~$\varphi$ is~\linear{$\kf$}.
	We also have
	\[
		\varphi^{-1}( [x,y] )
		=
		\varphi^{-1}( [ \varphi(\varphi^{-1}(x)), \varphi(\varphi^{-1}(y)) ] )
		=
		\varphi^{-1}( \varphi( [ \varphi^{-1}(x), \varphi^{-1}(y) ] ) )
		=
		[ \varphi^{-1}(x), \varphi^{-1}(y) ]
	\]
	for any two elements~$x$,~$y$ of~$\hlie$.
	This shows that~$\varphi^{-1}$ is a homomorphism of Lie~algebras.
\end{proof}


\begin{remark}
	We have the notion of an \defemph{isomorphism of~\liealgebras{$\kf$}}\index{isomorphism!of Lie algebras} because \liealgebras{$\kf$} form a category.
	More explicitely, a homomorphism of Lie~algebras~$\varphi$ from a Lie~algebra~$\glie$ to a Lie~algebra~$\hlie$ is an isomorphism (of Lie~algebras) if and only if there exists a homomorphism of Lie~algebras~$\psi$ from~$\hlie$ to~$\glie$ such that both~$\psi \circ \varphi = \id_{\glie}$ and~$\varphi \circ \psi = \id_{\hlie}$.

	We can now give a reinterpretation of \cref{inverse of homomorphism of lie algebras is again a homomorphism of lie algebras}:
	a homomorphism of Lie~algebras is an isomorphism of Lie~algebras if and only if it is bijective.
\end{remark}


\begin{example}[Classification of abelian Lie~algebras]
	\index{classification!of abelian Lie algebras}
	Let~$\kf$ be any field.

	Two abelian~{\liealgebras{$\kf$}} are isomorphic (as Lie~algebras) if and only if they are isomorphic as~{\vectorspaces{$\kf$}}, because every vector space isomorphism between them is already an isomorphism of Lie~algebras.
	It follows that there exists precisely one abelian~{\liealgebra{$\kf$}} of each dimension up to isomorphism.
\end{example}


\begin{example}[Classification of {\onedimensional} Lie~algebras]
	\index{classification!of one-dimensional Lie algebras}
	\index{one-dimensional Lie algebras}
	Let~$\kf$ be any field and let~$\glie$ be a~{\onedimensional}~\liealgebra{$\kf$}.
	The Lie~bracket of~$\glie$ is zero since it is an alternating bilinear map on a one-dimensional vector space.
	The Lie~algebra~$\glie$ is therefore abelian.
	It follows that there exists precisely one {\onedimensional}~\liealgebra{$\kf$} up to isomorphism, namely the abelian one.
\end{example}


\begin{example}[Classification of {\twodimensional} Lie~algebras]
	\index{classification!of two-dimensional Lie algebras}
	\index{two-dimensional Lie algebras}
	There exists precisely one {\twodimensional}, abelian~\liealgebra{$\kf$} up to isomorphism.

	Let now~$\glie$ be a {\twodimensional}, non-abelian~\liealgebra{$\kf$}.
	Let~$x'$,~$y'$ be any basis of~$\glie$.
	The commutator space~$[\glie, \glie]$ is nonzero because~$\glie$ is non-abelian, and~$[\glie, \glie]$ is spanned by the single commutator~$[x',y']$ because the Lie~bracket is alternating.
	We thus have~$[\glie, \glie] = \gen{ [x', y'] }_{\kf}$ with~$[x', y']$ being nonzero.

	Let now~$x \defined [x',y']$ and let~$y$ be some element of~$\glie$ that is linearly independent to~$x$.
	Then~$x$,~$y$ is a basis of~$\glie$, and we have seen above that the commutator~$[x,y]$.
	This commutator is also contained in the commutator space~$[\glie, \glie]$, which is spanned by~$x$.
	The element~$[x,y]$ is thus a nonzero scalar multiple of~$x$.
	By rescaling the basis vector~$y$ we may therefore assume that~$[x,y]$ equals~$x$.

	We have thus shown that any two-dimensional, non-abelian Lie~algebra admits a basis~$x$,~$y$ on that its Lie~bracket has a specific form, namely
	\[
		[x,y] = x \,.
	\]
	This shows that there exists at most one {\twodimensional}, non-abelian~\liealgebra{$\kf$} up to isomorphism.

	To show the existence of such a two-dimensional, non-abelian~\liealgebra{$\kf$} we consider the two matrices
	\[
		x
		\defined
		\begin{pmatrix}
			0 & 1 \\
			0 & 0
		\end{pmatrix}
		=
		E_{12}  \,,
		\qquad
		y
		\defined
		\begin{pmatrix}
			0 & 0 \\
			0 & 1
		\end{pmatrix}
		=
		E_{22}  \,.
	\]
	We have
	\[
		[x,y]
		=
		[E_{12}, E_{22}]
		=
		E_{12} E_{22} - E_{22} E_{12}
		=
		E_{12} - 0
		=
		E_{12}
		=
		x \,.
	\]
	The linear subspace of~$\gllie(2, \kf)$ which is spanned by the two matrices~$x$ and~$y$ is therefore a Lie~subalgebra of~$\gllie(2, \kf)$.
	The matrices~$x$ and~$y$ are linearly independent, whence this Lie~subalgebra is {\twodimensional}.
	It is also non-abelian because the matrices~$x$ and~$y$ do not commute.

	We have altogether shown that there exists precisely two {\twodimensional}~\liealgebras{$\kf$} up to isomorphism.
	One of them is abelian, and the other one is non-abelian and admits a basis~$x$,~$y$ with~$[x,y] = x$.
\end{example}

\begin{example}[Three-dimensional Lie~algebras]
	\label{infinitely many three-dimensional lie algebras}
	\index{three-dimensional Lie algebras}
	The number of isomorphism classes of {\threedimensional} Lie~algebras over a field~$\kf$ depends on the cardinality of~$\kf$.

	If the field~$\kf$ is finite, then there exist only finitely many {\threedimensional} Lie~algebras up to isomorphism.
	Indeed, every such Lie~algebra is isomorphic to the vector space~$\kf^3$ together with a suitable Lie~bracket~$[\ph, \ph]$ on~$\kf^3$.
	But there exist only finitely many Lie~brackets on~$\kf^3$ because~$\kf^3$ is finite.

	If the field~$\kf$ is infinite, then there exist infinitely many {\threedimensional}~\liealgebras{$\kf$} up to isomorphism.
	We show this by constructing a family of Lie~algebras~$\glie_\tau$ where~$\tau$ ranges through~$\kf$, such that~$\glie_\tau$ and~$\glie_\upsilon$ are isomorphic if and only if~$\tau = \upsilon$ or~$\tau \upsilon = 1$.

	The Lie~algebra~$\glie_\tau$ is given by a the vector space with basis~$t$,~$x$,~$y$ together with the Lie~bracket given by
	\[
		[t, x] = x \,,
		\quad
		[t, y] = \tau y \,,
		\quad
		[x, y] = 0 \,.
	\]
	Such a Lie~algebra exists since it can be realized as the Lie~subalgebra of~$\gllie(3, \kf)$ that is spanned by the three matrices
	\[
		t
		=
		\begin{pmatrix}
			0 &   &       \\
				& 1 &       \\
				&   & \tau
		\end{pmatrix} \,,
		\qquad
		x
		=
		E_{21} 
		=
		\begin{pmatrix}
			0 & 0 & 0 \\
			1 & 0 & 0 \\
			0 & 0 & 0
		\end{pmatrix} \,,
		\qquad
		y
		=
		E_{31}
		=
		\begin{pmatrix}
			0 & 0 & 0 \\
			0 & 0 & 0 \\
			1 & 0 & 0
		\end{pmatrix} \,.
	\]

	We first observe that for~$\tau \neq 0$ we can replace the basis~$t$,~$x$,~$y$ of~$\glie_\tau$ by the slighty rescaled basis~$t'$,~$x'$,~$y'$ given by
	\[
		t' \defined \frac{t}{\tau} \,,
		\quad
		x' \defined x \,,
		\quad
		y' \defined y \,.
	\]
	The Lie~bracket of~$\glie_\tau$ is on this rescaled basis given by the relations
	\[
		[t', x'] = \frac{1}{\tau} x' \,,
		\quad
		[t', y'] = y' \,,
		\quad
		[y', x'] = 0 \,.
	\]
	We find from these relations that the unique linear map~$\varphi \colon \glie_{\tau} \to \glie_{1/\tau}$ given by
	\[
		\varphi(t') = t \,,
		\quad
		\varphi(x') = y \,,
		\quad
		\varphi(y') = x
	\]
	is an isomorphism of Lie~algebras.
	This shows that~$\glie_{\tau}$ is isomorphic to~$\glie_{\upsilon}$ if~$\tau \neq 0$ and~$\upsilon = 1/\tau$, i.e. if~$\tau \upsilon = 1$.
	
	Let now~$\glie$ be~$\glie_\tau$ for some scalar~$\tau$ in~$\kf$.
	The commutator space~$[\glie, \glie]$ is spanned by~$x$ and~$y$ if~$\tau$ is nonzero, and otherwise spanned by~$x$.
	We can hence distinguish the Lie~algebra~$\glie_0$ from the Lie~algebras~$\glie_\upsilon$ for~$\upsilon \neq 0$ by considering the dimension of~$[\glie, \glie]$.

	To further distinguish between the Lie~algebras~$\glie_\upsilon$ with~$\upsilon \neq 0$ let~$t'$ be some element of~$\glie$ that does not belong to the commutator space~$[\glie, \glie]$.
	We find from the equality~$[\glie, \glie] = \gen{x,y}_{\kf}$ that the element~$t'$ is of the form
	\[
		t'
		=
		\alpha t + \beta x + \gamma y
	\]
	for some coefficients~$\alpha$,~$\beta$,~$\gamma$ in~$\kf$ with~$\alpha \neq 0$.
	We consider the endomorphism
	\[
		\ad(t')
		=
		[t', \ph]
		\colon
		\glie
		\to
		\glie \,.
	\]
	With respect to the basis~$t$,~$x$,~$y$ of~$\glie$ we have
	\[
		\ad(t)
		\equiv
		\begin{pmatrix}
			0 & 0 & 0     \\
			0 & 1 & 0     \\
			0 & 0 & \tau
		\end{pmatrix} \,,
		\quad
		\ad(x)
		\equiv
		\begin{pmatrix*}[r]
			 0  & 0 & 0 \\
			-1  & 0 & 0 \\
			 0  & 0 & 0
		\end{pmatrix*} \,,
		\quad
		\ad(y)
		\equiv
		\begin{pmatrix*}[r]
			0     & 0 & 0 \\
			0     & 0 & 0 \\
			-\tau & 0 & 0
		\end{pmatrix*}
	\]
	and therefore
	\begin{align*}
		\ad(t')
		&=
		\ad(\alpha t + \beta x + \gamma y)
		\\
		&=
		\alpha \ad(t) + \beta \ad(x) + \gamma \ad(y)
		\\
		&\equiv
		\alpha
		\begin{pmatrix}
			0 & 0 & 0     \\
			0 & 1 & 0     \\
			0 & 0 & \tau
		\end{pmatrix}
		+
		\beta
		\begin{pmatrix*}[r]
			 0  & 0 & 0 \\
			-1  & 0 & 0 \\
			 0  & 0 & 0
		\end{pmatrix*}
		+
		\gamma
		\begin{pmatrix*}[r]
			0     & 0 & 0 \\
			0     & 0 & 0 \\
			-\tau & 0 & 0
		\end{pmatrix*}
		\\
		&=
		\begin{pmatrix}
			 0          & 0       & 0           \\
			-\beta      & \alpha  & 0           \\
			-\gamma\tau & 0       & \alpha \tau
		\end{pmatrix} \,.
	\end{align*}
	The eigenvalues of~$\ad(t')$ are thus given by~$0$,~$\alpha$, and~$\alpha \tau$.
	We find from this that the two quotients of the two non-zero eigenvalues of~$\ad(t')$ are given by~$\tau$ and~$1 / \tau$.

	The above calculations shows how the set~$\{ \tau, 1 / \tau \}$ can be reconstructed from the Lie~algebra~$\glie = \glie_\tau$:
	the values~$\tau$ and~$1/\tau$ are precisely the quotients of the nonzero eigenvalues of~$\ad(t')$ where~$t'$ is any element of~$\glie$ not contained in the commutator space~$[\glie, \glie]$.
	It follows that for~$\glie_\tau$ and~$\glie_\upsilon$ to be isomorphic we need the equality~$\{ \tau , 1/\tau \} = \{ \upsilon, 1/\upsilon \}$.
	This means that~$\tau = \upsilon$ or~$\tau = 1 / \upsilon$, i.e.~$\tau \upsilon = 1$.
\end{example}


\subsection{Universal Property of the Product}


\begin{proposition}[Universal property of the product]
	\label{products of lie algebras}
	\index{universal property!of product of Lie algebras}
	Let~$\glie_\lambda$ with~$\lambda$ in~$\Lambda$ be a family of~\liealgebras{$\kf$}.
	\begin{enumerate}
		\item
			The canonical projection
			\[
			 \pi_\mu
			 \colon
			 \prod_{\lambda \in \Lambda} \glie_\lambda
			 \to
			 \glie_\mu \,,
			 \quad
			 ( x_\lambda )_\lambda
			 \mapsto
			 x_\mu
			\]
			is for every index~$\mu$ in~$\Lambda$ a homomorphism of Lie~algebras.
		\item
			Let~$\hlie$ be a~\liealgebra{$\kf$}.
			A map~$\varphi$ from~$\hlie$ to~$\prod_{\lambda \in \Lambda} \glie$ is a homomorphism of Lie~algebras if and only if it is a homomorphisms of Lie~algebras in each coordinate, i.e.\ if and only if for every index~$\mu$ in~$\Lambda$ the composite~$\pi_\mu \circ \varphi$ is a homomorphism of Lie~algebras from~$\hlie$ to~$\glie_\mu$.
		\qed
	\end{enumerate}
\end{proposition}


\begin{proof}
	\leavevmode
	\begin{enumerate}
		\item
			This holds because the Lie~algebra structure on~$\prod_{\lambda \in \Lambda} \glie_{\lambda}$ is defined componentwise.
		\item
			If~$\varphi$ is a homomorphism of Lie~algebras from~$\hlie$ to~$\prod_{\lambda \in \Lambda} \glie_\lambda$, then the composite~$\pi_\mu \circ \varphi$ is for every index~$\mu$ in~$\Lambda$ a composite of homomorphism of Lie~algebras, and therefore again a homomorphism of Lie~algebras.

			Suppose on the other hand that each composite~$\pi_\lambda \circ \varphi$ is a homomorphism of Lie~algebras.
			The map~$\varphi$ is then linear in each coordinate, and thus linear.
			We also have for every two elements~$x$,~$y$ of~$\hlie$ and for every index~$\mu$ in~$\Lambda$ the equalities
			\begin{align*}
				\pi_\mu\bigl( \varphi( [ x, y ] ) \bigr)
				&=
				(\pi_\mu \circ \varphi)( [x, y] )
				\\
				&=
				[ (\pi_\mu \circ \varphi)(x), (\pi_\mu \circ \varphi)(y) ]
				\\
				&=
				[ \pi_\mu( \varphi(x) ), \pi_\mu( \varphi(y) ) ]
				\\
				&=
				\pi_\mu\bigl( [\varphi(x), \varphi(y)] \bigr) \,,
			\end{align*}
			and thus altogether the equality
			\[
				\varphi( [x,y] ) = [ \varphi(x), \varphi(y) ] \,.
			\]
			This shows that the linear map~$\varphi$ is a homomorphism of Lie~algebras.
		\qedhere
	\end{enumerate}
\end{proof}


\begin{remark}
	\Cref{products of lie algebras} shows that the category~$\cLie{\kf}$ has all small products.

	Given three~\liealgebras{$\kf$}~$\glie$,~$\hlie_1$,~$\hlie_2$ and two homomorphisms of Lie~algebras
	\[
		\varphi_1 \colon \glie \to \hlie_1 \,,
		\quad
		\varphi_2 \colon \glie \to \hlie_2 \,,
	\]
	the set
	\[
		\{
			x \in \glie
		\suchthat
			\varphi_1(x) = \varphi_2(x)
		\}
	\]
	is a Lie~subalgebra of~$\glie$.
	This then shows that the category~$\cLie{\kf}$ has binary equalizers.

	Together, this shows that the category~$\cLie{\kf}$ has all small limits, i.e. it is complete.
	We will see in \cref{existence of small colimits} that the category~$\cLie{\kf}$ also has all small colimits, i.e. that it is cocomplete.
\end{remark}


\subsection{Universal Property of the External Direct Sum}


\begin{example}[Universal property of the external direct sum]
	\label{homomorphism out of direct sum}
	\index{universal property!of direct sum of Lie algebras}
	Let~$\glie_\lambda$ with~$\lambda$ in~$\Lambda$ be a family of~\liealgebras{$\kf$}.
	For every index~$\mu$ in~$\Lambda$ we have a~\linear{$\kf$}, injective inclusion map
	\[
		\iota_\mu
		\colon
		\glie_\mu
		\to
		\bigoplus_{\lambda \in \Lambda} \glie_\lambda
	\]
	that is given by
	\[
		\iota_\mu(x)
		=
		(x_\lambda)_\lambda
		\quad\text{with}\quad
		x_\lambda
		=
		\begin{cases*}
			x & if~$\lambda = \mu$, \\
			0 & otherwise,
		\end{cases*}
		\qquad
		\text{for all~$\lambda \in \Lambda$.}
	\]
	These inclusion maps are homomorphisms of Lie~algebras because the Lie~bracket on~$\bigoplus_{\lambda \in \Lambda} \glie_\lambda$ is defined componentwise.
	We can therefore regard every Lie~algebra~$\glie_\mu$ as a Lie~subalgebra of the direct sum~$\bigoplus_{\lambda \in \Lambda} \glie_\lambda$ via the inclusion map~$\iota_\mu$.

	Let~$\hlie$ be another~\liealgebra{$\kf$}.
	We have a {\onetoonetext} correspondence
	\begin{align}
		\left\{
			\begin{tabular}{c}
				\linear{$\kf$} maps \\
				$\varphi \colon \bigoplus_{\lambda \in \Lambda} \glie_\lambda \to \hlie$
			\end{tabular}
		\right\}
		&\onetoone
		\left\{
			\begin{tabular}{c}
				$(\varphi_\lambda)_\lambda$
			\end{tabular}
		\suchthat*
			\begin{tabular}{c}
				\linear{$\kf$} maps\\
				$\varphi_\lambda \colon \glie_\lambda \to \hlie$
			\end{tabular}
		\right\}
		\label{universal property of vector space direct sum}
	\intertext{that is given by}
		\SwapAboveDisplaySkip
		\varphi
		&\mapsto
		(\varphi \circ \iota_\lambda)_\lambda \,,
		\notag
		\\
		\biggl(
			(x_\lambda)_\lambda
			\mapsto
			\sum_{\lambda \in \Lambda}
			\varphi_\lambda(x_\lambda)
		\biggr)
		&\mapsfrom
		(\varphi_\lambda)_\lambda \,.
		\notag
	\end{align}
	We may think about the composite~$\varphi \circ \iota_\lambda$ as the restriction of~$\varphi$ to~$\glie_\lambda$, when~$\glie_\lambda$ is regarded as a Lie~subalgebra of~$\glie$ via the inclusion~$\iota_\lambda$.

	If a~\linear{$\kf$} map~$\varphi$ from~$\bigoplus_{\lambda \in \Lambda} \glie_\lambda$ to~$\hlie$ is a homomorphism of Lie~algebras, then each composite~$\varphi \circ \iota_\mu$ is a homomorphism of Lie~algebras from~$\glie_\mu$ to~$\hlie$, because it is a composite of two homomorphisms of Lie~algebras.
	We note that the Lie~algebras~$\glie_\mu$ commute with each other in~$\bigoplus_{\lambda \in \Lambda} \glie_\lambda$, in the sense that
	\[
		[ \iota_\mu(\glie_\mu), \iota_\kappa(\glie_\kappa) ] = 0
		\qquad
		\text{for any two distinct indices~$\mu$ and~$\kappa$}.
	\]
	It follows that the images of the composites~$\varphi \circ \iota_\lambda$ commute with each other in~$\hlie$, because
	\[
		[
			(\varphi \circ \iota_\mu)(\glie_\mu),
			(\varphi \circ \iota_\kappa)(\glie_\kappa)
		]
		=
		\varphi( [\iota_\mu(\glie_\mu), \iota_\kappa(\glie_\kappa)] )
		=
		\varphi(0)
		=
		0
	\]
	for any two distinct indices~$\mu$ and~$\kappa$.

	Suppose on the other hand that~$\varphi_\lambda \colon \glie_\lambda \to \hlie$ with~$\lambda$ in~$\Lambda$ is a family of homomorphisms of Lie~algebras whose images commute with each other.
	Let~$\varphi$ be the~\linear{$\kf$} from~$\bigoplus_{\lambda \in \Lambda} \glie_\lambda$ to~$\hlie$ corresponding to the maps~$\varphi_\lambda$.
	More explicitely, we have
	\[
		\varphi( (x_\lambda)_\lambda )
		=
		\sum_{\lambda \in \Lambda}
		\varphi_\lambda(x_\lambda)
		\qquad
		\text{for every~$(x_\lambda)_\lambda \in \bigoplus_{\lambda \in \Lambda} \glie_\lambda$.}
	\]
	Then the map~$\varphi$ is again a homomorphism of Lie~algebras.
	Indeed, let~$(x_\lambda)_\lambda$ and~$(y_\lambda)_\lambda$ be two elements of~$\bigoplus_{\lambda \in \Lambda} \glie_\lambda$.
	We have
	\begin{align*}
		[
			\varphi( (x_\lambda)_\lambda ),
			\varphi( (y_\lambda)_\lambda )
		]
		&=
		\Biggl[
			\sum_{\lambda \in \Lambda}
			\varphi_\lambda( x_\lambda ) ,
			\sum_{\mu \in \Lambda}
			\varphi_\mu( y_\mu )
		\Biggr]
		\\
		&=
		\sum_{\lambda, \mu \in \Lambda}
		\underbrace{
			[ \varphi_\lambda( x_\lambda ) , \varphi_\mu( y_\mu ) ]
		}_{
			\text{$= 0$ for~$\lambda \neq \mu$}
		}
		\\
		&=
		\sum_{\lambda \in \Lambda} [ \varphi_\lambda( x_\lambda ), \varphi_\lambda( y_\lambda ) ]
		\\
		&=
		\sum_{\lambda \in \Lambda} \varphi_\lambda( [ x_\lambda , y_\lambda ] )
		\\
		&=
		\varphi( ([x_\lambda, y_\lambda])_\lambda )
		\\
		&=
		\varphi( [ (x_\lambda)_\lambda, (y_\lambda)_\lambda ] ) \,.
	\end{align*}
	This shows that~$\varphi$ is indeed a homomorphism of Lie~algebras.

	It follows from the above assertions that the {\onetoonetext} correspondence~\eqref{universal property of vector space direct sum} restricts to a {\onetoonetext} correspondence
	\begin{align*}
		\left\{
			\begin{tabular}{c}
				Lie~algebra \\
				homomorphisms \\
				$\varphi \colon \bigoplus_{\lambda \in \Lambda} \glie_\lambda \to \hlie$
			\end{tabular}
		\right\}
		&\onetoone
		\left\{
			\begin{tabular}{c}
				$(\varphi_\lambda)_\lambda$
			\end{tabular}
		\suchthat*
			\begin{tabular}{c}
				Lie~algebra homomorphisms \\
				$\varphi_\lambda \colon \glie_\lambda \to \hlie$ whose images \\
				commute with each other
			\end{tabular}
		\right\} \,.
	\end{align*}
\end{example}


\subsection{Internal Direct Sums}


\begin{example}
	\label{direct sum of ideals}
%  If~$I$ and~$J$ are two Lie~algebra over the same field~$\kf$ then their product~$I \times J$ contains the linear subspaces~$I' \defined I \times 0$ and~$J' \defined 0 \times J$ as ideals.
%  These ideals are isomorphic to~$I$ and~$J$ as Lie~algebras via the isomorphisms
%  \begin{alignat*}{2}
%    I
%    &\to
%    I'  \,,
%    &
%    \quad
%    x
%    &\mapsto
%    (x,0) \,,
%    \\
%    J
%    &\to
%    J'  \,,
%    &
%    \quad
%    y
%    &\mapsto
%    (0,y) \,.
%  \end{alignat*}
	Let~$\glie$ be a Lie~algebra and let~$I_\lambda$ with~$\lambda$ in~$\Lambda$ be a family of ideals of~$\glie$ such that~$I = \bigoplus_{\lambda \in \Lambda} I_\lambda$ as vector spaces.
	In other words, the underlying vector space of~$\glie$ is the internal direct sum of the underlying vector spaces of the ideals~$I_\lambda$.
	Let~$\glie$ be a~{\liealgebra{$\kf$}} and let~$I_\lambda$ with~$\lambda$ in~$\Lambda$ be a family of ideals of~$\glie$ such that
	\[
		\glie
		=
		\bigoplus_{\lambda \in \Lambda}
		I_\lambda
	\]
	as vector spaces.
	It then follows for any two distinct indices~$\mu$ and~$\kappa$ that the ideals~$I_\mu$ and~$I_\kappa$ commute with each other in~$\glie$.
	Indeed, the commutator subspace~$[I_\mu, I_\kappa]$ is contained in~$I_\mu$ because~$I_\mu$ is an ideal in~$\glie$.
	This commutator subspace is for the same reason also contained in~$I_\kappa$.
	It follows that~$[I_\mu, I_\kappa]$ is contained in~$I_\mu \cap I_\kappa = 0$, which means that~$[I_\mu, I_\kappa]$ vanishes.

	It follows that the isomorphism of vector spaces
	\[
		\bigoplus_{\lambda \in \Lambda}
		I_\lambda
		\to
		\glie \,,
		\quad
		(x_\lambda)_\lambda
		\mapsto
		\sum_{\lambda \in \Lambda} x_\lambda
	\]
	is an isomorphism of Lie~algebras.

	Let~$\hlie$ be another~\liealgebra{$\kf$}.
	It now follows from \cref{homomorphism out of direct sum} that a homomorphism of Lie~algebras~$\varphi$ from~$\glie$ to~$\hlie$ is \enquote{the same} as a collection of Lie~algebra homomorphisms~$\varphi_\lambda$ from~$I_\lambda$ to~$\hlie$ where~$\lambda$ ranges through~$\Lambda$ and such that the images of~$\varphi_\mu$ and~$\varphi_\kappa$ commute in~$\hlie$ for any two distinct indices~$\mu$ and~$\kappa$.
\end{example}


\begin{definition}
	In the situation of \cref{direct sum of ideals} the Lie~algebra~$\glie$ is the \defemph{internal direct sum}\index{internal direct sum of lie algebras}\index{direct sum of Lie algebras!internal} the ideals~$I_\lambda$.
\end{definition}


\subsection{Universal Property of the Quotient}


\begin{theorem}[Homomorphism theorem]
	\index{homomorphism theorem!for Lie algebras}
	\label{homomorphism theorem}
	Let~$\glie$ be a Lie~algebra and let~$I$ be an ideal of~$\glie$.
	\begin{enumerate}
		\item
			The canonical projection map
			\[
				\pi
				\colon
				\glie
				\to
				\glie/I \,,
				\quad
				x
				\mapsto
				\class{x}
			\]
			is a homomorphism of Lie~algebras.
	\end{enumerate}
	Let~$\hlie$ be another Lie~algebra.
	\begin{enumerate}[resume*]
		\item
			Let~$\psi$ be a homomorphism of Lie~algebras from~$\glie/I$ to~$\hlie$.
			The composite~$\psi \circ \pi$ is a homomorphism of Lie~algebras from~$\glie$ to~$\hlie$ whose kernel contains~$I$.
		\item
			Let conversely~$\varphi$ be a homomorphism of Lie~algebras from~$\glie$ to~$\hlie$.
			The homomorphism~$\varphi$ factors through a homomorphism of Lie~algebras~$\psi$ from~$\glie/I$ to~$\hlie$ that makes the diagram
			\[
				\begin{tikzcd}
					\glie
					\arrow{r}[above]{\varphi}
					\arrow{d}[left]{\pi}
					&
					\hlie
					\\
					\glie/I
					\arrow[dashed]{ur}[below right]{\psi}
					&
					{}
				\end{tikzcd}
			\]
			commute if and only if the ideal~$I$ is contained in the kernel of~$\ker(\varphi)$.
			The homomorphism~$\psi$ is unique and given by
			\[
				\psi( \class{x} )
				=
				\varphi(x)
				\qquad
				\text{for every~$x \in \glie$.}
			\]
			The image and kernel of~$\psi$ are given by~$\im(\psi) = \im(\varphi)$ and~$\ker(\psi) = \ker(\varphi)/I$.
		\qed
	\end{enumerate}
\end{theorem}


%TODO: Add a proof.


\begin{remark}
	The homomorphism theorem is also known as the \defemph{universal property of the quotient}\index{universal property!of quotient Lie algebra}.
\end{remark}


\begin{corollary}[Isomorphism theorems]
	\index{isomorphism theorems!for Lie algebras}
	\label{isomorphism theorems}
	Let~$\glie$ be a Lie~algebra.
	\begin{enumerate}
		\item
			\label{first isomorphism theorem}
			Let~$\hlie$ be another Lie~algebra and let~$\varphi$ be a homomorphism of Lie~algebras from~$\glie$ to~$\hlie$.
			The homomorphism~$\varphi$ induces a well-defined isomorphism of Lie~algebras
			\[
				\glie / {\ker(\varphi)}
				\to
				\im(\varphi)  \,,
				\quad
				\class{x}
				\mapsto
				\class{\varphi(x)} \,.
			\]
		\item
			Let~$I$ and~$J$ be two ideals of~$\glie$ such that~$I$ is contained in~$J$.
			The quotient~$J/I$ is an ideal of the Lie~algebra~$\glie/I$ and the natural isomorphism of vector spaces
			\[
				(\glie/I) / (J/I)
				\to
				\glie/I \,,
				\quad
				\class{ \class{x} }
				\mapsto
				\class{x}
			\]
			is already a natural isomorphism of Lie~algebras.
		\item
			Let~$\hlie$ is a Lie~subalgebra of~$\glie$ and let~$I$ is an ideal of~$\glie$.
			Then the sum~$\hlie + I$ is a Lie~subalgebra of~$\glie$ that contains the ideal~$I$, and the intersection~$\hlie \cap I$ is an ideal of~$\hlie$.
			The natural isomorphism of vector spaces
			\begin{gather*}
				\hlie/(\hlie \cap I)
				\to
				(\hlie + I)/I \,,
				\quad
				\class{x}
				\mapsto
				\class{x}
			\end{gather*}
			is already a natural isomorphism of Lie~algebras.
		\qed
	\end{enumerate}
\end{corollary}


%\begin{lemma}
%  \label{homomorphisms respect commutators of sets}
%  If~$\varphi \colon \glie \to \hlie$ is a homomorphism of Lie~algebras then
%  \[
%    \varphi([X,Y])
%    =
%    [\varphi(X), \varphi(Y)]
%  \]
%  for any two subsets~$X$,~$Y$ of~$\glie$.
%  \qed
%\end{lemma}


\begin{proposition}
	Let~$\glie$ and~$\hlie$ be two Lie~algebras and let~$\varphi$ be a homomorphism of Lie~algebras from~$\glie$ to~$\hlie$.
	\begin{enumerate}
		\item
			For every Lie~subalgebra~$\klie$ of~$\hlie$ its preimage~$\varphi^{-1}(\klie)$ is a Lie~subalgebra of~$\glie$.
		\item
			For every Lie~subalgebra~$\klie$ of~$\glie$ its image~$\varphi(\klie)$ is a Lie~subalgebra of~$\hlie$.
		\item
			For every ideal~$I$ of~$\hlie$ its preimage~$\varphi^{-1}(I)$ is an ideal of~$\glie$.
		\item
			Suppose the homomorphism~$\varphi$ is surjective.
			Then for every ideal of~$I$ of~$\glie$ its image~$\varphi(I)$ is an ideal of~$\hlie$.
		\qed
	\end{enumerate}
\end{proposition}


\begin{warning}
	Let~$\glie$ and~$\hlie$ be two Lie~algebras and let~$\varphi$ be a homomorphism of Lie~algebras from~$\glie$ to~$\hlie$.
	For an ideal~$I$ in~$\glie$ is it in general not true that the image~$\varphi(I)$ is an ideal of~$\hlie$.

	Indeed, let~$\hlie$ be some Lie~algebra and let~$\glie$ be a Lie~subalgebra of~$\hlie$ that is not an ideal of~$\hlie$.
	Then~$I \defined \glie$ is an ideal of~$\glie$ and the inclusion map~$\iota$ from~$\glie$ to~$\hlie$ is a homomorphism of Lie~algebras.
	But the image~$\iota(I) = \hlie$ is by construction not an ideal of~$\hlie$.
\end{warning}


\begin{proposition}[Correspondence theorem]
	\index{correspondence theorem!for Lie algebras}
	\label{correspondence theorem}
	Let~$I$ be an ideal of a Lie~algebra~$\glie$ and let
	\[
		\pi
		\colon
		\glie
		\to
		\glie / I
	\]
	denote the canonical quotient map.
	\begin{enumerate}
		\item
			Let~$\hlie$ be a Lie~subalgebra of~$\glie$ that contains the ideal~$I$.
			The quotient~$\hlie/I$ is a Lie~subalgebra of of~$\glie/I$, and this construction induces a {\onetoonetext} correspondence
			\begin{align*}
				\left\{
					\begin{tabular}{@{}c@{}}
						Lie~subalgebras $\hlie$ of~$\glie$ \\
						containing~$I$
					\end{tabular}
				\right\}
				&\longleftrightarrow
				\{ \text{Lie~subalgebras~$\klie$ of~$\glie/I$} \} \,,
				\\
				\hlie
				&\mapsto
				\hlie/I \,,
				\\
				\pi^{-1}(\klie)
				&\mapsfrom
				\klie \,.
		\intertext{
		\item
			This correspondence restricts to a {\onetoonetext} correspondence
		}
				\left\{
					\begin{tabular}{@{}c@{}}
						Lie~ideals~$J$ in~$\glie$ \\
						containing~$I$
					\end{tabular}
				\right\}
				&\longleftrightarrow
				\{ \text{Lie~ideals~$K$ in~$\glie/I$} \} \,,
				\\
				J
				&\mapsto
				J/I \,,
				\\
				\pi^{-1}(K)
				&\mapsfrom
				K \,.
			\end{align*}
		\item
			Let~$J$ be an ideal of~$\glie$ containing~$I$.
			For the corresponding ideal~$J/I$ of~$\glie/I$ the resulting quotient Lie~algebras~$\glie/J$ and~$(\glie/I)/(J/I)$ are isomorphic via the third isomorphism theorem.
		\qed
	\end{enumerate}
\end{proposition}


\subsection{Universal Property of the Abelianization}


\begin{example}[Universal property of abelianization]
	\index{universal property!of abelianization}
	\label{universal property of abelianization}
	Let~$\glie$ be a~\liealgebra{$\kf$} and let~$\hlie$ be an abelian~\liealgebra{$\kf$}.
	If~$\varphi$ is any homomorphism of Lie~algebras from~$\glie$ to~$\hlie$, then the image of~$\varphi$ is a Lie~subalgebra of~$\glie$, which is then again abelian.
	It follows from the isomorphism between~$\im(\varphi)$ and~$\glie / {\ker(\varphi)}$ that the quotiont Lie~algebra~$\glie / {\ker(\varphi)}$ is abelian.
	The ideal~$\ker(\varphi)$ must therefore contain the commutator ideal~$[\glie, \glie]$, as seen in \cref{quotient is abelian iff moded out the commutator ideal}.
	The homomorphism~$\varphi$ does therefore factor through a homomorphism from~$\glie^{\ab}$ to~$\hlie$.
	This construction gives a {\onetoonetext} correspondence
	\[
		\left\{
			\begin{tabular}{c}
				homomorphisms of \\
				Lie~algebras~$\glie \to \hlie$
			\end{tabular}
		\right\}
		\onetoone
		\left\{
			\begin{tabular}{c}
				homomorphisms of \\
				Lie~algebras $\glie^{\ab} \to \hlie$
			\end{tabular}
		\right\}
		=
		\{
			\textstyle
			\text{linear maps~$\glie^{\ab} \to \hlie$}
		\} \,.
	\]
\end{example}


\begin{proposition}
	\label{functoriality of abelianization}
	Let~$\glie$ and~$\hlie$ be two~\liealgebras{$\kf$} and let~$\varphi$ be a homomorphism of Lie~algebras from~$\glie$ to~$\hlie$.
	The homomorphism~$\varphi$ induces a homomorphism of Lie~algebras
	\[
		\varphi^{\ab}
		\colon
		\glie^{\ab}
		\to
		\hlie^{\ab} \,,
		\quad
		\class{x}
		\mapsto
		\class{\varphi(x)} \,.
	\]
\end{proposition}


\begin{proof}
	The homomorphism~$\varphi$ maps the commutator ideal~$[\glie, \glie]$ into the commutator ideal~$[\hlie, \hlie]$.
	It thus induces a homomorphism from~$\glie / [\glie, \glie]$ to~$\hlie / [\hlie, \hlie]$ as desired.
\end{proof}


\begin{remark}
	Let~$\cLieab{\kf}$\glsadd{category abelian lie algebras} denote the full subcategory\index{category!of abelian Lie algebras} of~$\cLie{\kf}$ whose class of objects consists of those~\liealgebras{$\kf$} which are abelian.
	Let~$U$ be the forgetful functor from~$\cLieab{\kf}$ to~$\cLie{\kf}$ (i.e. the inclusion functor).

	We also have a functor~$(-)^{\ab}$ from~$\cLie{\kf}$ to~$\cLieab{\kf}$.
	This functor assigns to each~\liealgebra{$\kf$}~$\glie$ its abelianization~$\glie^{\ab}$ and to each homomorphism of Lie~algebras~$\varphi$ from~$\glie$ to~$\hlie$ the induced homomorphism of Lie~algebras~$\varphi^{\ab}$ from~$\glie^{\ab}$ to~$\hlie^{\ab}$ as introduced in~\cref{functoriality of abelianization}.

	The discussion in \cref{universal property of abelianization} tells us that the abelianization functor~$(\ph)^{\ab}$ is left adjoint\index{adjunction} to the forgetful functor~$U$.
\end{remark}


\subsection{Universal Property of Extension of Scalars}


\begin{proposition}[Universal property of extension of scalars]
	\index{universal property!of extension of scalars}
	Let~$\glie$ be a~\liealgebra{$\kf$}, let~$A$ be a commutative~\algebra{$\kf$} and let~$\hlie$ be an~\liealgebra{$A$}.
	Every homomorphism of~\liealgebras{$\kf$}~$\varphi$ from~$\glie$ to~$\hlie$ extends uniquely to a homomorphism of~\liealgebras{$A$}~$\psi$ from~$A \tensor_{\kf} \glie$ to~$\hlie$.
\end{proposition}


\begin{proof}
	We know from linear algebra that the homomorphism~$\varphi$ extends uniquely to an~\linear{$A$} map~$\psi$ from~$A \tensor_{\kf} \glie$ to~$\hlie$.
	The compatibility of~$\psi$ with the Lie~brackets of~$A \tensor_{\kf} \glie$ and~$\hlie$ can be checked on an~\generating{$A$} set of~$A \tensor_{\kf} \glie$.
	Such a generating set is given by~$\glie$, on which~$\psi$ agrees with~$\varphi$, and where it is therefore compatible with the Lie~brackets by assumption.
\end{proof}


\subsection{Anti-homomorphisms}


\begin{recall}
	\index{anti-homomorphism!of algebras@of \enquote{algebras}}
	\label{anti-homomorphisms for general algebras}
	Let~$A$ and~$B$ be two~\enquote{\algebras{$\kf$}}.
	A map~$\Phi$ from~$A$ ot~$B$ is an \defemph{anti-homomorphism} if it is~{\linear{$\kf$}} and satisfies the condition
	\[
		\Phi(ab)
		=
		\Phi(b) \Phi(a)
		\qquad
		\text{for all~$a, b \in A$.}
	\]
	For a map~$\Phi$ from~$A$ to~$B$ the following conditions are equivalent.
	\begin{equivalenceslist*}
		\item
			$\Phi$ is an anti-homomorphism from~$A$ to~$B$.
		\item
			$\Phi$ is a homomorphism from~$A^{\op}$ to~$B$.
		\item
			$\Phi$ is a homomorphism from~$A$ to~$B^{\op}$.
		\item
			$\Phi$ is an anti-homomorphism from~$A^{\op}$ to~$B^{\op}$.
	\end{equivalenceslist*}

	An anti-homomorphism~$\Phi$ from~$A$ to~$B$ is an \defemph{anti-isomorphism}\index{anti-isomorphism!of algebras@of \enquote{algebras}} if there exists another anti-homomorphism~$\Psi$ from~$B$ to~$A$ with both~$\Psi \circ \Phi = \id_A$ and~$\Phi \circ \Psi = \id_B$.
	For a map~$\Phi$ from~$A$ to~$B$ the following conditions are equivalent.
	\begin{equivalenceslist*}
		\item
			$\Phi$ is an anti-isomorphism from~$A$ to~$B$.
		\item
			$\Phi$ is an isomorphism from~$A^{\op}$ to~$B$.
		\item
			$\Phi$ is an isomorphism from~$A$ to~$B^{\op}$.
		\item
			$\Phi$ is an anti-isomorphism from~$A^{\op}$ to~$B^{\op}$.
	\end{equivalenceslist*}

	It follows in particular that an anti-homomorphism of~\enquote{\algebras{$\kf$}} is an anti-isomorphism if and only if it is bijective.
\end{recall}


\begin{fluff}
	The general definitions of anti-homomorphisms and anti-isomorphisms for \enquote{algebras} from \cref{anti-homomorphisms for general algebras} can in particular be applied to Lie~algebras.
	We will make this explicit now:
\end{fluff}


\begin{definition}
	Let~$\glie$ and~$\hlie$ be two~\liealgebras{$\kf$}.
	\begin{enumerate}
		\item
			A map~$\varphi$ from~$\glie$ and~$\hlie$ is an \defemph{anti-homomorphism}\index{anti-homomorphism!of Lie algebras} of Lie~algebras if it is~\linear{$\kf$} and satisfies the condition
			\[
				\varphi([x,y]) = [\varphi(y), \varphi(x)]
				\qquad
				\text{for all~$x, y \in \glie$.}
			\]
		\item
			An anti-homomorphism of Lie~algebras~$\varphi$ from~$\glie$ to~$\hlie$ is an \defemph{anti-isomorphism}\index{anti-isomorphism!of Lie algebras} if there exists an anti-homomorphism of Lie~algebras~$\psi$ from~$\hlie$ to~$\glie$ such that both~$\varphi \circ \psi = \id_{\hlie}$ and~$\psi \circ \varphi = \id_{\glie}$.
	\end{enumerate}
\end{definition}


\begin{remark}
	\label{characterizations of anti-homomorphisms for lie algebras}
	The characterizations of anti-homomorphisms and anti-isomorphisms of \enquote{algebras} from \cref{anti-homomorphisms for general algebras} apply in particular to anti-homomorphisms and anti-isomorphisms of Lie~algebras.
	We will therefore not spill out these equivalences a second time.
	Instead, we advise the reader to take the equivalences in \cref{anti-homomorphisms for general algebras} and replace~$A$ by~$\glie$ and~$B$ by~$\hlie$.
\end{remark}


\begin{proposition}
	\label{lie algebra isomorphic to its opposite}
	Every Lie~algebra~$\glie$ is isomorphic to its opposite Lie~algebra~$\glie^{\op}$ via the map
	\[
		\glie
		\to
		\glie^{\op}\,,
		\quad
		x
		\mapsto
		-x^{\op}  \,.
	\]
\end{proposition}


\begin{proof}
	The map~$\varphi$ from~$\glie$ to~$\glie^{\op}$ given by~$x \mapsto -x^{\op}$ for every~$x \in \glie$ is linear, and it is an anti-homomorphism of Lie~algebras because
	\[
		\varphi([x, y])
		=
		-[x,y]^{\op}
		=
		-[ y^{\op}, x^{\op} ]
		=
		[ x^{\op}, y^{\op} ]
		=
		[ -x^{\op}, -y^{\op} ]
		=
		[ \varphi(x), \varphi(y) ]
	\]
	for all~$x, y \in \glie$.
	It is also bijective, and thus an isomorphism of Lie~algebras.
\end{proof}


\begin{corollary}
	We have for any two~\liealgebras{$\kf$}~$\glie$ and~$\hlie$ a {\onetoonetext} correspondence given by
	\begin{align*}
		\SwapAboveDisplaySkip
		\left\{
			\begin{tabular}{c}
				anti-homomorphisms \\
				of Lie~algebras~$\glie \to \hlie$
			\end{tabular} 
		\right\}
		&\onetoone
		\left\{
			\begin{tabular}{c}
				homomorphisms \\
				of Lie~algebras~$\glie \to \hlie$
			\end{tabular}
		\right\} \,,
		\\
		\varphi
		&\mapsto
		-\varphi \,,
		\\
		-\varphi
		&\mapsfrom
		\varphi \,.
	\end{align*}
\end{corollary}


\begin{proof}
	This is a combination of \cref{anti-homomorphisms for general algebras}, respectively \cref{characterizations of anti-homomorphisms for lie algebras}, and \cref{lie algebra isomorphic to its opposite}.
\end{proof}





