\section{Simple Lie~Algebras}
\label{simple lie algebras}


\begin{definition}
	A Lie~algebra~$\glie$ is \defemph{simple}\index{simple Lie algebra} if it is non-abelian and the only ideals of~$\glie$ are~$0$ and~$\glie$.
\end{definition}


\begin{remark}
	\leavevmode
	\begin{enumerate}
		\item
			Any simple Lie~algebra is in particular nonzero.
		\item
			A Lie~algebra is simple if and only if it is non-abelian and contains precisely two ideals.
	\end{enumerate}
\end{remark}


\begin{examples}
	\label{examples for simple lie algebras}
	\leavevmode
	\begin{enumerate}
		\item
			The Lie~algebra~$\gllie(n, \kf)$ isn’t simple.
			For~$n = 0$ it is zero, for~$n = 1$ it is abelian, and for~$n \geq 1$ it contains~$\sllie(n,\kf)$ as a nonzero, proper ideal.
		\item
			The Lie~algebra~$\sllie(2, \kf)$ is simple if and only if the field~$\kf$ does not have characteristic~$2$.
			To prove this we consider the standard basis
			\[
				e
				=
				\begin{pmatrix}
					0 & 1 \\
					0 & 0
				\end{pmatrix} \,,
				\qquad
				h
				=
				\begin{pmatrix*}[r]
					1 &  0  \\
					0 & -1
				\end{pmatrix*} \,,
				\qquad
				f
				=
				\begin{pmatrix}
					0 & 0 \\
					1 & 0
				\end{pmatrix}
			\]
			of~$\sllie(2, \kf)$, on which the Lie~bracket of~$\sllie(2, \kf)$ is given by the relations
			\[
				[h,e] = 2e  \,,
				\qquad
				[h,f] = -2 f \,,
				\qquad
				[e,f] = h \,.
			\]
			
			If the field~$\kf$ is of characteristic~$2$, then the element~$h$ spans a {\onedimensional} ideal, which then shows that~$\sllie(2, \kf)$ is not simple.
			
			Let us consider in the following the case that the field~$\kf$ is not of characteristic~$2$.
			Let~$I$ be a nonzero ideal of~$\sllie(2, \kf)$ and let~$x$ be some nonzero element of~$I$.
			We may write the element~$x$ as a linear combination
			\[
				x = \alpha e + \beta h + \gamma f
			\]
			for some scalar~$\alpha$,~$\beta$,~$\gamma$ in~$\kf$.
			It follows that
			\[
				[e,x]
				=
				-2 \beta e + \gamma h \,,
				\qquad \text{and thus}\qquad
				[e,[e,x]]
				=
				-2 \gamma e \,.
			\]
			The elements~$[e,x]$ and~$[e,[e,x]]$ are both again contained in~$I$ because~$I$ is an ideal of~$\sllie(2, \kf)$.

			It now follows that the ideal~$I$ contains the basis vector~$e$.
			If the scalar~$\gamma$ is nonzero, then it follows from the equality~$[e,[e,x]] = -2 \gamma e$ that
			\[
				e
				=
				- \frac{ [e,[e,x]] }{ 2\gamma } \,.
			\]
			If~$\gamma$ vanishes but the scalar~$\beta$ is nonzero, then it follows from the equality$[e,x] = -2 \beta e$ that
			\[
				e
				=
				-\frac{ [e,x] }{ 2\beta } \,.
			\]
			If both~$\beta$ and~$\gamma$ vanish, then it follows that~$\alpha$ must be nonzero because~$x$ is nonzero, and it then follows from the equality~$x = \alpha e$ that
			\[
				e = \frac{x}{\alpha} \,.
			\]
			We find in each case that the basis vector~$e$ is contained in the ideal~$I$, because the elements~$x$,~$[e,x]$ and~$[e,[e,x]]$ are contained in~$I$.
			
			It further follows that the basis elements~$h$ and~$f$ are also contained in~$I$ because~$h = [e,f]$ and~$f = -[h,f]/2$.
			We have thus altogether found that the ideals~$I$ contains all three standard basis vectors of~$\sllie(2 \kf)$, and must therefore equal~$\sllie(2, \kf)$.
	\end{enumerate}
\end{examples}


\begin{fluff}
	A famous theorem due to Ado -- that we will not attempt to prove in this lecture -- states that this conclusion actually holds for every finite-dimensional Lie~algebra.
\end{fluff}


\begin{theorem}[Ado, first version]
	\label{Ado’s theorem}
	Every finite-dimensional Lie~algebra is isomorphic to a linear Lie~algebra.
\end{theorem}


\begin{fluff}
	We finish this \lcnamecref{simple lie algebras} by proving the general simplicity of~$\sllie(n, \kf)$.
\end{fluff}


\begin{theorem}[Simplicity of~$\sllie(n, \kf)$]
	\label{sln is simple}
	Let~$\kf$ be any field.
	The Lie~algebra~$\sllie(n, \kf)$ is simple if and only if~$n \geq 2$ and the characteristic of~$\kf$ does not divide~$n$.
\end{theorem}


\begin{proof}
	The assertion holds for~$n = 0$ and~$n = 1$ because~$\sllie(2, \kf)$ is then zero, and therefore not simple.
	We have also proven the assertion for~$n = 2$ in \cref{examples for simple lie algebras}.
	We will therefore assume in the following that~$n \geq 3$.

	Suppose first that the characteristic of~$\kf$ divides~$n$.
	The identity matrix is then contained in~$\sllie(n, \kf)$, which shows that the center of~$\sllie(n, \kf)$ is nonzero.
	It follows from \cref{commutator and center of simple} that~$\sllie(n, \kf)$ is not simple.

	Suppose in the following that the characteristic of~$\kf$ does not divide~$n$.
	We abbrevite~$\sllie(n, \kf)$ as~$\glie$, and denote by~$\hlie$ its Lie~subalgebra of traceless diagonal matrices.
	We consider the decomposition
	\begin{equation}
		\label{decomposition of sln for simplicity}
		\glie
		=
		\hlie \oplus \bigoplus_{i \neq j} \gen{ E_{ij} }_{\kf} \,.
	\end{equation}
	Let~$I$ be a nonzero ideal of~$\glie$.
	We show in the following that~$I$ already equals~$\glie$.
	To do this let~$A$ be a nonzero matrix contained in~$I$.
	With respect to the decomposition~\eqref{decomposition of sln for simplicity} we can write the matrix~$A$ as
	\[
		A = A' + \sum_{i \neq j} \alpha_{ij} E_{ij}
	\]
	for some coefficients~$\alpha_{ij}$ in~$\kf$.

	Let us first show that the ideal~$I$ contains some nonzero diagonal matrix.
	If all coefficients~$\alpha_{ij}$ vanish, then~$A = A'$ is this desired matrix.
	Suppose otherwise that the coefficient~$\alpha_{ij}$ does not vanish for some indices~$i$ and~$j$.
	For every~$k = 1, \dotsc, n$ let~$D_k$ be the diagonal matrix whose entries are~$1/n$, except for the~\howmanyth{$k$} one, which is~$1/n - 1$.
	The matrix~$D_k$ has trace zero and is thus contained in~$\glie$.
	It follows from \cref{background on diagonal matrices} that the commutator~$[D_i, A]$ arises from~$A$ by
	\begin{itemize}
		\item
			deleting all diagonal entries of~$A$,
		\item
			multiplying the~\howmanyth{$i$} column with~$1$, and the~\howmanyth{$i$} row with~$-1$ (except for the deleted diagonal entry~$A_{ii}$), and
		\item
		  removing all other entries of~$A$.
	\end{itemize}
	It follows that the matrix~$[D_i, [D_j, A]]$ is again contained in~$I$ and that it is given by
	\[
		\widetilde{A}
		\defined
		- \alpha_{ij} E_{ij} - \alpha_{ji} E_{ji} \,.
	\]
	It follows that the matrix
	\[
		\Bigl[ E_{ji}, - \widetilde{A} \Bigr]
		=
		\alpha_{ij} ( E_{jj} - E_{ii} )
	\]
	is contained in~$I$.
	This is the desired nonzero diagonal matrix in~$I$.

	Let us now show that the ideal~$I$ contains some matrix~$E_{ij}$.
	We consider for this a nonzero diagonal matrix~$D$ contained in~$I$.
	The matrix~$D$ cannot be a scalar multiple of the identity matrix because the characteristic of~$\kf$ does not divide the size~$n$.
	It follows that there exist two indices~$i$ and~$j$ such that the diagonal entries~$D_{ii}$ and~$D_{jj}$ are distinct.
	It follows that the matrix
	\[
		\frac{1}{D_{ii} - D_{jj}} [D, E_{ij}]
		=
		E_{ij}
	\]
	is again contained in the ideal~$I$.

	We now show that the ideal~$I$ contains every matrix~$E_{kl}$ where~$k$ and~$l$ are distinct.
	We have already shown that the ideal~$I$ contains some matrix~$E_{ij}$.
	We distinguish now between the following cases:
	\begin{itemize}
		\item
			If~$(i,j) = (k,l)$, then~$E_{kl} = E_{ij}$ is contained in~$I$.
		\item
			If~$i = k$ but~$j \neq l$, then the matrix~$E_{jl}$ is contained in~$\glie$, and it follows that
			\[
				[ E_{ij}, E_{jl} ]
				=
				[ E_{kj}, E_{jl} ]
				=
				E_{kl}
			\]
			is indeed contained in~$I$.
		\item
			Similarly, if~$i \neq k$ but~$j = l$, then the matrix~$E_{ki}$ is contained in~$\glie$, and it follows that
			\[
				[ E_{ki}, E_{ij} ]
				=
				[ E_{ki}, E_{il} ]
				=
				E_{kl}
			\]
			is indeed contained in~$I$.
		\item
			Suppose that~$i \neq j$ and~$j \neq i$.
			There exists an index~$k$ that is distinct from both~$i$ and~$j$ because~$n \geq 3$.
			It follows that the matrix~$E_{ki}$ is contained in~$\glie$, and hence that the matrix
			\[
				[E_{ki}, E_{ij}]
				=
				E_{kj}
			\]
			is again contained in~$I$.
			It follows from the previous case that~$E_{kl}$ is contained in~$I$.
	\end{itemize}
	It now follows that the Lie subalgebra~$\hlie$ of~$\glie$ is contained in~$I$.
	Indeed, a vector space generating set of~$\hlie$ is given by the matrices~$E_{ii} - E_{jj}$ with~$i \neq j$.
	We have already seen that the matrices~$E_{ij}$ and~$E_{ji}$ are contained in~$I$, whence it follows that
	\[
		[E_{ij}, E_{ji}]
		=
		E_{ii} - E_{jj}
	\]
	is contained in~$I$.

	We have altogether shown that the ideal~$I$ contains both the Lie~subalgebra~$\hlie$ of~$\glie$ and all matrices~$E_{ij}$ with~$i \neq j$.
	It follows from the decomposition~\eqref{decomposition of sln for simplicity} that~$I$ equals~$\glie$.
\end{proof}


\begin{remark}
	It follows from \cref{sln is simple} and \cref{commutator and center of simple} that
	\[
		[\sllie(n, \kf), \sllie(n, \kf)]
		=
		\sllie(n, \kf) \,.
	\]
	In other words, every matrix of trace zero is a linear combination of commutators.
	It was proven in \cite{albert_muckenhoupt_matrices_trace_zero} that every matrix of trace zero is already a commutator itself
\end{remark}





