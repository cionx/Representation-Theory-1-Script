\section{Tensor Algebra and Symmetric Algebra}


% \begin{example}[Monoid algebra]
%   In the following all monoids will be written multiplicaitvely unless otherwise mentioned.
%   The neutral element of a monoid~$M$ will be denoted by~$1$ or~$1_M$.
%   Given two monoids~$M$ and~$N$ a map~$f \colon M \to N$ is a homomorphism of monids if~$f(m \cdot m') = f(m) \cdot f(m')$ for all~$m, m' \in M$ and~$f(1_M) = 1_N$.
%   If~$M$ is any monoid then the identity~$\id_M$ is a homomorphism and if~$f \colon M \to N$ and~$g \colon N \to P$ are composable homomorphisms of monoids then their composition~$g \circ f \colon M \to P$ is again a homomorphism of monoids.
%   The resulting category of monoids is denoted by~$\cMon$.
%   
%   \begin{description}
%     \item[Construction:]
%       If~$M$ is a monoid then the monoid algebra~\gls*{monoid algebra} is the (free) vector space with basis~$M$ together with the unique bilinear extension~$\kf[M] \times \kf[M] \to \kf[M]$ of the multiplication~$M \times M \to M$ as its multiplication.
%   
%       This means that the elements of~$\kf[M]$ are formal {\linear{$\kf$}} combinations~$\sum_{m \in M} a_m m$ with~$a_m = 0$ for all but finitely many~$m \in M$.
%       The multiplication of two such elements is given by
%       \[
%         \left(
%           \sum_{m \in M} a_m m
%         \right)
%         \left(
%           \sum_{n \in M} b_n n
%         \right)
%         =
%         \sum_{m, n \in M} (a_m b_n) m n  \,.
%       \]
%       We identify every element~$m \in M$ with the corresponding element~$1 \cdot m \in \kf[M]$.
%       The product~$m \cdot n$ of two elements~$m, n \in M$ in~$\kf[M]$ is then the same as their product in~$M$.
%       The associativity of the multiplication of~$\kf[M]$ follows from the associativity of the multiplication of~$M$, and the neutral element of~$M$ is given by the multiplicative neutral element for~$\kf[M]$.
%       
%     \item[Universal Property:]
%       If~$A$ is any~{\algebra{$\kf$}} then~$(A, \cdot)$ is a multiplicative monoid, which we will denote by~$A^-$.
%       If~$M$ is any monoid and~$f \colon M \to A^-$ is a monoid hommorphism then~$f$ extends uniquely to an algebra homomorphism~$F \colon \kf[M] \to A$.
%       The algebra homomorphism~$F$ is given on elements by
%       \[
%         F\left( \sum_{m \in M} a_m m \right)
%         =
%         \sum_{m \in M} a_m f(m) \,.
%       \]
%       On the other hand every algebra homomorphism~$\kf[M] \to A$ restricts to a monoid homomorphism~$M \to A^-$.
%       This construction results in a {\onetoonetext} correspondence
%       \[
%         \{
%           \text{monoid homomorphisms~$M \to A^-$}
%         \}
%         \onetoone
%         \{
%           \text{algebra homomorphisms~$\kf[M] \to A$}
%         \}  \,.
%       \]
%     
%     \item[Uniqueness:]
%       The monoid algebra~$\kf[M]$ together with the inclusion~$i \colon M \to \kf[M]$ is uniquely determined by its universal property up to isomorphism:
%   \end{description}
% \end{example}


% \begin{recall}[Free algebra]
%   Let~$I$ be any set.
%   The \defemph{noncommutative polynomial algebra}~$\kf\gen{X_i \suchthat i \in I}$ has as a basis the set of all monomials
%   \[
%     X_{i_1} \dotsm X_{i_n}
%     \qquad
%     \text{with~$i_1, \dotsc, i_n$}
%   \]
%   and the multiplication is on these basis elements given by
%   \[
%     X_{i_1} \dotsm X_{i_n}
%     \cdot
%     X_{j_1} \dotsm X_{j_m}
%     =
%     X_{i_1} \dotsm X_{i_n} X_{j_1} \dotsm X_{j_m} \,.
%   \]
%   In contrast to the usual (commutative) polynomial algebra~$\kf[X_i \suchthat i \in I]$ the variables~$X_i$ are not required to commute with each other.
%   
%   We can alternatively construct~$\kf\gen{X_i \suchthat i \in I}$ as the monomial algebra of the free monoid on~$I$:
%   Let~$M$ be the set of all words in~$I$, i.e.\ the set of all finite sequences
%   \[
%     (i_1, \dotsc, i_n)
%     \qquad
%     \text{with~$i_1, \dotsc, i_n \in I$}  \,.
%   \]
%   Then~$M$ is a monoid with respect to concatenation of words given by
%   \[
%     (i_1, \dotsc, i_n) (j_1, \dotsc, j_m)
%     =
%     (i_1, \dotsc, i_n, j_1, \dotsc, j_m)
%   \]
%   for all words~$(i_1, \dotsc, i_n), (j_1, \dotsc, j_m) \in M$.
%   The neutral element of~$M$ is given by the empty word~$()$.
% \end{recall}





\subsection{Reviewing the Tensor Algebra}


\begin{recall}[Tensor algebra]
	Let~$V$ be a vector space.
	\begin{description}
		\item[Construction]
			For all vectors~$v_1, \dotsc, v_d$ of~$V$ we denote the resulting simple tensor~$v_1 \tensor \dotsb \tensor v_d$ in the tensor power~$V^{\tensor d}$\glsadd{tensor power}\index{tensor power} by~$(v_1, \dotsc, v_d)$\glsadd{simple tensor}.
			The zeroth tensor power~$V^{\tensor 0}$ has as the empty simple tensor~$()$ as a basis.
			We will therefore identify the tensor power~$V^{\tensor 0}$ with the ground field~$\kf$, so that empty simple tensor~$()$ corresponds to the element~$1$ of~$\kf$.
			
			For all exponents~$p, q \geq 0$ we define a multiplication map
			\[
				\mu_{p,q}
				\colon
				V^{\tensor p} \times V^{\tensor q}
				\to
				V^{\tensor (p+q)} \,,
				\quad
				(x,y)
				\mapsto
				x y
			\]
			that is on simple tensors~$(v_1, \dotsc, v_p)$ and~$(v_{p+1}, \dotsc, v_{p+q})$ given by concatination, i.e. by
			\[
				(v_1, \dotsc, v_p) \cdot (v_{p+1}, \dotsc, v_{p+q})
				\defined
				(v_1, \dotsc, v_{p+q})  \,.
			\]
			We note that for~$p = 0$ or~$q = 0$ this multiplication is just the scalar multiplication that is part of the vector space structure of~$V$.
			These multiplication maps~$\mu_{p,q}$ fit together associatively in the sense that we have for all~$p, q, r \geq 0$ and all tensors~$x \in V^{\tensor p}$,~$y \in V^{\tensor q}$ and~$z \in V^{\tensor r}$ the equality
			\[
				x \cdot (y \cdot z)
				=
				(x \cdot y) \cdot z \,.
			\]
			
			Let~$\Tensor(V) \defined \bigoplus_{d \geq 0} V^{\tensor d}$\glsadd{tensor algebra}.
			We can fit together the multiplication maps~$\mu_{p,q}$ with~$p, q \geq 0$ into a single multiplication map
			\[
				\mu
				\colon
				\Tensor(V) \times \Tensor(V)
				\to
				\Tensor(V)  \,,
				\quad
				(x,y)
				\mapsto
				xy 
			\]
			that is given on any two elements~$x$,~$y$ of~$\Tensor(V)$ with~$x = (x_d)_{d \geq 0}$ and~$y = (y_d)_{d \geq 0}$ by
			\[
				x y
				=
				\left(
					\sum_{p+q = d} x_p y_q
				\right)_{d \geq 0} \,.
			\]
			This multiplication~$\mu$ is built precisely such that it follows from the bilinearity of the multiplications~$\mu_{p,q}$ that the multipliation~$\mu$ is again bilinear.
			It follows from the associativities of the multiplications~$\mu_{p,q}$ that the multiplication~$\mu$ is associative.
			We may identify the ground field~$\kf$ with the zeroth tensor power~$V^{\tensor 0}$ and then with the corresponding direct summand of~$\Tensor(V)$.
			In this way, we regard~$\kf$ as a linear subspace of~$\Tensor(V)$.
			We have seen above that the element~$1$ of~$\kf$ is then unital for the multiplication of~$\Tensor(V)$.
			We have thus altogether constructed a~{\algebra{$\kf$}}~$\Tensor(V)$.
			The algebra~$\Tensor(V)$ is the \defemph{tensor algebra}\index{tensor algebra} of~$V$.
			
			We may identify the vector space~$V$ with the first tensor power~$V^{\otimes 1}$ and then with the corresponding direct summand of~$\Tensor(V)$.
			In this way, we regard the vector space~$V$ as a linear subspace of~$\Tensor(V)$.
			We have for all elements~$v_1, \dotsc, v_n$ of~$V$ that
			\[
				v_1 \dotsm v_n
				=
				(v_1) \dotsm (v_n)
				=
				(v_1, \dotsc, v_n)
				=
				v_1 \tensor \dotsb \tensor v_n  \,.
			\]
			It follows in particular that the algebra~$\Tensor(V)$ is generated by~$V$.
			
			We will more generally identify for every natural number~$d$ the tensor power~$V^{\tensor d}$ with the corresponding direct summand of~$\Tensor(V)$.
			Every element of the tensor algebra~$\Tensor(V)$ is then a linear combination of simple tensors.
		
		\item[Universal Property]
			The tensor algebra~$\Tensor(V)$ can be though of as the \enquote{free~{\algebra{$\kf$}} on~$V$} in at least two ways:
			\begin{description}
				\item[Informal]
					The tensor algebra~$\Tensor(V)$ arises from the vector space~$V$ by starting with the elements of~$V$ and then adding all kinds of expressions that can be constructed from the elements of~$V$ via algebra operations.
					But it follows from the axioms of a~\algebra{$\kf$} that many of these expressions have to be the same, so that we only end up with expressions of a certain form.
					
					Let us be a bit more explicit:
					Suppose that a~{\algebra{$\kf$}}~$A$ contains the vector space~$V$ as a linear subspace.
					Then it also contains every product of the form~$v_1 \dotsm v_n$ with~$v_1, \dotsc, v_n$ in~$V$, and hence sums of such products, i.e. elements of the form
					\begin{equation}
						\label{general form of an element in the generated subalgebra}
						\sum_{i=1}^r v_{1i} \dotsm, v_{n_i i}
					\end{equation}
					for natural numbers~$r, n_1, \dotsc, n_r$ and vectors~$v_{ij}$ in~$V$.
					If we continue to combine elements of the form~\eqref{general form of an element in the generated subalgebra} via algebra operations then we do not gain any new elements, since we can simplify the resulting expressions via the axioms of a~\algebra{$\kf$} into something of the form~\eqref{general form of an element in the generated subalgebra}.
					
					But it may happen that two expressions of the form~\eqref{general form of an element in the generated subalgebra} are equal in~$A$ even though this does not follow pureley from the axioms of a~\algebra{$\kf$}.
					Consider for example the polynomial ring~$A = \kf[x, y]$ and the linear subspace~$V = \gen{x, y}_{\kf}$.
					It follows from the axioms of a~\algebra{$\kf$} that the two elements~$x (x+y)$ and~$x^2 + xy$ of~$A$ are equal.
					But it does not follow from these axioms that~$xy = yx$, even though this holds in~$A$.
					There are hence certain additional \emph{relations} between the elements~$x$ and~$y$ of~$V$ in the ambient \algebra{$\kf$}~$\kf[x,y]$.
					
					In the tensor algebra~$\Tensor(V)$ this does not happen.
					If two elements of~$A$ built from elements of~$V$ via algebra operations are equal, then this equality can be derived from the algebra axioms.
					There hence exist no additional relations between the elements of~$V$ in~$\Tensor(V)$.
					(The only required condition is~$V$ being a linear subspace of~$\Tensor(V)$, i.e. that addition and scalar multiplication in~$V$ does coincide with the one coming from~$\Tensor(V)$.)
					
					The tensor algebra~$\Tensor(V)$ is in this way the \enquote{freest} way of expanding the vector space~$V$ into a~\algebra{$\kf$}.
				\item[Formal]
					Let~$i$ be the inclusion map from~$V$ to~$\Tensor(V)$.
					This map is~{\linear{$\kf$}}.
					If~$A$ is a~{\algebra{$\kf$}} and~$f$ is a linear map from~$V$ to~$A$, then any~\linear{$\kf$} map~$f$ extends uniquely to a homomorphism of algebras~$f^+$ from~$\Tensor(V)$ to~$A$.
					In other words, there exists a unique homomorphism of algebras~$f^+$ from~$\Tensor(V)$ to~$A$ that makes the triangular diagram
					\[
						\begin{tikzcd}
							V
							\arrow{r}[above]{f}
							\arrow{d}[left]{i}
							&
							A
							\\
							\Tensor(V)
							\arrow[dashed]{ur}[below right]{f^+}
							&
							{}
						\end{tikzcd}
					\]
					commute.
					The algebra homomorphism~$f^+$ is given by
					\[
						f^+(v_1 \tensor \dotsb \tensor v_d)
						=
						f(v_1) \dotsm f(v_d)
					\]
					for all~$d \geq 0$ and~$v_1, \dotsc, v_n \in V$.
					This construction results in a {\onetoonetext} correspondence
					\begin{align*}
						\left\{
							\begin{tabular}{c}
								\linear{$\kf$} maps \\
								$f \colon V \to A$
							\end{tabular}
					\right\}
						&\onetoone
						\left\{
							\begin{tabular}{c}
								algebra homomorphisms \\
								$\Phi \colon \Tensor(V) \to A$
							\end{tabular}
						\right\} \,,
						\\
						f
						&\mapsto
						f^+ \,,
						\\
						\restrict{\Phi}{V}
						&\mapsfrom
						\Phi \,.
					\end{align*}
					The tensor algebra~$\Tensor(V)$ together with the linear map~$i$ from~$V$ to~$\Tensor(V)$ is in this sense the \enquote{universal way}\index{universal property!of the tensor algebra} of assigning a~{\algebra{$\kf$}} to the vector space~$V$.
			\end{description}
			The formal explanation relates to the informal explanation as follows:
			If~$A$ is any~{\algebra{$\kf$}} that contains~$V$ as a linear subspace, then the inclusion map~$i$ from~$V$ to~$A$ extend uniquely to a homomorphism of algebras~$i^+$ from~$\Tensor(V)$ to~$A$.
			Every relation between expression of the form~\eqref{general form of an element in the generated subalgebra} that holds in~$\Tensor(V)$ must then also hold in~$A$.
			The relations that hold in~$\Tensor(V)$ are the therefore precisely those relations which hold in \emph{every}~{\algebra{$\kf$}} containing~$V$.
			
		\item[Uniqueness]
			The above universal property determines the tensor algebra~$\Tensor(V)$ together with the inclusion map~$i$ from~$V$ to~$\Tensor(V)$ uniquely up to unique isomorphism, in the following sense.

			Let~$T$ be another~{\algebra{$\kf$}} and let~$j$ be a linear map from~$V$ to~$T$.
			Suppose that for every~{\algebra{$\kf$}}~$A$ and every~{\linear{$\kf$}} map~$f$ from~$V$ to~$A$ there exists a unique homomorphism of algebras~$\Phi$ from~$T$ to~$A$ that makes the triangular diagram
			\[
				\begin{tikzcd}
					V
					\arrow{r}[above]{f}
					\arrow{d}[left]{j}
					&
					A
					\\
					T
					\arrow[dashed]{ur}[below right]{\Phi}
					&
					{}
				\end{tikzcd}
			\]
			commute.
			Then there exist unique homomorphisms of algebras~$\Phi$ from~$A$ to~$T$ and a unique homomorphism of algebras~$\Psi$ from~$T$ to~$A$ that make the triangular diagrams
			\[
				\begin{tikzcd}[column sep = {3em,between origins}]
					{}
					&
					V
					\arrow{dl}[above left]{i}
					\arrow{dr}[above right]{j}
					&
					{}
					\\
					\Tensor(V)
					\arrow[dashed]{rr}[below]{\Phi}
					&
					{}
					&
					T
				\end{tikzcd}
				\qquad\text{and}\qquad
				\begin{tikzcd}[column sep = {3em,between origins}]
					{}
					&
					V
					\arrow{dl}[above left]{j}
					\arrow{dr}[above right]{i}
					&
					{}
					\\
					T
					\arrow[dashed]{rr}[below]{\Psi}
					&
					{}
					&
					\Tensor(V)
				\end{tikzcd}
			\]
			commute.
			It then follows that the composites~$\Psi \circ \Phi$ and~$\Phi \circ \Psi$ make the triangular diagrams
			\[
				\begin{tikzcd}[column sep = small]
					{}
					&
					V
					\arrow{dl}[above left]{i}
					\arrow{dr}[above right]{i}
					&
					{}
					\\
					\Tensor(V)
					\arrow[dashed]{rr}[below]{\Psi \circ \Phi}
					&
					{}
					&
					\Tensor(V)
				\end{tikzcd}
				\qquad\text{and}\qquad
				\begin{tikzcd}[column sep = small]
					{}
					&
					V
					\arrow{dl}[above left]{j}
					\arrow{dr}[above right]{j}
					&
					{}
					\\
					T
					\arrow[dashed]{rr}[below]{\Phi \circ \Psi}
					&
					{}
					&
					T
				\end{tikzcd}
			\]
			commute.
			The algebra homomorphisms~$\Psi \circ \Phi$ and~$\Phi \circ \Psi$ are unique with this propert by the universal properties of~$(\Tensor(V), i)$ and~$(T, j)$.
			But the identities~$\id_{\Tensor(V)}$ and~$\id_T$ also make these diagrams commute.
			We therefore find that~$\Psi \circ \Phi = \id_{\Tensor(V)}$ and~$\Phi \circ \Psi = \id_{T}$.
			The homomorphisms of algebras~$\Phi$ and~$\Psi$ are therefore mutually inverse isomorphisms of algebras.
		
		\item[Functoriality]
			Let~$V$ and~$W$ be two vector spaces and let~$f$ be a~\linear{$\kf$} map from~$V$ to~$W$.
			We can consider the following diagram.
			\[
				\begin{tikzcd}
					V
					\arrow{r}[above]{f}
					\arrow{d}[left]{i_V}
					&
					W
					\arrow{d}[right]{i_W}
					\\
					\Tensor(V)
					&
					\Tensor(W)
				\end{tikzcd}
			\]
			By applying the universal property of the tensor algebra~$\Tensor(V)$ to the composite~$i_W \circ f$, it follows that there exists a unique algebra homomorphism~$\Tensor(f)$\glsadd{tensor algebra on morphisms}\index{induced homomorphism!of tensor algebras} from~$\Tensor(V)$ to~$\Tensor(W)$ that makes the square diagram
			\[
				\begin{tikzcd}[column sep = large, row sep = huge]
					V
					\arrow{r}[above]{f}
					\arrow{d}[left]{i_V}
					\arrow[dashed]{dr}[above right]{i_W \circ f}
					&
					W
					\arrow{d}[right]{i_W}
					\\
					\Tensor(V)
					\arrow[dashed]{r}[below]{\Tensor(f)}
					&
					\Tensor(W)
				\end{tikzcd}
			\]
			commute.
			This homomorphism is given by
			\[
				f(v_1 \tensor \dotsb v_d)
				=
				f(v_1) \dotsm f(v_d)
			\]
			for all~$d \geq 0$ and~$v_1, \dotsc, v_d \in V$.

			This induced algebra homorphism is functorial in the following sense:
			\begin{itemize}
				\item
					It holds that~$\Tensor(\id_V) = \id_{\Tensor(V)}$.
					Indeed, the commutativity of the square 
					\[
						\begin{tikzcd}[column sep = large]
							V
							\arrow{r}[above]{f}
							\arrow{d}
							&
							V
							\arrow{d}
							\\
							\Tensor(V)
							\arrow[dashed]{r}[below]{\Tensor(\id_V)}
							&
							\Tensor(V)
						\end{tikzcd}
					\]
					shows that the identity~$\id_{\Tensor(V)}$ satifies the defining property of the induced algebra homomorphism~$\Tensor(f)$.
				\item
					Let~$U$,~$V$,~$W$ be three vector spaces.
					Let~$f$ be a linear map from~$U$ to~$V$ and let~$g$ be a linear map from~$V$ to~$W$.
					Then
					\[
						\Tensor(g \circ f)
						=
						\Tensor(g) \circ \Tensor(f) \,.
					\]
					Indeed, it follows from the commutativity of the diagram
					\[
						\begin{tikzcd}[column sep = large]
							U
							\arrow[dashed, bend left=45]{rr}[above]{g \circ f}
							\arrow{r}[above]{f}
							\arrow{d}
							&
							V
							\arrow{r}[above]{g}
							\arrow{d}
							&
							W
							\arrow{d}
							\\
							\Tensor(U)
							\arrow{r}[below]{\Tensor(f)}
							\arrow[dashed, bend right=45]{rr}[below]{\Tensor(g) \circ \Tensor(f)}
							&
							\Tensor(V)
							\arrow{r}[below]{\Tensor(g)}
							&
							\Tensor(W)
						\end{tikzcd}
					\]
					that the subdiagram
					\[
						\begin{tikzcd}[column sep = huge]
							U
							\arrow{r}[above]{g \circ f}
							\arrow{d}
							&
							W
							\arrow{d}
							\\
							\Tensor(U)
							\arrow[dashed]{r}[below]{\Tensor(g) \circ \Tensor(f)}
							&
							\Tensor(W)
						\end{tikzcd}
					\]
					commutes.
					This shows that the composite~$\Tensor(g) \circ \Tensor(f)$ satisfies the defining property of the induced homomorphism of algebras~$\Tensor(g \circ f)$.
			\end{itemize}
			
			We see from the above discussion that the assignment~$\Tensor$ defines a functor from~$\cVect{\kf}$ to~$\cAlg{\kf}$.
			The universal property of the tensor algebra states that the functor~$\Tensor$ is left adjoint\index{adjunction} to the forgetful functor from~$\cAlg{\kf}$ to~$\cVect{\kf}$.
		
		\item[Description via a basis]
			Let~$(v_i)_{i \in I}$ be a basis of~$V$
			Then for every natural number~$d$ the tensor power~$V^{\tensor d}$ inherits a basis from~$V$, given by all simple tensors
			\begin{equation}
				\label{basis element of tensor power}
				v_{i_1} \tensor \dotsb \tensor v_{i_d}
			\end{equation}
			with~$i_1, \dotsc, i_d \in I$.
			It follows that the tensor algebra~$\Tensor(V)$ has a basis given by all such simple tensors with~$d \geq 0$ and~$i_1, \dotsc, i_d \in I$.

			The product of two such basis vectors is again a basis vector.
			We may think about the basis vector~\eqref{basis element of tensor power} of~$\Tensor(V)$ as the finite word~$i_1 \dotso i_d$ in the alphabet~$I$, and about the multiplication of two basis vectors as the concatenation of words.
			
			If we think about the basis vector~$v_i$ of~$V$ as a formal variable~$X_i$ then we see that the tensor algebra~$\Tensor(V)$ is isomorphic to the noncommutative polynomial algebra~$\kf\gen{X_i \suchthat i \in I}$\glsadd{noncommutative polynomial algebra}.

			This can also seen via adjunctions.
			Indeed, we have the following commutative diagram of forgetful functors.
			\[
				\begin{tikzcd}
					\cAlg{\kf}
					\arrow{d}
					\arrow[bend left = 60]{dd}
					\\
					\cVect{\kf}
					\arrow{d}
					\\
					\cSet
				\end{tikzcd}
			\]
			It follow that the resulting diagram of left-adjoint functors
			\[
				\begin{tikzcd}
					\cAlg{\kf}
					\\
					\cVect{\kf}
					\arrow{u}[left]{\Tensor}
					\\
					\cSet
					\arrow{u}[left]{F}
					\arrow[bend right = 60]{uu}[right]{\kf\gen{X_i \suchthat i \in (-)}}
				\end{tikzcd}
			\]
			commutes up to natural isomorphism.
			The vector space~$V$ has a basis indexed by~$I$ and is therefore isomorphic to free vector space~$F(I)$.
			Hence
			\[
				\Tensor(V)
				\cong
				\Tensor(F(I))
				\cong
				\kf\gen{X_i \suchthat i \in I} \,.
			\]
	\end{description}
\end{recall}





\subsection{Reviewing the Symmetric Algebra}


\begin{recall}[Symmetric power]
	Let~$V$ be a vector space and let~$d$ be a natural number.
	The~{\howmanyth{$d$}} \defemph{symmetric power}~$\Symm^d(V)$\glsadd{symmetric power}\index{symmetric power} is the quotient vector space of the tensor power~$V^{\tensor d}$ by its linear subspace~$U_d$ that is generated by all differences of the form
	\[
			v_1 \tensor \dotsb \tensor v_d
		- v_{\sigma(1)} \tensor \dotsb \tensor v_{\sigma(d)}
	\]
	with~$v_1, \dotsc, v_d \in V$ and~$\sigma \in \symm_d$.
	Hence
	\[
		\Symm^d(V)
		=
		V^{\tensor d} / U_d
		=
		V^{\tensor d}
		/
		\gen{
				v_1 \tensor \dotsb \tensor v_d
			- v_{\sigma(1)} \tensor \dotsb \tensor v_{\sigma(d)} 
		\suchthat
			v_1, \dotsc, v_n \in V,
			\sigma \in \symm_n
		}_{\kf} \,.
	\]
	We can identify the zeroth symmetric power~$\Symm^0(V)$ with the zeroth tensor prower~$V^{\tensor 0}$ and thus with~$\kf$, because~$U_0$ vanishes.
	For all vectors~$v_1, \dotsc, v_n$ in~$V$ we denote the residue class of the simple tensor~$v_1 \tensor \dotsb \tensor v_d$ in~$\Symm^d(V)$ by~$v_1 \dotsm v_d$\glsadd{simple symmetric tensor}, and call this a \defemph{simple symmetric tensor}\index{simple symmetric tensor}\index{symmetric tensor}.
	We have by construction of~$\Symm^d(V)$ that
	\[
		v_1 \dotsm v_d
		=
		v_{\sigma(1)} \dotsm v_{\sigma(d)}
	\]
	for all~$v_1, \dotsc, v_n \in V$ and~$\sigma \in \symm_d$.

	The symmetric power~$\Symm^d(V)$ is in the following sense universal with this property:
	The map
	\[
		\alpha
		\colon
		\underbrace{ V \times \dotsb \times V }_d
		\to
		\Symm^d(V)  \,,
		\quad
		(v_1, \dotsc, v_d)
		\mapsto
		v_1 \dotsm v_d
	\]
	is symmetric and multilinear.
	If~$\beta$ is any symmetric, multilinear map from the~\fold{$d$} product~$V \times \dotsb \times V$ to another vector space~$W$, then there exists a unique linear map~$f$ from~$\Symm^d(V)$ to~$W$ that makes the triangular diagram
	\[
		\begin{tikzcd}[row sep = large]
			V \times \dotsb \times V
			\arrow{d}[left]{\alpha}
			\arrow{dr}[above right]{\beta}
			&
			{}
			\\
			\Symm^d(V)
			\arrow[dashed]{r}[below]{f}
			&
			W
		\end{tikzcd}
	\]
	commute.
	A linear map from~$\Symm^d(V)$ to~$W$ is in this sense \enquote{the same} as a symmetric, multilinear map from the~\fold{$d$} product~$V \times \dotsb \times V$ to~$W$.
	
	If~$(v_i)_{i \in I}$ is a basis of~$V$ such that~$(I, \leq)$ is a linearly ordered set then the ordered monomials
	\[
		v_{i_1} \dotsm v_{i_d}
		\qquad
		\text{with~$i_1 \leq \dotsb \leq i_d$}
	\]
	form a basis of the symmetric power~$\Symm^d(V)$.
	If~$V$ is of finite dimension~$n$ then it follows that
	\[
		\dim \Symm^d(V)
		=
		\binom{n+d-1}{d}  \,.
	\]
	(This can be seen via stars and bars.)
\end{recall}


\begin{recall}[Symmetric algebra]
	Let~$V$ be a vector space.
	Just as the tensor powers~$V^{\otimes d}$ can be used to construct the tensor algebra~$\Tensor(V)$, we can also use the symmetric powers~$\Symm^d(V)$ to construct the \defemph{symmetric algebra}~$\Symm(V)$\glsadd{symmetric algebra}\index{symmetric algebra}.
	Just as the tensor algebra~$\Tensor(V)$ is the free~\algebra{$\kf$} on~$V$, the symmetric algebra~$\Symm(V)$ is the free symmetric algebra on~$V$.

	The argumentation for the symmetric algebra is analogous to that for the tensor algebra.
	We will therefore skip some of the details.
	
	\begin{description}
		\item[Construction]
			For all vectors~$v_1, \dotsc, v_d$ in~$V$ we denote the corresponding simple symmetric tensor in~$\Symm^d(V)$ by~$v_1 \dotsm v_d$.
			We can define on the direct sum~$\Symm(V) \defined \bigoplus_{d \geq 0} \Symm^d(V)$ a multiplication such that
			\[
				(v_1 \dotsm v_p) \cdot (v_{p+1} \dotsm v_{p+q})
				=
				v_1 \dotsm v_p v_{p+1} \dotsm v_{p+q}
			\]
			for all~$p, q \geq 0$ and~$v_1, \dotsc, v_{p+q} \in V$.
			By identifying the zeroth symmetric power~$\Symm^0(V)$ with the ground field~$\kf$ we make~$\Symm(V)$ into an associative~{\algebra{$\kf$}}.
			This algebra is commutative since we have
			\begin{align*}
				\SwapAboveDisplaySkip
				(v_1 \dotsm v_p) \cdot (v_{p+1} \dotsm v_{p+q})
				&=
				v_1 \dotsm v_p v_{p+1} \dotsm v_{p+q}
				\\
				&=
				v_{p+1} \dotsm v_{p+q} v_1 \dotsm v_p
				\\
				&=
				(v_{p+1} \dotsm v_{p+q}) \cdot (v_1 \dotsm v_p)
			\end{align*}
			for all~$p, q \geq 0$ and~$v_1, \dotsc, v_{p+q} \in V$. 
			We can identify the vector space~$V$ with the first symmetric power~$\Symm^1(V)$ and then with the corresponding direct summand of~$\Symm(V)$.
			We can more generally identify every symmetric power~$\Symm^d(V)$ with the corresponding direct summand of~$\Symm(V)$.
			Every element of the symmetric algebra~$\Symm(V)$ is then a a linear combination of simple symmetric tensors.
			
		\item[Universal property]
			The symmetric algebra~$\Symm(V)$ is in the following sense the \enquote{free commutative~{\algebra{$\kf$}}} on the vector space~$V$.

			Let~$i$ be the inclusion map from~$V$ to~$\Symm(V)$.
			There exists for every symmetric~{\algebra{$\kf$}}~$A$ and every linear map~$f$ from~$V$ to~$A$ a unique homomorphism of algebras~$f^+$ from~$\Symm(V)$ to~$A$ that makes the triangular diagram
			\[
				\begin{tikzcd}
					V
					\arrow{r}[above]{f}
					\arrow{d}[left]{i}
					&
					A
					\\
					\Symm(V)
					\arrow[dashed]{ur}[below right]{f^+}
					&
					{}
				\end{tikzcd}
			\]
			commute.
			The homomorphism~$f^+$ is given by
			\[
				f^+(v_1 \dotsm v_d)
				=
				f(v_1) \dotsm f(v_d)
			\]
			for all~$d \geq 0$ and~$v_1, \dotsc, v_d \in V$.
			This construction results in a {\onetoonetext} correspondence\index{universal property!of the symmetric algebra}
			\begin{align*}
				\left\{
					\begin{tabular}{c}
						\linear{$\kf$} maps \\
						$f \colon V \to A$
					\end{tabular}
			\right\}
				&\onetoone
				\left\{
					\begin{tabular}{c}
						algebra homomorphisms \\
						$\Phi \colon \Symm(V) \to A$
					\end{tabular}
				\right\} \,,
				\\
				f
				&\mapsto
				f^+ \,,
				\\
				\restrict{\Phi}{V}
				&\mapsfrom
				\Phi \,.
			\end{align*}
			
			It follows that a relations between elements of~$V$ holds in the symmetric algebra~$\Symm(V)$ if and only if it holds in every commutative algebra that contains~$V$.
			
		\item[Uniqueness]
			Let~$S$ be a commutative~{\algebra{$\kf$}} and let~$j$ be a linear map from~$V$ to~$S$.
			Suppose that the pair~$(S, j)$ satisfies the same universal property as the symmetric algebra~$(\Symm(V), i)$.
			Then there exists a unique homomorphism of algebras~$\Phi$ from~$\Symm(V)$ and a unique homomorphisms of algebras~$\Psi$ from~$S$ to~$\Symm(V)$ that make the triangular diagrams
			\[
				\begin{tikzcd}[column sep = {3em,between origins}]
					{}
					&
					V
					\arrow{dl}[above left]{i}
					\arrow{dr}[above right]{j}
					&
					{}
					\\
					\Symm(V)
					\arrow[dashed]{rr}[below]{\Phi}
					&
					{}
					&
					S
				\end{tikzcd}
				\qquad\text{and}\qquad
				\begin{tikzcd}[column sep = {3em,between origins}]
					{}
					&
					V
					\arrow{dl}[above left]{j}
					\arrow{dr}[above right]{i}
					&
					{}
					\\
					S
					\arrow[dashed]{rr}[below]{\Psi}
					&
					{}
					&
					\Symm(V)
				\end{tikzcd}
			\]
			commute.
			These homomorphisms~$\Phi$ and~$\Psi$ are mutually inverse isomorphisms of algebras.
			
		\item[Functoriality]
			Let~$V$ and~$W$ be two vector spaces and let~$f$ be a linear map from~$V$ to~$W$.
			Then there exists a unique homomorphism of algebras~$\Symm(f)$\glsadd{symmetric algebra on morphisms}\index{induced homomorphism!on symmetric algebras} from~$\Symm(V)$ to~$\Symm(W)$ that makes the square diagram
			\[
				\begin{tikzcd}
					V
					\arrow{r}[above]{f}
					\arrow{d}
					&
					W
					\arrow{d}
					\\
					\Symm(V)
					\arrow[dashed]{r}[below]{\Symm(f)}
					&
					\Symm(W)
				\end{tikzcd}
			\]
			commmute.
			This homomorphism~$\Symm(f)$ is given by
			\[
				\Symm(f)(v_1 \dotsm v_d)
				=
				f(v_1) \dotsm f(v_d)
			\]
			for all~$d \geq 0$ and~$v_1, \dotsc, v_d \in V$.

			It holds that~$\Symm(\id_V) = \id_{\Symm(V)}$ and it holds for all composable~{\linear{$\kf$}} maps~$f$ from~$U$ to~$V$ and~$g$ from~$V$ to~$W$ that~$\Symm(g \circ f) = \Symm(g) \circ \Symm(f)$.
			The assignment~$\Symm$ is thus a functor from~$\cVect{\kf}$ to~$\cCAlg{\kf}$, where~$\cCAlg{\kf}$\glsadd{category commutative algebras}\index{category!of commutative algebras} denotes the category of commutative~{\algebras{$\kf$}}.
			
		\item[Description via a basis]
			Let~$(v_i)_{i \in I}$ be a basis of$~V$ where~$(I, \leq)$ is a linearly ordered set.
			Then the symmetric power~$\Symm^d(V)$ inherits a basis from~$V$ given by all simple symmetric tensors
			\[
				v_{i_1} \dotsm v_{i_d}
				\qquad
				\text{where~$i_1 \leq \dotsb \leq i_d$} \,.
			\]
			It follows that the symmetric algebra~$\Symm(V)$ has the simple symmetric tensors~$v_{i_1} \dotsm v_{i_d}$ with~$d \geq 0$ and~$i_1, \dotsc, i_d \in I$ such that~$i_1 \leq \dotsb \leq i_d$ as a basis.
			This basis may also be written as
			\[
				v_{i_1}^{n_1} \dotsm v_{i_r}^{n_r}
			\]
			with~$r \geq 0$ and~$i_1, \dotsc, i_r \in I$ such that~$i_1 < \dotsb < i_r$, and~$n_1, \dotsc, n_r \geq 0$.
			(This second description of the induced basis on~$\Symm(V)$ is connected to the first description via~$d = n_1 + \dotsb + n_r$).
			
			We see from this description that the symmetric algebra~$\Symm(V)$ is isomorphic to the commutative polynomial algebra~$\kf[X_i \suchthat i \in I]$, which is the free commutative~{\algebra{$\kf$}} on the generators~$X_i$ with~$i \in I$.
			This can again be explained by considering the commutative diagram of forgetful functors
			\[
				\begin{tikzcd}
					\cCAlg{\kf}
					\arrow{d}
					\arrow[bend left = 60]{dd}
					\\
					\cVect{\kf}
					\arrow{d}
					\\
					\cSet
				\end{tikzcd}
			\]
			from which we see that the resulting diagram of left adjoints\index{adjunction}
			\[
				\begin{tikzcd}
					\cCAlg{\kf}
					\\
					\cVect{\kf}
					\arrow{u}[left]{\Symm}
					&
					\\
					\cSet
					\arrow{u}[left]{F}
					\arrow[bend right = 60]{uu}[right]{\kf[X_i \suchthat i \in (-)]}
					&
					{}
				\end{tikzcd}
			\]
			commutes up to natural isomorphism.
			
		\item[Contruction via the tensor algebra]
			The symmetric algebra~$\Symm(V)$ can also be constructed as a quotient of the tensor algebra~$\Tensor(V)$.
			We give multiple ways how to seethis and think about it.
			Let in the following~$i$ denote the inclusion map from~$V$ to~$\Tensor(V)$ and let~$j$ denote the inclusion map from~$V$ to~$\Symm(V)$.
			\begin{itemize}
				\item
					Let~$I$ be the commutator ideal\index{commutator ideal} of~$\Tensor(V)$, i.e. the two-sided ideal of~$\Tensor(V)$ generated by all elements of the form
					\[
						x y - y x
					\]
					where~$x$ and~$y$ are elements of~$\Tensor(V)$.
					Let~$\Pi$ denote the canonical projection homomorphism from~$\Tensor(V)$ onto~$\Tensor(V)/I$.
					The quotient algebra~$\Tensor(V)/I$ is commutative, whence there exists by the universal property of the symmetric algebra~$\Symm(V)$ a unique homomorphism of algebras~$\Phi$ from~$\Symm(V)$ to~$\Tensor(V)/I$ that makes the diagram
					\[
						\begin{tikzcd}[column sep = small]
							{}
							&
							V
							\arrow[bend right]{ddl}[above left]{i}
							\arrow[bend left]{dr}[above right]{j}
							&
							{}
							\\
							{}
							&
							{}
							&
							\Tensor(V)
							\arrow{d}[right]{\Pi}
							\\
							\Symm(V)
							\arrow[dashed]{rr}[above]{\Phi}
							&
							{}
							&
							\Tensor(V)/I
						\end{tikzcd}
					\]
					commute.
					The homomorphism~$\Phi$ is on the generating set~$V$ of~$\Symm(V)$ given by~$\Phi(v) = \class{v}$ for every~$v \in V$.

					We get on the other hand from the universal property of the tensor algebra~$\Tensor(V)$ a unique homomorphism ofalgebras~$\Psi'$ from~$\Tensor(V)$ to~$\Symm(V)$ that makes the diagram
					\[
						\begin{tikzcd}[column sep = small]
							{}
							&
							V
							\arrow[bend right]{dl}[above left]{j}
							\arrow[bend left]{ddr}[above right]{i}
							&
							{}
							\\
							\Tensor(V)
							\arrow[bend left=25, dashed]{drr}[above right, pos=0.35]{\Psi'}
							\arrow{d}[left]{\Pi}
							&
							{}
							&
							{}
							\\
							\Tensor(V)/I
							&
							{}
							&
							\Symm(V)
						\end{tikzcd}
					\]
					commute.
					The commutator ideal~$I$ is contained in the kernel of~$\Psi'$ because the algebra~$\Symm(V)$ is commutative.
					There hence exists a unique homomorphism of algebras~$\Psi$ from~$\Tensor(V)/I$ to~$\Symm(V)$ that makes the diagram
					\[
						\begin{tikzcd}[column sep = small]
							{}
							&
							V
							\arrow[bend right]{dl}[above left]{j}
							\arrow[bend left]{ddr}[above right]{i}
							&
							{}
							\\
							\Tensor(V)
							\arrow[bend left=25]{drr}[above right, pos=0.37]{\Psi'}
							\arrow{d}[left]{\Pi}
							&
							{}
							&
							{}
							\\
							\Tensor(V)/I
							\arrow[dashed]{rr}[above]{\Psi}
							&
							{}
							&
							\Symm(V)
						\end{tikzcd}
					\]
					commute.
					The homomorphism~$\Psi'$ is on the algebra generating set~$\{ \class{v} \suchthat v \in V \}$ of~$\Tensor(V) / I$ given by by~$\Psi(\class{v}) = v$ for every~$v \in V$.
					
					It follows from the explicit descriptions of~$\Phi$ and~$\Psi$ on algebra generatorthat these homomorphisms are mutually inverse isomorphisms of~\algebras{$\kf$}.
					Thus~$\Symm(V) \cong \Tensor(V)/I$ via the isomorphism~$f$.
					
					We note that the commutator ideal~$I$ is already generated by the commutators~$v \tensor w - w \tensor v$ where~$v$ and~$w$ range through~$V$.
					Indeed, let~$J$ be the ideal generated by these elements.
					Then on the one hand~$J$ is contained in~$I$.
					But on the other hand the quotient~$\Tensor(V)/J$ is already commutative because it is generated by the residue classes~$\class{v}$ with~$v$ in~$V$, and these generators commute.
					The commutator ideal~$I$ is therefore contained in the kernel of the canonical projection homomorphism from~$\Tensor(V)$ to~$\Tensor(V)/J$.
					But this kernel is the ideal~$J$, so~$I$ is contained in~$J$.
					
				\item
					The above argumentation is not surprising if we remember that the tensor algebra~$\Tensor(V)$ is the universal~{\algebra{$\kf$}} on~$V$ and that quotiening out the commutator ideal~$I$ is the universal way of making an algebra commutative.
					The quotient algebra~$\Tensor(V)/I$ therefore ought to be the universal commutative~{\algebra{$\kf$}}.
					
					This motivation can be formalized by observing that the diagram of forgetful functors
					\[
						\begin{tikzcd}
							\cCAlg{\kf}
							\arrow{d}
							\arrow[bend left = 60]{dd}
							\\
							\cAlg{\kf}
							\arrow{d}
							\\
							\cVect{\kf}
						\end{tikzcd}
					\]
					commutes.
					It follows that the resulting diagram of left adjoint functors\index{adjunction}
					\[
						\begin{tikzcd}
							\cCAlg{\kf}
							\\
							\cAlg{\kf}
							\arrow{u}[left]{C}
							\\
							\cVect{\kf}
							\arrow{u}[left]{\Tensor}
							\arrow[bend right = 60]{uu}[right]{\Symm}
						\end{tikzcd}
					\]
					commutes up to natural isomorphism.
					In this diagram, the adjoint fuctor~$C$ of the forgetful functor from~$\cCAlg{\kf}$ to~$\cAlg{\kf}$ is given by quotiening out the commutator ideal, i.e. by
					\[
						C(A)
						=
						A / \ideal{ ab - ba \suchthat a, b \in A }
					\]
					on objects.
					It follows that~$\Symm(V) \cong C(\Tensor(V)) = \Tensor(V)/I$.
				\item
					The above argumentation be also expressed via Yoneda’s lemma.
					Inddeed, for every commutative~\algebra{$\kf$}~$A$ there exist natural bijections
					\begin{align*}
						{}&
						\{ \textstyle\text{algebra homomorphisms~$\Symm(V) \to A$} \}
						\\
						\cong{}&
						\{ \textstyle\text{{\linear{$\kf$}} maps~$V \to A$} \}
						\\
						\cong{}&
						\{ \textstyle\text{algebra homomorphisms~$\Tensor(V) \to A$} \}
						\\
						\cong{}&
						\{ \textstyle\text{algebra homomorphisms~$\Tensor(V)/I \to A$} \} \,,
					\end{align*}
					It now follows from Yoneda’s~lemma that~$\Symm(V) \cong \Tensor(V)/I$.
			\end{itemize}
	\end{description}
\end{recall}


\begin{remark}
	One can similarly construct the \defemph{exterior algebra}~$\Exterior(V) = \bigoplus_{d \geq 0} \Exterior^d(V)$\glsadd{exterior algebra}\index{exterior algebra} of a vector space~$V$ by using the exterior powers~$\Exterior^d(V)$\glsadd{exterior power}\index{exterior power} instead of the the tensor powers~$V^{\otimes d}$ or symmetric powers~$\Symm^d(V)$.
	If~$A$ is another~\algebra{$\kf$}, then a homomorphism of algebras~$\Phi$ from~$\Exterior(V)$ to~$A$ is \enquote{the same} as a~\linear{$\kf$} map~$f$ from~$V$ to~$A$ such that~$f(v)^2 = 0$ for ever~$v \in V$.
% TODO: Do we have to worry about char(k) = 2?
	It thus follows from a similar argumentation as for the symmetric algebra that~$\Exterior(V)$ is isomorphic to the quotient algebra~$\Tensor(V)/I$ where~$I$ is the two-sided ideal in~$\Tensor(V)$ generated by the elements~$v \tensor v$ where~$v$ ranges through~$V$.
	
	If the vector space~$V$ is finite-dimensional, then the exterior algebra~$\Exterior(V)$ is again finite-dimensional.
	Its dimension is given by~$\dim( \Exterior(V) ) = 2^{\dim(V)}$.
	This behavior is different to both that of the tensor algebra~$\Tensor(V)$ and that of the symmetric algebra~$\Symm(V)$, which are infinite-dimensional whenever the vector space~$V$ is nonzero.
\end{remark}





