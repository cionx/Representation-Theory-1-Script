\section{Special Kinds of Extensions}


\begin{fluff}
	We will now classify certain kinds of extensions up to equivalence.
	For this we will show that the Lie~algebras in question are equivalent to standard extensions given by~$(\kappa, \theta)$ such that~$\kappa$ and~$\theta$ have to satisfy certain additional conditions.
	We then express under what conditions two such pairs~$(\kappa_1, \theta_1)$ and~$(\kappa_2, \theta_2)$ give equivalent conditions.
\end{fluff}



\subsection{Trivial Extensions}


\begin{example}
	\label{construction of trivial extension}
	Let~$\glie$ and~$\hlie$ be two~\liealgebras{$\kf$}.
	We consider the vector space~$\glie \oplus \hlie$ together with the componentwise Lie~bracket
	\[
		[ (x,c), (y,d) ]
		=
		( [x,y], [c,d] )
	\]
	for all~$(x,c), (y,d) \in \glie \oplus \hlie$.
	This makes the vector space~$\glie \oplus \hlie$ into a standard extension of~$\glie$ by~$\hlie$.
\end{example}


\begin{definition}
	Let~$\glie$ and~$\hlie$ be two~\liealgebras{$\kf$}.
	\begin{enumerate}
		\item
			The extension of~$\glie$ by~$\hlie$ from \cref{construction of trivial extension} is the \defemph{trivial extension}\index{trivial extension of Lie algebras}\index{extension of Lie algebras!trivial} of~$\glie$ by~$\hlie$.
		\item
			An arbitrary extension of~$\glie$ by~$\hlie$ is \defemph{trivial}\index{trivial extension of Lie algebras}\index{extension of Lie algebras!trivial} if it is equivalent to the trivial extension.
	\end{enumerate}
\end{definition}


\begin{remark}
	\leavevmode
	\begin{enumerate}
		\item
			We have choosen our definitions so that \emph{the} trivial extension is one specific extension, whereas \emph{being trivial} is a property of an extension.
		\item
			Let~$\glie$ and~$\hlie$ be two~\liealgebras{$\kf$}.
			Under the correspondence from \cref{structure of extensions} the trivial extension of~$\glie$ by~$\hlie$ corresponds to the pair~$(\kappa, \theta)$ with~$\kappa = 0$ and~$\theta = 0$.
	\end{enumerate}
\end{remark}


\begin{definition}
	Let
	\[
		0
		\to
		\hlie
		\xto{\varphi}
		\elie
		\xto{\psi}
		\glie
		\to
		0
	\]
	be an extension of Lie~algebras.
	A \defemph{retract}\index{retract}\index{short exact sequence of Lie algebras!retract} of this extension is a homomorphism of Lie~algebras~$\rho$ from~$\elie$ to~$\hlie$ such that
	\[
		\rho \circ \varphi
		=
		\id_{\hlie} \,.
	\]
\end{definition}


\begin{lemma}
	\label{having a retract is invariant under equivalence of extensions}
	Let~$\glie$ and~$\hlie$ be two~\liealgebras{$\kf$} and let~$\elie$ be an extension of~$\glie$ by~$\hlie$ that admits a rectract.
	Then every extension equivalent to~$\elie$ also admits a retract.
\end{lemma}


\begin{proof}
	The extension~$\elie$ is given by a short exact sequence of Lie~algebras
	\[
		0
		\to
		\hlie
		\xto{\varphi}
		\elie
		\xto{\psi}
		\glie
		\to
		0 \,.
	\]
	Let
	\[
		0
		\to
		\hlie
		\xto{\varphi'}
		\elie'
		\xto{\psi'}
		\glie
		\to
		0
	\]
	be an equivalent extension.
	This means that there exists a homorphism of Lie~algebras~$\alpha$ from~$\elie$ to~$\elie'$ that makes the diagram
	\[
		\begin{tikzcd}
			0
			\arrow{r}
			&
			\hlie
			\arrow{r}[above]{\varphi}
			\arrow[equal]{d}
			&
			\elie
			\arrow{r}[above]{\psi}
			\arrow[dashed]{d}[right]{\alpha}
			&
			\glie
			\arrow{r}
			\arrow[equal]{d}
			&
			0
			\\
			0
			\arrow{r}
			&
			\hlie
			\arrow{r}[above]{\varphi'}
			&
			\elie'
			\arrow{r}[above]{\psi'}
			&
			\glie
			\arrow{r}
			&
			0
		\end{tikzcd}
	\]
	commute.
	If~$\rho$ is a retract for the second extension then the composite~$\rho \circ \alpha$ will be a retract for the first extension.
	Indeed, we have~$\rho \circ \varphi' = \id_{\hlie}$ and therefore
	\[
		\rho \circ \alpha \circ \varphi
		=
		\rho \circ \varphi'
		=
		\id_{\hlie}
	\]
	by the commutativity of the above diagram.
	This shows that the first extension admits a retract if the second one does.
	The reverse implication follows from this first implication because equivalence of extensions is an equivalence relation.
\end{proof}


\begin{proposition}[Characterizations of trivial extensions]
	\index{characterization of trivial extensions}
	\index{trivial extension of Lie algebras}
	\index{extension of Lie algebras!trivial}
	For an extension of Lie algebras
	\[
		0
		\to
		\hlie
		\xto{\varphi}
		\elie
		\xto{\psi}
		\glie
		\to
		0
	\]
	the following conditions are equivalent.
	\begin{equivalenceslist*}
		\item
			\label{extension is trivial}
			The extension is trivial.
		\item
			\label{extension is equivalent to 0 0}
			The extension is equivalent to the standard extension~$(\kappa, \theta)$ with~$\kappa = 0$ and~$\theta = 0$.
		\item
			\label{extension admits a retract}
			The extension admits a retract.
		\item
			\label{image of h admits a direct complement}
			There exists an ideal~$J$ of~$\elie$ such that~$\elie$ is the (internal) direct sum of the two ideals~$\varphi(\hlie)$ and~$J$.
	\end{equivalenceslist*}
%  Suppose furthermore that the underlying vector space of~$\elie$ is given by the direct sum~$\glie \oplus \hlie$.
%  Let~$(\kappa, \theta)$ be the pair correspondung to the Lie~bracket of~$\elie$ under the bijection from \cref{structure of extensions}.
%  Then the following condition is also equivalent to the ones above.
%  \begin{equivalenceslist*}[resume*]
%    \item
%      \label{trivial via kappa and theta}
%      There exists a linear map~$\varphi$ from~$\glie$ to~$\hlie$ such that
%      \[
%        \theta(x) = \ad_{\hlie}(\varphi(x)) \,,
%        \quad
%        \kappa(x,y) = [\varphi(x), \varphi(y)] - \varphi([x,y])
%      \]
%      for all~$x, y \in \glie$.
%      % TODO: Give an explicit explanation of these conditions.
%      % The first condition means that \theta takes values in \ad(\hlie), the Lie algebra of inner derivations
%      % The second condition means that \kappa is the coboundary associated to \varphi.
%      % But what about the interplay between these conditions?
%  \end{equivalenceslist*}
\end{proposition}

\begin{proof}
	We denote the trivial representation of~$\glie$ by~$\hlie$ as~$\glie \oplus \hlie$.
	\begin{implicationlist}
		\item[\ref{extension is trivial}~$\iff$~\ref{extension is equivalent to 0 0}]
			The standard extension given by~$(0,0)$ is precisely the trivial extension.
			This equivalence is therefore just a reformulation of the definition of a trivial extension.
		\item[\ref{extension is trivial}~$\implies$~\ref{extension admits a retract}]
			The trivial extension admits a retract, namly the canonical projection map from~$\glie \oplus \hlie$ to~$\hlie$, given by~$(x,c) \mapsto c$.
			It follows from \cref{having a retract is invariant under equivalence of extensions} that every trivial representation admits a retract.
		\item[\ref{extension admits a retract}~$\implies$~\ref{image of h admits a direct complement}]
			Let~$\rho \colon \elie \to \hlie$ be a retract of the given extension.
			We consider for the ideal~$J$ the kernel of~$\rho$.
			The composite~$\varphi \circ \rho$ is an idempotent endomorphism of~$\elie$, whence~$\elie$ is the direct sum of the kernel and the image of~$\varphi \circ \rho$.
			It follows from the condition~$\rho \circ \varphi = \id_{\hlie}$ that~$\rho$ is surjective and~$\varphi$ is injective, so
			\[
				\ker(\varphi \circ \rho) = \ker(\rho) = J \,,
				\qquad
				\im(\varphi \circ \rho) = \im(\varphi) = \varphi(\hlie) \,.
			\]
			We have thus shown that~$\elie$ is the direct sum of~$\varphi(\hlie)$ and~$J$.
		\item[\ref{image of h admits a direct complement}~$\implies$~\ref{extension is trivial}]
			It also follows from the injectivity of~$\varphi$ that~$\varphi$ restricts to an isomorphism of Lie~algebras from~$\hlie$ to~$\varphi(\hlie)$.
			We denote the restriction by~$\varphi'$.
			We similarly find that the homomorphism~$\psi$ restricts to an isomorphism~$\psi'$ from~$J$ to~$\glie$.
			This follows from the fact that~$\psi$ is surjective and~$J$ is a precisely a direct complement to~$\varphi(\hlie)$, and thus a direct complement to the kernel of~$\psi$.

			It follows from \cref{direct sum of ideals} that the map
			\[
				\alpha
				\colon
				J \oplus \varphi(\hlie)
				\to
				\elie \,,
				\quad
				(x, y)
				\mapsto
				x + y
			\]
			is an isomorphism of Lie~algebras.
			This isomorphism makes the diagram
			\[
				\begin{tikzcd}[sep = large]
					0
					\arrow{r}
					&
					\varphi(\hlie)
					\arrow{r}[above]{\iota}
					\arrow{d}[right]{(\varphi')^{-1}}
					&
					J \oplus \varphi(\hlie)
					\arrow{r}[above]{\pi}
					\arrow[dashed]{d}[right]{\alpha}
					&
					J
					\arrow{r}
					\arrow{d}[right]{\psi'}
					&
					0
					\\
					0
					\arrow{r}
					&
					\hlie
					\arrow{r}[above]{\varphi}
					&
					\elie
					\arrow{r}[above]{\psi}
					&
					\glie
					\arrow{r}
					&
					0
				\end{tikzcd}
			\]
			commute, where~$\iota$ denotes the inclusion map and~$\pi$ denotes the canonical projection map.

			We also find that the map
			\[
				\beta
				\colon
				\glie \oplus \hlie
				\to
				J \oplus \varphi(\hlie) \,,
				\quad
				(x,y)
				\mapsto
				\bigl( (\psi')^{-1}(x) , \varphi'(y) \bigr)
			\]
			is again an isomorphism of Lie~algebras.
			This isomorphism makes the diagram
			\[
				\begin{tikzcd}[sep = large]
					0
					\arrow{r}
					&
					\hlie
					\arrow{r}[above]{\iota}
					\arrow{d}[right]{\varphi'}
					&
					\glie \oplus \hlie
					\arrow{r}[above]{\pi}
					\arrow[dashed]{d}[right]{\beta}
					&
					\glie
					\arrow{r}
					\arrow{d}[right]{(\psi')^{-1}}
					&
					0
					\\
					0
					\arrow{r}
					&
					\varphi(\hlie)
					\arrow{r}[above]{\iota}
					&
					J \oplus \varphi(\hlie)
					\arrow{r}[above]{\pi}
					&
					J
					\arrow{r}
					&
					0
				\end{tikzcd}
			\]
			commute, where~$\iota$ denotes the respective inclusion maps, and~$\pi$ the respective canonical projection maps.

			We have altogether the following commutative diagram:
			\[
				\begin{tikzcd}[sep = large]
					0
					\arrow{r}
					&
					\hlie
					\arrow{r}[above]{\iota}
					\arrow{d}[right]{\varphi'}
					&
					\glie \oplus \hlie
					\arrow{r}[above]{\pi}
					\arrow[dashed]{d}[right]{\beta}
					&
					\glie
					\arrow{r}
					\arrow{d}[right]{(\psi')^{-1}}
					&
					0
					\\
					0
					\arrow{r}
					&
					\varphi(\hlie)
					\arrow{r}[above]{\iota}
					\arrow{d}[right]{(\varphi')^{-1}}
					&
					J \oplus \varphi(\hlie)
					\arrow{r}[above]{\pi}
					\arrow{d}[right]{\alpha}
					&
					J
					\arrow{r}
					\arrow{d}[right]{\psi'}
					&
					0
					\\
					0
					\arrow{r}
					&
					\hlie
					\arrow{r}[above]{\varphi}
					&
					\elie
					\arrow{r}[above]{\psi}
					&
					\glie
					\arrow{r}
					&
					0
				\end{tikzcd}
			\]
			By deleting the middle row of this commutative diagram we arrive at the following commutative diagram:
			\[
				\begin{tikzcd}[sep = large]
					0
					\arrow{r}
					&
					\hlie
					\arrow{r}[above]{\iota}
					\arrow[equal]{d}
					&
					\glie \oplus \hlie
					\arrow{r}[above]{\pi}
					\arrow[dashed]{d}[right]{\alpha \circ \beta}
					&
					\glie
					\arrow{r}
					\arrow[equal]{d}
					&
					0
					\\
					0
					\arrow{r}
					&
					\hlie
					\arrow{r}[above]{\varphi}
					&
					\elie
					\arrow{r}[above]{\psi}
					&
					\glie
					\arrow{r}
					&
					0
				\end{tikzcd}
			\]
			This shows that the homomorphismp~$\alpha \circ \beta$ gives an equivalence between the trivial extension~$\glie \oplus \hlie$ and the given extension.
%    \item[\ref{extension is trivial}~$\iff$~\ref{trivial via kappa and theta}]
%      Under the correspondence from \cref{structure of extensions} the trivial representation of corresponds to the pair~$(0,0)$.
%      The claim thus follows from \cref{structure of extensions}.
		\qedhere
	\end{implicationlist}
\end{proof}

\begin{example}
	We have a short exact sequence of Lie~algebras given by
	\[
		0
		\to
		\sllie(n, \kf)
		\xto{\iota}
		\gllie(n, \kf)
		\xto{\tr}
		\kf
		\to
		0
	\]
	where~$\iota$ denotes the inclusion map.
	The general linear Lie~algebra~$\gllie(n, \kf)$ is thus an extension of the {\onedimensional} Lie~algebra~$\kf$ by the special linear Lie~algebra~$\sllie(n, \kf)$\index{special linear Lie algebra}.
	If the characteristic of~$\kf$ does not divide the size~$n$ then the one-dimensional ideal~$\gen{ \Id }_{\kf}$ of~$\gllie(n, \kf)$ is a direct complement of~$\sllie(n, \kf)$.
	We find in this case that the above extension is trivial.
	% TODO: What happens if the characteristic does divide n?
	% This problem should follow from a classification of ideals in gl, which should be done after root space decompositions have been introduced.
\end{example}

\begin{fluff}
	Let~$\glie$ and~$\hlie$ be two~\liealgebras{$\kf$}.
	It may happen that a standard extension~$(\kappa, \theta)$ is trivial even if neither~~$\kappa$ nor~$\theta$ vanishes.
	Indeed, the standard extension~$(\kappa, \theta)$ is trivial if and only if it is equivalent to the standard extension~$(0,0)$.
	According to \cref{structure of extensions} this happens if and only if there exists a linear map~$\varphi$ from~$\glie$ to~$\Der(\hlie)$ such that
	\[
		\theta(x) = \ad_{\hlie}(\varphi(x)) \,,
		\qquad
		\kappa(x,y) = [\varphi(x), \varphi(y)] - \varphi([x,y])
	\]
	for all~$x, y \in \glie$.

	Indeed, we observe that for any linear map~$\theta$ from~$\hlie$ to~$\Der(\hlie)$ we can define maps~$\kappa$ from~$\glie \times \glie$ to~$\hlie$ and~$\theta$ from~$\glie$ to~$\Der(\hlie)$ via the above two formulas.
	The map~$\kappa$ is then bilinear and the map~$\theta$ is linear.
	Moreover, the maps~$\kappa$ and~$\theta$ satisfy the compatibility conditions from \cref{structure of extensions} because
	\begin{gather*}
		\SwapAboveDisplaySkip
		\begin{aligned}
			{}&
			\kappa([x,y], z) + \kappa([y,z], x) + \kappa([z,x], y)
			\\
			={}&
			[\varphi([x,y]), \varphi(z)] - \varphi([[x,y],z])
			\\
			{}&
			+ [\varphi([y,z]), \varphi(x)] - \varphi([[y,z],x])
			\\
			{}&
			+ [\varphi([z,x]), \varphi(y)] - \varphi([[z,x],y])
			\\
			={}&
			[\varphi([x,y]), \varphi(z)]
			+ [\varphi([y,z]), \varphi(x)]
			+ [\varphi([z,x]), \varphi(y)]
			\\
			{}&
			- \varphi( [[x,y],z] + [[y,z],x] + [[z,x],y] )
			\\
			={}&
			[\varphi([x,y]), \varphi(z)]
			+ [\varphi([y,z]), \varphi(x)]
			+ [\varphi([z,x]), \varphi(y)]
		\end{aligned}
	\shortintertext{and}
		\begin{aligned}
			{}&
			\theta(x)(\kappa(y,z))
			+ \theta(y)(\kappa(z,x))
			+ \theta(z)(\kappa(x,y))
			\\
			={}&
			[ \varphi(x), [\varphi(y), \varphi(z)] - \varphi([y,z]) ]
			\\
			{}&
			+ [ \varphi(y), [\varphi(z), \varphi(x)] - \varphi([z,x]) ]
			\\
			{}&
			+ [ \varphi(z), [\varphi(x), \varphi(y)] - \varphi([x,y]) ]
			\\
			={}&
			[ \varphi(x), [\varphi(y), \varphi(z)] ]
			+ [ \varphi(y), [\varphi(z), \varphi(x)] ]
			+ [ \varphi(z), [\varphi(x), \varphi(y)] ]
			\\
			{}&
			- [ \varphi(x), \varphi([y,z]) ]
			- [ \varphi(y), \varphi([z,x]) ]
			- [ \varphi(z), \varphi([x,y]) ]
			\\
			={}&
			- [ \varphi(x), \varphi([y,z]) ]
			- [ \varphi(y), \varphi([z,x]) ]
			- [ \varphi(z), \varphi([x,y]) ]
			\\
			={}&
			[ \varphi([x,y]), \varphi(z) ]
			+ [ \varphi([y,z]), \varphi(x) ]
			+ [ \varphi([z,x]), \varphi(y) ] \,,
		\end{aligned}
	\end{gather*}
	as well as
	\begin{align*}
		\SwapAboveDisplaySkip
		{}&
		[\theta(x), \theta(y)] - \ad_{\hlie}( \kappa(x,y) )
		\\
		={}&
		[ \ad_{\hlie}(\varphi(x)), \ad_{\hlie}(\varphi(y)) ] - \ad_{\hlie}( [\varphi(x), \varphi(y)] - \varphi([x,y]) )
		\\
		={}&
		\ad_{\hlie}( [\varphi(x), \varphi(y)] ) - \ad_{\hlie}( [\varphi(x), \varphi(y)] ) + \ad_{\hlie}( \varphi([x,y]) )
		\\
		={}&
		\ad_{\hlie}( \varphi([x,y]) )
		\\
		={}&
		\theta([x,y])
	\end{align*}
	for all~$x, y, z \in \glie$.
	Hence every linear map~$\varphi$ from~$\glie$ to~$\hlie$ gives rise to a trivial standard extension.
\end{fluff}



\subsection{Abelian Extensions}

\begin{definition}
	An extension of Lie~algebras
	\[
		0 \to \hlie \to \elie \to \glie \to 0
	\]
	is \defemph{abelian}\index{abelian extension of Lie algebras}\index{extension of Lie algebras!abelian} if the Lie~algebra~$\hlie$ is abelian.
\end{definition}


\begin{fluff}
	\label{discussing abelian extensions}
	Let~$\glie$ be a~\liealgebra{$\kf$}.
	We denote by~$\AbEx(\glie)$\glsadd{abelian extensions} the class of abelian extensions of~$\glie$.
	To better understand the equivalence classes of abelian extensions we will split up the class~$\mathcal{A}$ into pairwise disjoint subclasses, such that no two equivalent abelian extensions are contained in the same subclass.

	We denote for every abelian~\liealgebra{$\kf$}~$\hlie$ by~$\AbEx(\glie, \hlie)$\glsadd{abelian extensions by h} the class of abelian extensions of~$\glie$ by~$\hlie$.
	This gives a disjoint decomposition
	\[
		\AbEx(\glie)
		=
		\coprod_{\hlie}
		\AbEx(\glie, \hlie)
	\]
	where~$\hlie$ runs through the class of abelian~\liealgebras{$\kf$}.
	There exist no equivalences of extensions between an extension in~$\AbEx(\glie, \hlie)$ and an extension in~$\AbEx(\glie, \klie)$ whenever~$\hlie$ and~$\klie$ are distinct.

	Let~$\hlie$ be an abelian~\liealgebra{$\kf$}.
	We want to further decompose the class~$\AbEx(\glie, \hlie)$.
	Let for this
	\[
		0 \to \hlie \to \elie \to \glie \to 0
	\]
	be an extension of Lie~algebras.
	We know that this extension is equivalent to some standard extensions~$(\kappa, \theta)$.
	The parameter~$\theta$ is a linear map from~$\glie$ to~$\Der(\hlie)$ such that
	\[
		\theta( [x,y] )
		=
		[ \theta(x), \theta(y) ] + \ad_{\hlie}( \kappa(x,y) )
	\]
	for all~$x, y \in \elie$.
	But the Lie~algebra~$\hlie$ is abelian, so the term~$\ad_{\hlie}( \kappa(x,y) )$ vanishes.
	We thus find that~$\theta$ needs to be a homomorphism of Lie algebras from~$\glie$ to~$\Der(\hlie)$.
	Moreover, since~$\hlie$ is abelian, we have~$\Der(\hlie) = \gllie(\hlie)$.
	We thus find that~$\theta$ is a homomorphism of Lie~algebras from~$\glie$ to~$\gllie(\hlie)$.

	If the given extension is equivalent to another standard extension~$(\kappa', \theta')$, then it follows from \cref{structure of extensions} that~$\theta = \theta' + \ad_{\hlie}(\varphi(\ph))$ for some linear map~$\varphi$ from~$\glie$ to~$\hlie$.
	But we then have~$\ad_{\hlie}(\varphi(\ph)) = 0$ because~$\hlie$ is abelian, and therefore~$\theta = \theta'$.
	This shows that the homomorphism~$\theta$ is unique.

	We find from the above that we can further decompose the class~$\AbEx(\glie, \hlie)$ as
	\[
		\AbEx(\glie, \hlie)
		=
		\coprod_{\theta}
		\AbEx(\glie, \hlie, \theta)
		\glsadd{abelian extensions by h via theta}
	\]
	where~$\theta$ ranges through the set of Lie~algebra homomorphisms from~$\glie$ to~$\gllie(\hlie)$.
	We thus have the disjoint decomposition
	\[
		\AbEx(\glie)
		=
		\coprod_{\hlie, \theta}
		\AbEx(\glie, \hlie, \theta) \,.
	\]

	We can reparametrize this decomposition.
	For this we note that a pair~$(\hlie, \theta)$ consisting of an abelian Lie~algebra~$\hlie$ and a homomorphism of Lie~algebras~$\theta$ from~$\glie$ to~$\hlie$ is the same as a representation~$M$ of~$\glie$, in the following way.
	\index{representation!associated to an abelian extension}
	Given on the one hand~$\hlie$ and~$\theta$ as above, the pair~$(\hlie, \theta)$ is a representation of~$\glie$, with the action of~$\glie$ on~$\hlie$ given by~$x \act c = \theta(x)(c)$ for all~$x \in \glie$ and~$c \in \hlie$.
	Given on the other hand a representation~$(M, \rho)$ of~$\glie$ we regard~$M$ as an abelian Lie~algebra and let~$\theta$ be~$\rho$.
	We have thus a decomposition
	\[
		\AbEx(\glie)
		=
		\coprod_M \AbEx(\glie, M)
		\glsadd{abelian extensions associated to M}
	\]
	where~$M$ runs through the class of representations of~$\glie$.

	To determine in which subclass~$\AbEx(\glie, M)$ a given abelian extension
	\[
		0
		\to
		\hlie
		\xto{\varphi}
		\elie
		\xto{\psi}
		\glie
		\to
		0
	\]
	is contained we proceed as follows.
	We identify the abelian Lie~algebra~$\hlie$ with an abelian ideal of~$\elie$ via the homomorphism~$\varphi$.
	For an element~$x$ of~$\elie$ let~$y$ be a preimage of~$x$ under~$\psi$.
	We define an action of~$x$ act on~$\hlie$ via~$x \act c \defined [y, c]$.
	This action is well-defined:
	if~$y'$ is another preimage of~$x$ under~$\psi$ then the difference~$y - y'$ is contained in the kernel of~$\psi$, i.e. in~$\hlie$.
	Then
	\[
		[y, c]
		=
		[y' + y - y', c]
		=
		[y', c] + [y - y', c]
		=
		[y', c] + 0
		=
		[y', c]
	\]
	for all~$c \in \hlie$ because~$\hlie$ is abelian.
	This action of~$\glie$ on~$\hlie$ makes~$\hlie$ into a representation of~$\glie$.
	This is precisely the desired representation~$M$ for which the given extension is contained in~$\AbEx(\glie, M)$.

	Let us summarize our results.
	To every abelian extension of~$\glie$ we have associated a representation~$M$ of~$\mathfrak{h}$.
	For an abelian extension
	\[
		0
		\to
		\hlie
		\to
		\elie
		\to
		\glie
		\to
		0
	\]
	the associated representation~$M$ is given by~$\hlie$ as a vector space, and the action of an element~$x$ of~$\glie$ on~$M$ is given by taking a preimage~$y$ of~$x$ in~$\elie$ and acting with~$y$ on~$\hlie$ via the adjoint action of~$\elie$.
	If two abelian extensions of~$\glie$ are equivalent then the resulting representations of~$\glie$ are equal.
\end{fluff}


\begin{theorem}[Structure of abelian extensions]
	\index{structure theorem!for abelian extensions}
	\index{abelian extension of Lie algebras!structure theorem}
	\label{structure of abelian extensions}
	Let~$\glie$ be a Lie algebra.
	\begin{enumerate}
		\item
			Suppose that
			\[
				0
				\to
				\hlie
				\to
				\elie
				\to
				\glie
				\to
				0
			\]
			is an abelian extension of~$\glie$.
			For every element~$x$ of~$\glie$ let~$y$ be a preimage of~$x$ in~$\elie$, and define an action of~$x$ on~$\hlie$ via~$x \act c = [y,c]$ for all~$c \in \hlie$.
			This action makes~$\hlie$ into a representation of~$\glie$.
		\item
			If two abelian extensions of~$\glie$ are equivalent then their associated representations are equal.
		\item
			Let~$\AbEx(\glie)$ be the class of abelian extensions of~$\glie$.
			For every representation~$M$ of~$\glie$ let~$\AbEx(\glie, M)$ be the subclass of~$\AbEx(\glie)$ consisting of all those abelian extensions of~$\elie$ whose associated representation is~$M$.
			Then
			\[
				\AbEx(\glie)
				=
				\coprod_M \AbEx(\glie, M)
			\]
			where~$M$ ranges through the class of representations of~$\glie$.
	\end{enumerate}
	Let now~$M$ be a representation of~$\glie$.
	Let~$\theta$ be the corresponding homomorphism of Lie~algebras from~$\glie$ to~$\gllie(M)$.
	We regard~$M$ as an abelian Lie~algebra, which we denote by~$\hlie$, and regard~$\theta$ as a homomorphism of Lie~algebras from~$\glie$ to~$\Der(\hlie)$.
	\begin{enumerate}[resume*]
		\item
			The standard extensions contained in~$\AbEx(\glie, M)$ are precisely those standard extensions of~$\glie$ by~$\hlie$ of the form~$(\kappa, \theta)$ with~$\theta$ as above.
		\item
			An alternating, bilinear map~$\kappa$ from~$\glie \times \glie$ to~$\hlie$ defines a standard extension~$(\kappa, \theta)$ of~$\glie$ by~$\hlie$, which is then necessarily contained in~$\AbEx(\glie, M)$, if and only if it satisfies the condition
			\[
				\kappa( [x,y], z ) + \kappa( [y,z], x ) + \kappa( [z,x], y )
				=
				x \act \kappa(y,z) + y \act \kappa(z,x) + z \act \kappa(x,y)
			\]
			for all~$x, y, z \in \glie$.%
			\footnote{
				We want to point out that the right hand side of this equation depends on~$\theta$.
			}
		\item
			Two such maps~$\kappa_1$ and~$\kappa_2$ result in equivalent standard extensions~$(\kappa_1, \theta)$ and~$(\kappa_2, \theta)$ of~$\glie$ by~$\hlie$ if and only if there exists a linear map~$\varphi$ from~$\glie$ to~$\hlie$ such that
			\[
				\kappa_1(x,y)
				- \kappa_2(x,y)
				=
				x \act \varphi(y)
				- y \act \varphi(x)
				- \varphi( [x,y] )
			\]
			for all~$x, y \in \glie$.
	\end{enumerate}
\end{theorem}


\begin{proof}
	This follows from \cref{structure of extensions} and the discussion in \cref{discussing abelian extensions}.
\end{proof}



\subsection{Central Extensions}

\begin{definition}
	An extension of Lie~algebras
	\[
		0 \to \hlie \to \elie \to \glie \to 0
	\]
	is \defemph{central}\index{central extension of Lie algebras}\index{extension of Lie algebras!central} if the image of~$\hlie$ in~$\elie$ is contained in the center of~$\elie$.
\end{definition}


\begin{remark}
	Suppose that
	\[
		0 \to \hlie \to \elie \to \glie \to 0
	\]
	is a central extensions of Lie~algebras.
	Then~$\hlie$ is isomorphic to its image in~$\elie$, and thus abelian.
	This means that every central extension is in particular an abelian extension.
	It therefore only makes sense to talk about central extensions of a Lie~algebras~$\glie$ by a Lie~algebra~$\hlie$ if the Lie~algebra~$\hlie$ is abelian, as otherwise there are now such extensions.
\end{remark}


\begin{proposition}
	\label{central iff theta is zero}
	Let~$\glie$ and~$\hlie$ be two~\liealgebras{$\kf$} with~$\hlie$ being abelian.
	A standard extension~$(\kappa, \theta)$ of~$\glie$ by~$\hlie$ is central if and only if~$\theta$ vanishes.
\end{proposition}


\begin{proof}
	The Lie~bracket~$[\ph, \ph]$ of the standard extension~$(\kappa, \theta)$ is given by
	\[
		[ (x,c), (y,d) ]
		=
		( [x,y], \kappa(x,y) + \theta(x)(d) - \theta(y)(c) + [c,d] )
	\]
	for all~$(x,c), (y,d) \in \glie \oplus \hlie$.
	We have~$[c,d] = 0$ for all~$c, d \in \hlie$ because the Lie~algebra~$\hlie$ is abelian.
	Thus
	\[
		[ (x,c), (0,d) ]
		=
		( 0, \theta(x)(d) )
	\]
	for all~$(x,c) \in \glie \oplus \hlie$ and~$d \in \hlie$.
	It follows that the image of~$\hlie$ in~$\glie \oplus \hlie$ is central if and only if
	\[
		\theta(x)(d) = 0
		\qquad
		\text{for all~$x \in \glie$ and~$d \in \hlie$,}
	\]
	i.e. if and only if~$\theta = 0$.
\end{proof}



\begin{theorem}[Structure of central extensions]
	\index{structure theorem!for central extensions}
	\index{central extension of Lie algebras!structure theorem}
	\label{structure of central extensions}
	Let~$\glie$ and~$\hlie$ be two~\liealgebras{$\kf$} where~$\hlie$ is abelian.
	\begin{enumerate}
		\item
			Every central extension of~$\glie$ by~$\hlie$ is equivalent to a central, standard extension of~$\glie$ by~$\hlie$.
		\item
			We have a {\onetoonetext} correspondence
			\[
				\left\{
					\begin{tabular}{c}
						central standard \\
						extensions of~$\glie$ by~$\hlie$
					\end{tabular}
				\right\}
				\onetoone
				\left\{
					\begin{tabular}{c}
						bilinear, alternating maps \\
						$\kappa \colon \glie \times \glie \to \hlie$ such that \\
						$\kappa([x,y], z) + \kappa([y,z], x) + \kappa([z, x], y) = 0$
					\end{tabular}
				\right\}
			\]
			where the extension corresponding to a map~$\kappa$ is the standard extension~$(\kappa, 0)$.
		\item
			The Lie~bracket on the extension~$\elie$ corresponding to~$\kappa$ is given by
			\[
				[ (x,c), (y,d) ]
				=
				( [x,y], \kappa(x,y) )
			\]
			for all~$(x,c), (y,d) \in \elie$.
		\item
			Let
			\[
				\kappa_1, \kappa_2
				\colon
				\glie \times \glie
				\to
				\hlie
			\]
			be two maps as above, and let~$\elie_1$ and~$\elie_2$ be the correspond central extensions.
			Then we have a {\onetoonetext} correspondence given by
			\[
				\left\{
					\begin{tabular}{c}
						equivalences \\ of extensions \\
						$\alpha \colon \elie_1 \to \elie_2$
					\end{tabular}
				\right\}
				\onetoone
				\left\{
					\begin{tabular}{c}
						linear maps~$\varphi \colon \glie \to \hlie$ with \\
						$\kappa_1(x,y) = \kappa_2(x,y) - \varphi([x,y])$ \\
						for all~$x, y \in \glie$
					\end{tabular}
				\right\} \,.
			\]
			It follows in particular that the two extensions~$\elie_1$ and~$\elie_2$ are equivalent if and only if there exists a linear map~$\varphi$ from~$\glie$ to~$\hlie$ with
			\[
				\kappa_1(x,y)
				=
				\kappa_2(x,y)
				-
				\varphi([x,y])
			\]
			for all~$x, y \in \glie$.
	\end{enumerate}
\end{theorem}


\begin{proof}
	The \lcnamecref{structure of central extensions} follows from \cref{structure of extensions} thanks to \cref{central iff theta is zero}.
\end{proof}


\begin{remark}
	Let~$\glie$ be a Lie~algebra.
	Under the decomposition~$\AbEx(\glie) = \coprod_M \AbEx(\glie, M)$ from \cref{structure of abelian extensions}, the subclass of central extensions on the left hand side corresponds on the right hand side to~$\coprod_{M'} \AbEx(\glie, M')$ where~$M'$ ranges through the class of trivial representations of~$\glie$.
\end{remark}


%\begin{remark}
%  One can also think about central extenions, and our above structure theorem about central extensions, in a homological way.
%  We do so as follows:
%  \begin{enumerate}
%    \item
%      Let~$\elie$ be a Lie~algebra.
%      We can form a chain complex~$\Exterior^\bullet(\elie)$ whose~\howmanyth{$n$} term is given by the exterior power~$\Exterior^n(\elie)$ for every~$n \geq 0$, and which vanishes in negative degrees.
%      The differential of this chain complex is given by the maps
%      \begin{align*}
%        d_n
%        \colon
%        \bigwedge^n(\elie)
%        &\to
%        \bigwedge^{n-1}(\elie) \,,
%        \\
%        x_1 \wedge \dotsb \wedge x_n
%        &\mapsto
%        \sum_{1 \leq i < j \leq n}
%        (-1)^{i+j-1}
%        [x_i, x_j] \wedge x_1 \wedge \dotsb \wedge \widehat{x_i} \wedge \dotsb \wedge \widehat{x_j} \wedge \dotsb \wedge x_n
%      \end{align*}
%      for every~$n \geq 1$, and~$d_n \defined 0$ for every~$n \leq 0$.
%  \end{enumerate}
%\end{remark}

%\begin{example}[Central extensions]
%  In the notation of \cref{general approach to extensions} we find that an extension
%  \[
%    0
%    \to
%    I
%    \xto{\iota}
%    \glie \oplus \hlie
%    \xto{\pi}
%    \glie
%    \to
%    0
%  \]
%  is central if and only
%  \begin{equation}
%    \label{central extension condition on elements}
%    [(x,c), (0,d)] = 0
%    \qquad
%    \text{for all~$x \in \glie$ and~$c, d \in \hlie$.}
%  \end{equation}
%  We have
%  \[
%    [(x,c), (0,d)]
%    =
%    (0, \theta(x)(d) + [c,d])
%    \qquad
%    \text{for all~$x \in \glie$,~$c, d \in \hlie$,}
%  \]
%  so we find that the condition~\eqref{central extension condition on elements} is equivalent to the combination of the two conditions~$\theta = 0$ and also~$[c,d] = 0$ for all~$c, d \in \hlie$.
%  Indeed, by choosing~$d = c$ we see that~$\theta(x)(c) = 0$ for all~$x \in \glie$ and~$c \in \hlie$, and thus~$\theta = 0$.
%  And from this we then find that~$[c,d] = 0$ for all~$c, d \in \hlie$.
%
%  We now find from the general formula
%  \begin{equation}
%    \label{central extension formula}
%    [ (x,c), (y,d) ]
%    =
%    ( [x,y], \kappa(x,y) )
%  \end{equation}
%  for all~$(x,c), (y,d) \in \glie \oplus \hlie$.
%  
%  Given any bilinear form~$\kappa \colon \glie \times \glie \to \hlie$ we have already seen that \eqref{central extension formula} defines a bilinear bracket on~$\glie \oplus \hlie$ that is alternating if and only if~$\kappa$ is alternating.
%  We find for the Jacobi identity that
%  \[
%    [ (x,c), [ (y,d), (z,e) ] ]
%    =
%    [ (x,c), ([y,z], \kappa(y,z)) ]
%    =
%    ( [x,[y,z]], \kappa(x, [y,z]) )
%  \]
%  for all~$(x,c), (y,d), (z,e) \in \glie \oplus \hlie$ and therefore
%  \begin{align*}
%    {}&
%      [(x,c), [(y,d), (z,e)]]
%    + [(y,d), [(z,e), (x,c)]]
%    + [(z,e), [(x,c), (y,d)]]
%    \\
%    ={}&
%    (
%      [x,[y,z]] + [y,[z,x]] + [z,[x,y]],
%      \kappa(x, [y,z]) + \kappa(y, [z,x]) + \kappa(z, [x,y])
%    )
%    \\
%    ={}&
%    ( 0, \kappa(x, [y,z]) + \kappa(y, [z,x]) + \kappa(z, [x,y]) ) \,.
%  \end{align*}
%  We hence find that the bracket~$[\ph, \ph]$ on~$\glie \oplus \hlie$ satisfies the Jacobi identity if and only if the bilinear map~$\kappa$ satisfies the similar looking identity
%  \[
%    \kappa(x, [y,z]) + \kappa(y, [z,x]) + \kappa(z, [x,y])
%    =
%    0 \,.
%  \]
%  This condition is the \defemph{{\twococycle} condition}, and a bilinear map~$\kappa \colon \glie \to \glie \to \hlie$ that is both alternating is satisfies the {\twococycle} condition is a {\twococycle}.
%  
%  We have overall constructed for all Lie~algebras~$\glie$ and~$\hlie$ a {\onetoonetext} correspondence between
%  \begin{itemize}
%    \item
%      Lie~brackets~$[\ph, \ph]$ on the vector space~$\glie \oplus \hlie$ that make the standard short exact sequence
%      \[
%        0
%        \to
%        I
%        \to
%        \glie
%        \oplus
%        I
%        \to
%        \glie
%        \to
%        0
%      \]
%      into a central extension of~$\glie$ by~$\hlie$ and
%    \item
%      {\twococycles}~$\kappa \colon \glie \times \glie \to \hlie$,
%  \end{itemize}
%  and this correspondence is given by~$[(x,c), (y,d)] = ([x,y], \kappa(x,y))$.
%\end{example}



\subsection{Split Extensions and Semidirect Products}

\begin{fluff}
	Another important kind of extensions are so called semidirect products.
	We will give three characterizations of these extensions:
	internal semidirect product, external semidirect products, and split extensions.
	We will also explain the equivalence of these characterizations.
\end{fluff}

\subsubsection{Internal Semidirect Products}


\begin{definition}
	Let~$\elie$ be a Lie~algebra.
	Let~$\glie$ be a Lie~subalgebra of~$\elie$ and let~$\hlie$ be an ideal of~$\elie$.
	If~$\elie = \glie \oplus \hlie$ as vector spaces, then~$\elie$ is the \defemph{internal semidirect product} of~$\glie$ by~$\hlie$, or of~$\glie$ and~$\hlie$.
	This is then denoted by~$\elie = \glie \ltimes \hlie$\glsadd{internal semidirect product}\index{internal semidirect product}\index{semidirect product!index}.
\end{definition}


\begin{fluff}
	\label{calculating structure of internal semidirect product}
	Let~$\elie$ be a Lie~algebra and suppose that~$\elie$ is the semidirect product of a Lie~subalgebra~$\glie$ by an ideal~$\hlie$.
	The adjoint action of~$\elie$ on itself restricts to an action of~$\glie$ on~$\hlie$ because~$\hlie$ is an ideal of~$\elie$.
	The action of~$\elie$ on itself happens by derivatations, whence the action of~$\glie$ on~$\hlie$ is also by derivations.
	This action is given by a homomorphism of Lie~algebras
	\[
		\theta
		\colon
		\glie
		\to
		\Der(\hlie) \,.
	\]
	Any two elements of~$\elie$ may be written uniquely as sums~$x + c$ and~$y + d$ with~$x$,~$y$ in~$\glie$ and~$c$,~$d$ in~$\hlie$.
	The Lie~bracket of~$\elie$ can be expressed in terms of the homomorphisms~$\theta$ as
	\begin{align*}
		[x + c, y + d]_{\elie}
		&=
		[x, y]_{\elie}
		+ [x, d]_{\elie}
		+ [c, y]_{\elie}
		+ [c, d]_{\elie}
		\\
		&=
		[x, y]_{\elie}
		+ [x, d]_{\elie}
		- [y, c]_{\elie}
		+ [c, d]_{\elie}
		\\
		&=
		[x, y]_{\glie}
		+ \theta(x)(d)
		- \theta(y)(c)
		+ [c, d]_{\hlie} \,.
	\end{align*}
	The term~$[x,y]_{\glie}$ is again contained in the Lie~subalgebra~$\glie$, and the term~$\theta(x)(d) - \theta(y)(c) + [c,d]_{\hlie}$ is again contained in the ideal~$\hlie$.
\end{fluff}


\begin{example}
	The Lie~algebra~$\trianglie(n, \kf)$\index{upper triangular matrices} is the direct sum of its Lie~subalgebra of diagonal matrices~$\diaglie(n, \kf)$\index{diagonal matrices} and its ideal of strictly upper triangular matrices~$\upperlie(n, \kf)$\index{strictly upper triangular matrices}.
	The Lie~algebra~$\trianglie(n, \kf)$ is thus the internal semidirect product of~$\diaglie(n, \kf)$ by~$\upperlie(n, \kf)$.
\end{example}


\begin{example}
	\label{affine lie algebra is an internal semidirect product}
	The Lie~algebra of affine transformations~$\afflie(n, \kf)$\index{affine transformations} contains the Lie~subalgebra
	\[
		\glie
		\defined
		\begin{pmatrix}
			\gllie(n, \kf)  & 0 \\
			0               & 0
		\end{pmatrix}
	\]
	and the ideal
	\[
		\hlie
		\defined
		\begin{pmatrix}
			0 & \kf^n \\
			0 & 0
		\end{pmatrix} \,.
	\]
	We have~$\afflie(n, \kf) = \glie \oplus \hlie$ as vector spaces.
	The Lie~algebra~$\afflie(n, \kf)$ is therefore the internal semidirect product of~$\glie$ by~$\hlie$.
\end{example}


% TODO: Express aff(n,k) as an external semidirect product.


\begin{example}
	Let~$\heisenberglie$ be the Heisenberg Lie~algebra\index{Heisenberg Lie algebra} from \cref{examples for lie algebras}.
	The linear subspace~$\hlie$ of~$\heisenberglie$ spanned by the elements~$p_1, \dotsc, p_n, c$ is an ideal of~$\heisenberglie$, and the linear subspace~$\glie$ of~$\heisenberglie$ spanned by the elements~$q_1, \dotsc, q_n$ is a Lie~subalgebra of~$\elie$.
	The Lie~algebra~$\heisenberglie$ is thus the internal semidirect product of~$\glie$ by~$\hlie$.
\end{example}

\subsubsection{External Semidirect Products}

\begin{definition}
	Let~$\glie$ and~$\hlie$ be two~\liealgebras{$\kf$} and let~$\theta$ be a homomorphism of Lie~algebras from~$\glie$ to~$\Der(\hlie)$.
	The standard extension~$(\theta, 0)$ of~$\glie$ by~$\hlie$ is the \defemph{external semidirect product}\index{external semidirect product}\index{semidirect product!external} of~$\glie$ by~$\hlie$ along~$\theta$.
	It is denoted by~$\glie \ltimes_\theta \hlie$\glsadd{external semidirect product}.
\end{definition}


\begin{remark}
	An external semidirect product~$\glie \ltimes_\theta \hlie$ is given by the external direct sum~$\glie \oplus \hlie$ on the level of vector spaces.
	Its Lie~bracket is given by
	\[
		[ (x,c), (y,d) ]
		=
		( [x,y], \theta(x)(d) - \theta(y)(c) + [c,d] )
	\]
	for all~$(x,c), (y,d) \in \glie \ltimes_\theta \hlie$.
	We have in particular
	\[
		[(x,0), (0,c)]
		=
		( 0, \theta(x)(c) )
	\]
	for all~$x \in \glie$ and~$c \in \hlie$.
\end{remark}


\begin{fluff}
	\label{internal semidirect product is also external}
	Let~$\elie$ be a Lie~algebra, let~$\glie$ be a Lie~subalgebra of~$\elie$ and let~$\hlie$ be an ideal of~$\elie$.
	Suppose that~$\elie$ is the internal semidirect product of~$\glie$ by~$\hlie$.
	We have seen in~\cref{calculating structure of internal semidirect product} that the Lie~bracket on~$\elie$ is uniquely determined by the Lie~brackets of~$\glie$ and~$\hlie$ and a certain homomorphism of Lie~algebras~$\theta$ from~$\glie$ to~$\Der(\hlie)$.
	Let~$\elie_{\ext}$ be the external semidirect product of~$\glie$ by~$\hlie$ along~$\theta$.
	Then the linear map
	\[
		\varphi
		\colon
		\elie_{\ext}
		\to
		\elie \,,
		\quad
		(x, c)
		\mapsto
		x + c
	\]
	is an isomorphism of vector spaces, and according to the calculations in \cref{calculating structure of internal semidirect product} it is an isomorphism of Lie~algebras.
	In this way, we see that every internal semidirect product is isomorphic to an external semidirect product.
\end{fluff}


\begin{fluff}
	Let on the other hand~$\glie$ and~$\hlie$ be two~\liealgebras{$\kf$} and let~$\theta$ be a homomorphism of Lie~algebras from~$\glie$ to~$\Der(\hlie)$.
	Let~$\elie$ be the external semidirect product of~$\glie$ by~$\hlie$ along~$\theta$.
	Then~$\elie = \glie \oplus \hlie$ as vector spaces, where the right hand side denotes an external direct sum.
	The linear subspace~$\glie'$ of~$\elie$ given by~$\glie' \defined \glie \oplus 0$ is a Lie~subalgebra of~$\elie$, and the linear subspace~$\hlie'$ given by~$\hlie' \defined 0 \oplus \hlie$ is an ideal of~$\elie$.
	We also see that~$\elie = \elie' \oplus \hlie'$ on the level of vector spaces, where the right hand side denotes an internal direct sum.
	We see from this that~$\elie$ is the internal direct sum of~$\glie'$ by~$\hlie'$.

	We know from \cref{calculating structure of internal semidirect product} that the Lie~bracket of~$\elie$ can be expressed by the Lie~brackets of~$\glie'$ and~$\hlie'$ together with a certain homomorphism of Lie~algebras~$\theta'$ from~$\glie'$ to~$\Der(\hlie')$.
	We can identify~$\glie'$ and~$\hlie'$ with~$\glie$ and~$\hlie$ via the isomorphism of Lie~algebras
	\begin{alignat*}{2}
		\glie
		&\to
		\glie' \,,
		&\quad
		x
		&\mapsto
		(x, 0) \,,
		\\
		\hlie
		&\to
		\hlie' \,,
		&\quad
		c
		&\mapsto
		(0, c) \,.
	\end{alignat*}
	The homomorphism~$\theta'$ corresponds to the original homomorphism~$\theta$ under these isomorphism.

	We have now explained how every external semidirect product can be regarded as an internal semidirect product.
\end{fluff}


\begin{example}
	Let~$\glie$ be a Lie~algebra and let~$\hlie$ be a representation of~$\glie$.
	We may regard~$\hlie$ as an abelian Lie~algebra.
	Then~$\Der(\hlie) = \gllie(\hlie)$.
	We can therefore regard the homomorphism of Lie~algebras~$\rho$ that corresponds to the action of~$\glie$ on~$\hlie$ as a homomorphism of Lie~algebras from~$\glie$ to~$\Der(\hlie)$.
	This allows us to form the external semidirect product~$\glie \ltimes_\rho \hlie$.
	It is given by the vector space~$\glie \oplus \hlie$ together with the Lie~bracket given by
	\[
		[ (x, c), (y, d) ]
		=
		( [x,y], x \act d - y \act c )
	\]
	for all~$x, y \in \glie$ and~$c, d \in \hlie$.
\end{example}


\begin{example}
	Let~$\hlie$ be the abelian two-dimensional Lie~algebra with basis~$x$,~$y$ and let~$\glie$ be the one-dimensional Lie~algebra with basis~$t$.
	Then~$\Der(\hlie) = \gllie(\hlie)$ as seen in \cref{derivations of abelian lie algebra}, and every element~$f$ of~$\gllie(\hlie)$ determines a unique homomorphism of Lie algebras from~$\glie$ to~$\Der(\hlie)$ that maps~$t$ to~$f$.
	We denote for every scalar~$\tau$ in~$\kf$ by~$f_\tau$ the endomorphism of~$\hlie$ given by
	\[
		f_\tau(x) = x \,,
		\quad
		f_\tau(y) = \tau y \,,
	\]
	and by~$\theta_\tau$ the resulting homomorphism of Lie~algebras from~$\glie$ to~$\Der(\hlie)$ that maps~$t$ to~$f_\tau$.
	The external semidirect product~${\kf} \ltimes_{\theta_t} \hlie$ has the elements~$x$,~$y$,~$t$ as a basis, and its Lie~bracket is given on this basis by
	\[
		[x,y] = 0 \,,
		\quad
		[t,x] = x \,,
		\quad
		[t,y] =  \tau y \,.
	\]
	We have thus constructed the Lie~algebras~$\glie_\tau$ from \cref{infinitely many three-dimensional lie algebras} as semidirect products.
\end{example}

\subsubsection{Split Extensions}

\begin{definition}
	Let
	\[
		0
		\to
		\hlie
		\xto{\varphi}
		\elie
		\xto{\psi}
		\glie
		\to
		0
	\]
	be an extension of Lie~algebras.
	A \defemph{section}\index{section}\index{short exact sequence of Lie algebras!section} or \defemph{split}\index{split}\index{short exact sequence of Lie algebras!split} of this extension is a homomorphism of Lie~algebras~$\sigma$ from~$\glie$ to~$\elie$ such that
	\[
		\psi \circ \sigma
		=
		\id_{\glie} \,.
	\]
	The extension \defemph{splits} if it admits a split.
\end{definition}


\begin{lemma}
	\label{having a section is invariant under equivalence of extensions}
	Let~$\glie$ and~$\hlie$ be two~\liealgebras{$\kf$}.
	If an extension~$\elie$ of~$\glie$ by~$\hlie$ admits a split, then every extension of~$\glie$ by~$\hlie$ that is equivalent to~$\elie$ also admits a split.
\end{lemma}


\begin{proof}
	Let the extension~$\elie$ be given by
	\[
		0
		\to
		\hlie
		\xto{\varphi}
		\elie
		\xto{\psi}
		\glie
		\to
		0 \,2
	\]
	and let
	\[
		0
		\to
		\hlie
		\xto{\varphi'}
		\elie'
		\xto{\psi'}
		\glie
		\to
		0
	\]
	be an extensions of~$\glie$ by~$\hlie$ that is equivalent to~$\elie$.
	Let~$\alpha$ be an isomorphism of Lie~algebras from~$\elie$ to~$\elie'$ which makes the diagram
	\[
		\begin{tikzcd}
			0
			\arrow{r}
			&
			\hlie
			\arrow{r}[above]{\varphi}
			\arrow[equal]{d}
			&
			\elie
			\arrow{r}[above]{\psi}
			\arrow[dashed]{d}[right]{\alpha}
			&
			\glie
			\arrow{r}
			\arrow[equal]{d}
			&
			0
			\\
			0
			\arrow{r}
			&
			\hlie
			\arrow{r}[above]{\varphi'}
			&
			\elie'
			\arrow{r}[above]{\psi'}
			&
			\glie
			\arrow{r}
			&
			0
		\end{tikzcd}
	\]
	commute.
	If~$\sigma$ is a split for the first extension, then the composite~$\alpha \circ \sigma$ is a split for the second extension.
	This shows that the second extension admits a split if the first one does.
	The converse implication follows from this first implication because equivalence of extensions is an equivalence relation.
\end{proof}


\begin{proposition}
	Let~$\glie$ and~$\hlie$ be two Lie~algebras and let
	\begin{equation}
		\label{extension that may split}
		0
		\to
		\hlie
		\xto{\varphi}
		\elie
		\xto{\psi}
		\glie
		\to
		0
	\end{equation}
	be an extension of~$\glie$ by~$\hlie$.
	The following conditions on this extension are equivalent.
	\begin{equivalenceslist}
		\item
			\label{the extension splits}
			The extension~\eqref{extension that may split} splits.
		\item
			\label{is an internal semidirect product}
			There exists a Lie~subalgebra~$\glie'$ of~$\elie$ such that~$\elie$ is the internal semidirect product of~$\glie'$ by~$\varphi(\hlie)$.
		\item
			\label{is an external semidirect product}
			The extension~\eqref{extension that may split} is isomorphic (as extensions) to an external semidirect product of~$\glie$ by~$\hlie$.
	\end{equivalenceslist}
\end{proposition}


\begin{proof}
	\leavevmode
	\begin{implicationlist}
		\item[\ref{the extension splits}~$\implies$~\ref{is an internal semidirect product}]
			There exists by assumption a split~$\sigma$ for the extension~\eqref{extension that may split}, i.e. a homomorphism of Lie~algebras from~$\glie$ to~$\elie$ with~$\sigma \circ \psi = \id_{\glie}$.
			Let~$\glie'$ be the image of~$\sigma$.
			The composite~$\varepsilon \defined \sigma \circ \psi$ is an idempotent endomorphism of Lie~algebras of~$\elie$.
			It follows that
			\[
				\elie
				=
				\im(\varepsilon) \oplus \ker(\varepsilon)
			\]
			as vector spaces, with~$\im(\varepsilon)$ being a Lie~subalgebra of~$\elie$ and~$\ker(\varepsilon)$ beinan ideal of~$\elie$.
			The homomorphism~$\psi$ is surjective, so
			\[
				\im(\varepsilon)
				=
				\im(\sigma \circ \psi)
				=
				\im(\sigma)
				=
				\glie' \,.
			\]
			The homomorphism~$\sigma$ is injective (because it admits a left inverse, namely~$\psi$), so
			\[
				\ker(\varepsilon)
				=
				\ker(\sigma \circ \psi)
				=
				\ker(\psi)
				=
				\im(\varphi)
				=
				\varphi(\hlie) \,.
			\]
			We have overall shown that~$\elie$ is the internal semidirect product of~$\glie'$ by~$\varphi(\hlie)$.
		\item[\ref{is an internal semidirect product}~$\implies$~\ref{is an external semidirect product}]
			This follows from the discussion in \cref{internal semidirect product is also external} by noting that the constructed isomorphism of Lie~algebras~$\varphi$ is already an isomorphism of extensions.
		\item[\ref{is an external semidirect product}~$\implies$~\ref{the extension splits}]
			According to \cref{having a section is invariant under equivalence of extensions} we may assume that~$\elie$ is an external semidirect product of~$\glie$ by~$\hlie$ along a homomorphism of Lie~algebras~$\theta$ from~$\glie$ to~$\Der(\hlie)$.
			The linear map
			\[
				\glie
				\to
				\elie \,,
				\quad
				x
				\mapsto
				(x,0)
			\]
			is then a homomorphism of Lie~algebras, and the desired split.
		\qedhere
	\end{implicationlist}
\end{proof}


\begin{remark}
	\label{each class of abelian extensions contains precisely one split extension}
	Let~$\glie$ be a Lie~algebra and let~$M$ be a representation of~$\glie$.
	We regard the underlying vector space of~$M$ as an abelian Lie~algebra~$\hlie$ and denote by~$\theta$ the homomorphism of Lie~algebras from~$\hlie$ to~$\Der(\hlie) = \gllie(\hlie) = \gllie(M)$ that corresponds to the action of~$\elie$ on~$M$.
	We consider the resulting class of abelian extensions~$\AbEx(\glie, M)$ from~\cref{structure of abelian extensions}.

	The external semidirect product~$(0,\theta)$ is an abelian extension of~$\glie$ by~$\hlie$.
	We have seen in~\cref{structure of abelian extensions} that this extension is contained in~$\AbEx(\glie, M)$.

	Suppose on the other hand that~$\elie$ is some split extension of~$\glie$ by~$\hlie$ contained in~$\AbEx(\glie, M)$.
	The extension~$\elie$ is then isomorphic to a standard extension of the form~$(0, \theta')$.
	It follows from \cref{structure of abelian extensions} that~$\theta' = \theta$.

	We have thus shown that the class of extensions~$\AbEx(\glie, M)$ contains precisely one split extension up to equivalence.
\end{remark}





