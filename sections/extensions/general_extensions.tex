\section{General Extensions}


\begin{definition}
	A \defemph{short exact sequence of Lie~algebras}\index{short exact sequence of Lie algebras} is a short exact sequence
	\begin{equation}
		\label{general extension}
		0 
		\to
		\hlie
		\xto{\varphi}
		\elie
		\xto{\psi}
		\glie
		\to
		0
	\end{equation}
	of vector spaces where~$\hlie$,~$\elie$, and~$\glie$ are Lie algebras and both~$\varphi$ and~$\psi$ are homomorphisms of Lie algebras.
	The short exact sequence~\eqref{general extension} is an~\defemph{extension}\index{extension of Lie algebras} of~$\elie$ by~$\glie$.
\end{definition}


\begin{remark}
	\leavevmode
	\begin{enumerate}
		\item
			By abuse of notation we often say that \enquote{$\elie$ is an extension of~$\glie$ by~$\hlie$} when talking about an extension
			of~$\glie$ by~$\hlie$ of the form
			\[
				0 
				\to
				\hlie
				\to
				\elie
				\to
				\glie
				\to
				0 \,.
			\]
		\item
			Let
			\[
				0
				\to
				\hlie
				\xto{\varphi}
				\elie
				\xto{\psi}
				\glie
				\to
				0
			\]
			be a short exact sequence of Lie~algebras.
			Then the homomorphism~$\varphi$ is injective and identifies the Lie~algebra~$\hlie$ with the kernel of~$\psi$, which is an ideal of~$\elie$.
		\item
			Let~$\glie$ and~$\hlie$ be two~\liealgebra{$\kf$}.
			An extension~$\elie$ of~$\glie$ by~$\hlie$ encodes the following information:
			the Lie~algebra structure of~$\hlie$, the Lie~algebra structure of~$\glie$, and some kind of interaction between~$\glie$ and~$\hlie$.
			How exactly these informations are encoded into~$\elie$ relies on the specific choice of homomorphism~$\varphi$ and~$\psi$ that are part of the extension.
	\end{enumerate}
\end{remark}


\begin{example}
	\label{generic short exact sequence}
	Let~$I$ is an ideal of a Lie~algebra~$\glie$.
	Let~$\iota$ be the inclusion map from~$I$ to~$\glie$ and let~$\pi$ be the canonical projection map from~$\glie$ to~$\glie/I$.
	Then
	\[
		0
		\to
		I
		\xto{\iota}
		\glie
		\xto{\pi}
		\glie/I
		\to
		0
	\]
	is a short exact sequence of Lie~algebras.
\end{example}


\begin{definition}
	\label{equivalence of extensions}
	Let~$\hlie$ and~$\glie$ be two Lie~algebras and let
	\[
		0 
		\to
		\hlie
		\xto{\varphi}
		\elie
		\xto{\psi}
		\glie
		\to
		0
		\qquad\text{and}\qquad
		0 
		\to
		\hlie
		\xto{\varphi'}
		\elie'
		\xto{\psi'}
		\glie
		\to
		0
	\]
	be two extensions of~$\glie$ by~$\hlie$.
	These extensions are \defemph{equivalent}\index{equivalence of extensions}\index{extension of Lie algebras!equivalence} if there exists an isomorphism of Lie~algebras~$\alpha$ from~$\elie$ to~$\elie'$ that makes the following diagram commute.
	\[
		\begin{tikzcd}[column sep = large]
			0
			\arrow{r}
			&
			\hlie
			\arrow{r}[above]{\varphi}
			\arrow[equal]{d}
			&
			\elie
			\arrow{r}[above]{\psi}
			\arrow[dashed]{d}[right]{\alpha}
			&
			\glie
			\arrow{r}
			\arrow[equal]{d}
			&
			0
			\\
			0
			\arrow{r}
			&
			\hlie
			\arrow{r}[above]{\varphi'}
			&
			\elie'
			\arrow{r}[above]{\psi'}
			&
			\glie
			\arrow{r}
			&
			0
		\end{tikzcd}
	\]
\end{definition}


\begin{proposition}
	Let~$\hlie$ and~$\glie$ be two Lie~algebras.
	Equivalence of extensions is an equivalence relation on the class of extensions of~$\glie$ by~$\hlie$.
	\qed
\end{proposition}


\begin{remark}
	In \cref{equivalence of extensions} it sufficies to require for~$\varphi$ to be a homomorphism of Lie~algebras.
	It then follows from the five lemma that~$\varphi$ is already an isomorphism of vector spaces, and thus an isomorphism of Lie~algebras.
\end{remark}


\begin{fluff}
	In the following we want to classify certain classes of extensions.
	But this needs some preparation.
	We will start by showing that every extensions is, up to equivalence, of a certain standard form.
\end{fluff}


\begin{definition}
	Let~$\glie$ and~$\hlie$ be two~\liealgebras{$\kf$}.
	An extension of~$\glie$ by~$\hlie$ is of \defemph{standard form}\index{standard extension of Lie algebras}\index{extension of Lie algebras!standard}, or simply \defemph{standard}%
	\footnote{
		This is not a standard definition.
		It is made up by the author of these notes to make things more explicit.
	}, if its underlying short exact sequence of vector spaces is given by
	\[
		0
		\to
		\hlie
		\xto{\iota}
		\glie \oplus \hlie
		\xto{\pi}
		\glie
		\to
		0 \,,
	\]
	where~$\iota$ denotes the inclusion map, given by~$c \mapsto (0,c)$, and~$\pi$ denotes the canonical projection map, given by~$(x,c) \mapsto x$.
\end{definition}


\begin{warning}
	In a standard extension of~$\glie$ by~$\hlie$ the middle term~$\glie \oplus \hlie$ carries the structure of a Lie~algebra, which makes the inclusion map~$\iota$ and projection map~$\pi$ into homomorphisms of Lie~algebras.
	But this Lie~algebra structure on the vector space~$\glie \oplus \hlie$ does not have to be given by the componentwise Lie~bracket of the direct sum!
	We will see in \cref{structure of extensions} how such Lie~bracket on~$\glie \oplus \hlie$ has to look like.
\end{warning}


\begin{proposition}
	\label{every extension is equivalent to a standard one}
	Let~$\glie$ and~$\hlie$ be two~\liealgebras{$\kf$}.
	Every extension of~$\glie$ by~$\hlie$ is equivalent to a standard extension.
\end{proposition}


\begin{proof}
	Let
	\[
		0
		\to
		\hlie
		\xto{\varphi}
		\elie
		\xto{\psi}
		\glie
		\to
		0
	\]
	be an extension of~$\glie$ by~$\hlie$.
	This extension is in particular a short exact sequence of vector spaces, and it splits on the level of vector spaces.
	There hence exists an isomorphism of vector spaces~$\alpha$ from~$\elie$ to~$\glie \oplus \hlie$ that makes the diagram
	\begin{equation}
		\label{constructing equivalence with standard extension}
		\begin{tikzcd}[column sep = large]
			0
			\arrow{r}
			&
			\hlie
			\arrow{r}[above]{\varphi}
			\arrow[equal]{d}
			&
			\elie
			\arrow{r}[above]{\psi}
			\arrow[dashed]{d}[right]{\alpha}
			&
			\glie
			\arrow{r}
			\arrow[equal]{d}
			&
			0
			\\
			0
			\arrow{r}
			&
			\hlie
			\arrow{r}[above]{\iota}
			&
			\glie \oplus \hlie
			\arrow{r}[above]{\pi}
			&
			\glie
			\arrow{r}
			&
			0
		\end{tikzcd}
	\end{equation}
	commute.
	We can pushforward the Lie~bracket of~$\elie$ to a Lie~bracket on~$\glie \oplus \hlie$ along~$\alpha$.
	This makes~$\glie \oplus \hlie$ into a Lie~algebra and the map~$\alpha$ into an isomorphism of Lie~algebras.
	It follows from the commutativity of the diagram~\eqref{constructing equivalence with standard extension} that
	\[
		\iota = \alpha \circ \varphi
		\quad\text{and}\quad
		\pi = \psi \circ \alpha^{-1}
	\]
	are again homomorphisms of Lie~algebras.
	We hence find that
	\[
		0
		\to
		\hlie
		\xto{\iota}
		\glie \oplus \hlie
		\xto{\pi}
		\glie
		\to
		0
	\]
	is a standard extension of~$\glie$ by~$\hlie$.
	This standard extension is equivalent to the original extension by the commutativity of the diagram~\eqref{constructing equivalence with standard extension}.
\end{proof}


\begin{fluff}
	\label{general approach to extensions}
	Let~$\hlie$ and~$\glie$ be Lie~algebras.
	We will now investigate how the Lie~bracket on a standard extension of~$\glie$ by~$\hlie$ has to look like.
	We will then also express under what conditions two standard extensions are equivalent.
	The following discussion proceeds in multiple steps.
	\begin{enumerate}
		\item
			We consider a standard extension
			\[
				0
				\to
				\hlie
				\xto{\iota}
				\elie
				\xto{\pi}
				\glie
				\to
				0 \,.
			\]
			We want to understand how the Lie~bracket on~$\elie$ looks like.
			It follows from the bilinearity of the Lie~bracket~$[\ph, \ph]$ on~$\elie$ that
			\begin{align*}
				{}&
				[(x, c), (y,d)]
				\\
				={}&
					[(x,0), (y,0)]
				+ [(x,0), (0,d)]
				+ [(0,c), (y,0)]
				+ [(0,c), (0,d)]
				\\
				={}&
					[(x,0), (y,0)]
				+ [(x,0), (0,d)]
				- [(y,0), (0,c)]
				+ [(0,c), (0,d)]
			\end{align*}
			for all~$(x,c), (y,d) \in \elie$.

			We observe that
			\[
				\pi( [(x,0), (y,0)] )
				=
				[ \pi( (x,0) ), \pi( (y,0) ) ]
				=
				[x, y]
			\]
			because~$\pi$ is a homorphism of Lie~algebras.
			The commutator~$[(x,0), (y,0)]$ is therefore of the form
			\[
				[(x,0), (y,0)]
				=
				( [x,y], \kappa(x,y) )
			\]
			for all~$x, y \in \glie$ for some function
			\[
				\kappa
				\colon
				\glie \times \glie
				\to
				\hlie \,.
			\]
			It follows from the bilinearity of the Lie~bracket~$[\ph, \ph]$ on~$\elie$ that this map~$\kappa$ is again bilinear.

			We note that the two commutators~$[(x,0), (0,c)]$ and~$[(y,0), (0,d)]$ are again contained in~$0 \oplus \hlie$ because this is an ideal of~$\glie \oplus \hlie$, namely the kernel of the homomorphism~$\pi$.
			It follows together with the bilinearity of the Lie~bracket~$[\ph, \ph]$ on~$\elie$ that there exists a unique linear map
			\[
				\theta'
				\colon
				\glie
				\to
				\gllie(\hlie)
			\]
			with
			\[
				[(z,0), (0,e)]
				=
				[0, \theta(z)(e)]
			\]
			for all~$z \in \glie$ and~$e \in \hlie$.
			The map~$[(z,0), \ph] = \ad_{\glie \oplus \hlie}((z,0))$ is a derivation of~$\elie$ by \cref{lie algebras act adjoint by derivations}.
			It follows that the map~$\theta'$ corestricts to a linear map
			\[
				\theta
				\colon
				\glie
				\to
				\Der(\hlie) \,.
			\]
			For this map~$\theta$ we now have
			\[
				[(x,0), (0,d)]
				=
				[0, \theta(x)(d)]
				\quad\text{and}\quad
				[(y,0), (0,e)]
				=
				[0, \theta(y)(e)] \,.
			\]

			The last remaining commutator~$[(0,c), (0,d)]$ can be computed by using that the inclusion map~$\iota$ is a homomorphism of Lie~algebras from~$\hlie$ to~$\elie$, and therefore
			\[
				[(0,c), (0,d)]
				=
				[\iota(c), \iota(d)]
				=
				\iota([c,d])
				=
				(0, [c,d]) \,.
			\]

			We find altogether that the Lie bracket on~$\elie$ can be expressed with the help of the two maps~$\kappa$ and~$\theta$ as
			\begin{align*}
				{}&
				[ (x,c), (y,d) ]
				\\
				={}&
				[(x,0), (y,0)]
				+ [(x,0), (0,d)]
				- [(y,0), (0,c)]
				+ [(0,c), (0,d)]
				\\
				={}&
					( [x,y], \kappa(x,y) )
				+ ( 0, \theta(x)(d) )
				- ( 0, \theta(y)(c) )
				+ ( 0, [c,d] )
				\\
				={}&
				(
					[x,y],
					\kappa(x,y) + \theta(x)(d) - \theta(y)(c) + [c,d]
				) \,.
			\end{align*}
			The Lie~algebra structure of the standard extension~$\elie$ is therefore uniquely described by the two maps~$\kappa$ and~$\theta$.
		\item
			In the second step of this discussion we want to better understand the above map~$\kappa$ and~$\theta$.
			To do so, let
			\[
				\kappa
				\colon
				\glie \times \glie
				\to
				\hlie
			\]
			be any bilinear map, and let
			\[
				\theta
				\colon
				\glie
				\to
				\Der(\hlie)
			\]
			be any linear map.
			We define a bracket~$[\ph, \ph]$ on the vector space~$\glie \oplus \hlie$ via the previously deduced formula, i.e. via
			\begin{equation}
				\label{definition of lie bracket depending on kappa and theta}
				[(x,c), (y,d)]
				\defined
				(
					[x,y] ,
					\kappa(x,y) + \theta(x)(d) - \theta(y)(c) + [c,d]
				)
			\end{equation}
			for all~$(x,c), (y,d) \in \glie \oplus \hlie$.
			This bracket~$[\ph, \ph]$ on~$\glie \oplus \hlie$ is bilinear, and both the inclusion map~$\iota$ from~$\hlie$ to~$\glie \oplus \hlie$ and the canonical projection map~$\pi$ from~$\glie \oplus \hlie$ to~$\glie$ respect this bracket.
			There is only thing missing for
			\[
				0
				\to
				\hlie
				\xto{\iota}
				\glie \oplus \hlie
				\xto{\pi}
				\glie
				\to
				0
			\]
			to be an extension of Lie~algebras, namely that the bracket~$[\ph, \ph]$ on~$\glie \oplus \hlie$ makes it into a Lie~algebra, i.e. that it is actually a Lie~bracket.
			For this we need~$[\ph, \ph]$ to be alternating and to satisfy the Jacobi identity.

			We have for every element~$(x,c)$ of~$\glie \oplus \hlie$ that
			\[
				[(x,c), (x,c)]
				=
				( [x,x], \kappa(x,x) + \theta(x)(c) - \theta(x)(c) + [c,c] )
				=
				(0, \kappa(x,x) ) \,.
			\]
			We therefore find that the bracket~$[\ph, \ph]$ on~$\glie \oplus \hlie$ is alternating if and only if the bilinear map~$\kappa$ is alternating.

			We can also express the Jacobi identity\index{Jacobi identity!for standard extensions} for the bracket~$[\ph, \ph]$ on~$\glie \oplus \hlie$ in terms of conditions on~$\kappa$ ond~$\theta$, as we will now painstakingly demonstrate.
			We first calculate the iterated commutator
			\[
				[(x,c), [(y,d), (z,e)]]
			\]
			for all~$(x,c), (y,d), (z,e) \in \glie \oplus \hlie$ via the formula~\eqref{definition of lie bracket depending on kappa and theta} as
			\begin{align}
				% first term
				{}&
				[(x,c), [(y,d), (z,e)]]
				\notag
				\\
				% second torm
				={}&
				[(x,c), ([y,z], \kappa(y,z) + \theta(y)(e) - \theta(z)(d) + [d,e])]
				\notag
				\\
				% third term
				={}&
				\bigl(
					[x, [y, z]], 
				\notag
				\\
				{}&
					\kappa( x, [y, z] )
					+ \theta(x)
						\bigl(
							\kappa(y,z) + \theta(y)(e) - \theta(z)(d) + [d,e]
						\bigr)
				\notag
				\\
				{}&
				\hphantom{\biggl(}
					- \theta([y,z])(c)
					+ \bigl[
							c,
							\kappa(y,z) + \theta(y)(e) - \theta(z)(d) + [d,e]
						\bigr]
				\bigr)
				\notag
				\\
				% fourth term
				={}&
				\smash{\bigl(}
					[x, [y, z]],
				\notag
				\\
				{}&
				\hphantom{\bigl(}
					\kappa( x, [y, z] )
					+ \theta(x)( \kappa(y,z) )
				\label{ugly row 1}
				\\
				{}&
				\hphantom{\bigl(}
					+ \theta(x)( \theta(y)(e) )
					- \theta(x)( \theta(z)(d) )
				\label{ugly row 2}
				\\
				{}&
				\hphantom{\bigl(}
					+ \theta(x)( [d,e] )
				\label{ugly row 3}
				\\
				{}&
				\hphantom{\bigl(}
					- \theta( [y,z] )(c)
				\label{ugly row 4}
				\\
				{}&
				\hphantom{\bigl(}
					+ [c, \kappa(y,z)]
				\label{ugly row 5}
				\\
				{}&
				\hphantom{\bigl(}
					+ [c, \theta(y)(e)]
					- [c, \theta(z)(d)]
				\label{ugly row 6}
				\\
				{}&
				\hphantom{\bigl(}
					+ [c, [d, c]]
				\bigr) \,.
				\label{ugly row 7}
			\end{align}
			For every expression~$F$ in the three arguments~$(x,c)$,~$(y,d)$,~$(z,e)$ of~$\glie \oplus \hlie$ we will in the following use the abbreviation
			\[
				\sum_{\cyc} F( (x,c), (y,d), (z,e) )
				\glsadd{cyclic sum}
			\]
			to denote the cyclic sum\index{cyclic sum}
			\[
				F( (x,c), (y,d), (z,e) )
				+ F( (y,d), (z,e), (x,c) )
				+ F( (z,e), (x,c), (y,d) ) \,.
			\]
			We want to examine which conditions the two maps~$\kappa$ and~$\theta$ need to satisfy to ensure that
			\[
				\sum_{\cyc}
				[(x,c), [(y,d), (z,e)]]
				=
				0
				\qquad
				\text{for all~$(x,c), (y,d), (z,e) \in \glie \oplus \hlie$.}
			\]
			We see from~\eqref{ugly row 1} that the first entry of~$\sum_{\cyc} [(x,c), [(y,d), (z,e)]]$ is given by the cyclic sum
			\[
				\sum_{\cyc} [x,[y,z]] \,.
			\]
			But this terms vanishes because the Lie~bracket of~$\glie$ satisfies the Jacobi identity.
			We therefore only have to worry about the second entry of~$\sum_{\cyc} [(x,c), [(y,d), (z,e)]]$.
			For this second entry we can make the following observations.
			\begin{itemize*}
				\item
					For the term~\eqref{ugly row 7} we have
					\[
						\sum_{\cyc} [c,[d,e]] = 0
					\]
					because the Lie~bracket on~$\hlie$ satisfies the Jacobi identity.
					We therefore do not have to worry about the term~\eqref{ugly row 7}.
				\item
					The term~\eqref{ugly row 3} can be expanded as
					\[
						\theta(x)([d,e])
						=
						[\theta(x)(d), e] + [d, \theta(x)(e)]
					\]
					because~$\theta(x)$ is a derivation on~$\hlie$.
					We have
					\[
						\sum_{\cyc} [\theta(x)(d), e]
						=
						\sum_{\cyc} [\theta(y)(e), c]
						=
						-\sum_{\cyc} [c, \theta(y)(e)]
					\]
					and similarly
					\[
						\sum_{\cyc} [d, \theta(x)(e)]
						=
						\sum_{\cyc} [c, \theta(z)(d)] \,.
					\]
					With this we find that the cyclic sums of the term~\eqref{ugly row 3} and~\eqref{ugly row 6} cancel out.
					We therefore do not have to worry about these terms from now on.
				\item
					In the special case~$c, d, e = 0$ all terms except~\eqref{ugly row 1} vanish.
					We therefore find that~$\kappa$ and~$\theta$ have to satisfy the condition.
					\[
						\sum_{\cyc}
						(
							\kappa(x, [y,z])
							+ \theta(x)( \kappa(y,z))
						)
						=
						0 \,.
					\]
					By using that~$\kappa$ is alternating and therefore anti-symmetric, we may rewrite this condition as
					\begin{align*}
						0
						&=
						\sum_{\cyc}
						(
							\kappa(x, [y,z])
							+ \theta(x)( \kappa(y,z))
						)
						\\
						&=
						\sum_{\cyc} \kappa(x, [y,z])
						+
						\sum_{\cyc} \theta(x)( \kappa(y,z))
						\\
						&=
						-\sum_{\cyc} \kappa([y,z], x)
						+
						\sum_{\cyc} \theta(x)( \kappa(y,z))
						\\
						&=
						-\sum_{\cyc} \kappa([x,y], z)
						+
						\sum_{\cyc} \theta(x)( \kappa(y,z)) \,,
					\end{align*}
					and therefore as
					\[
						\sum_{\cyc} \kappa([x,y], z)
						=
						\sum_{\cyc} \theta(x)( \kappa(y,z) ) \,.
					\]
					We have thus arrived at the condition
					\begin{equation}
						\label{nonabelian cocycle condition}
						\begin{aligned}
							{}&
							\kappa([x,y], z)
							+ \kappa([y, z], x)
							+ \kappa([z, x], y)
							\\
							={}&
							\theta(x)( \kappa(y,z) )
							+ \theta(y)( \kappa(z,x) )
							+ \theta(z)( \kappa(x,y) )
						\end{aligned}
					\end{equation}
					for all~$x, y, z \in \glie$.

					We have derived this condition on~$\kappa$ and~$\theta$ by considering the special case~$c, d, e = 0$.
					However, as long as the condition~\eqref{nonabelian cocycle condition} is satisfied,  the term~\eqref{ugly row 1} will vanish in the cyclic sum of~$[(x,c), [(y,d), (z,e)]]$ for arbitrary~$c, d, e \in \hlie$.
					So as long as condition~\eqref{nonabelian cocycle condition} is satisfied we do not have to worry about the term~\eqref{ugly row 1}.
				\item
					It remains to consider the terms~\eqref{ugly row 2},~\eqref{ugly row 4} and~\eqref{ugly row 5}.
					For those terms we calculate that
					\begin{align}
						{}&
						\sum_{\cyc}
						\Bigl(
							\theta(x)( \theta(y)(e) )
							-
							\theta(x)( \theta(z)(d) )
							-
							\theta([y,z])(c)
							+
							[c, \kappa(y,z)]
						\Bigr)
						\label{unrotated condition on kappa and theta}
						\\
						={}&
						\sum_{\cyc}
						\theta(x)( \theta(y)(e) )
						-
						\sum_{\cyc}
						\theta(x)( \theta(z)(d) )
						-
						\sum_{\cyc}
						\theta([y,z])(c)
						+
						\sum_{\cyc}
						[c, \kappa(y,z)]
						\notag
						\\
						={}&
						\sum_{\cyc}
						\theta(x)( \theta(y)(e) )
						-
						\sum_{\cyc}
						\theta(y)( \theta(x)(e) )
						-
						\sum_{\cyc}
						\theta([x,y])(e)
						+
						\sum_{\cyc}
						[e, \kappa(x,y)]
						\notag
						\\
						={}&
						\sum_{\cyc}
						\Bigl(
							\theta(x)( \theta(y)(e) )
							-
							\theta(y)( \theta(x)(e)
							-
							\theta([x,y])(e)
							+
							[e, \kappa(x,y)]
						\Bigr) \,.
						\label{rotated conditon on kappa and theta}
					\end{align}
					We note that in the cyclic sum~\eqref{rotated conditon on kappa and theta} each summand depends on only of the three variables~$c$,~$d$,~$e$, und it does so in a linear way.
					It follows that the sum~\eqref{rotated conditon on kappa and theta} vanishes for arbitrary choices of~$(x,c)$,~$(y,d)$,~$(z,e)$ if and only if each summand vanishes.
					(Indeed, we can isolate one of the three summands by considering the special case~$c, d = 0$.
					From this we see that this summand has to vanish on its own.
					But we can also isolate the other two summands by consider the special cases~$c, e = 0$ and~$d, e = 0$.
					So each summand has to vanish on its own.)

					All three summands give the same condition, namely that
					\[
						\theta(x)( \theta(y)(e) )
						-
						\theta(y)( \theta(x)(e)
						-
						\theta([x,y])(e)
						+
						[e, \kappa(x,y)]
						=
						0
					\]
					for all~$x, y \in \glie$ and~$e \in \hlie$.
					We may rewrite this condition as
					\begin{equation}
						\label{rewritten condition on kappa and theta}
						\theta([x,y])
						=
						[\theta(x), \theta(y)]
						-
						\ad_{\hlie}( \kappa(x,y) )
						\qquad
						\text{for all~$x, y \in \glie$.}
					\end{equation}
			\end{itemize*}
			We have altogether shown that the bracket~$[\ph, \ph]$ on~$\glie \oplus \hlie$ satisfies the Jacobi identity if and only if it satisfies the two conditions~\eqref{nonabelian cocycle condition} and~\eqref{rewritten condition on kappa and theta}.

			We have now shown that the bracket~$[\ph, \ph]$ on~$\glie \oplus \hlie$ is a Lie~bracket if and only if the map~$\kappa$ is alternating and the two maps~$\kappa$ and~$\theta$ satisfy the compatibility conditions~\eqref{nonabelian cocycle condition} and~\eqref{rewritten condition on kappa and theta}
	\end{enumerate}

	We have overall constructed a bijection between Lie~brackets on~$\glie \oplus \hlie$ that make the short exact sequence
	\[
		0
		\to
		\hlie
		\xto{\iota}
		\glie \oplus \hlie
		\xto{\pi}
		\glie
		\to
		0
	\]
	into a standard extension of~$\glie$ by~$\hlie$, and certains pairs~$(\kappa, \theta)$ consisting of a map~$\kappa$ from~$\glie \times \glie$ to~$\hlie$ and a map~$\theta$ from~$\glie$ to~$\Der(\hlie)$.

	We now wish to understand under what conditions two such pairs~$(\kappa_1, \theta_1)$ and~$(\kappa_2, \theta_2)$ give equivalent extensions.\index{equivalence of standard extensions}
	For this we proceed again in multiple steps.

	\begin{enumerate}[resume*]
		\item
			Let~$(\kappa_1, \theta_1)$ and~$(\kappa_2, \theta_2)$ be two pairs of maps
			\[
				\kappa_1, \kappa_2
				\colon
				\glie \times \glie
				\to
				\hlie \,,
				\quad
				\theta_1, \theta_2
				\colon
				\glie
				\to
				\Der(\hlie)
			\]
			such that the maps~$\kappa_1$ and~$\kappa_2$ are bilinear and alternating, the maps~$\theta_1$ and~$\theta_2$ are linear, and the compatibility conditions \eqref{nonabelian cocycle condition} and~\eqref{rewritten condition on kappa and theta} are satisfied for both~$\kappa_1$ and~$\theta_1$, as well as for~$\kappa_2$ and~$\theta_2$.
			Let~$[\ph, \ph]_1$ and~$[\ph, \ph]_2$ be the Lie~brackets on~$\glie \oplus \hlie$ corresponding to the pairs~$(\kappa_1, \theta_1)$ and~$(\kappa_2, \theta_2)$.
			More explicitely, we have
			\[
				[(x,c), (y,d)]_i
				=
				(
					[x,y],
					\kappa_i(x,y) + \theta_i(x)(d) - \theta_i(y)(c) + [c,d]
				)
			\]
			for all~$(x,c), (y,d) \in \glie \oplus \hlie$ and~$i = 1, 2$.

			The two resulting extensions of~$\glie$ by~$\hlie$ given by~$[\ph, \ph]_1$ and~$[\ph, \ph]_2$ are equivalent if and only if there exists an homomorphism of Lie~algebras
			\[
				\alpha
				\colon
				( \glie \oplus \hlie, [\ph, \ph]_1 )
				\to
				( \glie \oplus \hlie, [\ph, \ph]_1 )
			\]
			that makes the following diagram commute.
			\begin{equation}
				\label{diagram for equivalence of standard extensions}
				\begin{tikzcd}
					0
					\arrow{r}
					&
					\hlie
					\arrow{r}[above]{\iota}
					\arrow[equal]{d}
					&
					\glie \oplus \hlie
					\arrow{r}[above]{\pi}
					\arrow[dashed]{d}[right]{\alpha}
					&
					\glie
					\arrow{r}
					\arrow[equal]{d}
					&
					0
					\\
					0
					\arrow{r}
					&
					\hlie
					\arrow{r}[above]{\iota}
					&
					\glie \oplus \hlie
					\arrow{r}[above]{\pi}
					&
					\glie
					\arrow{r}
					&
					0
				\end{tikzcd}
			\end{equation}
			We will now explain under what conditions on the pairs~$(\kappa_1, \theta_1)$ and~$(\kappa_2, \theta_2)$ such an isomorphism~$\alpha$ exists, and how it can be constructed.
		\item
			We first want to understand how a linear map
			\[
				\alpha
				\colon
				\glie \oplus \hlie
				\to
				\glie \oplus \hlie
			\]
			has to look like to make the diagram~\eqref{diagram for equivalence of standard extensions} commute.
			The maps~$\iota$,~$\pi$ and~$\alpha$ are linear maps between direct sums, and can therefore be written in the usual matrix calculus as
			\[
				\iota
				\equiv
				\begin{pmatrix}
					0
					\\
					1
				\end{pmatrix} \,,
				\quad
				\alpha
				\equiv
				\begin{pmatrix}
					\alpha_{11} & \alpha_{12} \\
					\alpha_{21} & \alpha_{22}
				\end{pmatrix} \,,
				\quad
				\pi
				\equiv
				\begin{pmatrix}
					1 & 0
				\end{pmatrix}
			\]
			for some unique linear maps
			\[
				\alpha_{11}
				\colon
				\glie
				\to
				\glie \,,
				\quad
				\alpha_{12}
				\colon
				\glie
				\to
				\hlie \,,
				\quad
				\alpha_{21}
				\colon
				\hlie
				\to
				\glie \,,
				\quad
				\alpha_{22}
				\colon
				\hlie
				\to
				\hlie \,.
			\]
			The commutativity of the diagram~\eqref{diagram for equivalence of standard extensions} can be encoded by the two conditions
			\[
				\begin{pmatrix}
					\alpha_{11} & \alpha_{12} \\
					\alpha_{21} & \alpha_{22}
				\end{pmatrix}
				\begin{pmatrix}
					0 \\
					1
				\end{pmatrix}
				=
				\begin{pmatrix}
					0 \\
					1
				\end{pmatrix} \,,
				\quad
				\begin{pmatrix}
					1 & 0
				\end{pmatrix}
				\begin{pmatrix}
					\alpha_{11} & \alpha_{12} \\
					\alpha_{21} & \alpha_{22}
				\end{pmatrix}
				=
				\begin{pmatrix}
					1 & 0
				\end{pmatrix} \,.
			\]
			We can apply the usual rules of matrix multiplication to find that these two conditions are furthermore equivalent to the combination of the three conditions
			\[
				\alpha_{11} = 1 \,,
				\quad
				\alpha_{12} = 0 \,,
				\quad
				\alpha_{22} = 1 \,.
			\]
			We have thus found that the diagram~\eqref{diagram for equivalence of standard extensions} commutes if and only if the linear map~$\alpha$ is of the form
			\[
				\alpha
				=
				\begin{pmatrix}
					1       & 0 \\
					\varphi & 1
				\end{pmatrix}
			\]
			for some linear map
			\[
				\varphi
				\colon
				\glie
				\to
				\hlie \,.
			\]
			This means more explicitely that
			\[
				\alpha(x,c)
				=
				\alpha(x, \varphi(x) + c)
				\qquad
				\text{for all~$(x,c) \in \glie \oplus \hlie$.}
			\]
			We now understand how~$\alpha$ has to look like to make the diagram~\eqref{diagram for equivalence of standard extensions} commute.
		\item
			We will now examine what condition the linear map~$\varphi$ has to satisfy in terms of the two pairs~$(\kappa_1, \theta_1)$ and~$(\kappa_2, \theta_2)$ for~$\alpha$ to be a homomorphism of Lie~algebras.
			The map~$\alpha$ is such a homomorphism if and only if the equality
			\[
				\alpha( [ t, u ]_1 )
				=
				[ t, u ]_2
			\]
			is satisfied for any two elements~$t$ and~$u$ of~$\glie \oplus \hlie$.
			By the bilinearity of the Lie~brackets~$[\ph, \ph]_1$ and~$[\ph, \ph]_2$ and the linearity of~$\alpha$ it sufficies to consider the cases in which~$t$ is of the form~$(x,0)$ or~$(0,c)$, and similarly~$u$ is of the form~$(y,0)$ or~$(0,d)$.
			\begin{itemize}
				\item
					For~$t = (x,0)$ and~$u = (y,0)$ we find on the one hand
					\[
						\alpha( [t,u] )
						=
						\alpha( [ (x,0) , (y,0) ] )
						=
						\alpha( [x,y], \kappa_1(x,y) )
						=
						( [x,y], \varphi( [x,y] ) + \kappa_1(x,y) )
					\]
					and on the other hand
					\begin{align*}
						[ \alpha(t), \alpha(u) ]
						&=
						[ \alpha(x,0), \alpha(y,0) ]
						\\
						&=
						[ (x, \varphi(x)), (y, \varphi(y)) ]
						\\
						&=
						( [x,y], \kappa_2(x,y) + \theta_2(x)(\varphi(y)) - \theta_2(y)(\varphi(x)) + [\varphi(x), \varphi(y)] ) \,.
					\end{align*}
					We thus arrive in this case at the condition
					\[
						\kappa_1(x,y)
						=
						\kappa_2(x,y)
						+ \theta_2(x)(\varphi(y))
						- \theta_2(y)(\varphi(x))
						+ [\varphi(x), \varphi(y)]
						- \varphi( [x,y] )
					\]
					for all~$x, y \in \glie$.
				\item
					For~$t = (x,0)$ and~$u = (0,d)$ we find on the one hand
					\[
						\alpha( [v,w] )
						=
						\alpha( [ (x,0), (0,d) ] )
						=
						\alpha( 0, \theta_1(x)(d) )
						=
						(0, \theta_1(x)(d))
					\]
					and on the other hand
					\[
						[ \alpha(t), \alpha(u) ]
						=
						[ \alpha(x,0), \alpha(0,d) ]
						=
						[ (x, \varphi(x)), (0,d) ]
						=
						( 0, \theta_2(x)(d) + [\varphi(x), d] ) \,.
					\]
					We therefore arrive in this case at the condition
					\begin{equation}
						\label{first occurance of condition}
						\theta_1(x)(d)
						=
						\theta_2(x)(d)
						+ [\varphi(x), d]
					\end{equation}
					for all~$x \in \glie$ and~$d \in \hlie$.
				\item
					In the case of~$t = (0,c)$ and~$u = (y,0)$ we find on the one hand
					\[
						\alpha( [t,u] )
						=
						\alpha( [ (0,c), (y,0) ] )
						=
						\alpha( 0, -\theta_1(y)(c) )
						=
						( 0, -\theta_1(y)(c) )
					\]
					and on the other hand
					\[
						[ \alpha(t), \alpha(u) ]
						=
						[ \alpha(0,c), \alpha(y,0) ]
						=
						[ (0,c), (y, \varphi(y) ]
						=
						( 0, -\theta_2(y)(c) + [c, \varphi(y)] ) \,.
					\]
					We therefore arrive in this case at the condition
					\[
						\theta_1(y)(c)
						=
						\theta_2(y)(c) + [\varphi(y), c]
					\]
					for all~$y \in \glie$ and~$c \in \hlie$.
					But this is the same condition as in the previous case, i.e. condition~\eqref{first occurance of condition}.
				\item 
					In the case of~$t = (0,c)$ and~$u = (0,d)$ we have on the one hand
					\[
						\alpha( [t,u] )
						=
						\alpha( [ (0,c), (0,d) ] )
						=
						\alpha( 0, [c,d] )
						=
						(0, [c,d])
					\]
					and on the other hand
					\[
						[ \alpha(t), \alpha(u) ]
						=
						[ \alpha(0,c), \alpha(0,d) ]
						=
						[ (0,c), (0,d) ]
						=
						( 0, [c,d] ) \,.
					\]
					We don’t get any additional condition from this last case.
			\end{itemize}
			We find altogether that the map~$\alpha$ is a homomorphism of Lie~algebras if and only if the linear map~$\varphi$ satisfies the two conditions
			\begin{gather*} 
				\theta_1(x)
				=
				\theta_2(x)
				+ \ad_{\hlie}(\varphi(x))
			\shortintertext{and}
				\kappa_1(x,y)
				=
				\kappa_2(x,y)
				+ \theta_2(x)(\varphi(y))
				- \theta_2(y)(\varphi(x))
				+ [\varphi(x), \varphi(y)]
				- \varphi( [x,y] )
			\end{gather*}
			for all~$x, y \in \glie$.
	\end{enumerate}
\end{fluff}


\begin{fluff}
	We have now classified standard extensions in terms of certain pairs~$(\kappa, \theta)$, and have examined under what condition two such pairs~$(\kappa_1, \theta_1)$ and~$(\kappa_2, \theta_2)$ give equivalent extensions.
	This will greatly help us in the upcoming classifications of certain classes of extensions because every extension is equivalent to a standard one.

	Let us now finish our discussion of general extensions by summarizing our findings.
\end{fluff}


\begin{theorem}[Structure of extensions]
	\index{structure theorem!for extensions}
	\index{extension of Lie algebras!structure theorem}
	\label{structure of extensions}
	Let~$\glie$ and~$\hlie$ be two~\liealgebras{$\kf$}.
	\begin{enumerate}
		\item
			Every extension of~$\glie$ by~$\hlie$ is equivalent to a standard extension of~$\glie$ by~$\hlie$.
		\item
			Given a standard extension~$\elie$ of~$\glie$ by~$\hlie$, the Lie~bracket~$[\ph, \ph]$ on~$\elie$ is of the form
			\begin{equation}
				\label{formula for lie bracket on standard extension}
				[(x,c), (y,d)]
				=
				( [x,y], \kappa(x,y) + \theta(x)(d) - \theta(y)(c) + [c,d] )
			\end{equation}
			for all~$(x,c), (y,d) \in \elie$, for some bilinear map
			\begin{equation}
				\label{kappa function}
				\kappa
				\colon
				\glie \times \glie
				\to
				\hlie
			\end{equation}
			and some linear map
			\begin{equation}
				\label{theta function}
				\theta
				\colon
				\glie
				\to
				\Der(\hlie) \,.
			\end{equation}
			The maps~$\kappa$ and~$\theta$ are uniquely determined by the formulas
			\begin{alignat*}{2}
				[(x,0), (y,0)]
				&=
				([x,y], \kappa(x,y))
				&
				\qquad
				&\text{for all~$x, y \in \glie$}
			\shortintertext{and}
				[(x,0), (0,c)]
				&=
				(0, \theta(x)(c))
				&
				\qquad
				&\text{for all~$x \in \glie$ and~$c \in \hlie$.}
			\end{alignat*}
			The map~$\kappa$ is alternating, and the maps~$\kappa$ and~$\theta$ satisfy the two compatibility conditions
			\begin{gather}
				\begin{aligned}
					{}&
					\kappa([x,y], z)
					+ \kappa([y,z], x)
					+ \kappa([z,x], y)
					\label{first compatibility condition for extensions}
					\\
					={}&
					\theta(x)( \kappa(y,z) )
					+ \theta(y)( \kappa(z,x) )
					+ \theta(z)( \kappa(x,y) )
				\end{aligned}
			\shortintertext{and}
				\theta([x,y])
				=
				[\theta(x), \theta(y)] - \ad_{\hlie}(\kappa(x,y))
				\label{second compatibility condition for extensions}
			\end{gather}
			for all~$x, y, z \in \glie$.
		\item
			Let on the other hand~$\kappa$ be a bilinear map as in \eqref{kappa function} and let~$\theta$ be a linear map as in~\eqref{theta function}, such that~$\kappa$ is alternating and the two compatibility conditions~\eqref{first compatibility condition for extensions} and~\eqref{second compatibility condition for extensions} are satisfied.
			Then the formula~\eqref{formula for lie bracket on standard extension} defines a Lie~bracket on the vector space~$\glie \oplus \hlie$ which makes it into a standard extension of~$\glie$ by~$\hlie$.
		\item
			The above two constructions are mutually inverse and result in a {\onetoonetext} correspondence
			\[
				\left\{
					\begin{tabular}{c}
						Lie~brackets on~$\glie \oplus \hlie$ \\
						that make it into a \\
						standard extension \\
						of~$\glie$ by~$\hlie$
					\end{tabular}
				\right\}
				\onetoone
				\left\{
					\begin{tabular}{c}
						$(\kappa, \theta)$
					\end{tabular}
				\suchthat*
					\begin{tabular}{c}
						$\kappa \colon \glie \times \glie \to \hlie$ is bilinear, \\
						$\theta \colon \glie \to \Der(\hlie)$ is linear, \\
						$\kappa$ is alternating, and the \\
						conditions~\eqref{first compatibility condition for extensions} and~\eqref{second compatibility condition for extensions} hold
					\end{tabular}
				\right\} \,.
			\]
	\end{enumerate}
	Let now~$(\kappa_1, \theta_1)$ and~$(\kappa_2, \theta_2)$ be two pairs as described above, and let~$\elie_1$ and~$\elie_2$ be the standard extensions corresponding to these pairs.
	\begin{enumerate}[resume*]
		\item
			Let~$\alpha$ be an equivalence of extensions from~$\elie_1$ to~$\elie_2$, i.e. a homomorphism of Lie~algebras
			\[
				\alpha
				\colon
				(\elie_1, [\ph, \ph]_1)
				\to
				(\elie_2, [\ph, \ph]_2)
			\]
			that makes the following diagram commute:
			\[
				\begin{tikzcd}
					0
					\arrow{r}
					&
					\hlie
					\arrow{r}[above]{\iota}
					\arrow[equal]{d}
					&
					\elie_1
					\arrow{r}[above]{\pi}
					\arrow[dashed]{d}[right]{\alpha}
					&
					\glie
					\arrow{r}
					\arrow[equal]{d}
					&
					0
					\\
					0
					\arrow{r}
					&
					\hlie
					\arrow{r}[above]{\iota}
					&
					\elie_2
					\arrow{r}[above]{\pi}
					&
					\glie
					\arrow{r}
					&
					0
				\end{tikzcd}
			\]
			Then the map~$\alpha$ is of the form
			\begin{equation}
				\label{formula for equivalence depending on varphi}
				\alpha(x,c)
				=
				\alpha(x, \varphi(x) + c)
				\qquad
				\text{for every~$(x,c) \in \elie_1$}
			\end{equation}
			for some linear map
			\begin{equation}
				\label{functional definition of varphi}
				\varphi
				\colon
				\glie
				\to
				\hlie \,.
			\end{equation}
			The map~$\varphi$ is unique, and it satisfies the two conditions
			\begin{gather} 
				\label{first condition for equivalence}
				\theta_1(x)
				=
				\theta_2(x)
				+ \ad_{\hlie}(\varphi(x))
			\shortintertext{and}
				\label{second condition for equivalence}
				\kappa_1(x,y)
				=
				\kappa_2(x,y)
				+ \theta_2(x)(\varphi(y))
				- \theta_2(y)(\varphi(x))
				+ [\varphi(x), \varphi(y)]
				- \varphi( [x,y] )
			\end{gather}
			for all~$x, y \in \glie$.
		\item
			Suppose on the other hand that~$\varphi$ is a linear map as in~\eqref{functional definition of varphi} which satisfies the two conditions~\eqref{first condition for equivalence} and~\eqref{second condition for equivalence}.
			Then formula~\eqref{formula for equivalence depending on varphi} defines an equivalence of extensions~$\alpha$ from~$\elie_1$ to~$\elie_2$.
		\item
			The above two constructions are mutually inverse and result in a {\onetoonetext} correspondence
			\[
				\left\{
					\begin{tabular}{c}
						$\alpha$
					\end{tabular}
				\suchthat*
					\begin{tabular}{c}
						$\alpha \colon \elie_1 \to \elie_2$ is \\
						an equivalence \\
						of extensions
					\end{tabular}
				\right\}
				\onetoone
				\left\{
					\begin{tabular}{c}
						$\varphi$
					\end{tabular}
				\suchthat*
					\begin{tabular}{c}
						$\varphi \colon \glie \to \hlie$ is linear \\
						and satisfies the two \\
						conditions~\eqref{first condition for equivalence} and~\eqref{second condition for equivalence}
					\end{tabular}
				\right\}
			\]
			It holds in particular that the extensions~$\elie_1$ and~$\elie_2$ are equivalent if and only if there exists a linear map~$\varphi$ from~$\glie$ to~$\hlie$ satisfying conditions~\eqref{first condition for equivalence} and~\eqref{second condition for equivalence}.
	\end{enumerate}
\end{theorem}


\begin{remark}
	The author of these notes learned about the general form of the compatibility conditions~\eqref{first condition for equivalence} and~\eqref{second condition for equivalence} from~\cite{nlab_extension_of_lie_algebras} and~\cite{nlab_nonabelian_lie_algebra_cohomology}, with further help from~\cite{alekseevsky_michor_ruppert_extension_of_lie_algebras}.
	According to these sources, these two compatiblity conditions express that the pair~$(\kappa, \theta)$ is a~\cocycle{$2$}\index{2-cocycle}\index{coycle@2-cocycle} in nonabelian Lie~algebra cohomology\index{nonabelian Lie algebra cohomology}\index{Lie algebra cohomology!nonabelian}.
\end{remark}


\begin{warning}
	Suppose that we are given any extension of Lie~algebras
	\[
		0 \to \hlie \to \elie \to \glie \to 0 \,.
	\]
	Then according to \cref{structure of extensions} this extension is equivalent to a standard extension~$\elie'$.
	This equivalence is realized by a suitable isomorphism of vector spaces~$\alpha$ from~$\elie$ to~$\elie'$.

	The Lie~bracket of~$\elie'$ can now be parametrized by a pair~$(\kappa, \theta)$ as explained in \cref{structure of extensions}.
	The Lie~bracket of~$\elie$ can therefore be expressed by the pair~$(\kappa, \theta)$.
	However, this pair~$(\kappa, \theta)$ does not only depend on the Lie~bracket of~$\elie$, but also on the choice of isomorphism~$\alpha$.
\end{warning}


\begin{definition}
	In the bijection from \cref{structure of extensions} the standard extension corresponding to a pair~$(\kappa, \theta)$ is \defemph{the standard extension given by~$(\kappa, \theta)$}\index{standard extension of Lie algebras}, or simple \defemph{the standard extension~$(\kappa, \theta)$}.
\end{definition}





