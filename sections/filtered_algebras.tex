\section{Filtered \texorpdfstring{$\kf$}{k}-Algebras}



\subsection{Definition}

\begin{definition}
  Let~$A$ be a~\algebra{$\kf$}.
  A \defemph{filtration}\index{filtration} of~$A$ is an increasing sequence
  \[
    A_{(0)}
    \subseteq
    A_{(1)}
    \subseteq
    A_{(2)}
    \subseteq
    \dotsb
  \]
  of linear subspaces of~$A$ such that~$A = \bigcup_{p \geq 0} A_{(p)}$ and~$A_{(p)} A_{(q)} \subseteq A_{(p+q)}$ for all~$p, q \geq 0$, as well as~$1 \in A_{(0)}$.
  A \defemph{filtered~\algebra{$\kf$}} is a~\algebra{$\kf$}~$A$ together with a filtration of~$A$.
\end{definition}


\begin{remark}
  \label{filtration conventions}
  \leavevmode
  \begin{enumerate}
    \item
      We often say that~\enquote{$A$ is a filtered algebra} without explicitely mentioning the filtration.
      The parts of the filtration will then be denoted by~$A_{(p)}$.
    \item
      If~$A$ is a filtered~{\algebra{$\kf$}} then we set~$A_{(-p)} \defined 0$ for all~$p < 0$ for convenience.
      The relation~$A_{(p)} A_{(q)} \subseteq A_{(p+q)}$ does then holds for all~$p, q \in \Integer$.
  \end{enumerate}
\end{remark}


\begin{definition}
  Let~$A$ and~$B$ be two filtered~\algebras{$\kf$}.
  \begin{enumerate}
    \item
      A~\defemph{homomorphism of filtered algebras} from~$A$ to~$B$ is a homomorphism of algebras~$\Phi$ from~$A$ to~$B$ such that~$\Phi( A_{(p)} ) \subseteq B_{(p)}$ for all~$p \geq 0$.
    \item
      Let~$\Phi$ be a homomorphism of filtered algebras from~$A$ to~$B$.
      Then the restiction of~$\Phi$ to a linear map from~$A_{(p)}$ to~$B_{(p)}$ is denoted by~$\Phi_{(p)}$ for all~$p \geq 0$.
  \end{enumerate}
\end{definition}


\begin{remark}
  Let~$A$,~$B$ and~$C$ be filtered~\algebras{$\kf$}.
  \begin{enumerate}
    \item
      The identity map~$\id_A$ is a homomorphism of filtered algebras from~$A$ to~$A$.
    \item
      Let~$\Phi$ be a homomorphism of filtered algebras from~$A$ to~$B$ and let~$\Psi$ be a homomorphisms of filtered~\algebras{$\kf$} from~$B$ to~$C$.
      Their composite~$\Psi \circ \Phi$ is a homomorphism of filtered algebras from~$A$ to~$C$.
  \end{enumerate}
  This shows that filtered~\algebras{$\kf$} together with homomorphisms of filtered algebras between them form a category.
  We will denote this category by~$\cfAlg{\kf}$\glsadd{filtered algebras}.
  \begin{enumerate}[resume]
    \item
      A homomorphism of filtered algebras~$\Phi$ from~$A$ to~$B$ is an isomorphism if and only if there exists a homomorphism of filtered algebras~$\Psi$ from~$B$ to~$A$ with~$\Psi \circ \Phi = \id_A$ and~$\Phi \circ \Psi = \id_B$.
  \end{enumerate}
\end{remark}


\begin{warning}
  Let~$A$ and~$B$ be two filtered~{\algebras{$\kf$}}.
  A bijective homomorphism of filtered algebra~$\Phi$ from $A$ to~$B$ is not necessarily an isomorphism of filtered algebras.
  
  Indeed, let~$A$ be some~\algebra{$\kf$} with~$A \neq \kf$.
  We can endow~$A$ with a filtration
  \[
    A_{(0)}
    =
    \kf
    \subseteq
    A
    =
    A
    =
    A
    =
    \dotsb
  \]
  which results in a filtered~{\algebra{$\kf$}}~$B_1$.
  But we can also endow~$A$ with the filtration
  \[
    A_{(0)}
    =
    A
    =
    A
    =
    A
    =
    A
    =
    \dotsb
  \]
  which results in a filtered~{\algebra{$\kf$}}~$B_2$.
  The identity~$\id_A$ is a bijective homomorphism of filtered~algebras from~$B_1$ to~$B_2$.
  But this is not an isomorphism of filtered algebras because its set-theoretic inverse, which is just~$\id_A$ again, does not map the filtration of~$B_2$ into the filtration of~$B_1$.
\end{warning}


\begin{remark}
  \leavevmode
  \begin{enumerate}
    \item
      Let~$A$ be a~\algebra{$\kf$}.
      Any grading~$A = \bigoplus_{p \geq 0} A_p$ of~$A$ results in a filtration~$A = \bigcup_{p \geq 0} A_{(p)}$ of~$A$ given by
      \[
        A_{(p)}
        =
        \bigoplus_{q \leq p} A_q
      \]
      for all~$p \geq 0$.
      We can therefore regard every graded~{\algebra{$\kf$}} as a filtered~{\algebra{$\kf$}}.
      Every homomorphism of graded algebras is then also a hommorphism of filtered algebras.
      This construction gives us therefore a forgetful functor from~$\cgAlg{\kf}$ to~$\cfAlg{\kf}$.
    \item 
      Let~$A$ be a filtered algebra, let~$I$ is a two-sided ideal in~$A$ andl let~$\Pi$ be the canonical quotient homomorphism from~$A$ to~$A/I$.
      The quotient algebra~$A/I$ inherits from~$A$ a filtration given by~$(A/I)_{(p)} \defined \Pi(A_{(p)})$ for all~$p \geq 0$.
  \end{enumerate}
\end{remark}


\begin{examples}
  \label{examples for filtered algebras}
  \leavevmode
  \begin{enumerate}
    \item
      Let~$V$ be a vector space.
      The tensor algebra~$\Tensor(V)$, the symmetric algebra~$\Symm(V)$ and the exterior algebra~$\Exterior(V)$ admit gradings as explained in \cref{examples for graded algebras}.
      They can hence be endowed with the resulting filtrations.
    \item
      Let~$\glie$ be a Lie~algebra.
      The universal enveloping algebra~$\Univ(\glie)$ inherits from the tensor algebra~$\Tensor(\glie)$ a filtration~$\Univ(\glie) = \bigcup_{p \geq 0} \Univ(\glie)_{(p)}$.
      This filtration is given by
      \[
        \Univ(\glie)_{(p)}
        =
        \gen{
          \class{x_1 \dotsm x_q}
        \suchthat
          q \leq p,
          x_1, \dotsc, x_q \in \glie
        }_{\kf}
      \]
      for all~$p \geq 0$.
    \item
      Let~$M$ be a multiplicative monoid and let
      \[
        M_{(0)}
        \subseteq
        M_{(1)}
        \subseteq
        M_{(2)}
        \subseteq
        M_{(3)}
        \subseteq
        \dotsb
      \]
      be a filtration of~$M$, i.e. it holds that~$M = \bigcup_{p \geq 0} M_{(p)}$ with~$M_{(p)} M_{(q)} \subseteq M_{(p+q)}$ for all~$p, q \geq 0$, and~$1 \in M_{(0)}$.
      The monoid algebra~$A \defined \kf[M]$ inherits a filtration given by
      \[
        A_{(p)}
        =
        \gen{ M_{(p)} }_{\kf}
      \]
      for all~$p \geq 0$.
    \item
      Let us give a special case of the previous example.

      Let~$G$ be a group and let~$S$ be a generating set of~$G$.
      The \defemph{length} of an element~$g$ of~$G$ with respect to the generating set~$S$ is given by
      \[
        \ell_S(g)
        \defined
        \min
        \left\{
          n \geq 0
        \suchthat*
          \begin{tabular}{@{}c@{}}
            there exist~$s_1, \dotsc, s_n \in S$ \\
            and~$\varepsilon_1, \dots, \varepsilon_n \in \{1 -1\}$ \\
            with~$g = s_1^{\varepsilon_1} \dotsm s_n^{\varepsilon_n}$
          \end{tabular}
        \right\}  \,.
      \]
      The length function is subadditive in the sense that
      \[
        \ell_S(gh)
        \leq
        \ell_S(g) + \ell_S(h)
      \]
      for all~$g, h \in G$.
      For all~$p \geq 0$ let
      \[
        G_{(p)}
        \defined
        \{
          g \in G
        \suchthat
          \ell_S(g) \leq p
        \}  \,.
      \]
      This is the ball of radius~$p$ with respect to the generating set~$S$.
      These subsets given a filtration of the group~$G$.
      The condition~$G_{(p)} G_{(q)} \subseteq G_{(p + q)}$ for all~$p, q \geq 0$ follows from the subadditivity of the length function~$\ell_S$.
      It follows that the group algebra~$\kf[G]$ inherits a filtration given by
      \[
        \kf[G]_{(p)}
        \defined
        \gen{ G_{(p)} }_{\kf}
        =
        \gen{
          g \in G
        \suchthat
          \ell_S(g) \leq p
        }_{\kf} 
      \]
      for all~$p \geq 0$.
  \end{enumerate}
\end{examples}


\begin{definition}
  Let~$A$ be a filtered algebra.
  The \defemph{degree}\index{degree!filtration} of an element~$x$ of~$A$ is the minimal natural number~$p$ for which the element~$x$ is contained~$A_{(p)}$.
  This degree is denoted by~$\deg(x)$.
\end{definition}


\begin{example}
  \leavevmode
  \begin{enumerate}
    \item
      Let~$A \defined \kf[t]$ be the polynomial ring in one variable.
      The standard grading~$\kf[t] = \bigoplus_{i \geq 0} \kf t^i$ gives the filtration
      \[
        A_{(0)}
        =
        \kf
        =
        \gen{ 1 }_{\kf}
        \subseteq
        \gen{ 1, t }_{\kf}
        \subseteq
        \gen{ 1, t, t^2 }_{\kf}
        \subseteq
        \gen{ 1, t, t^2, t^3 }_{\kf}
        \subseteq
        \dotsb
      \]
      The degree of a nonzero polynomial with respect to this filtration is the usual degree of a polynomial.
    \item
      Let more generall~$A$ be a graded algebra with associated filtration.
      Then the degree of a nonzero element~$x$ of~$A$ with homogeneous decomposition~$x = \sum_{p \geq 0} x_p$ is the maximal natural number~$p$ for which the homogeneous component~$x_p$ is nonzero.
  \end{enumerate}
\end{example}



\subsection{The Associated Graded Algebra}

\subsubsection{Construction}

\begin{definition}
  Let~$A$ be filtered algebra.
  Two elements~$x$ and~$y$ ~$A$ are \defemph{equal up to smaller degree}\index{equal up to smaller degree}\index{up to smaller degree}\index{degree!up to smaller} if either~$x = y$ or~$\deg(x-y) < \deg(x), \deg(y)$.
\end{definition}


\begin{proposition}
  Let~$A$ be a filtered algebra and let~$x$ and~$y$ be two elements of~$A$.
  The elements~$x$ and~$y$ are equal up to smaller degree if and only if they have the same degree~$d$ and their difference~$x - y$ is contained in~$A_{(d-1)}$.
\end{proposition}


\begin{proof}
  Suppose first that that the elements~$x$ and~$y$ are equal up to smaller degree.
  Let~$x$ be of degree~$d$ and let~$y$ be of degree~$d'$ with~$d \geq d'$.
  If~$x = y$ then~$d = d'$ and the difference~$x - y = 0$ is contained in~$A_{(d-1)}$.

  Suppose now in addition that~$x \neq y$.
  Then by assumption~$\deg(x - y) < \deg(y) = d'$.
  It follows that the difference~$x - y$ is contained in~$A_{(d')}$.
  But the element~$y$ is also contained in~$A_{(d')}$.
  It follows that~$x = (x-y) + y$ is contained in~$A_{(d')}$.
  Therefore~$d' \leq d$ by choice of~$d$, and thus~$d = d'$.
  It also follows from the condition~$\deg(x-y) < \deg(x) = d$ that the difference~$x-y$ is contained in~$A_{(d-1)}$.
 
  Suppose now on the other hand that the two elements~$x$ and~$y$ have the same degree~$d$ and that their difference~$x - y$ is contained in~$A_{(d-1)}$.
  If~$d = 0$ then~$A_{(d-1)} = A_{-1} = 0$ and it follws that~$x = y$.
  If~$d > 0$ then~$d-1$ is a natural number and it follows from the condition~$x - y \in A_{(d-1)}$ that~$\deg(x-y) < d = \deg(x), \deg(y)$.
\end{proof}


\begin{corollary}
  Let~$A$ be a filtered~\algebra{$\kf$}.
  The notion of \enquote{being equal up to smaller degree} is an equivalence relation on~$A$.
  The equivalence class of an element~$x$ of~$A$ of degree~$d$ is given by the coset~$x + A_{(d-1)}$.
  \qed
\end{corollary}


\begin{fluff}
  Let~$A$ be a filtered~{\algebra{$\kf$}} and let~$\sim$ be the equivalence relation \enquote{equal up to smaller degree} on~$A$.
  
  When calculating in~$A$ we sometimes want to replace an element~$x \in A$ by another element~$x' \in A$ that is equal to~$x$ up to smaller degree, while hoping that the result ouf our calculation also stays the same up to smaller degree.
  This can be useful if terms of smaller degree are not important in the given situation, or if they can be dealt with by induction.
  
  We would therefore like to have the properties
  \begin{equation}
    \label{wanted compatibility}
    x + y
    \sim
    x' + y' \,,
    \qquad
    x \cdot y
    \sim
    x' \cdot y' \,,
    \qquad
    \lambda x \sim \lambda x'
  \end{equation}
  for all elements~$x$,~$x'$~$y$,~$y'$ of~$A$ with~$x \sim x'$ and~$y \sim y'$ and all scalars~$\lambda$ in~$\kf$.
  This would then mean that the quotient set~$A/{\sim}$ inherits from~$A$ the structure of a~{\algebra{$\kf$}}, which would allow us to do calculations \enquote{up to smaller degree} by switching from~$A$ to~$A/{\sim}$.
  
  But alas the properties~\eqref{wanted compatibility} do not hold in general.%
  \begin{itemize}
    \item
      The equivalence relation~$\sim$ is not necessarily compatible with addition.

      As a counterexample we can take for~$A$ the polynomial algebra~$\kf[t]$ wih the filtration induced by the standard grading~$A = \bigoplus_{d \geq 0} \gen{ t^d }_{\kf}$.
      We can the consider the elements~$x = t$ and~$x' = t+1$ as well as the elements~$y, y' = -t$.
      Then~$x$ and~$x'$ are equal up to smaller degree and~$y$ and~$y'$ are equal, but~$x + y = 0$ and~$x' + y' = 1$ are not equal up to smaller degree.
    \item
      The equivalence relation~$\sim$ also hasn’t to be compatible with multiplication.
      
      We take on the~\algebra{$\kf$}~$\kf[t]$ the standard grading~$\kf[t] = \bigoplus_{d \geq 0} \gen{ t^d }_{\kf}$.
      The ideal~$\ideal{I}$ of~$\kf[t]$ is homogeneous, whence the quotient~$A \defined \kf[t] / \ideal{ t^2 }$ inherits a grading given by
      \[
        A
        =
        \kf
        \oplus
        \kf t
        \oplus
        0
        \oplus
        0
        \oplus
        \dotsb
      \]
      The resulting filtration of~$A$ is given by
      \[
        A_0
        =
        \kf
        \subsetneq
        A
        =
        A
        =
        A
        =
        \dotsb
      \]
      The two elements~$x = t$ and~$x' = t+1$ are equal up to smaller degree but for~$y, y' = t$ the two products~$xy = 0$ and~$x' y' = t$ are not equal up to smaller degree.
    \item
      The equivalence relation~$\sim$ is actually compatible with scalar multiplication.

      Suppose that~$x$ and~$x'$ are two element of~$A$ that are equal up to smaller degree.
      This means that~$x$ and~$x'$ are of the same degree~$d$ and that th difference~$x - x'$ is contained in~$A_{(d-1)}$.
      For~$\lambda = 0$ we have~$\lambda x = 0 = \lambda x'$ and hence~$\lambda x \sim \lambda x'$.
      For~$\lambda \neq 0$ we observe that~$x$ is contained in~$A_{(d')}$ for some~$d' \geq 0$ if and only if~$\lambda x$ is contained in~$A_{(d')}$, and similar for~$x'$.
      We thus find that~$\lambda x$ and~$\lambda x'$ have tha same degrees as~$x$ as~$x'$, and thus the same degree.
      The difference~$\lambda x - \lambda x' = \lambda (x - x')$ is contained in~$A_{(d-1)}$ because~$x - x'$ is contained in~$A_{(d-1)}$.
  \end{itemize}

  We can also argue in a more abstract way why the set~$A / {\sim}$ can in general not inherit the structure of a~\algebra{$\kf$} from~$A$.
  We would otherwise have~$A/{\sim} = A/I$ for a two-sided ideal~$I$ of~$A$, which would be given by~$I = \{x \in A \suchthat x \sim 0\}$.
  But the only element that is equal to~$0$ up to smaller degree is~$0$ itself.
  Hence~$I = 0$ and we would have that~$A/{\sim} = A / I = A / 0 = A$.
  This would mean that~$\sim$ is the trivial equivalence relation.
\end{fluff}


\begin{fluff}
  The following construction gives us a way to circumvent these above problems and to calculate with elements of~$A$ \enquote{up to smaller degree} in a rigorous way.
\end{fluff}


\begin{construction}[The associated graded algebra]
  \label{construction of associated graded}
  To every filtered~{\algebra{$\kf$}}~$A$ we can associate a graded~{\algebra{$\kf$}} as follows.

  For every~$p \geq 0$ let
  \[
    \gr[p](A)
    \defined
    A_{(p)} / A_{(p-1)}
  \]
  where we use the convention~$A_{(-1)} = 0$ from \cref{filtration conventions}.
  We denote the residue class of an element~$x$ of~$A_{(p)}$ in~$\gr[p](A)$ by~$\fclass{x}_p$.
  
  The multiplication of~$A$ restricts for any two natural numbers~$p$,~$q$ to a bilinear map
  \[
    A_{(p)} \times A_{(q)} \to A_{(p+q)} \,,
  \]
  which in turn induces a well-defined bilinear map
  \[
    \mu_{p,q}
    \colon
    \gr[p](A) \times \gr[q](A)
    \to
    \gr[p+q](A) \,,
    \quad
    (\fclass{x}_p, \fclass{y}_q)
    \mapsto
    \fclass{xy}_{p+q}  \,.
  \]
  We will write~$\fclass{x}_p \cdot \fclass{y}_q$ or just~$\fclass{x}_p \fclass{y}_q$ instead of~$\mu_{p,q}(\fclass{x}_p, \fclass{y}_q)$.
  To see that the multipliction map~$\mu_{p,q}$ is well-defined let~$x$,~$x'$ be elements of~$A_{(p)}$ and let~$y$,~$y'$ be elements of~$A_{(q)}$ such that~$x - x ' \in A_{(p-1)}$ and~$y - y' \in A_{(q-1)}$.
  Then
  \begin{align*}
    x y
    &=
    (x' + (x - x')) (y' + (y - y'))
    \\
    &=
    x' y' + (x - x') y' + x' (y - y') + (x - x')(y - y')
    \\
    &\in
    x' y' + A_{(p-1)} A_{(q)} + A_{(p)} A_{(q-1)} + A_{(p-1)} A_{(q-1)}
    \\
    &\subseteq
    x' y' + A_{(p+q-1)} + A_{(p+q-1)} + A_{(p+q-2)}
    \\
    &\subseteq
    x' y' + A_{(p+q-1)} \,,
  \end{align*}
  and hence~$\fclass{xy}_{p+q} = \fclass{x'y'}_{p+q}$.
  
  The element~$[1]_0$ of~$\gr[0](A)$ satisfies the property~$[1]_0 \cdot \fclass{x}_p = \fclass{x}_p$ for all~$p \geq 0$,~$\fclass{x}_p \in \gr[p](A)$, and the multiplications~$\mu_{p,q}$ are relatively associative in the sense that
  \[
    \fclass{x}_p \cdot (\fclass{y}_q \cdot [z]_r)
    =
    [xyz]_{p+q+r}
    =
    (\fclass{x}_p \cdot \fclass{y}_q) \cdot [z]_r
  \]
  for all~$p, q, r \geq 0$,~$\fclass{x}_p \in \gr[p](A)$,~$\fclass{y}_q \in \gr[q](A)$ and~$[z]_r \in \gr[r](A)$.
  It follows from \cref{external description of graded algebras} that on the direct sum
  \[
    \gr(A)
    \defined
    \bigoplus_{p \geq 0} \gr[p](A)
  \]
  the partial multiplications~$\mu_{p,q}$ assemble into a single multiplication
  \[
    \mu
    \colon
    \gr(A) \times \gr(A)
    \to
    \gr(A)
  \]
  which makes~$\gr(A)$ into a graded~{\algebra{$\kf$}}.
  The multiplicative neutral element of~$\gr(A)$ is given by~$[1]_0$.
\end{construction}


\begin{definition}
  For a filtered~{\algebra{$\kf$}}~$A$ the graded algebra~\gls*{associated graded} resulting from~$A$ by \cref{construction of associated graded} is the \defemph{associated graded algebra}\index{associated graded algebra} of~$A$.
\end{definition}


\begin{fluff}
  Let~$A$ be a filtered algebra.
  \begin{enumerate}
    \item
      To every element~$x$ of~$A$ we can associate an element in~$\gr(A)$ as follows.

      We observe that if~$x$ is of degree~$d$ then the residue class~$\fclass{x}_p$ is not defined for~$p < d$ because~$x$ is not contained in~$A_{(q)}$, whereas this residue class is defined for all~$p \geq d$.
      But for~$p > d$ we find that~$\fclass{x}_p = 0$ because the element~$x_p$ is contained in~$A_{(p-1)}$.
      The only interesting residue class associated to~$x$ is therefore~$\fclass{x}_p$.
      We also note that this residue class~$\fclass{x}_p$ is nonzero if~$x$ is nonzero, because then~$x$ is not contained in~$A_{(p-1)}$.
    \item
      We observe that two elements~$x$ and~$y$ of~$A$ give the same associated element in~$\gr(A)$ if and only if~$x$ and~$y$ are equal up to smaller degree.
      To show this we denote by~$\gamma$ the set-theoretic function from~$A$ to~$\gr(A)$ that maps ever every element~$x$ of~$A$ to the associated element of~$\gr(A)$.
      
      Suppose first that the elements~$x$ and~$y$ are equal up to smaller degree.
      Then~$x$ and~$y$ are of the same degree~$d$ and the difference~$x - y$ is containted in~$A_{(d-1)}$.
      Hence
      \[
        \gamma(x) = \fclass{x}_d = \fclass{y}_d = \gamma(y) \,.
      \]
      
      Suppose now that~$\gamma(x) = \gamma(y)$.
      If the elements~$x$ and~$y$ had different degrees then~$\gamma(x)$ and~$\gamma(y)$ would be contained in different homogeneous parts of~$\gr(A)$.
      This would contradict the assumption~$\gamma(x) = \gamma(y)$.
      The elements~$x$ and~$y$ are thus of the same degree~$d$.
      In now further follows from the equality
      \[
        \fclass{x}_d
        =
        \gamma(x)
        =
        \gamma(y)
        =
        \fclass{y}_d
      \]
      that~$\fclass{x - y}_d = 0$ and thus~$x - y \in A_{(d-1)}$.
      This shows that~$x$ and~$y$ are equal up to smaller degree.
    \item
      We observe that the image of the map~$\gamma$ consists precisely of the homogeneous elements of~$\gr(A)$.
      If an element~$x$ of~$A$ has degree~$d$ then the associated element~$\fclass{x}_d$ is homogeneous of degree~$d$.
  \end{enumerate}
\end{fluff}


\begin{warning}
  \label{generators of associated graded}
  Let~$A$ be a filtered~{\algebra{$\kf$}}
  \begin{enumerate}
    \item
      The map~$\gamma$ from~$A$ to~$\gr(A)$ which assigns to each element~$x$ of~$A$ the corresponding element of~$\gr(A)$ is very much not a homomorphism.
      It is in general neither multiplicative nor additive.
    \item
      \label{generators of associated graded part}
      Suppose that~$x_i$ with~$i$ in~$I$ is a generating set for the algebra~$A$.
      Then the associated elements~$\gamma(x_i)$ of~$\gr(A)$ do not have to form a generating set for~$\gr(A)$.
      We will give an explicit counterexample in~\cref{converse to warning about generating set for the associated graded}.
  \end{enumerate}
\end{warning}


\begin{remark}
  Let~$A$ be a filtered algebra with finite algebra generating set~$x_1, \dotsc, x_n$ such that each generator~$x_i$ is of degree~$d_i$.
  Then the associated elements in~$\gr(A)$ form a set of algebra generators of~$\gr(A)$ if and only if the generators~$x_1, \dotsc, x_n$ are compatible with the filtration of~$A$ in the sense that for every~$p \geq 0$ the linear subspace~$A_{(p)}$ of~$A$ is spanned by all those monomials~$x_{i_1} \dotsm x_{i_r}$ with~$r \geq 0$,~$d_{i_r} + \dotsb + d_{i_r} \leq p$.
  We refer to \cite{associated_generated} for more details on this.
\end{remark}

\subsubsection{Functoriality}

\begin{construction}
  Let~$A$ and~$B$ be two filtered~\algebra{$\kf$}.
  
  Let~$\Phi$ be a homomorphism of filtered algebras from~$A$ to~$B$.
  Then for every natural number~$p$ the restriction~$\Phi_{(p)}$ induces a linear map
  \[
    \gr[p](\Phi)
    \colon
    \gr[p](A)
    \to
    \gr[p](B) \,,
    \quad
    \fclass{x}_p
    \mapsto
    [\Phi(x)]_p  \,.
  \]
  These linear maps come together to form a linear map
  \[
    \gr(\Phi)
    \colon
    \gr(A)
    \to
    \gr(B)  \,.
  \]
  This map is already homomorphism of algebras because
  \begin{align*}
    \gr(\Phi)( \fclass{x}_p \cdot \fclass{y}_q )
    &=
    \gr(\Phi)( \fclass{xy}_{p+q} )
    \\
    &=
    [ \Phi(xy) ]_{p+q}
    \\
    &=
    [ \Phi(x) \cdot \Phi(y) ]_{p+q}
    \\
    &=
    [ \Phi(x) ]_p \cdot [ \Phi(y) ]_q
    \\
    &=
    \gr(\Phi)(\fclass{x}_p) \cdot \gr(\Phi)(\fclass{y}_q) \,.
  \end{align*}
  for all~$p, q \geq 0$,~$\fclass{x}_p \in \gr[p](A)$,~$\fclass{y}_q \in \gr[q](A)$, and
  \[
    \gr(\Phi)([1]_0)
    =
    [ \Phi(1) ]_0
    =
    [1]_0 \,.
  \]
  The map~$\gr(\Phi)$ is thus a homomorphism of graded algebras.

  It holds for every filtered~\algebra{$\kf$}~$A$ that~$\gr(\id_A) = \id_{\gr(A)}$.
  It also holds for every homomorphism of filtered algebras~$\Phi$ from~$A$ to~$B$ and every homomorphism of filtered algebras~$\Psi$ from~$B$ to~$C$ that~$\gr(\Psi \circ \Phi) = \gr(\Psi) \circ \gr(\Phi)$.
  
  We have hence constructed a functor~$\gr$ from~$\cfAlg{\kf}$ to~$\cgAlg{\kf}$.
\end{construction}


\begin{proposition}
  Let~$A$ be a graded~{\algebra{$\kf$}}.
  Then the linear map~$\Phi$ from~$A$ to~$\gr(A)$ given by~$\Phi(x) = \fclass{x}_p$ for all~$p \geq 0$,~$x \in A_p$ is an isomorphism of graded~algebras.
\end{proposition}


\begin{proof}
  We have for every natural number~$p$ that~$A_{(p)} = \bigoplus_{q=0}^p A_q$ and hence
  \[
    \gr[p](A)
    =
    A_{(p)} / A_{(p-1)}
    \cong
    A_p
  \]
  as vector spaces.
  It follows that~$\Phi$ is an isomorphism of vector spaces.
  We have for all natural numbers~$p$,~$q$ and homogeneous elements~$x$ in~$A_p$ and~$y$ in~$A_q$ that
  \[
    \Phi(x) \cdot \Phi(y)
    =
    \fclass{x}_p \cdot \fclass{y}_q
    =
    \fclass{xy}_{p+q}
    =
    \Phi(xy) \,,
  \]
  where the last equality holds because the element~$xy$ is homogeneous of degree~$p + q$.
  This shows that~$\Phi$ is multiplicative and hence already an isomorphism of graded algebras.
\end{proof}


\begin{warning}
  If~$A$ is a filtered~{\algebra{$\kf$}} for which the filtration does not come from a grading then~$A$ and~$\gr(A)$ are not isomorphic as filtered algebras (where the filtration of~$\gr(A)$ is induced by its grading).
\end{warning}

\subsubsection{Zero Divisors}

\begin{definition}
  Let~$A$ be a ring.
  \begin{enumerate}
    \item
      An element~$x$ of~$A$ is a \defemph{left zero divisor}\index{zero divisor!left} if there exists a nonzero element~$y$ of~$A$ with~$xy = 0$.
    \item
      An element~$x$ of~$A$ is a \defemph{right zero divisor}\index{zero divisor!right} if there exists a nonzero element~$y$ of~$A$ with~$yx = 0$.
    \item
      An element~$x$ of~$A$ is a \defemph{zero divisor}\index{zero divisor} if it is a left zero divisor or a right zero divisor.
  \end{enumerate}
\end{definition}


\begin{remark}
  Let~$A$ be a ring.
  \begin{enumerate}
    \item
      An element~$x$ of~$A$ is a left zero divisor in~$A$ if and only if the left multiplication map
      \[
        A \to A \,,
        \quad
        y \mapsto xy
      \]
      is not injective.
      The element~$x$ is a right zero divisor in~$A$ if and only if the right multiplication map
      \[
        A \to A \,,
        \quad
        y \mapsto yx
      \]
      is not injective.
    \item
      The zero element of~$A$ is both a left zero divisor and right zero divisor if~$A$ is nonzero.
      If~$A$ is the zero algebra then the zero element is not a zero divisor.
  \end{enumerate}
\end{remark}


\begin{proposition}
  \label{associated graded algebra and zero divisors}
  Let~$A$ be a filtered~{\algebra{$\kf$}}.
  \begin{enumerate}
    \item
      \label{associated graded has left zero divisors}
      If~$\gr(A)$ doesn’t contain any nonzero left zero divisor then neither does~$A$.
    \item
      \label{associated graded has right zero divisors}
      If~$\gr(A)$ doesn’t contain any nonzero right zero divisor then neither does~$A$.
    \item
      If~$\gr(A)$ doesn’t contain any nonzero zero divisor then neither does~$A$.
  \end{enumerate}
\end{proposition}


\begin{proof}
  \leavevmode
  \begin{enumerate}
    \item
      Suppose that the algebra~$A$ contains a nonzero left zero divisor~$x$.
      Then there exists some nonzero element~$y$ of~$A$ with~$xy = 0$.
      If the element~$x$ is of degree~$p$ and the element~$y$ is of degree~$q$ then both~$\fclass{x}_p$ and~$\fclass{y}_q$ are nonzero.
      But
      \[
        \fclass{x}_p \cdot \fclass{y}_q
        =
        \fclass{xy}_{p+q}
        =
        [0]_{p+q}
        =
        0 \,.
      \]
      This shows that the element~$\fclass{x}_p$ is again a nonzero left zero divisor.
    \item
      This can be shown in the same way as part~\ref*{associated graded has left zero divisors}.
    \item
      This is a combination of parts~\ref*{associated graded has left zero divisors} and~\ref*{associated graded has right zero divisors}.
    \qedhere
  \end{enumerate}
\end{proof}


\begin{remark}
  The converse of \cref{associated graded algebra and zero divisors} does not hold.
  More explicitely, it may happen that the algebra~$\gr(A)$ has nonzero zero divisors even though~$A$ doesn’t have any.
  To see this let~$A$ be any~\algebra{$\kf$} with~$\kf \subsetneq A$ and consider the filtration
  \[
    A_{(0)}
    =
    \kf
    \subsetneq
    A
    \subseteq
    A
    \subseteq
    A
    \subseteq
    \dotsb
  \]
  Then~$\gr[0](A) = \kf$, the homogeneous part~$\gr[1](A) = A / {\kf}$ is nonzero, but all homogeneous parts~$\gr[p](A)$ with~$p \geq 2$ vanish.
  Thus~$\gr[1](A) \cdot \gr[1](A) = 0$ even though~$\gr[1](A)$ is nonzero.
  This means that every elements of~$\gr[1](A)$ is both a left zero divisor and right zero divisor.
\end{remark}

\subsubsection{Ideals and Chain Conditions}

\begin{proposition}
  \label{associated graded ideals}
  Let~$A$ be a filtered algebra. 
  \begin{enumerate}
    \item
      Let~$I$ be some kind of ideal of~$A$.
      For every natural number~$p$ let
      \[
        \gr[p](I)
        \defined
        \bigl( A_{(p-1)} + I \cap A_{(p)} \bigr) / A_{(p-1)} \,,
      \]
      and let~$\gr[p](I) \defined \bigoplus_{p \geq 0} \gr[p](I)$.%
      \footnote{The term~$(A_{(p-1)} + I) \cap A_{(p)} = A_{(p-1)} + (I \cap A_{(p)})$ does not depend on the choice of parenthesization because the lattice of subspaces is modular.}
      This linear subspace~$\gr(I)$ of~$\gr(A)$ is a homogeneous ideal of~$A$, of the same kind as the original ideal~$I$.
    \item
      \label{pulling back generating set from graded ideal}
      Let~$S$ be a subset of~$I$.
      If the associated elements~$[s]_{\deg(s)}$ with~$s$ in~$S$ generate the homogeneous ideal~$\gr(I)$ then~$S$ generates the original ideal~$I$.
    \item
      The mapping
      \begin{align*}
        \{ \text{ideals in~$A$} \}
        &\to
        \{ \text{homogeneous ideals in~$\gr(A)$} \}  \,,
        \\
        I
        &\mapsto
        \gr(I)
      \end{align*}
      is strictly order preserving, i.e.\ if~$I$ and~$J$ are ideals of~$A$ such that~$I$ is strictly contained in~$J$ then~$\gr(I)$ is also strictly contained in~$\gr(J)$.
  \end{enumerate}
\end{proposition}


\begin{proof}
  \leavevmode
  \begin{enumerate}
    \item
      The graded components of~$\gr(A)$ can be described as
      \begin{align*}
        \gr[p](A)
        &=
        \bigl( A_{(p-1)} + I \cap A_{(p)} \bigr) / A_{(p-1)}
        \\
        &=
        \{
          y + A_{(p-1)}
        \suchthat
          y \in A_{(p-1)} + I \cap A_{(p)}
        \}
        \\
        &=
        \{
          y + A_{(p-1)}
        \suchthat
          y \in I \cap A_{(p)}
        \}
        \\
        &=
        \{
          \fclass{ y }_{p}
        \suchthat
          y \in I \cap A_{(p)}
        \}
      \end{align*}
      for all~$p \geq 0$.

      Suppose now that~$I$ is a left ideal of~$A$.
      It holds for every element~$x$ of~$A_{(p)}$ and every element~$y$ of~$I \cap A_{(q)}$ that
      \[
        xy
        \in
        A_{(p)} (I \cap A_{(q)})
        \subseteq
        A_{(p)} I \cap A_{(p)} A_{(q)}
        \subseteq
        I \cap A_{(p+q)} \,,
      \]
      whence the product~$\fclass{x}_p \cdot \fclass{y}_q = [xy]_{p+q}$ is contained in~$\gr[p+q](I)$.
      Together with the above description of~$\gr[p](A)$ this shows that~$\gr(I)$ is a left ideal of~$\gr(A)$.

      We can show in the same way that~$\gr(I)$ is a right ideal of~$\gr(A)$ if~$I$ is a right ideal of~$A$.
      By combining these two cases it follows that~$\gr(I)$ is a two-sided ideal of~$\gr(A)$ if~$I$ is a two-sided ideal of~$A$.
    \item
      Let~$x$ be an element of~$I$.
      If~$x = 0$ then the assertion holds.

      In general the element~$\fclass{x}_{\deg(x)}$ is contained in~$\gr(I)$ and is therefore of the form
      \[
        \fclass{x}_{\deg(x)}
        =
        \sum_{s \in S} b_s \cdot [s]_{\deg(s)}
      \]
      for some coefficients~$b_s$ in~$\gr(A)$.
      We can now replace each coefficient~$b_s$ by its homogeneous component of degree~$\deg(x) - \deg(s)$ to assume that~$b_s$ is homogeneous.
      This means that~$b_s$ is contained in~$\gr[\deg(x)-\deg(s)](A)$ and thus of the form~$b_s = [a_s]_{\deg(x)-\deg(s)}$ for some element~$a_s$ of~$A_{(\deg(x)-\deg(s))}$.
      We have in particular~$a_s = 0$ whenever~$\deg(s) > \deg(x)$.
      It now follows that
      \begin{align*}
        \fclass{x}_{\deg(x)}
        &=
        \sum_{s \in S} b_s \cdot [s]_{\deg(s)}
        \\
        &=
        \sum_{s \in S} [a_s]_{\deg(x) - \deg(s)} \cdot [s]_{\deg(s)}
        \\
        &=
        \sum_{s \in S} [a_s s]_{\deg(x)}
        \\
        &=
        \left[ \sum_{s \in S} a_s s \right]_{\deg(x)} \,.
      \end{align*}
      This means that the difference~$x - \sum_{s \in S} a_s s$ is contained in~$A_{(\deg(x)-1)}$.
      We can now procced by induction to express this difference as a linear combination of~$S$ with coefficients in~$A$.
    \item
      Let~$I$ and~$J$ be two ideal of~$A$ such that~$I$ is contained in~$J$.
      Then~$\gr[p](I)$ is contained in~$\gr[p](J)$ for all~$p \geq 0$, and hence~$\gr(I)$ is contained in~$\gr(J)$.
      Suppose now that~$\gr(I)$ equals~$\gr(J)$.
      We need to show that~$I$ already equals~$J$.

      The ideal~$\gr(I)$ is homogeneous and thus generated by homogeneous elements.
      There hence exists some subset~$S$ of~$I$ such that~$\gr(I)$ is generated by the elements~$[s]_{\deg(s)}$ with~$s$ in~$S$.
      (Here we use that the homogeneous elements of~$\gr(I)$ are precisely those elements of the form~$\fclass{x}_{\deg(x)}$ with~$x$ in~$I$.)
      It follows from the equality~$\gr(I) = \gr(J)$ and part~\ref*{pulling back generating set from graded ideal} that the set~$S$ is a generating set of both the ideals~$I$ and~$J$.
      Thus~$I = J$.
    \qedhere
  \end{enumerate}
\end{proof}


\begin{corollary}
  \label{universal enveloping reflects chain conditions}
  Let~$A$ be a fieltered~\algebra{$\kf$}.
  If the associated graded algebra~$\gr(A)$ is left noetherian, right noetherian, left artinian or right artinian then the same property holds for~$A$.
\end{corollary}


\begin{proof}
  Every strictly increasing sequence of left ideals in~$A$ results by \cref{associated graded ideals} in a strictly increasing sequence of (homogeneous) left ideals in~$\gr(A)$.
  So if~$A$ is not left noetherian then neither is~$\gr(A)$.
  The other assertions can be shown in the same way.
\end{proof}


\begin{remark}
  \leavevmode
  \begin{enumerate}
    \item
      The idea of \cref{associated graded ideals} is taken from \cite[6.7,~6.9]{noncommutative_noetherian}.
    \item
      Let~$A$ be a filtered~\algebra{$\kf$}.
      If~$a$ is an element of~$A$ then one may think about the associated element~$[a]_{\deg(a)}$ of~$\gr(A)$ as the \enquote{leading term of~$a$}.
      We note that if~$A$ is already a graded~{\algebra{$\kf$}} then under the identification of~$\gr(A)$ with~$A$ the element~$[a]_{\deg(a)}$ is precisely the leading term of~$a$ in the usual sense.
      
      In view of this interpretation of~$[a]_{\deg(a)}$ one might compare the proof of part~\ref{pulling back generating set from graded ideal} of \cref{associated graded ideals} to the proof of Hilbert’s basis theorem.
      One may also compare the associated graded ideal~$\gr(I)$ to the \enquote{ideal of leading coefficients} as considered in the theory of Gröbner bases.
    \item
      Let~$A$ be a filtered~\algebra{$\kf$} and let~$I$ be an ideal of~$A$.
      Then~$I$ inherits from~$A$ a filtration given by~$I_{(p)} = I \cap A_{(p)}$ for all~$p \geq 0$.
      Then
      \begin{align*}
        \gr[p](I)
        &=
        \bigl(A_{(p-1)} + I \cap A_{(p)} \bigr) / A_{(p-1)}
        \\
        &\cong
        \bigl( I \cap A_{(p)} \bigr) / \bigl( A_{(p-1)} \cap I \cap A_{(p)} \bigr)
        \\
        &=
        \bigl( I \cap A_{(p)} \bigr) / \bigl( I \cap A_{(p-1)} \bigr)
        \\
        &=
        I_{(p)} / I_{(p-1)}
      \end{align*}
      by the second isomorphism theorem.
      This justifies the notation~$\gr[p](I)$.
  \end{enumerate}
\end{remark}

\subsubsection{Bases}

\begin{proposition}
  \label{checking basis via associated graded}
  Let~$A$ be filtered algebra and let~$s_i$ with~$i$ in~$I$ be elements of~$A$.
  Suppose that~$s_i$ is contained in~$A_{(d_i)}$ with~$d_i \geq 0$ for all~$i \in I$.
  If the elements~$\fclass{ s_i }_{d_i}$ form a vector space basis of~$\gr(A)$ then the elements~$s_i$ form a basis of~$A$.
\end{proposition}


\begin{proof}
  We first show that the elements~$s_i$ span the algebra~$A$ as a vector space.
  Let~$x$ be an element of~$A$.
  This element is contained in~$A_{(d)}$ for some~$d \geq -1$.
  We show that~$x$ is a linear combination of the elements~$s_i$ by induction over~$d$.
  The assertion holds for~$d = -1$ because then~$x = 0$.

  For the induction step we write the residue class~$\fclass{ x }_d$ as a linear combination
  \[
    \fclass{ x }_d
    =
    \sum_{i \in I}
    \lambda_i \fclass{ s_i }_{d_i} \,.
  \]
  It folllows by comparing degrees that
  \[
    \fclass{ x }_d
    =
    \sum_{i \in I, d_i = d}
    \lambda_i \fclass{ s_i }_d
    =
    \fclass*{
      \sum_{i \in I, d_i = d}
      \lambda_i s_i
    }_d \,.
  \]
  This shows that the difference~$x - \sum_{i \in I, d_i = d} \lambda_i s_i$ is contained in~$A_{(d-1)}$.
  It follows from the induction hypothesis that this difference can be written as the linear combination of the elements~$s_i$.
  It follows that the element~$x$ can be written as such a linear combination.

  We now show that the elements~$s_i$ are linearly independent.
  We suppose that there exists a linear combination
  \[
    0
    =
    \lambda_1 s_{i_1} + \dotsb + \lambda_n s_{i_n}
  \]
  whose coefficients~$\lambda_1, \dotsc, \lambda_n$ are nonzero, with~$n \geq 1$.
  We may assume that~$d_{i_1} \leq \dotsb \leq d_{i_n}$.
  Let~$d \defined d_{i_n}$ be the largest occuring degree and let~$j$ be the minimal index with~$d_{i_j} = d$.
  Then~$\fclass{ s_{i_k} }_d = 0$ for all~$k = 0, \dotsc, j-1$ and~$d_{i_j}, \dotsc, d_{i_n} = d$.
  It follows that the elements~$\fclass{ s_{i_j} }_d, \dotsc, \fclass{ s_{i_n} }_d$ are linearly independent by assumption, and that
  \[
    0
    =
    \lambda_j \fclass{ s_{i_j} }_d + \dotsb + \lambda_n \fclass{ s_{i_n} }_d \,.
  \]
  Thus~$\lambda_j = \dotsc = \lambda_n$.
  But this contradicts the assumption that these coefficients are nonzero.
\end{proof}





