\section{The Abstract Jordan Decomposition}


\begin{lemma}
  \label{ss linear lie algebra contains ss and nilpotent parts}
 Let~$\glie$ be a finite dimensional semisimple linear Lie~algebra.
 Then~$\glie$ contains the semisimple and nilpotent parts of all its elements.
\end{lemma}


\begin{proof}
  Let~$\glie$ be a Lie~subalgebra of~$\gllie(V)$ for a finite dimensional vector space~$V$.
  Let~$\hat{\glie}$ be the Lie~subalgebra of~$\gllie(V)$ that is generated by the semisimple and nilpotent parts of all~$x \in \glie$.
  We observe that~$\hat{\glie}$ has the following properties:
  \begin{itemize}
    \item
      The Lie~algebra~$\hat{\glie}$ contains~$\glie$:
      For every~$x \in \glie$ we have~$x_s, x_n \in \hat{\glie}$ and hence~$x = x_s + x_n \in \hat{\glie}$.
    \item
      The Lie~algebra~$\glie$ is in fact an ideal in~$\hat{\glie}$:
      We need to show that every~$y \in \hat{\glie}$ normalizes~$\glie$.
      It sufficies to consider the case that~$y$ is a Lie~algebra generator~$y = x_s$ or~$y = x_n$ for some~$x \in \glie$ because the normalizer~$\normallie_{\hat{\glie}}(\glie)$ is a Lie~subalgebra of~$\hat{\glie}$.
      We have~$\ad(x)(\glie) = [x,\glie] \subseteq \glie$ and hence~$\ad(x)_s(\glie) \subseteq \glie$ and~$\ad(x)_n(\glie) \subseteq \glie$ by \cref{concrete jordan decomposition}.
      We know from \cref{concrete jordan decomposition compatible with adjoint representation} that~$\ad(x)_s = \ad(x_s)$ and~$\ad(x)_n = \ad(x_n)$.
      Thus~$x_s$ and~$x_n$ normalize~$\glie$.
    \item
      Every~{\subrepresentation{$\glie$}}~$W$ of~$V$ (with respect to the natural action) is also a~{\subrepresentation{$\hat{\glie}$}}:
      It follows for every~$x \in \glie$ from~$x(W) \subseteq W$ that also~$x_s(W) \subseteq W$ and~$x_n(W) \subseteq W$.
      The stabilizer~$\glie_W = \{ z \in \gllie(V) \suchthat z(W) \subseteq W \}$ is a Lie~subalgebra of~$\gllie(V)$ that contains the generators of~$\hat{\glie}$ and thus all of~$\hat{\glie}$.
      This means that~$W$ is a~{\subrepresentation{$\hat{\glie}$}}.
    \item
      For every~$y \in \hat{\glie}$ and every~{\subrepresentation{$\glie$}}~$W$ of~$V$ the restriction~$\restrict{y}{W}$ (which is well-defined because~$W$ is a~{\subrepresentation{$\hat{\glie}$}}) has trace zero:
      It suffcies to show this holds for the Lie~algebra generators~$x_s$ and~$x_n$ with~$x \in \glie$ because~$\sllie(V) = \{z \in \gllie(V) \suchthat \tr(z) = 0 \}$ is a Lie~subalgebra of~$\gllie(V)$.
      The restriction~$\restrict{x_n}{W}$ has trace zero because it is again nilpotent.
      The restriction~$\restrict{x}{W}$ has trace zero by \cref{representation of semisimple lie algebra are traceless}, hence also~$\tr(\restrict{x_s}{W}) = \tr(\restrict{x}{W}) - \tr(\restrict{x_n}{W}) = 0$.
  \end{itemize}
  
  It follows from this properties that already~$\hat{\glie} = \glie$.
  Indeed, there exists by \cref{decomposition into orthogonals for semisimple} an ideal~$I$ of~$\glie$ with~$\hat{\glie} = \glie \oplus I$ because~$\glie$ is semisimple.
  Let~$x \in I$.
  For every irreducible subrepresentations~$W$ of~$V$ the restriction~$\restrict{x}{W} \colon W \to W$ is well-defined because~$W$ is a~{\subrepresentation{$\hat{\glie}$}}.
  We have~$[\glie, I] = 0$ because~$\hat{\glie} = \glie \oplus I$ is a decomposition into ideals, hence~$[x, \glie] = 0$.
  This means that~$\restrict{x}{W}$ is an endomorphism of~$W$ as a~{\representation{$\glie$}}.
  It follows from Schur’s~lemma that~$\restrict{x}{W}$ acts by multiplication by some scalar~$\lambda \in \kf$.
  But we have seen that~$\restrict{x}{W}$ has trace zero, so~$\lambda = 0$.
  This shows that~$\restrict{x}{W} = 0$ for every irreducible subrepresentation~$W$ of~$V$.
  It follows that~$x = 0$ because~$V$ is completely reducible.
  This shows that~$I = 0$ and hence~$\hat{\glie} = \glie$.
%   Let~$\nlie$ be the normalizer of~$\glie$ in~$\gllie(V)$.
%   This is a Lie~subalgebra of~$\gllie(V)$ that contains $\glie$.
%   We have for~$x \in \glie$ that~$\ad_{\gllie(V)}(x)(\glie) \subseteq \glie$ and hence~$(\ad_{\gllie(V)}(x))_s(\glie) \subseteq \glie$ and~$(\ad_{\gllie(V)}(x))_n(\glie) \subseteq \glie$.
%   We know from~\cref{ss and nilpotent implies ad-ss and ad-nilpotent} that~$(\ad_{\gllie(V)}(x))_s = \ad(x_s)$ and $\ad(x)_n = \ad(x_n)$ where~$x = x_s + x_n$ is the concrete Jordan~decomposition of~$x$.
%   Hence $x_s, x_n \in \nlie$.
%  
%   We observe that~$\glie$ is an ideal in~$\nlie$, but a proper one because~$\id_V \in \nlie$ while~$\id_V \notin \glie$ since $\centerlie(\glie) \neq 0$.
%   We will now shrink down~$\nlie$ to~$\glie$ by adding additional restraints, while maintaining the property that~$\nlie$ contains the simisimple and nilpotent part of all its element.
%  
%   For every~{\subrepresentation{$\glie$}}~$W$ of~$V$ the stabilizer
%   \[
%     \glie_W
%     \defined
%     \{
%       x \in \gllie(V)
%     \suchthat
%       x(W) \subseteq W,
%       \tr \restrict{x}{W} = 0
%     \}  \,.
%   \]
%   Then~$\glie_W$ is a Lie~subalgebra of $\gllie(V)$.
%   We know from \cref{representation of semisimple lie algebra are traceless} that every~$x \in \glie$ is traceless, hence that~$\glie$ is contained in~$\glie_W$.
%   For every~$x \in \glie_W$ the nilpotent part~$x_n$ again satifies~$x_n(W) \subseteq W$ by \cref{concrete jordan decomposition} and it has trace zero because it is nilpotent.
%   This shows that the Lie~algebra~$\glie_W$ contains the nilpotent part~$x_n$, and hence also the semisimple part~$x_s = x - x_n$.
%   
%   Let~$\hat{\glie}$ be the intersection of the normalizer~$\nlie$ with all the stabilizers~$\glie_W$ where~$W$ ranges through the~{\subrepresentations{$\glie$}} of~$V$.
%   Then~$\hat{\glie}$ is a Lie~subalgebra of~$\gllie(V)$ that contains~$\glie$ as an ideal and that contains the nilpotent and semisimple parts of all its elements.
%   We will now show that~$\glie = \hat{\glie}$.
% 
%   By \cref{decomposition into orthogonals for semisimple} there exists an ideal~$I$ in~$\hat{\glie}$ with~$\hat{\glie} = \glie \oplus I$.
%   We show that~$x = 0$ for every~$x \in I$, which means that~$x(v) = 0$ for every~$v \in V$.
%   We know from Weyl’s~theorem that~$V$ decomposes into irreducible~{\subrepresentations{$\glie$}}.
%   It sufficies to show that~$x(v) = 0$ for every irreducible~{\subrepresentation{$\glie$}}~$W$ of~$V$ and every~$x \in W$.
%   
%   The natural action of~$\hat{\glie}$ on~$V$ restricts by construction of~$W$ to an irreducible representation
%   
% %   Let $x \in I$ and~$W$ be an irreducible~{\subrepresentation{$\glie$}} of~$V$.
% %   The natural aciton of~$\glie$ on~$W$ extends to the natural action of~$\hat{\glie}$ on~$W$ and~$W$ is also irreducible as a~{\representation{$\hat{\glie}$}}
% %   Because $[\g,I] = 0$ it follows that $x \in I$ is a homomorphisms of representations of $\glie$. By Schur’s Lemma it follows that $x = \lambda \id_W$ for some $\lambda \in k$. Because $\lambda \dim W = \tr x|_W = 0$ this scalar has to be zero. Hence $x = 0$ and therefore $I = 0$. Thus $\hat{\glie} = \glie$.
\end{proof}


\begin{remark}
  Our proof of \cref{ss linear lie algebra contains ss and nilpotent parts} is a variantion of the standard proof as found in~\cite[Theorem~6.4]{humphreys} and~\cite[Proposition~I.\S6.3]{bourbaki_lie}.
\end{remark}


\begin{definition}
  Let~$\glie$ be a semisimple Lie~algebra.
  An element~$x \in \glie$ is \defemph{\textup(abstract\textup) semisimple}\index{semisimple!element} (resp.\ \defemph{\textup(abstract\textup) nilpotent}\index{nilpotent!element}) if~$x$ acts semisimple (resp.\ nilpotent) on every finite dimensional representation of~$\glie$, i.e.\ if for every finite dimensional representation~$(V,\rho)$ of~$\glie$ the endomorphism~$\rho(x)$ is semisimple (resp.\ nilpotent).
\end{definition}


\begin{remark}
  If~$f \colon \glie \to \hlie$ is an isomorphism of finite dimensional semisimple Lie~algebras then an element~$x \in \glie$ is semisimple (resp.\ nilpotent) if and only if the corresponding element~$\varphi(x) \in \hlie$ is semisimple (resp.\ nilpotent).
\end{remark}


\begin{lemma}
  \label{characterizations of semisimple elements}
  Let~$\glie$ be a linear semisimple Lie~algebra.
  For any element~$x \in \glie$ the following conditions are equivalent:
  \begin{equivalenceslist}
    \item
      \label{concrete semisimple}
      $x$ is concrete semisimple, i.e.\ semisimple as an endomorphism of~$V$.
    \item
      \label{adsemisimple}
      $x$ is~{\adsemisimple}, i.e.~$\ad_{\glie}(x)$ is semisimple as an endomorphism of~$\glie$.
    \item
      \label{abstract semisimple}
      $x$ is abstract semisimple, i.e.~$\rho(x)$ is a semisimple endomorphism of~$V$ for every finite dimensional representation~$(V,\rho)$ of~$\glie$.
    \item
      \label{rep adsemisimple}
      $\rho(x)$ is~{\adsemisimple} in~$\rho(\glie)$ for every finite dimensional representation~$(V,\rho)$ of~$\glie$.
  \end{equivalenceslist}
  The analogous statement for nilpotency also holds.
\end{lemma}


\begin{proof}
  We have seen the implication~\ref*{concrete semisimple}~$\implies$~\ref*{adsemisimple} in \cref{ss and nilpotent implies ad-ss and ad-nilpotent} and deduced in \cref{concrete jordan decomposition compatible with adjoint representation} that for every~$x \in \glie$ with concrete Jordan~decomposition~$x = x_s + x_n$ the concrete Jordan decomposition of the endomorphism~$\ad(x)$ is given by~$\ad(x) = \ad(x_s) + \ad(x_n)$.
  The adjoint representation of~$\glie$ is faithful because~$\centerlie(\glie) = 0$, whence
  \begin{align*}
    \text{$x$ is concrete semisimple}
    &\iff
    x = x_s
    \\
    &\iff
    \ad(x) = \ad(x_s)
    \\
    &\iff
    \text{$\ad(x)$ is concrete semisimple}
    \\
    &\iff
    \text{$x$ is~{\adsemisimple}} \,.
  \end{align*}
  This shows the equivalence~\ref*{concrete semisimple}~$\iff$~\ref*{adsemisimple}.
  The equivalence~\ref*{abstract semisimple}~$\iff$~\ref*{abstract semisimple} follows because~$\rho(\glie)$ is again a linear semisimple Lie~algebra.
  The implication~\ref*{adsemisimple}~$\implies$~\ref*{rep adsemisimple} holds because~$\ad(\rho(x)) \colon \rho(\glie) \to \rho(\glie)$ is the endomorphism induced by~$\ad(x) \colon \glie \to \glie$.
  The implication~\ref*{abstract semisimple}~$\implies$~\ref*{concrete semisimple} follows by considering the natural representation.
  (We could equivalently consider the adjoint representation for the implication~\ref*{abstract semisimple}~$\implies$~\ref*{adsemisimple}.)
  
  For the nilpotent case we argue in the same way.
\end{proof}


\begin{theorem}[Abstract Jordan~decomposition]
  \label{abstract jordan decomposition}
  Let~$\glie$ be a finite dimensional semisimple Lie~algebra.
  \begin{enumerate}
    \item
      \label{existence of unique jd}
      Every element~$x \in \glie$ admits a unique decomposition~$x = x_s + x_n$ such that~$x_s$ is semisimple,~$x_n$ is nilpotent and the summands~$x_s$ and~$x_n$ commute.
    \item
      \label{abstract and concrete jd coincide}
      If~$\glie$ is linear then this decomposition coincides with the concrete Jordan decomposition.
    \item
      \label{functoriality of jordan decomposition}
      The Jordan decomposition is functorial:
      If~$f \colon \glie \to \hlie$ is a homomorphism of Lie~algebras where~$\hlie$ is another finite dimensional semisimple Lie~algebra then~$f(x_s) = f(x)_s$ and~$f(x)_n = f(x_n)$ for every~$x \in \glie$.
  \end{enumerate}
\end{theorem}


\begin{definition}
  The decomposition from \cref{abstract jordan decomposition} is the \defemph{\textup(abstract\textup) Jordan~decomposition}\index{Jordan decomposition!abstract}.
\end{definition}


\begin{remark}
  If~$\glie$ is a finite dimensional semisimple Lie~algebra then the abstract Jordan decomposition~$x = x_s + x_n$ of~$x \in \glie$ is built precisely so that for every finite dimensional representation~$(V,\rho)$ of~$\glie$ the induced decomposition~$\rho(x) = \rho(x_s) + \rho(x_n)$ is precisely the concrete Jordan~decomposition of~$\rho(x)$.
  If~$(V,\rho)$ is faithful then the abstract Jordan decomposition in~$\glie$ can conversely be calculated from the concrete Jordan decomposition in~$\glie(V)$.
  This applies in particular to the adjoint representation.
\end{remark}

\begin{proof}[Proof of \cref{abstract jordan decomposition}]
  If~$\glie$ is linear then the concrete Jordan decomposition restricts to~$\glie$ by \cref{ss linear lie algebra contains ss and nilpotent parts}.
  Then parts~\ref*{existence of unique jd} and~\ref*{abstract and concrete jd coincide} follow with \cref{characterizations of semisimple elements} from the concrete Jordan decomposition.
  In the general case the semisimple Lie~algebra~$\glie$ is isomorphic to a linear Lie~algebra via the adjoint representation and hence part~\ref*{existence of unique jd} follows from the linear case.
  
  For part~\ref*{functoriality of jordan decomposition} let~$x \in \glie$.
  If~$x$ is semisimple then every finite dimensional representation~$(V, \rho)$ of~$\hlie$ gives a finite dimensional representation~$(V, \rho \circ f)$ of~$\glie$ for which the endomorphism~$\rho(\spacing f(x))$ is then semisimple.
  This shows that~$f(x)$ is again semisimple and we find in the same way that~$f(x)$ is nilpotent if~$x \in \glie$ is nilpotent.
  If~$x \in \glie$ has the abstract Jordan decomposition~$x = x_s + x_n$ then it follows that~$f(x) = f(x_s) + f(x_n)$ is the abstract Jordan decomposition of~$f(x)$.
\end{proof}


\begin{remark}
  We have taken our approach from~\cite[I.\S6.3]{bourbaki_lie}.
  In~\cite{humphreys} the abstract Jordan decomposition is first constructed by considering the adjoint representation:
  It is shown in~\cite[\S 4.2, \S 5.3]{humphreys} that~$\Der(\glie)$ contains the semisimple and nilpotent parts of all its elements (for any Lie~algebra~$\glie$) and it is then shown that every derivation of~$\glie$ is inner (if~$\glie$ is semisimple).
  Then~$\ad \colon \glie \to \Der(\glie)$ is an isomorphism and one can pull back the concrete Jordan decomposition in~$\Der(\glie)$ to the abstract Jordan decomposition~$\glie$.
  For this approach one has to \emph{define} an element~$x \in \glie$ as semisimple (resp.\ nilpotent) if it is~{\adsemisimple} (resp.\ {\adnilpotent}).
  In~\cite[\S 6]{humphreys} it is then shown with \cref{ss linear lie algebra contains ss and nilpotent parts} that in the linear case the concrete and abstract Jordan decompositions coincide, and that the Jordan decomposition is functorial.
  From this one then gets the equivalence with our definition of semisimple (resp.\ nilpotent) elements.
\end{remark}


\begin{example}
  The element~$h \in \sllie_2(\kf)$ is semisimple and hence acts semisimply on every finite dimensional~{\representation{$\sllie_2(\kf)$}}.
  This matches what we have seen in the classification of finite dimensional~{\representations{$\sllie_2(\kf)$}}.
  We will see in the next section how this generalizes to finite semisimple Lie~algebras.
\end{example}


% 
% \begin{proof}[Proof of \cref{abstract jordan decomposition}]
%   \leavevmode
%   \begin{enumerate}[start=3]
%     \item
%       We consider first the case that~$\glie$ is a linear Lie~algebra, say a Lie~subalgebra of~$\gllie(V)$ for some finite dimensional vector space~$V$.
%       If~$x \in \glie$ admits a decomposition~$x = x_s + x_n$ with~$x_s$ abstract semisimiple,~$x_n$ abstract nilpotent and~$[x_s, x_n] = 0$.
%       Then for the natural representation~$\rho \colon \glie \inclusion \gllie(V)$ we have~$\rho(x) = \rho(x_s) + \rho(x_n)$ with~$\rho(x_s)$ concret semisimple,~$\rho(x_n)$ concret nilpotent and~$[\rho(x_s), \rho(x_n)] = \rho([x_s, x_n]) = \rho(0) = 0$.
%       This means that~$\rho(x) = \rho(x_s) + \rho(x_n)$ is the concrete Jordan decomposition of~$\rho(x)$.
%       But~$\rho$ is the inclusion of~$\glie$ into~$\gllie(V)$, so~$\rho(x') = x'$ for every~$x' \in \glie$.
%       This shows that if such a decomposition exists, then it coincides with the concrete Jordan decomposition.
%       This shows part~\ref*{abstract and concrete jd coincide} and the uniqueness in part~\ref*{existence of unique jd} for linear Lie~algebras.
%       
%       Let~$x \in \glie$ with concrete Jordan~decomposition~$x = x_s + x_n$.
%       By \cref{ss linear lie algebra contains ss and nilpotent parts} both~$x_s$ and~$x_n$ are again contained in~$\glie$, and in light of part~\ref*{abstract and concrete jd coincide} we need to show that~$x_s$ is abstract semisimple and that~$x_n$ is abstract nilpotent.
%       So let~$(W,\rho)$ be any finite dimensional representation of~$\glie$.
%       By \cref{decomposition into orthogonals for semisimple}
%       
%       
%     \item
%       Let~$f \colon \glie \to \hlie$ be a homomorphism of Lie~algebras between semisimple Lie~algebras~$\glie$ and~$\hlie$.
%       Let~$x \in \glie$ be semisimple.
%       Every finite dimensional representation~$(V, \rho)$ of~$\hlie$ gives a finite dimensional~{\representation{$\glie$}}~$(V, \rho \circ f)$ for which the endomorphism~$\rho(f(x))$ is then semisimple.
%       This shows that~$f(x)$ is again semisimple, and we find in the same way that~$f(x)$ is nilpotent if~$x \in \glie$ is nilpotent.
%       If~$x \in \glie$ admits a decomposition~$x = x_s + x_n$ as in~\ref*{existence of unique jd} then~$f(x) = f(x_s) + f(x_n)$ with~$f(x_s)$ semisimple,~$f(x_n)$ nilpotent and~$[\spacing f(x_s), f(x_n)] = f([x_s, x_n]) = f(0) = 0$.
%       Hence~$f(x_s) = f(x)_s$ and~$f(x_n) = f(x)_n$ by the uniqueness in of the decomposition in part~\ref*{existence of unique jd}.
%       This shows part~\ref*{functoriality of jordan decomposition}.
%   \end{enumerate}
% 
% %   Let~$f \colon \glie \to \hlie$ be a homomorphism of Lie~algebras between semisimple Lie~algebras~$\glie$ and~$\hlie$.
% %   Let~$x \in \glie$ be semisimple.
% %   Every finite dimensional representation~$(V, \rho)$ of~$\hlie$ gives a finite dimensional~{\representation{$\glie$}}~$(V, \rho \circ f)$ for which the endomorphism~$\rho(f(x))$ is then semisimple.
% %   This shows that~$f(x)$ is again semisimple, and we find in the same way that~$f(x)$ is nilpotent if~$x \in \glie$ is nilpotent.
% %   If~$x \in \glie$ admits the Jordan~decomposition~$x = x_s + x_n$ then~$f(x) = f(x_s) + f(x_n)$ with~$f(x_s)$ semisimple,~$f(x_n)$ nilpotent and~$[\spacing f(x_s), f(x_n)] = f([x_s, x_n]) = f(0) = 0$.
% %   This shows part~\ref*{functoriality of jordan decomposition}.
% %   
% %   For part~\ref*{jordan decomposition on reps} we observe that the image~$\rho(\glie)$ is again semisimple and hence by part~\ref*{abstract and concrete jd coincide} the concrete and abstract Jordan decomposition in~$\rho(\glie)$ coincide.
% %   The assertion now follows from part~\ref*{functoriality of jordan decomposition}.
% %   
% %   To show parts~\ref*{existence of unique jd} and~\ref*{jordan decomposition on reps} we observe that by parts~\ref*{jordan decomposition on reps} and~\ref*{abstract and concrete jd coincide} the Jordan decomposition of any element~$x \in \glie$ can be calculated via any faithful representation~$V$ and the concrete Jordan decomposition in~$\gllie(V)$.
% %   
% %   So let~$V$ be a faithful representation of~$\glie$ (e.g.\ the adjoint representation, which is faithful because~$\ker \ad_{\glie} = \centerlie(\glie) = 0$).
% \end{proof}


% \begin{lemma}\label{lem: generalized eigenspace decomposition for derivations}
%  Let $A$ be a finite dimensional $k$-algebra, not necessarily associative nor unital, and $\delta \in \Der A$. For every $\lambda \in k$ let
%  \[
%   A_\lambda
%   \defined \bigcup_{n \in \N} \ker(\delta-\lambda I)^n
%   = \{x \in A \mid \text{$(\delta-\lambda I)^m(x) = 0$ for some $m \in \N$}\}
%  \]
%  be the generalized eigenspace of $\delta$ with respect to $\lambda$. Then $A = \bigoplus_{\lambda \in k} A_\lambda$ and
%  \[
%   A_\lambda A_\mu \subseteq A_{\lambda + \mu} \quad \text{for all $\lambda, \mu \in k$}.
%  \]
% \end{lemma}
% \begin{proof}
%  For this proof abbreviate $1 \defined I$. That $A = \bigoplus_{\lambda \in k} A_\lambda$ is a standard fact from linear algebra. Let $x \in A_\lambda$ and $y \in A_\mu$. It needs to be shown that $xy \in A_{\lambda + \mu}$, i.e.\ $(\delta-(\lambda+\mu)I)^m(xy) = 0$ for some $m \in \N$. Because $(\delta-\lambda I)^{m_1}(x) = 0$ and $(\delta-\mu I)^{m_2}(y) = 0$ for some $m_1, m_2 \in \N$ this follows from the formula
%  \[
%   (\delta-(\lambda+\mu)I)^n(xy)
%   = \sum_{i=0}^n \binom{n}{i} (\delta-\lambda I)^i(x) (\delta-\mu I)^{n-i}(y)
%   \quad\text{for every $n \in \N$},
%  \]
%  which can be shown by induction over $n \in \N$: For $n = 0$ the statement holds. Suppose it holds for some $n \in \N$. Then
%  \begingroup
%  \allowdisplaybreaks
%  \begin{align*}
%   &\, (\delta-(\lambda+\mu)I)^{n+1}(xy)
%   = (\delta-(\lambda+\mu)I)((\delta-(\lambda+\mu)I)^n(xy)) \\
%   =&\, (\delta-(\lambda+\mu)I)\left(
%        \sum_{i=0}^n \binom{n}{i} (\delta-\lambda I)^i(x) (\delta-\mu I)^{n-i}(y)
%        \right) \\
%   =&\,  \sum_{i=0}^n \binom{n}{i} \delta\left( (\delta-\lambda I)^i(x) \right) (\delta-\mu I)^{n-i}(y)
%        +\sum_{i=0}^n \binom{n}{i} (\delta-\lambda I)^i(x) \delta\left( (\delta-\mu I)^{n-i}(y) \right) \\
%    &\, -\sum_{i=0}^n \binom{n}{i} \lambda(\delta-\lambda I)^i(x) (\delta-\mu I)^{n-i}(y)
%        -\sum_{i=0}^n \binom{n}{i} (\delta-\lambda I)^i(x) \mu(\delta-\mu I)^{n-i}(y) \\
%   =&\,  \sum_{i=0}^n \binom{n}{i} (\delta-\lambda I)^{i+1}(x) (\delta-\mu I)^{n-i}(y)
%        +\sum_{i=0}^n \binom{n}{i} (\delta-\lambda I)^i(x) (\delta-\mu I)^{n+1-i}(y) \\
%   =&\,  \sum_{i=1}^{n+1} \binom{n}{i-1} (\delta-\lambda I)^i(x) (\delta-\mu I)^{n+1-i}(y)
%        +\sum_{i=0}^n \binom{n}{i} (\delta-\lambda I)^i(x) (\delta-\mu I)^{n+1-i}(y) \\
%   =&\,  \sum_{i=1}^n \left(\binom{n}{i-1}+\binom{n}{i}\right) (\delta-\lambda I)^i(x) (\delta-\mu I)^{n+1-i}(y) \\
%    &\, +(\delta-\mu I)^{n+1}(y) + (\delta-\lambda I)^{n+1}(x) \\
%   =&\,  \sum_{i=1}^n \binom{n+1}{i} (\delta-\lambda I)^i(x) (\delta-\mu I)^{n+1-i}(y)
%        +(\delta-\mu I)^{n+1}(y) + (\delta-\lambda I)^{n+1}(x) \\
%   =&\, \sum_{i=0}^{n+1} \binom{n+1}{i} (\delta-\lambda I)^i(x) (\delta-\mu I)^{n+1-i}(y)
%  \qedhere
%  \end{align*}
%  \endgroup
% \end{proof}
% 
% \begin{remark}
%  Notice that the formula in the proof of Lemma~\ref{lem: generalized eigenspace decomposition for derivations} is just a generalization of the binomial theorem, which follows by setting $\delta = 0$.
% \end{remark}
% 
% 
% \begin{lemma}\label{lem: derivations contain the ss and np part of all its elements}
%  Let $A$ be a finite dimensional $k$-algebra, not necessarily associative nor unital, e.g.\ a Lie algebra. Then $\Der(A)$ contains the semisimple and nilpotent parts (in $\End_k(A)$) of all its elements.
% \end{lemma}
% \begin{proof}
%  Let $\delta \in \Der(A)$ and for every $\lambda \in k$ let $A_\lambda$ be the generalized eigenspace with respect to $\lambda$. To show that $\Der(A)$ contains the semisimple and nilpotent part of $\delta$ is sufficies to do so for the semisimple part, which will be denoted by $\sigma$. As seen in the proof of Theorem~\ref{thrm: concrete Jordan decomposition} (the concrete Jordan decomposition) $\sigma$ acts on $A_\lambda$ by multiplication with $\lambda$ for every $\lambda \in k$. If $x \in A_\lambda$ and $y \in A_\mu$ then $xy \in A_{\lambda + \mu}$ and thus
%  \[
%   \sigma(xy)
%   = (\lambda + \mu)(xy)
%   = (\lambda x)y + x(\mu y)
%   = \sigma(x)y + x\sigma(y).
%  \]
%  Because $A = \bigoplus_{\lambda \in k} A_\lambda$ it follows that $\sigma$ is a derivation of $A$.
% \end{proof}
% 
% 
% \begin{lemma}\label{lem: direct complement of semisimple ideals}
%  Let $\glie$ be a finite dimensional Lie algebra and $I \subideal \glie$ a semisimple ideal. Then there exists an ideal $J \subideal \glie$ with $\g = I \oplus J$.
% \end{lemma}
% \begin{proof}
%  Because $I \subideal \glie$ is an ideal it follows that $\kappa_I = \kappa_\g|_{I \times I}$. Because $I$ is semisimple it also follows that $\kappa_I$ is non-degenerate. It follows for the linear map
%  \[
%   \varphi \colon \g \to I^*, \quad x \mapsto \kappa(x, \cdot)
%  \]
%  that $\varphi|_I \colon I \to I^*$ is an isomorphism. Hence $I \cap \ker \varphi = 0$ with
%  \[
%   \dim I = \dim I^* = \dim \g - \dim \ker \varphi.
%  \]
%  Let $J \defined I^\perp$ be the orthogonal complement of $I$ with respect to the Killing form. From Lemma~\ref{lem: orthogonal complement of an ideal is again an ideal} it is known that $J$ is an ideal in $\glie$. Because $J = \ker \varphi$ it follows from the previous observations that $I \cap J = 0$ and $\dim \g = \dim I + \dim J$ and thus $\g = I \oplus J$.
% \end{proof}
% 
% 
% \begin{lemma}\label{lem: every derivation of a semisimple Lie algebra is inner}
%  Let $\glie$ be a finite dimensional semisimple Lie algebra. Then every derivation of $\glie$ is inner.
% \end{lemma}
% \begin{proof}
%  As seen in Lemma~\ref{lem: inner derivations are in ideal} and its proof the inner derivations form an ideal $I \subideal \Der(\glie)$ with $[\delta,\ad(x)] = \ad(\delta(x))$ for every $\delta \in \Der(\glie)$ and $x \in \glie$. By Lemma~\ref{lem: direct complement of semisimple ideals} there exists an ideal $J \subideal \Der(\glie)$ with $\Der(\glie) = I \oplus J$. Let $\delta \in J$. Then for any $x \in \glie$
%  \[
%   \ad(\delta(x)) = [\delta, \ad(x)] \in J \cap I = 0,
%  \]
%  and thus $\ad(\delta(x)) = 0$. Because $\ad$ is injective it follows that $\delta(x) = 0$ for every $x \in \glie$ and thus $\delta = 0$. Therefore $J = 0$ and thus $\Der(A) = I$.
% \end{proof}
% 
% 
% \begin{theorem}[Abstract Jordan decomposition] \label{thrm: abstract Jordan decomposition}
%  Let $\glie$ be a finite dimensional semisimple Lie algebra. Then for any $x \in \glie$ there exist unique elements $x_s, x_n \in \glie$ such that $\ad(x) = \ad(x_s) + \ad(x_n)$ is the concrete Jordan decomposition of $\ad(x)$ with $\ad(x_s) = \ad(x)_s$ and $\ad(x_n) = \ad(x)_n$.
%  
%  Then $x = x_s + x_n$ and an arbitrary element of $\glie$ commutes with $x$ if and only if it commutes both with $x_s$ and $x_n$. In particular $x$, $x_s$ and $x_n$ are pairwise commuting.
% \end{theorem}
% \begin{proof}
%  By Lemma~\ref{lem: every derivation of a semisimple Lie algebra is inner} the adjoint representation $\ad \colon \g \to \ad(\glie) = \Der(\glie)$ is an isomorphism of Lie algebras. By Lemma~\ref{lem: derivations contain the ss and np part of all its elements} $\Der(\glie)$ contains the semisimple and nilpotent parts of all its elements. By setting $x_s \coloneq \ad^{-1}(\ad(x)_s)$ and $x_n \defined \ad^{-1}(\ad(x)_n)$ the theorem follows from the properties of the concrete Jordan decomposition as stated in Theorem~\ref{thrm: concrete Jordan decomposition}.
% \end{proof}
% 
% 
% \begin{definition}\label{defi: definition of abstract ss and np}
%  Let $\glie$ be a finite dimensional semisimple Lie algebra and $x \in \glie$. Then the decomposition $x = x_s + x_n$ as is Theorem~\ref{thrm: abstract Jordan decomposition} is the \emph{abstract Jordan decomposition} of $x$ where $x_s$ is the \emph{abstract semisimple part} and $x_n$ is the \emph{abstract nilpotent part} of $x$.
% \end{definition}
% 
% 
% \begin{remark}
%  By Definition~\ref{defi: definition of abstract ss and np} an element of a finite dimensional semisimple Lie algebra is semisimple if and only if it is $\ad$-semisimple and nilpotent if and only if it is $\ad$-nilpotent.
% \end{remark}
% 
% 
% \begin{proposition}\label{prop: abstract and concrete Jordan decomposition coincide}
%  Let $\glie$ be a finite dimensional semisimple linear Lie algebra. Then the abstract and concrete Jordan decomposition coincide.
% \end{proposition}
% \begin{proof}
%  Let $x \in \glie$, $x = xc_s + x_n$ the abstract Jordan decomposition of $x$ and $x = y_s + y_n$ the concrete Jordan decomposition of $x$. From Lemma~\ref{lem: semisimple linear Lie algebra containing the semisimple and nilpotent parts of its elements} it follows that $y_s, y_n \in \glie$ and by Lemma~\ref{lem: ss and nilpotent implies ad-ss and ad-nilpotent} $\ad(x) = \ad(y_s) + \ad(y_n)$ is the concrete Jordan decomposition of $\ad(x)$ with $\ad(y_s) = \ad(x)_s$ and $\ad(y_n) = \ad(x)_n$. Hence $y_s$ is the abstract semisimple part of $x$ and $y_n$ is the abstract nilpotent part of $x$.
% \end{proof}
% 
% 
% \begin{remark}
%  By Proposition~\ref{prop: abstract and concrete Jordan decomposition coincide} we need not to distinguish between the concrete and abstract Jordan decomposition when working with a finite dimensional semisimple linear Lie algebra, which is why we will stop doing so.
% \end{remark}
% 
% 
% % TODO: Examples all around the Jordan decomposition
% 
% 
% \begin{lemma}\label{lem: functoriality of the Jordan decomposition for linear Lie algebras}
%  Let $\glie$ be a finite dimensional semisimple Lie algbra, $\rho \colon \g \to \gllie(V)$ a finite dimensional representation of $\glie$ and $x \in \glie$. If $x = x_s + x_n$ is the Jordan decomposition of $x$ then $\rho(x) = \rho(x_s) + \rho(x_n)$ is the Jordan decomposition of $\rho(x)$ with $\rho(x_s) = \rho(x)_s$ and $\rho(x_n) = \rho(x)_n$.
% \end{lemma}
% \begin{proof}
%  Let $x \in \glie$ be semisimple (resp.\ nilpotent). Then $x$ is $\ad_\glie$-semisimple (resp.\ $\ad_\glie$-nilpotent). It follows that $\rho(x)$ is $\ad_{\rho(\glie)}$-semisimple (resp.\ $\ad_{\rho(\glie)}$-nilpotent) and therefore semisimple (resp.\ nilpotent).
% \end{proof}
% 
% 
% \begin{corollary}[Functoriality of the Jordan decomposition] \label{cor: functoriality of the Jordan decomposition}
%  Let $\g_1$ and $\g_2$ be finite dimensional semisimple Lie algebras and $\phi \colon \g_1 \to \g_2$ a homomorphism of Lie algebras. Then $\phi$ preserves the Jordan decomposition, i.e.\ $\phi(x_s) = \phi(x)_s$ and $\phi(x_n) = \phi(x)_n$ for every $x \in \g_1$.
% \end{corollary}
% \begin{proof}
%  By Lemma~\ref{lem: functoriality of the Jordan decomposition for linear Lie algebras} the homomorphism of Lie algebras $\ad_{\g_2} \!\circ\, \phi \colon \g_1 \to \gl(\g_2)$ preserves the Jordan decomposition. The homomorphism $\ad_{\g_2}^{-1} \colon \ad(\g_2) \to \g_2$ preserves the Jordan decomposition by the definition of the abstract Jordan decomposition. Hence $\phi = \ad_{\g_2}^{-1} \circ \ad_{\g_2}\! \circ\, \phi$ preserves the Jordan decomposition as well.
% \end{proof}
% 
% 
% 




