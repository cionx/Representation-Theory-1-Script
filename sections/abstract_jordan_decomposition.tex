\section{The abstract Jordan decomposition}
In this section the concrete Jordan decomposition (Theorem~\ref{thrm: concrete Jordan decomposition}) is generalized to finite dimensional semisimple Lie algebras. The approach taken is mostly from \cite[\S 4.2, \S 5.3, \S 5.4, \S 6.4]{Humphreys} with some inspiration from the lecture.


\begin{lem}\label{lem: semisimple linear Lie algebra containing the semisimple and nilpotent parts of its elements}
 Let $\g \subseteq \gl(V)$ be a finite dimensional semisimple linear Lie algebra. Then $\g$ contains the semisimple and nilpotent parts of all its elements.
\end{lem}
\begin{proof}
 It sufficies to show the statement for $V \neq 0$ because otherwise $\g = \gl(V) = 0$.
 
 Let $N \coloneqq N_{\gl(V)}(\g)$. This is a Lie subalgebra of $\gl(V)$ containing $\g$. Let $x \in \g$. Because $\ad(x)(\g) \subseteq \g$ it follows that $\ad(x_s)(\g) \subseteq \g$ and $\ad(x_n)(\g) \subseteq \g$ with the abbreviation $\ad = \ad_{\gl(V)}$. From Lemma~\ref{lem: ss and nilpotent implies ad-ss and ad-nilpotent} it follows that $\ad(x)_s = \ad(x_s)$ and $\ad(x)_n = \ad(x_n)$ where $x = x_s + x_n$ is the concrete Jordan decomposition of $x$. Hence $x_s, x_n \in N$.
 
 Notice that $\g \subsetneq N$ because $\id_V \in N$ but $\id_V \notin \g$ since $Z(\g) \neq 0$. The idea is to now shrink $N$ down to $\g$ by adding additional restraints while maintaining the property that the semisimle and nilpotent parts of all element of $\g$ are contained.
 
 For any $\g$-subrepresentation $W \subseteq V$ let
 \[
  \g_W \coloneqq \{x \in \gl(V) \mid x(W) \subseteq W, \tr x|_W = 0\};
 \]
 notice that this is a Lie subalgebra of $\gl(V)$. Also notice that together with Corollary~\ref{cor: representation of semisimple Lie algebra are traceless} it follows that $\g \subseteq \g_W$ for every $\g$-subrepresentation $W \subseteq V$. Let
 \[
  \hat{\g} \coloneqq N \cap \bigcap \{\g_W \mid \text{$W \subseteq V$ is a $\g$-subrepresentation}\}.
 \]
 Notice that for any $\g$-subrepresentation $W \subseteq V$ and $x \in \g$ it follows from $x(W) \subseteq W$ that als $x_s(W) \subseteq W$ and $x_n(W) \subseteq W$. Because $x_n|_W$ is nilpotent it also follows that $\tr x_n|_W = 0$ and therefore also $\tr x_s|_W = \tr x|_W - \tr x_n|_W = 0$. Hence $x_s, x_n \in \g_W$.
 
 Combining the above observations it follows that $\hat{\g}$ is a Lie subalgebra of $\gl(V)$ which contains $\g$ as an ideal and for which $x_s, x_n \in \hat{\g}$ for every $x \in \g$. By Lemma~\ref{lem: direct complement of semisimple ideals} there exists an ideal $I \subseteq \hat{\g}$ with $\hat{\g} = \g \oplus I$. Let $x \in I$ and $W \subseteq V$ be an irreducible subrepresentation. By construction of $\hat{\g}$ the action of $\g$ on $W$ extends to the natural action of $\hat{\g}$ on $W$. Because $[\g,I] = 0$ it follows that $x \in I$ is a homomorphisms of representations of $\g$. By Schur’s Lemma it follows that $x = \lambda \id_W$ for some $\lambda \in k$. Because $\lambda \dim W = \tr x|_W = 0$ this scalar has to be zero. Hence $x = 0$ and therefore $I = 0$. Thus $\hat{\g} = \g$.
\end{proof}


\begin{lem}\label{lem: generalized eigenspace decomposition for derivations}
 Let $A$ be a finite dimensional $k$-algebra, not necessarily associative nor unital, and $\delta \in \Der A$. For every $\lambda \in k$ let
 \[
  A_\lambda
  \coloneqq \bigcup_{n \in \N} \ker(\delta-\lambda I)^n
  = \{x \in A \mid \text{$(\delta-\lambda I)^m(x) = 0$ for some $m \in \N$}\}
 \]
 be the generalized eigenspace of $\delta$ with respect to $\lambda$. Then $A = \bigoplus_{\lambda \in k} A_\lambda$ and
 \[
  A_\lambda A_\mu \subseteq A_{\lambda + \mu} \quad \text{for all $\lambda, \mu \in k$}.
 \]
\end{lem}
\begin{proof}
 For this proof abbreviate $1 \coloneqq I$. That $A = \bigoplus_{\lambda \in k} A_\lambda$ is a standard fact from linear algebra. Let $x \in A_\lambda$ and $y \in A_\mu$. It needs to be shown that $xy \in A_{\lambda + \mu}$, i.e.\ $(\delta-(\lambda+\mu)I)^m(xy) = 0$ for some $m \in \N$. Because $(\delta-\lambda I)^{m_1}(x) = 0$ and $(\delta-\mu I)^{m_2}(y) = 0$ for some $m_1, m_2 \in \N$ this follows from the formula
 \[
  (\delta-(\lambda+\mu)I)^n(xy)
  = \sum_{i=0}^n \binom{n}{i} (\delta-\lambda I)^i(x) (\delta-\mu I)^{n-i}(y)
  \quad\text{for every $n \in \N$},
 \]
 which can be shown by induction over $n \in \N$: For $n = 0$ the statement holds. Suppose it holds for some $n \in \N$. Then
 \begingroup
 \allowdisplaybreaks
 \begin{align*}
  &\, (\delta-(\lambda+\mu)I)^{n+1}(xy)
  = (\delta-(\lambda+\mu)I)((\delta-(\lambda+\mu)I)^n(xy)) \\
  =&\, (\delta-(\lambda+\mu)I)\left(
       \sum_{i=0}^n \binom{n}{i} (\delta-\lambda I)^i(x) (\delta-\mu I)^{n-i}(y)
       \right) \\
  =&\,  \sum_{i=0}^n \binom{n}{i} \delta\left( (\delta-\lambda I)^i(x) \right) (\delta-\mu I)^{n-i}(y)
       +\sum_{i=0}^n \binom{n}{i} (\delta-\lambda I)^i(x) \delta\left( (\delta-\mu I)^{n-i}(y) \right) \\
   &\, -\sum_{i=0}^n \binom{n}{i} \lambda(\delta-\lambda I)^i(x) (\delta-\mu I)^{n-i}(y)
       -\sum_{i=0}^n \binom{n}{i} (\delta-\lambda I)^i(x) \mu(\delta-\mu I)^{n-i}(y) \\
  =&\,  \sum_{i=0}^n \binom{n}{i} (\delta-\lambda I)^{i+1}(x) (\delta-\mu I)^{n-i}(y)
       +\sum_{i=0}^n \binom{n}{i} (\delta-\lambda I)^i(x) (\delta-\mu I)^{n+1-i}(y) \\
  =&\,  \sum_{i=1}^{n+1} \binom{n}{i-1} (\delta-\lambda I)^i(x) (\delta-\mu I)^{n+1-i}(y)
       +\sum_{i=0}^n \binom{n}{i} (\delta-\lambda I)^i(x) (\delta-\mu I)^{n+1-i}(y) \\
  =&\,  \sum_{i=1}^n \left(\binom{n}{i-1}+\binom{n}{i}\right) (\delta-\lambda I)^i(x) (\delta-\mu I)^{n+1-i}(y) \\
   &\, +(\delta-\mu I)^{n+1}(y) + (\delta-\lambda I)^{n+1}(x) \\
  =&\,  \sum_{i=1}^n \binom{n+1}{i} (\delta-\lambda I)^i(x) (\delta-\mu I)^{n+1-i}(y)
       +(\delta-\mu I)^{n+1}(y) + (\delta-\lambda I)^{n+1}(x) \\
  =&\, \sum_{i=0}^{n+1} \binom{n+1}{i} (\delta-\lambda I)^i(x) (\delta-\mu I)^{n+1-i}(y)
 \qedhere
 \end{align*}
 \endgroup
\end{proof}


\begin{rem}
 Notice that the formula in the proof of Lemma~\ref{lem: generalized eigenspace decomposition for derivations} is just a generalization of the binomial theorem, which follows by setting $\delta = 0$.
\end{rem}


\begin{lem}\label{lem: derivations contain the ss and np part of all its elements}
 Let $A$ be a finite dimensional $k$-algebra, not necessarily associative nor unital, e.g.\ a Lie algebra. Then $\Der(A)$ contains the semisimple and nilpotent parts (in $\End_k(A)$) of all its elements.
\end{lem}
\begin{proof}
 Let $\delta \in \Der(A)$ and for every $\lambda \in k$ let $A_\lambda$ be the generalized eigenspace with respect to $\lambda$. To show that $\Der(A)$ contains the semisimple and nilpotent part of $\delta$ is sufficies to do so for the semisimple part, which will be denoted by $\sigma$. As seen in the proof of Theorem~\ref{thrm: concrete Jordan decomposition} (the concrete Jordan decomposition) $\sigma$ acts on $A_\lambda$ by multiplication with $\lambda$ for every $\lambda \in k$. If $x \in A_\lambda$ and $y \in A_\mu$ then $xy \in A_{\lambda + \mu}$ and thus
 \[
  \sigma(xy)
  = (\lambda + \mu)(xy)
  = (\lambda x)y + x(\mu y)
  = \sigma(x)y + x\sigma(y).
 \]
 Because $A = \bigoplus_{\lambda \in k} A_\lambda$ it follows that $\sigma$ is a derivation of $A$.
\end{proof}


\begin{lem}\label{lem: direct complement of semisimple ideals}
 Let $\g$ be a finite dimensional Lie algebra and $I \subideal \g$ a semisimple ideal. Then there exists an ideal $J \subideal \g$ with $\g = I \oplus J$.
\end{lem}
\begin{proof}
 Because $I \subideal \g$ is an ideal it follows that $\kappa_I = \kappa_\g|_{I \times I}$. Because $I$ is semisimple it also follows that $\kappa_I$ is non-degenerate. It follows for the linear map
 \[
  \varphi \colon \g \to I^*, \quad x \mapsto \kappa(x, \cdot)
 \]
 that $\varphi|_I \colon I \to I^*$ is an isomorphism. Hence $I \cap \ker \varphi = 0$ with
 \[
  \dim I = \dim I^* = \dim \g - \dim \ker \varphi.
 \]
 Let $J \coloneqq I^\perp$ be the orthogonal complement of $I$ with respect to the Killing form. From Lemma~\ref{lem: orthogonal complement of an ideal is again an ideal} it is known that $J$ is an ideal in $\g$. Because $J = \ker \varphi$ it follows from the previous observations that $I \cap J = 0$ and $\dim \g = \dim I + \dim J$ and thus $\g = I \oplus J$.
\end{proof}


\begin{lem}\label{lem: every derivation of a semisimple Lie algebra is inner}
 Let $\g$ be a finite dimensional semisimple Lie algebra. Then every derivation of $\g$ is inner.
\end{lem}
\begin{proof}
 As seen in Lemma~\ref{lem: inner derivations are in ideal} and its proof the inner derivations form an ideal $I \subideal \Der(\g)$ with $[\delta,\ad(x)] = \ad(\delta(x))$ for every $\delta \in \Der(\g)$ and $x \in \g$. By Lemma~\ref{lem: direct complement of semisimple ideals} there exists an ideal $J \subideal \Der(\g)$ with $\Der(\g) = I \oplus J$. Let $\delta \in J$. Then for any $x \in \g$
 \[
  \ad(\delta(x)) = [\delta, \ad(x)] \in J \cap I = 0,
 \]
 and thus $\ad(\delta(x)) = 0$. Because $\ad$ is injective it follows that $\delta(x) = 0$ for every $x \in \g$ and thus $\delta = 0$. Therefore $J = 0$ and thus $\Der(A) = I$.
\end{proof}


\begin{thrm}[Abstract Jordan decomposition] \label{thrm: abstract Jordan decomposition}
 Let $\g$ be a finite dimensional semisimple Lie algebra. Then for any $x \in \g$ there exist unique elements $x_s, x_n \in \g$ such that $\ad(x) = \ad(x_s) + \ad(x_n)$ is the concrete Jordan decomposition of $\ad(x)$ with $\ad(x_s) = \ad(x)_s$ and $\ad(x_n) = \ad(x)_n$.
 
 Then $x = x_s + x_n$ and an arbitrary element of $\g$ commutes with $x$ if and only if it commutes both with $x_s$ and $x_n$. In particular $x$, $x_s$ and $x_n$ are pairwise commuting.
\end{thrm}
\begin{proof}
 By Lemma~\ref{lem: every derivation of a semisimple Lie algebra is inner} the adjoint representation $\ad \colon \g \to \ad(\g) = \Der(\g)$ is an isomorphism of Lie algebras. By Lemma~\ref{lem: derivations contain the ss and np part of all its elements} $\Der(\g)$ contains the semisimple and nilpotent parts of all its elements. By setting $x_s \coloneq \ad^{-1}(\ad(x)_s)$ and $x_n \coloneqq \ad^{-1}(\ad(x)_n)$ the theorem follows from the properties of the concrete Jordan decomposition as stated in Theorem~\ref{thrm: concrete Jordan decomposition}.
\end{proof}


\begin{defi}\label{defi: definition of abstract ss and np}
 Let $\g$ be a finite dimensional semisimple Lie algebra and $x \in \g$. Then the decomposition $x = x_s + x_n$ as is Theorem~\ref{thrm: abstract Jordan decomposition} is the \emph{abstract Jordan decomposition} of $x$ where $x_s$ is the \emph{abstract semisimple part} and $x_n$ is the \emph{abstract nilpotent part} of $x$.
\end{defi}


\begin{rem}
 By Definition~\ref{defi: definition of abstract ss and np} an element of a finite dimensional semisimple Lie algebra is semisimple if and only if it is $\ad$-semisimple and nilpotent if and only if it is $\ad$-nilpotent.
\end{rem}


\begin{prop}\label{prop: abstract and concrete Jordan decomposition coincide}
 Let $\g$ be a finite dimensional semisimple linear Lie algebra. Then the abstract and concrete Jordan decomposition coincide.
\end{prop}
\begin{proof}
 Let $x \in \g$, $x = xc_s + x_n$ the abstract Jordan decomposition of $x$ and $x = y_s + y_n$ the concrete Jordan decomposition of $x$. From Lemma~\ref{lem: semisimple linear Lie algebra containing the semisimple and nilpotent parts of its elements} it follows that $y_s, y_n \in \g$ and by Lemma~\ref{lem: ss and nilpotent implies ad-ss and ad-nilpotent} $\ad(x) = \ad(y_s) + \ad(y_n)$ is the concrete Jordan decomposition of $\ad(x)$ with $\ad(y_s) = \ad(x)_s$ and $\ad(y_n) = \ad(x)_n$. Hence $y_s$ is the abstract semisimple part of $x$ and $y_n$ is the abstract nilpotent part of $x$.
\end{proof}


\begin{rem}
 By Proposition~\ref{prop: abstract and concrete Jordan decomposition coincide} we need not to distinguish between the concrete and abstract Jordan decomposition when working with a finite dimensional semisimple linear Lie algebra, which is why we will stop doing so.
\end{rem}


% TODO: Examples all around the Jordan decomposition


\begin{lem}\label{lem: functoriality of the Jordan decomposition for linear Lie algebras}
 Let $\g$ be a finite dimensional semisimple Lie algbra, $\rho \colon \g \to \gl(V)$ a finite dimensional representation of $\g$ and $x \in \g$. If $x = x_s + x_n$ is the Jordan decomposition of $x$ then $\rho(x) = \rho(x_s) + \rho(x_n)$ is the Jordan decomposition of $\rho(x)$ with $\rho(x_s) = \rho(x)_s$ and $\rho(x_n) = \rho(x)_n$.
\end{lem}
\begin{proof}
 Let $x \in \g$ be semisimple (resp.\ nilpotent). Then $x$ is $\ad_\g$-semisimple (resp.\ $\ad_\g$-nilpotent). It follows that $\rho(x)$ is $\ad_{\rho(\g)}$-semisimple (resp.\ $\ad_{\rho(\g)}$-nilpotent) and therefore semisimple (resp.\ nilpotent).
\end{proof}


\begin{cor}[Functoriality of the Jordan decomposition] \label{cor: functoriality of the Jordan decomposition}
 Let $\g_1$ and $\g_2$ be finite dimensional semisimple Lie algebras and $\phi \colon \g_1 \to \g_2$ a homomorphism of Lie algebras. Then $\phi$ preserves the Jordan decomposition, i.e.\ $\phi(x_s) = \phi(x)_s$ and $\phi(x_n) = \phi(x)_n$ for every $x \in \g_1$.
\end{cor}
\begin{proof}
 By Lemma~\ref{lem: functoriality of the Jordan decomposition for linear Lie algebras} the homomorphism of Lie algebras $\ad_{\g_2} \!\circ\, \phi \colon \g_1 \to \gl(\g_2)$ preserves the Jordan decomposition. The homomorphism $\ad_{\g_2}^{-1} \colon \ad(\g_2) \to \g_2$ preserves the Jordan decomposition by the definition of the abstract Jordan decomposition. Hence $\phi = \ad_{\g_2}^{-1} \circ \ad_{\g_2}\! \circ\, \phi$ preserves the Jordan decomposition as well.
\end{proof}






















