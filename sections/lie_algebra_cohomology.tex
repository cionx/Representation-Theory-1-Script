\chapter{Lie Algebra (Co)homology I}
\label{lie algebra cohomology}


\begin{fluff}
  In this \lcnamecref{lie algebra cohomology} we give an introduction to Lie algebra cohomology and Lie algebra homology.
  In this first introduction we try to avoid assuming previous knowledge of homological algebra as far as possible.
\end{fluff}





\section{Calculations}

\begin{recall}
  \leavevmode
  \begin{enumerate}
    \item
      A \defemph{chain complex} of vector spaces consists of a sequence of vector spaces~$(X_n)_{n \in \Integer}$ together with a linear map~$d_n$ from~$X_n$ to~$X_{n-1}$ for every index~$n \in \Integer$ such that~$d_{n-1} \circ d_n = 0$ for every~$n \in \Integer$.
      Such a chain complex can be graphically depicted as a diagram
      \[
        \dotsb
        \from
        X_{n-2}
        \xfrom{d_{n-1}}
        X_{n-1}
        \xfrom{d_n}
        X_n
        \xfrom{d_{n+1}}
        X_{n+1}
        \from
        \dotsb
      \]
      We often denote such a chain complex~$( (X_n)_{n \in \Integer}, (d_n)_{n \in \Integer} )$ as~$(X_\bullet, d_\bullet)$, or simply as~$X_\bullet$.

      A \defemph{cochain complex} of vector spaces consists of a sequence of vector spaces~$(X^n)_{n \in \Integer}$ together with a linear map~$d^n$ from~$X^n$ to~$X^{n+1}$ for every index~$n \in \Integer$ such that~$d^{n+1} \circ d^n = 0$ for every~$n \in \Integer$.
      Such a cochain complex can be graphically depicted as a diagram
      \[
        \dotsb
        \to
        X^{n-1}
        \xto{d^{n-1}}
        X^n
        \xto{d^n}
        X^{n+1}
        \xto{d^{n+1}}
        X^{n+2}
        \to
        \dotsb
      \]
      We often denote such a cochain complex~$( (X^n)_{n \in \Integer}, (d^n)_{n \in \Integer} )$ as~$(X^\bullet, d^\bullet)$, or simply as~$X^\bullet$.
    \item
      Let~$X_\bullet = ( (X_n)_{n \in \Integer}, (d_n)_{n \in \Integer} )$ be a chain complex and let~$n$ be some integer.
      The linear subspace~$\ker(d_n)$ of~$X_n$ is denoted by~$\Cycle_n(X_\bullet)$ and its elements are the~\defemph{\cycles{$n$}} of~$X_\bullet$.
      The linear subspace~$\im(d_{n+1})$ of~$X_n$ is denoted by~$\Boundary_n(X_\bullet)$ and its elements are the~\defemph{\boundaries{$n$}} of~$X_\bullet$.
      It follows from the condition~$d_n \circ d_{n+1} = 0$ that~$\Boundary_n(X_\bullet)$ is a linear subspace of~$\Cycle_n(X_\bullet)$.
      The quotient vector space~$\Cycle_n(X_\bullet) / {\Boundary_n(X_\bullet)}$ is the~\defemph{\howmanyth{$n$} homology} of~$X_\bullet$.
      It is denoted by~$\Homology_n(X_\bullet)$.

      Let~$X^\bullet = ( (X^n)_{n \in \Integer}, (d^n)_{n \in \Integer} )$ be a cochain complex and let~$n$ be some integer.
      The linear subspace~$\ker(d^n)$ of~$X^n$ is denoted by~$\Cycle^n(X^\bullet)$ and its elements are the~\defemph{\cocycles{$n$}} of~$X$.
      The linear subspace~$\im(d^{n-1})$ of~$X^n$ is denoted by~$\Boundary^n(X^\bullet)$ and its elements are the~\defemph{\coboundaries{$n$}} of~$X$.
      It follows from the condition~$d^n \circ d^{n-1} = 0$ that~$\Boundary^n(X^\bullet)$ is a linear subspace of~$\Cycle^n(X^\bullet)$.
      The quotient vector space~$\Cycle^n(X^\bullet) / {\Boundary^n(X^\bullet)}$ is the~\defemph{\howmanyth{$n$} cohomology} of~$X$.
      It is denoted by~$\Homology^n(X^\bullet)$.
  \end{enumerate}
\end{recall}


\begin{lemma}
  \label{alternating in multiple arguments}
  Let~$\glie$ be a Lie~algebra and let~$x_1, \dotsc, x_n$ be elements of~$\glie$.
  Then
  \[
    \sum_{1 \leq i < j \leq n}
    (-1)^{i+j}
    [x_i, x_j] \wedge x_1 \wedge \dotsb \wedge \widehat{x_i} \wedge \dotsb \wedge \widehat{x_j} \wedge \dotsb x_n
    =
    0
  \]
  whenever~$x_r$ equals~$x_s$ for some~$r \neq s$.
\end{lemma}


\begin{proof}
  We may assume that~$r < s$.
  We denote the expression on the left hand side of the claimed identity by~$h_n(x_1, \dotsc, x_n)$.
  We show the identity by induction over~$n$.
  It holds for~$n = 0$ and~$n = 1$ because the sum is empty.
  For~$n \geq 2$ we split off those summands with~$j = n$, which gives us the formula
  \begin{align*}
    h_n(x_1, \dotsc, x_n)
    =
    h_{n-1}(x_1, \dotsc, x_{n-1}) \wedge x_n
    +
    \sum_{p=1}^{n-1}
    (-1)^{p+n}
    [x_p, x_n] \wedge x_1 \wedge \dotsb \wedge \widehat{x_p} \wedge \dotsb \wedge x_{n-1} \,.
  \end{align*}
  We now distinguish between two cases.

  Suppose first that~$s < n$.
  Then~$h_{n-1}(x_1, \dotsc, x_{n-1}) = 0$ by induction.
  In the sum
  \[
    \sum_{p=1}^{n-1}
    (-1)^{p+n}
    [x_p, x_n] \wedge x_1 \wedge \dotsb \wedge \widehat{x_p} \wedge \dotsb \wedge x_{n-1}
  \]
  one those summands with~$p = r, s$ are possibly nonzero.
  We thus need to show that these two summands
  \[
    (-1)^{r+n}
    [x_r, x_n] \wedge x_1 \wedge \dotsb \wedge \widehat{x_r} \wedge \dotsb \wedge x_{n-1}
  \]
  and
  \[
    (-1)^{s+n}
    [x_s, x_n] \wedge x_1 \wedge \dotsb \wedge \widehat{x_s} \wedge \dotsb \wedge x_{n-1}
  \]
  cancel out.
  This holds because the simple wedges
  \[
    [x_r, x_n] \wedge x_1 \wedge \dotsb \wedge \widehat{x_r} \wedge \dotsb \wedge x_{n-1}
    =
    [x_r, x_n] \wedge x_1 \wedge \dotsb \wedge \widehat{x_r} \wedge \dotsb \wedge x_s \wedge \dotsb \wedge x_{n-1}
  \]
  and
  \[
    [x_s, x_n] \wedge x_1 \wedge \dotsb \wedge \widehat{x_s} \wedge \dotsb \wedge x_{n-1}
    =
    [x_s, x_n] \wedge x_1 \wedge \dotsb \wedge x_r \wedge \dotsb \wedge \widehat{x_s} \wedge \dotsb \wedge x_{n-1}
  \]
  differ only by the sign~$(-1)^{s-r-1}$.

  Suppose now that~$s = n$.
  Then~$x_r = x_s = x_n$.
  In the term
  \[
    h_{n-1}(x_1, \dotsc, x_n)
    =
    \sum_{1 \leq i < j \leq n-1}
    (-1)^{i+j}
    [x_i, x_j] \wedge x_1 \wedge \dotsb \wedge \widehat{x_i} \wedge \dotsb \wedge \widehat{x_j} \wedge \dotsb x_n
  \]
  those summands for~$i, j \neq r$ contain~$x_r$ as a wedge factor.
  When passing from~$h_{n-1}(x_1, \dotsc, x_{n-1})$ to~$h_{n-1}(x_1, \dotsc, x_{n-1}) \wedge x_n$ all those summands disappear because~$x_r = x_n$.
  We therefore only have to worry about those summands with~$i = r$ or~$j = r$, in the sense that
  \begin{equation}
    \label{first term as sum of two sums}
    \begin{aligned}
      h_{n-1}(x_1, \dotsc, x_{n-1}) \wedge x_n
      ={}&
      \sum_{i=1}^{r-1}
      (-1)^{i+r}
      [x_i, x_r] \wedge x_1 \wedge \dotsb \wedge \widehat{x_i} \wedge \dotsb \wedge \widehat{x_r} \wedge \dotsb \wedge x_n
      \\
      {}&
      +
      \sum_{j=r+1}^{n-1}
      (-1)^{r+j}
      [x_r, x_j] \wedge x_1 \wedge \dotsb \wedge \widehat{x_r} \wedge \dotsb \wedge \widehat{x_j} \wedge \dotsb \wedge x_n \,.
    \end{aligned}
  \end{equation}
  We have to show that this term is the negative of
  \[
    \sum_{p=1}^{n-1}
    (-1)^{p+n}
    [x_p, x_n] \wedge x_1 \wedge \dotsb \wedge \widehat{x_p} \wedge \dotsb \wedge x_{n-1} \,.
  \]
  We note that in this sum the summand for~$p = r$ vanishes since~$x_r = x_n$ and thus~$[x_r, x_n] = 0$.
  This sum can therefore be split up into
  \begin{equation}
    \label{second term as sum of two sums}
    \begin{aligned}
      {}&
      \sum_{i=1}^{r-1}
      (-1)^{i+n}
      [x_i, x_n] \wedge x_1 \wedge \dotsb \wedge \widehat{x_i} \wedge \dotsb \wedge x_{n-1}
      \\
      {}&
      +
      \sum_{j=r+1}^{n-1}
      (-1)^{j+n}
      [x_j, x_n] \wedge x_1 \wedge \dotsb \wedge \widehat{x_j} \wedge \dotsb \wedge x_{n-1} \,.
    \end{aligned}
  \end{equation}

  It follows for every~$i = 1, \dotsc, r-1$ from the equality~$x_r = x_n$ that
  \begin{align*}
    {}&
    [x_i, x_r] \wedge x_1 \wedge \dotsb \wedge \widehat{x_i} \wedge \dotsb \wedge \widehat{x_r} \wedge \dotsb \wedge x_n
    \\
    ={}&
    [x_i, x_r] \wedge x_1 \wedge \dotsb \wedge \widehat{x_i} \wedge \dotsb \wedge \widehat{x_r} \wedge \dotsb \wedge x_{n-1} \wedge x_n
    \\
    ={}&
    [x_i, x_n] \wedge x_1 \wedge \dotsb \wedge \widehat{x_i} \wedge \dotsb \wedge \widehat{x_r} \wedge \dotsb \wedge x_{n-1} \wedge x_r
    \\
    ={}&
    (-1)^{n-r-1}
    [x_i, x_r] \wedge x_1 \wedge \dotsb \wedge \widehat{x_i} \wedge \dotsb \wedge x_{n-1}
  \end{align*}
  and therefore
  \begin{align*} 
    {}&
    \sum_{i=1}^{r-1}
    (-1)^{i+r}
    [x_i, x_r] \wedge x_1 \wedge \dotsb \wedge \widehat{x_i} \wedge \dotsb \wedge \widehat{x_r} \wedge \dotsb \wedge x_n
    \\
    ={}&
    \sum_{i=1}^{r-1}
    (-1)^{i+n-1}
    [x_i, x_n] \wedge x_1 \wedge \dotsb \wedge \widehat{x_i} \wedge \dotsb \wedge x_{n-1}
    \\
    ={}&
    -
    \sum_{i=1}^{r-1}
    (-1)^{i+n}
    [x_i, x_n] \wedge x_1 \wedge \dotsb \wedge \widehat{x_i} \wedge \dotsb \wedge x_{n-1} \,.
  \end{align*}
  This shows that the fist sum in~\eqref{first term as sum of two sums} cancels out the first sum in~\eqref{second term as sum of two sums}.

  It follows similarly for every~$j = r+1, \dotsc, n-1$ from the equality~$x_n = x_r$ that
  \begin{align*}
    {}&
    [x_r, x_j] \wedge x_1 \wedge \dotsb \wedge \widehat{x_r} \wedge \dotsb \wedge \widehat{x_j} \wedge \dotsb \wedge x_n
    \\
    ={}&
    [x_r, x_j] \wedge x_1 \wedge \dotsb \wedge \widehat{x_r} \wedge \dotsb \wedge \widehat{x_j} \wedge \dotsb \wedge x_{n-1} \wedge x_n
    \\
    ={}&
    [x_n, x_j] \wedge x_1 \wedge \dotsb \wedge \widehat{x_r} \wedge \dotsb \wedge \widehat{x_j} \wedge \dotsb \wedge x_{n-1} \wedge x_r
    \\
    ={}&
    (-1)^{n-r}
    [x_n, x_j] \wedge x_1 \wedge \dotsb \wedge \widehat{x_j} \wedge \dotsb \wedge x_{n-1}
  \end{align*}
  and therefore
  \begin{align*}
    {}&
    \sum_{j = r+1}^{n-1}
    (-1)^{r+j}
    [x_r, x_j] \wedge x_1 \wedge \dotsb \wedge \widehat{x_r} \wedge \dotsb \wedge \widehat{x_j} \wedge \dotsb \wedge x_n
    \\
    ={}&
    \sum_{j = r+1}^{n-1}
    (-1)^{j+n}
    [x_n, x_j] \wedge x_1 \wedge \dotsb \wedge \widehat{x_j} \wedge \dotsb \wedge x_{n-1}
    \\
    ={}&
    -\sum_{j = r+1}^{n-1}
    (-1)^{j+n}
    [x_j, x_n] \wedge x_1 \wedge \dotsb \wedge \widehat{x_j} \wedge \dotsb \wedge x_{n-1}
  \end{align*}
  This shows that the second sum in~\eqref{first term as sum of two sums} cancels out the second sum in~\eqref{second term as sum of two sums}.
\end{proof}


\begin{definition}
  Let~$V$ and~$W$ be two~\vectorspaces{$\kf$}~$V$ and~$W$.
  The vector space of multilinear, alternating maps from the~\fold{$n$} product~$V \times \dotsb \times V$ to~$W$ is denoted by
  \[
    \Alt^n(V,W) \,.
  \]
\end{definition}


\begin{recall}
  \label{expressing alt with exterior powers}
  We can identify~$\Alt^n(V, W)$ with~$\Hom_{\kf}( \Exterior^n(V), W)$ by the universal property of the exterior power.
  More explicitely, for an element~$\omega'$ of~$\Hom_{\kf}( \Exterior^n(V), W)$ the corresponding element~$\omega$ of~$\Alt^n(V,W)$ is given by
  \[
    \omega(v_1, \dotsc, v_n)
    =
    \omega'( v_1 \wedge \dotsb \wedge v_n )
  \]
  for all~$v_1, \dotsc, v_n \in V$.
\end{recall}


\begin{definition}
  \label{construction of lie algebra homology and cohomology}
  Let~$\glie$ be a Lie~algebra and let~$M$ be a representation of~$\glie$.
  \begin{enumerate}
    \item
      For every integer~$n$ let
      \[
        \Chain_n(\glie, M)
        \defined
        \begin{cases*}
          \Exterior^n(\glie) \otimes M  & if~$n \geq 0$,  \\
          0                             & otherwise.
        \end{cases*}
      \]
    \item
      For every integer~$n$ let
      \[
        \Chain^n(\glie, M)
        \defined
        \begin{cases*}
          \Alt^n(\glie, M)  & if~$n \geq 0$,  \\
          0                 & otherwise.
        \end{cases*}
      \]
  \end{enumerate}
\end{definition}


\begin{proposition}
  Let~$\glie$ be a Lie~algebra and let~$M$ be a representation of~$\glie$.
  \begin{enumerate}
    \item
      There exists for every integer~$n$ with~$n \geq 1$ a unique linear map
      \[
        d_n
        \colon
        \Chain_n(\glie, M)
        \to
        \Chain_{n-1}(\glie, M)
      \]
      given by
      \begin{align*}
        x_1 \wedge \dotsb \wedge x_n \otimes m
        \mapsto
        {}&
        \sum_{1 \leq i < j \leq n}
        (-1)^{i+j}
        [x_i, x_j] \wedge x_1 \wedge \dotsb \wedge \widehat{x_i} \wedge \dotsb \wedge \widehat{x_j} \wedge \dotsb \wedge x_n \otimes m
        \\
        {}&
        +
        \sum_{i=1}^n
        (-1)^i
        x_1 \wedge \dotsb \wedge \widehat{x_i} \wedge \dotsb \wedge x_n \otimes (x_i \act m)
      \end{align*}
      for all~$x_1, \dotsc, x_n \in \glie$ and~$m \in M$.
    \item
      Together with the maps~$d_n \defined 0$ for every~$n \leq 0$ these maps satisfy the identity~$d_{n-1} \circ d_n = 0$ for every integer~$n$.
  \end{enumerate}
  We have thus constructed a chain complex~$\Chain_\bullet(\glie, M)$.
  \begin{enumerate}[resume*]
    \item
      There exists for every integer~$n$ with~$n \geq 0$ a unique linear map
      \[
        d^n
        \colon
        \Chain^n( \glie, M )
        \to
        \Chain^{n+1}( \glie, M)
      \]
      given by
      \begin{align*}
        d^n(\omega)(x_1, \dotsc, x_{n+1})
        \mapsto
        {}&
        \sum_{1 \leq i < j \leq n+1}
        (-1)^{i+j} \omega([x_i, x_j], x_1, \dotsc, \widehat{x_i}, \dotsc, \widehat{x_j}, \dotsc, x_{n+1})
        \\
        {}&
        +
        \sum_{i=1}^{n+1}
        (-1)^{i+1}
        x_i \act \omega(x_1, \dotsc, \widehat{x_i}, \dotsc, x_{n+1})
      \end{align*}
      for all~$x_1, \dotsc, x_{n+1} \in \glie$ and~$m \in M$.
    \item
      Together with the maps~$d_n \defined 0$ for every~$n < 0$ these maps satisfy the identity~$d^{n+1} \circ d^n = 0$ for every integer~$n$.
  \end{enumerate}
  We have thus constructed a cochain complex~$\Chain^\bullet(\glie, M)$.
\end{proposition}


\begin{proof}
  \leavevmode
  \begin{enumerate}
    \item
      There exists a multilinear map
      \[
        \widetilde{d_n}
        \colon
        \underbrace{ \glie \times \dotsb \times \glie }_{n} \times M
        \to
        \Exterior^{n-1}(\glie) \otimes M
      \]
      given by
      \begin{align}
        (x_1, \dotsc, x_n, m)
        \mapsto
        {}&
        \sum_{1 \leq i < j \leq n}
        (-1)^{i+j}
        [x_i, x_j] \wedge x_1 \wedge \dotsb \wedge \widehat{x_i} \wedge \dotsb \wedge \widehat{x_j} \wedge \dotsb \wedge x_n \otimes m
        \label{first term for homology}
        \\
        {}&
        +
        \sum_{i=1}^n
        (-1)^i
        x_1 \wedge \dotsb \wedge \widehat{x_i} \wedge \dotsb \wedge x_n \otimes (x_i \act m)
        \label{second term for homology}
      \end{align}
      for all~$x_1, \dotsc, x_n \in \glie$ and~$m \in M$.
      It sufficies to show that this map is alternating in~$x_1, \dotsc, x_n$, i.e. that
      \[
        \widetilde{d_n}(x_1, \dotsc, x_n, m) = 0
      \]
      whenever there exist some indices~$r$,~$s$ with~$1 \leq r < s \leq n$ and~$x_r = x_s$.%
%      \footnote{
%        If this condition holds then we can consider for every fixed element~$m$ of~$M$ the resulting mulilinear map~$\widetilde{h_m}$ given by~$\widetilde{d_n}(-, -, \dotsc, -, m)$.
%        This map is then multilinear and alternating, and thus induces a linear map~$h_m$ from~$\Exterior^n(\glie)$ to~$\Exterior^{n-1}(\glie) \otimes M$.
%        It follows from the multilinearity of the original map~$\widetilde{d_n}$ that the resulting family of linear maps~$(h_m)_{m \in M}$ assembles into a bilinear map from~$\Exterior^n(\glie) \times M$ into~$\Exterior^{n-1}(\glie) \otimes M$.
%        This bilinear map corresponds to the desired linear map~$d_n$ under the universal property of the tensor product.
%      }

      Suppose that~$x_r = x_s$ for~$r$,~$s$ as above.
      The term~\eqref{first term for homology} vanshes by \cref{alternating in multiple arguments}
      For the term~\eqref{second term for homology} we note that every summand with~$i \neq r, s$ vanishes since~$x_r = x_s$.
      The two remaining summands are given by
      \begin{equation}
        \label{first summand for second term}
        (-1)^r x_1 \wedge \dotsb \wedge \widehat{x_r} \wedge \dotsb \wedge x_n \otimes (x_r \act m)
      \end{equation}
      and
      \begin{equation}
        \label{second summand for second term}
        (-1)^s x_1 \wedge \dotsb \wedge \widehat{x_s} \wedge \dotsb \wedge x_n \otimes (x_s \act m) \,.
      \end{equation}
      The tensor factors~$x_r \act m$ and~$x_s \act m$ are equal because~$x_r$ equals~$x_s$.
      The simple wedges
      \[
        x_1 \wedge \dotsb \wedge \widehat{x_r} \wedge \dotsb \wedge x_n
        =
        x_1 \wedge \dotsb \wedge \widehat{x_r} \wedge \dotsb \wedge x_s \wedge \dotsb \wedge x_n
      \]
      and
      \[
        x_1 \wedge \dotsb \wedge \widehat{x_s} \wedge \dotsb \wedge x_n
        =
        x_1 \wedge \dotsb \wedge x_r \wedge \dotsb \wedge \widehat{x_s} \wedge \dotsb \wedge x_n
      \]
      differ only by the sign~$(-1)^{s-r-1}$ because~$x_r$ equals~$x_s$.
      The two summands~\eqref{first summand for second term} and~~\eqref{second summand for second term} differ therefore only by the sign~$-1$.
      They hence cancel out.
    \item
      The adjoint action of~$\glie$ on itself and the action on~$M$ induce for every natural number~$n$ an action of~$\glie$ on~$\Exterior^n(V) \otimes M$.
      This action is given by
      \begin{align*}
        x \act (x_1 \wedge \dotsb \wedge x_n \otimes m)
        &=
        (x \act (x_1 \wedge \dotsb \wedge x_n)) \otimes m
        + x_1 \wedge \dotsb \wedge x_n \otimes (x \act m)
        \\
        &=
        \sum_{i=1}^n x_1 \wedge \dotsb \wedge (x \act x_i) \wedge \dotsb \wedge x_n \otimes m
        + x_1 \wedge \dotsb \wedge x_n \otimes (x \act m) \,.
        \\
        &=
        \sum_{i=1}^n x_1 \wedge \dotsb \wedge (x \act x_i) \wedge \dotsb \wedge x_n \otimes m
        + x_1 \wedge \dotsb \wedge x_n \otimes (x \act m) \,.
      \end{align*}

      We now observe for every~$n \geq 2$ that
      \begin{equation}
        \label{important identity for homology}
        d_n( x \wedge t \otimes m )
        =
        - x \wedge d_{n-1}( t \otimes m)
        - x \act (t \otimes m)
      \end{equation}
      for every element~$x$ of~$\glie$, every element~$t$ of~$\Exterior^{n-1}(\glie)$ and every element~$m$ of~$M$.
      To show this identity it sufficies to consider for~$t$ a simple wedge~$x_1 \wedge \dotsb \wedge x_{n+1}$ where~$x_1, \dotsc, x_{n-1}$ are elements of~$\glie$.
      By setting
      \[
        y_1 \defined x \,,
        \quad
        y_2 \defined x_1 \,,
        \quad
        \dotsc \,,
        \quad
        y_n \defined x_{n-1}
      \]
      we find that
      \begin{align}
        {}&
        d_n( x \wedge (x_1 \dotsb \wedge x_{n-1}) \otimes m )
        \notag
        \\
        ={}&
        d_n ( x \wedge x_1 \dotsb \wedge x_{n-1} \otimes m )
        \notag
        \\
        ={}&
        d_n ( y_1 \wedge \dotsb \wedge y_n \otimes m )
        \notag
        \\
        ={}&
        \sum_{1 \leq i < j \leq n}
        (-1)^{i + j}
        [y_i, y_j] \wedge y_1 \wedge \dotsb \wedge \widehat{y_i} \wedge \dotsb \wedge \widehat{y_j} \wedge \dotsb \wedge y_n
        \otimes m
        \label{first term to be expanded for homology}
        \\
        {}&
        +
        \sum_{i=1}^n
        (-1)^i
        y_1 \wedge \dotsb \wedge \widehat{y_i} \wedge \dotsb \wedge y_n \otimes (y_i \act m) \,.
        \label{second term to be expanded for homology}
      \end{align}
      We expand the term~\eqref{first term to be expanded for homology} as
      \begin{align*}
        {}&
        \sum_{1 \leq i < j \leq n}
        (-1)^{i + j}
        [y_i, y_j] \wedge y_1 \wedge \dotsb \wedge \widehat{y_i} \wedge \dotsb \wedge \widehat{y_j} \wedge \dotsb \wedge y_n
        \otimes m
        \\
        ={}&
        \sum_{2 \leq i < j \leq n}
        (-1)^{i + j}
        [y_i, y_j] \wedge y_1 \wedge \dotsb \wedge \widehat{y_i} \wedge \dotsb \wedge \widehat{y_j} \wedge \dotsb \wedge y_n
        \otimes m
        \\
        {}&
        +
        \sum_{j=2}^n
        (-1)^{j+1}
        [y_1, y_j] \wedge y_2 \wedge \dotsb \wedge \widehat{y_j} \wedge \dotsb \wedge y_n
        \otimes m
        \\
        ={}&
        -
        \sum_{2 \leq i < j \leq n}
        (-1)^{i + j}
        y_1 \wedge [y_i, y_j] \wedge y_2 \wedge \dotsb \wedge \widehat{y_i} \wedge \dotsb \wedge \widehat{y_j} \wedge \dotsb \wedge y_n
        \otimes m
        \\
        {}&
        -
        \sum_{j=2}^n
        y_2 \wedge \dotsb \wedge [y_1, y_j] \wedge \dotsb \wedge y_n
        \otimes m
        \\
        ={}&
        -
        \sum_{2 \leq i < j \leq n}
        (-1)^{i + j}
        x \wedge [x_{i-1}, x_{j-1}] \wedge x_1 \wedge \dotsb \wedge \widehat{x_{i-1}} \wedge \dotsb \wedge \widehat{x_{j-1}} \wedge \dotsb \wedge x_{n-1}
        \otimes m
        \\
        {}&
        -
        \sum_{j=2}^n
        x_1 \wedge \dotsb \wedge [x, x_{j-1}] \wedge \dotsb \wedge x_{n-1}
        \otimes m
        \\
        ={}&
        -
        \sum_{1 \leq i < j \leq n-1}
        (-1)^{i + j}
        x \wedge [x_i, x_j] \wedge x_1 \wedge \dotsb \wedge \widehat{x_i} \wedge \dotsb \wedge \widehat{x_j} \wedge \dotsb \wedge x_{n-1}
        \otimes m
        \\
        {}&
        -
        \sum_{j=1}^{n-1}
        x_1 \wedge \dotsb \wedge [x, x_j] \wedge \dotsb \wedge x_{n-1}
        \otimes m
      \end{align*}
      We also expand the term~\eqref{second term to be expanded for homology} as
      \begin{align*}
        {}&
        \sum_{i=1}^n
        (-1)^i
        y_1 \wedge \dotsb \wedge \widehat{y_i} \wedge \dotsb \wedge y_n \otimes (y_i \act m)
        \\
        ={}&
        \sum_{i=2}^n
        (-1)^i
        y_1 \wedge \dotsb \wedge \widehat{y_i} \wedge \dotsb \wedge y_n \otimes (y_i \act m)
        -
        y_2 \wedge \dotsb \wedge y_n \otimes (y_1 \act m)
        \\
        ={}&
        \sum_{i=2}^n
        (-1)^i
        x \wedge x_1 \wedge \dotsb \wedge \widehat{x_{i-1}} \wedge \dotsb \wedge x_{n-1} \otimes (x_{i-1} \act m)
        -
        x_1 \wedge \dotsb \wedge x_{n-1} \otimes (x \act m)
        \\
        ={}&
        \sum_{i=1}^{n-1}
        (-1)^{i+1}
        x \wedge x_1 \wedge \dotsb \wedge \widehat{x_i} \wedge \dotsb \wedge x_{n-1} \otimes (x_i \act m)
        -
        x_1 \wedge \dotsb \wedge x_{n-1} \otimes (x \act m)
        \\
        ={}&
        -
        \sum_{i=1}^{n-1}
        (-1)^i
        x \wedge x_1 \wedge \dotsb \wedge \widehat{x_i} \wedge \dotsb \wedge x_{n-1} \otimes (x_i \act m)
        -
        x_1 \wedge \dotsb \wedge x_{n-1} \otimes (x \act m)
      \end{align*}
      Putting these calculations together we find that
      \begin{align*}
        {}&
        d_n( x \wedge (x_1 \wedge \dotsb \wedge x_n) \otimes m )
        \\
        % first group of terms
        ={}&
        -
        \sum_{1 \leq i < j \leq n-1}
        (-1)^{i + j}
        x \wedge [x_i, x_j] \wedge x_1 \wedge \dotsb \wedge \widehat{x_i} \wedge \dotsb \wedge \widehat{x_j} \wedge \dotsb \wedge x_{n-1}
        \otimes m
        \\
        {}&
        -
        \sum_{j=1}^{n-1}
        x_1 \wedge \dotsb \wedge [x, x_j] \wedge \dotsb \wedge x_{n-1}
        \otimes m
        \\
        {}&
        -
        \sum_{i=1}^{n-1}
        (-1)^i
        x \wedge x_1 \wedge \dotsb \wedge \widehat{x_i} \wedge \dotsb \wedge x_{n-1} \otimes (x_i \act m)
        -
        x_1 \wedge \dotsb \wedge x_{n-1} \otimes (x \act m)
        \\
        % second group of terms
        ={}&
        -
        \sum_{1 \leq i < j \leq n-1}
        (-1)^{i + j}
        x \wedge [x_i, x_j] \wedge x_1 \wedge \dotsb \wedge \widehat{x_i} \wedge \dotsb \wedge \widehat{x_j} \wedge \dotsb \wedge x_{n-1}
        \otimes m
        \\
        {}&
        -
        \sum_{i=1}^{n-1}
        (-1)^i
        x \wedge x_1 \wedge \dotsb \wedge \widehat{x_i} \wedge \dotsb \wedge x_{n-1} \otimes (x_i \act m)
        \\
        {}&
        -
        \sum_{j=1}^{n-1}
        x_1 \wedge \dotsb \wedge [x, x_j] \wedge \dotsb \wedge x_{n-1}
        \otimes m
        -
        x_1 \wedge \dotsb \wedge x_{n-1} \otimes (x \act m)
        \\
        % third group of terms
        ={}&
        -
        x \wedge
        \Biggl(
        \sum_{1 \leq i < j \leq n-1}
        (-1)^{i + j}
        [x_i, x_j] \wedge x_1 \wedge \dotsb \wedge \widehat{x_i} \wedge \dotsb \wedge \widehat{x_j} \wedge \dotsb \wedge x_{n-1}
        \otimes m
        \\
        {}&
        -
        \sum_{i=1}^{n-1}
        (-1)^i
        x_1 \wedge \dotsb \wedge \widehat{x_i} \wedge \dotsb \wedge x_{n-1} \otimes (x_i \act m)
        \Biggr)
        \\
        {}&
        -
        \sum_{j=1}^{n-1}
        x_1 \wedge \dotsb \wedge [x, x_j] \wedge \dotsb \wedge x_{n-1}
        \otimes m
        -
        x_1 \wedge \dotsb \wedge x_{n-1} \otimes (x \act m)
        \\
        ={}&
        - x \wedge d_{n-1}(x_1 \wedge \dotsb \wedge x_{n-1} \otimes m)
        - x \act (x_1 \wedge \dotsb \wedge x_{n-1} \otimes m) \,.
      \end{align*}
      We have thus shown the identity~\eqref{important identity for homology}.

      We now show that the linear maps~$d_n$ are actually homomorphisms of representations for every~$n \geq 1$.
      We show this by induction over~$n$.
      For~$n = 1$ we have~$\Exterior^1(\glie) = \glie$ and
      \begin{align*}
        x \act d_1(x_1 \otimes m)
        &=
        - x \act x_1 \act m
        \\
        &=
        - ( [x, x_1] \act m + x_1 \act x \act m )
        \\
        &=
        - [x, x_1] \act m - x_1 \act x \act m
        \\
        &=
        d_1( [x, x_1] \otimes m )
        + d_1( x_1 \otimes (x \act m) )
        \\
        &=
        d_1\bigl( [x, x_1] \otimes m + x_1 \otimes (x \act m) \bigr)
        \\
        &=
        d_1( x \act (x_1 \otimes m) ) \,.
      \end{align*}
      This shows that~$d_1$ is a homomorphism of representations.
      If~$d_n$ is a homomorphism of representations for some~$n \geq 1$ then we have for every element~$x$ of~$\glie$, every element~$x_1$ of~$\glie$, every element~$t$ of~$\Exterior^n(\glie)$ and every element~$m$ of~$M$ that
      \begin{align*}
        {}&
        d_{n+1}( x \act (x_1 \wedge t \otimes m) )
        \\
        ={}&
        d_{n+1}
        \bigl(
          [x, x_1] \wedge t \otimes m
          + x_1 \wedge (x \act t) \otimes m
          + x_1 \wedge t \otimes (x \act m)
        \bigr)
        \\
        ={}&
        d_{n+1}\bigl( [x, x_1] \wedge t \otimes m \bigr)
        + d_{n+1}\bigl( x_1 \wedge (x \act t) \otimes m \bigr)
        + d_{n+1}\bigl( x_1 \wedge t \otimes (x \act m) \bigr)
        \\
        ={}&
        - [x, x_1] \wedge d_n(t \otimes m)
        - [x, x_1] \act (t \otimes m)
        - x_1 \wedge d_n( (x \act t) \otimes m )
        - x_1 \act ( (x \act t) \otimes m )
        \\
        {}&
        - x_1 \wedge d_n( t \otimes (x \act m) )
        - x_1 \act ( t \otimes (x \act m) )
        \\
        ={}&
        - [x, x_1] \wedge d_n(t \otimes m)
        - [x, x_1] \act (t \otimes m)
        - x_1 \wedge d_n( (x \act t) \otimes m )
        - x_1 \wedge d_n( t \otimes (x \act m) )
        \\
        {}&
        - x_1 \act ( (x \act t) \otimes m )
        - x_1 \act ( t \otimes (x \act m) )
        \\
        ={}&
        - [x, x_1] \wedge d_n(t \otimes m)
        - [x, x_1] \act (t \otimes m)
        - x_1 \wedge d_n\bigl( (x \act t) \otimes m + t \otimes (x \act m) \bigr)
        \\
        {}&
        - x_1 \act \bigl( (x \act t) \otimes m + t \otimes (x \act m) \bigr)
        \\
        ={}&
        - [x, x_1] \wedge d_n(t \otimes m)
        - [x, x_1] \act (t \otimes m)
        - x_1 \wedge d_n( x \act (t \otimes m) )
        - x_1 \act ( x \act (t \otimes m) )
        \\
        ={}&
        - [x, x_1] \wedge d_n(t \otimes m)
        - [x, x_1] \act (t \otimes m)
        - x_1 \wedge ( x \act d_n(t \otimes m) )
        - x_1 \act ( x \act (t \otimes m) )
        \\
        ={}&
        - [x, x_1] \wedge d_n(t \otimes m)
        - x_1 \wedge ( x \act d_n(t \otimes m) )
        - [x, x_1] \act (t \otimes m)
        - x_1 \act ( x \act (t \otimes m) )
        \\
        ={}&
        - x \act ( x_1 \wedge d_n(t \otimes m) )
        - x \act x_1 \act (t \otimes m)
        \\
        ={}&
        x \act ( - x_1 \wedge d_n(t \otimes m) - x_1 \act (t \otimes m) )
        \\
        ={}&
        x \act d_{n+1}( x_1 \wedge t \otimes m ) \,.
      \end{align*}
      This then shows that~$d_{n+1}$ is also a homomorphism of representations.

      For every element~$x$ of~$\glie$ let~$\theta_n^x$ denote the action of~$x$ on~$\Exterior^n(\glie) \otimes M$, i.e. the linear map
      \[
        \theta_n^x
        \colon
        \Exterior^n(\glie) \otimes M
        \to
        \Exterior^n(\glie) \otimes M \,,
        \quad
        z
        \mapsto
        x \act z \,.
      \]
      We also have for every element~$x$ of~$\glie$ an auxiliary linear map
      \[
        \sigma_n^x
        \colon
        \Exterior^n(\glie)
        \to
        \Exterior^{n+1}(\glie)
      \]
      which is given by
      \[
        \sigma_n^x( t \otimes m)
        =
        x \wedge t \otimes m
      \]
      for every element~$t$ of~$\Exterior^n(\glie)$ and every element~$m$ of~$M$.
      We have shown in~\eqref{important identity for homology} that
      \[
        d_n \circ \sigma_{n-1}^x
        =
        - \sigma_{n-1}^x \circ d_{n-1}
        - \theta_{n-1}^x
      \]
      for every element~$x$ of~$\glie$ and every~$n \geq 2$.
      We have also shown that
      \[
        d_n \circ \theta_n^x
        =
        \theta_{n-1}^x \circ d_n
      \]
      for every element~$x$ of~$\glie$ and every~$n \geq 1$.

      We can now show the desired identity
      \[
        d_{n-1} \circ d_n = 0
      \]
      for~$n \in \Integer$ by induction over~$n$.
      The identity holds for~$n \leq 1$ because then~$d_{n-1} = 0$.
      It also holds for~$n = 2$ because
      \begin{align*}
        d_1( d_2( x_1 \wedge x_2 \otimes m) )
        &=
        d_1( -[x_1, x_2] \otimes m - x_2 \wedge (x_1 \act m) + x_1 \wedge (x_2 \act m) )
        \\
        &=
        - d_1( [x_1, x_2] \otimes m )
        - d_1( x_2 \wedge (x_1 \act m) )
        + d_1( x_1 \wedge (x_2 \act m) )
        \\
        &=
        [x_1, x_2] \act m
        + x_2 \act (x_1 \act m)
        - x_1 \act (x_2 \act m)
        \\
        &=
        0
      \end{align*}
      for any two elements~$x_1$,~$x_2$ of~$\glie$ and every element~$m$ of~$M$.%
      \footnote{
        It should be noted that the condition~$d_1 \circ d_2 = 0$ is satisfied precisely because~$M$ is a representation of~$\glie$.
      }
      For~$n \geq 3$ it sufficies to show that
      \[
        d_{n-1}( d_n( x \wedge t \otimes m) )
        =
        0
      \]
      for every element~$x$ of~$\glie$, every element~$t$ of~$\Exterior^{n-1}(\glie)$ and every element~$m$ of~$M$.
      It hence sufficies to show that
      \[
        d_{n-1} \circ d_n \circ \sigma_{n-1}^x = 0 \,.
      \]
      for every element~$x$ of~$\glie$.
      We have
      \begin{align*}
        d_{n-1} \circ d_n \circ \sigma_{n-1}^x
        &=
        d_{n-1} \circ (- \sigma_n^x \circ d_{n-1} - \theta_{n-1}^x)
        \\
        &=
        - d_{n-1} \circ \sigma_{n-1}^x \circ d_{n-1}
        - d_{n-1} \circ \theta_{n-1}^x
        \\
        &=
        - ( - \sigma_{n-2}^x \circ d_{n-2} - \theta_{n-2}^x ) \circ d_{n-1}
        - d_{n-1} \circ \theta_{n-1}^x
        \\
        &=
        \sigma_{n-2} \circ d_{n-2} \circ d_{n-1}
        + \theta_{n-2}^x \circ d_{n-1}
        - d_{n-1} \circ \theta_{n-1}^x
        \\
        &=
        \sigma_{n-1} \circ 0
        + 0
        \\
        &=
        0
      \end{align*}
      because~$d_{n-2} \circ d_{n-1} = 0$ by induction hypothesis and because~$\theta_{n-2}^x \circ d_{n-1} = d_{n-1} \circ \theta_{n-1}^x$.
    \item
      We identify~$\Alt^n(\glie, M)$ with~$\Hom_{\kf}( \Exterior^n(\glie), M)$ as explained in \cref{expressing alt with exterior powers}.
      Let~$\omega$ be an element of~$\Hom_{\kf}( \Exterior^n(\glie), M )$.
      We get a map
      \[
        \kappa'
        \colon
        \underbrace{ \glie \times \dotsb \times \glie }_{n+1}
        \to
        M
      \]
      given by
      \begin{align*}
        \kappa'(x_1, \dotsc, x_{n+1})
        \defined
        {}&
        \sum_{1 \leq i < j \leq n+1}
        (-1)^{i+j}
        \omega
        (
          [x_i, x_j] \wedge x_1 \wedge \dotsb \wedge \widehat{x_i} \wedge \dotsb \wedge \widehat{x_j} \wedge \dotsb \wedge x_{n+1}
        )
        \\
        {}&
        +
        \sum_{i=1}^{n+1}
        (-1)^{i+1}
        x_i \act \omega(x_1 \wedge \dotsb \wedge \widehat{x_i} \wedge \dotsb \wedge x_{n+1})
      \end{align*}
      for all elements~$x_1, \dotsc, x_{n+1}$ of~$\glie$.
      This map is multilinear.
      We claim that is it also alternating.
      To show this we assume that~$x_r = x_s$ for some indices~$r$,~$s$ with~$1 \leq r < s \leq n+1$.
      We have
      \begin{align}
        \kappa'(x_1, \dotsc, x_{n+1})
        \defined
        {}&
        \omega
        \Biggl(
          \sum_{1 \leq i < j \leq n+1}
          (-1)^{i+j}
          [x_i, x_j] \wedge x_1 \wedge \dotsb \wedge \widehat{x_i} \wedge \dotsb \wedge \widehat{x_j} \wedge \dotsb \wedge x_{n+1}
        \Biggr)
        \label{first term for cohomology}
        \\
        {}&
        +
        \sum_{i=1}^{n+1}
        (-1)^{i+1}
        x_i \act \omega(x_1 \wedge \dotsb \wedge \widehat{x_i} \wedge \dotsb \wedge x_{n+1})
        \label{second term for cohomology}
      \end{align}
      It follows from \cref{alternating in multiple arguments} that the term~\eqref{first term for cohomology} vanishes.
      For the term~\eqref{second term for cohomology} we note that the summands for~$i \neq r,s$ vanish since~$\omega$ is alternating and~$x_r = x_s$.
      It remains to show that the two summands
      \[
        (-1)^{r+1}
        x_r \act \omega(x_1 \wedge \dotsb \wedge \widehat{x_r} \wedge \dotsb \wedge x_{n+1})
      \]
      and
      \[
        (-1)^{s+1}
        x_s \act \omega(x_1 \wedge \dotsb \wedge \widehat{x_s} \wedge \dotsb \wedge x_{n+1})
      \]
      cancel out.
      This holds because the two simple wedges
      \[
        x_1 \wedge \dotsb \wedge \widehat{x_r} \wedge \dotsb \wedge x_{n+1}
        =
        x_1 \wedge \dotsb \wedge \widehat{x_r} \wedge \dotsb \wedge x_s \wedge \dotsb \wedge x_{n+1}
      \]
      and
      \[
        x_1 \wedge \dotsb \wedge \widehat{x_s} \wedge \dotsb \wedge x_{n+1}
        =
        x_1 \wedge \dotsb \wedge x_r \wedge \dotsb \wedge \widehat{x_s} \wedge \dotsb \wedge x_{n+1}
      \]
      differ only by the sign~$(-1)^{s-r-1}$, because~$x_r = x_s$.

      The map~$\kappa'$ is multilinear and alternating and thus induces a linear map~$\kappa$ from~$\Exterior^{n+1}(\glie)$ to~$M$ by the universal property of the exterior power.
      This map~$\kappa$ is precisely the desired linear map~$d^n(\omega)$.
      We have thus shows that the element~$d^n(\omega)$ of~$\Hom_{\kf}( \Exterior^{n+1}(\glie), M)$ is well-defined.
      The map~$d^n$ is thus well-defined.
      It is also linear.
    \item
      The identity~$d^{n+1} \circ d^n = 0$ holds whenever~$n < 0$ since then~$d^n = 0$.
      In the following we consider the case~$n \geq 0$.

      The dual~$M^*$ is again a representation of~$\glie$ via the action
      \[
        (x \act \varphi)(m)
        =
        - x \act \varphi(m)
      \]
      for all~$x \in \glie$,~$\varphi \in M^*$ and~$m \in M$.
      We may dualize the linear map
      \[
        d_{n+1}
        \colon
        \Exterior^{n+1}(\glie) \otimes M^*
        \to
        \Exterior^n(\glie) \otimes M^*
      \]
      to get the linear map
      \[
        d_{n+1}^*
        \colon
        \Bigl( \Exterior^n(\glie) \otimes M^* \Bigr)^*
        \to
        \Bigl( \Exterior^{n+1}(\glie) \otimes M^* \Bigr)^* \,.
      \]
      We have for every natural number~$k$ the isomorphisms
      \begin{align*}
        \Bigl( \Exterior^k(\glie) \otimes M^* \Bigr)^*
        &=
        \Hom\Bigl( \Exterior^k(\glie) \otimes M^*, \kf \Bigr)
        \\
        &\cong
        \Hom\Bigl( \Exterior^k(\glie), \Hom_{\kf}( M^*, \kf ) \Bigr)
        \\
        &=
        \Hom\Bigl( \Exterior^k(\glie), M^{**} \Bigr)
        \\
        &\cong
        \Alt^k( \glie, M^{**} ) \,.
      \end{align*}
      The natural inclusion
      \[
        \ev
        \colon
        M
        \to
        M^{**} \,,
        \quad
        m
        \mapsto
        \ev_m
      \]
      given by~$\ev_m(\varphi) = \varphi(m)$ for all~$m \in M$ and~$\varphi \in M^*$ induces for every natural number~$k$ an inclusion
      \[
        \Alt^k(\glie, M)
        \to
        \Alt^k(\glie, M^{**}) \,,
        \quad
        \omega
        \mapsto
        {\ev} \circ \omega \,.
      \]
      Overall we may regard the vector space~$\Alt^k(\glie, M)$ as linear subspace of~$( \Exterior^k(\glie) \otimes M^* )^*$ for every natural number~$k$.

      We claim that under these identifications the linear map~$d_{n+1}^*$ restricts to a linear map from~$\Alt^n(\glie, M)$ to~$\Alt^{n+1}(\glie, M)$, and that this restriction is precisely the linear map~$d^n$.

      To see this we make the identification of~$\Alt^k(\glie, \kf)$ with~$( \Exterior^k(\glie) \otimes M^* )^*$ more explicit.
      Let~$\omega$ be an element of~$\Alt^k(\glie, M)$.
      The resulting element~$\omega_1$ of~$\Alt^k(\glie, M^{**})$ is given by
      \[
        \omega_1(x_1, \dotsc, x_k)
        =
        \ev_{\omega(x_1, \dotsc, x_k)}
      \]
      for all~$x_1, \dotsc, x_k \in \glie$.
      The corresponding element~$\omega_2$ of~$\Hom_{\kf}( \Exterior^k(\glie), M^{**} )$ is given by
      \[
        \omega_2(t)
        =
        \ev_{\omega(t)}
      \]
      for every element~$t$ of~$\Exterior^k(\glie)$.
      The corresponding element~$\omega_3$ of~$( \Exterior^k(\glie) \otimes M^* )^*$ is given by
      \[
        \omega_3(t \otimes \varphi)
        =
        \omega_2(t)(\varphi)
        =
        \ev_{\omega(t)}(\varphi)
        =
        \varphi(\omega(t))
      \]
      for all~$t \in \Exterior^k(\glie)$ and~$\varphi \in M^*$.
      Instead of~$\omega_3$ we write~$\overline{\omega}$.
      We have thus seen that the linear inclusion
      \[
        \overline{ (-) }
        \colon
        \Alt^k(\glie, M)
        \to
        \Bigl( \Exterior^k(\glie) \otimes M^* \Bigr)^*
      \]
      is explicitely given by
      \[
        \overline{\omega}(t \otimes \varphi)
        =
        \varphi(\omega(t))
      \]
      for all~$t \in \Exterior^k(\glie)$ and~$\varphi \in M^*$.

      In this notation our claim becomes
      \begin{equation}
        \label{embedding of cochain complex into dual of chain complex}
        d_{n+1}^*( \overline{\omega} )
        =
        \overline{ d^n(\omega) }
        \qquad
        \text{for every~$\omega \in \Alt^n(\glie, M)$.}
      \end{equation}
      Ths identity follows from the calculation
      \begin{align*}
        {}&
        d_{n+1}^*( \overline{\omega} )(x_1 \wedge \dotsb x_{n+1} \otimes \varphi)
        \\
        ={}&
        \overline{\omega}( d_{n+1}( x_1 \wedge \dotsb \wedge x_{n+1} \otimes \varphi ) )
        \\
        ={}&
        \overline{\omega}
        \Biggl(
          \sum_{1 \leq i < j \leq n+1}
          (-1)^{i+j}
          [x_i, x_j] \wedge x_1 \wedge \dotsb \wedge \widehat{x_i} \wedge \dotsb \wedge \widehat{x_j} \wedge \dotsb \wedge x_{n+1} \otimes \varphi
        \\
        {}&
        \phantom{
          \omega
          \Biggl(
        }
          +
          \sum_{i=1}^{n+1}
          (-1)^i
          x_1 \wedge \dotsb \wedge \widehat{x_i} \wedge \dotsb \wedge x_{n+1} \otimes (x_i \act \varphi)
        \Biggr)
        \\
        ={}&
        \sum_{1 \leq i < j \leq n+1}
        (-1)^{i+j}
        \overline{\omega}
        (
          [x_i, x_j] \wedge x_1 \wedge \dotsb \wedge \widehat{x_i} \wedge \dotsb \wedge \widehat{x_j} \wedge \dotsb \wedge x_{n+1} \otimes \varphi
        )
        \\
        {}&
        +
        \sum_{i=1}^{n+1}
        (-1)^i
        \overline{\omega}
        (
          x_1 \wedge \dotsb \wedge \widehat{x_i} \wedge \dotsb \wedge x_{n+1} \otimes (x_i \act \varphi)
        )
        \\
        ={}&
        \sum_{1 \leq i < j \leq n+1}
        (-1)^{i+j}
        \varphi( \omega( [x_i, x_j] \wedge x_1 \wedge \dotsb \wedge \widehat{x_i} \wedge \dotsb \wedge \widehat{x_j} \wedge \dotsb \wedge x_{n+1} ) )
        \\
        {}&
        +
        \sum_{i=1}^{n+1}
        (-1)^i
        (x_i \act \varphi)( \omega( x_1 \wedge \dotsb \wedge \widehat{x_i} \wedge \dotsb \wedge x_{n+1} ) )
        \\
        ={}&
        \sum_{1 \leq i < j \leq n+1}
        (-1)^{i+j}
        \varphi( \omega( [x_i, x_j] \wedge x_1 \wedge \dotsb \wedge \widehat{x_i} \wedge \dotsb \wedge \widehat{x_j} \wedge \dotsb \wedge x_{n+1} ) )
        \\
        {}&
        +
        \sum_{i=1}^{n+1}
        (-1)^{i+1}
        \varphi( x_i \act \omega( x_1 \wedge \dotsb \wedge \widehat{x_i} \wedge \dotsb \wedge x_{n+1} ) )
        \\
        ={}&
        \varphi
        \Biggl(
          \sum_{1 \leq i < j \leq n+1}
          (-1)^{i+j}
          \omega( [x_i, x_j] \wedge x_1 \wedge \dotsb \wedge \widehat{x_i} \wedge \dotsb \wedge \widehat{x_j} \wedge \dotsb \wedge x_{n+1} )
        \\
          {}&
          \phantom{
            \varphi\Biggl(
          }
          +
          \sum_{i=1}^{n+1}
          (-1)^{i+1}
          x_i \act \omega( x_1 \wedge \dotsb \wedge \widehat{x_i} \wedge \dotsb \wedge x_{n+1} )
        \Biggr)
        \\
        ={}&
        \varphi( d^n(\omega)(x_1 \wedge \dotsb \wedge x_n) )
        \\
        ={}&
        \overline{ d^n(\omega) } (x_1 \wedge \dotsb \wedge x_n \otimes \varphi)
      \end{align*}
      for all~$x_1, \dotsc, x_n \in \glie$ and~$\varphi \in M^*$.
      It now follows from the identity~$d_{n+1} \circ d_{n+2} = 0$ by dualizing that~$d_{n+2}^* \circ d_{n+1}^* = 0$, and thus~$d^{n+1} \circ d^n = 0$ by restriction.
    \qedhere
  \end{enumerate}
\end{proof}





\section{Definitions and Basic Properties}


\begin{definition}
  Let~$\glie$ be a Lie~algebra and let~$M$ be a representation of~$\glie$.
  \begin{enumerate}
    \item
      The chain complex~$\Chain_\bullet(\glie, M)$ from \cref{construction of lie algebra homology and cohomology} is the \defemph{Lie algebra chain complex} of~$\glie$ with coefficients in~$M$.
      The homology of this chain complex is the \defemph{Lie algebra homology} of~$\glie$ with coefficients in~$M$, and it is denoted by~$\Homology_\bullet(\glie, M)$.
    \item
      The cochain complex~$\Chain^\bullet(\glie, M)$ from \cref{construction of lie algebra homology and cohomology} is the \defemph{Lie algebra cochain complex} of~$\glie$ with coefficients in~$M$.
      The cohomology of this chain complex is the \defemph{Lie algebra cohomology} of~$\glie$ with coefficients in~$M$, and it is denoted by~$\Homology^\bullet(\glie, M)$.
  \end{enumerate}
\end{definition}


\begin{recall}
  \leavevmode
  \begin{enumerate}
    \item
      A \defemph{homomorphism} of chain complexes from~$X_\bullet$ to~$Y_\bullet$ is a family~$(f_n)_{n \in \Integer}$ of linear maps~$f_n$ from~$X_n$ to~$Y_n$ such that the following diagram commutes.
      \[
        \begin{tikzcd}[sep = large]
          \dotsb
          &
          X_{n-1}
          \arrow{l}
          \arrow{d}[right]{f_{n-1}}
          &
          X_n
          \arrow{l}[above]{d_n}
          \arrow{d}[right]{f_n}
          &
          X_{n+1}
          \arrow{l}[above]{d_{n+1}}
          \arrow{d}[right]{f_{n+1}}
          &
          \dotsb
          \arrow{l}
          \\
          \dotsb
          &
          Y_{n-1}
          \arrow{l}
          &
          Y_n
          \arrow{l}[above]{d_n}
          &
          Y_{n+1}
          \arrow{l}[above]{d_{n+1}}
          &
          \dotsb
          \arrow{l}
        \end{tikzcd}
      \]

      A \defemph{homomorphism} of cochain complexes from~$X^\bullet$ to~$Y^\bullet$ is a family~$(f^n)_{n \in \Integer}$ of linear maps~$f^n$ from~$X^n$ to~$Y^n$ such that the following diagram commutes.
      \[
        \begin{tikzcd}[sep = large]
          \dotsb
          \arrow{r}
          &
          X^{n-1}
          \arrow{d}[right]{f^{n-1}}
          \arrow{r}[above]{d^{n-1}}
          &
          X^n
          \arrow{d}[right]{f^n}
          \arrow{r}[above]{d^n}
          &
          X^{n+1}
          \arrow{d}[right]{f^{n+1}}
          \arrow{r}
          &
          \dotsb
          \\
          \dotsb
          \arrow{r}
          &
          Y^{n-1}
          \arrow{r}[above]{d^{n-1}}
          &
          Y^n
          \arrow{r}[above]{d^n}
          &
          Y^{n+1}
          \arrow{r}
          &
          \dotsb
        \end{tikzcd}
      \]
    \item
      If~$f_\bullet = (f_n)_{n \in \Integer}$ is a homomorphism of chain complexes from~$X_\bullet$ to~$Y_\bullet$ then
      \[
        f_n( \Cycle_n(X_\bullet) )
        \subseteq
        \Cycle_n(Y_\bullet)
        \quad\text{and}\quad
        f_n( \Boundary_n(X_\bullet) )
        \subseteq
        \Boundary_n(Y_\bullet)
      \]
      for every integer~$n$.
      It follows that the homomorphism~$f$ induces a linear map
      \[
        \Homology_n( f_\bullet )
        \colon
        \Homology_n( X_\bullet )
        \to
        \Homology_n( Y_\bullet ) \,,
        \quad
        \class{ x }
        \mapsto
        \class{ f_n(x) }
      \]
      for every integer~$n$.

      If~$f^\bullet = (f^n)_{n \in \Integer}$ is a homomorphism of chain complexes from~$X^\bullet$ to~$Y^\bullet$ then
      \[
        f^n( \Cycle^n(X^\bullet) )
        \subseteq
        \Cycle^n(Y^\bullet)
        \quad\text{and}\quad
        f^n( \Boundary^n(X^\bullet) )
        \subseteq
        \Boundary^n(Y^\bullet)
      \]
      for every integer~$n$.
      It follows that the homomorphism~$f$ induces a linear map
      \[
        \Homology^n( f^\bullet )
        \colon
        \Homology^n( X^\bullet )
        \to
        \Homology^n( Y^\bullet ) \,,
        \quad
        \class{ x }
        \mapsto
        \class{ f^n(x) }
      \]
      for every integer~$n$.
  \end{enumerate}
\end{recall}


\begin{proposition}
  \label{construction of induced homomorphism of (co)chain complexes}
  Let~$\glie$ be a Lie~algebra and let~$M$ and~$N$ be two representations of~$\glie$.
  Let~$f$ be a homomorphism of representations from~$M$ to~$N$.
  \begin{enumerate}
    \item
      For every integer~$n$ let
      \[
        f_n
        \defined
        \begin{cases*}
          \id \otimes f & if~$n \geq 0$, \\
          0             & if~$n < 0$.
        \end{cases*}
      \]
      This family~$(f_n)_{n \in \Integer}$ is a homomorphism of chain complexes from~$\Chain_\bullet(\glie, M)$ to~$\Chain_\bullet(\glie, N)$.
    \item
      For every integer~$n$ let
      \[
        f^n
        \colon
        \Chain^n(\glie, M)
        \to
        \Chain^n(\glie, N) \,,
        \quad
        \omega
        \mapsto
        f \circ \omega
      \]
      if~$n \geq 0$, and~$f^n = 0$ if~$n < 0$.
      This family~$(f^n)_{n \in \Integer}$ is a homomorphism of chain complexes from~$\Chain^\bullet(\glie, M)$ to~$\Chain^\bullet(\glie, N)$.
  \end{enumerate}
\end{proposition}


\begin{proof}
  \leavevmode
  \begin{enumerate}
    \item
      We need to show that
      \[
        f_{n-1} \circ d_n
        =
        d_n \circ f_n
      \]
      for every integer~$n$.
      This identity holds for~$n \leq 0$ because then both sides map into the zero vector space.
      For~$n \geq 1$ we have
      \begin{align*}
        {}&
        f_{n-1}( d_n( x_1 \wedge \dotsb \wedge x_n \otimes m ) )
        \\
        ={}&
        f_{n-1}
        \Biggl(
          \sum_{1 \leq i < j \leq n}
          (-1)^{i + j}
          [x_i, x_j] \wedge x_1 \wedge \dotsb \wedge \widehat{x_i} \wedge \dotsb \wedge \widehat{x_j} \wedge \dotsb \wedge x_n \otimes m
        \\
        {}&
          \phantom{
            f_{n-1}
            \Biggl(
          }
          +
          \sum_{i=1}^n
          (-1)^i
          x_1 \wedge \dotsb \wedge \widehat{x_i} \wedge \dotsb \wedge x_n \otimes (x_i \act m)
        \Biggr)
        \\
        ={}&
        \sum_{1 \leq i < j \leq n}
        (-1)^{i + j}
        [x_i, x_j] \wedge x_1 \wedge \dotsb \wedge \widehat{x_i} \wedge \dotsb \wedge \widehat{x_j} \wedge \dotsb \wedge x_n \otimes f(m)
        \\
        {}&
        +
        \sum_{i=1}^n
        (-1)^i
        x_1 \wedge \dotsb \wedge \widehat{x_i} \wedge \dotsb \wedge x_n \otimes f(x_i \act m)
        \\
        ={}&
        \sum_{1 \leq i < j \leq n}
        (-1)^{i + j}
        [x_i, x_j] \wedge x_1 \wedge \dotsb \wedge \widehat{x_i} \wedge \dotsb \wedge \widehat{x_j} \wedge \dotsb \wedge x_n \otimes f(m)
        \\
        {}&
        +
        \sum_{i=1}^n
        (-1)^i
        x_1 \wedge \dotsb \wedge \widehat{x_i} \wedge \dotsb \wedge x_n \otimes ( x_i \act f(m) )
        \\
        ={}&
        d_n( x_1 \wedge \dotsb \wedge x_n \otimes f(m) )
        \\
        ={}&
        d_n( f_n( x_1 \wedge \dotsb \wedge x_n \otimes m ) )
      \end{align*}
      for all~$x_1, \dotsc, x_n \in \glie$ and~$m \in M$.
    \item
      We need to show that
      \[
        f^{n+1} \circ d^n
        =
        d^n \circ f^n
      \]
      for every integer~$n$.
      This holds for~$n < 0$ because then both sides are the zero map.
      For~$n \geq 0$ we have
      \begin{align*}
        {}&
        f^{n+1}( d^n(\omega) )(x_1, \dotsc, x_{n+1})
        \\
        ={}&
        f( d^n(\omega)(x_1, \dotsc, x_{n+1}) )
        \\
        ={}&
        f
        \Biggl(
          \sum_{1 \leq i < j \leq n+1}
          (-1)^{i+j}
          \omega( [x_i, x_j], x_1, \dotsc, \widehat{x_i}, \dotsc, \widehat{x_j}, \dotsc, x_{n+1} )
        \\
        {}&
        \phantom{
          f
          \Biggl(
        }
          +
          \sum_{i=1}^{n+1}
          (-1)^{i+1}
          x_i \act \omega(x_1, \dotsc, \widehat{x_i}, \dotsc, x_{n+1})
        \Biggr)
        \\
        ={}&
        \sum_{1 \leq i < j \leq n+1}
        (-1)^{i+j}
        f( \omega( [x_i, x_j], x_1, \dotsc, \widehat{x_i}, \dotsc, \widehat{x_j}, \dotsc, x_{n+1} ) )
        \\
        {}&
        +
        \sum_{i=1}^{n+1}
        (-1)^{i+1}
        x_i \act f( \omega(x_1, \dotsc, \widehat{x_i}, \dotsc, x_{n+1}) )
        \\
        ={}&
        \sum_{1 \leq i < j \leq n+1}
        (-1)^{i+j}
        f^n(\omega)( [x_i, x_j], x_1, \dotsc, \widehat{x_i}, \dotsc, \widehat{x_j}, \dotsc, x_{n+1} )
        \\
        {}&
        +
        \sum_{i=1}^{n+1}
        (-1)^{i+1}
        x_i \act f^n(\omega)(x_1, \dotsc, \widehat{x_i}, \dotsc, x_{n+1})
        \\
        ={}&
        d^n( f^n(\omega) )(x_1, \dotsc, x_{n+1})
      \end{align*}
      for all~$\omega \in \Chain^n(\glie, M)$ and~$x_1, \dotsc, x_n \in \glie$.
    \qedhere
  \end{enumerate}
\end{proof}


\begin{definition}
  Let~$\glie$ be a Lie~algebra and let~$M$ and~$N$ be two representations of~$\glie$.
  Let~$f$ be a homomorphism of representations from~$M$ to~$N$.
  \begin{enumerate}
    \item
      The induced homomorphism of chain complexes from~$\Chain_\bullet(\glie, M)$ to~$\Chain_\bullet(\glie, N)$ from \cref{construction of induced homomorphism of (co)chain complexes} is denoted by~$\Chain_\bullet(f)$.
      The homomorphism from~$\Homology_\bullet(\glie, M)$ to~$\Homology_\bullet(\glie, N)$ induced by~$\Chain_\bullet(f)$ is denoted by~$\Homology_\bullet(f)$.
    \item
      The induced homomorphism of cochain complexes from~$\Chain^\bullet(\glie, M)$ to~$\Chain^\bullet(\glie, N)$ from \cref{construction of induced homomorphism of (co)chain complexes} is denoted by~$\Chain^\bullet(f)$.
      The homomorphism from~$\Homology^\bullet(\glie, M)$ to~$\Homology^\bullet(\glie, N)$ induced by~$\Chain^\bullet(f)$ is denoted by~$\Homology^\bullet(f)$.
  \end{enumerate}
\end{definition}


\begin{remark}
  We have for every integer~$n$ functors
  \begin{alignat*}{2}
    \Chain_n(\glie, -)
    &\colon
    \cRep{\glie}
    \to
    \cVect{\kf} \,,
    &
    \qquad
    \Homology_n(\glie, -)
    &\colon
    \cRep{\glie}
    \to
    \cVect{\kf} \,,
  \intertext{as well as functors}
    \Chain^n(\glie, -)
    &\colon
    \cRep{\glie}
    \to
    \cVect{\kf} \,,
    &
    \qquad
    \Homology^n(\glie, -)
    &\colon
    \cRep{\glie}
    \to
    \cVect{\kf} \,.
  \end{alignat*}
\end{remark}


\begin{recall}
  \leavevmode
  \begin{enumerate}
    \item
      Let~$X_\bullet$ be a chain complex.
      A \defemph{subcomplex} of~$X_\bullet$ is a family~$(S_n)_{n \in \Integer}$ of linear subspaces~$S_n$ of~$X_n$ such that
      \[
        d_n( S_n )
        \subseteq
        S_{n-1}
      \]
      for every integer~$n$.
      If~$(S_n)_{n \in \Integer}$ is a subcomplex then it becomes a chain complex in its own right by restricting the differentials of~$X_\bullet$. 

      Let~$X^\bullet$ be a chain complex.
      A \defemph{subcomplex} of~$X^\bullet$ is a family~$(S^n)_{n \in \Integer}$ of linear subspaces~$S^n$ of~$X^n$ such that
      \[
        d^n( S^n )
        \subseteq
        S^{n+1}
      \]
      for every integer~$n$.
      If~$(S^n)_{n \in \Integer}$ is a subcomplex then it becomos a cochain complex in its own right by restricting the differentials of~$X^\bullet$. 
    \item
      Let~$X_\bullet$ and~$Y_\bullet$ be two chain complexes and let~$f_\bullet$ be a homomorphism of chain complexes from~$X_\bullet$ to~$Y_\bullet$.
      Then
      \[
        d_n( \im(f_n) )
        \subseteq
        \im( f_{n-1} )
      \]
      for every integer~$n$.
      The sequence~$( \im(f_n) )_{n \in \Integer}$ is therefore a subcomplex of~$Y_\bullet$.
      This subcomplex is the \defemph{image} of~$f_\bullet$, and it is denoted by~$\im( f_\bullet )$.
      Similarly
      \[
        d_n( \ker(f_n) )
        \subseteq
        \ker( f_{n-1} )
      \]
      for every integer~$n$.
      The sequence~$( \ker(f_n) )_{n \in \Integer}$ is therefore a subcomplex of~$X_\bullet$.
      This subcomplex is the \defemph{kernel} of~$f_\bullet$, and it is denoted by~$\ker( f_\bullet )$.

      Let~$X^\bullet$ and~$Y^\bullet$ be two chain complexes and let~$f^\bullet$ be a homomorphism of chain complexes from~$X^\bullet$ to~$Y^\bullet$.
      Then
      \[
        d^n( \im(f^n) )
        \subseteq
        \im( f^{n+1} )
      \]
      for every integer~$n$.
      The sequence~$( \im(f^n) )_{n \in \Integer}$ is therefore a subcomplex of~$Y^\bullet$.
      This subcomplex is the \defemph{image} of~$f^\bullet$, and it is denoted by~$\im( f^\bullet )$.
      Similarly
      \[
        d^n( \ker(f^n) )
        \subseteq
        \ker( f^{n+1} )
      \]
      for every integer~$n$.
      The sequence~$( \ker(f^n) )_{n \in \Integer}$ is therefore a subcomplex of~$X^\bullet$.
      This subcomplex is the \defemph{kernel} of~$f^\bullet$, and it is denoted by~$\ker( f^\bullet )$.
    \item
      A sequence
      \[
        \dotsb
        \to
        X_\bullet
        \xto{f_\bullet}
        Y_\bullet
        \xto{g_\bullet}
        Z_\bullet
        \to
        \dotsb
      \]
      of of chain complexes is \defemph{exact} at~$Y_\bullet$ if the image of~$f_\bullet$ equals the kernel of~$g_\bullet$.
      This happens if and only if the sequence of vector spaces
      \[
        \dotsb
        \to
        X_n
        \xto{f_n}
        Y_n
        \xto{g_n}
        Z_n
        \to
        \dotsb
      \]
      is exact at~$Y_n$ for every integer~$n$.
      An exact sequence of the form
      \[
        0
        \to
        X_\bullet
        \to
        Y_\bullet
        \to
        Z_\bullet
        \to
        0
      \]
      is a \defemph{short exact sequence} of chain complexes.%
      \footnote{
        Here~$0$ denotes the zero chain complex, whose components and entries all are zero.
      }
      This means precisely that for every integer~$n$ the sequence of vector spaces
      \[
        0
        \to
        X_n
        \to
        Y_n
        \to
        Z_n
        \to
        0
      \]
      is short exact.

      A sequence
      \[
        \dotsb
        \to
        X^\bullet
        \xto{f^\bullet}
        Y^\bullet
        \xto{g^\bullet}
        Z^\bullet
        \to
        \dotsb
      \]
      of of chain complexes is \defemph{exact} at~$Y^\bullet$ if the image of~$f^\bullet$ equals the kernel of~$g^\bullet$.
      This happens if and only if the sequence of vector spaces
      \[
        \dotsb
        \to
        X^n
        \xto{f^n}
        Y^n
        \xto{g^n}
        Z^n
        \to
        \dotsb
      \]
      is exact at~$Y^n$ for every integer~$n$.
      An exact sequence of the form
      \[
        0
        \to
        X^\bullet
        \to
        Y^\bullet
        \to
        Z^\bullet
        \to
        0
      \]
      is a \defemph{short exact sequence} of cochain complexes.
      This means precisely that for every integer~$n$ the sequence of vector spaces
      \[
        0
        \to
        X^n
        \to
        Y^n
        \to
        Z^n
        \to
        0
      \]
      is short exact.
    \item
      Let
      \[
        0
        \to
        X_\bullet
        \xto{f_\bullet}
        Y_\bullet
        \xto{g_\bullet}
        Z_\bullet
        \to
        0
      \]
      be a short exact sequence of chain complexes.
      Then the homologies of the chain complexes~$X_\bullet$,~$Y_\bullet$,~$Z_\bullet$ fit into a long exact sequence
      \[
        \begin{tikzcd}[column sep = large]
          {}
          &
          \dotsb
          \arrow{r}
          \arrow[d, phantom, ""{coordinate, name=Y}]
          &
          \Homology_{n+1}( Z_\bullet )
          \arrow[ dll,
            rounded corners,
            to path={ -- ([xshift=2ex]\tikztostart.east)
                      |- (Y)
                      -| ([xshift=-2ex]\tikztotarget.west)
                      -- (\tikztotarget) }
          ]
          \\[1em]
          \Homology_n( X_\bullet )
          \arrow{r}[above]{ \Homology_n( f_\bullet ) }
          &
          \Homology_n( Y_\bullet )
          \arrow{r}[above]{ \Homology_n( g_\bullet ) }
          \arrow[d, phantom, ""{coordinate, name=Z}]
          &
          \Homology_n( Z_\bullet )
          \arrow[ dll,
            rounded corners,
            to path={ -- ([xshift=2ex]\tikztostart.east)
                      |- (Z)
                      -| ([xshift=-2ex]\tikztotarget.west)
                      -- (\tikztotarget) }
          ]
          \\[1em]
          \Homology_{n-1}( X_\bullet )
          \arrow{r}
          &
          \dotsb
          &
          {}
        \end{tikzcd}
      \]

      Let
      \[
        0
        \to
        X^\bullet
        \xto{f^\bullet}
        Y^\bullet
        \xto{g^\bullet}
        Z^\bullet
        \to
        0
      \]
      be a short exact sequence of cochain complexes.
      Then the cohomologies of the cochain complexes~$X^\bullet$,~$Y^\bullet$,~$Z^\bullet$ fit into a long exact sequence
      \[
        \begin{tikzcd}[column sep = large]
          {}
          &
          \dotsb
          \arrow{r}
          \arrow[d, phantom, ""{coordinate, name=Y}]
          &
          \Homology^{n-1}( Z^\bullet )
          \arrow[ dll,
            rounded corners,
            to path={ -- ([xshift=2ex]\tikztostart.east)
                      |- (Y)
                      -| ([xshift=-2ex]\tikztotarget.west)
                      -- (\tikztotarget) }
          ]
          \\[1em]
          \Homology^n( X^\bullet )
          \arrow{r}[above]{ \Homology^n( f^\bullet ) }
          &
          \Homology^n( Y^\bullet )
          \arrow{r}[above]{ \Homology^n( g^\bullet ) }
          \arrow[d, phantom, ""{coordinate, name=Z}]
          &
          \Homology^n( Z^\bullet )
          \arrow[ dll,
            rounded corners,
            to path={ -- ([xshift=2ex]\tikztostart.east)
                      |- (Z)
                      -| ([xshift=-2ex]\tikztotarget.west)
                      -- (\tikztotarget) }
          ]
          \\[1em]
          \Homology^{n-1}( X^\bullet )
          \arrow{r}
          &
          \dotsb
          &
          {}
        \end{tikzcd}
      \]
  \end{enumerate}
\end{recall}


\begin{recall}
  \label{exactness for vector spaces}
  Let
  \[
    0
    \to
    U
    \xto{f}
    V
    \xto{g}
    W
    \to
    0
  \]
  be a short exact sequence of vector spaces and let~$E$ be another vector space.
  \begin{enumerate}
    \item
      The sequence
      \[
        0
        \to
        E \otimes U
        \xto{\id_E \otimes f}
        E \otimes V
        \xto{\id_E \otimes g}
        E \otimes W
        \to
        0
      \]
      is again short exact.
    \item
      The sequence
      \[
        0
        \to
        \Hom_{\kf}(E, U)
        \xto{f_*}
        \Hom_{\kf}(E, V)
        \xto{g_*}
        \Hom_{\kf}(E, W)
        \to
        0
      \]
      is again short exact.
  \end{enumerate}
\end{recall}


\begin{proposition}
  Let~$\glie$ be a Lie~algebra and let
  \[
    0
    \to
    N
    \xto{f}
    M
    \xto{g}
    P
    \to
    0
  \]
  be a short exact sequence of representations of~$\glie$.
  \begin{enumerate}
    \item
      The resulting sequence
      \begin{equation}
        \label{short exact sequence for lie algebra chain complex}
        0
        \to
        \Chain_\bullet(\glie, N)
        \xto{ \Chain_\bullet(f) }
        \Chain_\bullet(\glie, M)
        \xto{ \Chain_\bullet(f) }
        \Chain_\bullet(\glie, P)
        \to
        0
      \end{equation}
      of chain complexes is again short exact.
    \item
      The Lie algebra homologies of~$\glie$ with coefficients in~$N$,~$M$,~$P$ fit into a long exact sequence
      \[
        \begin{tikzcd}[column sep = large]
          {}
          &
          \dotsb
          \arrow{r}
          \arrow[d, phantom, ""{coordinate, name=Y}]
          &
          \Homology_{n+1}(\glie, P)
          \arrow[ dll,
            rounded corners,
            to path={ -- ([xshift=2ex]\tikztostart.east)
                      |- (Y)
                      -| ([xshift=-2ex]\tikztotarget.west)
                      -- (\tikztotarget) }
          ]
          \\[1em]
          \Homology_n(\glie, N)
          \arrow{r}[above]{ \Homology_n(f) }
          &
          \Homology_n(\glie, M)
          \arrow{r}[above]{ \Homology_n(g) }
          \arrow[d, phantom, ""{coordinate, name=Z}]
          &
          \Homology_n(\glie, P)
          \arrow[ dll,
            rounded corners,
            to path={ -- ([xshift=2ex]\tikztostart.east)
                      |- (Z)
                      -| ([xshift=-2ex]\tikztotarget.west)
                      -- (\tikztotarget) }
          ]
          \\[1em]
          \Homology_{n-1}(\glie, N)
          \arrow{r}
          &
          \dotsb
          &
          {}
        \end{tikzcd}
      \]
    \item
      The resulting sequence
      \begin{equation}
        \label{short exact sequence for lie algebra cochain complex}
        0
        \to
        \Chain^\bullet(\glie, N)
        \xto{ \Chain^\bullet(f) }
        \Chain^\bullet(\glie, M)
        \xto{ \Chain^\bullet(f) }
        \Chain^\bullet(\glie, P)
        \to
        0
      \end{equation}
      of cochain complexes is again short exact.
    \item
      The Lie algebra cohomologies of~$\glie$ with coefficients in~$N$,~$M$,~$P$ fit into a long exact sequence
      \[
        \begin{tikzcd}[column sep = large]
          {}
          &
          \dotsb
          \arrow{r}
          \arrow[d, phantom, ""{coordinate, name=Y}]
          &
          \Homology^{n-1}(\glie, P)
          \arrow[ dll,
            rounded corners,
            to path={ -- ([xshift=2ex]\tikztostart.east)
                      |- (Y)
                      -| ([xshift=-2ex]\tikztotarget.west)
                      -- (\tikztotarget) }
          ]
          \\[1em]
          \Homology^n(\glie, N)
          \arrow{r}[above]{ \Homology^n(f) }
          &
          \Homology^n(\glie, M)
          \arrow{r}[above]{ \Homology^n(g) }
          \arrow[d, phantom, ""{coordinate, name=Z}]
          &
          \Homology^n(\glie, P)
          \arrow[ dll,
            rounded corners,
            to path={ -- ([xshift=2ex]\tikztostart.east)
                      |- (Z)
                      -| ([xshift=-2ex]\tikztotarget.west)
                      -- (\tikztotarget) }
          ]
          \\[1em]
          \Homology^{n+1}(\glie, N)
          \arrow{r}
          &
          \dotsb
          &
          {}
        \end{tikzcd}
      \]
  \end{enumerate}
\end{proposition}


\begin{proof}
  \leavevmode
  \begin{enumerate}
    \item
      The exactness of the given sequence can be checked degreewise, i.e. it sufficies to show that for every integer~$n$ the sequence
      \[
        0
        \to
        \Chain_n(\glie, N)
        \xto{\Chain_n(f)}
        \Chain_n(\glie, M)
        \xto{\Chain_n(g)}
        \Chain_n(\glie, P)
        \to
        0
      \]
      is exact.
      If~$n$ is negative then this is the zero short exact sequence.
      For~$n \geq 0$ this is the sequence
      \[
        0
        \to
        \Exterior^n(\glie) \otimes N
        \xto{\id \otimes f}
        \Exterior^n(\glie) \otimes M
        \xto{\id \otimes g}
        \Exterior^n(\glie) \otimes P
        \to
        0 \,.
      \]
      The exactness of this sequence follows from \cref{exactness for vector spaces}.
    \item
      The short exact sequence of chain complexes~\eqref{short exact sequence for lie algebra chain complex} gives a long exact sequence in homology.
    \item
      The exactness of the given sequence can be checked degreewise, i.e. it sufficies to show that for every integer~$n$ the sequence
      \[
        0
        \to
        \Chain^n(\glie, N)
        \xto{\Chain^n(f)}
        \Chain^n(\glie, M)
        \xto{\Chain^n(g)}
        \Chain^n(\glie, P)
        \to
        0
      \]
      is exact.
      If~$n$ is negative then this is the zero short exact sequence.
      For~$n \geq 0$ this is the sequence
      \[
        0
        \to
        \Alt^n(\glie, N)
        \xto{f_*}
        \Alt^n(\glie, M)
        \xto{g_*}
        \Alt^n(\glie, P)
        \to
        0 \,.
      \]
      Under the natural isomorphism~$\Alt^n(\glie, -) \cong \Hom_{\kf}( \Exterior^n(\glie), -)$ this sequence becomes the sequence
      \[
        0
        \to
        \Hom\Bigl( \Exterior^n(\glie), N \Bigr)
        \xto{f_*}
        \Hom\Bigl( \Exterior^n(\glie), N \Bigr)
        \xto{g_*}
        \Hom\Bigl( \Exterior^n(\glie), N \Bigr)
        \to
        0 \,.
      \]
      The exactness of this sequence follows from \cref{exactness for vector spaces}.
    \item
      The short exact sequence of cochain complexes~\eqref{short exact sequence for lie algebra cochain complex} gives a long exact sequence in cohomology.
    \qedhere
  \end{enumerate}
\end{proof}





\section{Computation in Small Degrees}


\begin{fluff}
  We will now compute Lie algebra cohomology and Lie algebra homology in small degrees.
\end{fluff}



\subsection{Cohomology and Homology in Negative Degrees}

\begin{fluff}
  Let~$\glie$ be a Lie~algebra and let~$M$ be a representation of~$\glie$.
  The chain complex~$\Chain_\bullet(\glie, M)$ vanishes in negative degrees, and the same goes for the cochain complex~$\Chain^\bullet(\glie, M)$.
  It follows that both the Lie algebra homology~$\Homology_\bullet(\glie, M)$ and the Lie algebra cohomology~$\Homology^\bullet(\glie, M)$ vanish in negative degrees.
\end{fluff}



\subsection{Zeroth Cohomology}

\begin{fluff}
  \label{zeroth cohomology}
  Let~$\glie$ be a Lie~algebra and let~$M$ be a representation of~$\glie$.
  Then
  \[
    \Chain^0(\glie, M)
    =
    \Hom\Bigl( \Exterior^0(\glie), M \Bigr)
    \cong
    \Hom_{\kf}( \kf, M )
    \cong
    M
  \]
  and
  \[
    \Chain^1(\glie, M)
    =
    \Hom\Bigl( \Exterior^1(\glie), M \Bigr)
    \cong
    \Hom_{\kf}( \glie, M ) \,.
  \]
  The differential~$d^1$ is under these identifications given by the linear map
  \[
    M
    \to
    \Hom_{\kf}(\glie, M) \,,
    \quad
    m
    \mapsto
    ( x \mapsto x \act m ) \,.
  \]
  It follows that
  \[
    \Cycle^0(\glie, M)
    =
    \ker(d^1)
    \cong
    \{
      m \in M
    \suchthat
      x \cdot m
    \}
    =
    M^{\glie} \,.
  \]
  We also have~$\Boundary^0(\glie, M) = 0$ (because~$\Chain^{-1}(\glie, M) = 0$) and thus
  \[
    \Homology^0(\glie, M)
    =
    \Cycle^0(\glie, M) / { \Boundary^0(\glie, M) }
    =
    \Cycle^0(\glie, M) / 0
    \cong
    \Cycle^0(\glie, M)
    \cong
    M^{\glie} \,.
  \]
  This isomorphism is moreover natural in~$M$, whence
  \[
    \Homology^0(\glie, -)
    \cong
    (-)^{\glie} \,.
  \]
\end{fluff}



\subsection{Zeroth Homology}

\begin{definition}
  Let~$\glie$ be a Lie~algebra and let~$M$ be a representation of~$\glie$.
  The quotient representation
  \[
    M / \glie M
  \]
  is the representation of~\defemph{coinvariants} of~$M$.
  It is denoted by~$M_{\glie}$.
\end{definition}


\begin{proposition}[Functoriality of Coinvariants]
  \label{functoriality of coinvariants}
  Let~$\glie$ be a Lie~algebra and let~$M$ and~$N$ be two representations of~$\glie$.
  Let~$f$ be a homomorphism of representations from~$M$ to~$N$.
  Then~$f$ induces a homomorphism of representations
  \[
    M_{\glie}
    \to
    N_{\glie} \,,
    \quad
    \class{m}
    \mapsto
    \class{f(m)} \,.
  \]
\end{proposition}


\begin{proof}
  This holds because the homomorphisms~$f$ maps the subrepresentation~$\glie M$ of~$M$ into the subrepresentation~$\glie M$ of~$N$.
\end{proof}


\begin{remark}
  Let~$\glie$ be a Lie~algebra.
  \begin{enumerate}
    \item
      The representation~$M_{\glie}$ is the largest quotient representation of~$M$ on which~$\glie$ acts trivially.
    \item
      We can extend our discussion from \cref{invariants are right adjoint} as follows.

      We have a functor~$(-)_{\glie}$ from~$\cRep{\glie}$ to~$\Tcat$.
      This functor assigns to each~\representation{$\glie$}~$M$ its space of coinvariants~$M_{\glie}$.
      To each homomorphism of representations~$f$ from~$M$ to~$N$ it assigns the induced homomorphism of representations from~$M_{\glie}$ and~$N_{\glie}$ (as explained in \cref{functoriality of coinvariants}).

      If~$M$ is any~\representation{$\glie$} and~$N$ is any trivial~\representation{$\glie$} then the quotient map~$p$ from~$M$ to~$M_{\glie}$ induces an isomorphism of vector spaces
      \[
        p^*
        \colon
        \Hom_{\glie}(M_{\glie}, N)
        \to
        \Hom_{\glie}(M, N) \,,
        \quad
        f
        \mapsto
        f \circ p \,.
      \]
      This isomorphism is natural in both~$M$ and~$N$.
      The functor~$(-)_{\glie}$ is therefore left adjoint to the inclusion functor~$T$.
      We have overall the adjunctions
      \[
        (-)_{\glie}
        \dashv
        T
        \dashv
        (-)^{\glie} \,.
      \]
    \item
      The category~$\mathcal{T}$ is isomorphic to~$\cRep{\kf}$.
      We will therefore often regard~$(-)^{\glie}$ and~$(-)_{\glie}$ as functors from~$\cRep{\glie}$ to~$\cVect{\kf}$.
  \end{enumerate}
\end{remark}


\begin{fluff}
  Let~$\glie$ be a Lie~algebra and let~$M$ be a representation of~$\glie$.
  We have
  \[
    \Chain_0(\glie, M)
    =
    \Exterior^0(\glie) \otimes M
    \cong
    \kf \otimes M
    \cong
    M
  \]
  and
  \[
    \Chain_1(\glie, M)
    =
    \Exterior^1(\glie) \otimes M
    \cong
    \glie \otimes M \,.
  \]
  Under these isomorphisms the differential~$d_1$ corresponds to the linear map
  \[
    \glie \otimes M
    \to
    M \,,
    \quad
    x \otimes m
    \mapsto
    - x \act m \,.
  \]
  It follows that
  \[
    \Boundary_0(\glie, M)
    =
    \im(d_1)
    \cong
    \glie M
  \]
  under the above identification of~$\Chain_0(\glie, M)$ with~$M$.
  We also have
  \[
    \Cycle_0(\glie, M)
    =
    \Chain_0(\glie, M)
  \]
  because~$d_{-1} = 0$.
  Under the above identification of~$\Chain_0(\glie, M)$ with~$M$ we hence find that~$\Cycle_0(\glie, M)$ corresponds to~$M$.
  It follows that
  \[
    \Homology_0(\glie, M)
    =
    \Cycle_0(\glie, M) / { \Boundary_0(\glie, M) }
    \cong
    M / \glie M
    =
    M_{\glie} \,.
  \]
  This isomorphism is moreover natural in~$M$, whence
  \[
    \Homology_0(\glie, -)
    \cong
    (-)_{\glie} \,.
  \]
\end{fluff}



\subsection{First Cohomology}

\begin{definition}
  Let~$\glie$ be a Lie~algebra and let~$M$ be a representation of~$\glie$.
  A map~$\delta$ from~$\glie$ to~$M$ is a \defemph{derivation} if it is linear and satisfies the condition
  \[
    \delta( [x,y] )
    =
    x \act \delta(y) - y \act \delta(x)
    \qquad
    \text{for all~$x, y \in \glie$.}
  \]
  The set of derivations of~$M$ is denoted by~$\Der(\glie, M)$.
\end{definition}


\begin{proposition}
  \label{inner derivations of representations}
  Let~$\glie$ be a Lie~algebra and let~$M$ be a representation of~$\glie$.
  For every element~$m$ of~$M$ the map
  \[
    \delta_m
    \colon
    \glie
    \to
    M \,,
    \quad
    x
    \mapsto
    x \act m
  \]
  is a derivation of~$M$.
\end{proposition}


\begin{proof}
  We compute that
  \[
    \delta_m([x,y])
    =
    [x,y] \act m
    =
    x \act y \act m - y \act x \act m
    =
    x \act \delta_m(y) - y \act \delta_m(x)
  \]
  for all~$x, y \in \glie$.
\end{proof}


\begin{definition}
  Let~$\glie$ be a Lie~algebra and let~$M$ be a representation of~$\glie$.
  A derivation~$\delta$ of~$M$ is \defemph{inner} if there exists an element~$m$ of~$M$ with
  \[
    \delta(x) = x \act m
    \qquad
    \text{for all~$x \in \glie$.}
  \]
  The set of inner dervations of~$M$ is denoted by~$\InnDer(\glie, M)$.
\end{definition}


\begin{remark}
  Let~$\glie$ be a Lie~algebra.
  For the adjoint representation of~$\glie$ the above notions of derivations and inner derivations coincide with the notions from \cref{definition of derivations} and \cref{definition of inner derivations}.
\end{remark}


\begin{proposition}
  Let~$\glie$ be Lie~algebra and let~$M$ be a representation of~$\glie$.
  \begin{enumerate}
    \item
      The set~$\Der(\glie, M)$ is a linear subspace of~$\Hom_{\kf}(\glie, M)$.
    \item
      The set~$\InnDer(\glie, M)$ is a linear subspace of~$\Der(\glie, M)$.
    \qed
  \end{enumerate}
\end{proposition}


\begin{definition}
  Let~$\glie$ be a Lie~algebra and let~$M$ be a representation of~$\glie$.
  The quotient vector space~$\OutDer(\glie, M) \defined \Der(\glie, M) / { \InnDer(\glie, M) }$ is the space of \defemph{outer derivations} of~$M$.
\end{definition}


\begin{proposition}
  \label{functiorality of outer derivations}
  Let~$\glie$ be a Lie~algebra and let~$M$ and~$N$ be two representations of~$\glie$.
  Let~$f$ be a homomorphism of representations from~$M$ to~$N$.
  \begin{enumerate}
    \item
      The linear map~$f_* \colon \Der(\glie, M) \to \Der(\glie, N)$ given by~$\delta \mapsto f \circ \delta$ is a well-defined.
    \item
      The map~$f_*$ maps the inner derivations of~$M$ to inner derivations of~$N$.
    \item
      The map~$f_*$ induces a linear map from~$\OutDer(\glie, M)$ to~$\OutDer(\glie, N)$ given by~$\class{\delta} \mapsto \class{f \circ \delta}$.
  \end{enumerate}
\end{proposition}


\begin{proof}
  \leavevmode
  \begin{enumerate}
    \item
      Let~$\delta$ be a derivation of~$M$.
      Then
      \begin{align*}
        (f \circ \delta)([x,y])
        &=
        f( \delta([x,y]) )
        \\
        &=
        f( x \act \delta(y) - y \act \delta(x) )
        \\
        &=
        f( x \act \delta(y) ) - f( y \act \delta(x) )
        \\
        &=
        x \act f( \delta(y) ) - y \act f( \delta(x) )
        \\
        &=
        x \act (f \circ \delta)(y) - y \act (f \circ \delta)(x)
      \end{align*}
      for all~$x, y \in \glie$.
      This shows that the composite~$f \circ \delta$ is again a derivation.
    \item
      For every element~$m$ of~$M$ let~$\delta_m$ denote the resulting inner derivation of~$M$.
      Then
      \[
        f_*(\delta_m)(x)
        =
        (f \circ \delta_m)(x)
        =
        f( \delta_m(x) )
        =
        f( x \act m )
        =
        x \act f(m)
        =
        \delta_{f(m)}(x)
      \]
      for all~$x \in \glie$ and thus~$f_*( \delta_m ) = \delta_{f(m)}$.
    \item
      The linear map~$f_*$ from~$\Der(\glie, M)$ to~$\Der(\glie, N)$ maps the linear subspace~$\InnDer(\glie, M) )$ of~$\Der(\glie, M)$ into the linear subspace~$\InnDer(\glie, N)$ of~$\Der(\glie, N)$.
      It thus induces a well-defined linear map from~$\Der(\glie, M) / {\InnDer(\glie, M)}$ to~$\Der(\glie, N) / {\InnDer(\glie, N)}$ which is given by~$\class{\delta} \mapsto \class{f_*(\delta)}$.
      This is the desired induced map.
    \qedhere
  \end{enumerate}
\end{proof}


\begin{remark}
  It follows with the help of~\cref{functiorality of outer derivations} that we have functors
  \[
    \Der(\glie, -),
    \InnDer(\glie, -),
    \OutDer(\glie, -)
    \colon
    \cRep{\glie}
    \to
    \cVect{\kf} \,.
  \]
\end{remark}


\begin{fluff}
  Let~$\glie$ be a Lie~algebra and let~$M$ be a representation of~$\glie$.
  We will now compute~$\Homology^1(\glie, M)$.
  
  As in \cref{zeroth cohomology} we will identify~$\Chain^0(\glie, M)$ with~$M$ and~$\Chain^1(\glie, M)$ with~$\Hom_{\kf}(\glie, M)$.
  The differential~$d^0$ is under these identifications given by the linear map
  \[
    d^0
    \colon
    M
    \to
    \Hom_{\kf}(\glie, M) \,,
    \quad
    d^1(m)(x)
    =
    x \act m \,.
  \]
  The differential~$d^1$ is under these identifications the linear map
  \[
    d^1
    \colon
    \Hom_{\kf}(\glie, M)
    \to
    \Alt^2(\glie, M) \,,
  \]
  given by
  \[
    d^1(\varphi)(x_1, x_2)
    =
    - \varphi( [x_1, x_2] )
    - x_1 \act \varphi(x_2)
    + x_2 \act \varphi(x_1)
  \]
  for all~$\varphi \in \Hom_{\kf}(\glie, M)$ and~$x_1, x_2 \in \glie$.

  We see from these explicit descriptions that an element~$\delta$ of~$\Hom_{\kf}(\glie, M)$ is a one-cocoycle if and only if is satisfies the identity
  \[
    \delta( [x_1, x_2] )
    =
    x_1 \act \delta(x_2) - x_2 \act \delta(x_1) \,,
  \]
  for all~$x_1, x_2 \in \glie$, i.e. if and only if~$\delta$ is a derivation.
  We also see that an element~$\delta$ of~$\Hom_{\kf}(\glie, M)$ is a one-coboundary if and only if there exists an element~$m$ of~$M$ such that
  \[
    \delta(x)
    =
    x \act m \,,
  \]
  for all~$x \in \glie$, i.e. if and only if~$\delta$ is an inner derivation.
  
  We see from these computations that
  \[
    \Homology^1(\glie, M)
    =
    \Cycle^1(\glie, M) / {\Boundary^1(\glie, M)}
    \cong
    \Der(\glie, M) / {\InnDer(\glie, M)}
    =
    \OutDer(\glie, M) \,.
  \]
  The above isomorphism~$\Chain^1(\glie, M) \cong \Hom_{\kf}(\glie, M)$ is natural in~$M$, and so the induced isomorphism~$\Homology^1(\glie, M) \cong \OutDer(\glie, M)$ is again natural in~$M$.
  We have thus shown that
  \[
    \Homology^1(\glie, -)
    \cong
    \OutDer(\glie, -) \,.
  \]
\end{fluff}




\subsection{First Homology}

\begin{fluff}
  Let~$\glie$ be a Lie~algebra.
  We will compute~$\Homology_1(\glie, \kf)$, where we denote by~$\kf$ the trivial representation of~$\glie$.

  We have
  \[
    \Chain_n(\glie, \kf)
    =
    \Exterior^n(\glie) \otimes \kf
    \cong
    \Exterior^n(\glie)
  \]
  for every natural number~$n$.
  Under these identifications the differential~$d_n$ for~$n \geq 1$ is given by
  \[
    d_n(x_1 \wedge \dotsb \wedge x_n)
    =
    \sum_{1 \leq i < j \leq n}
    (-1)^{i+j}
    [x_i, x_j] \wedge x_1 \wedge \dotsb \wedge \widehat{x_i} \wedge \dotsb \wedge \widehat{x_j} \wedge \dotsb \wedge x_n
  \]
  for all~$x_1, \dotsc, x_n \in \glie$.

  For the computation of~$\Homology_1(\glie, \kf)$ we also identify~$\Exterior^0(\glie)$ with~$\kf$ and we identify~$\Exterior^1(\glie)$ with~$\glie$.
  We have thus identified the chain complex~$\Chain_\bullet(\glie, \kf)$ with a chain complex of the form
  \[
    \dotsb
    \to
    \Exterior^2(\glie)
    \xto{- [-,-]}
    \glie
    \xto{0}
    \kf
    \to
    0
    \to
    0
    \to
    \dotsb \,,
  \]
  where the torm~$\glie$ sits in degree~$1$.
  We now see that
  \[
    \Homology_1(\glie, \kf)
    =
    \Cycle_1(\glie, \kf) / { \Boundary_1(\glie, \kf) }
    \cong
    \glie / [\glie, \glie]
    =
    \glie^{\ab} \,.
  \]
\end{fluff}


\begin{recall}
  If~$Y$ is a vector space then we can apply the functor~$(-) \otimes Y$ to the chain complex
  \[
    \dotsb
    \to
    X_{n+1}
    \xto{d_{n+1}}
    X_n
    \xto{d_n}
    X_{n-1}
    \to
    \dotsb
  \]
  to get a new chain complex
  \[
    \dotsb
    \to
    X_{n+1} \otimes Y
    \xto{d_{n+1} \otimes \id}
    X_n \otimes Y
    \xto{d_n \otimes \id}
    X_{n-1} \otimes Y
    \to
    \dotsb
  \]
  This chain complex is denoted by~$X_\bullet \otimes Y$.
  It holds that
  \[
    \Boundary_n( X_\bullet \otimes Y )
    =
    \im( d_{n+1} \otimes \id_Y )
    =
    \im( d_{n+1} ) \otimes \im( \id_Y )
    =
    \Boundary_n( X_\bullet ) \otimes Y
  \]
  for every integer~$n$, as well as
  \[
    \Cycle_n( X_\bullet \otimes Y )
    =
    \ker( d_n \otimes \id_Y )
    =
    \ker( d_n ) \otimes Y + X_n \otimes \ker( \id_Y )
    =
    \Cycle_n( X_\bullet ) \otimes Y
  \]
  for every integer~$n$.
  It follows that
  \begin{align*}
    \Homology_n( X_\bullet \otimes Y )
    &=
    \Cycle_n( X_\bullet \otimes Y) / { \Boundary_n( X_\bullet \otimes Y ) }
    \\
    &\cong
    \Cycle_n( X_\bullet ) \otimes Y / { \Boundary_n( X_\bullet) \otimes Y }
    \\
    &\cong
    ( \Cycle_n( X_\bullet ) / { \Boundary_n( X_\bullet ) } ) \otimes Y
    \\
    &=
    \Homology_n( X_\bullet ) \otimes Y
  \end{align*}
  for every integer~$n$.
\end{recall}


\begin{example}
  Let~$\glie$ be a Lie~algebra and let~$M$ be a representation of~$\glie$.
  Let~$N$ be a trivial representation of~$\glie$.
  Then
  \[
    \Chain_\bullet(\glie, M \otimes N)
    \cong
    \Chain_\bullet(\glie, M) \otimes N
  \]
  and therefore
  \begin{align*}
    \Homology_n(\glie, M \otimes N)
    &=
    \Homology_n( \Chain_\bullet(\glie, M \otimes N) )
    \\
    &\cong
    \Homology_n( \Chain_\bullet(\glie, M) \otimes N )
    \\
    &\cong
    \Homology_n( \Chain_\bullet(\glie, M) ) \otimes N
    \\
    &\cong
    \Homology_n(\glie, M) \otimes N \,.
  \end{align*}
\end{example}


\begin{fluff}
  Let~$\glie$ be a Lie~algebra and let~$M$ be a trivial representation of~$\glie$.
  Then
  \[
    \Homology_1(\glie, M)
    \cong
    \Homology_1(\glie, \kf \otimes M)
    \cong
    \Homology_1(\glie, \kf) \otimes M
    \cong
    \glie^{\ab} \otimes M \,.
  \]
\end{fluff}





\subsection{Second Cohomology}


\begin{fluff}
  Let~$\hlie$ be a Lie~algebra and let~$I$ be an abelian Lie~algebra.
  We may equivalently regard~$I$ simply as a vector space, or equivalently as a trivial representation of~$\hlie$.

  An element~$\kappa$ of~$\Chain^2(\hlie, I) = \Alt^2(\hlie, I)$ is a cocycle if and only if it satisfies the identity
  \[
    \kappa([x_1, x_2], x_3) - \kappa([x_1, x_3], x_2) + \kappa(x_1, [x_2, x_3])
    =
    0
  \]
  for all~$x_1, x_2, x_3 \in \hlie$.
  We have seen in \cref{structure of central extensions} that this means that~$\kappa$ defines a central extension of~$\hlie$ by~$I$.

  A two-coycle~$\omega$ is a coboundary if and only if there exists an element~$\varphi$ of~$\Alt^1(\hlie, I)$, i.e. of~$\Hom_{\kf}(\hlie, I)$, such that
  \[
    \omega(x_1, x_2)
    =
    -\varphi([x_1, x_2])
  \]
  for all~$x_1, x_2 \in \hlie$.
  It follows from \cref{structure of central extensions} that two two-coycles~$\kappa_1$ und~$\kappa_2$ define equivalent central extensions if and only if their difference~$\kappa_1 - \kappa_2$ is a two-coboundary.
  
  We find overall that~$\Homology^2(\hlie, I)$ is in {\onetoonetext} correspondence with the equivalence classes of central extensions of~$\hlie$ by~$I$.
\end{fluff}








