\chapter{Lie~Algebras}


% TODO: Examples for Derivations
%       k[x_1, ... x_n]
%       abelian Lie algebras
%       Example where algebra and Lie algebra derivations differ (commutative algebras)

% TODO: Every derivation of a semisimple Lie algebra is inner

% TODO: solvable Lie algebras are never ss (unless 0)

% TODO: Levi decomposition7

% TODO: Lie algebra of a bilinear form
%       Show that they are in general semisimple
%       Show via root space decomposition that they are simple

% TODO: sl(n, k) simple
%       Show this by considering the root space decomposition of gl(n, k)
%       to show that the only ideals are sl(n, k) and scalar matrices
%       then use that sl_n = [gl_n, gl_n] is a characteristic ideal,
%       or because sl_n is a direct summand

% TODO: Example: Construct some strange looking Lie algebras as semidirect products.

% TODO: When is a semidirect product semisimple?





\section{Definition and Examples}


\begin{definition}
  Let~$\glie$ be a vector space over some field~$\kf$.
  A~{\bilinear{$\kf$}} map
  \[
    [-,-]\glsadd{lie bracket}
    \colon
    \glie \times \glie
    \to
    \glie
  \]
  is a \defemph{Lie~bracket}\index{Lie!bracket} if it satisfies the following two conditions.
  \begin{enumerate}
    \item
    $[-, -]$ is alternating\index{alternating}, i.e.~$[x,x] = 0$ for every~$x \in \glie$.
    \item
    $[-, -]$ satisfies the \defemph{Jacobi identity}\index{Jacobi identity}
    \[
      [x,[y,z]] + [y,[z,x]] + [z,[x,y]]
      =
      0
      \qquad
      \text{for all~$x, y, z \in \glie$.}
    \]
  \end{enumerate}
  A~{\vectorspace{$\kf$}}~$\glie$ together with a Lie~bracket~$[-,-]$ is a~\defemph{\liealgebra{$\kf$}}\index{Lie!algebra}.
\end{definition}


\begin{remark}
  A Lie~bracket~$[-, -]$ on a vector space~$\glie$ is always antisymmetric, i.e.
  \[
    [y,x] = -[x,y]
    \qquad
    \text{for all~$x, y \in \glie$,}
  \]
  because
  \[
    0
    =
    [x+y, x+y]
    =
      \underbrace{ [x,x] }_{= 0}
    + [x,y]
    + [y,x]
    + \underbrace{ [y,y] }_{= 0}
    =
    [x,y] + [y,x] \,.
  \]
\end{remark}


\begin{remark}
  The Jacobi identity\index{Jacobi identity} can be rewritten as
  \[
    [x,[y,z]]
    =
    [[x,y],z] + [y,[x,z]]
    \qquad
    \text{for all~$x, y, z \in \glie$}
  \]
  by using the antisymmetry of the Lie~bracket.
\end{remark}


\begin{remark}
  One can more generally define the notion of an~\liealgebra{$A$}, where~$A$ is any commutative ring.
  Such a Lie~algebra consists of an~\module{$A$}~$\glie$ together with an~\bilinear{$A$} map~$[-,-] \colon \glie \times \glie \to \glie$ that is alternating and satisfies the Jacobi identity.
\end{remark}


\begin{definition}
  \leavevmode
  \begin{enumerate}
    \item
      A~\defemph{\enquote{\algebra{$\kf$}}}\index{\enquote{$k$-algebra}} is a~\vectorspace{$\kf$}~$A$ together wtih a bilinear map~$A \times A \to A$.
      This map is the \defemph{multiplication} of~$A$.
    \item
      A~\defemph{\algebra{$\kf$}}\index{k-algebra@$k$-algebra} is an~\enquote{\algebra{$\kf$}} that is both associative and unital.
  \end{enumerate}
\end{definition}

\begin{examples}
  \label{examples for lie algebras}
  \leavevmode
  \begin{enumerate}
    \item
      Any associative \enquote{\algebra{$\kf$}}~$A$ becomes a~{\liealgebra{$\kf$}} via
      \[
        [a,b]
        \defined
        ab - ba
        \qquad
        \text{for all~$a, b \in A$.}
      \]
      Indeed, this map~$[-, -] \colon A \times A \to A$ is bilinear and alternating, and it follows from the associativity of the multiplication of~$A$ that
      \begin{align*}
         {}&  [a,[b,c]] + [b,[c,a]] + [c,[a,b]] \\
        ={}&  [a, (bc-cb)] + [b, (ca-ac)] + [c, (ab-ba)] \\
        ={}&  a(bc-cb)-(bc-cb)a + b(ca-ac) - (ca-ac)b + c(ab-ba) - (ab-ba)c \\
        ={}&  \underbracket{abc}_{1}
              - \underbracket{acb}_{2}
              - \underbracket{bca}_{3}
              + \underbracket{cba}_{4}
              + \underbracket{bca}_{3}
              - \underbracket{bac}_{5}
              - \underbracket{cab}_{6}
              + \underbracket{acb}_{2}
              + \underbracket{cab}_{6}
              - \underbracket{cba}_{4}
              - \underbracket{abc}_{1}
              + \underbracket{bac}_{5} \\
        ={}&  0
      \end{align*}
      for all~$a, b, c \in A$.
      The element~$[a,b]$ of~$A$ is the \defemph{commutator}\index{commutator} of the two elements~$a$ and~$b$.
      It holds that~$[a, b] = 0$ if and only if the elements~$a$ and~$b$ commute.

      Let us emphazise some special cases of this general construction.
      \begin{enumerate}
        \item
          The~{\algebra{$\kf$}} of~($n \times n$)-matrices,~$\Mat(n, \kf)$, becomes a Lie~algebra via
          \[
            [A,B]
            \defined
            AB - BA
            \qquad
            \text{for all~$A, B \in \Mat(n, \kf)$.}
          \]
          This Lie~algebra is the \defemph{general linear Lie~algebra}\index{general linear Lie algebra}, and it is denoted by~$\gllie(n, \kf)$\glsadd{general lie matrix}.
        \item
          For any~{\vectorspace{$\kf$}}~$V$ the~{\algebra{$\kf$}}~$\End_{\kf}(V)$ becomes a Lie~algebra via
          \[
            [\phi_1, \phi_2]
            \defined
            \phi_1 \circ \phi_2 - \phi_2 \circ \phi_1
            \qquad
            \text{for all~$\phi_1, \phi_2 \in \End_{\kf}(V)$.}
          \]
          This Lie~algebra is the \defemph{general linear Lie~algebra}\index{general linear Lie algebra} of~$V$, and it is denoted by~\gls*{general lie endomorphism}.
      \end{enumerate}
    \item
      The \defemph{Heisenberg Lie~algebra}\index{Heisenberg Lie algebra}\index{Lie algebra!Heisenberg}~$\Hlie$ consists of the~\dimensional{$(2n+1)$}~{\vectorspace{$\kf$}} with basis
      \[
        P_1 \,, \dotsc \,, P_n \,,
        \quad
        Q_1 \,, \dotsc \,, Q_n \,,
        \quad
        C
      \]
      together with the Lie~bracket~$[-,-]$ that is given on these basis elements by
      \begin{itemize}
        \item
          $[P_i, P_j] = 0$ and~$[Q_i, Q_j] = 0$ for all~$i, j = 1, \dotsc, n$,
        \item
          $[P_i, C] = [Q_i, C] = 0$ for every~$i = 1, \dotsc, n$,
        \item
          $[P_i, Q_j] = \delta_{ij} C$ for all~$i,j = 1, \dotsc, n$.
      \end{itemize}
      In \cref{examples for representations} we will realize~$\Hlie$ as a Lie~subalgebra of~$\gllie(V)$ for a suitable~\vectorspace{$\kf$}~$V$.
      This will in particular show that~$\Hlie$ is indeed a Lie~algebra.
  \end{enumerate}
\end{examples}


\begin{recall}
  For calculations in~$\gllie(n, \kf)$ it is often useful to remember that
  \[
    E_{ij} E_{kl}
    =
    \begin{cases*}
      E_{il}  & if~$j = k$, \\
      0       & otherwise,
    \end{cases*}
  \]
  where the matrices~$E_{ij}$ with~$i,j = 1, \dotsc, n$ are the standard basis matrices of~$\gllie(n, \kf)$.
  One may think about the matrix~$E_{ij}$ as \enquote{going from~$j$ to~$i$}.
  The composition~$E_{ij} E_{kl}$ does then \enquote{go from~$l$ to~$i$} if the positions~$j$ and~$k$ match, and if they don’t fit, then this composition vanishes.
  
  This intuition can be formalized by observing that
  \[
    E_{ij} e_k
    =
    \begin{cases*}
      e_i & if~$j = k$, \\
      0   & otherwise,
    \end{cases*}
  \]
  where~$e_1, \dotsc, e_n$ denotes the standard basis vectors of~$\kf^n$.
  The above formula tells us that the matrix~$E_{ij}$ maps one of the standard basis vectors (namely~$e_j$) to another standard basis vector (namely~$e_i$), but filters out all other standard basis vectors.
\end{recall}


\begin{remark}[Pre-Lie~algebras]
  Let~$A$ be an~\enquote{\algebra{$\kf$}}.
  Even if we don’t require the multiplication of~$A$ to be associative, we can still define a bilinear bracket~$[-,-]$ on~$A$ via
  \[
    [a,b]
    \defined
    ab-ba
    \qquad
    \text{for all~$a, b \in A$.}
  \]
  This bracket~$[-,-]$ is alternating.
  We have seen in \cref{examples for lie algebras} that it satisfies the Jacobi identity if the multiplication of~$A$ is associative, and is then a Lie~bracket.
  But the converse does not hold true:
  it may happen that~$[-,-]$ is a Lie~bracket even though~$A$ is not associative.
  
  One important example of this are \defemph{pre-Lie~algebras}\index{pre-Lie algebra}:
  We say that~$A$ is a pre-Lie~algebra if its multiplication satisfies the condition
  \begin{equation}
    \label{abstract equation for pre-lie algebra}
    [L_a, L_b]
    =
    L_{[a,b]}
    \qquad
    \text{for all~$a, b \in A$,}
  \end{equation}
  where the left hand side is the endomorphism commutator~$[L_a, L_b] = L_a L_b - L_b L_a$, and where we denote for every element~$a$ of~$A$ by~$L_a$ the left multiplication with~$a$, i.e.\ the map
  \[
    L_a
    \colon
    A
    \to
    A \,,
    \quad
    x
    \mapsto
    ax \,.
  \]
  Writing out the condition~\eqref{abstract equation for pre-lie algebra} yields the equivalent condition
  \begin{equation}
    \label{pre lie algebra condition written out}
    a(bc) - b(ac)
    =
    (ab-ba)c
    \qquad
    \text{for all~$a, b, c \in A$,}
  \end{equation}
  or equivalently the condition
  \[
    a(bc) - (ab)c
    =
    b(ac) - (ba)c
    \qquad
    \text{for all~$a, b, c \in A$.}
  \]
  If~$A$ is a pre-Lie~algebra then the resulting commutator bracket~$[-,-]$ on~$A$ is still a Lie~bracket, even if~$A$ itself is not associative.
  Indeed, the Bracket~$[-,-]$ alternating and it follows from the characterization~\eqref{pre lie algebra condition written out} of a pre-Lie~algebra that it also satisfies the Jacobi identity, because
  \begin{align*}
    {}&
    [a, [b,c] ] + [b, [c, a]] + [c, [a, b]]
    \\
    ={}&
    a (bc - cb) - (bc - cb) a
    + b (ca - ac) - (ca - ac) b
    + c (ab - ba) - (ab - ba) c
    \\
    ={}&
    a (bc) - a (cb) - (bc - cb) a
    + b (ca) - b (ac) - (ca - ac) b
    + c (ab) - c (ba) - (ab - ba) c
    \\
    ={}&
    \underbracket{a (bc)}_{1}
    - \underbracket{a (cb)}_{2}
    - \underbracket{b (ca)}_{3}
    - \underbracket{c (ba)}_{4}
    + \underbracket{b (ca)}_{3}
    - \underbracket{b (ac)}_{5}
    - \underbracket{c (ab)}_{6}
    - \underbracket{a (cb)}_{2}
    \\
    {}&
    + \underbracket{c (ab)}_{6}
    - \underbracket{c (ba)}_{4}
    - \underbracket{a (bc)}_{1}
    - \underbracket{b (ac)}_{5}
    \\
    ={}&
    0
  \end{align*}
  for all~$a, b, c \in A$.

  Every associative \enquote{algebra} is in particular a pre-Lie~algebra, and thus we see that the Lie~algebra structure on an associative \enquote{algebra} actually comes from its pre-Lie~algebra structure.
  The situation is hence as follows:
  \[
    \text{associative \enquote{algebra}}
    \to
    \text{pre-Lie~algebra}
    \to
    \text{Lie~algebra}
  \]
 Pre-Lie~algebras were first introduced by Gerstenhaber and we refer to~\cite{gerstenhaber} to see this concept in action.
\end{remark}


\begin{definition}
  Let~$\glie$ be a~{\liealgebra{$\kf$}}.
  \begin{enumerate}
    \item
      A \defemph{Lie~subalgebra}\index{Lie!subalgebra} of~$\glie$ is a~{\linear{$\kf$}} subspace~$\hlie$ of~$\glie$ such that
      \[
        [x,y] \in \hlie
        \qquad
        \text{for all~$x, y \in \hlie$.}
      \]
    \item
      A \defemph{Lie~ideal}\index{Lie!ideal}, or simply~\defemph{ideal} of~$\glie$ is a~{\linear{$\kf$}} subspace~$I$ of~$\glie$ such that
      \[
        [x,y] \in I
        \qquad
        \text{for all~$x \in \glie$,~$y \in I$.}
      \]
      That~$I$ is an ideal in~$\glie$ is denoted by~$I \ideal \glie$\glsadd{lie ideal}.
  \end{enumerate}
\end{definition}


\begin{remark}
  \label{on the notion of ideals}
  Let~$\glie$ be a Lie~algebra.
  \begin{enumerate}
    \item
      It is not necessary to distinguish between left ideals, right ideals and two-sided ideals in~$\glie$ because the Lie~bracket of~$\glie$ is antisymmetric.
    \item
      Every ideal in~$\glie$ is in particular a Lie~subalgebra of~$\glie$.
      This is different to the setting of rings%
      \footnote{
        We consider only unital rings, unless otherwise specified.
      }, where a proper ideal is never a subring.
      Lie~algebras do instead behave more like groups, where a normal subgroup is in particular a subgroup.
  \end{enumerate}
\end{remark}


\begin{remark}
  Let~$\glie$ be a Lie~algebra.
  Every Lie~subalgebra~$\hlie$ of~$\glie$ becomes a Lie~algebra in its own right by restricting the Lie~bracket of~$\glie$ to~$\hlie$.
  It follows in particular that every ideal in~$\glie$ is again a Lie~algebra in its own right.
\end{remark}


\begin{definition}
  Let~$\glie$ be a Lie~algebra.
  The \defemph{center}\index{center} of~$\glie$ is the set
  \[
    \centerlie(\glie)\glsadd{center}
    \defined
    \{
      x \in \glie
    \suchthat
      \text{$[x,y] = 0$ for every~$y \in \glie$}
    \}  \,.
  \]
\end{definition}


\begin{proposition}
  For any Lie~algebra~$\glie$ its center~$\centerlie(\glie)$ is an ideal in~$\glie$.
  \qed
\end{proposition}


\begin{definition}
  Let~$\glie$ be a~\liealgebra{$\kf$}.
  For any two subsets~$X$,~$Y$ of~$\glie$ the linear subspace of~$\glie$ given by
  \[
    \gls*{commutator space}
    \defined
    \gen{
      [x,y]
    \suchthat
      x \in X,
      y \in Y
    }_{\kf}
  \]
  is the \defemph{commutator}\index{commutator} of~$X$ and~$Y$.
\end{definition}


\begin{definition}
  A Lie~algebra~$\glie$ is \defemph{abelian}\index{abelian} if~$[x,y] = 0$ for all~$x, y \in \glie$.
\end{definition}


\begin{proposition}
  For any Lie~algebra~$\glie$ the following three conditions are equivalent.
  \begin{equivalenceslist}
    \item
      $\glie$ is abelian.
    \item
      $\centerlie(\glie) = \glie$.
    \item
      $[\glie, \glie] = 0$.
    \qed
  \end{equivalenceslist}
\end{proposition}


\begin{examples}
  \leavevmode
  \begin{enumerate}
    \item
      A~{\algebra{$\kf$}}~$A$ is commutative if and only if it is abelian as a Lie~algebra.
    \item
      In any Lie~algebra~$\glie$ every element~$x$ of~$\glie$ spans a {\onedimensional} abelian Lie~subalgebra of~$\glie$, given by~$\gen{x}_{\kf} = \{ \lambda x \suchthat \lambda \in \kf \}$.
    \item
      Every vector space~$\glie$ can be made into an abelian Lie~algebra is precisely one way, namely by setting~$[x, y] \defined 0$ for all~$x, y \in \glie$.
  \end{enumerate}
\end{examples}


\begin{proposition}
  \label{construction of new ideals}
  Let~$\glie$ be a Lie~algebra.
  \begin{enumerate}
    \item
    If~$I_\lambda$ with~$\lambda \in \Lambda$ is a collection of ideals of~$\glie$ then their intersection~$\bigcap_{\lambda \in \Lambda} I_\lambda$ and their sum~$\sum_{\lambda \in \Lambda} I_\lambda$ are again ideals in~$\glie$.
    \item
    If~$I$ and~$J$ are two ideals in~$\glie$ then their commutator~$[I,J]$ is again an ideal in~$\glie$.
  \end{enumerate}
\end{proposition}


\begin{proof}
  \leavevmode
  \begin{enumerate}
    \item
      This follows from direct calculation.
    \item
      It sufficies to show that~$[x,[y,z]] \in [I,J]$ for all~$x \in \glie$,~$y \in I$ and~$z \in J$, because the commutator space~$[I,J]$ is spanned by the elements~$[y,z]$ with~$y \in I$ and~$z \in J$ .
      By using the Jacobi identity and that~$I$,~$J$ are ideals of~$\glie$ we find that
      \[
        [x,[y,z]]
        =
        [[x,y], z] + [y, [x,z]]
        \subseteq
        [I, z] + [y, J]
        \subseteq
        [I, J] + [J, I]
        =
        [I,J] \,.
      \]
      Here we use for the last equality that~$[J,I] = -[I,J] = [I,J]$ by the antisymmetry of the Lie~bracket.
   \qedhere
 \end{enumerate}
\end{proof}


\begin{example}
  Any Lie~algebra~$\glie$ is an ideal of itself.
  It thus follows from \cref{construction of new ideals} that the commutator space
  \[
    [\glie, \glie]
    =
    \gen{ [x,y] \suchthat x, y \in \glie }
  \]
  is again an ideal in~$\glie$.
  This is the \defemph{commutator ideal}\index{commutator!ideal} of~$\glie$.
\end{example}


\begin{definition}
  A Lie~algebra~$\glie$ is~\defemph{linear}\index{linear} if it is a Lie~subalgebra of~$\gllie(V)$ for some finite-dimensional vector space~$V$, or a Lie~subalgebra of~$\gllie(n, \kf)$ for some~$n$.
\end{definition}


\begin{examples}[Linear Lie~algebras]
  \leavevmode
  \begin{enumerate}
    \item
      Let~$\glie \defined \gllie(n, \kf)$.
      Then
      \[
        \sllie(n, \kf)
        \defined
        \{
          A \in \gllie(n, \kf)
        \suchthat
          \tr(A) = 0
        \}
        \glsadd{special lie matrix}
      \]
      is an ideal in~$\glie$, namely the commutator ideal~$[\glie, \glie]$.
      Indeed, we have for all~$A, B \in \glie$ that
      \[
        \tr(AB) = \tr(BA)
      \]
      and therefore
      \[
          \tr( [A,B] )
        = \tr(AB-BA)
        = \tr(AB) - \tr(BA)
        = \tr(AB) - \tr(AB)
        = 0  \,.
      \]
      This shows that the commutator ideal~$[\glie, \glie]$ is contained~$\sllie(n, \kf)$.
      We observe on the other hand that the matrix space~$\sllie(n, \kf)$ has a basis given by the matrices~$E_{ij}$ for~$1 \leq i \neq j \leq n$ together with the matrices~$E_{11} - E_{ii}$ for~$i = 2, \dotsc, n$.
      Each of these matrices is a commutator, namely
      \[
          [E_{ij}, E_{jj}]
          =
          E_{ij} E_{jj} - \underbrace{ E_{jj} E_{ij} }_{=0}
          =
          E_{ij}
      \]
      for~$1 \leq i \neq j \leq n$ and
      \[
        [E_{1i}, E_{i1}]
        =
        E_{1i} E_{i1} - E_{i1} E_{1i}
        =
        E_{11} - E_{ii}
      \]
      for~$i = 2, \dotsc, n$.
      This shows that~$\sllie(n, \kf)$ is contained in the commutator Ideal~$[\glie, \glie]$.
      
      The Lie~algebra~$\sllie(n, \kf)$ is the \defemph{special linear Lie~algebra}\index{special linear Lie algebra}.
    \item
      If~$V$ is any finite-dimensional~{\vectorspace{$\kf$}} then
      \[
        \sllie(V)
        \defined
        [\gllie(V), \gllie(V)]
        =
        \{
          f \in \gllie(V)
        \suchthat
          \tr(f) = 0
        \}
        \glsadd{special lie endomorphism}
      \]
      is the \defemph{special linear Lie~algebra}\index{special linear Lie algebra} of~$V$.
    \item
      The set of upper triangular matrices\index{upper triangular matrices}\index{triangular matrices!upper}, given by
      \[
        \tlie(n, \kf)
        \defined
        \left\{
          \begin{pmatrix}
              a_{11}
            & \cdots
            & \cdots
            & a_{1n}
            \\
              {}
            & \ddots
            & {}
            & \vdots
            \\
              {}
            & {}
            & \ddots
            & \vdots
            \\
              {}
            & {}
            & {}
            & a_{nn}
          \end{pmatrix}
        \suchthat*
          \text{$a_{ij} \in \kf$ for all~$1 \leq i \leq j \leq n$}
        \right\} \,,
        \glsadd{triangular lie matrix}
      \]
      is a Lie~subalgebra of~$\gllie(n, \kf)$.
      This holds because the set~$\tlie(n, \kf)$ is a~{\subalgebra{$\kf$}} of~$\Mat(n, \kf)$, and hence closed under the commutator bracket~$[A,B] = AB - BA$.
   
    \item
      The set of strictly upper triangular matrices\index{stricly upper triangular matrices}\index{triangular matrices!strictly upper}
      \[
        \nlie(n, \kf)
        \defined
        \left\{
          \begin{pmatrix}
              0
            & a_{12}
            & \cdots
            & a_{1n}
            \\
              {}
            & \ddots
            & \ddots
            & \vdots
            \\
              {}
            & {}
            & \ddots
            & a_{n-1,n}
            \\
              {}
            & {}
            & {}
            & 0
          \end{pmatrix}
        \suchthat*
          \text{$a_{ij} \in \kf$ for all~$1 \leq i < j \leq n$}
        \right\}
        \glsadd{strictly triangular lie matrix}
      \]
      is a Lie~subalgebra of~$\tlie(n, \kf)$, and therefore also a Lie~subalgebra of~$\gllie(n, \kf)$.
      The space~$\nlie(n, \kf)$ is even an ideal in~$\tlie(n, \kf)$, namely the commutator ideal~$[\tlie(n, \kf), \tlie(n, \kf)]$.
    
      Indeed, for any two upper triangular matrices~$A$,~$B$ their two products~$AB$ and~$BA$ are again upper triangular, and both products have the same diagonal entries.
      The commutator~$[A,B] = AB - BA$ is therefore a strictly upper triangular matrix, which shows that the commutator ideal~$[\tlie(n, \kf), \tlie(n, \kf)]$ is contained in~$\nlie(n, \kf)$.
      We know on the other hand that~$\nlie(n, \kf)$ has as a basis the matrices~$E_{ij}$ with~$1 \leq i < j \leq n$.
      Each of those matrices can be written as a commutator, namely as
      \[
        [E_{ii}, E_{ij}]
        =
        E_{ii} E_{ij} - \underbrace{ E_{ij} E_{ii} }_{= 0}
        =
        E_{ij} \,,
      \]
      with both~$E_{ii}$ and~$E_{ij}$ being elements of~$\tlie(n, \kf)$.
      This shows that~$\nlie(n, \kf)$ is contained in the commutator ideal~$[\tlie(n, \kf), \tlie(n, \kf)]$.
    \item
      The set of diagonal matrices
      \[
        \dlie(n, \kf)
        =
        \left\{
          \begin{pmatrix}
              d_1
            & {}
            & {}
            \\
              {}
            & \ddots
            & {}
            \\
              {}
            & {}
            & d_n
          \end{pmatrix}
        \suchthat*
          d_1, \dotsc, d_n \in \kf
        \right\}
      \]
      is an~{\dimensional{$n$}} abelian Lie~subalgebra of~$\gllie(n, \kf)$.
      A basis of~$\dlie(n, \kf)$ is given by the matrices~$E_{ii}$ with~$i = 1, \dotsc, n$.
  \end{enumerate}
\end{examples}


\begin{remark}
  The Lie~algebra~$\sllie(2,\kf)$ will play a crucial role in the second part of these notes.
  We have already observed in the above discussion that~$\sllie(2,\kf)$ has a basis given by the three matrices
  \[
    e
    \defined
    \begin{pmatrix}
      0 & 1 \\
      0 & 0
    \end{pmatrix} \,,
    \qquad
    h
    \defined
    \begin{pmatrix*}[r]
      1 &  0  \\
      0 & -1
    \end{pmatrix*}  \,,
    \qquad
    f
    \defined
    \begin{pmatrix}
      0 & 0 \\
      1 & 0
    \end{pmatrix} \,.
  \]
  The Lie~bracket~$[-,-]$ of~$\sllie(2, \kf)$ is on these basis elements given by
  \[
    [h, e]
    =
    2e  \,,
    \qquad
    [h, f]
    =
    -2 f \,,
    \qquad
    [e,f]
    =
    h \,.
  \]
  We will see these relations quite a lot later on.
\end{remark}


\begin{fluff}
  The above basis of~$\sllie(2, \kf)$ is so important that we give it a special name.
\end{fluff}


\begin{definition}
 Let~$\kf$ be any field.
 The basis
 \[
    e\glsadd{standard basis e}
    =
    \begin{pmatrix}
      0 & 1 \\
      0 & 0
    \end{pmatrix} \,,
    \qquad
    h\glsadd{standard basis h}
    =
    \begin{pmatrix*}[r]
      1 &  0  \\
      0 & -1
    \end{pmatrix*}  \,,
    \qquad
    f\glsadd{standard basis f}
    =
    \begin{pmatrix}
      0 & 0 \\
      1 & 0
    \end{pmatrix} \,.
  \]
  of the Lie~algebra~$\sllie(2, \kf)$ is its \defemph{standard basis}\index{standard basis}.
\end{definition}


\begin{definition}
  Let~$\glie$ be a Lie~algebra.
  \begin{enumerate}
    \item
      Let~$U$ be a linear subspace of~$\glie$.
      The set
      \[
        \gls*{normalizer}
        \defined
        \{
          x \in \glie
        \suchthat
          \text{$[x,y] \in U$ for every~$y \in U$}
        \}
      \]
      is the \defemph{normalizer}\index{normalizer} of~$U$ in~$\glie$, and the set
      \[
        \gls*{centralizer of space}
        \defined
        \{
          x \in \glie
        \suchthat
          \text{$[x,y] = 0$ for every~$y \in U$}
        \}
      \]
      is the \defemph{centralizer}\index{centralizer} of~$U$ in~$\glie$.
    \item
      For any element~$x$ of~$\glie$ its \defemph{centralizer}\index{centralizer} is the set
      \[
        \gls*{centralizer of element}
        \defined
        \{
          y \in \glie
        \suchthat
          {}
          [x,y] = 0
        \} \,.
      \]
  \end{enumerate}
\end{definition}


\begin{proposition}
  Let~$\glie$ be a Lie~algebra.
  Let~$U$ be a linear subspace of~$\glie$.
  \begin{enumerate}
    \item
      The normalizer~$\normallie_{\glie}(U)$ is a Lie~subalgebras of~$\glie$.
    \item
      The linear subspace~$U$ of~$\glie$ is a Lie~subalgebra if and only if its normalizer~$\normallie_{\glie}(U)$ is all of~$\glie$.
    \item
      The centralizer~$\centerlie_{\glie}(U)$ is an ideal of the normalizer~$\normallie_{\glie}(U)$, and a Lie~subalgebra of~$\glie$.
  \end{enumerate}
  Let~$x$ be an element of~$\glie$.
  \begin{enumerate}[resume*]
    \item
      The centralizer~$\centerlie_{\glie}(x)$ is a Lie~subalgebra of~$\glie$.
 \end{enumerate}
\end{proposition}


\begin{proof}
  \leavevmode
  \begin{enumerate}
    \item
      Let~$x$,~$y$ be two elements of~$\normallie_{\glie}(U)$.
      It follows for every element~$z$ of~$U$ from the Jacobi~identity that
      \[
        [[x,y], z]
        =
        - [z, [x,y]]
        =
        - [[z,x], y] - [x, [z,y]
        \in
        - [U, y] - [x, U]
        \subseteq
        - U - U
        =
        U \,.
      \]
      This shows that the commutator~$[x,y]$ is again contained in~$\normallie_{\glie}(U)$.
    \item
      This follows immediately from the definitions.
    \item
      Let~$x$ be any element of the normalizer~$\normallie_{\glie}(U)$ and let~$y$ be any element of the centralizer~$\centerlie_{\glie}(U)$.
      We need to show that the commutator~$[x,y]$ is again contained in the centralizer~$\centerlie_{\glie}(U)$.
      In other words, we have to show that
      \[
        [[x,y],z] = 0
        \qquad
        \text{for every~$z \in U$.}
      \]
      It follows from the choice of~$x$ that the commutator~$[x,z]$ is again contained in~$U$, and it thus follows from the choice of~$y$ that both~$[[x,z],y] = 0$ and~$[y,z] = 0$.
      We now find with some help from the Jacobi identity that
      \[
        [[x,y], z]
        =
        -[z, [x,y]]
        =
        -[[z, x], y] - [x, [z, y]]
        =
        [[x,z], y] + [x, [y, z]]
        =
        0 \,.
      \]
      This shows that the centralizer~$\centerlie_{\glie}(U)$ is indeed an ideal of the normalizer~$\normallie_{\glie}(U)$.

      It follows from this that~$\centerlie_{\glie}(U)$ is a Lie~subalgebra of~$\normallie_{\glie}(U)$, und therefore also a Lie~subalgebra of~$\glie$.
    \item 
      The linear span~$\kf x$ is a linear subspace of~$\glie$ with~$\centerlie_{\glie}(x) = \centerlie_{\glie}(\kf x)$.
      It therefore follows from the previous assertion that~$\centerlie_{\glie}(x)$ is a Lie~subalgebra of~$\glie$.
    \qedhere
  \end{enumerate}
\end{proof}





\section{New Lie~Algebras From Old Ones}


\begin{proposition}
  \label{construction of product lie algebra}
  Let~$\glie_{\lambda}$ with~$\lambda \in \Lambda$ be a family of Lie~algebras.
  Then the mapping
  \[
    \bigl[ (x_\lambda)_\lambda, (y_\lambda)_\lambda \bigr]
    \defined
    ( [x_\lambda, y_\lambda] )_\lambda
    \qquad
    \text{for all~$(x_\lambda)_\lambda, (y_\lambda)_\lambda \in \prod_{\lambda \in \Lambda} \glie_\lambda$}
  \]
  is a Lie~bracket on the vector space~$\prod_{\lambda \in \Lambda} \glie_\lambda$.
  \qed
\end{proposition}


\begin{definition}
  In the situation of \cref{construction of product lie algebra} the resulting Lie~algebra~$\prod_{\lambda \in \Lambda} \glie_\lambda$\glsadd{product of lie algebras} is the \defemph{(direct) product} of the Lie~algebras~$\glie_\lambda$.
\end{definition}


\begin{proposition}
  \label{construction of direct sum of lie algebras}
  Let~$\glie_\lambda$ with~$\lambda \in \Lambda$ be a family of Lie~algebras.
  Then the direct sum~$\bigoplus_{\lambda \in \Lambda} \glie_\lambda$ is a Lie~subalgebra of the product~$\prod_{\lambda \in \Lambda} \glie_\lambda$.
  Is does therefore inherit the structure of a Lie~algebra from~$\prod_{\lambda \in \Lambda} \glie_\lambda$.
\end{proposition}


\begin{proof}
  Let~$(x_\lambda)_\lambda$ and~$(y_\lambda)_\lambda$ be two elements of~$\bigoplus_{\lambda \in \Lambda} \glie_\lambda$.
  Then there exist only finitely many indices~$\lambda \in \Lambda$ for which one of the vectors~$x_\lambda$ or~$y_\lambda$ does not vanish.
  There hence exist only finitely many indices~$\lambda \in \Lambda$ for which~$[x_\lambda, y_\lambda]$ does not vanish.
  This shows that~$[ (x_\lambda)_\lambda, (y_\lambda)_\lambda ]$ is again contained in~$\bigoplus_{\lambda \in \Lambda} \glie_\lambda$.
\end{proof}


\begin{definition}
  In the situation of \cref{construction of direct sum of lie algebras} the resulting Lie~algebra~$\bigoplus_{\lambda \in \Lambda} \glie_\lambda$ is the \defemph{external direct sum}\glsadd{direct sum of lie algebras} of the Lie~algebras~$\glie_\lambda$.
\end{definition}


\begin{proposition}
  \label{construction of quotient lie algebra}
  Let~$\glie$ be a Lie~algebra and let~$I$ an ideal of~$\glie$.
  Then the quotient vector space~$\glie/I$ inherits from~$\glie$ the structure of a Lie~algebra via the bracket
  \[
    [x+I, y+I]
    \defined
    [x,y] + I
    \qquad
    \text{for all~$x, y \in \glie$.}
  \]
\end{proposition}


\begin{proof}
  The bilinear map~$[-,-]$ on~$\glie/I$ is well-defined:
  suppose that ~$x$,~$y$,~$x'$,~$y'$ are elements of~$\glie$ such that both~$x$ and~$x'$, as well as~$y$ and~$y'$ represent the same element of~$\glie / I$.
  Then~$x - x' \in I$ and~$y - y' \in I$ and thus
  \begin{align*}
    [x,y] + I
    &=
    [x' + (x-x'), y' + (y-y')]
    \\
    &=
    [x',y']
    + \underbrace{[x', y-y']}_{\in I}
    + \underbrace{[x-x', y']}_{\in I}
    + \underbrace{[x-x', y-y']}_{\in I} \,.
  \end{align*}
  This shows that the difference~$[x, y] - [x', y']$ is again contained in~$I$, which means that~$[x, y]$ and~$[x', y']$ represent the same element of~$\glie / I$.
  
  It remains to show that the bracket~$[-,-]$ on~$\glie / I$ is bilinear, alternating, and satisfies the Jacobi identity.
  But these properties are inherited from the bracket of~$\glie$, as they can be checked on representatives for the elements of~$\glie / I$.
\end{proof}


\begin{definition}
  The Lie~algebra~$\glie / I$\glsadd{quotient lie algebra} from \cref{construction of quotient lie algebra} is the \defemph{quotient}\index{quotient! Lie algebra}\index{Lie algebra!quotient} of~$\glie$ by~$I$.
\end{definition}


\begin{example}
  Let~$\glie$ be a Lie~algebra.
  If~$I$ is any ideal of~$\glie$ then the quotient~$\glie / I$ is abelian if and only if~$I$ contains the commutator ideal~$[\glie, \glie]$.
  This holds because
  \begin{align*}
    {}&
    \text{$\glie / I$ is abelian}
    \\
    \iff{}&
    \text{$[ \class{x}, \class{y}] = 0$ for all~$x, y \in \glie$}
    \\
    \iff{}&
    \text{$\class{[x,y]} = 0$ for all~$x, y \in \glie$}
    \\
    \iff{}&
    \text{$[x,y] \in I$ for all~$x, y \in \glie$}
    \\
    \iff{}&
    [\glie, \glie] \subseteq I \,.
  \end{align*}
\end{example}


\begin{proposition}
  \label{quasi extension of scalars for lie algebras}
  \index{extension!of scalars}
  Let~$\glie$ be a Lie~algebra over a field~$\kf$ and let~$A$ be an commutative~{\algebra{$\kf$}}.
  (This entails that~$A$ is associative and unital.)
  \begin{enumerate}
    \item
      The tensor product~$A \tensor_{\kf} \glie$ becomes an~\liealgebra{$A$} via the bracket
      \begin{alignat*}{2}
        [a \tensor x, b \tensor y]
        &\defined
        (ab) \tensor [x,y]
        &\qquad
        &\text{for all simple tensors~$a \tensor x, b \tensor y \in A \tensor_{\kf} \glie$.}
    \intertext{
    \item
      The tensor product~$\glie \tensor_{\kf} A$ becomes an~\liealgebra{$A$} via the bracket
    }
        [x \tensor a, y \tensor b]
        &\defined
        [x,y] \tensor (ab)
        &\qquad
        &\text{for all simple tensors~$x \tensor a, y \tensor b \in \glie \tensor_{\kf} A$.}
      \end{alignat*}
  \end{enumerate}
\end{proposition}


\begin{proof}
  We prove only the first assertion, as the second one can be shown in the same way.

  The Lie~bracket of~$\glie$ is~\bilinear{$\kf$} and can therefore be regarded as a~\linear{$\kf$} map
  \[
    l \colon \glie \tensor_{\kf} \glie \to \glie \,.
  \]
  By the functoriality of the extension of scalars we now get an induced~\linear{$A$} map
  \[
    \id_A \tensor l
    \colon
    A \tensor_{\kf} \glie \tensor_{\kf} \glie
    \to
    A \tensor_{\kf} \glie \,,
  \]
  which maps any simple tensor~$a \tensor x \tensor y$ onto the simple tensor~$a \tensor [x,y]$.
  We now recall that there exists a unique isomorphism of~\modules{$A$}
  \[
    A \tensor_{\kf} \glie \tensor_{\kf} \glie
    \cong
    ( A \tensor_{\kf} \glie ) \tensor_A ( A \tensor_{\kf} \glie )
  \]
  under which a simple tensor~$a \tensor x \tensor b \tensor y$ of the right hand side corresponds to the simple tensor~$(ab) \tensor x \tensor y$ of the left hand side.
  By considering the composition
  \[
    (A \tensor_{\kf} \glie) \tensor_A (A \tensor_{\kf} \glie)
    \to
    A \otimes_{\kf} \glie \otimes_{\kf} \glie
    \xto{\id_A \otimes l}
    A \tensor_{\kf} \glie
  \]
  we get altogether an~{\linear{$A$}} map
  \[
    (A \tensor_{\kf} \glie) \tensor_A (A \tensor_{\kf} \glie)
    \to
    A \tensor_{\kf} \glie
  \]
  which maps any simple tensor~$a \tensor x \tensor b \tensor y$ onto the the simple tensor~$(ab) \tensor [x,y]$.
  We may reinterpret this~\linear{$A$} map as an~\bilinear{$A$} map
  \[
    [-,-]
    \colon
    (A \tensor_{\kf} \glie) \times (A \tensor_{\kf} \glie)
    \to
    A \tensor_{\kf} \glie
  \]
  that is given on (pairs of) simple tensors by
  \[
    [a \tensor x, b \tensor y]
    =
    (ab) \tensor [x,y] \,.
  \]
  This~\bilinear{$A$} map is precisely the desired bracket.

  It remains to show this bracket~$[-,-]$ on~$A \tensor_{\kf} \glie$ is already a Lie~bracket.
  For this we need to show that it is alternating and that it satisfies the Jacobi identity.

  To see that it is alternating let~$t$ be an elements of~$A \tensor_{\kf} \glie$.
  We may write this element as sums of simple tensors, say~$t = \sum_{i=1}^n a_i \tensor x_i$.
  We can then compute the commutator~$[t, t]$ as
  \[
    [t, t]
    =
    \Biggl[
      \sum_{i=1}^n a_i \tensor x_i,
      \sum_{j=1}^n a_j \tensor x_j
    \Biggr]
    =
    \sum_{i, j = 1}^n
    [ a_i \tensor x_i, a_j \tensor x_j ]
    =
    \sum_{i, j = 1}^n
    ( a_i a_j ) \tensor [ x_i, x_j ] \,.
  \]
  The summands for~$i = j$ vanish since in this case~$[x_i, x_j] = [x_i, x_i] = 0$.
  For~$i \neq j$ the two occuring summands~$(a_i a_j) \tensor [x_i, x_j]$ and~$(a_j a_i) \tensor [x_j, x_i]$ cancel each other because~$a_i a_j = a_j a_i$ while~$[x_i, x_j] = -[x_j, x_i]$.
  Thus overall~$[t, t] = 0$.
  This shows that the bracket~$[-,-]$ on~$A \tensor_{\kf} \glie$ is alternating.

  It remains to show that the bracket on~$A \tensor_{\kf} \glie$ satisfies the Jacobi identity.
  We hence want to show that the map
  \begin{gather*}
    J
    \colon
    (A \tensor_{\kf} \glie) \times (A \tensor_{\kf} \glie) \times (A \tensor_{\kf} \glie)
    \to
    (A \tensor_{\kf} \glie)
  \shortintertext{given by}
    (t, u, v)
    \mapsto
    [t, [u, v]] + [u, [v, t]] + [v, [t, u]]
  \end{gather*}
  is the zero map.
  The map~$J$ is~\trilinear{$A$} by the~\bilinearity{$A$} of the bracket~$[-,-]$ on~$A \tensor_{\kf} \glie$.
  It hence sufficies to show that the map~$J$ is zero on an~\linear{$A$} generating set of~$A \tensor_{\kf} \glie$.
  Such a generating set is given by the simple tensors~$1 \tensor x$ with~$x \in \glie$.
  We find for these generators that
  \begin{align*}
    {}&
    J(1 \tensor x, 1 \tensor y, 1 \tensor z)
    \\
    ={}&
      [1 \tensor x, [1 \tensor y, 1 \tensor z]]
    + [1 \tensor y, [1 \tensor z, 1 \tensor x]]
    + [1 \tensor z, [1 \tensor x, 1 \tensor y]]
    \\
    ={}&
    1 \tensor \bigl( [x, [y, z]] + [y, [z, x]] + [z, [x, y]] \bigr)
    \\
    ={}&
    1 \tensor 0
    \\
    ={}&
    0
  \end{align*}
  for all~$x, y, z \in \glie$ by the Jacobi identity for~$\glie$.
  This shows that~$J = 0$, as desired.
\end{proof}


\begin{example}
  \index{extension!of scalars}
  Let~$\Lf/\kf$ be a field extension and let~$\glie$ be a Lie~algebra over~$\kf$.
  Then the tensor product~$\Lf \tensor_{\kf} \glie$ is an~\liealgebra{$\Lf$}, via the Lie~bracket that is given by
  \[
    [\lambda \tensor x, \mu \tensor y]
    = 
    (\lambda \mu) \tensor [x,y]
    \qquad
    \text{for all simple tensor~$\lambda \tensor x, \mu \tensor y \in \Lf \tensor_{\kf} \glie$.}
  \]
  We can moreover regard~$\glie$ as an~\liesubalgebra{$\kf$} of~$\Lf \tensor_{\kf} \glie$ via the inclusion
  \[
    \glie
    \to
    \Lf \tensor_{\kf} \glie \,,
    \quad
    x
    \mapsto
    1 \tensor x \,.
  \]
\end{example}


\begin{example}
  Let~$\glie$ be a Lie~algebra and let~$\kf[t, t^{-1}]$ be the~\algebra{$\kf$} of Laurant polynomials over~$\kf$.
  Then
  \[
    \gls*{loop lie algebra}
    \defined
    \glie \tensor_{\kf} \kf[t, t^{-1}]
  \]
  with the Lie~bracket as in~\cref{quasi extension of scalars for lie algebras} is the \defemph{loop \textup(Lie\textup)~algebra}\index{loop Lie algebra} of~$\glie$.
\end{example}
 
 
% \begin{example}
%   Another example for constructing new Lie~algebras out of old ones are \defemph{central extensions}:
%   Let~$\glie$ be any~$\kf$-Lie~algebra.
%   Then let
%   \[
%     \widetilde{\glie}
%     \defined
%     \glie \oplus \kf
%     =
%     \{
%       x + \lambda c
%     \suchthat
%       x \in \glie,
%       \lambda \in \kf
%     \},
%   \]
%   where we understand~$c$ as a formal variable.
%   Suppose that~$\kappa \colon \glie \times \glie \to \kf$ is a~{\bilinear{$\kf$}} map satisfying the following properties:
%   \begin{enumerate}
%   \item
%     $\kappa$ is antisymmetric, i.e.~$\kappa(x,y) = -\kappa(y,x)$ for all~$x,y \in \glie$.
%   \item
%     $\kappa$ satisfies the \defemph{$2$-cocycle condition}\index{$2$-cocycle condition}
%     \[
%       \kappa([x,y],z) + \kappa([y,z],x) + \kappa([z,x],y) = 0
%     \]
%     for all~$x, y, z \in \glie$.
%   \end{enumerate}
%   Then~$\widetilde{\glie}$ becomes a Lie~algebra via
%   \[
%     [x + \lambda c, y + \mu c]
%     \defined
%     [x,y] + \kappa(x,y) c
%   \]
%   for all~$x, y \in \glie$ and~$\lambda, \mu \in \kf$.
%   Note that the element~$c$ is central in~$\widetilde{\glie}$ in the sense that~$[x,c] = 0$ for all~$x \in \glie$.
%   
%   Take for example~$\glie \defined \gllie(n, \kf)$.
%   We can then define a symmetric bilinear form~$(-,-)_{\tr}$ on~$\glie$ via
%   \[
%     (A,B)_{\tr}
%     \defined
%     \tr(AB)
%   \]
%   for all~$A, B \in \glie$.
%   We can use~$(-,-)_{\tr}$ to define on the Loop~algebra~$\looplie(\glie)$ a~{\bilinear{$\kf[t,t^{-1}]$}} form~$(-,-)$ via
%   \[
%     \looplie(\glie) \times \looplie(\glie)
%     \to
%     \kf[t,t^{-1}] \,,
%     \quad
%     (x \tensor p, y \tensor q)
%     \mapsto
%     (x,y)_{\tr} \, pq \,.
%   \]
%   We get from this a bilinear form a~{\twococycle}~$\kappa \colon \looplie(\glie) \times \looplie(\glie) \to \kf$ via
%   \[
%     \kappa(a,b)
%     \defined
%     \Res\left( \frac{\partial a}{\partial t}, b \right) \,.
%   \]
%   The bilinear form~$\kappa$ is also antisymmetric:
%   Let~$a = x \tensor t^i$ and~$b = y \tensor t^{j}$ with~$x,y \in \glie$ and~$i,j \in \Integer$.
%   Then
%   \begin{align*}
%     \kappa(x \tensor t^i, y \tensor t^{j})
%     &=
%     \Res(i x \tensor t^{i-1}, y \tensor t^{j})
%     \\
%     &= 
%     \Res(i t^{i+j-1} (x,y)_{\tr})
%     \\
%     &=
%     \begin{cases}
%       i (x,y)_{\tr} & \text{if~$i+j = 0$} \,,\\
%                   0 & \text{otherwise}  \,.
%     \end{cases}
%   \end{align*}
%   In the same way we find that
%   \[
%     \kappa(y \tensor t^{j}, x \tensor t^i)
%     =
%     \begin{cases}
%        j (x,y)_{\tr} & \text{if~$i+j = 0$} \,, \\
%                    0 & \text{otherwise}  \,.
%     \end{cases}
%   \]
%   Since~$(\cdot,\cdot)_{\tr}$ is symmetric we find that
%   \begin{align*}
%     \kappa(x \tensor t^i, y \tensor t^{j})
%     &=
%     \begin{cases}
%     i (x,y)_{\tr} & \text{if~$i+j = 0$} \,, \\
%                 0 & \text{otherwise}  \,,
%     \end{cases} \\
%     &=
%     \begin{cases}
%     -j (x,y)_{\tr} & \text{if~$i+j = 0$}  \,, \\
%                   0 & \text{otherwise}  \,,
%     \end{cases} \\
%     &=
%     -\kappa(y \tensor t^{j}, x \tensor t^i) \,.
%   \end{align*}
% \end{example}
% 
% 
% \begin{remark}
%   During the rest of these notes we will never see the Loop algebra again.
% \end{remark}


\begin{recall}
  If~$A$ is a~\enquote{\algebra{$\kf$}} then we can form the \defemph{opposite \enquote{algebra}}\index{opposite!algebra}~$A^{\op}$\glsadd{opposite algebra} of~$A$.

  The \enquote{algebras}~$A$ and~$A^{\op}$ have the same underlying~\vectorspace{$\kf$}.
  For every element~$a$ of~$A$ the corresponding element of~$A^{\op}$ (which is just~$a$ itself) is denoted by~$a^{\op}$.
  The multplication on~$A^{\op}$ is given by
  \[
    a^{\op} \cdot b^{\op}
    \defined
    (b \cdot a)^{\op}
    \qquad
    \text{for all~$a, b \in A$.}
  \]
  It holds that~$(A^{\op})^{\op} = A$.

  The \enquote{algebra}~$A$ is associative if and only if its oppositie \enquote{algebra}~$A^{\op}$ is associative, and an element~$1$ of~$A$ is a multiplicative neutral element of~$A$ if and only if~$1^{\op}$ is a multiplicative neutral element of~$A^{\op}$.
  It follows that~$A$ is a~{\algebra{$\kf$}} if and only if~$A^{\op}$ is one.
\end{recall}


\begin{proposition}
  If~$\glie$ is a~\liealgebra{$\kf$} with Lie~bracket~$[-,-]$ then the opposite \enquote{algebra}~$\glie^{\op}$, whose bracket is given by
  \[
    [x^\op, y^\op]
    =
    [y,x]^{\op}
    =
    -[x,y]^{\op}
    \qquad
    \text{for all~$x, y \in \glie$,}
  \]
  is again a Lie~algebra.
  \qed
\end{proposition}


\begin{definition}
  Let~$\glie$ be a Lie algebra.
  The Lie algebra~$\glie^{\op}$\glsadd{opposite lie algebra} is the \defemph{opposite Lie~algebra}\index{opposite!Lie-algebra} of~$\glie$.
\end{definition}





\section{Homomorphisms of Lie~Algebras}



\subsection{Definition and Basic Examples}


\begin{definition}
  Let~$\glie$ and~$\hlie$ be two~\liealgebras{$\kf$}.
  A~\linear{$\kf$} map~$\varphi \colon \glie \to \hlie$ is a \defemph{homomorphism of Lie~algebras}\index{homomorphism!of Lie algebras} if it satisfies the condition
 \[
   \varphi([x,y])
   =
   [\varphi(x), \varphi(y)]
   \qquad
   \text{for all~$x, y \in \glie$.}
 \]
\end{definition}


\begin{proposition}
  Let~$\varphi \colon \glie \to \hlie$ be a homomorphism of Lie~algebras.
  \begin{enumerate}
    \item
      The image of~$\varphi$ is a Lie~subalgebra of~$\hlie$.
    \item
      The kernel of~$\varphi$ is an ideal of~$\glie$.
  \end{enumerate}
\end{proposition}


\begin{proof}
  \leavevmode
  \begin{enumerate}
    \item
      The image of~$\varphi$ is a linear subspace of~$\hlie$ because~$\varphi$ is linear.
      For any two element~$y_1$,~$y_2$ of~$\im(\varphi)$ there exists elements~$x_1$,~$x_2$ of~$\glie$ with~$y_1 = \varphi(x_1)$ and~$y_2 = \varphi(x_2)$.
      We have
      \[
        [y_1, y_2]
        =
        [ \varphi(x_1), \varphi(x_2) ]
        =
        \varphi( [x_1, x_2] ) \,,
      \]
      which shows that the commutator~$[y_1, y_2]$ is again contained in the image of~$\varphi$.
    \item
      The kernel of~$\varphi$ is a linear subspace of~$\glie$ because~$\varphi$ is linear.
      Let~$x$ by any element of~$\glie$ and let~$y$ any element of~$\ker(\varphi)$.
      We have
      \[
        \varphi( [x, y] )
        =
        [ \varphi(x), \varphi(y) ]
        =
        [ \varphi(x), 0 ]
        =
        0 \,,
      \]
      which shows that the commutator~$[x, y]$ is again contained in the kernel of~$\varphi$.
    \qedhere
  \end{enumerate}
\end{proof}


\begin{examples}
  \label{homomorphisms of lie algebras}
  Let~$\glie$,~$\hlie$,~$\klie$ be Lie~algebras over the same field~$\kf$.
  \begin{enumerate}
    \item
      \label{identity is a homomorphism of lie algebras}
      The identity~$\id_{\glie} \colon \glie \to \glie$ is a homomorphism of Lie~algebras.
    \item
      \label{composite of homomorphisms of lie algebras}
      If~$\varphi \colon \glie \to \hlie$ and~$\psi \colon \hlie \to \klie$ are two composable homomorphsims of Lie~algebras then their composite~$\psi \circ \varphi \colon \glie \to \klie$ is again a homomorphism of Lie~algebras.
    \item
      If~$\hlie$ is a Lie~subalgebra of~$\glie$ then the inclusion~$\hlie \to \glie$ is a homomorphism of Lie~algebras.
    \item
      Suppose that~$\glie$ and~$\hlie$ are both abelian.
      Then any~\linear{$\kf$} map~$\glie \to \hlie$ is already a homomorphism of Lie~algebras.
    \item
      For every element~$x$ of~$\glie$ let~$\ad(x)$\glsadd{adjoint representation} denote the map
      \[
        \ad(x)
        \colon
        \glie
        \to
        \glie \,,
        \qquad
        y
        \mapsto
        [x,y] \,.
        \index{adjoint representation}
      \]
      It follows from the bilinearity of the Lie~bracket of~$\glie$ that this describes a~\linear{$\kf$} map
      \[
        \ad
        \colon
        \glie
        \to
        \gllie(\glie) \,.
      \]
      It follows from the Jacobi identity that the map~$\ad$ is already a homomorphism of Lie~algebras.
      Indeed, we have for all element~$x$,~$y$,~$z$ of~$\glie$ that
      \begin{align*}
          \ad([x,y])(z)
          &=
          [[x,y], z]
          \\
          &=
          - [z, [x,y]]
          \\
          &=
          - [[z,x], y] - [x, [z,y]]
          \\
          &=
          [x, [y,z]] - [y, [x,z]]
          \\
          &=
          \ad(x)\ad(y)(z) - \ad(y)\ad(x)(z)
          \\
          &=
          [\ad(x), \ad(y)](z) \,,
      \end{align*}
      and therefore~$\ad([x,y]) = [ \ad(x), \ad(y) ]$.
    \item
      \label{algebra homomorphisms are lie algebra homomorphisms}
      If~$A$ and~$B$ are two associative~{\algebras{$\kf$}} then every homomorphism of~{\algebras{$\kf$}}
      \[
        \varphi \colon A \to B
      \]
      is also a homomorphism of Lie~algebras.
      Indeed, we have for all elements~$a$,~$b$ of~$A$ that
      \[
        \varphi([a,b])
        =
        \varphi(ab - ba)
        =
        \varphi(a) \varphi(b) - \varphi(b) \varphi(a)
        =
        [\varphi(a), \varphi(b)] \,.
      \]
    \item
      Let~$\glie$ be a Lie~algebra over an arbitary field~$\kf$.
      If~$\varphi \colon \sllie(2, \kf) \to \glie$ is a homomorphism of Lie~algebras then the images
      \[
        E \defined \varphi(e)  \,,
        \qquad
        H \defined \varphi(h)  \,,
        \qquad
        F \defined \varphi(f)
      \]
      satisfy the relations
      \begin{equation}
        \label{relations for sl2 tripel}
        [H, E] = 2E  \,,
        \qquad
        [H, F] = -2F  \,,
        \qquad
        [E, F] = H \,.
      \end{equation}
      On the other hand, every such triple~$(E', H', F')$ of elements of~$\glie$ satisfying the above relations (with~$X$ replaced by~$X'$ for~$X \in \{ E, H, F \}$) gives rise to a unique homomorphism of Lie~algebras~$\varphi' \colon \sllie(2, \kf) \to \glie$, which is given by
      \[
        \varphi'(e) = E' \,,
        \qquad
        \varphi'(h) = H' \,,
        \qquad
        \varphi'(f) = F' \,.
      \] 
      These two constructions are mutually inverse.
      We have thus constructed a bijection
      \begin{align*}
        \left\{
          \begin{tabular}{c}
            homomorphisms of \\
            Lie~algebras~$\phi \colon \sllie(2, \kf) \to \glie$
          \end{tabular}
        \right\}
        &\to
        \left\{
          \begin{tabular}{c}
            triples $(E, H, F)$ of elements of~$\glie$ \\
            which satisfy the relations~\eqref{relations for sl2 tripel}
          \end{tabular}
        \right\} \,,
        \\
        \varphi
        &\mapsto
        \bigl( \varphi(e), \varphi(h), \varphi(f) \bigr) \,.
      \end{align*}
  %     Such triples will play an important role later on.
  \end{enumerate}
\end{examples}


\begin{remark}
  It follows from part~\ref{identity is a homomorphism of lie algebras} and part~\ref{composite of homomorphisms of lie algebras} of \cref{homomorphisms of lie algebras} that~\liealgebras{$\kf$} form a category, which we will denote by
  \[
    \cLie{\kf} \,.
  \]
  The class of objects of~$\cLie{\kf}$ is the class of~\liealgebras{$\kf$}.
  For any two Lie~algebras~$\glie$,~$\hlie$ over~$\kf$ we have
  \[
    \Hom_{\cLie{\kf}}(\glie, \hlie)
    =
    \{
      \text{homomorphisms of Lie algebras~$\textstyle \glie \to \hlie$} 
    \} \,.
  \]
  The composition of morphisms in~$\cLie{\kf}$ is the usual composition of functions.
  The identity of an object in~$\cLie{\kf}$ is the usual identity function.

  We have a forgetful functor
  \[
    \cLie{\kf}
    \to
    \cVect{\kf}
  \]
  which assigns to every~\liealgebra{$\kf$} its underlying~\vectorspace{$\kf$}.
  It follows from part~\ref{algebra homomorphisms are lie algebra homomorphisms} of \cref{homomorphisms of lie algebras} that we also have a forgetful functor
  \[
    \cAlg{\kf}
    \to
    \cLie{\kf} \,.
  \]
  The forgetful functor~$\cAlg{\kf} \to \cVect{\kf}$ is the composite of the forgetful functors~$\cAlg{\kf} \to \cLie{\kf}$ and~$\cLie{\kf} \to \cVect{\kf}$.
\end{remark}


\begin{proposition}
  \label{inverse of homomorphism of lie algebras is again a homomorphism of lie algebras}
  Let~$\glie$ and~$\hlie$ be two~\liealgebras{$\kf$} and let~$\varphi$ be a bijective homomorphism of Lie~algebras from~$\glie$ to~$\hlie$.
  The set-theoretic inverse map~$\varphi^{-1}$ is a homomorphism of Lie~algebras from~$\hlie$ to~$\glie$.
\end{proposition}


\begin{proof}
  The inverse~$\varphi^{-1}$ is~\linear{$\kf$} because the original map~$\varphi$ is~\linear{$\kf$}.
  We also have
  \[
    \varphi^{-1}( [x,y] )
    =
    \varphi^{-1}( [ \varphi(\varphi^{-1}(x)), \varphi(\varphi^{-1}(y)) ] )
    =
    \varphi^{-1}( \varphi( [ \varphi^{-1}(x), \varphi^{-1}(y) ] ) )
    =
    [ \varphi^{-1}(x), \varphi^{-1}(y) ]
  \]
  for any two elements~$x$,~$y$ of~$\hlie$.
  This shows that~$\varphi^{-1}$ is a homomorphism of Lie~algebras.
\end{proof}


\begin{remark}
  We have the notion of an \defemph{isomorphism of Lie~algebras}\index{isomorphism!of Lie algebras} because \liealgebras{$\kf$} form a category.
  More explicitely, a homomorphism of Lie~algebras~$\varphi \colon \glie \to \hlie$ is an isomorphism if and only if there exists a homomorphism of Lie~algebras~$\psi \colon \hlie \to \glie$ with both~$\psi \circ \varphi = \id_{\glie}$ and~$\varphi \circ \psi = \id_{\hlie}$.

  We can now give a reinterpretation of \cref{inverse of homomorphism of lie algebras is again a homomorphism of lie algebras}:
  a homomorphism of Lie~algebras is an isomorphism of Lie~algebras if and only if it is bijective.
\end{remark}


\begin{example}[Classification of abelian Lie~algebras]
  Let~$\kf$ be any field.

  Two abelian~{\liealgebras{$\kf$}} are isomorphic (as Lie~algebras) if and only if they are isomorphic as~{\vectorspaces{$\kf$}}, because every vector space isomorphism between them is already an isomorphism of Lie~algebras.
  It follows that there exists up to isomorphism precisely one abelian~{\liealgebra{$\kf$}} of each dimension.
\end{example}


\begin{example}[Classification of one- and {\twodimensional} Lie~algebras]
  Let~$\kf$ be any field.
  We can classify the {\onedimensional} and {\twodimensional}~\liealgebras{$\kf$} up to isomorphism by hand.
  
  If~$\glie$ is a~{\onedimensional}~\liealgebra{$\kf$} then the Lie~bracket of~$\glie$ is zero because it is alternating, so~$\glie$ is abelian.
  It follows that there exists up to isomorphism precisely one one-dimensional~\liealgebra{$\kf$}, namely the abelian one.
 
  For the {\twodimensional} Lie algebras we have already seen that there exists up to isomorphism precisely one abelian one.
  We will now also show that there exists up to isomorphism precisely one non-abelian one.
  
  Suppose first that~$\glie$ is any {\twodimensional}, non-abelian~\liealgebra{$\kf$}.
  Let~$\tilde{x}$,~$\tilde{y}$ be a basis of~$\glie$.
  We have~$[\glie, \glie] \neq 0$ because~$\glie$ is non-abelian, and~$[\glie, \glie]$ is spanned by the single element~$[\tilde{x}, \tilde{y}]$ because the Lie~bracket is alternating.
  Hence~$[\glie, \glie] = \gen{ [\tilde{x}, \tilde{y}] }_{\kf}$ with~$[\tilde{x}, \tilde{y}] \neq 0$.

  Let~$x \defined [\tilde{x},\tilde{y}]$ and let~$y$ be some element of~$\glie$ that is linearly independent to~$x$.
  Then~$x$,~$y$ is again a basis of~$\glie$.
  We have~$[x,y] \neq 0$ because~$\glie$ is non-abelian, and~$[x,y] \in [\glie, \glie] = \kf x$.
  By rescaling the basis vector~$y$ we may therefore assume that~$[x,y] = x$.

  We have shown that any two-dimensional, non-abelian Lie~algebra admits a basis~$x$,~$y$ on which its Lie~bracket has a specific form, namely
  \[
    [x,y] = x \,.
  \]
  This shows that there is up to isomorphism at most one such Lie~algebra.
 
  To show the existence of a two-dimensional, non-abelian Lie~algebra we consider the two matrices
  \[
    x
    \defined
    \begin{pmatrix}
      0 & 1 \\
      0 & 0
    \end{pmatrix}
    =
    E_{12}  \,,
    \qquad
    y
    \defined
    \begin{pmatrix}
      0 & 0 \\
      0 & 1
    \end{pmatrix}
    =
    E_{22}  \,.
  \]
  A direct calculation shows that
  \[
    [x,y]
    =
    [E_{12}, E_{22}]
    =
    E_{12} E_{22} - E_{22} E_{12}
    =
    E_{12} - 0
    =
    E_{12}
    =
    x \,.
  \]
  The linear subspace of~$\gllie(2, \kf)$ which is spanned by~$x$,~$y$ is therefore a Lie~subalgebra of~$\gllie(2, \kf)$.
  It is two-dimensional because the two matrices~$x$,~$y$ are linearly independent, and it is non-abelian because~$x$,~$y$ do not commute with each other (since~$[x,y] \neq 0$).

  We have altogether shown that up to isomorphism there are precisely two {\twodimensional} Lie~algebras over~$\kf$.
  One of them is abelian, and the other one admits a basis~$x$,~$y$ with~$[x,y] = x$.
\end{example}

\begin{example}[Three-dimensional Lie~algebras]
  \label{infinitely many three-dimensional lie algebras}
  The number of three-dimensional Lie~algebras over a field~$\kf$ depends on the cardinality of~$\kf$.

  If the field~$\kf$ is finite then there exist up to isomorphism only finitely many {\threedimensional} Lie~algebras:
  Every such Lie~algebra is isomorphic to the vector space~$\kf^3$ together with a suitable Lie~bracket~$[-,-] \colon \kf^3 \times \kf^3 \to \kf^3$.
  But there exist only finitely many Lie~brackets on~$\kf^3$ because~$\kf^3$ is finite.

  If the field~$\kf$ is infinite then there exist up to isomorphism inifinitely many Lie~algebras over~$\kf$.
  We show this by constructing a family of Lie~algebras~$\glie_\tau$ with~$\tau \in \kf$, such that~$\glie_\tau$ and~$\glie_\upsilon$ are isomorphic if and only if~$\tau = \upsilon$ or~$\tau \upsilon = 1$.

  The Lie~algebra~$\glie_\tau$ is given by a the vector space with basis~$t$,~$x$,~$y$ together with the Lie~bracket given by
  \[
    [t, x] = x \,,
    \quad
    [t, y] = \tau y \,,
    \quad
    [x, y] = 0 \,.
  \]
  Such a Lie~algebra exists since it can be realized as the Lie~subalgebra of~$\gllie(3, \kf)$ which is spanned by the three matrices
  \[
    t
    =
    \begin{pmatrix}
      0 &   &       \\
        & 1 &       \\
        &   & \tau
    \end{pmatrix} \,,
    \qquad
    x
    =
    E_{21} 
    =
    \begin{pmatrix}
      0 & 0 & 0 \\
      1 & 0 & 0 \\
      0 & 0 & 0
    \end{pmatrix} \,,
    \qquad
    y
    =
    E_{31}
    =
    \begin{pmatrix}
      0 & 0 & 0 \\
      0 & 0 & 0 \\
      1 & 0 & 0
    \end{pmatrix} \,.
  \]

  We first note that for~$\tau \neq 0$ we can replace the basis~$t$,~$x$,~$y$ of~$\glie_\tau$ by the slighty rescaled basis~$t'$,~$x'$,~$y'$ that is given by
  \[
    t' \defined \frac{t}{\tau} \,,
    \quad
    x' \defined x \,,
    \quad
    y' \defined y \,.
  \]
  The Lie~bracket of~$\glie_\tau$ is on this rescaled basis given by the relations
  \[
    [t', x'] = \frac{1}{\tau} x' \,,
    \quad
    [t', y'] = y' \,,
    \quad
    [y', x'] = 0 \,.
  \]
  We find from these relations that the unique linear map~$\varphi \colon \glie_{\tau} \to \glie_{\tau^{-1}}$ given by
  \[
    \varphi(t') = t \,,
    \quad
    \varphi(x') = y \,,
    \quad
    \varphi(y') = x
  \]
  is an isomorphism of Lie~algebras.
  This shows that~$\glie_{\tau}$ is isomorphic to~$\glie_{\upsilon}$ if~$\upsilon = \tau^{-1}$, i.e.\ if~$\tau \upsilon = 1$.
  
  Let on the other hand~$\glie \defined \glie_\tau$ for some~$\tau \in \glie$.
  The commutator space~$[\glie, \glie]$ is given by~$\gen{x, y}$ if~$\tau \neq 0$ and by~$\gen{x}$ if~$\tau = 0$.
  We can hence distinguish the case~$\tau = 0$ from all the other cases by considering the dimension of~$[\glie, \glie]$.

  To further distinguish the cases~$\tau \neq 0$ let~$t'$ be some element of~$\glie$ that does not belong to the commutator space~$[\glie, \glie] = \gen{x,y}$.
  Then~$t' = \alpha t + \beta x + \gamma y$ for some coefficients~$\alpha, \beta, \gamma \in \kf$ with~$\alpha \neq 0$.
  We consider the endomorphism
  \[
    \ad(t')
    =
    [t', -]
    \colon
    \glie
    \to
    \glie \,.
  \]
  With respect to the basis~$t$,~$x$,~$y$ of~$\glie$ we have
  \[
    \ad(t)
    \equiv
    \begin{pmatrix}
      0 & 0 & 0     \\
      0 & 1 & 0     \\
      0 & 0 & \tau
    \end{pmatrix} \,,
    \quad
    \ad(x)
    \equiv
    \begin{pmatrix*}[r]
       0  & 0 & 0 \\
      -1  & 0 & 0 \\
       0  & 0 & 0
    \end{pmatrix*} \,,
    \quad
    \ad(y)
    \equiv
    \begin{pmatrix}
      0     & 0 & 0 \\
      0     & 0 & 0 \\
      -\tau & 0 & 0
    \end{pmatrix}
  \]
  and therefore
  \begin{align*}
    \ad(t')
    &=
    \ad(\alpha t + \beta x + \gamma y)
    \\
    &=
    \alpha \ad(t) + \beta \ad(x) + \gamma \ad(y)
    \\
    &\equiv
    \alpha
    \begin{pmatrix}
      0 & 0 & 0     \\
      0 & 1 & 0     \\
      0 & 0 & \tau
    \end{pmatrix}
    +
    \beta
    \begin{pmatrix*}[r]
       0  & 0 & 0 \\
      -1  & 0 & 0 \\
       0  & 0 & 0
    \end{pmatrix*}
    +
    \gamma
    \begin{pmatrix}
      0     & 0 & 0 \\
      0     & 0 & 0 \\
      -\tau & 0 & 0
    \end{pmatrix}
    \\
    &=
    \begin{pmatrix}
       0          & 0       & 0           \\
      -\beta      & \alpha  & 0           \\
      -\gamma\tau & 0       & \alpha \tau
    \end{pmatrix} \,.
  \end{align*}
  The eigenvalues of~$\ad(t')$ are thus given by~$0$,~$\alpha$,~$\alpha \tau$.
  We find in particular that the quotients of the two non-zero eigenvalues of~$\ad(t')$ are given by~$\tau$ and~$1 / \tau$.

  The above calculations shows how the set~$\{ \tau, 1 / \tau \}$ can be reconstructed from the Lie~algebra~$\glie = \glie_\tau$:
  the values~$\tau$ and~$1/\tau$ are precisely the quotients of the nonzero eigenvalues of~$\ad(t')$ where~$t'$ is any element of~$\glie$ not contained in~$[\glie, \glie]$.
  It follows that for~$\glie_\tau$ and~$\glie_\upsilon$ to be isomorphic we need that~$\{ \tau , 1/\tau \} = \{ \upsilon, 1/\upsilon \}$.
  This means that~$\tau = \upsilon$ or~$\tau = 1 / \upsilon$, i.e.~$\tau \upsilon = 1$.
\end{example}



\subsection{Homomorphisms Involving Products of Lie~Algebras}


\begin{proposition}
  \label{products of lie algebras}
  Let~$\glie_\lambda$ with~$\lambda \in \Lambda$ be a family of~\liealgebras{$\kf$}.
  \begin{enumerate}
    \item
      The canonical projection
      \[
       \pi_\mu
       \colon
       \prod_{\lambda \in \Lambda} \glie_\lambda
       \to
       \glie_\mu \,,
       \quad
       ( x_\lambda )_\lambda
       \mapsto
       x_\mu
      \]
      is for every index~$\mu \in \Lambda$ a homomorphism of Lie~algebras.
    \item
      Let~$\hlie$ be another~\liealgebra{$\kf$}.
      A map~$\varphi \colon \hlie \to \prod_{\lambda \in \Lambda} \glie$ is a homomorphism of Lie algebras if and only if it is a homomorphisms of Lie algebras in each coordinate, i.e.\ if and only if for every index~$\mu \in \Lambda$ the composite~$\pi_\mu \circ \varphi \colon \hlie \to \glie_\mu$ is a homomorphism of Lie~algebras.
    \qed
  \end{enumerate}
\end{proposition}


\begin{proof}
  \leavevmode
  \begin{enumerate}
    \item
      This holds because the Lie~algebra structure on~$\prod_{\lambda \in \Lambda} \glie_{\lambda}$ is defined componentwise.
    \item
      If~$\varphi$ is a homomorphism of Lie~algebras then the composite~$\pi_\mu \circ \varphi$ is for every index~$\mu \in \Lambda$ again a homomorphism of Lie~algebras because the projection~$\pi_\mu$ is a homomorphism of Lie~algebras.

      Suppose on the other hand that each composite~$\pi_\lambda \circ \varphi$ is a homomorphism of Lie~algebras.
      We have for any two element~$x$,~$y$ of~$\hlie$ and for every index~$\mu \in \Lambda$ the equalities
      \begin{align*}
        \pi_\mu\bigl( \varphi( [ x, y ] ) \bigr)
        &=
        (\pi_\mu \circ \varphi)( [x, y] )
        \\
        &=
        [ (\pi_\mu \circ \varphi)(x), (\pi_\mu \circ \varphi)(y) ]
        \\
        &=
        [ \pi_\mu( \varphi(x) ), \pi_\mu( \varphi(y) ) ]
        \\
        &=
        \pi_\mu\bigl( [\varphi(x), \varphi(y)] \bigr) \,,
      \end{align*}
      and thus altogether
      \[
        \varphi( [x,y] ) = [ \varphi(x), \varphi(y) ] \,.
      \]
      This shows that the linear map~$\varphi$ is a homomorphism of Lie~algebras.
    \qedhere
  \end{enumerate}
\end{proof}


\begin{remark}
  \Cref{products of lie algebras} asserts that the category~$\cLie{\kf}$ has all (small) products.

  Given three~\liealgebras{$\kf$}~$\glie$,~$\hlie_1$,~$\hlie_2$ and two homomorphisms of Lie~algebras
  \[
    f_1 \colon \glie \to \hlie_1 \,,
    \quad
    f_2 \colon \glie \to \hlie_2
  \]
  one can also verify that the set
  \[
    \{
      x \in \glie
    \suchthat
      f_1(x) = f_2(x)
    \}
  \]
  is a Lie~subalgebra of~$\glie$.
  This then shows that the category~$\cLie{\kf}$ has (binary) equalizers.

  Together this shows that the category~$\cLie{\kf}$ has all (small) limits, i.e.\ it is complete.
  We will see later that~$\cLie{\kf}$ is also cocomplete.
  % TODO: Add a reference to the position, once it is written.
\end{remark}


\subsection{Homomorphisms Involving Direct Sums of Lie~Algebras}


\begin{example}[Homomorphisms out of direct sums]
  \label{homomorphism out of direct sum}
  Let~$\glie_\lambda$ with~$\lambda \in \Lambda$ be a family of~\liealgebras{$\kf$}.
  For every index~$\mu \in \Lambda$ we have a~\linear{$\kf$}, injective map
  \[
    \iota_\mu
    \colon
    \glie_\mu
    \to
    \bigoplus_{\lambda \in \Lambda} \glie_\lambda \,,
  \]
  which is given by
  \[
    \iota_\mu(x)
    =
    (y_\lambda)_\lambda
    \quad\text{with}\quad
    y_\lambda
    =
    \begin{cases*}
      x & if~$\lambda = \mu$, \\
      0 & otherwise,
    \end{cases*}
    \qquad
    \text{for every~$\lambda \in \Lambda$.}
  \]
  These inclusions are homomorphisms of Lie~algebras.
  We can therefore regard every Lie~algebra~$\glie_\mu$ as a Lie~subalgebra of the direct sum~$\bigoplus_{\lambda \in \Lambda} \glie_\lambda$.

  Let~$\hlie$ be another~\liealgebra{$\kf$}.
  We have a {\onetoonetext} correspondence
  \begin{align*}
    \left\{
      \begin{tabular}{c}
        \linear{$\kf$} maps \\
        $\varphi \colon \bigoplus_{\lambda \in \Lambda} \glie_\lambda \to \hlie$
      \end{tabular}
    \right\}
    &\onetoone
    \left\{
      (\varphi_\lambda)_\lambda
    \suchthat*
      \begin{tabular}{@{}c@{}}
        \linear{$\kf$} maps\\
        $\varphi_\lambda \colon \glie_\lambda \to \hlie$
      \end{tabular}
    \right\}
  \intertext{which is given by}
    \SwapAboveDisplaySkip
    \varphi
    &\mapsto
    (\varphi \circ \iota_\lambda)_\lambda \,,
    \\
    \biggl(
      (x_\lambda)_\lambda
      \mapsto
      \sum_{\lambda \in \Lambda}
      \varphi_\lambda(x_\lambda)
    \biggr)
    &\mapsfrom
    (\varphi_\lambda)_\lambda \,.
  \end{align*}
  We may think about the composite~$\varphi \circ \iota_\lambda$ as the restriction of~$\varphi$ to~$\glie_\lambda$, when~$\glie_\lambda$ is regarded as a Lie~subalgebra of~$\glie$ via the inclusion~$\iota_\lambda$.

  If a~\linear{$\kf$} map~$\varphi$ from~$\bigoplus_{\lambda \in \Lambda} \glie_\lambda$ to~$\hlie$ is a homomorphism of Lie~algebras, then each composite~$\varphi \circ \iota_\mu$ is a homomorphism of Lie~algebras from~$\glie_\mu$ to~$\hlie$, because the inclusions~$\iota_\mu$ are homomorphism of Lie~algebras.
  We note that the Lie~algebras~$\glie_\mu$ -- when regarded as Lie~subalgebras of~$\bigoplus_{\lambda \in \Lambda} \glie_\lambda$ via the inclusions~$\iota_\mu$ -- commute with each other, in the sense that
  \[
    [ \iota_\mu(x), \iota_\kappa(y) ] = 0
    \qquad
    \text{for all~$\mu \neq \kappa$ and~$x \in \glie_\mu$,~$y \in \glie_\kappa$.}
  \]
  It follows that the images of the composites~$\varphi \circ \iota_\lambda$ commute with each other, because
  \[
    [
      (\varphi \circ \iota_\mu)(x),
      (\varphi \circ \iota_\kappa)(y)
    ]
    =
    \varphi( [\iota_\mu(x), \iota_\kappa(y)] )
    =
    \varphi(0)
    =
    0
  \]
  for all indices~$\mu \neq \kappa$ and elements~$x \in \glie_\mu$,~$y \in \glie_\kappa$.

  Suppose on the other hand that~$\varphi_\lambda \colon \glie_\lambda \to \hlie$ with~$\lambda \in \Lambda$ are homomorphisms of Lie~algebras whose images commute with each other.
  Let~$\varphi$ be the~\linear{$\kf$} from~$\bigoplus_{\lambda \in \Lambda} \glie_\lambda$ to~$\hlie$ corresponding to the maps~$\varphi_\lambda$.
  More explicitely, we have
  \[
    \varphi( (x_\lambda)_\lambda )
    =
    \sum_{\lambda \in \Lambda}
    \varphi_\lambda(x_\lambda)
    \qquad
    \text{for every~$(x_\lambda)_\lambda \in \bigoplus_{\lambda \in \Lambda} \glie_\lambda$.}
  \]
  Then the map~$\varphi$ is again a homomorphism of Lie~algebras.
  Indeed, let~$(x_\lambda)_\lambda$ and~$(y_\lambda)_\lambda$ be two elements of~$\bigoplus_{\lambda \in \Lambda} \glie_\lambda$.
  We have on the one hand
  \begin{align*}
    [
      \varphi( (x_\lambda)_\lambda ),
      \varphi( (y_\lambda)_\lambda )
    ]
    &=
    \Biggl[
      \sum_{\lambda \in \Lambda}
      \varphi_\lambda( x_\lambda ) ,
      \sum_{\mu \in \Lambda}
      \varphi_\mu( y_\mu )
    \Biggr]
    \\
    &=
    \sum_{\lambda, \mu \in \Lambda}
    \underbrace{
      [ \varphi_\lambda( x_\lambda ) , \varphi_\mu( y_\mu ) ]
    }_{
      \text{$= 0$ for~$\lambda \neq \mu$}
    }
    \\
    &=
    \sum_{\lambda \in \Lambda} [ \varphi_\lambda( x_\lambda ), \varphi_\lambda( y_\lambda ) ]
    \\
    &=
    \sum_{\lambda \in \Lambda} \varphi_\lambda( [ x_\lambda , y_\lambda ] ) \,,
  \end{align*}
  and on the other hand we have
  \[
    \varphi\bigl( [ (x_\lambda)_\lambda, (y_\lambda)_\lambda ] \bigr)
    =
    \varphi\bigl( ( [x_\lambda, y_\lambda] )_\lambda \bigr)
    =
    \sum_{\lambda_1}^n \varphi_\lambda( [ x_\lambda, y_\lambda ] ) \,.
  \]
  This shows that~$\varphi$ is indeed a homomorphism of Lie~algebras.

  It follows from the above assertions that the previous {\onetoonetext} correspondce restricts to a {\onetoonetext} correspondence
  \begin{align*}
    \left\{
      \begin{tabular}{c}
        Lie~algebra \\
        homomorphisms \\
        $\varphi \colon \bigoplus_{\lambda \in \Lambda} \glie_\lambda \to \hlie$
      \end{tabular}
    \right\}
    &\onetoone
    \left\{
      (\varphi_\lambda)_\lambda
    \suchthat*
      \begin{tabular}{@{}c@{}}
        Lie~algebra homomorphisms \\
        $\varphi_\lambda \colon \glie_\lambda \to \hlie$ whose images \\
        commute with each other
      \end{tabular}
    \right\} \,.
  \end{align*}
\end{example}


\begin{example}
  \label{direct sum of ideals}
%  If~$I$ and~$J$ are two Lie~algebra over the same field~$\kf$ then their product~$I \times J$ contains the linear subspaces~$I' \defined I \times 0$ and~$J' \defined 0 \times J$ as ideals.
%  These ideals are isomorphic to~$I$ and~$J$ as Lie~algebras via the isomorphisms
%  \begin{alignat*}{2}
%    I
%    &\to
%    I'  \,,
%    &
%    \quad
%    x
%    &\mapsto
%    (x,0) \,,
%    \\
%    J
%    &\to
%    J'  \,,
%    &
%    \quad
%    y
%    &\mapsto
%    (0,y) \,.
%  \end{alignat*}
  Let~$\glie$ be a~{\liealgebra{$\kf$}} and let~$I_\lambda$ with~$\lambda \in \Lambda$ be a family of ideals of~$\glie$ such that
  \[
    \glie
    =
    \bigoplus_{\lambda \in \Lambda}
    I_\lambda
  \]
  as vector spaces.
  It then follows for any two distinct indices~$\mu$,~$\kappa$ that~$[I_\mu, I_\kappa] = 0$.
  Indeed, we have for all~$x \in I_\mu$ and~$y \in I_\kappa$ that~$[x,y] \in I_\mu$ because~$I_\mu$ is an ideal, but also~$[x,y] \in I_\kappa$ because~$I_\kappa$ is an ideal.
  Hence~$[x,y]$ is contained in~$I_\mu \cap I_\kappa$.
  But~$I_\mu \cap I_\kappa = 0$, so~$[x,y] = 0$.
  The isomorphism of vector spaces
  \[
    \bigoplus_{\lambda \in \Lambda}
    I_\lambda
    \to
    \glie \,,
    \quad
    (x_\lambda)_\lambda
    \mapsto
    \sum_{\lambda \in \Lambda} x_\lambda
  \]
  is therefore an isomorphism of Lie~algebras.

  Let~$\hlie$ be another~\liealgebra{$\kf$}.
  It now follows from \cref{homomorphism out of direct sum} that a homomorphism of Lie~algebras~$\varphi$ from~$\glie$ to~$\hlie$ is \enquote{the same} as a collection of Lie~algebra homomorphisms~$\varphi_\lambda$ from~$I_\lambda$ to~$\hlie_\lambda$ with~$\lambda \in \Lambda$, such that the images of~$\varphi_\mu$ and~$\varphi_\kappa$ commute for any two distinct indices~$\mu$ and~$\kappa$.
\end{example}


\begin{definition}
  In the situation of \cref{direct sum of ideals} the Lie~algebra~$\glie$ is the \defemph{internal direct sum} the Lie~subalgebras~$I_\lambda$.
\end{definition}


\subsection{Homomorphisms Involving Quotients of Lie~Algebras}


\begin{theorem}[Homomorphism theorem]
  Let~$\glie$ be a Lie~algebra and let~$I$ be an ideal in~$\glie$.
  \begin{enumerate}
    \item
      The canonical projection
      \[
        \pi
        \colon
        \glie
        \to
        \glie/I \,,
        \quad
        x
        \mapsto
        \class{x}
      \]
      is a homomorphism of Lie~algebras.
  \end{enumerate}
  Let~$\hlie$ be another Lie~algebra.
  \begin{enumerate}[resume*]
    \item
      Let~$\psi \colon \glie/I \to \hlie$ be a homomorphism of Lie~algebras.
      The composite~$\psi \circ \pi \colon \glie \to \hlie$ is a homomorphism of Lie~algebras with~$I \subseteq \ker(\psi \circ \pi)$.
    \item
      Let~$\varphi \colon \glie \to \hlie$ be a homomorphism of Lie~algebras.
      The homomorphism~$\varphi$ factors through a homomorphism of Lie~algebras~$\psi \colon \glie/I \to \hlie$ that makes the diagram
      \[
        \begin{tikzcd}
          \glie
          \arrow{r}[above]{\varphi}
          \arrow{d}[left]{\pi}
          &
          \hlie
          \\
          \glie/I
          \arrow[dashed]{ur}[below right]{\psi}
          &
          {}
        \end{tikzcd}
      \]
      commute if and only if~$I \subseteq \ker(\varphi)$.
      The homomorphism~$\psi$ is unique and it is given by
      \[
        \psi( \class{x} )
        =
        \varphi(x)
        \qquad
        \text{for every~$x \in \glie$.}
      \]
      Its image and kernel are given by~$\im(\psi) = \im(\varphi)$ and~$\ker(\psi) = \ker(\varphi)/I$.
    \qed
  \end{enumerate}
\end{theorem}


%TODO: Add a proof.


\begin{corollary}[Isomorphism theorems]
  \index{isomorphism theorems!for Lie algebras}
  Let~$\glie$ be a Lie~algebra.
  \begin{enumerate}
    \item
      Let~$\varphi \colon \glie \to \hlie$ be a homomorphism of Lie~algebras.
      The homomorphism~$\varphi$ induces a well-defined isomorphism of Lie~algebras
      \[
        \psi
        \colon
        \glie / {\ker(\varphi)}
        \to
        \im(\varphi)  \,,
        \quad
        \class{x}
        \mapsto
        \class{\varphi(x)} \,.
      \]
    \item
      If~$I$ and~$J$ are two ideals in~$\glie$ with~$I \subseteq J$ then the quotient~$J/I$ is an ideal in~$\glie/I$ and the natural isomorphism of vector spaces
      \[
        (\glie/I) / (J/I)
        \to
        \glie/I \,,
        \quad
        \class{ \,\class{x}\, }
        \mapsto
        \class{x}
      \]
      is already a natural isomorphism of Lie~algebras.
    \item
      Let~$\hlie$ is a Lie~subalgebra of~$\glie$ and let~$I$ is an ideal in~$\glie$.
      Then the sume~$\hlie + I$ is a Lie~subalgebra of~$\glie$ that contains the ideal~$I$, and the intersection~$\hlie \cap I$ is an ideal in~$\hlie$.
      The natural isomorphism of vector spaces
      \begin{gather*}
        \hlie/(\hlie \cap I)
        \to
        (\hlie + I)/I \,,
        \quad
        \class{x}
        \mapsto
        \class{x}
      \end{gather*}
      is already a natural isomorphism of Lie~algebras.
    \qed
  \end{enumerate}
\end{corollary}


\begin{corollary}
  It holds for any Lie~algebra~$\glie$ that~$\glie/{\centerlie(\glie)} \cong \ad(\glie)$, with~$\ad(\glie)$ being a Lie~subalgebra~$\gllie(\glie)$.
\end{corollary}


\begin{proof}
  The center~$\centerlie(\glie)$ is precisely the kernel of the Lie~algebra homomorphism~$\ad \colon \glie \to \gllie(\glie)$.
  The assertion does therefore follow from the first isomorphism theorem.
\end{proof}


\begin{remark}
  The Lie~subalgebra~$\ad(\glie)$ of~$\gllie(\glie)$ is the canonical way of associating to the Lie~algebra~$\glie$ a linear Lie~algebra.
  We will later see that questions about the original Lie~algebra~$\glie$ can sometimes be reduced to the case of the linear Lie~algebra~$\ad(\glie)$, for which we can then apply linear algebra.
\end{remark}


\begin{example}
  Let~$\glie$ be a~\liealgebra{$\kf$} and let~$\hlie$ be an abelian~\liealgebra{$\kf$}.
  If~$\varphi \colon \glie \to \hlie$ is any homomorphism of Lie~algebras then the image of~$\varphi$ is a Lie~subalgebra of~$\glie$, which is then again abelian.
  It follows from the isomorphism~$\im(\varphi) \cong \glie / {\ker(\varphi)}$ that the quotiont~$\glie / {\ker(\varphi)}$ is abelian.
  The ideal~$\ker(\varphi)$ must therefore contain the commutator ideal~$[\glie, \glie]$.
  The homomorphism~$\varphi$ does therefore factor through a homomorphism~$\glie / [\glie, \glie] \to \hlie$, and this construction gives a bijection
  \[
    \left\{
      \begin{tabular}{c}
        homomorphisms of \\
        Lie~algebras~$\glie \to \hlie$
      \end{tabular}
    \right\}
    \onetoone
    \left\{
      \begin{tabular}{c}
        homomorphisms of \\
        Lie~algebras $\glie/[\glie, \glie] \to \hlie$
      \end{tabular}
    \right\} \,.
  \]
\end{example}


%\begin{lemma}
%  \label{homomorphisms respect commutators of sets}
%  If~$\varphi \colon \glie \to \hlie$ is a homomorphism of Lie~algebras then
%  \[
%    \varphi([X,Y])
%    =
%    [\varphi(X), \varphi(Y)]
%  \]
%  for any two subsets~$X$,~$Y$ of~$\glie$.
%  \qed
%\end{lemma}


\begin{lemma}
  Let~$\varphi \colon \glie \to \hlie$ be a homomorphism of Lie~algebras.
  \begin{enumerate}
    \item
      If~$\klie$ is a Lie~subalgebra of~$\hlie$ then the preimage~$\varphi^{-1}(\klie)$ is a Lie~subalgebra of~$\glie$.
    \item
      If~$\klie$ is a Lie~subalgebra of~$\glie$ then the image~$\varphi(\klie)$ is a Lie~subalgebra of~$\hlie$.
    \item
      If~$I$ is an ideal of~$\hlie$ then the preimage~$\varphi^{-1}(I)$ is an ideal of~$\glie$.
    \item
      If the homomorphism~$\varphi$ is surjective and~$I$ is an ideal of~$\glie$ then the image~$\varphi(I)$ is an ideal of~$\hlie$.
    \qed
  \end{enumerate}
\end{lemma}


\begin{proposition}[Correspondence theorem]
  \label{correspondence theorem!for Lie algebras}
  Let~$I$ be an ideal of a Lie~algebra~$\glie$ and let
  \[
    \pi \colon \glie \to \glie / I
  \] denote the canonical projection.
  If~$\hlie$ is a Lie~subalgebra of~$\glie$ that contains the ideal~$I$ then quotient~$\hlie/I$ is a Lie~subalgebra of of~$\glie/I$.
  This construction induces a {\onetoonetext} correspondence
  \begin{align*}
    \{ \text{Lie~subalgebras~$\hlie$ of~$\glie$ containing~$I$} \}
    &\longleftrightarrow
    \{ \text{Lie~subalgebras~$\klie$ of~$\glie/I$} \} \,,
    \\
    \hlie
    &\mapsto
    \hlie/I \,,
    \\
    \pi^{-1}(\klie)
    &\mapsfrom
    \klie \,.
  \end{align*}
  This correspondence restricts to a {\onetoonetext} correspondence
  \begin{align*}
    \{ \text{Lie~ideals~$J$ in~$\glie$} \}
    &\longleftrightarrow
    \{ \text{Lie~ideals~$K$ in~$\glie/I$} \} \,,
    \\
    J
    &\mapsto
    J/I \,,
    \\
    \pi^{-1}(K)
    &\mapsfrom
    K \,.
  \end{align*}
  If~$J$ is an ideal in~$\glie$ containing~$I$ then for the associated ideal~$J/I$ in~$\glie/I$ the quotients~$(\glie/I)/(J/I)$ and~$\glie/J$ stay isomorphic via the third isomorphism theorem.
  \qed
\end{proposition}



\subsection{Antihomomorphisms}


% 
% \begin{lemma}
%   Let~$\glie$ be a Lie~algebra.
%   Every element~$x \in \glie$ can be regarded as an element~$x^{\op} \in \glie^{\op}$, resulting in a map~$i \colon \glie \to \glie^{\op}$.
% \end{lemma}


\begin{recall}
  Let~$A$ and~$B$ be two~\enquote{\algebras{$\kf$}}.
  A map~$\varphi \colon A \to B$ is an \defemph{anti-homomorphism}\index{anti-homomorphism!of $k$-algebras} if it is~{\linear{$\kf$}} and satisfies the condition
  \[
    \varphi(ab)
    =
    \varphi(b) \varphi(a)
    \qquad
    \text{for all~$a, b \in A$.}
  \]
  It holds that a map from~$A$ to~$B$ is an anti-homomorphism if any only if it is a homomorphism from~$A^{\op}$ to~$B$, if and only if it is a homomorphism from~$A$ to~$B^{\op}$.

  An anti-homomorphism~$\varphi$ from~$A$ to~$B$ is an \defemph{anti-isomorphism}\index{anti-isomorphism!of $k$-algebras} if there exists another anti-homomorphism~$\psi$ from~$B$ to~$A$ with both~$\psi \circ \varphi = \id_A$ and~$\varphi \circ \psi = \id_B$.
  It holds that a map from~$A$ to~$B$ is an anti-isomorphism if and only if it is an isomorphism from~$A^{\op}$ to~$B$, if and only if it is an isomorphism from~$A$ to~$B^{\op}$, if and only if it is bijective.
\end{recall}


\begin{definition}
  A map~$\varphi \colon \glie \to \hlie$ between~{\liealgebras{$\kf$}}~$\glie$ and~$\hlie$ is an \defemph{anti-homomorphism} of~{\liealgebras{$\kf$}} if it is~{\linear{$\kf$}} and satisfies the condition
  \[
    \varphi([x,y]) = [\varphi(y), \varphi(x)]
    \qquad
    \text{for all~$x, y \in \glie$.}
  \]
\end{definition}


\begin{proposition}
  \label{antihomomorphisms correspond to opposite homomorphisms}
  For a map~$\varphi \colon \glie \to \hlie$ between two~{\liealgebras{$\kf$}}~$\glie$ and~$\hlie$ the following four conditions are equivalent:
  \begin{equivalenceslist}
    \item
      $\varphi$ is an anti-homomorphism of Lie~algebras from~$\glie$ to~$\hlie$.
    \item
      $\varphi$ is a homomorphism of Lie~algebras from~$\glie^{\op}$ to~$\hlie$.
    \item
      $\varphi$ is a homomorphism of Lie~algebras from~$\glie$ to~$\hlie^{\op}$.
    \item
      $\varphi$ is an anti-homomorphism of Lie~algebras from~$\glie^{\op}$ to~$\hlie^{\op}$.
      \qed
  \end{equivalenceslist}
\end{proposition}


\begin{corollary}
  \label{onetoone correspondence for antihomomorphism}
  We have for any two~\liealgebras{$\kf$}~$\glie$ and~$\hlie$ a {\onetoonetext} correspondence given by
  \begin{align*}
    \left\{
      \begin{tabular}{c}
        anti-homomorphisms \\
        of Lie~algebras~$\glie \to \hlie$
      \end{tabular} 
    \right\}
    &\onetoone
    \left\{
      \begin{tabular}{c}
        homomorphisms \\
        of Lie~algebras~$\glie^{\op} \to \hlie$
      \end{tabular}
    \right\} \,,
    \\
    \varphi
    &\onetoone
    \varphi \,.
  \end{align*}
  \qed
\end{corollary}


\begin{proposition}
  \label{lie algebra isomorphic to its opposite}
  Every Lie~algebra~$\glie$ is isomorphic to its opposite Lie~algebra~$\glie^{\op}$ via the map
  \[
    \glie
    \to
    \glie^{\op}\,,
    \quad
    x
    \mapsto
    -x^{\op}  \,.
  \]
\end{proposition}


\begin{proof}
  The map~$\varphi \colon \glie \to \glie^{\op}$,~$x \mapsto -x^{\op}$ is linear, and it is an anti-homomorphism of Lie~algebras because
  \[
    \varphi([x, y])
    =
    -[x,y]^{\op}
    =
    [ x^{\op}, y^{\op} ]
    =
    [ -x^{\op}, -y^{\op} ]
    =
    [ \varphi(x), \varphi(y) ]
  \]
  for all~$x, y \in \glie$.
  It is also bijective, and thus an isomorphism of Lie~algebras.
\end{proof}


\begin{corollary}
  We have for any two~\liealgebras{$\kf$}~$\glie$ and~$\hlie$ a {\onetoonetext} correspondence given by
  \begin{align*}
    \left\{
      \begin{tabular}{c}
        anti-homomorphisms \\
        of Lie~algebras~$\glie \to \hlie$
      \end{tabular} 
    \right\}
    &\onetoone
    \left\{
      \begin{tabular}{c}
        homomorphisms \\
        of Lie~algebras~$\glie \to \hlie$
      \end{tabular}
    \right\} \,,
    \\
    \varphi
    &\mapsto
    -\varphi \,,
    \\
    -\varphi
    &\mapsfrom
    \varphi \,.
  \end{align*}
\end{corollary}


\begin{proof}
  This is a combination of \cref{onetoone correspondence for antihomomorphism} and \cref{lie algebra isomorphic to its opposite}.
\end{proof}





\section{Derivations}


\begin{definition}
  Let~$A$ be a~\enquote{\algebra{$\kf$}}.
  A \defemph{derivation}\index{derivation} of~$A$ is a~{\linear{$\kf$}} map~$\delta \colon A \to A$ such that
  \[
    \delta(ab)
    =
    \delta(a) b + a \delta(b)
    \qquad
    \text{for all~$a, b \in A$.}
  \]
  The set of derivations of~$A$ is denoted by
  \[
    \Der(A)\glsadd{derivations}
    \defined
    \{
      \delta
      \colon
      A
      \textstyle\to
      A
    \suchthat
      \text{$\delta$ is a derivation of~$A$}
    \}  \,.
  \]
\end{definition}


\begin{proposition}
  Let~$A$ be a~\enquote{\algebra{$\kf$}}.
  The space of derivations~$\Der(A)$ is a Lie~subalgebra of~$\gllie(A)$.
\end{proposition}


\begin{proof}
  For any two element~$a$,~$b$ of~$A$ the auxiliary map
  \[
    h_{ab}
    \colon
    \gllie(A) \,,
    \quad
    \delta
    \mapsto
    \delta(ab) - \delta(a)b - a \delta(b)
  \]
  is linear.
  It follows that
  \[
    \Der(A)
    =
    \bigcap_{a, b \in A}
    \ker( h_{ab} )
  \]
  is a linear subspace of~$\gllie(A)$.
  If~$\delta_1$ and~$\delta_2$ are any two derivations of~$A$ then
  \[
    \delta_1( \delta_2(ab) )
    =
    \delta_1( \delta_2(a) b + a \delta_2(b) )
    =
    \delta_1(\delta_2(a)) + \delta_2(a) \delta_1(b)
    + \delta_1(a) \delta_2(b) + a \delta_1( \delta_2(b) )
  \]
  for all~$a, b \in A$, and therefore
  \begin{align*}
    [\delta_1, \delta_2](ab)
    &=
    (\delta_1 \delta_2 - \delta_2 \delta_1)(ab)
    \\
    &=
    \delta_1( \delta_2(ab) ) - \delta_2( \delta_1(ab) )
    \\
    &=
    \delta_1(\delta_2(a)) b + a \delta_1(\delta_2(b))
    - \delta_2(\delta_1(a)) b - a \delta_2(\delta_1(b))
    \\
    &=
    (\delta_1 \delta_2 - \delta_2 \delta_1)(a) b
    + a (\delta_1 \delta_2 - \delta_2 \delta_1)(b)
    \\
    &=
    [\delta_1, \delta_2](a) b + a [\delta_1, \delta_2](b) \,.
  \end{align*}
  This shows that the commutator~$[\delta_1, \delta_2]$ of any two derivations~$\delta_1$ and~$\delta_2$ of~$A$ is again such a derivation.
  This means that the linear subspace~$\Der(A)$ of~$\gllie(A)$ is indeed a Lie~subalgebra.
\end{proof}


\begin{remark}
  \label{derivations made explicit}
  Let us unravel the definition of a derivation in some special cases.
  \begin{enumerate}
    \item
      Let~$\glie$ be a Lie~algebra.
      A derivation of~$\glie$ is a vector space endomorphism~$\delta$ of~$\glie$ such that
      \[
        \delta([x,y])
        =
        [\delta(x), y] + [x, \delta(y)]
        \qquad
        \text{for all~$x, y \in \glie$.}
      \]
    \item
      Let~$A$ be a commutative~\algebra{$\kf$}.
      A derivation of~$A$ is a vector space endomorphism~$\delta$ of~$A$ such that
      \[
        \delta(ab)
        =
        \delta(a) b + a \delta(b)
        \qquad
        \text{for all~$a, b \in A$}.
      \]
      The~\vectorspace{$\kf$}~$\End_{\kf}(A)$ becomes an~\module{$A$} via the scalar multiplication
      \[
        (a \cdot \phi)(b)
        \defined
        a \cdot \phi(b)
        \qquad
        \text{for all~$a, b \in A$,~$\phi \in \gllie(A)$.}
      \]
      It follows from the commutativity of~$A$ that~$\Der(A)$ is an~\submodule{$A$} of~$\End_{\kf}(A)$, because
      \begin{align*}
        (a \delta)(b c)
        &=
        a \delta(b c)
        \\
        &=
        a ( \delta(b) c + b \delta(c) )
        \\
        &=
        a \delta(b) c + a b \delta(c)
        \\
        &=
        a \delta(b) + b a \delta(c)
        \\
        &=
        (a \delta)(b) + b (a \delta)(c)
      \end{align*}
      for all~$a, b, c \in A$ and~$\delta \in \Der(A)$.
  \end{enumerate}
\end{remark}


\begin{example}
  Let~$\glie$ be an abelian Lie~algebra.
  Every~\linear{$\kf$} endomorphism~$\delta$ of~$\glie$ is already a derivation of~$\glie$ because
  \[
    \delta([x,y])
    =
    0
    =
    [0, y] + [x, 0]
    =
    [\delta(x), y] + [x, \delta(y)]
  \]
  for all~$x, y \in \glie$.
  We thus have~$\Der(\glie) = \gllie(\glie)$.
\end{example}


\begin{lemma}
  \label{about the kernel of a derivation}
  Let~$A$ be an~\enquote{\algebra{$\kf$}} and let~$\delta$ be a derivation of~$A$.
  \begin{enumerate}
    \item
      If~$A$ is unital then~$\delta(1) = 0$.
    \item
      The set~$A' \defined \{ a \in A \suchthat \delta(a) = 0 \}$ is a subalgebra of~$A$.
      If~$A$ is unital then~$A'$ is a unital subalgebra, in the sense that~$1 \in A'$.
  \end{enumerate}
\end{lemma}


\begin{proof}
  \leavevmode
  \begin{enumerate}
    \item
      It holds that
      \[
        \delta(1)
        =
        \delta(1 \cdot 1)
        =
        \delta(1) \cdot 1 + 1 \cdot \delta(1)
        =
        \delta(1) + \delta(1) \,.
      \]
      By subtracting~$\delta(1)$ from both sides of this equation we find that~$\delta(1) = 0$.
    \item
      The set~$A'$ is a subalgebra of~$A$ because the derivation~$\delta$ is~\linear{$\kf$}.
      It holds for all~$a, b \in A'$ that
      \[
        \delta(ab)
        =
        \delta(a) b + a \delta(b)
        =
        0 \cdot a + b \cdot 0
        =
        0 \,,
      \]
      and thus~$ab \in A'$.
      If~$A$ is unital then it follows from the previous assertion that~$\delta(1) = 0$ and thus~$1 \in A'$.
    \qedhere
  \end{enumerate}
\end{proof}


\begin{corollary}
  \label{dervation is uniquely determined by algebra generators}
  Let~$A$ be a~\algebra{$\kf$} and let~$\delta_1$ and~$\delta_2$ be two derivations of~$A$.
  Let~$X$ be an algebra generating set of~$A$ and suppose that~$\delta_1(x) = \delta_2(x)$ for every element~$x$ of~$X$.
  Then already~$\delta_1 = \delta_2$.
\end{corollary}


\begin{proof}
  The difference~$\delta \defined \delta_1 - \delta_2$ is again a derivation of~$A$.
  It follows from \cref{about the kernel of a derivation} that the set
  \[
    A'
    \defined
    \{
      a \in A
    \suchthat
      \delta_1(a) = \delta_2(a)
    \}
    =
    \{
      a \in A
    \suchthat
      \delta(a) = 0
    \}
    =
    \ker(\delta)
  \]
  is a subalgebra of~$A$.
  This subalgebra contains by assumption the algebra generating set~$X$ of~$A$.
  This means that~$A' = A$ and thus~$\delta_1 = \delta_2$.
\end{proof}


\begin{example}
  \label{derivations of commutative polynomial algebra}
  We consider the commutative~\algebra{$\kf$}~$A \defined \kf[x_1, \dotsc, x_n]$.
  The partial derivative function
  \[
    \dd{x_i}
    \colon
    A
    \to
    A
  \]
  with~$i = 1, \dotsc, n$ satisfy the product rule
  \[
    \dd[(fg)]{x_i}
    =
    \dd[f]{x_i} g + f \dd[g]{x_i} \,,
  \]
  which means that each~$\dd{x_i}$ is a derivation of~$A$.
  It follows from \cref{derivations made explicit} that for all polynomials~$f_1, \dotsc, f_n$ the~\linear{$A$} combination
  \[
    \delta
    \defined
    f_1 \dd{x_1} + \dotsb + f_n \dd{x_n}
  \]
  is again a derivation of~$A$.
  We note that this derivation~$\delta$ is given on the algebra generators~$x_1, \dotsc, x_n$ of~$A$ by
  \[
    \delta(x_i)
    =
    \sum_{j=1}^n f_j \dd{x_i}{x_j}
    =
    \sum_{j=1}^n f_j \delta_{ij}
    =
    f_i
  \]
  for every~$i = 1, \dotsc, n$.
  This shows in particular that the polynomials~$f_1, \dotsc, f_n$ are uniquely determined by the derivation~$\delta$, from which it follows that the derivations~$\dd{x_1}, \dotsc, \dd{x_n}$ are linearly independent over~$A$.

  Suppose now that~$\delta$ is any derivation of~$A$.
  For every index~$i = 1, \dotsc, n$ let
  \[
    f_i \defined \delta(x_i) \,,
  \]
  and let overall~$\delta'$ be the derivation of~$A$ given by
  \[
    \delta'
    \defined
    f_1 \dd{x_1} + \dotsb + f_n \dd{x_n} \,.
  \]
  It follows from the above calculations that
  \[
    \delta'(x_i) = f_i = \delta(x_i)
    \qquad
    \text{for all~$i = 1, \dotsc, n$.}
  \]
  According to \cref{dervation is uniquely determined by algebra generators} this shows that already~$\delta = \delta'$.
  We have thus shown that the derivations~$\dd{x_1}, \dotsc, \dd{x_n}$ form an~\generating{$A$} set of~$\Der(A)$.

  We have altogether shown that~$\Der(A)$ is free as an~\module{$A$} with basis given by the partial derivative functions~$\dd{x_1}, \dotsc, \dd{x_n}$.
\end{example}


\begin{remark}
  For an~\algebra{$\kf$}~$A$ we consider the derivations of~$A$ as an algebra as well as the derivations of~$A$ as a Lie~algebra.
  These two notions do not need to coincide.
  Let us be more precise:

  Every dervation~$\delta$ of~$A$ as an algebra is also a derivation of~$A$ as a Lie~algebra, because
  \begin{align*}
    \delta([a,b])
    &=
    \delta(ab - ba)
    \\
    &=
    \delta(ab) - \delta(ba)
    \\
    &=
    ( \delta(a) b + a \delta(b) ) - ( \delta(b) a + b \delta(a) )
    \\
    &=
    \delta(a) b - b \delta(a) + a \delta(b) - \delta(b) a
    \\
    &=
    [\delta(a), b] + [a, \delta(b)]
  \end{align*}
  for all~$a, b \in A$.

  But not every derivation of~$A$ as a Lie~algebra also needs to be a derivation of~$A$ as an algebra.
  To see this we can consider the case that~$A$ is commutative and nonzero.
  Then~$A$ is abelian as a Lie~algebra and thus every vector space endomorphism of~$A$ is already a derivation of~$A$ as a Lie~algebra.
  But every derivation~$\delta$ of~$A$ as an algebra must satisfy the condition~$\delta(1) = 0$.
  So~$\delta = \id_A$ is a derivation of~$A$ as a Lie~algebra, but not a derivation of~$A$ as an algebra.
\end{remark}


\begin{proposition}
  \label{lie algebras act adjoint by derivations}
  \index{adjoint representation}
  Let~$\glie$ be a Lie~algebra.
  The map
  \[
    \ad(x)\glsadd{adjoint representation}
    \colon
    \glie
    \to
    \glie,
    \quad
    y
    \mapsto
    [x,y]
  \]
  is for every element~$x$ of~$\glie$ a derivation of~$\glie$.
\end{proposition}


\begin{proof}
  The map~$\ad$ is linear by the bilinearity of the Lie~bracket of~$\glie$.
  It follows from the Jacobi identity that
  \[
    \ad(x)([y,z])
    =
    [x,[y,z]]
    =
    [[x,y],z] + [y,[x,z]] \\
    =
    [\ad(x)(y), z] + [y, \ad(x)(z)]
  \]
  for all~$y,z \in \glie$.
\end{proof}


\begin{corollary}
  Let~$\glie$ be a Lie algebra.
  The map
  \[
    \ad
    \colon
    \glie
    \to
    \Der(\glie) \,,
    \quad
    x
    \mapsto
    \ad(x)
  \]
  is a well-defined homomorphism of Lie~algebras.
\end{corollary}


\begin{proof}
  We have seen in \cref{homomorphisms of lie algebras} that the map~$\ad' \colon \glie \to \gllie(\glie)$ given by~$x \mapsto \ad(x)$ is a homomorphism of Lie~algebras.
  We have also seen in \cref{lie algebras act adjoint by derivations} that the image of this homomorphism is already contained in the Lie~subalgebra~$\Der(\glie)$ of~$\gllie(\glie)$.
  It follows that~$\ad'$ restricts to the claimed homomorphism.
\end{proof}


\begin{definition}
 Let~$\glie$ be a Lie~algebra.
 A derivation of~$\glie$ is \defemph{inner}\index{inner derivation}\index{derivation!inner} if it is of the form~$\ad(x)$ for some element~$x$ of~$\glie$.
\end{definition}


\begin{lemma}
  \label{commutator of any derivation and inner derivation}
  Let~$\glie$ be a Lie~algebra.
  If~$x$ is any element of~$\glie$ and~$\delta$ is some derivation of~$\glie$ then
  \[
    [\delta, \ad(x)] = \ad(\delta(x)) \,.
  \]
\end{lemma}


\begin{proof}
  We have
  \begin{align*}
    \SwapAboveDisplaySkip
    [\delta, \ad(x)](y)
    &= 
    (\delta \ad(x) - \ad(x) \delta(x))(y)
    \\
    &=
    \delta([x,y]) - [x,\delta(y)]
    \\
    &=
    [\delta(x),y] + [x,\delta(y)] - [x,\delta(y)]
    \\
    &=
    [\delta(x),y]
    \\
    &=
    \ad(\delta(x))(y)
  \end{align*}
  for every~$y \in \glie$, and thus~$[\delta, \ad(x)] = \ad(\delta(x))$.
\end{proof}


\begin{corollary}
  \label{inner derivations are an ideal}
  The inner derivations of a Lie~algebra~$\glie$ form an ideal inside~$\Der(\glie)$.
\end{corollary}


\begin{proof}
  It follows from the linearity of the map~$\ad \colon \glie \to \Der(A)$ that the image~$\ad(\glie)$ is a linear subspace of~$\Der(A)$.
  That~$\ad(\glie)$ is already an ideal of~$\Der(A)$ follows from \cref{commutator of any derivation and inner derivation}.
\end{proof}


\begin{remark}
  If one thinks about Lie algebras as \enquote{derived} versions of (Lie) groups, then derivations are derived versions of automorphisms, and inner derivations are derived versions of inner automorphisms.
  We can therefore extend the comparision between groups and Lie~algebras from \cref{on the notion of ideals} to \cref{correspondence between groups and lie algebras}.
  \begin{table}
    \centering
    \begin{tabular}{ll}
      \toprule
      group theory
      &
      Lie~algebra theory
      \\
      \midrule
      subgroup
      &
      Lie~subalgebra
      \\
      normal subgroup
      &
      Lie~ideal
      \\
      automorphism
      &
      derivation
      \\
      inner automorphism
      &
      inner derivations
      \\
      \bottomrule
    \end{tabular}
    \caption{Correspondence between groups and Lie~algebras.}
    \label{correspondence between groups and lie algebras}
  \end{table}

  That the inner dervivations of a Lie algebra~$\glie$ form an ideal inside the Lie~algebra~$\Der(\glie)$ correspondence to the group theoretic statement, that the inner automorphisms of a group~$G$ form a normal subgroup of the automorphism group~$\Aut(G)$.
  Moreover, the center of~$\glie$ is precisely the kernel of the Lie~algebra homomorphism~$\ad \colon \glie \to \Der(\glie)$, just the center of~$G$ is precisely the kernel of the group homomorphism~$G \to \Aut(G)$.
\end{remark}



\subsection{Inner Automorphisms}


\begin{recall}
  Let~$V$ be a~\vectorspace{$\kf$} where~$\kf$ if a field of characteristic zero.
  Let~$f$ be a nilpotent endomorphism of~$V$.
  Then
  \[
    \exp(f)
    \defined
    \sum_{n=0}^\infty
    \frac{f^n}{n!}
  \]
  is again a well-defined endomorphism of~$V$.
  If~$\alpha$ is an automorphism of~$V$ then the conjugate~$\alpha f \alpha^{-1}$ is again nilpotent, and
  \[
    \exp( \alpha f \alpha^{-1} )
    =
    \sum_{n=0}^\infty
    \frac{ (\alpha f \alpha^{-1})^n }{n!}
    =
    \sum_{n=0}^\infty
    \frac{ \alpha f^n \alpha^{-1} }{n!}
    =
    \alpha
    \Biggl(
      \sum_{n=0}^\infty
      \frac{f^n}{n!}
    \Biggr)
    \alpha^{-1}
    =
    \alpha \exp(f) \alpha^{-1} \,.
  \]

  Let now~$g$ be another nilpotent endomorphism of~$V$ and suppose that the endomorphisms~$f$ and~$g$ commute.
  Then the sum~$f + g$ is again nilpotent.
  To be more precise, we have
  \[
    (f + g)^n
    =
    \sum_{k=0}^n
    \binom{n}{k}
    f^k g^{n-k}
  \]
  for every exponent~$n \in \Natural$.
  If~$f^{n'} = 0$ and~$g^{n''} = 0$ then by choosing~$n > n' + n''$ we find that~$f^k = 0$ or~$g^{n-k} = 0$ for every~$k = 0, \dotsc, n$, and thus~$(f + g)^n = 0$.

  We can therefore also consider the endomorphism~$\exp(f + g)$, and it follows from the commutativity of~$f$ and~$g$ that 
  \begingroup
  \allowdisplaybreaks
  \begin{align*}
    \exp(f + g)
    &=
    \sum_{n=0}^\infty
    \frac{(f + g)^n}{n!}
    \\
    &=
    \sum_{n=0}^\infty
    \frac{1}{n!}
    \sum_{k=0}^n
    \binom{n}{k}
    f^k g^{n-k}
    \\
    &=
    \sum_{n=0}^\infty
    \sum_{k=0}^n
    \frac{f^k g^{n-k}}{k!(n-k)!}
    \\
    &=
    \sum_{n=0}^\infty
    \sum_{k + l = n}
    \frac{f^k g^l}{k! l!}
    \\
    &=
    \sum_{k, l = 0}^\infty
    \frac{f^k g^l}{k! l!}
    \\
    &=
    \Biggl(
      \sum_{k=0}^\infty
      \frac{f^k}{k!}
    \Biggr)
    \Biggl(
      \sum_{l=0}^\infty
      \frac{g^l}{l!}
    \Biggr)
    \\
    &=
    \exp(f) \exp(g) \,.
  \end{align*}
  \endgroup
  It follows in particular for~$g = -f$ that
  \[
    \exp(f) \exp(-f) = \exp(f - f) = \exp(0) = \id_V \,,
  \]
  and similarly that~$\exp(-f) \exp(f) = \id_V$.
  This shows that~$\exp(f)$ is already a vector space automorphism of~$V$, with~$\exp(f)^{-1} = \exp(-f)$.
\end{recall}


\begin{lemma}
  Let~$A$ be a~\enquote{\algebra{$\kf$}} and let~$\delta$ be a derivation of~$A$.
  Then
  \[
    \delta^n(ab)
    =
    \sum_{k=0}^n
    \binom{n}{k}
    \delta^k(a) \delta^{n-k}(b)
    \qquad
    \text{for all~$n \in \Natural$,~$a, b \in A$.}
  \]
\end{lemma}


\begin{proof}
  We prove the formula by induction.
  It holds for~$n = 0$.
  If the formula holds for some exponent~$n$ then it also holds for the exponent~$n + 1$ because
  \begingroup
  \allowdisplaybreaks
  \begin{align*}
    \delta^{n+1}(ab)
    &=
    \delta( \delta^n(ab) )
    \\
    &=
    \delta
    \Biggl(
      \sum_{k=0}^n
      \binom{n}{k}
      \delta^k(a) \delta^{n-k}(b)
    \Biggr)
    \\
    &=
    \sum_{k=0}^n
    \binom{n}{k}
    \delta
    \bigl(
      \delta^k(a) \delta^{n-k}(b)
    \bigr)
    \\
    &=
    \sum_{k=0}^n
    \binom{n}{k}
    \bigl(
      \delta^{k+1}(a) \delta^{n-k}(b)
      +
      \delta^k(a) \delta^{n+1-k}(b)
    \bigr)
    \\
    &=
    \sum_{k=0}^n
    \binom{n}{k}
    \delta^{k+1}(a) \delta^{n-k}(b)
    +
    \sum_{k=0}^n
    \binom{n}{k}
    \delta^k(a) \delta^{n+1-k}(b)
    \\
    &=
    \sum_{k=1}^{n+1}
    \binom{n}{k-1}
    \delta^{k}(a) \delta^{n+1-k}(b)
    +
    \sum_{k=0}^n
    \binom{n}{k}
    \delta^k(a) \delta^{n+1-k}(b)
    \\
    &=
    \delta^{n+1}(a) b
    +
    \sum_{k=1}^n
    \binom{n}{k-1}
    \delta^{k}(a) \delta^{n+1-k}(b)
    +
    \sum_{k=1}^n
    \binom{n}{k}
    \delta^k(a) \delta^{n+1-k}(b)
    +
    a \delta^{n+1}(b)
    \\
    &=
    \delta^{n+1}(a) b
    +
    \sum_{k=1}^n
    \Biggl(
      \binom{n}{k-1}
      +
      \binom{n}{k}
    \Biggr)
    \delta^k(a) \delta^{n+1-k}(b)
    +
    a \delta^{n+1}(b)
    \\
    &=
    \delta^{n+1}(a) b
    +
    \sum_{k=1}^n
    \binom{n+1}{k}
    \delta^k(a) \delta^{n+1-k}(b)
    +
    a \delta^{n+1}(b)
    \\
    &=
    \sum_{k=0}^{n+1}
    \binom{n+1}{k}
    \delta^k(a) \delta^{n+1-k}(b) \,.
  \end{align*}
  \endgroup
  This proves the formula.
\end{proof}


\begin{proposition}
  \label{exponential of derivation is automorphism}
  Let~$A$ be a~\enquote{\algebra{$\kf$}} where~$\kf$ is a field of characteristic zero and let~$\delta$ be a nilpotent derivation of~$A$.
  Then
  \[
    \alpha
    \defined
    \exp(\delta)
    =
    \sum_{n=0}^\infty \frac{\delta^n}{n!}
  \]
  is an algebra automorphism of~$A$.
\end{proposition}


\begin{proof}
  We have for any two element~$a$,~$b$ of~$A$ that
  \begingroup
  \allowdisplaybreaks
  \begin{align*}
    \alpha(ab)
    &=
    \sum_{n=0}^\infty
    \frac{\delta^n}{n!}(ab)
    \\
    &=
    \sum_{n=0}^\infty
    \frac{1}{n!}
    \delta^n(ab)
    \\
    &=
    \sum_{n=0}^\infty
    \frac{1}{n!}
    \sum_{k=0}^n
    \binom{n}{k}
    \delta^k(a) \delta^{n-k}(b)
    \\
    &=
    \sum_{n=0}^\infty
    \sum_{k=0}^n
    \frac{\delta^k(a) \delta^{n-k}(b)}{k! (n-k)!}
    \\
    &=
    \sum_{n=0}^\infty
    \sum_{k+l=n}
    \frac{\delta^k(a) \delta^{n-k}(b)}{k! l!}
    \\
    &=
    \sum_{k,l = 0}^\infty
    \frac{\delta^k(a) \delta^{n-k}(b)}{k! l!}
    \\
    &=
    \Biggl(
      \sum_{k=0}^\infty
      \frac{\delta^k(a)}{k!}
    \Biggr)
    \Biggl(
      \sum_{l=0}^\infty
      \frac{\delta^l(b)}{l!}
    \Biggr)
    \\
    &=
    \Biggl(
      \sum_{k=0}^\infty
      \frac{\delta^k}{k!}
    \Biggr)
    (a)
    \Biggl(
      \sum_{l=0}^\infty
      \frac{\delta^l}{l!}
    \Biggr)
    (b)
    \\
    &=
    \alpha(a) \alpha(b) \,. 
  \end{align*}
  \endgroup
  This shows that~$\alpha$ preserves multiplication.
  It is also a vector space automorphism.
  Together this shows that~$\alpha$ is an algebra automorphism.
\end{proof}


\begin{corollary}
  Let~$\glie$ be a~\liealgebra{$\kf$} where~$\kf$ is a field of characteristic zero.
  Let~$x$ be an element of~$\glie$ such that~$\ad(x)$ is nilpotent.
  Then~$\exp(\ad(x))$ is a Lie~algebra automorphism of~$\glie$.
\end{corollary}


\begin{proof}
  We have seen in \cref{lie algebras act adjoint by derivations} that~$\ad(x)$ is a derivation of~$\glie$.
  The assertion thus follows from \cref{exponential of derivation is automorphism}.
\end{proof}


\begin{definition}
  Let~$\glie$ be a~\liealgebra{$\kf$} where~$\kf$ is a field of characteristic zero.
  \begin{enumerate}
    \item
      The group of \defemph{inner automorphisms}\index{inner!automorphism}\index{automorphism!inner} of~$\glie$ is the subgroup of~$\Aut_{\Lie}(\glie)$ which is generated by all automorphism of the form~$\exp(\ad(x))$ where~$x$ is any element of~$\glie$ for which~$\ad(x)$ is nilpotent.
      The group of inner automorphisms of~$\glie$ is denoted by~$\Inn(\glie)$.
    \item
      The elements of~$\Inn(\glie)$ are the \defemph{inner automorphisms}\index{inner!automorphism}\index{automorphism!inner} of~$\glie$.
  \end{enumerate}
\end{definition}


\begin{lemma}
  \label{adjoint and automorphisms}
  Let~$\glie$ be a Lie~algebra.
  Let~$x$ be any element of~$\glie$ and let~$\alpha$ be an automorphism of~$\glie$.
  Then
  \[
    \ad(\alpha(x))
    =
    \alpha \circ \ad(x) \circ \alpha^{-1} \,.
  \]
\end{lemma}


\begin{proof}
  We have for every element~$y$ of~$\glie$ that
  \[
    \ad(\alpha(x))(\alpha(y))
    =
    [\alpha(x), \alpha(y)]
    =
    \alpha( [x,y] )
    =
    \alpha( \ad(x)(y) )
  \]
  and therefore overall
  \[
    \ad(\alpha(x)) \circ \alpha
    =
    \alpha \circ \ad(x) \,.
  \]
  Composing both sides of this equation with~$\alpha^{-1}$ proves the assertion.
\end{proof}


\begin{lemma}
  \label{conjugation of inner automorphism}
  Let~$\glie$ be a~\liealgebra{$\kf$} where~$\kf$ is a field of characteristic zero.
  Let~$x$ be any element of~$\glie$ for which~$\ad(x)$ is nilpotent, and let~$\alpha$ be any automorphism of~$\glie$.
  Then~$\ad(\alpha(x))$ is again nilpotent and
  \[
    \alpha \exp(\ad(x)) \alpha^{-1}
    =
    \exp( \alpha(x) ) \,.
  \]
\end{lemma}


\begin{proof}
  It follows from \cref{adjoint and automorphisms} that the endomorphism~$\ad(\alpha(x)) = \alpha \ad(x) \alpha^{-1}$ is a conjugate of the nilpotent endomorphism~$\ad(x)$, and thus again nilpotent.
  We also find that
  \[
    \exp( \ad(\alpha(x)) )
    =
    \exp( \alpha \ad(x) \alpha^{-1} )
    =
    \alpha \exp(\ad(x)) \alpha^{-1} \,,
  \]
  as desired.
\end{proof}


\begin{proposition}
  Let~$\glie$ be a~\liealgebra{$\kf$} where~$\kf$ is a field of characteristic zero.
  Then~$\Inn(\glie)$ is a normal subgroup of~$\Aut_{\Lie}(\glie)$.
\end{proposition}


\begin{proof}
  The group~$\Inn(\glie)$ is generated by the set of automorphisms
  \[
    \{
      \exp(\ad(x))
    \suchthat
      \text{$x \in \glie$ such that~$\ad(x)$ is nilpotent}
    \} \,.
  \]
  It follows from \cref{conjugation of inner automorphism} that this generating set is closed under the conjugation by any automorphism of~$\glie$.
  It follows that~$\Inn(\glie)$ is also closed under this conjugation.
\end{proof}





\section{Simple Lie~Algebras}


\begin{definition}
 A Lie~algebra~$\glie$ is \defemph{simple}\index{simple!Lie algebra}\index{Lie!algebra!simple} if it is non-abelian and contains no ideals apart from~$0$ and~$\glie$ itself.
\end{definition}


\begin{remark}
  \leavevmode
  \begin{enumerate}
    \item
      Any simple Lie~algebra is in particular nonzero.
    \item
      A Lie~algebra is simple if and only if it is non-abelian and contains precisely two ideals.
  \end{enumerate}
\end{remark}


\begin{examples}
  \leavevmode
  \begin{enumerate}
    \item
      The Lie~algebra~$\gllie(n, \kf)$ isn’t simple.
      For~$n = 0$ it is zero, for~$n = 1$ it is abelian, and for~$n \geq 1$ it contains~$\sllie(n,\kf)$ as a nonzero, proper ideal.
    \item
      The Lie~algebra~$\glie \defined \sllie(2, \kf)$ is simple if and only if~$\ringchar \kf \neq 2$
      To prove this we consider the standard basis
      \[
        e
        =
        \begin{pmatrix}
          0 & 1 \\
          0 & 0
        \end{pmatrix} \,,
        \qquad
        h
        =
        \begin{pmatrix*}[r]
          1 &  0  \\
          0 & -1
        \end{pmatrix*} \,,
        \qquad
        f
        =
        \begin{pmatrix}
          0 & 0 \\
          1 & 0
        \end{pmatrix}
      \]
      of~$\sllie(2 \kf)$, on which the Lie~bracket of~$\sllie(2, \kf)$ is given by the relations
      \[
        [h,e] = 2e  \,,
        \qquad
        [h,f] = -2 f \,,
        \qquad
        [e,f] = h \,.
      \]
      
      If~$\ringchar(\kf) = 2$ then the element~$h$ spans a {\onedimensional} ideal, which then shows that~$\sllie_2(\kf)$ is not simple.
      
      Let us consider in the following the case~$\ringchar(\kf) \neq 2$.
      Let~$I$ be a nonzero ideal in~$\sllie_2(\kf)$ and let~$x$ be some nonzero element of~$I$.
      We may write the element~$x$ as a linear combination
      \[
        x = \alpha e + \beta h + \gamma f
      \]
      for some scalar~$\alpha, \beta, \gamma \in \kf$.
      It follows that
      \[
        [e,x]
        =
        -2 \beta e + \gamma h \,,
        \qquad \text{and thus}\qquad
        [e,[e,x]]
        =
        -2 \gamma e \,.
      \]
      Both~$[e,x]$ and~$[e,[e,x]]$ are again contained in~$I$ because~$I$ is an ideal.

      It now follows that the ideal~$I$ contains the basis vector~$e$:
      If~$\gamma \neq 0$ then it follows from the equality~$[e,[e,x]] = -2 \gamma e$ that
      \[
        e
        =
        - \frac{ [e,[e,x]] }{ 2\gamma } \,.
      \]
      If~$\gamma = 0$ but~$\beta \neq 0$ then it follows from the equality$[e,x] = -2 \beta e$ that
      \[
        e
        =
        -\frac{ [e,x] }{ 2\beta } \,.
      \]
      And if both~$\beta = 0$ and~$\gamma = 0$ then it follows that~$\alpha$ must be nonzero because~$x$ is nonzero, and it then follows from the equality~$x = \alpha e$ that
      \[
        e = \frac{x}{\alpha} \,.
      \]
      In each case we find that~$e$ is contained in~$I$, because the elements~$x$,~$[e,x]$ and~$[e,[e,x]]$ are contained in~$I$.
      
      It now further follows that both~$h = [e,f]$ and~$f = -[h,f]/2$ are again contained in~$I$.
      We have thus altogether found that the ideals~$I$ contains all three standard basis vectors of~$\sllie(2 \kf)$, and must therefore equal~$\sllie(2, \kf)$.
  \end{enumerate}
\end{examples}


\begin{proposition}
  \label{commutator and center of simple}
  Let~$\glie$ be a simple Lie~algebra.
  Then~$[\glie, \glie] = \glie$ and~$\centerlie(\glie) = 0$.
\end{proposition}


\begin{proof}
  The Lie~algebra~$\glie$ is nonabelian because it is simple.
  We therefore find that the commutator ideal~$[\glie, \glie]$ is a nonzero ideal of~$\glie$, and we also find that the center~$\centerlie(\glie)$ is a proper ideal of~$\glie$.
  But~$\glie$ admits by assumption only two ideals, namely the zero ideal and itself.
  It therefore follows that~$[\glie, \glie] = \glie$ and~$\centerlie(\glie) = 0$.
\end{proof}


\begin{remark}
  If~$\kf$ is a field of characteristic zero then the Lie~algebra~$\sllie(n, \kf)$ is simple for every~$n \geq 2$.
  We will prove this later on, after we have introduced the idea of a root space decomposition.
  % TODO: Add a proof of this using root spaces.
\end{remark}


\begin{corollary}
  \label{ad is injective for simple}
  Let~$\glie$ be a simple Lie~algebra.
  Then the homomorphism of Lie~algebras
  \[
    \ad
    \colon
    \glie
    \to
    \gllie(\glie)
  \]
  is injective.
\end{corollary}

\begin{proof}
  The kernel of~$\glie$ is precisely the center of~$\glie$, which vanishes by \cref{commutator and center of simple}.
\end{proof}

\begin{corollary}
  Every finite-dimensional, simple Lie~algebra is isomorphic to a linear Lie~algebra.
\end{corollary}

\begin{proof}
  Let~$\glie$ be a simple Lie~algebra.
  It follows from \cref{ad is injective for simple} that~$\glie$ is isomorphic to the linear Lie algebra~$\ad(\glie)$.
\end{proof}


\begin{fluff}
  A famous theorem due to Ado -- which we will not attempt to prove in this lecture -- states that this conclusion does actually hold for every finite-dimensional Lie~algebra.
\end{fluff}


\begin{theorem}[Ado, first version]
  \index{Ado’s theorem}
  Every finite-dimensional Lie~algebra is isomorphic to a linear Lie~algebra.
\end{theorem}






