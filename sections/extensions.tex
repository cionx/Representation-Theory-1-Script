\subsection{Extensions}


\begin{definition}
  A \defemph{short exact sequence of Lie~algebras}\index{short exact sequence!of Lie algebras} is a short exact sequence
  \begin{equation}
    \label{general extension}
    0 
    \to
    I
    \xlongto{f}
    \glie
    \xlongto{g}
    \hlie
    \to
    0
  \end{equation}
  of vector spaces such that~$I$,~$\glie$ and~$\hlie$ are Lie algebras and both~$f$ and~$g$ are Lie algebra homomorphisms.
  The short exact sequence~\eqref{general extension} is then an~\defemph{extension}\index{extension} of~$\hlie$ by~$I$.
\end{definition}


\begin{remark}
  \leavevmode
  \begin{enumerate}
    \item
      By abuse of notation we often say that \enquote{$\glie$ is an extension of~$\hlie$ by~$I$} to mean that there exists such an extension with~$\glie$ as its middle term.
    \item
      If~$0 \to I \xto{f} \glie \xto{g} \hlie \to 0$ is a short exact sequence of Lie~algebras then the image of~$I$ is the kernel of~$g$ and hence an ideal in~$\glie$.
  \end{enumerate}
\end{remark}


\begin{example}
  If~$I$ is any ideal in a Lie~algebra~$\glie$ then
  \[
    0
    \to
    I
    \xlongto{i}
    \glie
    \xlongto{p}
    \glie/I
    \to
    0
  \]
  is a short exact sequence of Lie~algebras where~$i$ is the inclusion~$x \to x$ and~$p$ is the canonical projection~$x \mapsto \class{x}$.
\end{example}


\begin{definition}
  \label{equivalence of extensions}
  Let~$I$ and~$\hlie$ be two Lie~algebras.
  Two extensions
  \[
    0 
    \to
    I
    \xlongto{f}
    \glie
    \xlongto{g}
    \hlie
    \to
    0
    \quad\text{and}\quad
    0 
    \to
    I
    \xlongto{f'}
    \glie'
    \xlongto{g'}
    \hlie
    \to
    0
  \]
  of~$\hlie$ by~$I$ are \defemph{equivalent}\index{equivalent extensions} if there exists an isomorphism of Lie~algebras~$\varphi \colon \glie \to \glie'$ that makes the following diagram commute:
  \[
    \begin{tikzcd}
      0
      \arrow{r}
      &
      I
      \arrow{r}
      \arrow[equal]{d}
      &
      \glie
      \arrow{r}
      \arrow[dashed]{d}[right]{\varphi}
      &
      \hlie
      \arrow{r}
      \arrow[equal]{d}
      &
      0
      \\
      0
      \arrow{r}
      &
      I
      \arrow{r}
      &
      \glie'
      \arrow{r}
      &
      \hlie
      \arrow{r}
      &
      0
    \end{tikzcd}
  \]
\end{definition}


\begin{lemma}
  Let~$I$ and~$\hlie$ be two Lie~algebras.
  Equivalences of extensions is an equivalence relation on the class of extensions of~$\hlie$ by~$I$.
\end{lemma}


\begin{remark}
  It follows from the five lemma that in \cref{equivalence of extensions} we do not have to require the homomorphism~$\varphi$ to be an isomorphism --- this automatically follows from the commutativity of the given diagram.
\end{remark}


\begin{remark}
  \label{general approach to extensions}
  Let~$I$ and~$\hlie$ be Lie~algebras.
  One general approach to understanding (certain equivalence classes of) extensions of~$\hlie$ by~$I$ is the following:
  
  Any such extension
  \begin{equation}
  \label{original extension}
    0
    \to
    I
    \xlongto{f}
    \glie
    \xlongto{g}
    \hlie
    \to
    0
  \end{equation}
  is in particular a short exact sequece of vector spaces, and splits as such.
  In other words, there exists an isomorphism of vector spaces~$\varphi \colon \glie \to \hlie \oplus I$ that makes the diagram
  \[
    \begin{tikzcd}[column sep = large]
      0
      \arrow{r}
      &
      I
      \arrow{r}
      \arrow[equal]{d}
      &
      \glie
      \arrow{r}
      \arrow[dashed]{d}[right]{\varphi}
      &
      \hlie
      \arrow{r}
      \arrow[equal]{d}
      &
      0
      \\
      0
      \arrow{r}
      &
      I
      \arrow{r}[above]{\iota}
      &
      \hlie \oplus I
      \arrow{r}[above]{\pi}
      &
      \hlie
      \arrow{r}
      &
      0
    \end{tikzcd}
  \]
  commute.
  Here we denote~$\iota \colon I \to \hlie \oplus I$ the canonical inclusion~$c \mapsto (0,c)$ and by~$\pi \colon \hlie \oplus I \to \hlie$ the canonical projection~$(x,c) \mapsto x$.
  It follows that the original extension~\eqref{original extension} is equivalent to an extension whose middle term is~$\hlie \oplus I$.
  Up to equivalence we can therefore assume that the extension of interest is of the form
  \[
    0
    \to
    I
    \xlongto{\iota}
    \hlie \oplus I
    \xlongto{\pi}
    \hlie
    \to
    0 \,.
  \]

  Suppose now that we are given such an extension.
  We have for the Lie~bracket on~$\hlie \oplus I$ that
  \begin{align*}
    {}&
    [(x, c), (y,d)]
    \\
    ={}&
      [(x,0), (y,0)]
    + [(x,0), (0,d)]
    + [(0,c), (y,0)]
    + [(0,c), (0,d)]
    \\
    ={}&
      [(x,0), (y,0)]
    + [(x,0), (0,d)]
    - [(y,0), (0,c)]
    + [(0,c), (0,d)]
  \end{align*}
  for all~$(x,c), (y,d) \in \hlie \oplus I$.
  We have that
  \[
    \pi( [(x,0), (y,0)] )
    =
    [ \pi( (x,0) ), \pi( (y,0) ) ]
    =
    [x, y]
  \]
  because~$\pi$ is a homorphism of Lie~algebras.
  The commutator~$[(x,0), (y,0)]$ is therefore of the form
  \[
    [(x,0), (y,0)]
    =
    ( [x,y], \kappa(x,y) )
  \]
  for some function
  \[
    \kappa
    \colon
    \hlie \times \hlie
    \to
    I \,.
  \]
  It follows from the bilinearity of the Lie~bracket~$[-,-]$ on~$\hlie \oplus I$ that the map~$\kappa$ is again bilinear.
  The commutators~$[(x,0), (0,c)]$ and~$[(y,0), (0,d)]$ are again contained in~$0 \oplus I$ because this is an ideal in~$\hlie \oplus I$ (since it is the kernel of~$\pi$).
  It follows that there exists a unique linear map
  \[
    \theta
    \colon
    \hlie
    \to
    \gllie(I)
  \]
  with
  \[
    [(z,0), (0,e)]
    =
    [0, \theta(z)(e)]
  \]
  for all~$z \in \hlie$ and~$e \in I$.
  We have seen in \cref{lie algebras act adjoint by derivations} that the action of~$[(z,0), -] = \ad_{\hlie \oplus I}((z,0))$ on~$\hlie \oplus I$ is by derivations.
  It follows that~$\theta$ is actually a linear map
  \[
    \theta
    \colon
    \hlie
    \to
    \Der(I) \,.
  \]
  We now have that
  \[
    [(x,0), (0,d)]
    =
    [0, \theta(x)(d)]
    \quad\text{and}\quad
    [(y,0), (0,e)]
    =
    [0, \theta(y)(e)] \,.
  \]
  The last remaining commutator~$[(0,c), (0,d)]$ can be computed by observing that
  \[
    [(0,c), (0,d)]
    =
    [\iota(c), \iota(d)]
    =
    \iota([c,d])
    =
    (0, [c,d])
  \]
  because the inclusion~$\iota \colon I \to \hlie \oplus I$ is a homomorphism of Lie~algebras.
  We find altogether that the Lie bracket on~$\hlie \oplus I$ can be expressed with help of the maps~$\kappa$ and~$\theta$ as
  \begin{align*}
    {}&
    [ (x,c), (y,d) ]
    \\
    ={}&
      ( [x,y], \kappa(x,y) )
    + ( 0, \theta(x)(d) )
    - ( 0, \theta(y)(c) )
    + ( 0, [c,d] )
    \\
    ={}&
    ( [x,y], \kappa(x,y) + \theta(x)(d) - \theta(y)(c) + [c,d] ) \,.
  \end{align*}
  The Lie~algebra structure of the extension~$\hlie \oplus I$ is therefore uniquely determined by the two maps~$\kappa \colon \hlie \times \hlie \to I$ and~$\theta \colon \hlie \to \Der(I)$.

  Suppose now that we are not yet given a Lie~bracket on~$\hlie \oplus I$ but instead a bilinear map~$\kappa \colon \hlie \times \hlie \to I$ and a linear map~$\theta \colon \hlie \to \Der(I)$.
  Then we can conversely define a bilinear bracket~$[-,-]$ on~$\hlie \oplus I$ via the above formula
  \[
    [ (x,c), (y,d) ]
    \defined
    ( [x,y], \kappa(x,y) + \theta(x)(d) - \theta(y)(c) + [c,d] )  \,.
  \]
  (Note that this formula involves three brackets:
  The one on~$\hlie \oplus I$ that is being defined, and the ones on~$\hlie$ and~$I$ that are used on the right hand side.)
  If this is a Lie bracket on~$\hlie \oplus I$ then this makes~$\hlie \oplus I$ into an extension of~$\hlie$ of~$I$, as both the inclusion~$\iota \colon I \to \hlie \oplus I$ and the projection~$\pi \colon \hlie \oplus I \to \hlie$ are then Lie~algebra homomorphisms.
  But of course not any choice of~$\kappa$ and~$\theta$ does define a Lie~bracket on~$\hlie \oplus I$ via the above formula:
  We need that~$[-,-]$ is alternating and that~$[-,-]$ satisfies the Jacobi identity.
  
  For the first condition we calculate that
  \[
    [(x,c), (x,c)]
    =
    ( [x,x], \kappa(x,x) + \theta(x)(c) - \theta(x)(c) + [c,c] )
    =
    (0, \kappa(x,x))
  \]
  for all~$(x, c) \in \hlie \oplus I$.
  This shows that~$[-,-]$ is alternating if and only if~$\kappa$ is alternating.
  We can similarly express that~$[-,-]$ satisfies the Jacobi identity in terms of both~$\kappa$ and~$\theta$.
  But we will not do this here, as the resulting terms become rather ugly.
  We will instead do this missing calculation for some special cases.
\end{remark}


\begin{warning}
  If we are given an extension~$0 \to I \to \glie \to \hlie \to 0$ then we may identify~$\glie$ with the direct sum~$\glie \oplus I$ as vector spaces and then describe the Lie~bracket of~$\glie$ in terms of a bilinear map~$\kappa \colon \hlie \times \hlie \to I$ and a linear map~$\theta \colon \hlie \to \Der(I)$.
  But these maps depend not only on the Lie~bracket of~$\glie$ but also on the choosen identification of~$\glie$ with~$\hlie \oplus I$!
\end{warning}



\subsubsection{Central Extensions}


\begin{definition}
  An extension~$0 \to I \to \glie \to \hlie \to 0$ of Lie~algebras is \emph{central}\index{central extension}\index{extension!central} if the image of~$I$ is contained in the center of~$\glie$.
\end{definition}


\begin{example}[Central extensions]
  In the notation of \cref{general approach to extensions} we find that an extension
  \[
    0
    \to
    I
    \xlongto{\iota}
    \hlie \oplus I
    \xlongto{\pi}
    \hlie
    \to
    0
  \]
  is central if and only~$[(x,c), (0,d)] = 0$ for all~$(x,c), (0,d) \in \hlie \oplus I$.
  This means precisely that~$\theta = 0$ and also~$[c,d] = 0$ for all~$c, d \in I$.
  Hence the Lie bracket on~$\hlie \oplus I$ is given by
  \begin{equation}
    \label{central extension formula}
    [ (x,c), (y,d) ]
    =
    ( [x,y], \kappa(x,y) )
  \end{equation}
  for all~$(x,c), (y,d) \in \hlie \oplus I$.
  
  Given any bilinear form~$\kappa \colon \hlie \times \hlie \to I$ we have already seen that \eqref{central extension formula} defines a bilinear bracket on~$\hlie \oplus I$ that is alternating if and only if~$\kappa$ is alternating.
  We find for the Jacobi identity that
  \[
    [ (x,c), [ (y,d), (z,e) ] ]
    =
    [ (x,c), ([y,z], \kappa(y,z)) ]
    =
    ( [x,[y,z]], \kappa(x, [y,z]) )
  \]
  for all~$(x,c), (y,d), (z,e) \in \hlie \oplus I$ and therefore
  \begin{align*}
    {}&
      [(x,c), [(y,d), (z,e)]]
    + [(y,d), [(z,e), (x,c)]]
    + [(z,e), [(x,c), (y,d)]]
    \\
    ={}&
    (
      [x,[y,z]] + [y,[z,x]] + [z,[x,y]],
      \kappa(x, [y,z]) + \kappa(y, [z,x]) + \kappa(z, [x,y])
    )
    \\
    ={}&
    ( 0, \kappa(x, [y,z]) + \kappa(y, [z,x]) + \kappa(z, [x,y]) ) \,.
  \end{align*}
  We hence find that the bracket~$[-,-]$ on~$\hlie \oplus I$ satisfies the Jacobi identity if and only if the bilinear map~$\kappa$ satisfies the similar looking identity
  \[
    \kappa(x, [y,z]) + \kappa(y, [z,x]) + \kappa(z, [x,y])
    =
    0 \,.
  \]
  This condition is the \defemph{{\twococycle} condition}\index{$2$-cocycle condition}, and a bilinear map~$\kappa \colon \hlie \to \hlie \to I$ that is both alternating is satisfies the {\twococycle} condition is a {\twococycle}.
  
  We have overall constructed for all Lie~algebras~$\hlie$ and~$I$ a {\onetoone} correspondence between
  \begin{itemize}
    \item
      Lie~brackets~$[-,-]$ on the vector space~$\hlie \oplus I$ that make the standard short exact sequence
      \[
        0
        \to
        I
        \to
        \hlie
        \oplus
        I
        \to
        \hlie
        \to
        0
      \]
      into a central extension of~$\hlie$ by~$I$ and
    \item
      {\twococycles}~$\kappa \colon \hlie \times \hlie \to I$,
  \end{itemize}
  and this correspondence is given by~$[(x,c), (y,d)] = ([x,y], \kappa(x,y))$.
\end{example}



\subsubsection{Split Extensions and Semidirect Products}


\begin{definition}
  A short exact sequence
  \[
    0
    \to
    I
    \xlongto{f}
    \glie
    \xlongto{g}
    \hlie
    \to
    0
  \]
  \defemph{splits}\index{extension!split} if there exists a Lie~algebra homomorphism~$s \colon \hlie \to \glie$ with~$g \circ s = \id_{\hlie}$.
  The homomorphism~$s$ is then a \defemph{split} for the short exact sequence.
  
  A \emph{split extension} is an extension that splits.%
  \footnote{This is meant to be funny.}
\end{definition}


\begin{example}[Split extensions]
  Let
  \[
    0
    \to
    I
    \xlongto{f}
    \glie
    \xlongto{g}
    \hlie
    \to
    0
  \]
  be any split extension of~$\hlie$ by~$I$, and let~$s \colon \hlie \to \glie$ be a split.
  Then there exists a (unique) homomorphism of vector spaces~$\varphi \colon \glie \to \hlie \oplus I$ that makes the resulting diagram
  \[
    \begin{tikzcd}[column sep = large]
      0
      \arrow{r}
      &
      I
      \arrow{r}[above]{f}
      \arrow[equal]{d}
      &
      \glie
      \arrow{r}[above]{g}
      \arrow[dashed]{d}[right]{\varphi}
      &
      \hlie
      \arrow{r}
      \arrow[equal]{d}
      &
      0
      \\
      0
      \arrow{r}
      &
      I
      \arrow{r}[above]{\iota}
      &
      \hlie \oplus I
      \arrow{r}[above]{\pi}
      &
      \hlie
      \arrow{r}
      &
      0
    \end{tikzcd}
  \]
  commute and such that the square diagram
  \[
    \begin{tikzcd}
      \glie
      \arrow[dashed]{d}[left]{\varphi}
      &
      \hlie
      \arrow{l}[above]{s}
      \arrow[equal]{d}
      \\
      \hlie \oplus I
      &
      \hlie
      \arrow{l}{\sigma}
    \end{tikzcd}
  \]
  commutes.
  Hence we may again assume that~$\glie = \hlie \oplus I$, that~$f$ is the canonical inclusion~$\iota \colon I \to \hlie \oplus I$, that~$g$ is the canonical projection~$\pi \colon \hlie \oplus I \to \hlie$ that~$s$ is the canonical inclusion~$\sigma \colon \hlie \to \hlie \oplus I$.
  
  We can now use the notations from \cref{general approach to extensions} to express the Lie bracket~$[-,-]$ of~$\hlie \oplus I$ via the bilinear form~$\kappa \colon \hlie \times \hlie \to I$ and the linear map~$\theta \colon \hlie \to \Der(I)$.
  
  The bilinear map~$\kappa$ is defined by the equality
  \[
    [(x,0), (y,0)]
    =
    ([x,y], \kappa(x,y))
  \]
  for all~$x, y \in \hlie$.
  The linear subspace~$\hlie \oplus 0$ of~$\hlie \oplus I$ is the image of the Lie~algebra homomorphism~$\sigma \colon \hlie \to \hlie \oplus I$ and hence a Lie~subalgebra of~$\hlie \oplus I$.
  We therefore find that the commutator~$[(x,0), (y,0)]$ is again contained in~$\hlie \oplus 0$ for all~$x, y \in \hlie$.
  This means that~$\kappa = 0$.
  
  The linear map~$\theta \colon \hlie \to I$ takes for~$x \in \hlie$ the restriction of the endomorphism~$\ad_{\hlie \oplus I}((x,0))$ to the ideal~$0 \oplus I$ and then identifying~$0 \oplus I$ with~$I$ to get~$\theta(x) \in \Der(I)$.
  This assignment may be written as the composition
  \[
    \theta
    \colon
    \hlie
    \xlongto{\sigma}
    \hlie \oplus I
    \xlongto{ \restrict{\ad(-)}{0 \oplus I} }
    \Der(0 \oplus I)
    \to
    \Der(I) \,.
  \]
  Each of the intermediate maps is a Lie~algebra homomorphism, so~$\theta$ is again a homomorphism.
  
  It remains to examine what further conditions such a homomorphism~$\theta$ needs to satisfy for the bracket~$[-,-]$ on~$\hlie \oplus I$ to satisfy the Jacobi identity.
  We claim that there are none.
  
  Indeed, we need to check that under the given constraints~($\kappa = 0$ and~$\theta$ is a homomorphism of Lie~algebras~$\hlie \to \Der(I)$) we already have the Jacobi identity
  \[
      [\alpha, [\beta, \gamma]]
    + [\beta, [\gamma, \alpha]]
    + [\gamma, [\alpha, \beta]]
    =
    0
  \]
  for all~$\alpha, \beta, \gamma \in \hlie \oplus I$.
  For this we use a trick taken from \cite{semidirect_jacobi}:
  Let us denote the left hand side of this equation by~$J(\alpha, \beta, \gamma)$.
  It follows from the bilinearity of~$[-,-]$ that~$J$ is trilinear.
  It also follows from~$[-,-]$ being alternating and thus skew-symmetric that~$J$ is again skew-symmetric.
  It therefore sufficies to check the condition~$J(\alpha, \beta, \gamma) = 0$ for the following four cases:
  \begin{itemize}
    \item
      If~$\alpha = (x,0)$,~$\beta = (y,0)$ and~$\gamma = (z,0)$ with~$x, y, z \in \hlie$ then
      \[
        [\alpha, [\beta, \gamma]]
        =
        [(x,0), [(y,0), (z,0)]]
        =
        [(x,0), ([y,z], 0)]
        =
        ([x,[y,z]], 0)
      \]
      and therefore
      \begin{align*}
        J(\alpha, \beta, \gamma)
        &=
          ([x,[y,z]], 0)
        + ([y,[z,x]], 0)
        + ([z,[x,y]], 0)
        \\
        &=
        ([x,[y,z]] + [y,[z,x]] + [z,[x,y]], 0)  \,.
      \end{align*}
      Hence~$J(\alpha, \beta, \gamma) = 0$ precisely because the Lie~bracket of~$\hlie$ satisfies the Jacobi identity.
    \item
      If~$\alpha = (x,0)$,~$\beta = (y,0)$ and~$\gamma = (0,c)$ with~$x, y \in \hlie$ and~$c \in I$ then
      \[
        [\alpha, [\beta, \gamma]]
        =
        [(x,0), [(y,0), (0,c)]]
        =
        [(x,0), [(0, \theta(y)(c)]]
        =
        ( 0, \theta(x)(\theta(y)(c)) )
      \]
      and similarly
      \[
        [\gamma, [\alpha, \beta]]
        =
        [(0,c), [(x,0), (y,0)]]
        =
        [(0,c), ([x,y], 0)]
        =
        ( 0, -\theta([x,y])(c) ) \,.
      \]
      It also follows that
      \[
        [\beta, [\gamma, \alpha]]
        =
        [(y,0), [(0,c), (x,0)]]
        =
        -[(y,0), [(x,0), (0,c)]]
        =
        ( 0, -\theta(y)(\theta(x)(c)) )
      \]
      and therefore
      \begin{align*}
        J(\alpha, \beta, \gamma)
        &=
          [(x,0), [(y,0), (0,c)]]
        + [(y,0), [(0,c), (x,0)]]
        + [(0,c), [(x,0), (y,0)]]
        \\
        &=
        ( 0, \theta(x)(\theta(y)(c)) - \theta([x,y])(c) - \theta(y)(\theta(x)(c)) ) \,.
      \end{align*}
      We hence find for this case that~$J(\alpha, \beta, \gamma) = 0$ precisely because~$\theta$ is a Lie~algebra homomorphism.
    \item
      If~$\alpha = (x,0)$,~$\beta = (0,c)$ and~$\gamma = (0,d)$ with~$x \in \hlie$ and~$c, d \in I$ then
      \[
        [\alpha, [\beta, \gamma]]
        =
        [(x,0), [(0,c), (0,d)]]
        =
        [(x,0), (0,[c,d])]
        =
        ( 0, \theta(x)([c,d]) )
      \]
      and similarly
      \[
        [\beta, [\gamma, \alpha]]
        =
        [(0,c), [(0,d), (x,0)]]
        =
        [(0,c), (0, -\theta(x)(d)]
        =
        ( 0, -[c,\theta(x)(d)] ) \,.
      \]
      It also follows that
      \begin{align*}
        [\gamma, [\alpha, \beta]]
        =
        [(0,d), [(x,0), (0,c)]]
        =
        - [(0,d), [(0,c), (x,0)]]
        &=
        ( 0, [d,\theta(x)(c)] )
        \\
        &=
        ( 0, -[\theta(x)(c), d] ) \,.
      \end{align*}
      It follows in combination that
      \begin{align*}
        J(\alpha, \beta, \gamma)
        &=
          [(x,0), [(0,c), (0,d)]]
        + [(0,c), [(0,d), (0,x)]]
        + [(0,d), [(x,0), (0,c)]]
        \\
        &=
        ( 0, \theta(x)([c,d]) - [c, \theta(x)(d)] - [\theta(x)(c), d] )
      \end{align*}
      which shows that~$J(\alpha, \beta, \gamma) = 0$ precisely because~$\theta(x)$ is a derivation of~$I$.
    \item
      If~$\alpha = (0,c)$,~$\beta = (0,d)$ and~$\gamma = (0,e)$ with~$c, d, e \in I$ then
      \[
        [\alpha, [\beta, \gamma]]
        =
        [(0,c), [(0,d), (0,e)]]
        =
        [(0,c), (0, [d,e])]
        =
        ( 0, [c,[d,e]] )
      \]
      and therefore
      \begin{align*}
        J(\alpha, \beta, \gamma)
        &=
          [(0,c), [(0,d), (0,e)]]
        + [(0,d), [(0,e), (0,c)]]
        + [(0,e), [(0,c), (0,d)]]
        \\
        &=
        ( 0, [c,[d,e]] + [d,[e,c]] + [e,[c,d]] )
      \end{align*}
      Which shows that~$J(\alpha, \beta, \gamma) = 0$ precisely because the Lie~bracket of~$I$ satisfies the Jacobi~identity.
  \end{itemize}
  
  We have now overall constructed for all Lie~algebras~$\hlie$ and~$I$ a {\onetoone} correspondence between
  \begin{itemize}
    \item
      Lie~brackets~$[-,-]$ on the vector space~$\hlie \oplus I$ that makes the standard short exact sequence
      \[
        0
        \to
        I
        \to
        \hlie
        \oplus
        I 
        \to
        \hlie
        \to
        0
      \]
      into a split extension of~$\hlie$ by~$I$ and
    \item
      Lie~algebra homomorphisms~$\theta \colon \hlie \to \Der(I)$,
  \end{itemize}
  and this correspondence is given by~$[(x,c), (y,d)] = ([x,y], \theta(x)(d) - \theta(y)(c) + [c,d])$.
\end{example}


\begin{definition}
  Let~$\hlie$ and~$I$ be two Lie~algebras and let~$\theta \colon \hlie \to \Der(I)$ be a homomorphism of Lie~algebras.
  The \defemph{semidirect product}~\gls*{semidirect product} of~$\hlie$ by~$I$ over~$\theta$ is the Lie~algebra~$\glie$ that is given by
  \begin{itemize}
    \item
      the underlying vector space~$\glie \defined \hlie \oplus I$ together with
    \item
      the Lie bracket~$[-,-]$ given by
      \[
        [(x,c), (y,d)]
        =
        ([x,y], \theta(x)(d) - \theta(y)(c) + [c,d])
      \]
      for all~$(x,c), (y,d) \in \glie$.
  \end{itemize}
\end{definition}


\begin{remark}
  Let~$\hlie$ and~$I$ be Lie~algebras and let~$\theta \colon \hlie \to \Der(I)$ be a Lie~algebra homomorphism.
  \begin{enumerate}
    \item
      It follows from the above discussion that~$\hlie \ltimes_\theta I$ is indeed a Lie algebra and that
      \[
        0
        \to
        I
        \xlongto{\iota}
        \hlie \ltimes_\theta I
        \xlongto{\pi}
        \hlie
        \to
        0
      \]
      is a split extension of~$\hlie$ by~$I$, where~$\iota(x) = (0,y)$ and~$\pi(x,y) = x$.
      Moreover, a split~$\sigma \colon \hlie \to \hlie \ltimes_\theta I$ of this extension is given by~$\sigma(x) = (x,0)$.
    \item
      We have also seen that whenever a Lie algebra~$\glie$ is a split extension of~$\hlie$ by~$I$ then~$\glie$ is already (equivalent to) a semidirect product of~$\hlie$ by~$I$ (over some suitable homomorphism of Lie~algebras~$\theta \colon \hlie \to \Der(I)$).
  \end{enumerate}
  The notions of split extensions and semidirect products are hence equivalent.
\end{remark}


\begin{remark}[Internal semidirect products]
  Given a Lie~algebra~$\glie$ we may ask ourselves if~$\glie$ can be decomposed as a semidirect product~$\glie \cong \hlie \ltimes_\theta I$;
  and if so, how~$I$ and~$\glie$ look like from the point of view of~$\glie$.
  
  We observe first that the semidirect product~$\glie = \hlie \ltimes_\theta I$ (where~$\theta \colon \hlie \to \Der(I)$ is some Lie~algebra homomorphism) contains the Lie~subalgebra~$\hlie' = \{ (x,0) \suchthat x \in \hlie \}$ and the ideal~$I' \defined \{ (0,c) \suchthat c \in I \}$.
  We have that~$\hlie' \cong \hlie$ and~$I' \cong I$ as Lie~algebras, and~$\glie = \hlie' \oplus I'$ as vector spaces.
  The homomorpism~$\theta \colon \hlie \to \Der(I)$ can be described by considering for~$x \in \hlie$ the corresponding element~$(x,0) \in \hlie'$, restricting the endomorphism~$\ad_{\glie}((x,0)) \in \Der(\glie)$ to an endomorphism~$\restrict{ \ad_{\glie}((x,0)) }{I'} \in \Der(I')$ and then using the isomorphism~$I' \cong I$ to arrive at~$\theta(x) \in \Der(I)$.
  
  Suppose on the other hand that~$\glie$ is any Lie~algebra and that~$\glie = \hlie \oplus I$ for some Lie~subalgebra~$\hlie$ of~$\glie$ and some ideal~$I$ of~$\glie$.
  Then~$\theta \colon \hlie \to \Der(I)$ given by~$x \mapsto \restrict{\ad_{\glie}(x)}{I}$ is a well-defined Lie algebra homomorphism, and we have that
  \begin{align*}
    [x + c, y + d]
    &=
    [x,y] + [x,d] + [c,y] + [c,d]
    \\
    &=
    [x,y] + [x,d] - [y,c] + [c,d]
    \\
    &=
    \underbrace{ [x,y] }_{\in \hlie} + \underbrace{ \theta(x)(d) - \theta(y)(c) + [c,d] }_{\in I}
  \end{align*}
  for all~$x, y \in \hlie$ and~$c, d \in I$.
  This shows that the vector space isomorphism
  \[
    \varphi
    \colon
    \hlie \oplus I
    \to
    \glie \,,
    \quad
    (x,c)
    \mapsto
    x + c
  \]
  is already an isomorphism of Lie algebras~$\varphi \colon \hlie \ltimes_\theta I \to \glie$.
\end{remark}


\begin{definition}
  Let~$\glie$ be a Lie~algebra, let~$\hlie$ be a Lie~subalgebra of~$\glie$ and let~$I$ be an ideal in~$\glie$.
  Then~$\glie$ is the \defemph{internal semidirect product}\index{internal semidirect product} of~$\hlie$ by~$I$ if~$\glie = \hlie \oplus I$.
  This is then denoted by~$\glie = \gls*{internal semidirect product}$.
\end{definition}


\begin{remark}
  The notion of an internal semidirect product does not depend on the additional information of a Lie~algebra homomorphism~$\theta \colon \hlie \to \Der(I)$.
  Instead this information is encoded in the Lie~algebra structure of~$\glie$, and related to the expected homomorphism~$\theta$ by~$\theta(x) = \restrict{\ad_{\glie}(x)}{I}$.
\end{remark}


\begin{example}
  The Lie algebra~$\glie = \tlie_n(\kf)$ can be written as a direct sum~$\tlie_n(\kf) = \dlie_n(\kf) \oplus \nlie_n(\kf)$ with~$\dlie_n(\kf)$ a Lie~subalgebra of~$\tlie_n(\kf)$ and~$\nlie_n(\kf)$ an ideal in~$\tlie_n(\kf)$.
  Hence
  \[
    \tlie_n(\kf)
    =
    \dlie_n(\kf) \ltimes \nlie_n(\kf) \,.
  \]
\end{example}


\begin{example}[Trivial extensions]
  \label{trivial extension is semidirect}
  For any two Lie~algebras~$I$ and~$\hlie$ the zero map~$\theta \colon \hlie \to \Der(I)$ is a homomorphism of Lie algebras.
  The resulting semidirect product~$\hlie \ltimes_\theta I$ is precisely the usual product~$\hlie \times I$.
  The corresponding split extension
  \[
    0
    \to
    I
    \xlongto{\iota}
    \hlie \times I
    \xlongto{\pi}
    \hlie
    \to
    0
  \]
  is given by canonical inclusion~$\iota(c) = (0,c)$ and the canonical projection~$\pi(x,y) = x$.
  This extension is the \emph{trivial extension}\index{trivial extension} of~$\hlie$ by~$I$.
  More generally any extension that is equivalent to the trivial extension is called \defemph{trivial}.
\end{example}


% \begin{remark}
%   \label{when semidirect is direct}
%   The converse to the above observation also holds:
%   Let~$I$ and~$\hlie$ be Lie~algebras and let~$\theta \colon \hlie \to \Der(I)$ be a homomorphism of Lie~algebras such that the resulting semidirect product~$\hlie \ltimes_\theta I$ is trivial.
%   Then already~$\theta = 0$.
%   
%   Indeed, if the extension
%   \[
%     0
%     \to
%     I
%     \xlongto{\iota}
%     \hlie \times I
%     \xlongto{\pi}
%     \hlie
%     \to
%     0
%   \]
%   is trivial
% 
% \end{remark}


\begin{warning}
  Let
  \[
    0
    \to
    I
    \xlongto{f}
    \glie
    \xlongto{g}
    \hlie
    \to
    0
  \]
  be an extension of Lie~algebras.
  The observant reader may have noticed that we have so far only considered splits~$s \colon \hlie \to \glie$, but not splits~$t \colon \glie \to I$.
  There is a good reason for this:
  
  Suppose that such a split exists, i.e.\ there exists a Lie~algebra homomorphism~$t \colon \glie \to I$ with~$t {} f = \id_I$.
  Then~$J \defined \ker t$ is an ideal in~$\glie$.
  If~$I'$ denotes the image of~$I$ in~$\glie$ then it follows that~$\glie = I' \oplus J$, the homomorphism~$f$ restricts to an isomorphism~$I \to I'$ and the homomorphism~$g$ restricts to an isomorphism~$J \to \hlie$.
  This means that the given extension is equivalent to the trivial extension, and hence trivial itself.
  
  We see on the other hand that the trivial extension admits both kinds of split, and thus the same holds for every extension that is trivial.
  
  This means altogether that an extension~$0 \to I \to \glie \to \hlie \to 0$ admits a split~$\glie \to I$ (of Lie~algebras) if and only if the extension is already trivial.
\end{warning}




