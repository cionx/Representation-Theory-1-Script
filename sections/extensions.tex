\chapter{Extensions}


% TODO: More general notion of equivalence of extensions

% TODO: Classification of semidirect products and central extensions
% (What did I mean by this?)





\section{General Extensions}

\begin{definition}
  A \defemph{short exact sequence of Lie~algebras}\index{short exact sequence!of Lie algebras} is a short exact sequence
  \begin{equation}
    \label{general extension}
    0 
    \to
    I
    \xto{\varphi}
    \glie
    \xto{\psi}
    \hlie
    \to
    0
  \end{equation}
  of vector spaces where~$I$,~$\glie$,~$\hlie$ are Lie algebras and both~$\varphi$ and~$\psi$ are homomorphisms of Lie algebras.
  The short exact sequence~\eqref{general extension} is an~\defemph{extension}\index{extension!of Lie algebras} of~$\hlie$ by~$I$.
\end{definition}


\begin{remark}
  \leavevmode
  \begin{enumerate}
    \item
      By abuse of notation we often say that \enquote{$\glie$ is an extension of~$\hlie$ by~$I$} when talking about an extension
      of~$\hlie$ by~$I$ of the form
      \[
        0 
        \to
        I
        \to
        \glie
        \to
        \hlie
        \to
        0 \,.
      \]
    \item
      Let
      \[
        0
        \to
        I
        \xto{\varphi}
        \glie
        \xto{\psi}
        \hlie
        \to
        0
      \]
      be a short exact sequence of Lie~algebras.
      Then the homomorphism~$\varphi$ is injective and identifies the Lie~algebra~$I$ with the kernel of~$\psi$, which is an ideal in~$\glie$.
    \item
      Let~$\hlie$ and~$I$ be two~\liealgebra{$\kf$}.
      An extension~$\glie$ of~$\hlie$ by~$I$ encodes the following information:
      the Lie~algebra structure of~$I$, the Lie~algebra structure of~$\hlie$, and some kind of interaction between~$\hlie$ and~$I$.
      How exactly these informations are encoded into~$\glie$ relies on the specific choice of homomorphism~$\varphi$ and~$\psi$ that are part of the extension.
  \end{enumerate}
\end{remark}


\begin{example}
  Let~$I$ is an ideal in a Lie~algebra~$\glie$.
  Let~$\iota$ be the inclusion from~$I$ to~$\glie$ and let~$\pi$ be the canonical projection from~$\glie$ to~$\glie/I$.
  Then
  \[
    0
    \to
    I
    \xto{\iota}
    \glie
    \xto{\pi}
    \glie/I
    \to
    0
  \]
  is a short exact sequence of Lie~algebras.
\end{example}


\begin{definition}
  \label{equivalence of extensions}
  Let~$I$ and~$\hlie$ be two Lie~algebras and let
  \[
    0 
    \to
    I
    \xto{\varphi}
    \glie
    \xto{\psi}
    \hlie
    \to
    0
    \qquad\text{and}\qquad
    0 
    \to
    I
    \xto{\varphi'}
    \glie'
    \xto{\psi'}
    \hlie
    \to
    0
  \]
  be two extensions of~$\hlie$ by~$I$.
  These extensions are \defemph{equivalent}\index{equivalent extensions} if there exists an isomorphism of Lie~algebras~$\alpha$ from~$\glie$ to~$\glie'$ that makes the following diagram commute:
  \[
    \begin{tikzcd}[column sep = large]
      0
      \arrow{r}
      &
      I
      \arrow{r}[above]{\varphi}
      \arrow[equal]{d}
      &
      \glie
      \arrow{r}[above]{\psi}
      \arrow[dashed]{d}[right]{\alpha}
      &
      \hlie
      \arrow{r}
      \arrow[equal]{d}
      &
      0
      \\
      0
      \arrow{r}
      &
      I
      \arrow{r}[above]{\varphi'}
      &
      \glie'
      \arrow{r}[above]{\psi'}
      &
      \hlie
      \arrow{r}
      &
      0
    \end{tikzcd}
  \]
\end{definition}


\begin{proposition}
  Let~$I$ and~$\hlie$ be two Lie~algebras.
  Equivalence of extensions is an equivalence relation on the class of extensions of~$\hlie$ by~$I$.
  \qed
\end{proposition}


\begin{remark}
  In \cref{equivalence of extensions} it sufficies to require for~$\varphi$ to be a homomorphism of Lie~algebras.
  It then follows from the five lemma that it is already an isomorphism of vector spaces, and thus an isomorphism of Lie~algebras.
\end{remark}


\begin{fluff}
  In the following we want to classify certain classes of extensions.
  But this needs some preparation.
  We will start by showing that every extensions is, up to equivalence, of a certain standard form.
\end{fluff}


\begin{definition}
  Let~$\hlie$ and~$I$ be two~\liealgebras{$\kf$}.
  An extension of~$\hlie$ by~$I$ is of \defemph{standard form}\index{extension!standard form}, or simply \defemph{standard}\index{extension!standard}%
  \footnote{
    This is not a standard definition.
    It is made up by the author of these notes to make the following discussion more readable.
  }, if its underlying short exact sequence of vector spaces is given by
  \[
    0
    \to
    I
    \xto{\iota}
    \hlie \oplus I
    \xto{\pi}
    \hlie
    \to
    0 \,,
  \]
  where~$\iota$ denotes the inclusion, given by~$c \mapsto (0,c)$, and~$\pi$ denotes the canonical projection, given by~$(x,c) \mapsto x$.
\end{definition}


\begin{warning}
  In a standard extension of~$\hlie$ by~$I$ the middle term~$\hlie \oplus I$ carries the structure of a Lie~algebra which makes the inclusion~$\iota$ and projection~$\pi$ into homomorphisms of Lie~algebras.
  But this Lie~algebra structure on the vector space~$\hlie \oplus I$ does not have to be given by the componentwise Lie~bracket of the direct sum!
  We will see in \cref{structure of extensions} how such Lie~bracket on~$\hlie \oplus I$ has to look like.
\end{warning}


\begin{proposition}
  \label{every extension is equivalent to a standard one}
  Let~$\hlie$ and~$I$ be two~\liealgebras{$\kf$}.
  Every extension of~$\hlie$ by~$I$ is equivalent to a standard extension.
\end{proposition}


\begin{proof}
  Let
  \[
    0
    \to
    I
    \xto{\varphi}
    \glie
    \xto{\psi}
    \hlie
    \to
    0
  \]
  be an extension of~$\hlie$ by~$I$.
  This extension is in particular a short exact sequence of vector spaces, and it splits on the level of vector spaces.
  There hence exists an isomorphism of vector spaces~$\alpha$ from~$\glie$ to~$\hlie \oplus I$ that makes the diagram
  \[
    \begin{tikzcd}[column sep = large]
      0
      \arrow{r}
      &
      I
      \arrow{r}[above]{\varphi}
      \arrow[equal]{d}
      &
      \glie
      \arrow{r}[above]{\psi}
      \arrow[dashed]{d}[right]{\alpha}
      &
      \hlie
      \arrow{r}
      \arrow[equal]{d}
      &
      0
      \\
      0
      \arrow{r}
      &
      I
      \arrow{r}[above]{\iota}
      &
      \hlie \oplus I
      \arrow{r}[above]{\pi}
      &
      \hlie
      \arrow{r}
      &
      0
    \end{tikzcd}
  \]
  commute.
  We can pushforward the Lie~bracket of~$\glie$ to a Lie~bracket on~$\hlie \oplus I$ along~$\alpha$.
  This makes~$\hlie \oplus I$ into a Lie~algebra and the map~$\alpha$ into an isomorphism of Lie~algebras.
  It follows from the commutativity of the above diagram that
  \[
    \iota = \alpha \circ \varphi \,,
    \quad
    \pi = \psi \circ \alpha^{-1}
  \]
  are again homomorphisms of Lie~algebras.
  We hence find that
  \[
    0
    \to
    I
    \xto{\iota}
    \hlie \oplus I
    \xto{\pi}
    \hlie
    \to
    0
  \]
  is a standard extension of~$\hlie$ by~$I$.
  This standard extension is equivalent to the original extension by the commutativity of the above diagram.
\end{proof}


\begin{fluff}
  \label{general approach to extensions}
  Let~$I$ and~$\hlie$ be Lie~algebras.
  We will now investigate how the Lie~bracket on a standard extension of~$\hlie$ by~$I$ has to look like.
  We will then also express und what conditions two standard extensions are equivalent.
  The following discussion proceeds in multiple steps.
  \begin{enumerate}
    \item
      We consider a standard extension
      \[
        0
        \to
        I
        \xto{\iota}
        \hlie \oplus I
        \xto{\pi}
        \hlie
        \to
        0 \,.
      \]
      We want to understand how the Lie~bracket on~$\hlie \oplus I$ has to look like.
      It follows from the bilinearity of the Lie~bracket~$[-,-]$ on~$\hlie \oplus I$ that
      \begin{align*}
        {}&
        [(x, c), (y,d)]
        \\
        ={}&
          [(x,0), (y,0)]
        + [(x,0), (0,d)]
        + [(0,c), (y,0)]
        + [(0,c), (0,d)]
        \\
        ={}&
          [(x,0), (y,0)]
        + [(x,0), (0,d)]
        - [(y,0), (0,c)]
        + [(0,c), (0,d)]
      \end{align*}
      for all~$(x,c), (y,d) \in \hlie \oplus I$.

      We observe that
      \[
        \pi( [(x,0), (y,0)] )
        =
        [ \pi( (x,0) ), \pi( (y,0) ) ]
        =
        [x, y]
      \]
      because~$\pi$ is a homorphism of Lie~algebras.
      The commutator~$[(x,0), (y,0)]$ is therefore of the form
      \[
        [(x,0), (y,0)]
        =
        ( [x,y], \kappa(x,y) )
      \]
      for all~$x, y \in \hlie$ for some function
      \[
        \kappa
        \colon
        \hlie \times \hlie
        \to
        I \,.
      \]
      It follows from the bilinearity of the Lie~bracket~$[-,-]$ on~$\hlie \oplus I$ that this map~$\kappa$ is again bilinear.

      We note that the two commutators~$[(x,0), (0,c)]$ and~$[(y,0), (0,d)]$ are again contained in~$0 \oplus I$ because this is an ideal in~$\hlie \oplus I$, namely the kernel of the homomorphism~$\pi$.
      It follows together with the bilinearity of the Lie~bracket~$[-,-]$ on~$\hlie \oplus I$ that there exists a unique linear map
      \[
        \theta'
        \colon
        \hlie
        \to
        \gllie(I)
      \]
      with
      \[
        [(z,0), (0,e)]
        =
        [0, \theta(z)(e)]
      \]
      for all~$z \in \hlie$,~$e \in I$.
      We have seen in \cref{lie algebras act adjoint by derivations} that the map~$[(z,0), -] = \ad_{\hlie \oplus I}((z,0))$ is a derivation of~$\hlie \oplus I$.
      It follows that the map~$\theta'$ corestricts to a linear map
      \[
        \theta
        \colon
        \hlie
        \to
        \Der(I) \,.
      \]
      For this map~$\theta$ we now have
      \[
        [(x,0), (0,d)]
        =
        [0, \theta(x)(d)]
        \quad\text{and}\quad
        [(y,0), (0,e)]
        =
        [0, \theta(y)(e)] \,.
      \]

      The last remaining commutator~$[(0,c), (0,d)]$ can be computed by using that the inclusion~$\iota$ is a homomorphism of Lie~algebras from~$I$ to~$\hlie \oplus I$, and therefore
      \[
        [(0,c), (0,d)]
        =
        [\iota(c), \iota(d)]
        =
        \iota([c,d])
        =
        (0, [c,d]) \,.
      \]

      We find altogether that the Lie bracket on~$\hlie \oplus I$ can be expressed with the help of the two maps~$\kappa$ and~$\theta$ as
      \begin{align*}
        {}&
        [ (x,c), (y,d) ]
        \\
        ={}&
        [(x,0), (y,0)]
        + [(x,0), (0,d)]
        - [(y,0), (0,c)]
        + [(0,c), (0,d)]
        \\
        ={}&
          ( [x,y], \kappa(x,y) )
        + ( 0, \theta(x)(d) )
        - ( 0, \theta(y)(c) )
        + ( 0, [c,d] )
        \\
        ={}&
        (
          [x,y],
          \kappa(x,y) + \theta(x)(d) - \theta(y)(c) + [c,d]
        ) \,.
      \end{align*}
      The Lie~algebra structure of the extension~$\hlie \oplus I$ is therefore uniquely described by the two maps~$\kappa$ and~$\theta$.
    \item
      In the second step of this discussion we want to better understand the above map~$\kappa$ and~$\theta$.
      To do so, let
      \[
        \kappa
        \colon
        \hlie \times \hlie
        \to
        I
      \]
      be any bilinear map, and let
      \[
        \theta
        \colon
        \hlie
        \to
        \Der(I)
      \]
      be any linear map.
      We define a bracket~$[-,-]$ on the vector space~$\hlie \oplus I$ via the previously deduced formula, i.e. via
      \begin{equation}
        \label{definition of lie bracket depending on kappa and theta}
        [(x,c), (y,d)]
        \defined
        (
          [x,y] ,
          \kappa(x,y) + \theta(x)(d) - \theta(y)(c) + [c,d]
        )
      \end{equation}
      for all~$(x,c), (y,d) \in \hlie \oplus I$.
      This bracket~$[-,-]$ on~$\hlie \oplus I$ is bilinear, and both the inclusion~$\iota$ from~$I$ to~$\hlie \oplus I$ and the canonical projection~$\pi$ from~$\hlie \oplus I$ to~$\hlie$ respect this bracket.
      There is only thing missing for
      \[
        0
        \to
        I
        \xto{\iota}
        \hlie \oplus I
        \xto{\pi}
        \hlie
        \to
        0
      \]
      to be an extension of Lie~algebras, namely that the bracket~$[-,-]$ on~$\hlie \oplus I$ makes it into a Lie~algebra, i.e. that it is actually a Lie~bracket.
      For this we need~$[-,-]$ to be alternating and to satisfy the Jacobi identity.

      We have for every element~$(x,c)$ of~$\glie \oplus I$ that
      \[
        [(x,c), (x,c)]
        =
        ( [x,x], \kappa(x,x) + \theta(x)(c) - \theta(x)(c) + [c,c] )
        =
        (0, \kappa(x,x) ) \,.
      \]
      We therefore find that the bracket~$[-,-]$ on~$\hlie \oplus I$ is alternating if and only if the bilinear map~$\kappa$ is alternating.

      We can also express the Jacobi identity for the bracket~$[-,-]$ on~$\hlie \oplus I$ in terms of conditions on~$\kappa$ ond~$\theta$, as we will now painstakingly demonstrate.
      We first calculate the iterated commutator
      \[
        [(x,c), [(y,d), (z,e)]]
      \]
      for all~$(x,c), (y,d), (z,e) \in \hlie \oplus I$ via the formula~\eqref{definition of lie bracket depending on kappa and theta} as
      \begin{align}
        % first term
        {}&
        [(x,c), [(y,d), (z,e)]]
        \notag
        \\
        % second torm
        ={}&
        [(x,c), ([y,z], \kappa(y,z) + \theta(y)(e) - \theta(z)(d) + [d,e])]
        \notag
        \\
        % third term
        ={}&
        \bigl(
          [x, [y, z]], 
        \notag
        \\
        {}&
          \kappa( x, [y, z] )
          + \theta(x)
            \bigl(
              \kappa(y,z) + \theta(y)(e) - \theta(z)(d) + [d,e]
            \bigr)
        \notag
        \\
        {}&
        \hphantom{\biggl(}
          - \theta([y,z])(c)
          + \bigl[
              c,
              \kappa(y,z) + \theta(y)(e) - \theta(z)(d) + [d,e]
            \bigr]
        \bigr)
        \notag
        \\
        % fourth term
        ={}&
        \smash{\bigl(}
          [x, [y, z]],
        \notag
        \\
        {}&
        \hphantom{\bigl(}
          \kappa( x, [y, z] )
          + \theta(x)( \kappa(y,z) )
        \label{ugly row 1}
        \\
        {}&
        \hphantom{\bigl(}
          + \theta(x)( \theta(y)(e) )
          - \theta(x)( \theta(z)(d) )
        \label{ugly row 2}
        \\
        {}&
        \hphantom{\bigl(}
          + \theta(x)( [d,e] )
        \label{ugly row 3}
        \\
        {}&
        \hphantom{\bigl(}
          - \theta( [y,z] )(c)
        \label{ugly row 4}
        \\
        {}&
        \hphantom{\bigl(}
          + [c, \kappa(y,z)]
        \label{ugly row 5}
        \\
        {}&
        \hphantom{\bigl(}
          + [c, \theta(y)(e)]
          - [c, \theta(z)(d)]
        \label{ugly row 6}
        \\
        {}&
        \hphantom{\bigl(}
          + [c, [d, c]]
        \bigr) \,.
        \label{ugly row 7}
      \end{align}
      For every expression~$F$ in the three arguments~$(x,c)$,~$(y,d)$,~$(z,e)$ of~$\hlie \oplus I$ we will in the following use the abbreviation
      \[
        \sum_{\cyc} F( (x,c), (y,d), (z,e) )
      \]
      to denote the cyclic sum
      \[
        F( (x,c), (y,d), (z,e) )
        + F( (y,d), (z,e), (x,c) )
        + F( (z,e), (x,c), (y,d) ) \,.
      \]
      We want to examine which conditions the two maps~$\kappa$ and~$\theta$ need to satisfy to ensure that
      \[
        \sum_{\cyc}
        [(x,c), [(y,d), (z,e)]]
        =
        0
        \qquad
        \text{for all~$(x,c), (y,d), (z,e) \in \hlie \oplus I$.}
      \]
      We see from~\eqref{ugly row 1} that the first entry of~$\sum_{\cyc} [(x,c), [(y,d), (z,e)]]$ is given by the cyclic sum
      \[
        \sum_{\cyc} [x,[y,z]] \,.
      \]
      But this terms vanishes because the Lie~bracket of~$\hlie$ satisfies the Jacobi identity.
      We therefore only have to worry about the second entry of~$\sum_{\cyc} [(x,c), [(y,d), (z,e)]]$.
      For this second entry we can make the following observations.
      \begin{itemize*}
        \item
          For term~\eqref{ugly row 7} we have
          \[
            \sum_{\cyc} [c,[d,e]] = 0
          \]
          because the Lie~bracket on~$I$ satisfies the Jacobi identity.
          We therefore do not have to worry about term~\eqref{ugly row 7}.
        \item
          Term~\eqref{ugly row 3} can be expanded as
          \[
            \theta(x)([d,e])
            =
            [\theta(x)(d), e] + [d, \theta(x)(e)]
          \]
          because~$\theta(x)$ is a derivation on~$I$.
          We have
          \[
            \sum_{\cyc} [\theta(x)(d), e]
            =
            \sum_{\cyc} [\theta(y)(e), c]
            =
            -\sum_{\cyc} [c, \theta(y)(e)]
          \]
          and similarly
          \[
            \sum_{\cyc} [d, \theta(x)(e)]
            =
            \sum_{\cyc} [c, \theta(z)(d)] \,.
          \]
          With this we find that the cyclic sums of term~\eqref{ugly row 3} and term~\eqref{ugly row 6} cancel out.
          We therefore do not have to worry about the terms~\eqref{ugly row 3} and~\eqref{ugly row 6}.
        \item
          In the special case~$c, d, e = 0$ all terms except~\eqref{ugly row 1} vanish.
          We therefore find that~$\kappa$ and~$\theta$ have to satisfy the condition.
          \[
            \sum_{\cyc}
            (
              \kappa(x, [y,z])
              + \theta(x)( \kappa(y,z))
            )
            =
            0 \,.
          \]
          By using that~$\kappa$ is alternating, und therefore anti-symmetric, we may rewrite this condition as
          \begin{align*}
            0
            &=
            \sum_{\cyc}
            (
              \kappa(x, [y,z])
              + \theta(x)( \kappa(y,z))
            )
            \\
            &=
            \sum_{\cyc} \kappa(x, [y,z])
            +
            \sum_{\cyc} \theta(x)( \kappa(y,z))
            \\
            &=
            -\sum_{\cyc} \kappa([y,z], x)
            +
            \sum_{\cyc} \theta(x)( \kappa(y,z))
            \\
            &=
            -\sum_{\cyc} \kappa([x,y], z)
            +
            \sum_{\cyc} \theta(x)( \kappa(y,z)) \,,
          \end{align*}
          and therefore as
          \[
            \sum_{\cyc} \kappa([x,y], z)
            =
            \sum_{\cyc} \theta(x)( \kappa(y,z) ) \,.
          \]
          We have thus arrived at the condition
          \begin{equation}
            \label{nonabelian cocycle condition}
            \begin{aligned}
              {}&
              \kappa([x,y], z)
              + \kappa([y, z], x)
              + \kappa([z, x], y)
              \\
              ={}&
              \theta(x)( \kappa(y,z) )
              + \theta(y)( \kappa(z,x) )
              + \theta(z)( \kappa(x,y) )
            \end{aligned}
          \end{equation}
          for all~$x, y, z \in \hlie$.

          We have derived this condition on~$\kappa$ and~$\theta$ by considering the special case~$c, d, e = 0$.
          However, as long as the condition~\eqref{nonabelian cocycle condition} is satisfied, term~\eqref{ugly row 1} will vanish in the cyclic sum of~$[(x,c), [(y,d), (z,e)]]$ for arbitrary~$c, d, e \in I$.
          So as long as condition~\eqref{nonabelian cocycle condition} is satisfied we do not have to worry about term~\eqref{ugly row 1}.
        \item
          It remains to consider the terms~\eqref{ugly row 2},~\eqref{ugly row 4} and~\eqref{ugly row 5}.
          For those terms we calculate that
          \begin{align}
            {}&
            \sum_{\cyc}
            \Bigl(
              \theta(x)( \theta(y)(e) )
              -
              \theta(x)( \theta(z)(d) )
              -
              \theta([y,z])(c)
              +
              [c, \kappa(y,z)]
            \Bigr)
            \label{unrotated condition on kappa and theta}
            \\
            ={}&
            \sum_{\cyc}
            \theta(x)( \theta(y)(e) )
            -
            \sum_{\cyc}
            \theta(x)( \theta(z)(d) )
            -
            \sum_{\cyc}
            \theta([y,z])(c)
            +
            \sum_{\cyc}
            [c, \kappa(y,z)]
            \notag
            \\
            ={}&
            \sum_{\cyc}
            \theta(x)( \theta(y)(e) )
            -
            \sum_{\cyc}
            \theta(y)( \theta(x)(e) )
            -
            \sum_{\cyc}
            \theta([x,y])(e)
            +
            \sum_{\cyc}
            [e, \kappa(x,y)]
            \notag
            \\
            ={}&
            \sum_{\cyc}
            \Bigl(
              \theta(x)( \theta(y)(e) )
              -
              \theta(y)( \theta(x)(e)
              -
              \theta([x,y])(e)
              +
              [e, \kappa(x,y)]
            \Bigr) \,.
            \label{rotated conditon on kappa and theta}
          \end{align}
          We note that in the cyclic sum~\eqref{rotated conditon on kappa and theta} each summand depends on only of the three variables~$c$,~$d$,~$e$, und it does so in a linear way.
          It follows that the sum~\eqref{rotated conditon on kappa and theta} vanishes for arbitrary choices of~$(x,c)$,~$(y,d)$,~$(z,e)$ if and only if each summand vanishes.
          (Indeed, we can isolate one of the three summands by considering the special case~$c, d = 0$.
          From this we see that this summand has to vanish on its own.
          But we can also isolate the other two summands by consider the special cases~$c, e = 0$ and~$d, e = 0$.
          So each summand has to vanish on its own.)

          All three summands give the same condition, namely that
          \[
            \theta(x)( \theta(y)(e) )
            -
            \theta(y)( \theta(x)(e)
            -
            \theta([x,y])(e)
            +
            [e, \kappa(x,y)]
            =
            0
          \]
          \text{for all~$x, y \in \hlie$,~$e \in I$.}
          We may rewrite this condition as
          \begin{equation}
            \label{rewritten condition on kappa and theta}
            \theta([x,y])
            =
            [\theta(x), \theta(y)]
            -
            \ad_I( \kappa(x,y) )
            \qquad
            \text{for all~$x, y \in \hlie$.}
          \end{equation}
      \end{itemize*}
      We have altogether shown that the bracket~$[-,-]$ on~$\hlie \oplus I$ satisfies the Jacobi identity if and only if it satisfies the two conditions~\eqref{nonabelian cocycle condition} and~\eqref{rewritten condition on kappa and theta}.

      We have now shown that the bracket~$[-,-]$ on~$\hlie \oplus I$ is a Lie~bracket if and only if the map~$\kappa$ is alternating and the two maps~$\kappa$ and~$\theta$ satisfy the compatibility conditions~\eqref{nonabelian cocycle condition} and~\eqref{rewritten condition on kappa and theta}
  \end{enumerate}
  We have overall constructed a bijection between Lie~brackets on~$\hlie \oplus I$ which make the short exact sequence
  \[
    0
    \to
    I
    \xto{\iota}
    \hlie \oplus I
    \xto{\pi}
    \hlie
    \to
    0
  \]
  into a (standard) extension of~$\hlie$ by~$I$ and certains pairs~$(\kappa, \theta)$ consisting of maps~$\kappa$ from~$\hlie \times \hlie$ to~$I$ and~$\theta$ from~$\hlie$ to~$\Der(I)$.

  We now wish to understand under what conditions two pairs~$(\kappa_1, \theta_1)$ and~$(\kappa_2, \theta_2)$ give equivalent extensions.
  We proceed again in multiple steps.
  \begin{enumerate}[resume*]
    \item
      Let~$(\kappa_1, \theta_1)$ and~$(\kappa_2, \theta_2)$ be two pairs of maps
      \[
        \kappa_1, \kappa_2
        \colon
        \hlie \times \hlie
        \to
        I \,,
        \quad
        \theta_1, \theta_2
        \colon
        \hlie
        \to
        \Der(I)
      \]
      such that the maps~$\kappa_1$ and~$\kappa_2$ are bilinear and alternating, the maps~$\theta_1$ and~$\theta_2$ are linear, and the compatibility conditions \eqref{nonabelian cocycle condition} and~\eqref{rewritten condition on kappa and theta} are satisfied for both~$\kappa_1$ and~$\theta_1$, as well as~$\kappa_2$ and~$\theta_2$.
      Let~$[-,-]_1$ and~$[-,-]_2$ be the Lie~brackets on~$\hlie \oplus I$ corresponding to the pairs~$(\kappa_1, \theta_1)$ and~$(\kappa_2, \theta_2)$.
      More explicitely, we have
      \[
        [(x,c), (y,d)]_i
        =
        (
          [x,y],
          \kappa_i(x,y) + \theta_i(x)(d) - \theta_i(y)(c) + [c,d]
        )
      \]
      for all~$(x,c), (y,d) \in \hlie \oplus I$,~$i = 1, 2$.

      These two resulting extensions of~$\hlie$ by~$I$ are equivalent if and only if there exists an homomorphism of Lie~algebras
      \[
        \alpha
        \colon
        ( \hlie \oplus I, [-,-]_1 )
        \to
        ( \hlie \oplus I, [-,-]_1 )
      \]
      that makes the following diagram commute:
      \begin{equation}
        \label{diagram for equivalence of standard extensions}
        \begin{tikzcd}
          0
          \arrow{r}
          &
          I
          \arrow{r}[above]{\iota}
          \arrow[equal]{d}
          &
          \hlie \oplus I
          \arrow{r}[above]{\pi}
          \arrow[dashed]{d}[right]{\alpha}
          &
          \hlie
          \arrow{r}
          \arrow[equal]{d}
          &
          0
          \\
          0
          \arrow{r}
          &
          I
          \arrow{r}[above]{\iota}
          &
          \hlie \oplus I
          \arrow{r}[above]{\pi}
          &
          \hlie
          \arrow{r}
          &
          0
        \end{tikzcd}
      \end{equation}
      We will now explain under what conditions on the pairs~$(\kappa_1, \theta_1)$ and~$(\kappa_2, \theta_2)$ such an isomorphism~$\alpha$ exists, and how it can be constructed.
    \item
      We first want to understand how a linear map
      \[
        \alpha
        \colon
        \hlie \oplus I
        \to
        \hlie \oplus I
      \]
      has to look like to make the diagram~\eqref{diagram for equivalence of standard extensions} commute.
      The maps~$\iota$,~$\pi$ and~$\alpha$ are linear maps between direct sums, and can therefore be written in the usual matrix calculus as
      \[
        \iota
        =
        \begin{pmatrix}
          0
          \\
          1
        \end{pmatrix} \,,
        \quad
        \alpha
        =
        \begin{pmatrix}
          \alpha_{11} & \alpha_{12} \\
          \alpha_{21} & \alpha_{22}
        \end{pmatrix} \,,
        \quad
        \pi
        =
        \begin{pmatrix}
          1 & 0
        \end{pmatrix}
      \]
      for some unique linear maps
      \[
        \alpha_{11}
        \colon
        \hlie
        \to
        \hlie \,,
        \quad
        \alpha_{12}
        \colon
        \hlie
        \to
        I \,,
        \quad
        \alpha_{21}
        \colon
        I
        \to
        \hlie \,,
        \quad
        \alpha_{22}
        \colon
        I
        \to
        I \,.
      \]
      The commutativity of the diagram~\eqref{diagram for equivalence of standard extensions} can be encoded by the two conditions
      \[
        \begin{pmatrix}
          \alpha_{11} & \alpha_{12} \\
          \alpha_{21} & \alpha_{22}
        \end{pmatrix}
        \begin{pmatrix}
          0 \\
          1
        \end{pmatrix}
        =
        \begin{pmatrix}
          0 \\
          1
        \end{pmatrix} \,,
        \quad
        \begin{pmatrix}
          1 & 0
        \end{pmatrix}
        \begin{pmatrix}
          \alpha_{11} & \alpha_{12} \\
          \alpha_{21} & \alpha_{22}
        \end{pmatrix}
        =
        \begin{pmatrix}
          1 & 0
        \end{pmatrix} \,.
      \]
      We can apply the usual rules of matrix multiplication to find that these two conditions are furthermore equivalent to the combination of the three conditions
      \[
        \alpha_{11} = 1 \,,
        \quad
        \alpha_{12} = 0 \,,
        \quad
        \alpha_{22} = 1 \,.
      \]
      We have thus found that the diagram~\eqref{diagram for equivalence of standard extensions} commutes if and only if the linear map~$\alpha$ is of the form
      \[
        \alpha
        =
        \begin{pmatrix}
          1       & 0 \\
          \varphi & 1
        \end{pmatrix}
      \]
      for some linear map
      \[
        \varphi
        \colon
        \hlie
        \to
        I \,.
      \]
      This means more explicitely that
      \[
        \alpha(x,c)
        =
        \alpha(x, \varphi(x) + c)
        \qquad
        \text{for all~$(x,c) \in \hlie \oplus I$.}
      \]
      Now we understand how~$\alpha$ has to look like to make the diagram~\eqref{diagram for equivalence of standard extensions} commute.
    \item
      We will now examine what condition the linear map~$\varphi$ has to satisfy in terms of the two pairs~$(\kappa_1, \theta_1)$ and~$(\kappa_2, \theta_2)$ to be a homomorphism of Lie~algebras.
      The map~$\alpha$ is such a homomorphism if and only if the equality
      \[
        \alpha( [ t, u ]_1 )
        =
        [ t, u ]_2
      \]
      is satisfied for any two elements~$t$,~$u$ of~$\hlie \oplus I$.
      By the bilinearity of the Lie~brackets~$[-,-]_1$ and~$[-,-]_2$ and the linearity of~$\alpha$ it sufficies to consider the cases in which~$t$ is of the form~$(x,0)$ or~$(0,c)$, and similarly~$u$ is of the form~$(y,0)$ or~$(0,d)$.
      \begin{itemize}
        \item
          For~$t = (x,0)$ and~$u = (y,0)$ we find on the one hand
          \[
            \alpha( [t,u] )
            =
            \alpha( [ (x,0) , (y,0) ] )
            =
            \alpha( [x,y], \kappa_1(x,y) )
            =
            ( [x,y], \varphi( [x,y] ) + \kappa_1(x,y) )
          \]
          and on the other hand
          \begin{align*}
            [ \alpha(t), \alpha(u) ]
            &=
            [ \alpha(x,0), \alpha(y,0) ]
            \\
            &=
            [ (x, \varphi(x)), (y, \varphi(y)) ]
            \\
            &=
            ( [x,y], \kappa_2(x,y) + \theta_2(x)(\varphi(y)) - \theta_2(y)(\varphi(x)) + [\varphi(x), \varphi(y)] ) \,.
          \end{align*}
          We thus arrive in this case at the condition
          \[
            \kappa_1(x,y)
            =
            \kappa_2(x,y)
            + \theta_2(x)(\varphi(y))
            - \theta_2(y)(\varphi(x))
            + [\varphi(x), \varphi(y)]
            - \varphi( [x,y] )
          \]
          for all~$x, y \in \hlie$.
        \item
          For~$t = (x,0)$ and~$u = (0,d)$ we find on the one hand
          \[
            \alpha( [v,w] )
            =
            \alpha( [ (x,0), (0,d) ] )
            =
            \alpha( 0, \theta_1(x)(d) )
            =
            (0, \theta_1(x)(d))
          \]
          and on the other hand
          \[
            [ \alpha(t), \alpha(u) ]
            =
            [ \alpha(x,0), \alpha(0,d) ]
            =
            [ (x, \varphi(x)), (0,d) ]
            =
            ( 0, \theta_2(x)(d) + [\varphi(x), d] ) \,.
          \]
          We therefore arrive in this case at the condition
          \begin{equation}
            \label{first occurance of condition}
            \theta_1(x)(d)
            =
            \theta_2(x)(d)
            + [\varphi(x), d]
          \end{equation}
          for all~$x \in \hlie$,~$d \in I$.
        \item
          In the case of~$t = (0,c)$ and~$u = (y,0)$ we find on the one hand
          \[
            \alpha( [t,u] )
            =
            \alpha( [ (0,c), (y,0) ] )
            =
            \alpha( 0, -\theta_1(y)(c) )
            =
            ( 0, -\theta_1(y)(c) )
          \]
          and on the other hand
          \[
            [ \alpha(t), \alpha(u) ]
            =
            [ \alpha(0,c), \alpha(y,0) ]
            =
            [ (0,c), (y, \varphi(y) ]
            =
            ( 0, -\theta_2(y)(c) + [c, \varphi(y)] ) \,.
          \]
          We therefore arrive in this case at the condition
          \[
            \theta_1(y)(c)
            =
            \theta_2(y)(c) + [\varphi(y), c]
          \]
          for all~$y \in \hlie$,~$c \in I$.
          But this is the same condition as in the previous case, i.e. condition~\eqref{first occurance of condition}.
        \item 
          In the case of~$t = (0,c)$ and~$u = (0,d)$ we have on the one hand
          \[
            \alpha( [t,u] )
            =
            \alpha( [ (0,c), (0,d) ] )
            =
            \alpha( 0, [c,d] )
            =
            (0, [c,d])
          \]
          and on the other hand
          \[
            [ \alpha(t), \alpha(u) ]
            =
            [ \alpha(0,c), \alpha(0,d) ]
            =
            [ (0,c), (0,d) ]
            =
            ( 0, [c,d] ) \,.
          \]
          We don’t get any additional condition from this last case.
      \end{itemize}
      We find altogether that the map~$\alpha$ is a homomorphism of Lie~algebras if and only if the linear map~$\varphi$ satisfies the two conditions
      \begin{gather*} 
        \theta_1(x)
        =
        \theta_2(x)
        + \ad_I(\varphi(x))
      \shortintertext{and}
        \kappa_1(x,y)
        =
        \kappa_2(x,y)
        + \theta_2(x)(\varphi(y))
        - \theta_2(y)(\varphi(x))
        + [\varphi(x), \varphi(y)]
        - \varphi( [x,y] )
      \end{gather*}
      for all~$x, y \in \hlie$.
  \end{enumerate}
\end{fluff}


\begin{fluff}
  We have now classified standard extensions in terms of certains pairs~$(\kappa, \theta)$, and have explained under what condition two such pairs~$(\kappa_1, \theta_1)$ and~$(\kappa_2, \theta_2)$ give equivalent extensions.
  This will greatly help us in the upcoming classifications of certain classes of extensions because every extension is equivalent to a standard one.

  Let us now finish our discussion of general extensions by summarizing our findings.
\end{fluff}


\begin{theorem}[Structure of extensions]
  \label{structure of extensions}
  Let~$\hlie$ and~$I$ be two~\liealgebras{$\kf$}.
  \begin{enumerate}
    \item
      Every extension of~$\hlie$ by~$I$ is equivalent to a standard extension.
    \item
      Given a standard extension of~$\hlie$ by~$I$, the Lie~bracket~$[-,-]$ on~$\hlie \oplus I$ is of the form
      \begin{equation}
        \label{formula for lie bracket on standard extension}
        [(x,c), (y,d)]
        =
        ( [x,y], \kappa(x,y) + \theta(x)(d) - \theta(y)(c) + [c,d] )
      \end{equation}
      for all~$(x,c), (y,d) \in \hlie \oplus I$, for some bilinear map
      \begin{equation}
        \label{kappa function}
        \kappa
        \colon
        \hlie \times \hlie
        \to
        I
      \end{equation}
      and some linear map
      \begin{equation}
        \label{theta function}
        \theta
        \colon
        \hlie
        \to
        \Der(I) \,.
      \end{equation}
      The maps~$\kappa$ and~$\theta$ are uniquely determined by the formulas
      \begin{alignat*}{2}
        [(x,0), (y,0)]
        &=
        ([x,y], \kappa(x,y))
        &
        \qquad
        &\text{for all~$x, y \in \hlie$}
      \shortintertext{and}
        [(x,0), (0,c)]
        &=
        (0, \theta(x)(c))
        &
        \qquad
        &\text{for all~$x \in \hlie$,~$c \in I$.}
      \end{alignat*}
      The map~$\kappa$ is alternating, and the maps~$\kappa$ and~$\theta$ satisfy the two compatibility conditions
      \begin{gather}
        \begin{aligned}
          {}&
          \kappa([x,y], z)
          + \kappa([y,z], x)
          + \kappa([z,x], y)
          \label{first compatibility condition for extensions}
          \\
          ={}&
          \theta(x)( \kappa(y,z) )
          + \theta(y)( \kappa(z,x) )
          + \theta(z)( \kappa(x,y) )
        \end{aligned}
      \shortintertext{and}
        \theta([x,y])
        =
        [\theta(x), \theta(y)] - \ad_I(\kappa(x,y))
        \label{second compatibility condition for extensions}
      \end{gather}
      for all~$x, y, z \in \hlie$.
    \item
      Let on the other hand~$\kappa$ be a bilinear map as in \eqref{kappa function} and let~$\theta$ be a linear map as in~\eqref{theta function}, such that~$\kappa$ is alternating and the two compatibility conditions~\eqref{first compatibility condition for extensions} and~\eqref{second compatibility condition for extensions} are satisfied.
      Then the formula~\eqref{formula for lie bracket on standard extension} defines a Lie~bracket on the vector space~$\hlie \oplus I$ which makes it into a standard extension of~$\hlie$ by~$I$.
    \item
      The above two constructions are mutually inverse and result in a {\onetoonetext} correspondence
      \[
        \left\{
          \begin{tabular}{c}
            Lie~brackets on~$\hlie \oplus I$ \\
            that make it into a \\
            standard extension \\
            of~$\hlie$ by~$I$
          \end{tabular}
        \right\}
        \onetoone
        \left\{
          (\kappa, \theta)
        \suchthat*
          \begin{tabular}{@{}c}
            $\kappa \colon \hlie \times \hlie \to I$ is bilinear, \\
            $\theta \colon \hlie \to \Der(I)$ is linear, \\
            $\kappa$ is alternating, and the \\
            conditions~\eqref{first compatibility condition for extensions} and~\eqref{second compatibility condition for extensions} hold
          \end{tabular}
        \right\} \,.
      \]
  \end{enumerate}
  Let now~$(\kappa_1, \theta_1)$ and~$(\kappa_2, \theta_2)$ be two pairs as described above, and let~$\glie_1$ and~$\glie_2$ be the standard extensions corresponding to these pairs.
  \begin{enumerate}[resume*]
    \item
      Let~$\alpha$ be an equivalence of extensions from~$\glie_1$ to~$\glie_2$, i.e. a homomorphism of Lie~algebras
      \[
        \alpha
        \colon
        (\hlie \oplus I, [-,-]_1)
        \to
        (\hlie \oplus I, [-,-]_2)
      \]
      that makes the following diagram commute:
      \[
        \begin{tikzcd}
          0
          \arrow{r}
          &
          I
          \arrow{r}[above]{\iota}
          \arrow[equal]{d}
          &
          \hlie \oplus I
          \arrow{r}[above]{\pi}
          \arrow[dashed]{d}[right]{\alpha}
          &
          \hlie
          \arrow{r}
          \arrow[equal]{d}
          &
          0
          \\
          0
          \arrow{r}
          &
          I
          \arrow{r}[above]{\iota}
          &
          \hlie \oplus I
          \arrow{r}[above]{\pi}
          &
          \hlie
          \arrow{r}
          &
          0
        \end{tikzcd}
      \]
      Then the map~$\alpha$ is of the form
      \begin{equation}
        \label{formula for equivalence depending on varphi}
        \alpha(x,c)
        =
        \alpha(x, \varphi(x) + c)
        \qquad
        \text{for every~$(x,c) \in \hlie \oplus I$}
      \end{equation}
      for some linear map
      \begin{equation}
        \label{functional definition of varphi}
        \varphi
        \colon
        \hlie
        \to
        I \,.
      \end{equation}
      The map~$\varphi$ is unique, and it satisfies the two conditions
      \begin{gather} 
        \label{first condition for equivalence}
        \theta_1(x)
        =
        \theta_2(x)
        + \ad_I(\varphi(x))
      \shortintertext{and}
        \label{second condition for equivalence}
        \kappa_1(x,y)
        =
        \kappa_2(x,y)
        + \theta_2(x)(\varphi(y))
        - \theta_2(y)(\varphi(x))
        + [\varphi(x), \varphi(y)]
        - \varphi( [x,y] )
      \end{gather}
      for all~$x, y \in \hlie$.
    \item
      Suppose on the other hand that~$\varphi$ is a linear map as in~\eqref{functional definition of varphi} which satisfies the two conditions~\eqref{first condition for equivalence} and~\eqref{second condition for equivalence}.
      Then formula~\eqref{formula for equivalence depending on varphi} defines an equivalence of extensions~$\alpha$ from~$\glie_1$ to~$\glie_2$.
    \item
      The above two constructions are mutually inverse and result in a {\onetoonetext} correspondence
      \[
        \left\{
          \alpha
        \suchthat*
          \begin{tabular}{@{}c@{}}
            $\alpha \colon \glie_1 \to \glie_2$ is \\
            an equivalence \\
            of extensions
          \end{tabular}
        \right\}
        \onetoone
        \left\{
          \varphi
        \suchthat*
          \begin{tabular}{@{}c@{}}
            $\varphi \colon \hlie \to I$ is linear \\
            and satisfies the two \\
            conditions~\eqref{first condition for equivalence} and~\eqref{second condition for equivalence}
          \end{tabular}
        \right\}
      \]
      It holds in particular that the extensions~$\glie_1$ and~$\glie_2$ are equivalent if and only if there exists a linear map~$\varphi$ from~$\hlie$ to~$I$ satisfying conditions~\eqref{first condition for equivalence} and~\eqref{second condition for equivalence}.
  \end{enumerate}
\end{theorem}


\begin{remark}
  The author learned about the general form of the compatibility conditions~\eqref{first condition for equivalence} and~\eqref{second condition for equivalence} from~\cite{extension_of_lie_algebras_nlab} and~\cite{nonabelian_lie_algebra_cohomology_nlab}, with further help from~\cite{extension_of_lie_algebras_arxiv}.
  According to these sources, these two compatiblity conditions express that the pair~$(\kappa, \theta)$ is a~\cocycle{$2$} in nonabelian Lie~algebra cohomology.
\end{remark}


\begin{warning}
  Suppose that we are given any extension of Lie~algebras
  \[
    0 \to I \to \glie \to \hlie \to 0 \,.
  \]
  Then according to \cref{structure of extensions} this extension is equivalent to a standard extesion.
  This equivalence is realized by a suitable isomorphism of vector spaces~$\alpha$ from~$\glie$ to~$\hlie \oplus I$.

  The Lie~bracket of~$\hlie \oplus I$ can now be parametrized by a pair~$(\kappa, \theta)$ as explained in \cref{structure of extensions}.
  The Lie~bracket of~$\glie$ can therefore be expressed by the pair~$(\kappa, \theta)$.
  However, this pair~$(\kappa, \theta)$ does not only depend on the Lie~bracket of~$\glie$, but also on the choice of isomorphism~$\alpha$.
\end{warning}


\begin{definition}
  In the bijection from \cref{structure of extensions} the standard extension corresponding to a pair~$(\kappa, \theta)$ is \defemph{the standard extension given by~$(\kappa, \theta)$}\index{extension!standard}, or simple \defemph{the standard extension~$(\kappa, \theta)$}\index{extension!of Lie algebras!standard}.
\end{definition}





\section{Special Kinds of Extensions}


\begin{fluff}
  We will now classify certain kinds of extensions up to equivalence.
  For this we will show that the Lie~algebras in question are equivalent to standard extensions given by~$(\kappa, \theta)$ such that~$\kappa$ and~$\theta$ have to satisfy certain additional conditions.
  We then express under what conditions two such pairs~$(\kappa_1, \theta_1)$ and~$(\kappa_2, \theta_2)$ give equivalent conditions.
\end{fluff}



\subsection{Trivial Extensions}


\begin{example}
  \label{construction of trivial extension}
  Let~$\hlie$ and~$I$ be two~\liealgebras{$\kf$}.
  We consider the vector space~$\hlie \oplus I$ together with the componentwise Lie~bracket
  \[
    [ (x,c), (y,d) ]
    =
    ( [x,y], [c,d] )
  \]
  for all~$(x,c), (y,d) \in \hlie \oplus I$.
  This makes the vector space~$\hlie \oplus I$ into a standard extension of~$\hlie$ by~$I$.
\end{example}


\begin{definition}
  Let~$\hlie$ and~$I$ be two~\liealgebras{$\kf$}.
  \begin{enumerate}
    \item
      The extension of~$\hlie$ by~$I$ from \cref{construction of trivial extension} is the \defemph{trivial extension}\index{trivial!extension of Lie algebras}\index{extension!of Lie algebras!trivial} of~$\hlie$ by~$I$.
    \item
      An arbitrary extension of~$\hlie$ by~$I$ is \defemph{trivial}\index{trivial!extension of Lie algebras}\index{extension!of Lie algebras!trivial} if it is equivalent to the trivial extension.
  \end{enumerate}
\end{definition}


\begin{remark}
  \leavevmode
  \begin{enumerate}
    \item
      We have choosen our definitions so that \emph{the} trivial extension is one specific extension, whereas \emph{being trivial} is a property of an extension.
    \item
      Let~$\hlie$ and~$I$ be two~\liealgebras{$\kf$}.
      Under the correspondence from \cref{structure of extensions} the trivial extension of~$\hlie$ by~$I$ corresponds to the pair~$(\kappa, \theta)$ with~$\kappa = 0$ and~$\theta = 0$.
  \end{enumerate}
\end{remark}


\begin{definition}
  Let
  \[
    0
    \to
    I
    \xto{\varphi}
    \glie
    \xto{\psi}
    \hlie
    \to
    0
  \]
  be an extension of Lie~algebras.
  A \defemph{retract}\index{retract}\index{extension!of Lie algebras!retract}  of this extension is a homomorphism of Lie~algebras~$\rho$ from~$\glie$ to~$I$ such that
  \[
    \rho \circ \varphi
    =
    \id_I \,.
  \]
\end{definition}


\begin{lemma}
  \label{having a retract is invariant under equivalence of extensions}
  Let~$\hlie$ and~$I$ be two~\liealgebras{$\kf$}
  If an extension of~$\hlie$ by~$I$ admits a retract then every equivalent extension also admits a retract.
\end{lemma}


\begin{proof}
  Let
  \[
    0
    \to
    I
    \xto{\varphi_1}
    \glie_1
    \xto{\psi_1}
    \hlie
    \to
    0
  \]
  and
  \[
    0
    \to
    I
    \xto{\varphi_2}
    \glie_2
    \xto{\psi_2}
    \hlie
    \to
    0
  \]
  be two equivalent extensions of~$\hlie$ by~$I$.
  There exists by assumption a homorphism of Lie~algebras~$\alpha$ from~$\glie_1$ to~$\glie_2$ that makes the diagram
  \[
    \begin{tikzcd}
      0
      \arrow{r}
      &
      I
      \arrow{r}[above]{\varphi_1}
      \arrow[equal]{d}
      &
      \glie_1
      \arrow{r}[above]{\psi_2}
      \arrow[dashed]{d}[right]{\alpha}
      &
      \hlie
      \arrow{r}
      \arrow[equal]{d}
      &
      0
      \\
      0
      \arrow{r}
      &
      I
      \arrow{r}[above]{\varphi_2}
      &
      \glie_2
      \arrow{r}[above]{\psi_2}
      &
      \hlie
      \arrow{r}
      &
      0
    \end{tikzcd}
  \]
  commute.
  If~$\rho$ is a retract for the second extension then the composite~$\rho \circ \alpha$ will be a retract for the first extension.
  Indeed, we have~$\rho \circ \varphi_2 = \id_I$ and therefore
  \[
    \rho \circ \alpha \circ \varphi_1
    =
    \rho \circ \varphi_2
    =
    \id_I
  \]
  by the commutativity of the above diagram.
  This shows that the first extension admits a retract if the second one does.
  The reverse implication follows from this first implication because equivalence of extensions is an equivalence relation.
\end{proof}


\begin{proposition}
  For an extension of Lie algebras
  \[
    0
    \to
    I
    \xto{\varphi}
    \glie
    \xto{\psi}
    \hlie
    \to
    0
  \]
  the following conditions are equivalent.
  \begin{equivalenceslist*}
    \item
      \label{extension is trivial}
      The extension is trivial.
    \item
      \label{extension is equivalent to 0 0}
      The extension is equivalent to the standard extension~$(\kappa, \theta)$ with~$\kappa = 0$ and~$\theta = 0$.
    \item
      \label{extension admits a retract}
      The extension admits a retract.
    \item
      \label{image of I admits a direct complement}
      There exists an ideal~$J$ of~$\glie$ such that~$\glie$ is the (internal) direct sum of the two ideals~$\varphi(I)$ and~$J$.
  \end{equivalenceslist*}
%  Suppose furthermore that the underlying vector space of~$\glie$ is given by the direct sum~$\hlie \oplus I$.
%  Let~$(\kappa, \theta)$ be the pair correspondung to the Lie~bracket of~$\glie$ under the bijection from \cref{structure of extensions}.
%  Then the following condition is also equivalent to the ones above.
%  \begin{equivalenceslist*}[resume*]
%    \item
%      \label{trivial via kappa and theta}
%      There exists a linear map~$\varphi$ from~$\hlie$ to~$I$ such that
%      \[
%        \theta(x) = \ad_I(\varphi(x)) \,,
%        \quad
%        \kappa(x,y) = [\varphi(x), \varphi(y)] - \varphi([x,y])
%      \]
%      for all~$x, y \in \hlie$.
%      % TODO: Give an explicit explanation of these conditions.
%      % The first condition means that \theta takes values in \ad(I), the Lie algebra of inner derivations
%      % The second condition means that \kappa is the coboundary associated to \varphi.
%      % But what about the interplay between these conditions?
%  \end{equivalenceslist*}
\end{proposition}

\begin{proof}
  We denote the trivial representation of~$\hlie$ by~$I$ as~$\hlie \oplus I$.
  \begin{implicationlist}
    \item[\ref{extension is trivial}~$\iff$~\ref{extension is equivalent to 0 0}]
      The standard extension given by~$(0,0)$ is precisely the trivial extension.
      This equivalence is therefore just a reformulation of the definition of a trivial extension.
    \item[\ref{extension is trivial}~$\implies$~\ref{extension admits a retract}]
      The trivial extension admits a retract, namly the canonical projection from~$\hlie \oplus I$ to~$I$, given by~$(x,c) \mapsto c$.
      It follows from \cref{having a retract is invariant under equivalence of extensions} that every trivial representation admits a retract.
    \item[\ref{extension admits a retract}~$\implies$~\ref{image of I admits a direct complement}]
      Let~$\rho \colon \glie \to I$ be a retract of the given extension.
      We consider for the ideal~$J$ the kernel of~$\rho$.
      The composite~$\varphi \circ \rho$ is an idempotent endomorphism of~$\glie$, whence~$\glie$ is the direct sum of its kernel and its image.
      It follows from the condtion~$\rho \circ \varphi = \id_I$ that~$\rho$ is surjective and~$\varphi$ is injective, so
      \[
        \ker(\varphi \circ \rho) = \ker(\rho) = J \,,
        \quad
        \im(\varphi \circ \rho) = \im(\varphi) = \varphi(I) \,.
      \]
      We have thus shown that~$\glie$ is the direct sum of~$\varphi(I)$ and~$J$.
    \item[\ref{image of I admits a direct complement}~$\implies$~\ref{extension is trivial}]
      It also follows from the injectivity of~$\varphi$ that it restricts to an isomorphism of Lie~algebras from~$I$ to~$\varphi(I)$.
      We denote the restriction by~$\varphi'$.
      We similarly find that the homomorphism~$\psi$ restricts to an isomorphism~$\psi'$ from~$J$ to~$\hlie$.
      This follows from the fact that~$\psi$ is surjective and~$J$ is a precisely a direct complement to~$\varphi(I)$, and thus a direct complement to the kernel of~$\psi$.

      It follows from \cref{direct sum of ideals} that the map
      \[
        \alpha
        \colon
        J \oplus \varphi(I)
        \to
        \glie \,,
        \quad
        (x, y)
        \mapsto
        x + y
      \]
      is an isomorphism of Lie~algebras.
      This isomorphism makes the diagram
      \[
        \begin{tikzcd}[sep = large]
          0
          \arrow{r}
          &
          \varphi(I)
          \arrow{r}[above]{\iota}
          \arrow{d}[right]{(\varphi')^{-1}}
          &
          J \oplus \varphi(I)
          \arrow{r}[above]{\pi}
          \arrow[dashed]{d}[right]{\alpha}
          &
          J
          \arrow{r}
          \arrow{d}[right]{\psi'}
          &
          0
          \\
          0
          \arrow{r}
          &
          I
          \arrow{r}[above]{\varphi}
          &
          \glie
          \arrow{r}[above]{\psi}
          &
          \hlie
          \arrow{r}
          &
          0
        \end{tikzcd}
      \]
      commute, where~$\iota$ denotes the inclusion and~$\pi$ denotes the canonical projection.

      We also find that the map
      \[
        \beta
        \colon
        \hlie \oplus I
        \to
        J \oplus \varphi(I) \,,
        \quad
        (x,y)
        \mapsto
        ( (\psi')^{-1}(x) , \varphi'(y) )
      \]
      is again an isomorphism of Lie~algebras.
      This isomorphism makes the diagram
      \[
        \begin{tikzcd}[sep = large]
          0
          \arrow{r}
          &
          I
          \arrow{r}[above]{\iota}
          \arrow{d}[right]{\varphi'}
          &
          \hlie \oplus I
          \arrow{r}[above]{\pi}
          \arrow[dashed]{d}[right]{\beta}
          &
          \hlie
          \arrow{r}
          \arrow{d}[right]{(\psi')^{-1}}
          &
          0
          \\
          0
          \arrow{r}
          &
          \varphi(I)
          \arrow{r}[above]{\iota}
          &
          J \oplus \varphi(I)
          \arrow{r}[above]{\pi}
          &
          J
          \arrow{r}
          &
          0
        \end{tikzcd}
      \]
      commute, where~$\iota$ denotes the respective inclusions, and~$\pi$ the respective canonical projections.

      We have altogether the following commutative diagram:
      \[
        \begin{tikzcd}[sep = large]
          0
          \arrow{r}
          &
          I
          \arrow{r}[above]{\iota}
          \arrow{d}[right]{\varphi'}
          &
          \hlie \oplus I
          \arrow{r}[above]{\pi}
          \arrow[dashed]{d}[right]{\beta}
          &
          \hlie
          \arrow{r}
          \arrow{d}[right]{(\psi')^{-1}}
          &
          0
          \\
          0
          \arrow{r}
          &
          \varphi(I)
          \arrow{r}[above]{\iota}
          \arrow{d}[right]{(\varphi')^{-1}}
          &
          J \oplus \varphi(I)
          \arrow{r}[above]{\pi}
          \arrow{d}[right]{\alpha}
          &
          J
          \arrow{r}
          \arrow{d}[right]{\psi'}
          &
          0
          \\
          0
          \arrow{r}
          &
          I
          \arrow{r}[above]{\varphi}
          &
          \glie
          \arrow{r}[above]{\psi}
          &
          \hlie
          \arrow{r}
          &
          0
        \end{tikzcd}
      \]
      By deleting the middle row of this commutative diagram we arrive at the following commutative diagram:
      \[
        \begin{tikzcd}[sep = large]
          0
          \arrow{r}
          &
          I
          \arrow{r}[above]{\iota}
          \arrow[equal]{d}
          &
          \hlie \oplus I
          \arrow{r}[above]{\pi}
          \arrow[dashed]{d}[right]{\alpha \circ \beta}
          &
          \hlie
          \arrow{r}
          \arrow[equal]{d}
          &
          0
          \\
          0
          \arrow{r}
          &
          I
          \arrow{r}[above]{\varphi}
          &
          \glie
          \arrow{r}[above]{\psi}
          &
          \hlie
          \arrow{r}
          &
          0
        \end{tikzcd}
      \]
      This shows that the homomorphismp~$\alpha \circ \beta$ gives an equivalence between the trivial extension~$\hlie \oplus I$ and the given extension.
%    \item[\ref{extension is trivial}~$\iff$~\ref{trivial via kappa and theta}]
%      Under the correspondence from \cref{structure of extensions} the trivial representation of corresponds to the pair~$(0,0)$.
%      The claim thus follows from \cref{structure of extensions}.
    \qedhere
  \end{implicationlist}
\end{proof}

\begin{example}
  We have a short exact sequence of Lie~algebras given by
  \[
    0
    \to
    \sllie(n, \kf)
    \xto{\iota}
    \gllie(n, \kf)
    \xto{\tr}
    \kf
    \to
    0
  \]
  where~$\iota$ denotes the inclusion map.
  The Lie~algebra~$\gllie(n, \kf)$ is thus an extension of~$\kf$ by~$\sllie(n, \kf)$.
  If the characteristic of~$\kf$ does not divide the size~$n$ then the one-dimensional ideal~$\gen{ \Id }_{\kf}$ of~$\gllie(n, \kf)$ is a direct complement of~$\sllie(n, \kf)$.
  We find in this case that the above extension is trivial.
  % TODO: What happens if the characteristic does divide n?
  % This problem should follow from a classification of ideals in gl, which should be done after root space decompositions have been introduced.
\end{example}

\begin{fluff}
  Let~$\hlie$ and~$I$ be two~\liealgebras{$\kf$}.
  It may happen that a standard extension~$(\kappa, \theta)$ is trivial even if neither~$\kappa = 0$ nor~$\theta = 0$.
  Indeed, the standard extension~$(\kappa, \theta)$ is trivial if and only if it is equivalent to the standard extension~$(0,0)$.
  According to \cref{structure of extensions} this happens if and only if there exists a linear map~$\varphi$ from~$\hlie$ to~$\Der(I)$ such that
  \[
    \theta(x) = \ad_I(\varphi(x)) \,,
    \qquad
    \kappa(x,y) = [\varphi(x), \varphi(y)] - \varphi([x,y])
  \]
  for all~$x, y \in \hlie$.

  Indeed, we observe that for any linear map~$\theta$ from~$I$ to~$\Der(I)$ we can define maps~$\kappa$ from~$\hlie \times \hlie$ to~$I$ and~$\theta$ from~$\hlie$ to~$\Der(I)$ via the above two formulas.
  The map~$\kappa$ is then bilinear and the map~$\theta$ is linear.
  Moreover, the maps~$\kappa$ and~$\theta$ satisfy the compatibility conditions from \cref{structure of extensions} because
  \begin{gather*}
    \begin{aligned}
      {}&
      \kappa([x,y], z) + \kappa([y,z], x) + \kappa([z,x], y)
      \\
      ={}&
      [\varphi([x,y]), \varphi(z)] - \varphi([[x,y],z])
      \\
      {}&
      + [\varphi([y,z]), \varphi(x)] - \varphi([[y,z],x])
      \\
      {}&
      + [\varphi([z,x]), \varphi(y)] - \varphi([[z,x],y])
      \\
      ={}&
      [\varphi([x,y]), \varphi(z)]
      + [\varphi([y,z]), \varphi(x)]
      + [\varphi([z,x]), \varphi(y)]
      \\
      {}&
      - \varphi( [[x,y],z] + [[y,z],x] + [[z,x],y] )
      \\
      ={}&
      [\varphi([x,y]), \varphi(z)]
      + [\varphi([y,z]), \varphi(x)]
      + [\varphi([z,x]), \varphi(y)]
    \end{aligned}
  \shortintertext{and}
    \begin{aligned}
      {}&
      \theta(x)(\kappa(y,z))
      + \theta(y)(\kappa(z,x))
      + \theta(z)(\kappa(x,y))
      \\
      ={}&
      [ \varphi(x), [\varphi(y), \varphi(z)] - \varphi([y,z]) ]
      \\
      {}&
      + [ \varphi(y), [\varphi(z), \varphi(x)] - \varphi([z,x]) ]
      \\
      {}&
      + [ \varphi(z), [\varphi(x), \varphi(y)] - \varphi([x,y]) ]
      \\
      ={}&
      [ \varphi(x), [\varphi(y), \varphi(z)] ]
      + [ \varphi(y), [\varphi(z), \varphi(x)] ]
      + [ \varphi(z), [\varphi(x), \varphi(y)] ]
      \\
      {}&
      - [ \varphi(x), \varphi([y,z]) ]
      - [ \varphi(y), \varphi([z,x]) ]
      - [ \varphi(z), \varphi([x,y]) ]
      \\
      ={}&
      - [ \varphi(x), \varphi([y,z]) ]
      - [ \varphi(y), \varphi([z,x]) ]
      - [ \varphi(z), \varphi([x,y]) ]
      \\
      ={}&
      [ \varphi([x,y]), \varphi(z) ]
      + [ \varphi([y,z]), \varphi(x) ]
      + [ \varphi([z,x]), \varphi(y) ] \,,
    \end{aligned}
  \end{gather*}
  as well as
  \begin{align*}
    {}&
    [\theta(x), \theta(y)] - \ad_I( \kappa(x,y) )
    \\
    ={}&
    [ \ad_I(\varphi(x)), \ad_I(\varphi(y)) ] - \ad_I( [\varphi(x), \varphi(y)] - \varphi([x,y]) )
    \\
    ={}&
    \ad_I( [\varphi(x), \varphi(y)] ) - \ad_I( [\varphi(x), \varphi(y)] ) + \ad_I( \varphi([x,y]) )
    \\
    ={}&
    \ad_I( \varphi([x,y]) )
    \\
    ={}&
    \theta([x,y])
  \end{align*}
  for all~$x, y, z \in \hlie$.
  Hence every linear map~$\varphi \colon \hlie \to I$ gives rise to a trivial standard extension.
\end{fluff}



\subsection{Abelian Extensions}

\begin{definition}
  An extension of Lie~algebras
  \[
    0 \to I \to \glie \to \hlie \to 0
  \]
  is \defemph{abelian} if the Lie~algebra~$I$ is abelian.
\end{definition}


\begin{fluff}
  \label{discussing abelian extensions}
  Let~$\hlie$ be a~\liealgebra{$\kf$}.
  We denote by~$\AbEx(\hlie)$ the class of abelian extensions of~$\hlie$.
  To better understand the equivalence classes of abelian extensions we will split up the class~$\mathcal{A}$ into pairwise disjoint subclasses, such that no two equivalent abelian extensions are contained in the same subclass.

  We denote for every abelian~\liealgebra{$\kf$}~$I$ by~$\AbEx(\hlie, I)$ the class of abelian extensions of~$\hlie$ by~$I$.
  This gives a decomposition
  \[
    \AbEx(\hlie)
    =
    \coprod_I
    \AbEx(\hlie, I)
  \]
  where~$I$ runs through the class of abelian~\liealgebras{$\kf$}.
  There exist no equivalences of extensions between extensions in~$\AbEx(\hlie, I)$ and extensions in~$\AbEx(\hlie, J)$ whenever~$I$ and~$J$ are distinct.

  Let~$I$ be an abelian~\liealgebra{$\kf$}.
  We want to further decompose the class~$\AbEx(\hlie, I)$.
  Let for this
  \[
    0 \to I \to \glie \to \hlie \to 0
  \]
  be an extension of Lie~algebras.

  We know that this extension is equivalent to a standard extensions~$(\kappa, \theta)$.
  The parameter~$\theta$ is in particular a linear map from~$\hlie$ to~$\Der(I)$ such that
  \[
    \theta( [x,y] )
    =
    [ \theta(x), \theta(y) ] + \ad_I( \kappa(x,y) )
  \]
  for all~$x, y \in \glie$.
  But the Lie~algebra~$I$ is abelian, so the term~$\ad_I( \kappa(x,y) )$ vanishes.
  We thus find that~$\theta$ needs to be a homomorphism of Lie algebras from~$\hlie$ to~$\Der(I)$.
  Moreover, since~$I$ is abelian we have~$\Der(I) = \gllie(I)$.
  We thus find that~$\theta$ is a homomorphism of Lie~algebras from~$\hlie$ to~$\gllie(I)$.

  If the given extension is equivalent to another standard extension~$(\kappa', \theta')$ then it follows from \cref{structure of extensions} that~$\theta = \theta' + \ad_I(\varphi(-))$ for some linear map~$\varphi$ from~$\hlie$ to~$I$.
  But we then have~$\ad_I(\varphi(-)) = 0$ because~$I$ is abelian, and therefore~$\theta = \theta'$.
  This shows that the homomorphism~$\theta$ is unique.

  We find from the above that we can further decompose the class~$\AbEx(\hlie, I)$ as
  \[
    \AbEx(\hlie, I)
    =
    \coprod_{\theta}
    \AbEx(\hlie, I, \theta)
  \]
  where~$\theta$ ranges through the set of Lie~algebra homomorphisms from~$\hlie$ to~$\gllie(I)$.
  We thus have the decomposition
  \[
    \AbEx(\hlie)
    =
    \coprod_{I, \theta}
    \AbEx(\hlie, I, \theta) \,.
  \]

  We can reparametrize this decomposition.
  For this we note that a pair~$(I, \theta)$ consisting of an abelian Lie~algebra~$I$ and a homomorphism of Lie~algebras~$\theta$ from~$\hlie$ to~$I$ is the same as a representation~$M$ of~$\hlie$, in the following way.
  Given on the one hand~$I$ and~$\theta$ as above, the pair~$(I, \theta)$ is a representation of~$\hlie$, with the action of~$\hlie$ on~$I$ given by~$x \act c = \theta(x)(c)$ for all~$x \in \hlie$,~$c \in I$.
  Given on the other hand a representation~$(M, \rho)$ of~$\hlie$ we regard~$M$ as an abelian Lie~algebra and let~$\theta = \rho$.

  We have thus a decomposition
  \[
    \AbEx(\hlie)
    =
    \coprod_M \AbEx(\hlie, M)
  \]
  where~$M$ runs through the class of~\representations{$\glie$}.
  To determine in which subclass~$\AbEx(\hlie, M)$ a given abelian extension
  \[
    0
    \to
    I
    \xto{\varphi}
    \glie
    \xto{\psi}
    \hlie
    \to
    0
  \]
  is contained we proceed as follows.

  We identify the abelian Lie~algebra~$I$ with an abelian ideal of~$\glie$ via the homomorphism~$\varphi$.
  For an element~$x$ of~$\glie$ let~$y$ be a preimage of~$x$ under~$\psi$.
  We define an action of~$x$ act on~$I$ via~$x \act c \defined [y, c]$.
  This action is well-defined:
  if~$y'$ is another preimage of~$x$ under~$\psi$ then the difference~$y - y'$ is contained in the kernel of~$\psi$, i.e. in~$I$.
  Then
  \[
    [y, c]
    =
    [y' + y - y', c]
    =
    [y', c] + [y - y', c]
    =
    [y', c] + 0
    =
    [y', c]
  \]
  for all~$c \in I$ because~$I$ is abelian.
  This action makes~$I$ into a representation of~$\hlie$.
  This is precisely the desired representation~$M$ for which the given extension is contained in~$\AbEx(\hlie, M)$.

  Let us summarize our results.
  To every abelian extension of~$\hlie$ we have associated a representation~$M$ of~$\mathfrak{h}$.
  For an abelian extension
  \[
    0
    \to
    I
    \to
    \glie
    \to
    \hlie
    \to
    0
  \]
  the representation~$M$ is given by~$I$ as a vector space, and the action of an element~$x$ of~$\hlie$ on~$M$ is given by taking a preimage~$y$ of~$x$ in~$\glie$ and acting wtih~$y$ on~$I$ via the adjoint action of~$\glie$.
  If two abelian extensions of~$\hlie$ are equal then the resulting representations of~$\hlie$ are equal.
\end{fluff}


\begin{theorem}[Structure of Abelian Extensions]
  \label{structure of abelian extensions}
  Let~$\hlie$ be a Lie algebra.
  \begin{enumerate}
    \item
      Suppose that
      \[
        0
        \to
        I
        \to
        \glie
        \to
        \hlie
        \to
        0
      \]
      is an abelian extension of~$\hlie$.
      For every element~$x$ of~$\hlie$ let~$y$ be a preimage of~$x$ in~$\glie$, and define an action of~$x$ on~$I$ via~$x \act c = [y,c]$ for all~$c \in I$.
      This action makes~$I$ into a representation of~$\hlie$.
    \item
      If two abelian extensions are equivalent then their associated representations are equal.
    \item
      Let~$\AbEx(\hlie)$ be the class of abelian extensions of~$\hlie$.
      For every representation~$M$ of~$\hlie$ let~$\AbEx(\hlie, M)$ be the subclass of~$\AbEx(\hlie)$ consisting of all those abelian extensions of~$\glie$ whose associated representation is~$M$.
      Then
      \[
        \AbEx(\hlie)
        =
        \coprod_M \AbEx(\hlie, M)
      \]
      where~$M$ ranges through the class of representations of~$\hlie$.
  \end{enumerate}
  Let now~$M$ be a representation of~$\hlie$.
  Let~$\theta$ be the corresponding homomorphism of Lie~algebras from~$\hlie$ to~$\gllie(M)$.
  We regard~$M$ as an abelian Lie~algebra, which we denote by~$I$, and regard~$\theta$ as a homomorphism of Lie~algebras from~$\hlie$ to~$\Der(I)$.
  \begin{enumerate}[resume*]
    \item
      The standard extensions contained in~$\AbEx(\hlie, M)$ are precisely those standard extensions of~$\hlie$ by~$I$ of the form~$(\kappa, \theta)$ with~$\theta$ as above.
    \item
      An alternating, bilinear map~$\kappa$ from~$\hlie \times \hlie$ to~$I$ defines a standard extension~$(\kappa, \theta)$ of~$\hlie$ by~$I$, which is then necessarily contained in~$\AbEx(\hlie, M)$, if and only if it satisfies the condition
      \[
        \kappa( [x,y], z ) + \kappa( [y,z], x ) + \kappa( [z,x], y )
        =
        x \act \kappa(y,z) + y \act \kappa(z,x) + z \act \kappa(x,y)
      \]
      for all~$x, y, z \in \hlie$.
    \item
      Two such maps~$\kappa_1$ and~$\kappa_2$ result in equivalent standard extensions~$(\kappa_1, \theta)$ and~$(\kappa_2, \theta)$ of~$\hlie$ by~$I$ if and only if there exists a linear map~$\varphi$ from~$\hlie$ to~$I$ such that
      \[
        \kappa_1(x,y)
        - \kappa_2(x,y)
        =
        x \act \varphi(y)
        - y \act \varphi(x)
        - \varphi( [x,y] )
      \]
      for all~$x, y \in \hlie$.
  \end{enumerate}
\end{theorem}


\begin{proof}
  This follows from \cref{structure of extensions} and \cref{discussing abelian extensions}.
\end{proof}



\subsection{Central Extensions}

\begin{definition}
  An extension of Lie~algebras
  \[
    0 \to I \to \glie \to \hlie \to 0
  \]
  is \defemph{central}\index{central extension}\index{extension!of Lie algebras!central} if the image of~$I$ in~$\glie$ is contained in the center of~$\glie$.
\end{definition}


\begin{remark}
  Suppose that
  \[
    0 \to I \to \glie \to \hlie \to 0
  \]
  is a central extensions of Lie~algebras.
  Then~$I$ is isomorphic to its image in~$\glie$, and thus abelian.
  This means that every central extension is in particular an abelian extension.
  It therefore only makes sense to talk about central extensions of a Lie~algebras~$\hlie$ by a Lie~algebra~$I$ if~$I$ is abelian.
\end{remark}


\begin{proposition}
  \label{central iff theta is zero}
  Let~$\hlie$ and~$I$ be two~\liealgebras{$\kf$} where~$I$ is abelian.
  A standard extension~$(\kappa, \theta)$ of~$\hlie$ by~$I$ is central if and only if~$\theta = 0$.
\end{proposition}


\begin{proof}
  The Lie~bracket~$[-,-]$ of the standard extension~$(\kappa, \theta)$ is given by
  \[
    [ (x,c), (y,d) ]
    =
    ( [x,y], \kappa(x,y) + \theta(x)(d) - \theta(y)(c) + [c,d] )
  \]
  for all~$(x,c), (y,d) \in \hlie \oplus I$.
  We have~$[c,d] = 0$ for all~$c, d \in I$ because the Lie~algebra~$I$ is abelian.
  Thus
  \[
    [ (x,c), (0,d) ]
    =
    ( 0, \theta(x)(d) )
  \]
  for all~$(x,c) \in \hlie \oplus I$,~$d \in I$.
  It follows that the image of~$I$ in~$\hlie \oplus I$ is central if and only if
  \[
    \theta(x)(d) = 0
    \qquad
    \text{for all~$x \in \hlie$,~$d \in I$,}
  \]
  i.e. if and only if~$\theta = 0$.
\end{proof}



\begin{theorem}[Structure of central extensions]
  \label{structure of central extensions}
  Let~$\hlie$ and~$I$ be two~\liealgebras{$\kf$} where~$I$ is abelian.
  \begin{enumerate}
    \item
      Every central extension of~$\hlie$ by~$I$ is equivalent to a central, standard extension of~$\hlie$ by~$I$.
    \item
      We have a {\onetoonetext} correspondence
      \[
        \left\{
          \begin{tabular}{@{}c@{}}
            central standard \\
            extensions of~$\hlie$ by~$I$
          \end{tabular}
        \right\}
        \onetoone
        \left\{
          \begin{tabular}{@{}c@{}}
            bilinear, alternating maps \\
            $\kappa \colon \hlie \times \hlie \to I$ such that \\
            $\kappa([x,y], z) + \kappa([y,z], x) + \kappa([z, x], y) = 0$
          \end{tabular}
        \right\}
      \]
      where the extension corresponding to a map~$\kappa$ is the standard extension~$(\kappa, 0)$.
    \item
      The Lie~bracket on the extension corresponding to~$\kappa$ is given by
      \[
        [ (x,c), (y,d) ]
        =
        ( [x,y], \kappa(x,y) )
      \]
      for all~$(x,c), (y,d) \in \hlie \oplus I$.
    \item
      Let
      \[
        \kappa_1, \kappa_2
        \colon
        \hlie \times \hlie
        \to
        I
      \]
      be two maps as above, and let~$\glie_1$ and~$\glie_2$ be the correspond central extensions.
      Then we have a {\onetoonetext} correspondence given by
      \[
        \left\{
          \begin{tabular}{@{}c@{}}
            equivalences \\ of extensions \\
            $\alpha \colon \glie_1 \to \glie_2$
          \end{tabular}
        \right\}
        \onetoone
        \left\{
          \begin{tabular}{@{}c@{}}
            linear maps~$\varphi \colon \hlie \to I$ with \\
            $\kappa_1(x,y) = \kappa_2(x,y) - \varphi([x,y])$ \\
            for all~$x, y \in \hlie$
          \end{tabular}
        \right\} \,.
      \]
      It follows in particular that the two extensions~$\glie_1$ and~$\glie_2$ are equivalent if and only if there exists a linear map~$\varphi \colon \hlie \to I$ with
      \[
        \kappa_1(x,y)
        =
        \kappa_2(x,y)
        -
        \varphi([x,y])
      \]
      for all~$x, y \in \hlie$.
  \end{enumerate}
\end{theorem}


\begin{proof}
  The \lcnamecref{structure of central extensions} follows from \cref{structure of extensions} thanks to \cref{central iff theta is zero}.
\end{proof}


\begin{remark}
  Let~$\hlie$ be a Lie~algebra.
  Under the decomposition~$\AbEx(\hlie) = \coprod_M \AbEx(\hlie, M)$ from \cref{structure of abelian extensions} the subclass of central extensions on the left hand side corresponds on the right hand side to~$\coprod_{M'} \AbEx(\hlie, M')$ where~$M'$ ranges through the class of trivial representations of~$\hlie$.
\end{remark}


%\begin{remark}
%  One can also think about central extenions, and our above structure theorem about central extensions, in a homological way.
%  We do so as follows:
%  \begin{enumerate}
%    \item
%      Let~$\glie$ be a Lie~algebra.
%      We can form a chain complex~$\Exterior^\bullet(\glie)$ whose~\howmanyth{$n$} term is given by the exterior power~$\Exterior^n(\glie)$ for every~$n \geq 0$, and which vanishes in negative degrees.
%      The differential of this chain complex is given by the maps
%      \begin{align*}
%        d_n
%        \colon
%        \bigwedge^n(\glie)
%        &\to
%        \bigwedge^{n-1}(\glie) \,,
%        \\
%        x_1 \wedge \dotsb \wedge x_n
%        &\mapsto
%        \sum_{1 \leq i < j \leq n}
%        (-1)^{i+j-1}
%        [x_i, x_j] \wedge x_1 \wedge \dotsb \wedge \widehat{x_i} \wedge \dotsb \wedge \widehat{x_j} \wedge \dotsb \wedge x_n
%      \end{align*}
%      for every~$n \geq 1$, and~$d_n \defined 0$ for every~$n \leq 0$.
%  \end{enumerate}
%\end{remark}

%\begin{example}[Central extensions]
%  In the notation of \cref{general approach to extensions} we find that an extension
%  \[
%    0
%    \to
%    I
%    \xto{\iota}
%    \hlie \oplus I
%    \xto{\pi}
%    \hlie
%    \to
%    0
%  \]
%  is central if and only
%  \begin{equation}
%    \label{central extension condition on elements}
%    [(x,c), (0,d)] = 0
%    \qquad
%    \text{for all~$x \in \hlie$ and~$c, d \in I$.}
%  \end{equation}
%  We have
%  \[
%    [(x,c), (0,d)]
%    =
%    (0, \theta(x)(d) + [c,d])
%    \qquad
%    \text{for all~$x \in \hlie$,~$c, d \in I$,}
%  \]
%  so we find that the condition~\eqref{central extension condition on elements} is equivalent to the combination of the two conditions~$\theta = 0$ and also~$[c,d] = 0$ for all~$c, d \in I$.
%  Indeed, by choosing~$d = c$ we see that~$\theta(x)(c) = 0$ for all~$x \in \hlie$ and~$c \in I$, and thus~$\theta = 0$.
%  And from this we then find that~$[c,d] = 0$ for all~$c, d \in I$.
%
%  We now find from the general formula
%  \begin{equation}
%    \label{central extension formula}
%    [ (x,c), (y,d) ]
%    =
%    ( [x,y], \kappa(x,y) )
%  \end{equation}
%  for all~$(x,c), (y,d) \in \hlie \oplus I$.
%  
%  Given any bilinear form~$\kappa \colon \hlie \times \hlie \to I$ we have already seen that \eqref{central extension formula} defines a bilinear bracket on~$\hlie \oplus I$ that is alternating if and only if~$\kappa$ is alternating.
%  We find for the Jacobi identity that
%  \[
%    [ (x,c), [ (y,d), (z,e) ] ]
%    =
%    [ (x,c), ([y,z], \kappa(y,z)) ]
%    =
%    ( [x,[y,z]], \kappa(x, [y,z]) )
%  \]
%  for all~$(x,c), (y,d), (z,e) \in \hlie \oplus I$ and therefore
%  \begin{align*}
%    {}&
%      [(x,c), [(y,d), (z,e)]]
%    + [(y,d), [(z,e), (x,c)]]
%    + [(z,e), [(x,c), (y,d)]]
%    \\
%    ={}&
%    (
%      [x,[y,z]] + [y,[z,x]] + [z,[x,y]],
%      \kappa(x, [y,z]) + \kappa(y, [z,x]) + \kappa(z, [x,y])
%    )
%    \\
%    ={}&
%    ( 0, \kappa(x, [y,z]) + \kappa(y, [z,x]) + \kappa(z, [x,y]) ) \,.
%  \end{align*}
%  We hence find that the bracket~$[-,-]$ on~$\hlie \oplus I$ satisfies the Jacobi identity if and only if the bilinear map~$\kappa$ satisfies the similar looking identity
%  \[
%    \kappa(x, [y,z]) + \kappa(y, [z,x]) + \kappa(z, [x,y])
%    =
%    0 \,.
%  \]
%  This condition is the \defemph{{\twococycle} condition}\index{$2$-cocycle condition}, and a bilinear map~$\kappa \colon \hlie \to \hlie \to I$ that is both alternating is satisfies the {\twococycle} condition is a {\twococycle}.
%  
%  We have overall constructed for all Lie~algebras~$\hlie$ and~$I$ a {\onetoonetext} correspondence between
%  \begin{itemize}
%    \item
%      Lie~brackets~$[-,-]$ on the vector space~$\hlie \oplus I$ that make the standard short exact sequence
%      \[
%        0
%        \to
%        I
%        \to
%        \hlie
%        \oplus
%        I
%        \to
%        \hlie
%        \to
%        0
%      \]
%      into a central extension of~$\hlie$ by~$I$ and
%    \item
%      {\twococycles}~$\kappa \colon \hlie \times \hlie \to I$,
%  \end{itemize}
%  and this correspondence is given by~$[(x,c), (y,d)] = ([x,y], \kappa(x,y))$.
%\end{example}



\subsection{Split Extensions and Semidirect Products}

\begin{fluff}
  Another important kind of extensions are so called semidirect products.
  We will give three characterizations of these extensions and explain the equivalence of these characterizations.
\end{fluff}

\subsubsection{Inner Semidirect Products}


\begin{definition}
  Let~$\glie$ be a Lie~algebra.
  Let~$\hlie$ be a Lie~subalgebra of~$\glie$ and let~$I$ be an ideal in~$\glie$.
  If~$\glie = \hlie \oplus I$ as vector spaces then~$\glie$ is the \defemph{internal semidirect product}\index{semidirect product!internal} of~$\hlie$ by~$I$.
  This is then denoted by~$\glie = \hlie \ltimes I$.
\end{definition}


\begin{fluff}
  \label{calculating structure of internal semidirect product}
  Let~$\glie$ be a Lie~algebra which is the semidirect product of a Lie~subalgebra~$\hlie$ by an ideal~$I$.
  The adjoint action of~$\glie$ on itself restricts to an action of~$\hlie$ on~$I$ because~$I$ is an ideal in~$\glie$.
  The action of~$\glie$ on itself is by derivatations, so the action of~$\hlie$ on~$I$ is also by derivations.
  This action is given by a homomorphism of Lie~algebras
  \[
    \theta
    \colon
    \hlie
    \to
    \Der(I) \,.
  \]
  Any two elements of~$\glie$ may be written uniquely as sums~$x + c$ and~$y + d$ with~$x$,~$y$ in~$\hlie$ and~$c$,~$d$ in~$I$.
  The Lie~bracket of~$\glie$ can be expressed in terms of the homomorphisms~$\theta$ as
  \begin{align*}
    [x + c, y + d]_{\glie}
    &=
    [x, y]_{\glie}
    + [x, d]_{\glie}
    + [c, y]_{\glie}
    + [c, d]_{\glie}
    \\
    &=
    [x, y]_{\glie}
    + [x, d]_{\glie}
    - [y, c]_{\glie}
    + [c, d]_{\glie}
    \\
    &=
    [x, y]_{\hlie}
    + \theta(x)(d)
    - \theta(y)(c)
    + [c, d]_I \,.
  \end{align*}
  The term~$[x,y]_{\hlie}$ is again contained in the Lie~subalgebra~$\hlie$, and the term~$\theta(x)(d) - \theta(y)(c) + [c,d]_I$ is again contained in the ideal~$I$.
\end{fluff}


\begin{example}
  Let~$\tlie(n, \kf)$ be the Lie~algebra of upper triangular matrices,~$\dlie(n, \kf)$ the Lie~algebra of diagonal matrices and~$\nlie(n, \kf)$ the Lie~algebra of strictly upper triangular matrices.
  Then~$\dlie(n, \kf)$ is a Lie~subalgebra of~$\tlie(n, \kf)$ and~$\nlie(n, \kf)$ is an ideal of~$\tlie(n, \kf)$, and~$\tlie(n, \kf)$ is the internal semidirect product of~$\dlie(n, \kf)$ by~$\nlie(n, \kf)$.
\end{example}


\begin{example}
  Let~$\Hlie$ be the Heisenberg Lie~algebra as in \cref{examples for lie algebras}.
  The linear subspace~$I$ of~$\Hlie$ spanned by the elements~$p_1, \dotsc, p_n, c$ is an ideal in~$\Hlie$, and the linear subspace~$\hlie$ of~$\Hlie$ spanned by the elements~$q_1, \dotsc, q_n$ is a Lie~subalgebra of~$\glie$.
  The Lie~algebra~$\Hlie$ is the inner semidirect product of~$\hlie$ by~$I$.
\end{example}

\subsubsection{External Semidirect Products}

\begin{definition}
  Let~$\hlie$ and~$I$ be two~\liealgebras{$\kf$} and let~$\theta$ be a homomorphism of Lie~algebras from~$\hlie$ to~$\Der(I)$.
  The standard extension given by~$(\theta, 0)$ is the \defemph{external semidirect product} of~$\hlie$ by~$I$ along~$\theta$.
  It is denoted by~$\hlie \ltimes_\theta I$.
\end{definition}


\begin{remark}
  An external semidirect product~$\hlie \ltimes_\theta I$ is given by the external direct sum~$\hlie \oplus I$ on the level of vector spaces.
  Its Lie~bracket is given by
  \[
    [ (x,c), (y,d) ]
    =
    ( [x,y], \theta(x)(d) - \theta(y)(c) + [c,d] )
  \]
  for all~$(x,c), (y,d) \in \hlie \ltimes_\theta I$.
  We have in particular
  \[
    [(x,0), (0,c)]
    =
    ( 0, \theta(x)(c) )
  \]
  for all~$x \in \hlie$,~$c \in I$.
\end{remark}


\begin{fluff}
  \label{internal semidirect product is also external}
  Suppose that a Lie~algebra~$\glie$ is the internal semidirect product of a Lie~subalgebra~$\hlie$ by an ideal~$I$.
  We have seen in~\cref{calculating structure of internal semidirect product} that the Lie~bracket on~$\glie$ is uniquely determined by the Lie~brackets of~$\hlie$ and~$I$ together with a certain homomorphism of Lie~algebras~$\theta$ from~$\hlie$ to~$\Der(I)$.
  Let~$\glie_{\ext}$ be the external semidirect product of~$\hlie$ by~$I$ along~$\theta$.
  Then the linear map
  \[
    \varphi
    \colon
    \glie_{\ext}
    \to
    \glie \,,
    \quad
    (x, c)
    \mapsto
    x + c
  \]
  is an isomorphism of vector spaces, and according to the calculations in \cref{calculating structure of internal semidirect product} it is already an isomorphism of Lie~algebras.
  In this way we see that every internal semidirect product is isomorphic to an external semidirect product.
\end{fluff}


\begin{fluff}
  Suppose conversely that~$\hlie$ and~$I$ are two~\liealgebras{$\kf$} and let~$\theta$ be a homomorphism of Lie~algebras from~$\hlie$ to~$\Der(I)$.
  Let~$\glie$ be the external semidirect product of~$\hlie$ by~$I$ along~$\theta$.
  Then~$\glie = \hlie \oplus I$ as vector spaces, where the right hand side denotes an external direct sum.
  The linear subspace~$\hlie'$ of~$\glie$ given by~$\hlie' \defined \hlie \oplus 0$ is a Lie~subalgebra of~$\glie$ and the linear subspace~$I'$ given by~$I' \defined 0 \oplus I$ is an ideal in~$\glie$.
  We also see that~$\glie = \glie' \oplus I'$ on the level of vector spaces, where the right hand side denotes an internal direct sum.
  We see from this that~$\glie$ is the internal direct sum of~$\hlie'$ by~$I'$.

  We know from \cref{calculating structure of internal semidirect product} that the Lie~bracket of~$\glie$ can be expressed by the Lie~brackets of~$\hlie'$ and~$I'$ together with a certain homomorphism of Lie~algebras~$\theta'$ from~$\hlie'$ to~$\Der(I')$.
  We can identify~$\hlie'$ and~$I'$ with~$\hlie$ and~$I$ via the isomorphism of Lie~algebras
  \begin{alignat*}{2}
    \hlie
    &\to
    \hlie' \,,
    &\quad
    x
    &\mapsto
    (x, 0) \,,
    \\
    I
    &\to
    I' \,,
    &\quad
    c
    &\mapsto
    (0, c) \,.
  \end{alignat*}
  The homomorphism~$\theta'$ corresponds to the original homomorphism~$\theta$ under these isomorphism.

  We have now explained how every external semidirect product can be regarded as an internal semidirect product.
\end{fluff}


\begin{example}
  Let~$I$ be the abelian two-dimensional Lie~algebra with basis~$x$,~$y$ and let~$\hlie$ be the one-dimensional Lie~algebra with basis~$t$.
  Then~$\Der(I) = \gllie(I)$, and every element~$f$ of~$\gllie(I)$ determined a unique homomorphism of Lie algebras from~$\hlie$ to~$\Der(I)$ which maps~$t$ to~$f$.
  We denote for every scalar~$\tau$ in~$\kf$ by~$f_\tau$ the endomorphism of~$I$ given by
  \[
    f_\tau(x) = x \,,
    \quad
    f_\tau(y) = \tau y \,,
  \]
  and by~$\theta_\tau$ the resulting homomorphism of Lie~algebras from~$\hlie$ to~$\Der(I)$ that maps~$t$ to~$f_\tau$.
  The external semidirect product~${\kf} \ltimes_{\theta_t} I$ has the elements~$x$,~$y$,~$t$ as a basis, and its Lie~bracket is given on this basis by
  \[
    [x,y] = 0 \,,
    \quad
    [t,x] = x \,,
    \quad
    [t,y] =  \tau y \,.
  \]
  We have thus constructed the Lie~algebras~$\glie_\tau$ from \cref{infinitely many three-dimensional lie algebras} as semidirect products.
\end{example}

\subsubsection{Split Extensions}

\begin{definition}
  Let
  \[
    0
    \to
    I
    \xto{\varphi}
    \glie
    \xto{\psi}
    \hlie
    \to
    0
  \]
  be an extension of Lie~algebras.
  A \defemph{section}\index{section}\index{extension!of Lie algebras!section} or \defemph{split}\index{split}\index{extension!of Lie algebras!split} of this extension is a homomorphism of Lie~algebras~$\sigma$ from~$\hlie$ to~$\glie$ such that
  \[
    \psi \circ \sigma
    =
    \id_{\hlie} \,.
  \]
  The extension \defemph{splits}\index{split}\index{extension!of Lie algebras!split} if it admits a split.
\end{definition}


\begin{lemma}
  \label{having a section is invariant under equivalence of extensions}
  Let~$\hlie$ and~$I$ be two~\liealgebras{$\kf$}
  If an extension of~$\hlie$ by~$I$ admits a split then every equivalent extension also admits a split.
\end{lemma}


\begin{proof}
  Let
  \[
    0
    \to
    I
    \xto{\varphi_1}
    \glie_1
    \xto{\psi_1}
    \hlie
    \to
    0
  \]
  and
  \[
    0
    \to
    I
    \xto{\varphi_2}
    \glie_2
    \xto{\psi_2}
    \hlie
    \to
    0
  \]
  be two equivalent extensions of~$\hlie$ by~$I$, and let~$\alpha$ be an isomorphism of Lie~algebras from~$\glie_1$ to~$\glie_2$ which makes the diagram
  \[
    \begin{tikzcd}
      0
      \arrow{r}
      &
      I
      \arrow{r}[above]{\varphi_1}
      \arrow[equal]{d}
      &
      \glie_1
      \arrow{r}[above]{\psi_2}
      \arrow[dashed]{d}[right]{\alpha}
      &
      \hlie
      \arrow{r}
      \arrow[equal]{d}
      &
      0
      \\
      0
      \arrow{r}
      &
      I
      \arrow{r}[above]{\varphi_2}
      &
      \glie_2
      \arrow{r}[above]{\psi_2}
      &
      \hlie
      \arrow{r}
      &
      0
    \end{tikzcd}
  \]
  commute.
  If~$\sigma$ is a split for the first extension then the composite~$\alpha \circ \sigma$ is a split for the second extension.
  This shows that the second extension admits a split if the first one does.
  The reverse implication follows from this first implication because equivalence of extensions is an equivalence relation.
\end{proof}


\begin{proposition}
  Let
  \begin{equation}
    \label{extension that may split}
    0
    \to
    I
    \xto{\varphi}
    \glie
    \xto{\psi}
    \hlie
    \to
    0
  \end{equation}
  be an extension of Lie~algebras.
  The following conditions on this extension are equivalent.
  \begin{equivalenceslist}
    \item
      \label{the extension splits}
      The extension~\eqref{extension that may split} splits.
    \item
      \label{is an internal semidirect product}
      There exists a Lie~subalgebra~$\hlie'$ of~$\glie$ such that~$\glie$ is the internal semidirect product of~$\hlie'$ by~$\varphi(I)$.
    \item
      \label{is an external semidirect product}
      The extension~\eqref{extension that may split} is isomorphic (as extensions) to an external semidirect product of~$\hlie$ by~$I$.
  \end{equivalenceslist}
\end{proposition}


\begin{proof}
  \leavevmode
  \begin{implicationlist}
    \item[\ref{the extension splits}~$\implies$~\ref{is an internal semidirect product}]
      There exists by assumption a split~$\sigma$ for the extension~\eqref{extension that may split}, i.e. a homomorphism of Lie~algebras from~$\hlie$ to~$\glie$ with~$\sigma \circ \psi = \id_{\hlie}$.
      Let~$\hlie'$ be the image of~$\sigma$.
      The composite~$\varepsilon \defined \sigma \circ \psi$ is an idempotent endomorphism of Lie~algebras of~$\glie$.
      Thus
      \[
        \glie
        =
        \im(\varepsilon) \oplus \ker(\varepsilon)
      \]
      as vector spaces, with~$\im(\varepsilon)$ being a Lie~subalgebra of~$\glie$ and~$\ker(\varepsilon)$ an ideal of~$\glie$.
      The homomorphism~$\psi$ is surjective, so
      \[
        \im(\varepsilon)
        =
        \im(\sigma \circ \psi)
        =
        \im(\sigma)
        =
        \hlie' \,.
      \]
      The homomorphism~$\sigma$ is injective (because it admits a left inverse, namely~$\psi$), so
      \[
        \ker(\varepsilon)
        =
        \ker(\sigma \circ \psi)
        =
        \ker(\psi)
        =
        \im(\varphi)
        =
        \varphi(I) \,.
      \]
      We have overall shown that~$\glie$ is the internal semidirect product of~$\hlie'$ by~$\varphi(I)$.
    \item[\ref{is an internal semidirect product}~$\implies$~\ref{is an external semidirect product}]
      This follows from the discussion in \cref{internal semidirect product is also external} by noting that the constructed isomorphism of Lie~algebras~$\varphi$ is already an isomorphism of extensions.
    \item[\ref{is an external semidirect product}~$\implies$~\ref{the extension splits}]
      According to \cref{having a section is invariant under equivalence of extensions} we may assume that~$\glie$ is an external semidirect product of~$\hlie$ by~$I$ along a homomorphism of Lie~algebras~$\theta$ from~$\hlie$ to~$\Der(I)$.
      The linear map
      \[
        \hlie
        \to
        \glie \,,
        \quad
        x
        \mapsto
        (x,0)
      \]
      is then a homomorphism of Lie~algebras and the desired split.
    \qedhere
  \end{implicationlist}
\end{proof}


\begin{remark}
  \label{each class of abelian extensions contains precisely one split extension}
  Let~$\hlie$ be a Lie~algebra and let~$M$ be a representation of~$\hlie$.
  We regard the underlying vector space of~$M$ as an abelian Lie~algebra~$I$ and denote by~$\theta$ the homomorphism of Lie~algebras from~$I$ to~$\Der(I) = \gllie(I) = \gllie(M)$ corresponding to the action of~$\glie$ on~$M$.
  We consider the resulting class of abelian extensions~$\AbEx(\hlie, M)$ from~\cref{structure of abelian extensions}.

  The external semidirect product~$(0,\theta)$ is an abelian extension of~$\hlie$ by~$I$.
  We have seen in~\cref{structure of abelian extensions} that this extension is contained in~$\AbEx(\hlie, M)$.

  Suppose on the other hand that~$\glie$ is some split extension of~$\hlie$ by~$I$ contained in~$\AbEx(\hlie, M)$.
  The extension~$\glie$ is then isomorphic to a standard extension~$(0, \theta')$.
  It follows from \cref{structure of abelian extensions} that~$\theta' = \theta$.

  We have thus shown that the class of extensions~$\AbEx(\hlie, M)$ contains precisely one split extension up to isomorphism.
\end{remark}


%\begin{definition}
%  A short exact sequence
%  \[
%    0
%    \to
%    I
%    \xto{f}
%    \glie
%    \xto{g}
%    \hlie
%    \to
%    0
%  \]
%  \defemph{splits}\index{extension!split} if there exists a Lie~algebra homomorphism~$s \colon \hlie \to \glie$ with~$g \circ s = \id_{\hlie}$.
%  The homomorphism~$s$ is then a \defemph{split} for the short exact sequence.
%  
%  A \emph{split extension} is an extension that splits.%
%  \footnote{This is meant to be funny.}
%\end{definition}
%
%
%\begin{example}[Split extensions]
%  Let
%  \[
%    0
%    \to
%    I
%    \xto{f}
%    \glie
%    \xto{g}
%    \hlie
%    \to
%    0
%  \]
%  be any split extension of~$\hlie$ by~$I$, and let~$s \colon \hlie \to \glie$ be a split.
%  Then there exists a (unique) homomorphism of vector spaces~$\varphi \colon \glie \to \hlie \oplus I$ that makes the resulting diagram
%  \[
%    \begin{tikzcd}[column sep = large]
%      0
%      \arrow{r}
%      &
%      I
%      \arrow{r}[above]{f}
%      \arrow[equal]{d}
%      &
%      \glie
%      \arrow{r}[above]{g}
%      \arrow[dashed]{d}[right]{\varphi}
%      &
%      \hlie
%      \arrow{r}
%      \arrow[equal]{d}
%      &
%      0
%      \\
%      0
%      \arrow{r}
%      &
%      I
%      \arrow{r}[above]{\iota}
%      &
%      \hlie \oplus I
%      \arrow{r}[above]{\pi}
%      &
%      \hlie
%      \arrow{r}
%      &
%      0
%    \end{tikzcd}
%  \]
%  commute and such that the square diagram
%  \[
%    \begin{tikzcd}
%      \glie
%      \arrow[dashed]{d}[left]{\varphi}
%      &
%      \hlie
%      \arrow{l}[above]{s}
%      \arrow[equal]{d}
%      \\
%      \hlie \oplus I
%      &
%      \hlie
%      \arrow{l}{\sigma}
%    \end{tikzcd}
%  \]
%  commutes.
%  Hence we may again assume that~$\glie = \hlie \oplus I$, that~$f$ is the canonical inclusion~$\iota \colon I \to \hlie \oplus I$, that~$g$ is the canonical projection~$\pi \colon \hlie \oplus I \to \hlie$ that~$s$ is the canonical inclusion~$\sigma \colon \hlie \to \hlie \oplus I$.
%  
%  We can now use the notations from \cref{general approach to extensions} to express the Lie bracket~$[-,-]$ of~$\hlie \oplus I$ via the bilinear form~$\kappa \colon \hlie \times \hlie \to I$ and the linear map~$\theta \colon \hlie \to \Der(I)$.
%  
%  The bilinear map~$\kappa$ is defined by the equality
%  \[
%    [(x,0), (y,0)]
%    =
%    ([x,y], \kappa(x,y))
%  \]
%  for all~$x, y \in \hlie$.
%  The linear subspace~$\hlie \oplus 0$ of~$\hlie \oplus I$ is the image of the Lie~algebra homomorphism~$\sigma \colon \hlie \to \hlie \oplus I$ and hence a Lie~subalgebra of~$\hlie \oplus I$.
%  We therefore find that the commutator~$[(x,0), (y,0)]$ is again contained in~$\hlie \oplus 0$ for all~$x, y \in \hlie$.
%  This means that~$\kappa = 0$.
%  
%  The linear map~$\theta \colon \hlie \to I$ takes for~$x \in \hlie$ the restriction of the endomorphism~$\ad_{\hlie \oplus I}((x,0))$ to the ideal~$0 \oplus I$ and then identifying~$0 \oplus I$ with~$I$ to get~$\theta(x) \in \Der(I)$.
%  This assignment may be written as the composition
%  \[
%    \theta
%    \colon
%    \hlie
%    \xto{\sigma}
%    \hlie \oplus I
%    \xto{ \restrict{\ad(-)}{0 \oplus I} }
%    \Der(0 \oplus I)
%    \to
%    \Der(I) \,.
%  \]
%  Each of the intermediate maps is a Lie~algebra homomorphism, so~$\theta$ is again a homomorphism.
%  
%  It remains to examine what further conditions such a homomorphism~$\theta$ needs to satisfy for the bracket~$[-,-]$ on~$\hlie \oplus I$ to satisfy the Jacobi identity.
%  We claim that there are none.
%  
%  Indeed, we need to check that under the given constraints~($\kappa = 0$ and~$\theta$ is a homomorphism of Lie~algebras~$\hlie \to \Der(I)$) we already have the Jacobi identity
%  \[
%      [\alpha, [\beta, \gamma]]
%    + [\beta, [\gamma, \alpha]]
%    + [\gamma, [\alpha, \beta]]
%    =
%    0
%  \]
%  for all~$\alpha, \beta, \gamma \in \hlie \oplus I$.
%  For this we use a trick taken from \cite{semidirect_jacobi}:
%  Let us denote the left hand side of this equation by~$J(\alpha, \beta, \gamma)$.
%  It follows from the bilinearity of~$[-,-]$ that~$J$ is trilinear.
%  It also follows from~$[-,-]$ being alternating and thus skew-symmetric that~$J$ is again skew-symmetric.
%  It therefore sufficies to check the condition~$J(\alpha, \beta, \gamma) = 0$ for the following four cases:
%  \begin{itemize}
%    \item
%      If~$\alpha = (x,0)$,~$\beta = (y,0)$ and~$\gamma = (z,0)$ with~$x, y, z \in \hlie$ then
%      \[
%        [\alpha, [\beta, \gamma]]
%        =
%        [(x,0), [(y,0), (z,0)]]
%        =
%        [(x,0), ([y,z], 0)]
%        =
%        ([x,[y,z]], 0)
%      \]
%      and therefore
%      \begin{align*}
%        J(\alpha, \beta, \gamma)
%        &=
%          ([x,[y,z]], 0)
%        + ([y,[z,x]], 0)
%        + ([z,[x,y]], 0)
%        \\
%        &=
%        ([x,[y,z]] + [y,[z,x]] + [z,[x,y]], 0)  \,.
%      \end{align*}
%      Hence~$J(\alpha, \beta, \gamma) = 0$ precisely because the Lie~bracket of~$\hlie$ satisfies the Jacobi identity.
%    \item
%      If~$\alpha = (x,0)$,~$\beta = (y,0)$ and~$\gamma = (0,c)$ with~$x, y \in \hlie$ and~$c \in I$ then
%      \[
%        [\alpha, [\beta, \gamma]]
%        =
%        [(x,0), [(y,0), (0,c)]]
%        =
%        [(x,0), [(0, \theta(y)(c)]]
%        =
%        ( 0, \theta(x)(\theta(y)(c)) )
%      \]
%      and similarly
%      \[
%        [\gamma, [\alpha, \beta]]
%        =
%        [(0,c), [(x,0), (y,0)]]
%        =
%        [(0,c), ([x,y], 0)]
%        =
%        ( 0, -\theta([x,y])(c) ) \,.
%      \]
%      It also follows that
%      \[
%        [\beta, [\gamma, \alpha]]
%        =
%        [(y,0), [(0,c), (x,0)]]
%        =
%        -[(y,0), [(x,0), (0,c)]]
%        =
%        ( 0, -\theta(y)(\theta(x)(c)) )
%      \]
%      and therefore
%      \begin{align*}
%        J(\alpha, \beta, \gamma)
%        &=
%          [(x,0), [(y,0), (0,c)]]
%        + [(y,0), [(0,c), (x,0)]]
%        + [(0,c), [(x,0), (y,0)]]
%        \\
%        &=
%        ( 0, \theta(x)(\theta(y)(c)) - \theta([x,y])(c) - \theta(y)(\theta(x)(c)) ) \,.
%      \end{align*}
%      We hence find for this case that~$J(\alpha, \beta, \gamma) = 0$ precisely because~$\theta$ is a Lie~algebra homomorphism.
%    \item
%      If~$\alpha = (x,0)$,~$\beta = (0,c)$ and~$\gamma = (0,d)$ with~$x \in \hlie$ and~$c, d \in I$ then
%      \[
%        [\alpha, [\beta, \gamma]]
%        =
%        [(x,0), [(0,c), (0,d)]]
%        =
%        [(x,0), (0,[c,d])]
%        =
%        ( 0, \theta(x)([c,d]) )
%      \]
%      and similarly
%      \[
%        [\beta, [\gamma, \alpha]]
%        =
%        [(0,c), [(0,d), (x,0)]]
%        =
%        [(0,c), (0, -\theta(x)(d)]
%        =
%        ( 0, -[c,\theta(x)(d)] ) \,.
%      \]
%      It also follows that
%      \begin{align*}
%        [\gamma, [\alpha, \beta]]
%        =
%        [(0,d), [(x,0), (0,c)]]
%        =
%        - [(0,d), [(0,c), (x,0)]]
%        &=
%        ( 0, [d,\theta(x)(c)] )
%        \\
%        &=
%        ( 0, -[\theta(x)(c), d] ) \,.
%      \end{align*}
%      It follows in combination that
%      \begin{align*}
%        J(\alpha, \beta, \gamma)
%        &=
%          [(x,0), [(0,c), (0,d)]]
%        + [(0,c), [(0,d), (0,x)]]
%        + [(0,d), [(x,0), (0,c)]]
%        \\
%        &=
%        ( 0, \theta(x)([c,d]) - [c, \theta(x)(d)] - [\theta(x)(c), d] )
%      \end{align*}
%      which shows that~$J(\alpha, \beta, \gamma) = 0$ precisely because~$\theta(x)$ is a derivation of~$I$.
%    \item
%      If~$\alpha = (0,c)$,~$\beta = (0,d)$ and~$\gamma = (0,e)$ with~$c, d, e \in I$ then
%      \[
%        [\alpha, [\beta, \gamma]]
%        =
%        [(0,c), [(0,d), (0,e)]]
%        =
%        [(0,c), (0, [d,e])]
%        =
%        ( 0, [c,[d,e]] )
%      \]
%      and therefore
%      \begin{align*}
%        J(\alpha, \beta, \gamma)
%        &=
%          [(0,c), [(0,d), (0,e)]]
%        + [(0,d), [(0,e), (0,c)]]
%        + [(0,e), [(0,c), (0,d)]]
%        \\
%        &=
%        ( 0, [c,[d,e]] + [d,[e,c]] + [e,[c,d]] )
%      \end{align*}
%      Which shows that~$J(\alpha, \beta, \gamma) = 0$ precisely because the Lie~bracket of~$I$ satisfies the Jacobi~identity.
%  \end{itemize}
%  
%  We have now overall constructed for all Lie~algebras~$\hlie$ and~$I$ a {\onetoonetext} correspondence between
%  \begin{itemize}
%    \item
%      Lie~brackets~$[-,-]$ on the vector space~$\hlie \oplus I$ that makes the standard short exact sequence
%      \[
%        0
%        \to
%        I
%        \to
%        \hlie
%        \oplus
%        I 
%        \to
%        \hlie
%        \to
%        0
%      \]
%      into a split extension of~$\hlie$ by~$I$ and
%    \item
%      Lie~algebra homomorphisms~$\theta \colon \hlie \to \Der(I)$,
%  \end{itemize}
%  and this correspondence is given by~$[(x,c), (y,d)] = ([x,y], \theta(x)(d) - \theta(y)(c) + [c,d])$.
%\end{example}
%
%
%\begin{definition}
%  Let~$\hlie$ and~$I$ be two Lie~algebras and let~$\theta \colon \hlie \to \Der(I)$ be a homomorphism of Lie~algebras.
%  The \defemph{semidirect product}~\gls*{semidirect product} of~$\hlie$ by~$I$ over~$\theta$ is the Lie~algebra~$\glie$ that is given by
%  \begin{itemize}
%    \item
%      the underlying vector space~$\glie \defined \hlie \oplus I$ together with
%    \item
%      the Lie bracket~$[-,-]$ given by
%      \[
%        [(x,c), (y,d)]
%        =
%        ([x,y], \theta(x)(d) - \theta(y)(c) + [c,d])
%      \]
%      for all~$(x,c), (y,d) \in \glie$.
%  \end{itemize}
%\end{definition}
%
%
%\begin{remark}
%  Let~$\hlie$ and~$I$ be Lie~algebras and let~$\theta \colon \hlie \to \Der(I)$ be a Lie~algebra homomorphism.
%  \begin{enumerate}
%    \item
%      It follows from the above discussion that~$\hlie \ltimes_\theta I$ is indeed a Lie algebra and that
%      \[
%        0
%        \to
%        I
%        \xto{\iota}
%        \hlie \ltimes_\theta I
%        \xto{\pi}
%        \hlie
%        \to
%        0
%      \]
%      is a split extension of~$\hlie$ by~$I$, where~$\iota(x) = (0,y)$ and~$\pi(x,y) = x$.
%      Moreover, a split~$\sigma \colon \hlie \to \hlie \ltimes_\theta I$ of this extension is given by~$\sigma(x) = (x,0)$.
%    \item
%      We have also seen that whenever a Lie algebra~$\glie$ is a split extension of~$\hlie$ by~$I$ then~$\glie$ is already (equivalent to) a semidirect product of~$\hlie$ by~$I$ (over some suitable homomorphism of Lie~algebras~$\theta \colon \hlie \to \Der(I)$).
%  \end{enumerate}
%  The notions of split extensions and semidirect products are hence equivalent.
%\end{remark}
%
%
%\begin{remark}[Internal semidirect products]
%  Given a Lie~algebra~$\glie$ we may ask ourselves if~$\glie$ can be decomposed as a semidirect product~$\glie \cong \hlie \ltimes_\theta I$;
%  and if so, how~$I$ and~$\glie$ look like from the point of view of~$\glie$.
%  
%  We observe first that the semidirect product~$\glie = \hlie \ltimes_\theta I$ (where~$\theta \colon \hlie \to \Der(I)$ is some Lie~algebra homomorphism) contains the Lie~subalgebra~$\hlie' = \{ (x,0) \suchthat x \in \hlie \}$ and the ideal~$I' \defined \{ (0,c) \suchthat c \in I \}$.
%  We have that~$\hlie' \cong \hlie$ and~$I' \cong I$ as Lie~algebras, and~$\glie = \hlie' \oplus I'$ as vector spaces.
%  The homomorpism~$\theta \colon \hlie \to \Der(I)$ can be described by considering for~$x \in \hlie$ the corresponding element~$(x,0) \in \hlie'$, restricting the endomorphism~$\ad_{\glie}((x,0)) \in \Der(\glie)$ to an endomorphism~$\restrict{ \ad_{\glie}((x,0)) }{I'} \in \Der(I')$ and then using the isomorphism~$I' \cong I$ to arrive at~$\theta(x) \in \Der(I)$.
%  
%  Suppose on the other hand that~$\glie$ is any Lie~algebra and that~$\glie = \hlie \oplus I$ for some Lie~subalgebra~$\hlie$ of~$\glie$ and some ideal~$I$ of~$\glie$.
%  Then~$\theta \colon \hlie \to \Der(I)$ given by~$x \mapsto \restrict{\ad_{\glie}(x)}{I}$ is a well-defined Lie algebra homomorphism, and we have that
%  \begin{align*}
%    [x + c, y + d]
%    &=
%    [x,y] + [x,d] + [c,y] + [c,d]
%    \\
%    &=
%    [x,y] + [x,d] - [y,c] + [c,d]
%    \\
%    &=
%    \underbrace{ [x,y] }_{\in \hlie} + \underbrace{ \theta(x)(d) - \theta(y)(c) + [c,d] }_{\in I}
%  \end{align*}
%  for all~$x, y \in \hlie$ and~$c, d \in I$.
%  This shows that the vector space isomorphism
%  \[
%    \varphi
%    \colon
%    \hlie \oplus I
%    \to
%    \glie \,,
%    \quad
%    (x,c)
%    \mapsto
%    x + c
%  \]
%  is already an isomorphism of Lie algebras~$\varphi \colon \hlie \ltimes_\theta I \to \glie$.
%\end{remark}
%
%
%\begin{definition}
%  Let~$\glie$ be a Lie~algebra, let~$\hlie$ be a Lie~subalgebra of~$\glie$ and let~$I$ be an ideal in~$\glie$.
%  Then~$\glie$ is the \defemph{internal semidirect product}\index{internal semidirect product} of~$\hlie$ by~$I$ if~$\glie = \hlie \oplus I$.
%  This is then denoted by~$\glie = \gls*{internal semidirect product}$.
%\end{definition}
%
%
%\begin{remark}
%  The notion of an internal semidirect product does not depend on the additional information of a Lie~algebra homomorphism~$\theta \colon \hlie \to \Der(I)$.
%  Instead this information is encoded in the Lie~algebra structure of~$\glie$, and related to the expected homomorphism~$\theta$ by~$\theta(x) = \restrict{\ad_{\glie}(x)}{I}$.
%\end{remark}
%
%
%\begin{example}
%  The Lie algebra~$\glie = \tlie(n, \kf)$ can be written as a direct sum~$\tlie(n, \kf) = \dlie(n, \kf) \oplus \nlie(n, \kf)$ with~$\dlie(n, \kf)$ a Lie~subalgebra of~$\tlie(n, \kf)$ and~$\nlie(n, \kf)$ an ideal in~$\tlie(n, \kf)$.
%  Hence
%  \[
%    \tlie(n, \kf)
%    =
%    \dlie(n, \kf) \ltimes \nlie(n, \kf) \,.
%  \]
%\end{example}
%
%
%\begin{example}[Trivial extensions]
%  \label{trivial extension is semidirect}
%  For any two Lie~algebras~$I$ and~$\hlie$ the zero map~$\theta \colon \hlie \to \Der(I)$ is a homomorphism of Lie algebras.
%  The resulting semidirect product~$\hlie \ltimes_\theta I$ is precisely the usual product~$\hlie \times I$.
%  The corresponding split extension
%  \[
%    0
%    \to
%    I
%    \xto{\iota}
%    \hlie \times I
%    \xto{\pi}
%    \hlie
%    \to
%    0
%  \]
%  is given by canonical inclusion~$\iota(c) = (0,c)$ and the canonical projection~$\pi(x,y) = x$.
%  This extension is the \emph{trivial extension}\index{trivial extension} of~$\hlie$ by~$I$.
%  More generally any extension that is equivalent to the trivial extension is called \defemph{trivial}.
%\end{example}
%

% \begin{remark}
%   \label{when semidirect is direct}
%   The converse to the above observation also holds:
%   Let~$I$ and~$\hlie$ be Lie~algebras and let~$\theta \colon \hlie \to \Der(I)$ be a homomorphism of Lie~algebras such that the resulting semidirect product~$\hlie \ltimes_\theta I$ is trivial.
%   Then already~$\theta = 0$.
%   
%   Indeed, if the extension
%   \[
%     0
%     \to
%     I
%     \xto{\iota}
%     \hlie \times I
%     \xto{\pi}
%     \hlie
%     \to
%     0
%   \]
%   is trivial
% 
% \end{remark}


%\begin{warning}
%  Let
%  \[
%    0
%    \to
%    I
%    \xto{f}
%    \glie
%    \xto{g}
%    \hlie
%    \to
%    0
%  \]
%  be an extension of Lie~algebras.
%  The observant reader may have noticed that we have so far only considered splits~$s \colon \hlie \to \glie$, but not splits~$t \colon \glie \to I$.
%  There is a good reason for this:
%  
%  Suppose that such a split exists, i.e.\ there exists a Lie~algebra homomorphism~$t \colon \glie \to I$ with~$t {} f = \id_I$.
%  Then~$J \defined \ker t$ is an ideal in~$\glie$.
%  If~$I'$ denotes the image of~$I$ in~$\glie$ then it follows that~$\glie = I' \oplus J$, the homomorphism~$f$ restricts to an isomorphism~$I \to I'$ and the homomorphism~$g$ restricts to an isomorphism~$J \to \hlie$.
%  This means that the given extension is equivalent to the trivial extension, and hence trivial itself.
%  
%  We see on the other hand that the trivial extension admits both kinds of split, and thus the same holds for every extension that is trivial.
%  
%  This means altogether that an extension~$0 \to I \to \glie \to \hlie \to 0$ admits a split~$\glie \to I$ (of Lie~algebras) if and only if the extension is already trivial.
%\end{warning}







